\bbook{Prophetia Habacuc}
{Habacuc}{images/genese_heading}


\bchapter
\mylettrine{O}nus quod vidit Habacuc propheta.


\begin{flushleft}\begin{verse}\vspace{6pt}${}^{2}$~Usquequo, Domine, clamabo,\\ et non exaudies~?\\ vociferabor ad te, vim patiens,\\ et non salvabis~?\\
${}^{3}$~Quare ostendisti mihi iniquitatem et laborem,\\ videre pr\ae dam et injustitiam contra me~?\\ Et factum est judicium, et contradictio potentior.\\
${}^{4}$~Propter hoc lacerata est lex,\\ et non pervenit usque ad finem judicium~;\\ quia impius pr\ae valet adversus justum,\\ propterea egreditur judicium perversum.\\
${}^{5}$~Aspicite in gentibus, et videte~;\\ admiramini, et obstupescite~:\\ quia opus factum est in diebus vestris,\\ quod nemo credet cum narrabitur.\\
${}^{6}$~Quia ecce ego suscitabo Chald\ae os,\\ gentem amaram et velocem,\\ ambulantem super latitudinem terr\ae ,\\ ut possideat tabernacula non sua.\\
${}^{7}$~Horribilis et terribilis est~:\\ ex semetipsa judicium et onus ejus egredietur.\\
${}^{8}$~Leviores pardis equi ejus,\\ et velociores lupis vespertinis~:\\ et diffundentur equites ejus~:\\ equites namque ejus de longe venient~;\\ volabunt quasi aquila festinans ad comedendum.\\
${}^{9}$~Omnes ad pr\ae dam venient,\\ facies eorum ventus urens~;\\ et congregabit quasi arenam captivitatem.\\
${}^{10}$~Et ipse de regibus triumphabit,\\ et tyranni ridiculi ejus erunt~;\\ ipse super omnem munitionem ridebit,\\ et comportabit aggerem, et capiet eam.\\
${}^{11}$~Tunc mutabitur spiritus, et pertransibit, et corruet~:\\ h\ae c est fortitudo ejus dei sui.\end{verse}\end{flushleft}


\begin{flushleft}\begin{verse}${}^{12}$~Numquid non tu a principio,\\ Domine, Deus meus, sancte meus,\\ et non moriemur~?\\ Domine, in judicium posuisti eum,\\ et fortem, ut corriperes, fundasti eum.\\
${}^{13}$~Mundi sunt oculi tui, ne videas malum,\\ et respicere ad iniquitatem non poteris.\\ Quare respicis super iniqua agentes,\\ et taces devorante impio justiorem se~?\\
${}^{14}$~Et facies homines quasi pisces maris,\\ et quasi reptile non habens principem.\\
${}^{15}$~Totum in hamo sublevavit,\\ traxit illud in sagena sua,\\ et congregavit in rete suum.\\ Super hoc l\ae tabitur, et exsultabit.\\
${}^{16}$~Propterea immolabit sagen\ae\ su\ae ,\\ et sacrificabit reti suo,\\ quia in ipsis incrassata est pars ejus,\\ et cibus ejus electus.\\
${}^{17}$~Propter hoc ergo expandit sagenam suam,\\ et semper interficere gentes non parcet.\end{verse}\end{flushleft}


\Needspace{2.5\baselineskip}\versal{2}\begin{flushleft}\begin{verse}\vspace{-19pt}Super custodiam meam stabo,\\ et figam gradum super munitionem~:\\ et contemplabor ut videam quid dicatur mihi,\\ et quid respondeam ad arguentem me.\end{verse}\end{flushleft}


${}^{2}$~Et respondit mihi Dominus, et dixit~: \begin{flushleft}\begin{verse}Scribe visum, et explana eum super tabulas,\\ ut percurrat qui legerit eum.\\
${}^{3}$~Quia adhuc visus procul~;\\ et apparebit in finem, et non mentietur~:\\ si moram fecerit, exspecta illum,\\ quia veniens veniet, et non tardabit.\\
${}^{4}$~Ecce qui incredulus est, non erit recta anima ejus in semetipso~;\\ justus autem in fide sua vivet.\\
${}^{5}$~Et quomodo vinum potantem decipit,\\ sic erit vir superbus, et non decorabitur~:\\ qui dilatavit quasi infernus animam suam,\\ et ipse quasi mors, et non adimpletur~:\\ et congregabit ad se omnes gentes,\\ et coacervabit ad se omnes populos.\\
${}^{6}$~Numquid non omnes isti super eum parabolam sument,\\ et loquelam \ae nigmatum ejus, et dicetur~:\\ V\ae\ ei qui multiplicat non sua~?\\ usquequo et aggravat contra se densum lutum~?\\
${}^{7}$~Numquid non repente consurgent qui mordeant te,\\ et suscitabuntur lacerantes te,\\ et eris in rapinam eis~?\\
${}^{8}$~Quia tu spoliasti gentes multas,\\ spoliabunt te omnes qui reliqui fuerint de populis,\\ propter sanguinem hominis,\\ et iniquitatem terr\ae , civitatis, et omnium habitantium in ea.\\
${}^{9}$~V\ae\ qui congregat avaritiam malam domui su\ae ,\\ ut sit in excelso nidus ejus,\\ et liberari se putat de manu mali~!\\
${}^{10}$~Cogitasti confusionem domui tu\ae~;\\ concidisti populos multos,\\ et peccavit anima tua.\\
${}^{11}$~Quia lapis de pariete clamabit,\\ et lignum, quod inter juncturas \ae dificiorum est, respondebit.\\
${}^{12}$~V\ae\ qui \ae dificat civitatem in sanguinibus,\\ et pr\ae parat urbem in iniquitate~!\\
${}^{13}$~Numquid non h\ae c sunt a Domino exercituum~?\\ laborabunt enim populi in multo igne,\\ et gentes in vacuum, et deficient.\\
${}^{14}$~Quia replebitur terra, ut cognoscant gloriam Domini,\\ quasi aqu\ae\ operientes mare.\\
${}^{15}$~V\ae\ qui potum dat amico suo mittens fel suum,\\ et inebrians ut aspiciat nuditatem ejus~!\\
${}^{16}$~Repletus es ignominia pro gloria~;\\ bibe tu quoque, et consopire.\\ Circumdabit te calix dexter\ae\ Domini,\\ et vomitus ignomini\ae\ super gloriam tuam.\\
${}^{17}$~Quia iniquitas Libani operiet te,\\ et vastitas animalium deterrebit eos\\ de sanguinibus hominum,\\ et iniquitate terr\ae , et civitatis, et omnium habitantium in ea.\\
${}^{18}$~Quid prodest sculptile, quia sculpsit illud fictor suus,\\ conflatile, et imaginem falsam~?\\ quia speravit in figmento fictor ejus, ut faceret simulacra muta.\\
${}^{19}$~V\ae\ qui dicit ligno~: Expergiscere~;\\ Surge, lapidi tacenti~!\\ Numquid ipse docere poterit~?\\ ecce iste coopertus est auro et argento,\\ et omnis spiritus non est in visceribus ejus.\\
${}^{20}$~Dominus autem in templo sancto suo~:\\ sileat a facie ejus omnis terra~!\end{verse}\end{flushleft}



\bchapter
\mylettrine{O}ratio Habacuc prophet\ae , pro ignorantiis.


\begin{flushleft}\begin{verse}\vspace{6pt}${}^{2}$~Domine, audivi auditionem tuam, et timui.\\ Domine, opus tuum, in medio annorum vivifica illud~;\\ in medio annorum notum facies~:\\ cum iratus fueris, misericordi\ae\ recordaberis.\\
${}^{3}$~Deus ab austro veniet,\\ et Sanctus de monte Pharan~:\\ operuit c\ae los gloria ejus,\\ et laudis ejus plena est terra.\\
${}^{4}$~Splendor ejus ut lux erit,\\ cornua in manibus ejus~:\\ ibi abscondita est fortitudo ejus.\\
${}^{5}$~Ante faciem ejus ibit mors,\\ et egredietur diabolus ante pedes ejus.\\
${}^{6}$~Stetit, et mensus est terram~;\\ aspexit, et dissolvit gentes,\\ et contriti sunt montes s\ae culi~:\\ incurvati sunt colles mundi ab itineribus \ae ternitatis ejus.\\
${}^{7}$~Pro iniquitate vidi tentoria \AE thiopi\ae~;\\ turbabuntur pelles terr\ae\ Madian.\\
${}^{8}$~Numquid in fluminibus iratus es, Domine~?\\ aut in fluminibus furor tuus~?\\ vel in mari indignatio tua~?\\ Qui ascendes super equos tuos, et quadrig\ae\ tu\ae\ salvatio.\\
${}^{9}$~Suscitans suscitabis arcum tuum,\\ juramenta tribubus qu\ae\ locutus es~;\\ fluvios scindes terr\ae .\\
${}^{10}$~Viderunt te, et doluerunt montes~;\\ gurges aquarum transiit~:\\ dedit abyssus vocem suam~;\\ altitudo manus suas levavit.\\
${}^{11}$~Sol et luna steterunt in habitaculo suo~:\\ in luce sagittarum tuarum ibunt,\\ in splendore fulgurantis hast\ae\ tu\ae .\\
${}^{12}$~In fremitu conculcabis terram~;\\ in furore obstupefacies gentes.\\
${}^{13}$~Egressus es in salutem populi tui,\\ in salutem cum christo tuo~:\\ percussisti caput de domo impii,\\ denudasti fundamentum ejus usque ad collum.\\
${}^{14}$~Maledixisti sceptris ejus,\\ capiti bellatorum ejus,\\ venientibus ut turbo ad dispergendum me~:\\ exsultatio eorum, sicut ejus qui devorat pauperem in abscondito.\\
${}^{15}$~Viam fecisti in mari equis tuis,\\ in luto aquarum multarum.\\
${}^{16}$~Audivi, et conturbatus est venter meus~;\\ a voce contremuerunt labia mea.\\ Ingrediatur putredo in ossibus meis,\\ et subter me scateat~:\\ ut requiescam in die tribulationis,\\ ut ascendam ad populum accinctum nostrum.\\
${}^{17}$~Ficus enim non florebit,\\ et non erit germen in vineis~;\\ mentietur opus oliv\ae ,\\ et arva non afferent cibum~:\\ abscindetur de ovili pecus,\\ et non erit armentum in pr\ae sepibus.\\
${}^{18}$~Ego autem in Domino gaudebo~;\\ et exsultabo in Deo Jesu meo.\\
${}^{19}$~Deus Dominus fortitudo mea,\\ et ponet pedes meos quasi cervorum~:\\ et super excelsa mea deducet me\\ victor in psalmis canentem.\end{verse}\end{flushleft}


