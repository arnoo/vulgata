\hyphenation{quod-cum-que}%
\hyphenation{quid-quam}%
\hyphenation{super-erit}%
\hyphenation{trans-tulit}%
\hyphenation{sex-centis}%
\hyphenation{sex-centa}%
\hyphenation{unus-quis-que}%
\newcommand{\addtitleline}[4]{\addtocontents{#1}{%
\protect\contentsline{#2}{#3}{#4}}}

\makeatletter
\renewcommand{\@pnumwidth}{2em}
\makeatother

\renewcommand{\contentsname}{Index Generalis}
\begin{center}\tableofcontents\end{center}
\clrdouble

\addtocontents{toc}{\protect\makebox[\columnwidth][r]{\textit{pag.}}}
\vspace{-18pt}{\centering \section*{Præfatio hujus editionis}}
\addcontentsline{toc}{subsection}{Præfatio hujus editionis}
\noindent Clementis Pap\ae~\textsc{viii} nomen casu Vulgat\ae\ adh\ae ret. H\ae c enim
sanctarum Scripturarum editio, cum Sixto~\textsc{v} regnante patrata fuisset,
illo mortuo tandem sub Clementis pontificatu publicata est. Attamen vere decet
ut iste Clemens, cum sit institutorum Concilii Tridentini \ae muli exemplar, hoc
bonum ejusdem Synodus fructum perficiat.

Sapientissimo fretus consilio S.~Philippi Nerii, Sancti Spiritus caritate
ex\ae stuantis, ad ecclesiasticam disciplinam reparandam, et errores
corruptelasque exstirpandas, indefesse laboravit~; et pro Ecclesi\ae\ libertate
acriter pugnavit. Tantus autem Clemens fidem catholicam tutus est opere et
sermone, quantus et ipsius Vulgata revelationem Dei ab imminentibus periculis
custodivit.

Hieronymi enim Doctoris maximi ex lingua hebr\ae a translatione, ita per
nongentos annos fruiti sunt doctissimi Ecclesi\ae\ filii, ut ceter\ae\
translationes latin\ae\ pene evanuissent~: illius autem innumerabilibus vicibus
descript\ae, \guillemotleft~tot fuisse exemplaria, quot codices~\guillemotright.
Quando ergo Lutherus flagitiosa rebellione ab Ecclesia descivit, hac confusione
abusus est ut verba Evangelii corrumperet, et suam falsam doctrinam
defenderet. Quam ob causam opus erat ut authentica divinarum Scripturarum
editio, diligenter emendata atque castigata, imprimeretur cum plena
Apostolic\ae\ Sedis auctoritate~; sacrosancta itaque Tridentina Synodus, Spiritu
Sancto gubernante, hoc jussit, et ab ea jussum, a Clemente \textsc{viii} est
consummatum.

Porro \textsc{xx}~s\ae culo incipiente, antiquissimis jam
alteris manuscriptis aliunde inventis, ut Bibli\ae\ editionem puram accuratamque
Ecclesia semper haberet sollicitus, Sanctus Pius~\textsc{x} Commissioni
Pontifici\ae\ de re biblica mandavit ut pristinam Hieronymi versionem, quantum
fieri potuit, componeret. Primo Card.~Gasquet, deinde D.~Henrico
Quentin, hoc onus delegatum est.  Novissime, defunctis illis, Abbatia Pontificia
S.~Hieronymi a Pio \textsc{xi} Rom\ae\ fundata est, ut monachi Benedictini
artis critic\ae\ peritissimi ibi opus emolirentur. Singulis deinceps annuentibus
Papis, minutim sancti libri emittebantur, et postremo anno salutis
\textsc{mcmxcviii} Biblia tota perfecta est.

Interim, eheu, interveniebant eventus infelices apud Ecclesiam. S.~Pius~\textsc{x}
voluerat constabiliri translationem pristinam S.~Hieronymi ut exinde confectura
esset Vulgata ad usum Ecclesi\ae\ quam accuratissime repurgata~; sed
illius voluntatem parvipendentes, malebant viri tumidi translationem
novam secundum gustus proprios facere, interpretationem vero Hieronymi negligere omnino. Quin etiam \ae tate nostra,
errores illi pessimi, quos idem Pius vehementissime condemnavit,
Ecclesiam Dei aperte infestant, et Lutherus dimidium errorum
co\ae qualum nostrorum non cogitavit. Oportet nos igitur Vulgatam Clementinam fideliter
amplexari donec, omnibus instauratis in Christo, Pii~\textsc{x} propositum
demum efficiatur.\enlargethispage{2\baselineskip}

\begin{flushright}
  \emph{Scribebam Londini, in festo D.~N.~J.~C.~Regis,\\ 31~Octobris 2004.}
\end{flushright}

\clearpage%
{\centering \section*{Ex Concilio Tridentino, sess.\ 4}}
%\addtocontents{toc}{\protect$\contentsline{subsection}{Ex Concilio Tridentino, sess.\ 4}{\protect\thepage}}
%\addtitleline{toc}{subsection}{Decreta Concilii Tridentini}{\thepage}
\addcontentsline{toc}{subsection}{Decreta Concilii Tridentini}

\begin{center}
\vspace{-9pt}\textit{8 aprilis 1546}
\end{center}

\begin{center}
\textsc{Decretum de canonicis Scripturis}
\end{center}
\noindent Sacrosancta \oe cumenica et generalis Tridentina Synodus, in Spiritu 
Sancto 
legitime congregata, pr\ae sidentibus in ea eisdem tribus Apostolic\ae\ Sedis 
legatis, hoc sibi perpetuo ante oculos proponens, ut, sublatis erroribus, 
puritas ipsa Evangelii in Ecclesia conservetur~: quod promissum ante per 
prophetas in Scripturis sanctis, Dominus noster Jesus Christus, Dei Filius, 
proprio ore primum promulgavit, deinde per suos apostolos, tamquam fontem omnis 
et salutaris veritatis et morum disciplin\ae, omni creatur\ae\
pr\ae dicari jussit~: perspiciensque hanc veritatem et disciplinam contineri 
in libris scriptis et sine scripto traditionibus, qu\ae\ ab ipsius 
Christi ore ab apostolis accept\ae, aut ab ipsis apostolis Spiritu Sancto 
dictante, quasi per manus tradit\ae, ad nos usque pervenerunt~: orthodoxorum 
Patrum exempla secuta, omnes libros tam Veteris quam Novi Testamenti, cum 
utriusque unus Deus sit auctor, nec non traditiones ipsas, tum ad fidem, tum ad 
mores pertinentes, tamquam vel oretenus a Christo, vel a Spiritu Sancto 
dictatas, et continua successione in Ecclesia catholica conservatas, pari 
pietatis affectu ac reverentia suscipit ac veneratur.

Sacrorum vero librorum indicem huic decreto ascribendum censuit, ne cui 
dubitatio suboriri possit, quinam sint, qui ab ipsa Synodo suscipiuntur. Sunt 
vero infrascripti~:

Testamenti Veteris~: quinque Moysi, id est Genesis, Exodus,
Leviticus, Numeri, 
Deuteronomium~; Josue, Judicum, Ruth, quatuor Regum, duo Paralipomenon, 
Esdr\ae\ primus et secundus, qui dicitur Nehemias, Tobias, Judith, Esther, Job, 
Psalterium Davidicum centum quinquaginta psalmorum, Parabol\ae, 
Ecclesiastes, Canticum Canticorum, Sapientia, Ecclesiasticus, Isaias, Jeremias 
cum Baruch, Ezechiel, Daniel, duodecim prophet\ae\ minores, id est~: 
Osee, Jo\"el, Amos, Abdias, Jonas, Mich\ae as, Nahum, Habacuc, Sophonias, 
Agg\ae us, Zacharias, Malachias~; duo Machab\ae orum, primus et secundus.

Testamenti Novi~: quatuor Evangelia, secundum Matth\ae um, Marcum, Lucam, 
Joannem~; Actus Apostolorum a Luca Evangelista conscripti~; quatuordecim 
epistol\ae\ Pauli Apostoli~: ad Romanos, du\ae\ ad Corinthios, ad Galatas, ad 
Ephesios, ad Philippenses, ad Colossenses, du\ae\ ad Thessalonicenses, 
du\ae\ ad Timotheum, ad Titum, ad Philemonem, ad Hebr\ae os~; Petri 
Apostoli du\ae~; Joannis Apostoli tres~; Jacobi Apostoli una~; Jud\ae\ 
Apostoli una et Apocalypsis Joannis Apostoli.

Si quis autem libros ipsos integros cum omnibus suis partibus, prout in 
Ecclesia catholica legi consueverunt, et in veteri Vulgata latina editione 
habentur, pro sacris et canonicis non susceperit, et traditiones pr\ae dictas 
sciens et prudens contempserit, anathema sit.\enlargethispage{2\baselineskip}

\newpage%
\begin{center}
\textsc{Decretum de editione et usu sacrorum librorum}
\end{center}
\noindent Insuper eadem sacrosancta Synodus, considerans non parum utilitatis 
accedere posse Ecclesi\ae\ Dei, si ex omnibus latinis editionibus, qu\ae\ 
circumferuntur sacrorum librorum, qu\ae\ nam pro authentica habenda sit, 
innotescat~: statuit et declarat, ut h\ae c ipsa vetus et Vulgata editio, 
qu\ae\ longo tot s\ae culorum usu in ipsa Ecclesia probata est, in 
publicis lectionibus, disputationibus, pr\ae dicationibus et expositionibus 
pro authentica habeatur, ut nemo illam rejicere quovis pr\ae textu audeat vel 
pr\ae sumat. 

Pr\ae terea ad co\"ercenda petulantia ingenia decernit, ut nemo, su\ae\ 
prudenti\ae\ innixus, in rebus fidei et morum ad \ae dificationem doctrin\ae\ 
christian\ae\ pertinentium, sacram Scripturam ad suos sensus contorquens, 
contra eum sensum, quem tenuit et tenet sancta mater Ecclesia, cujus est 
judicare de vero sensu et interpretatione Scripturarum sanctarum, aut etiam 
contra unanimen consensum Patrum, ipsam Scripturam sacram interpretari audeat, 
etiamsi hujusmodi interpretationes nullo umquam tempore in lucem edend\ae\ 
forent.
Qui contravenerint, per Ordinarios declarentur, et p\oe nis a jure 
statutis puniantur. 

Sed et impressoribus modum in hac parte, ut par est, imponere volens, qui jam 
sine modo, hoc est putantes sibi licere, quidquid libet, sine licentia 
superiorum ecclesiasticorum, ipsos sacr\ae\ Scriptur\ae\ libros, et super 
illis 
annotationes et expositiones quorumlibet indifferenter, s\ae pe tacito, 
s\ae pe etiam ementito prelo, et quod gravius est, sine nomine auctoris 
imprimunt, alibi etiam impressos libros hujusmodi temere venales habent~: 
decernit et statuit, ut posthac sacra Scriptura, potissimum vero h\ae c ipsa 
vetus et Vulgata editio, quam emendatissime imprimatur, nullique liceat 
imprimere, vel imprimi facere, quosvis libros de rebus sacris sine nomine 
auctoris, neque illos in futurum vendere, aut etiam apud se retinere, nisi 
primum examinati probatique fuerint ab Ordinario, sub p\oe na anathematis et 
pecuni\ae\ in canone Concilii novissimi Lateranensis apposita. Et si 
regulares fuerint, ultra examinationem et probationem hujusmodi, licentiam 
quoque a suis superioribus impetrare teneantur, recognitis per eos libris, 
juxta formam suarum ordinationum. Qui autem scripto eos communicant, vel 
evulgant, nisi antea examinati probatique fuerint, eisdem p\oe nis subjaceant, 
quibus impressores. Et qui eos habuerint vel legerint, nisi prodiderint 
auctorem, pro auctoribus habeantur. Ipsa vero hujusmodi librorum probatio in 
scriptis detur, atque ideo in fronte libri vel scripti vel impressi authentice 
appareat~; idque totum, hoc est, et probatio et examen, gratis fiat, ut 
probanda 
probentur et reprobentur improbanda. 

Post h\ae c temeritatem illam reprimere volens, qua ad profana qu\ae que 
convertuntur et torquentur verba et sententi\ae\ sacr\ae\ Scriptur\ae, ad 
scurrilia scilicet, fabulosa, vana, adulationes, detractiones, superstitiones, 
impias et diabolicas incantationes, divinationes, sortes, libellos etiam 
famosos, mandat et pr\ae cipit, ad tollendam hujusmodi irreverentiam et 
contemptum, et ne de cetero quisquam quomodolibet verba Scriptur\ae\ 
sacr\ae\ ad h\ae c et similia audeat usurpare, ut omnes hujus generis 
homines, temeratores et violatores verbi Dei, juris et arbitrii p\oe nis per 
Episcopos co\"erceantur.


\clearpage%
{\centering \section*{Clemens Papa \textsc{viii} ad perpetuam rei memoriam}}
\addcontentsline{toc}{subsection}{Constitutio Clementis viii}
\noindent Cum sacrorum Bibliorum Vulgat\ae\ editionis textus summis laboribus 
ac vigiliis restitutus, et quam accuratissime mendis expurgatus, benedicente 
Domino, ex nostra typographia Vaticana in lucem prodeat~; Nos, ut in posterum 
idem textus incorruptus, ut decet, conservetur, opportune providere volentes, 
auctoritate apostolica, tenore pr\ae sentium districtius inhibemus, ne intra 
decem annos a data pr\ae sentium numerandos, tam citra quam ultra montes, alibi 
quam in nostra Vaticana typographia, a quoquam imprimatur. Elapso autem pr\ae 
fato decennio, eam cautionem adhiberi pr\ae cipimus, ut nemo hanc sanctarum 
Scripturarum editionem typis mandare pr\ae sumat, nisi habito prius exemplari 
in typographia Vaticana excuso~: cujus exemplaris forma, ne minima quidem 
particula de textu mutata, addita, vel ab eo detracta, nisi aliquid occurrat, 
quod typographic\ae\ incuri\ae\ manifeste ascribendum sit, inviolabiliter 
observetur. 

Si quis vero typographus in quibuscumque regnis, civitatibus, provinciis, et 
locis tam nostr\ae\ et S.\ R.\ E.\ ditioni in temporalibus subjectis, quam non 
subjectis, hanc eamdem sacrarum Scripturarum editionem intra decennium pr\ae 
dictum quoquo modo, elapso autem decennio, aliter quam juxta hujusmodi 
exemplar, ut pr\ae fertur, imprimere, vendere, venales habere, aut alias edere 
vel evulgare~: aut si quis bibliopola a se vel ab aliis quibusvis, post datam 
pr\ae sentium, hujus editionis impressos libros, seu imprimendos a pr\ae fato 
restituto et correcto textu in aliquo discrepantes, seu ab alio, quam a 
typographo Vaticano, intra decennium excusos, pariter vendere, venales 
proponere, vel evulgare pr\ae sumpserit, ultra amissionem omnium librorum, et 
alias arbitrio nostro infligendas p\oe nas temporales, etiam majoris 
excommunicationis sententiam eo ipso incurrat~: a qua nisi a Romano Pontifice, 
pr\ae terquam in mortis articulo constitutus, absolvi non possit. 

Mandamus itaque universis et singulis, Patriarchis, Archiepiscopis, Episcopis, 
ceterisque Ecclesiarum et locorum, etiam regularium Pr\ae latis, ut pr\ae 
sentes litteras in suis quisque Ecclesiis et jurisdictionibus ab omnibus 
inviolate perpetuo observari curent ac faciant. Contradictores per censuras 
ecclesiasticas, aliaque opportuna juris et facti remedia, appellatione 
postposita, compescendo, invocato etiam ad hoc, si opus fuerit, auxilio brachii 
s\ae cularis~; non obstantibus constitutionibus, et ordinationibus apostolicis, 
ac in generalibus, provincialibus, vel synodalibus Conciliis editis 
generalibus, vel specialibus, ac quarumcumque Ecclesiarum, Ordinum, 
Congregationum, Collegiorum atque Universitatum, etiam studiorum generalium 
juramento, confirmatione apostolica, vel quavis firmitate alia roboratis, 
statutis, et consuetudinibus, ac privilegiis, indultis, ac litteris apostolicis 
in contrarium quomodocumque emanatis, et emanandis~: quibus omnibus ad hunc 
effectum latissime derogamus, ac derogatum esse decernimus. Volumus autem, ut 
pr\ae sentium trans sumptis etiam in ipsis voluminibus impressis eadem, in 
judicio et extra, fides ubique adhibeatur, qu\ae\ ipsis pr\ae sentibus 
adhiberetur, si forent exhibit\ae, vel ostens\ae. 

Datum Rom\ae, apud Sanctum Petrum sub Annulo Piscatoris, die \textsc{ix} 
novembris
\textsc{mdxcii}, Pont.\ Nostri anno~\textsc{i}.
\begin{flushright}\textsc{M.\ Vestrius Barbianus}\end{flushright}


\clearpage%
{\centering \section*{Pr\ae fatio ad lectorem}}
\addcontentsline{toc}{subsection}{Pr\ae fatio ad lectorem officialis}
\begin{center}
\vspace{-9pt}\textit{ex editione vaticana anni 1592}
\end{center}

\noindent In multis magnisque beneficiis, qu\ae\ per sacram Tridentinam Synodum 
Ecclesi\ae\ su\ae\ Deus contulit, id in primis numerandum videtur, quod inter 
tot latinas editiones divinarum Scripturarum, solam veterem ac Vulgatam, qu\ae\ 
longo tot s\ae culorum usu in Ecclesia probata fuerat, gravissimo Decreto 
authenticam declaravit. Nam, ut illud omittamus, quod ex recentibus editionibus 
non pauc\ae\ ad h\ae reses hujus temporis confirmandas licenter detort\ae\ 
videbantur~: ipsa certe tanta versionum varietas, atque diversitas magnam in 
Ecclesia Dei confusionem parere potuisset. Jam enim hac nostra \ae tate illud 
fere evenisse constat, quod sanctus Hieronymus tempore suo accidisse testatus 
est, tot scilicet fuisse exemplaria, quot codices~; cum unusquisque pro arbitrio 
suo adderet, vel detraheret.

Hujus autem veteris ac Vulgat\ae\ editionis tanta semper fuit auctoritas, 
tamque excellens pr\ae stantia, ut eam ceteris omnibus latinis editionibus 
longe anteferendam esse, apud \ae quos judices in dubium revocari non posset. 
Qui namque in ea libri continentur (ut a majoribus nostris quasi per manus 
traditum nobis est) partim ex sancti Hieronymi translatione, vel emendatione 
suscepti sunt~; partim retenti ex antiquissima quadam editione latina, quam 
sanctus Hieronymus communem et Vulgatam, sanctus Augustinus Italam, sanctus 
Gregorius Veterem translationem appellat.

Ac de Veteris quidem hujus, sive Ital\ae\ editionis sinceritate atque pr\ae 
stantia pr\ae clarum sancti Augustini testimonium exstat in secundo libro 
\emph{De doctrina christiana}, ubi latinis omnibus editionibus, qu\ae\ tunc plurim\ae\ 
circumferebantur, Italam pr\ae ferendam censuit, quod esset, ut ipse loquitur, 
verborum tenacior cum perspicuitate sententi\ae. De sancto vero Hieronymo 
multa exstant veterum Patrum egregia testimonia~: eum enim sanctus Augustinus 
hominem doctissimum, ac trium linguarum peritissimum vocat, atque ejus 
translationem ipsorum quoque hebr\ae orum testimonio veracem esse confirmat. 
Eumdem sanctus Gregorius ita pr\ae dicat, ut ejus translationem, quam novam 
appellat, ex hebr\ae o eloquio cuncta verius transfudisse dicat~: atque idcirco 
dignissimam esse, cui fides in omnibus habeatur. Sanctus autem Isidorus non uno 
in loco Hieronymianam versionem ceteris omnibus anteponit, eamque ab ecclesiis 
christianis communiter recipi ac probari affirmat, quod sit in verbis clarior, 
et veracior in sententiis. Sophronius quoque, vir eruditissimus, sancti 
Hieronymi translationem non latinis modo, sed etiam gr\ae cis valde probari 
animadvertens, tanti eam fecit, ut psalterium et prophetas ex Hieronymi 
versione in gr\ae cum eleganti sermone transtulerit. Porro qui secuti sunt, 
viri doctissimi, Remigius, Beda, Rabanus, Haymo, Anselmus, Petrus Damiani, 
Richardus, Hugo, Bernardus, Rupertus, Petrus Lombardus, Alexander, Albertus, 
Thomas, Bonaventura, ceterique omnes, qui his nongentis annis in Ecclesia 
floruerunt, sancti Hieronymi versione ita sunt usi, ut ceter\ae, qu\ae\ pene 
innumerabiles erant, quasi laps\ae\ de manibus theologorum, penitus 
obsoleverint. Quare non immerito catholica Ecclesia sanctum Hieronymum doctorem 
maximum, atque ad Scripturas sacras interpretandas divinitus excitatum ita 
celebrat, ut jam difficile non sit illorum omnium damnare judicium, qui vel tam 
eximii doctoris lucubrationibus non acquiescunt, vel etiam meliora, aut certe 
paria pr\ae stare se posse confidunt.

Ceterum ne tam fidelis translatio, tamque in omnes partes Ecclesi\ae\ utilis, 
vel injuria temporum, vel impressorum incuria, vel temere emendantium audacia, 
ulla ex parte corrumperetur, eadem sacrosancta Synodus Tridentina illud Decreto 
suo sapienter adjecit, ut h\ae c ipsa vetus ac Vulgata editio emendatissime, 
quoad fieri posset, imprimeretur~: neque ulli liceret eam sine facultate et 
approbatione superiorum excudere. Quo Decreto simul typographorum temeritati ac 
licenti\ae\ modum imposuit, et pastorum Ecclesi\ae\ in tanto bono quam 
diligentissime retinendo, et conservando, vigilantiam, atque industriam 
excitavit.

Et quamvis insignium Academiarum Theologi in editione Vulgata pristino suo 
nitori restituenda magna cum laude laboraverint~; quia tamen in tanta re nulla 
potest esse nimia diligentia, et codices manuscripti complures et vetustiores 
Summi Pontificis jussu conquisiti, atque in Urbem advecti erant~; et demum, 
quoniam executio generalium Conciliorum, et ipsa Scripturarum integritas ac 
puritas ad curam Apostolic\ae\ Sedis potissimum pertinere cognoscitur~; ideo 
Pius \textsc{iv} Pontifex Maximus pro sua in omnes Ecclesi\ae\ partes incredibili 
vigilantia, lectissimis aliquot sanct\ae\ roman\ae\ Ecclesi\ae\ Cardinalibus, 
aliisque tum Sacrarum litterarum, tum variarum linguarum peritissimis viris, 
eam provinciam demandavit, ut Vulgatam editionem latinam, adhibitis 
antiquissimis codicibus manuscriptis, inspectis quoque hebraicis, gr\ae cisque 
bibliorum fontibus~; consultis denique veterum Patrum commentariis, 
accuratissime castigarent. Quod itidem institutum Pius \textsc{v} prosecutus est. Verum 
conventum illum ob varias, gravissimasque Sedis Apostolic\ae\ occupationes 
jamdudum intermissum, Sixtus \textsc{v} divina providentia ad summum Sacerdotium 
evocatus, ardentissimo studio revocavit, et opus tandem confectum typis mandari 
jussit. Quod cum jam esset excusum, et ut in lucem emitteretur, idem Pontifex 
operam daret, animadvertens non pauca in sacra Biblia pr\ae li vitio 
irrepsisse, qu\ae\ iterata diligentia indigere viderentur, totum opus sub 
incudem revocandum censuit atque decrevit. Id vero cum morte pr\ae ventus pr\ae 
stare non potuisset, Gregorius \textsc{xiv} qui, post Urbani \textsc{vii} duodecim dierum 
Pontificatum, Sixto successerat, ejus animi intentionem executus perficere 
aggressus est, amplissimis aliquot Cardinalibus, aliisque doctissimis viris ad 
hoc iterum deputatis. Sed eo quoque, et qui illi successit, Innocentio \textsc{ix} 
brevissimo tempore de hac luce subtractis~; tandem sub initium Pontificatus 
Clementis \textsc{viii} qui nunc Ecclesi\ae\ univers\ae\ gubernacula tenet, opus, in 
quod Sixtus \textsc{v} intenderat, Deo bene juvante perfectum est.

Accipe igitur, christiane lector, eodem Clemente Summo Pontifice annuente, ex 
Vaticana typographia veterem ac Vulgatam sacr\ae\ Scriptur\ae\ editionem, 
quanta fieri potuit diligentia castigatam~: quam quidem sicut omnibus numeris 
absolutam, pro humana imbecillitate affirmare difficile est, ita ceteris 
omnibus, qu\ae\ ad hanc usque diem prodierunt, emendatiorem, purioremque esse, 
minime dubitandum. Et vero quamvis in hac Bibliorum recognitione in codicibus 
manuscriptis, hebr\ae is, gr\ae cisque fontibus, et ipsis veterum Patrum 
commentariis conferendis non mediocre studium adhibitum fuerit~; in hac tamen 
pervulgata lectione sicut nonnulla consulto mutata, ita etiam alia, qu\ae\ 
mutanda videbantur, consulto immutata relicta sunt~; tum quod ita faciendum esse 
ad offensionem populorum vitandam sanctus Hieronymus non semel admonuit~: tum 
quod facile fieri posse credendum est, ut majores nostri, qui ex hebr\ae is, et 
gr\ae cis latina fecerunt, copiam meliorum, et emendatiorum librorum habuerint, 
quam ii, qui post illorum \ae tatem ad nos pervenerunt, qui fortasse tam longo 
tempore identidem describendo minus puri, atque integri evaserunt~; tum denique 
quia sacr\ae\ Congregationi amplissimorum Cardinalium, aliisque eruditissimis 
viris ad hoc opus a Sede Apostolica delectis propositum non fuit, novam aliquam 
editionem cudere, vel antiquum interpretem ulla ex parte corrigere, vel 
emendare~; sed ipsam veterem, ac Vulgatam editionem latinam a mendis veterum 
librariorum, necnon pravarum emendationum erroribus repurgatam, su\ae\ 
pristin\ae\ integritati, ac puritati, quoad ejus fieri potuit, restituere~; 
eaque restituta, ut quam emendatissime imprimeretur juxta Concilii \oe cumenici 
Decretum pro viribus operam dare.

Porro in hac editione nihil non canonicum, nihil ascititium, nihil extraneum 
apponere visum est~: atque ea causa fuit, cur liber tertius et quartus Esdr\ae\ 
inscripti, quos inter canonicos libros sacra Tridentina Synodus non 
annumeravit, ipsa etiam Manass\ae\ regis Oratio, qu\ae\ neque hebraice, neque 
gr\ae ce quidem exstat, neque in manuscriptis antiquioribus invenitur, neque 
pars est ullius canonici libri, extra canonic\ae\ Scriptur\ae\ seriem posita 
sint~: et null\ae\ ad marginem concordanti\ae\ (qu\ae\ posthac inibi apponi non 
prohibentur), null\ae\ not\ae , null\ae\ vari\ae\ lectiones, null\ae\ denique 
pr\ae fationes, nulla argumenta ad librorum initia conspiciantur.

Sed sicut Apostolica Sedes industriam eorum non damnat, qui concordantias 
locorum, varias lectiones, pr\ae fationes sancti Hieronymi, et alia id genus in 
aliis editionibus inseruerunt~; ita quoque non prohibet, quin alio genere 
caracteris in hac ipsa Vaticana editione ejusmodi adjumenta pro studiosorum 
commoditate, atque utilitate in posterum adjiciantur~; ita tamen, ut lectiones 
vari\ae\ ad marginem ipsius textus minime annotentur.\enlargethispage{\baselineskip}
\addtocontents{toc}{\vspace{24pt}\textsc{Antiquum Testamentum}}
\addtocontents{toc}{\vspace{12pt}}

