\bbook{Actus Apostolorum}
{Actus Apostolorum}{images/genese_heading}


\bchapter{1}
\lettrine[lines=10,image=true,loversize=0.05,lraise=-0.03]{P}{}rimum quidem sermonem feci de omnibus, o Theophile, qu\ae\ cœpit Jesus facere et docere
${}^{2}$~usque in diem qua pr\ae cipiens Apostolis per Spiritum Sanctum, quos elegit, assumptus est~:
${}^{3}$~quibus et pr\ae buit seipsum vivum post passionem suam in multis argumentis, per dies quadraginta apparens eis, et loquens de regno Dei.


${}^{4}$~Et convescens, pr\ae cepit eis ab Jerosolymis ne discederent, sed exspectarent promissionem Patris, quam audistis (inquit) per os meum~:
${}^{5}$~quia Joannes quidem baptizavit aqua, vos autem baptizabimini Spiritu Sancto non post multos hos dies.
${}^{6}$~Igitur qui convenerant, interrogabant eum, dicentes~: Domine, si in tempore hoc restitues regnum Isra\"el~?
${}^{7}$~Dixit autem eis~: Non est vestrum nosse tempora vel momenta qu\ae\ Pater posuit in sua potestate~:
${}^{8}$~sed accipietis virtutem supervenientis Spiritus Sancti in vos, et eritis mihi testes in Jerusalem, et in omni Jud\ae a, et Samaria, et usque ad ultimum terr\ae .


${}^{9}$~Et cum h\ae c dixisset, videntibus illis, elevatus est~: et nubes suscepit eum ab oculis eorum.
${}^{10}$~Cumque intuerentur in c\ae lum euntem illum, ecce duo viri astiterunt juxta illos in vestibus albis,
${}^{11}$~qui et dixerunt~: Viri Galil\ae i, quid statis aspicientes in c\ae lum~? Hic Jesus, qui assumptus est a vobis in c\ae lum, sic veniet quemadmodum vidistis eum euntem in c\ae lum.


${}^{12}$~Tunc reversi sunt Jerosolymam a monte qui vocatur Oliveti, qui est juxta Jerusalem, sabbati habens iter.
${}^{13}$~Et cum introissent in cœnaculum, ascenderunt ubi manebant Petrus, et Joannes, Jacobus, et Andreas, Philippus, et Thomas, Bartholom\ae us, et Matth\ae us, Jacobus Alph\ae i, et Simon Zelotes, et Judas Jacobi.
${}^{14}$~Hi omnes erant perseverantes unanimiter in oratione cum mulieribus, et Maria matre Jesu, et fratribus ejus.


${}^{15}$~In diebus illis, exsurgens Petrus in medio fratrum, dixit (erat autem turba hominum simul, fere centum viginti)~:
${}^{16}$~Viri fratres, oportet impleri Scripturam quam pr\ae dixit Spiritus Sanctus per os David de Juda, qui fuit dux eorum qui comprehenderunt Jesum~:
${}^{17}$~qui connumeratus erat in nobis, et sortitus est sortem ministerii hujus.
${}^{18}$~Et hic quidem possedit agrum de mercede iniquitatis, et suspensus crepuit medius~: et diffusa sunt omnia viscera ejus.
${}^{19}$~Et notum factum est omnibus habitantibus Jerusalem, ita ut appellaretur ager ille, lingua eorum, Haceldama, hoc est, ager sanguinis.
${}^{20}$~Scriptum est enim in libro Psalmorum~: Fiat commoratio eorum deserta, et non sit qui inhabitet in ea~: et episcopatum ejus accipiat alter.
${}^{21}$~Oportet ergo ex his viris qui nobiscum sunt congregati in omni tempore quo intravit et exivit inter nos Dominus Jesus,
${}^{22}$~incipiens a baptismate Joannis usque in diem qua assumptus est a nobis, testem resurrectionis ejus nobiscum fieri unum ex istis.
${}^{23}$~Et statuerunt duos, Joseph, qui vocabatur Barsabas, qui cognominatus est Justus, et Mathiam.
${}^{24}$~Et orantes dixerunt~: Tu Domine, qui corda nosti omnium, ostende quem elegeris ex his duobus unum,
${}^{25}$~accipere locum ministerii hujus et apostolatus, de quo pr\ae varicatus est Judas ut abiret in locum suum.
${}^{26}$~Et dederunt sortes eis, et cecidit sors super Mathiam~: et annumeratus est cum undecim Apostolis.

\bchapter{2}
\lettrine[lines=3,image=true,loversize=0.05,lraise=-0.03]{E}{}t cum complerentur dies Pentecostes, erant omnes pariter in eodem loco~:
${}^{2}$~et factus est repente de c\ae lo sonus, tamquam advenientis spiritus vehementis, et replevit totam domum ubi erant sedentes.
${}^{3}$~Et apparuerunt illis dispertit\ae\ lingu\ae\ tamquam ignis, seditque supra singulos eorum~:
${}^{4}$~et repleti sunt omnes Spiritu Sancto, et cœperunt loqui variis linguis, prout Spiritus Sanctus dabat eloqui illis.
${}^{5}$~Erant autem in Jerusalem habitantes Jud\ae i, viri religiosi ex omni natione qu\ae\ sub c\ae lo est.
${}^{6}$~Facta autem hac voce, convenit multitudo, et mente confusa est, quoniam audiebat unusquisque lingua sua illos loquentes.
${}^{7}$~Stupebant autem omnes, et mirabantur, dicentes~: Nonne ecce omnes isti qui loquuntur, Galil\ae i sunt~?
${}^{8}$~et quomodo nos audivimus unusquisque linguam nostram in qua nati sumus~?
${}^{9}$~Parthi, et Medi, et \AE lamit\ae , et qui habitant Mespotamiam, Jud\ae am, et Cappadociam, Pontum, et Asiam,
${}^{10}$~Phrygiam, et Pamphyliam, \AE gyptum, et partes Liby\ae\ qu\ae\ est circa Cyrenen~: et adven\ae\ Romani,
${}^{11}$~Jud\ae i quoque, et Proselyti, Cretes, et Arabes~: audivimus eos loquentes nostris linguis magnalia Dei.
${}^{12}$~Stupebant autem omnes, et mirabantur ad invicem, dicentes~: Quidnam vult hoc esse~?
${}^{13}$~Alii autem irridentes dicebant~: Quia musto pleni sunt isti.


${}^{14}$~Stans autem Petrus cum undecim, levavit vocem suam, et locutus est eis~: Viri Jud\ae i, et qui habitatis Jerusalem universi, hoc vobis notum sit, et auribus percipite verba mea.
${}^{15}$~Non enim, sicut vos \ae stimatis, hi ebrii sunt, cum sit hora diei tertia~:
${}^{16}$~sed hoc est quod dictum est per prophetam Jo\"el~:
\begin{flushleft}\begin{verse}${}^{17}$~Et erit in novissimis diebus, dicit Dominus,\\ effundam de Spiritu meo super omnem carnem~:\\ et prophetabunt filii vestri et fili\ae\ vestr\ae ,\\ et juvenes vestri visiones videbunt,\\ et seniores vestri somnia somniabunt.\\
${}^{18}$~Et quidem super servos meos, et super ancillas meas,\\ in diebus illis effundam de Spiritu meo,\\ et prophetabunt~:\\
${}^{19}$~et dabo prodigia in c\ae lo sursum,\\ et signa in terra deorsum, sanguinem, et ignem, et vaporem fumi~:\\
${}^{20}$~sol convertetur in tenebras,\\ et luna in sanguinem,\\ antequam veniat dies Domini magnus et manifestus.\\
${}^{21}$~Et erit~: omnis quicumque invocaverit nomen Domini, salvus erit.\end{verse}\end{flushleft}


${}^{22}$~Viri Isra\"elit\ae , audite verba h\ae c~: Jesum Nazarenum, virum approbatum a Deo in vobis, virtutibus, et prodigiis, et signis, qu\ae\ fecit Deus per illum in medio vestri, sicut et vos scitis~:
${}^{23}$~hunc, definito consilio et pr\ae scientia Dei traditum, per manus iniquorum affligentes interemistis~:
${}^{24}$~quem Deus suscitavit, solutis doloribus inferni, juxta quod impossibile erat teneri illum ab eo.
${}^{25}$~David enim dicit in eum~: \begin{flushleft}\begin{verse}Providebam Dominum in conspectu meo semper~:\\ quoniam a dextris est mihi, ne commovear~:\\
${}^{26}$~propter hoc l\ae tatum est cor meum, et exsultavit lingua mea,\\ insuper et caro mea requiescet in spe~:\\
${}^{27}$~quoniam non derelinques animam meam in inferno,\\ nec dabis sanctum tuum videre corruptionem.\\
${}^{28}$~Notas mihi fecisti vias vit\ae~:\\ et replebis me jucunditate cum facie tua.\end{verse}\end{flushleft}


${}^{29}$~Viri fratres, liceat audenter dicere ad vos de patriarcha David, quoniam defunctus est, et sepultus~: et sepulchrum ejus est apud nos usque in hodiernum diem.
${}^{30}$~Propheta igitur cum esset, et sciret quia jurejurando jurasset illi Deus de fructu lumbi ejus sedere super sedem ejus~:
${}^{31}$~providens locutus est de resurrectione Christi, quia neque derelictus est in inferno, neque caro ejus vidit corruptionem.
${}^{32}$~Hunc Jesum resuscitavit Deus, cujus omnes nos testes sumus.
${}^{33}$~Dextera igitur Dei exaltatus, et promissione Spiritus Sancti accepta a Patre, effudit hunc, quem vos videtis et auditis.
${}^{34}$~Non enim David ascendit in c\ae lum~: dixit autem ipse~: \begin{flushleft}\begin{verse}Dixit Dominus Domino meo~:\\ Sede a dextris meis,\\
${}^{35}$~donec ponam inimicos tuos\\ scabellum pedum tuorum.\end{verse}\end{flushleft}


${}^{36}$~Certissime sciat ergo omnis domus Isra\"el, quia et Dominum eum et Christum fecit Deus hunc Jesum, quem vos crucifixistis.


${}^{37}$~His autem auditis, compuncti sunt corde, et dixerunt ad Petrum et ad reliquos Apostolos~: Quid faciemus, viri fratres~?
${}^{38}$~Petrus vero ad illos~: Pœnitentiam, inquit, agite, et baptizetur unusquisque vestrum in nomine Jesu Christi in remissionem peccatorum vestrorum~: et accipietis donum Spiritus Sancti.
${}^{39}$~Vobis enim est repromissio, et filiis vestris, et omnibus qui longe sunt, quoscumque advocaverit Dominus Deus noster.
${}^{40}$~Aliis etiam verbis plurimis testificatus est, et exhortabatur eos, dicens~: Salvamini a generatione ista prava.
${}^{41}$~Qui ergo receperunt sermonem ejus, baptizati sunt~: et apposit\ae\ sunt in die illa anim\ae\ circiter tria millia.


${}^{42}$~Erant autem perseverantes in doctrina Apostolorum, et communicatione fractionis panis, et orationibus.
${}^{43}$~Fiebat autem omni anim\ae\ timor~: multa quoque prodigia et signa per Apostolos in Jerusalem fiebant, et metus erat magnus in universis.
${}^{44}$~Omnes etiam qui credebant, erant pariter, et habebant omnia communia.
${}^{45}$~Possessiones et substantias vendebant, et dividebant illa omnibus, prout cuique opus erat.
${}^{46}$~Quotidie quoque perdurantes unanimiter in templo, et frangentes circa domos panem, sumebant cibum cum exsultatione et simplicitate cordis,
${}^{47}$~collaudantes Deum et habentes gratiam ad omnem plebem. Dominus autem augebat qui salvi fierent quotidie in idipsum.

\bchapter{3}
\lettrine[lines=3,image=true,loversize=0.05,lraise=-0.03]{P}{}etrus autem et Joannes ascendebant in templum ad horam orationis nonam.
${}^{2}$~Et quidam vir, qui erat claudus ex utero matris su\ae , bajulabatur~: quem ponebant quotidie ad portam templi, qu\ae\ dicitur Speciosa, ut peteret eleemosynam ab intro\"euntibus in templum.
${}^{3}$~Is cum vidisset Petrum et Joannem incipientes introire in templum, rogabat ut eleemosynam acciperet.
${}^{4}$~Intuens autem in eum Petrus cum Joanne, dixit~: Respice in nos.
${}^{5}$~At ille intendebat in eos, sperans se aliquid accepturum ab eis.
${}^{6}$~Petrus autem dixit~: Argentum et aurum non est mihi~: quod autem habeo, hoc tibi do~: in nomine Jesu Christi Nazareni surge, et ambula.
${}^{7}$~Et apprehensa manu ejus dextera, allevavit eum, et protinus consolidat\ae\ sunt bases ejus et plant\ae .
${}^{8}$~Et exiliens stetit, et ambulabat~: et intravit cum illis in templum ambulans, et exiliens, et laudans Deum.
${}^{9}$~Et vidit omnis populus eum ambulantem et laudantem Deum.
${}^{10}$~Cognoscebant autem illum, quod ipse erat qui ad eleemosynam sedebat ad Speciosam portam templi~: et impleti sunt stupore et extasi in eo quod contigerat illi.
${}^{11}$~Cum teneret autem Petrum et Joannem, cucurrit omnis populus ad eos ad porticum qu\ae\ appellatur Salomonis, stupentes.


${}^{12}$~Videns autem Petrus, respondit ad populum~: Viri Isra\"elit\ae , quid miramini in hoc, aut nos quid intuemini, quasi nostra virtute aut potestate fecerimus hunc ambulare~?
${}^{13}$~Deus Abraham, et Deus Isaac, et Deus Jacob, Deus patrum nostrorum glorificavit filium suum Jesum, quem vos quidem tradidistis, et negastis ante faciem Pilati, judicante illo dimitti.
${}^{14}$~Vos autem sanctum et justum negastis, et petistis virum homicidam donari vobis~:
${}^{15}$~auctorem vero vit\ae\ interfecistis, quem Deus suscitavit a mortuis, cujus nos testes sumus.
${}^{16}$~Et in fide nominis ejus, hunc quem vos vidistis et nostis, confirmavit nomen ejus~: et fides, qu\ae\ per eum est, dedit integram sanitatem istam in conspectu omnium vestrum.
${}^{17}$~Et nunc, fratres, scio quia per ignorantiam fecistis, sicut et principes vestri.
${}^{18}$~Deus autem, qu\ae\ pr\ae nuntiavit per os omnium prophetarum, pati Christum suum, sic implevit.
${}^{19}$~Pœnitemini igitur et convertimini, ut deleantur peccata vestra~:
${}^{20}$~ut cum venerint tempora refrigerii a conspectu Domini, et miserit eum qui pr\ae dicatus est vobis, Jesum Christum,
${}^{21}$~quem oportet quidem c\ae lum suscipere usque in tempora restitutionis omnium qu\ae\ locutus est Deus per os sanctorum suorum a s\ae culo prophetarum.
${}^{22}$~Moyses quidem dixit~: Quoniam prophetam suscitabit vobis Dominus Deus vester de fratribus vestris, tamquam me~: ipsum audietis juxta omnia qu\ae cumque locutus fuerit vobis.
${}^{23}$~Erit autem~: omnis anima qu\ae\ non audierit prophetam illum, exterminabitur de plebe.
${}^{24}$~Et omnes prophet\ae , a Samuel et deinceps, qui locuti sunt, annuntiaverunt dies istos.
${}^{25}$~Vos estis filii prophetarum, et testamenti quod disposuit Deus ad patres nostros, dicens ad Abraham~: Et in semine tuo benedicentur omnes famili\ae\ terr\ae .
${}^{26}$~Vobis primum Deus suscitans filium suum, misit eum benedicentem vobis~: ut convertat se unusquisque a nequitia sua.

\bchapter{4}
\lettrine[lines=3,image=true,loversize=0.05,lraise=-0.03]{L}{}oquentibus autem illis ad populum, supervenerunt sacerdotes, et magistratus templi, et sadduc\ae i,
${}^{2}$~dolentes quod docerent populum, et annuntiarent in Jesu resurrectionem ex mortuis~:
${}^{3}$~et injecerunt in eos manus, et posuerunt eos in custodiam in crastinum~: erat enim jam vespera.
${}^{4}$~Multi autem eorum qui audierant verbum, crediderunt~: et factus est numerus virorum quinque millia.
${}^{5}$~Factum est autem in crastinum, ut congregarentur principes eorum, et seniores, et scrib\ae , in Jerusalem~:
${}^{6}$~et Annas princeps sacerdotum, et Caiphas, et Joannes, et Alexander, et quotquot erant de genere sacerdotali.
${}^{7}$~Et statuentes eos in medio, interrogabant~: In qua virtute, aut in quo nomine fecistis hoc vos~?


${}^{8}$~Tunc repletus Spiritu Sancto Petrus, dixit ad eos~: Principes populi, et seniores, audite~:
${}^{9}$~si nos hodie dijudicamur in benefacto hominis infirmi, in quo iste salvus factus est,
${}^{10}$~notum sit omnibus vobis, et omni plebi Isra\"el, quia in nomine Domini nostri Jesu Christi Nazareni, quem vos crucifixistis, quem Deus suscitavit a mortuis, in hoc iste astat coram vobis sanus.
${}^{11}$~Hic est lapis qui reprobatus est a vobis \ae dificantibus, qui factus est in caput anguli~:
${}^{12}$~et non est in alio aliquo salus. Nec enim aliud nomen est sub c\ae lo datum hominibus, in quo oporteat nos salvos fieri.


${}^{13}$~Videntes autem Petri constantiam, et Joannis, comperto quod homines essent sine litteris, et idiot\ae , admirabantur, et cognoscebant eos quoniam cum Jesu fuerant~:
${}^{14}$~hominem quoque videntes stantem cum eis, qui curatus fuerat, nihil poterant contradicere.
${}^{15}$~Jusserunt autem eos foras extra concilium secedere~: et conferebant ad invicem,
${}^{16}$~dicentes~: Quid faciemus hominibus istis~? quoniam quidem notum signum factum est per eos omnibus habitantibus Jerusalem~: manifestum est, et non possumus negare.
${}^{17}$~Sed ne amplius divulgetur in populum, comminemur eis ne ultra loquantur in nomine hoc ulli hominum.
${}^{18}$~Et vocantes eos, denuntiaverunt ne omnino loquerentur neque docerent in nomine Jesu.
${}^{19}$~Petrus vero et Joannes respondentes, dixerunt ad eos~: Si justum est in conspectu Dei vos potius audire quam Deum, judicate.
${}^{20}$~Non enim possumus qu\ae\ vidimus et audivimus non loqui.
${}^{21}$~At illi comminantes dimiserunt eos, non invenientes quomodo punirent eos propter populum~: quia omnes clarificabant id quod factum fuerat in eo quod acciderat.
${}^{22}$~Annorum enim erat amplius quadraginta homo, in quo factum fuerat signum istud sanitatis.


${}^{23}$~Dimissi autem venerunt ad suos, et annuntiaverunt eis quanta ad eos principes sacerdotum et seniores dixissent.
${}^{24}$~Qui cum audissent, unanimiter levaverunt vocem ad Deum, et dixerunt~: Domine, tu es qui fecisti c\ae lum et terram, mare et omnia qu\ae\ in eis sunt~:
${}^{25}$~qui Spiritu Sancto per os patris nostri David pueri tui dixisti~: \begin{flushleft}\begin{verse}Quare fremuerunt gentes,\\ et populi meditati sunt inania~?\\
${}^{26}$~Astiterunt reges terr\ae ,\\ et principes convenerunt in unum\\ adversus Dominum, et adversus Christum ejus.\end{verse}\end{flushleft}


${}^{27}$~Convenerunt enim vere in civitate ista adversus sanctum puerum tuum Jesum, quem unxisti, Herodes et Pontius Pilatus, cum gentibus et populis Isra\"el,
${}^{28}$~facere qu\ae\ manus tua et consilium tuum decreverunt fieri.
${}^{29}$~Et nunc, Domine, respice in minas eorum, et da servis tuis cum omni fiducia loqui verbum tuum,
${}^{30}$~in eo quod manum tuam extendas ad sanitates, et signa, et prodigia fieri per nomen sancti filii tui Jesu.
${}^{31}$~Et cum orassent, motus est locus in quo erant congregati~: et repleti sunt omnes Spiritu Sancto, et loquebantur verbum Dei cum fiducia.


${}^{32}$~Multitudinis autem credentium erat cor unum et anima una~: nec quisquam eorum qu\ae\ possidebat, aliquid suum esse dicebat, sed erant illis omnia communia.
${}^{33}$~Et virtute magna reddebant Apostoli testimonium resurrectionis Jesu Christi Domini nostri~: et gratia magna erat in omnibus illis.
${}^{34}$~Neque enim quisquam egens erat inter illos. Quotquot enim possessores agrorum aut domorum erant, vendentes afferebant pretia eorum qu\ae\ vendebant,
${}^{35}$~et ponebant ante pedes Apostolorum. Dividebatur autem singulis prout cuique opus erat.
${}^{36}$~Joseph autem, qui cognominatus est Barnabas ab Apostolis (quod est interpretatum, Filius consolationis), Levites, Cyprius genere,
${}^{37}$~cum haberet agrum, vendidit eum, et attulit pretium, et posuit ante pedes Apostolorum.

\bchapter{5}
\lettrine[lines=3,image=true,loversize=0.05,lraise=-0.03]{V}{}ir autem quidam nomine Ananias, cum Saphira uxore suo vendidit agrum,
${}^{2}$~et fraudavit de pretio agri, conscia uxore sua~: et afferens partem quamdam, ad pedes Apostolorum posuit.
${}^{3}$~Dixit autem Petrus~: Anania, cur tentavit Satanas cor tuum, mentiri te Spiritui Sancto, et fraudare de pretio agri~?
${}^{4}$~nonne manens tibi manebat, et venundatum in tua erat potestate~? quare posuisti in corde tuo hanc rem~? non es mentitus hominibus, sed Deo.
${}^{5}$~Audiens autem Ananias h\ae c verba, cecidit, et expiravit. Et factus est timor magnus super omnes qui audierunt.
${}^{6}$~Surgentes autem juvenes amoverunt eum, et efferentes sepelierunt.
${}^{7}$~Factum est autem quasi horarum trium spatium, et uxor ipsius, nesciens quod factum fuerat, introivit.
${}^{8}$~Dixit autem ei Petrus~: Dic mihi mulier, si tanti agrum vendidistis~? At illa dixit~: Etiam tanti.
${}^{9}$~Petrus autem ad eam~: Quid utique convenit vobis tentare Spiritum Domini~? Ecce pedes eorum qui sepelierunt virum tuum ad ostium, et efferent te.
${}^{10}$~Confestim cecidit ante pedes ejus, et expiravit. Intrantes autem juvenes invenerunt illam mortuam~: et extulerunt, et sepelierunt ad virum suum.
${}^{11}$~Et factus est timor magnus in universa ecclesia, et in omnes qui audierunt h\ae c.


${}^{12}$~Per manus autem Apostolorum fiebant signa et prodigia multa in plebe. Et erant unanimiter omnes in porticu Salomonis.
${}^{13}$~Ceterorum autem nemo audebat se conjungere illis~: sed magnificabat eos populus.
${}^{14}$~Magis autem augebatur credentium in Domino multitudo virorum ac mulierum,
${}^{15}$~ita ut in plateas ejicerent infirmos, et ponerent in lectulis et grabatis, ut, veniente Petro, saltem umbra illius obumbraret quemquam illorum, et liberarentur ab infirmitatibus suis.
${}^{16}$~Concurrebat autem et multitudo vicinarum civitatum Jerusalem, afferentes \ae gros, et vexatos a spiritibus immundis~: qui curabantur omnes.


${}^{17}$~Exsurgens autem princeps sacerdotum, et omnes qui cum illo erant (qu\ae\ est h\ae resis sadduc\ae orum), repleti sunt zelo~:
${}^{18}$~et injecerunt manus in Apostolos, et posuerunt eos in custodia publica.
${}^{19}$~Angelus autem Domini per noctem aperiens januas carceris, et educens eos, dixit~:
${}^{20}$~Ite, et stantes loquimini in templo plebi omnia verba vit\ae\ hujus.
${}^{21}$~Qui cum audissent, intraverunt diluculo in templum, et docebant. Adveniens autem princeps sacerdotum, et qui cum eo erant, convocaverunt concilium, et omnes seniores filiorum Isra\"el~: et miserunt ad carcerem ut adducerentur.
${}^{22}$~Cum autem venissent ministri, et aperto carcere non invenissent illos, reversi nuntiaverunt,
${}^{23}$~dicentes~: Carcerem quidem invenimus clausum cum omni diligentia, et custodes stantes ante januas~: aperientes autem neminem intus invenimus.
${}^{24}$~Ut autem audierunt hos sermones magistratus templi et principes sacerdotum, ambigebant de illis quidnam fieret.
${}^{25}$~Adveniens autem quidam, nuntiavit eis~: Quia ecce viri quos posuistis in carcerem, sunt in templo, stantes et docentes populum.


${}^{26}$~Tunc abiit magistratus cum ministris, et adduxit illos sine vi~: timebant enim populum ne lapidarentur.
${}^{27}$~Et cum adduxissent illos, statuerunt in concilio~: et interrogavit eos princeps sacerdotum,
${}^{28}$~dicens~: Pr\ae cipiendo pr\ae cepimus vobis ne doceretis in nomine isto, et ecce replestis Jerusalem doctrina vestra~: et vultis inducere super nos sanguinem hominis istius.


${}^{29}$~Respondens autem Petrus et Apostoli, dixerunt~: Obedire oportet Deo magis quam hominibus.
${}^{30}$~Deus patrum nostrorum suscitavit Jesum, quem vos interemistis, suspendentes in ligno.
${}^{31}$~Hunc principem et salvatorem Deus exaltavit dextera sua ad dandam pœnitentiam Isra\"eli, et remissionem peccatorum~:
${}^{32}$~et nos sumus testes horum verborum, et Spiritus Sanctus, quem dedit Deus omnibus obedientibus sibi.
${}^{33}$~H\ae c cum audissent, dissecabantur, et cogitabant interficere illos.


${}^{34}$~Surgens autem quidam in concilio pharis\ae us, nomine Gamaliel, legis doctor, honorabilis univers\ae\ plebi, jussit foras ad breve homines fieri,
${}^{35}$~dixitque ad illos~: Viri Isra\"elit\ae , attendite vobis super hominibus istis quid acturi sitis.
${}^{36}$~Ante hos enim dies extitit Theodas, dicens se esse aliquem, cui consensit numerus virorum circiter quadringentorum~: qui occisus est, et omnes qui credebant ei, dissipati sunt, et redacti ad nihilum.
${}^{37}$~Post hunc extitit Judas Galil\ae us in diebus professionis, et avertit populum post se~: et ipse periit, et omnes quotquot consenserunt ei, dispersi sunt.
${}^{38}$~Et nunc itaque dico vobis, discedite ab hominibus istis, et sinite illos~: quoniam si est ex hominibus consilium hoc aut opus, dissolvetur~:
${}^{39}$~si vero ex Deo est, non poteritis dissolvere illud, ne forte et Deo repugnare inveniamini. Consenserunt autem illi.
${}^{40}$~Et convocantes Apostolos, c\ae sis denuntiaverunt ne omnino loquerentur in nomine Jesu, et dimiserunt eos.
${}^{41}$~Et illi quidem ibant gaudentes a conspectu concilii, quoniam digni habiti sunt pro nomine Jesu contumeliam pati.
${}^{42}$~Omni autem die non cessabant in templo et circa domos, docentes et evangelizantes Christum Jesum.

\bchapter{6}
\lettrine[lines=3,image=true,loversize=0.05,lraise=-0.03]{I}{}n diebus illis, crescente numero discipulorum, factum est murmur Gr\ae corum adversus Hebr\ae os, eo quod despicerentur in ministerio quotidiano vidu\ae\ eorum.
${}^{2}$~Convocantes autem duodecim multitudinem discipulorum, dixerunt~: Non est \ae quum nos derelinquere verbum Dei, et ministrare mensis.
${}^{3}$~Considerate ergo, fratres, viros ex vobis boni testimonii septem, plenos Spiritu Sancto et sapientia, quos constituamus super hoc opus.
${}^{4}$~Nos vero orationi et ministerio verbi instantes erimus.
${}^{5}$~Et placuit sermo coram omni multitudine. Et elegerunt Stephanum, virum plenum fide et Spiritu Sancto, et Philippum, et Prochorum, et Nicanorem, et Timonem, et Parmenam, et Nicolaum advenam Antiochenum.
${}^{6}$~Hos statuerunt ante conspectum Apostolorum~: et orantes imposuerunt eis manus.
${}^{7}$~Et verbum Domini crescebat, et multiplicabatur numerus discipulorum in Jerusalem valde~: multa etiam turba sacerdotum obediebat fidei.


${}^{8}$~Stephanus autem plenus gratia et fortitudine, faciebat prodigia et signa magna in populo.
${}^{9}$~Surrexerunt autem quidam de synagoga qu\ae\ appellatur Libertinorum, et Cyrenensium, et Alexandrinorum, et eorum qui erant a Cilicia, et Asia, disputantes cum Stephano~:
${}^{10}$~et non poterant resistere sapienti\ae , et Spiritui qui loquebatur.
${}^{11}$~Tunc summiserunt viros, qui dicerent se audivisse eum dicentem verba blasphemi\ae\ in Moysen et in Deum.
${}^{12}$~Commoverunt itaque plebem, et seniores, et scribas~: et concurrentes rapuerunt eum, et adduxerunt in concilium,
${}^{13}$~et statuerunt falsos testes, qui dicerent~: Homo iste non cessat loqui verba adversus locum sanctum, et legem~:
${}^{14}$~audivimus enim eum dicentem quoniam Jesus Nazarenus hic destruet locum istum, et mutabit traditiones quas tradidit nobis Moyses.
${}^{15}$~Et intuentes eum omnes qui sedebant in concilio, viderunt faciem ejus tamquam faciem angeli.

\bchapter{7}
\lettrine[lines=3,image=true,loversize=0.05,lraise=-0.03]{D}{}ixit autem princeps sacerdotum~: Si h\ae c ita se habent~?
${}^{2}$~Qui ait~: Viri fratres et patres, audite~: Deus glori\ae\ apparuit patri nostro Abrah\ae\ cum esset in Mesopotamia, priusquam moraretur in Charan,
${}^{3}$~et dixit ad illum~: Exi de terra tua, et de cognatione tua, et veni in terram quam monstravero tibi.
${}^{4}$~Tunc exiit de terra Chald\ae orum, et habitavit in Charan. Et inde, postquam mortuus est pater ejus, transtulit illum in terram istam, in qua nunc vos habitatis.
${}^{5}$~Et non dedit illi h\ae reditatem in ea, nec passum pedis~: sed repromisit dare illi eam in possessionem, et semini ejus post ipsum, cum non haberet filium.
${}^{6}$~Locutus est autem ei Deus~: Quia erit semen ejus accola in terra aliena, et servituti eos subjicient, et male tractabunt eos annis quadringentis~:
${}^{7}$~et gentem cui servierint, judicabo ego, dixit Dominus~: et post h\ae c exibunt, et servient mihi in loco isto.
${}^{8}$~Et dedit illi testamentum circumcisionis~: et sic genuit Isaac, et circumcidit eum die octavo~: et Isaac, Jacob~: et Jacob, duodecim patriarchas.
${}^{9}$~Et patriarch\ae\ \ae mulantes, Joseph vendiderunt in \AE gyptum~: et erat Deus cum eo,
${}^{10}$~et eripuit eum ex omnibus tribulationibus ejus, et dedit ei gratiam et sapientiam in conspectu pharaonis regis \AE gypti~: et constituit eum pr\ae positum super \AE gyptum, et super omnem domum suam.
${}^{11}$~Venit autem fames in universam \AE gyptum et Chanaan, et tribulatio magna~: et non inveniebant cibos patres nostri.
${}^{12}$~Cum audisset autem Jacob esse frumentum in \AE gypto, misit patres nostros primum~:
${}^{13}$~et in secundo cognitus est Joseph a fratribus suis, et manifestatum est Pharaoni genus ejus.
${}^{14}$~Mittens autem Joseph, accersivit Jacob patrem suum et omnem cognationem suam, in animabus septuaginta quinque.
${}^{15}$~Et descendit Jacob in \AE gyptum~: et defunctus est ipse, et patres nostri.
${}^{16}$~Et translati sunt in Sichem, et positi sunt in sepulchro, quod emit Abraham pretio argenti a filiis Hemor filii Sichem.


${}^{17}$~Cum autem appropinquaret tempus promissionis quam confessus erat Deus Abrah\ae , crevit populus, et multiplicatus est in \AE gypto,
${}^{18}$~quoadusque surrexit alius rex in \AE gypto, qui non sciebat Joseph.
${}^{19}$~Hic circumveniens genus nostrum, afflixit patres nostros ut exponerent infantes suos, ne vivificarentur.
${}^{20}$~Eodem tempore natus est Moyses, et fuit gratus Deo~: qui nutritus est tribus mensibus in domo patris sui.
${}^{21}$~Exposito autem illo, sustulit eum filia Pharaonis, et nutrivit eum sibi in filium.
${}^{22}$~Et eruditus est Moyses omni sapientia \AE gyptiorum, et erat potens in verbis et in operibus suis.
${}^{23}$~Cum autem impleretur ei quadraginta annorum tempus, ascendit in cor ejus ut visitaret fratres suos filios Isra\"el.
${}^{24}$~Et cum vidisset quemdam injuriam patientem, vindicavit illum, et fecit ultionem ei qui injuriam sustinebat, percusso \AE gyptio.
${}^{25}$~Existimabat autem intelligere fratres, quoniam Deus per manum ipsius daret salutem illis~: at illi non intellexerunt.
${}^{26}$~Sequenti vero die apparuit illis litigantibus~: et reconciliabat eos in pace, dicens~: Viri, fratres estis~: ut quid nocetis alterutrum~?
${}^{27}$~Qui autem injuriam faciebat proximo, repulit eum, dicens~: Quis te constituit principem et judicem super nos~?
${}^{28}$~Numquid interficere me tu vis, quemadmodum interfecisti heri \AE gyptium~?
${}^{29}$~Fugit autem Moyses in verbo isto~: et factus est advena in terra Madian, ubi generavit filios duos.
${}^{30}$~Et expletis annis quadraginta, apparuit illi in deserto montis Sina angelus in igne flamm\ae\ rubi.
${}^{31}$~Moyses autem videns, admiratus est visum. Et accedente illo ut consideraret, facta est ad eum vox Domini, dicens~:
${}^{32}$~Ego sum Deus patrum tuorum, Deus Abraham, Deus Isaac, et Deus Jacob. Tremefactus autem Moyses, non audebat considerare.
${}^{33}$~Dixit autem illi Dominus~: Solve calceamentum pedum tuorum~: locus enim in quo stas, terra sancta est.
${}^{34}$~Videns vidi afflictionem populi mei qui est in \AE gypto, et gemitum eorum audivi, et descendi liberare eos. Et nunc veni, et mittam te in \AE gyptum.
${}^{35}$~Hunc Moysen, quem negaverunt, dicentes~: Quis te constituit principem et judicem~? hunc Deus principem et redemptorem misit, cum manu angeli qui apparuit illi in rubo.
${}^{36}$~Hic eduxit illos faciens prodigia et signa in terra \AE gypti, et in rubro mari, et in deserto annis quadraginta.
${}^{37}$~Hic est Moyses, qui dixit filiis Isra\"el~: Prophetam suscitabit vobis Deus de fratribus vestris, tamquam me~: ipsum audietis.
${}^{38}$~Hic est qui fuit in ecclesia in solitudine cum angelo, qui loquebatur ei in monte Sina, et cum patribus nostris~: qui accepit verba vit\ae\ dare nobis.
${}^{39}$~Cui noluerunt obedire patres nostri~: sed repulerunt, et aversi sunt cordibus suis in \AE gyptum,
${}^{40}$~dicentes ad Aaron~: Fac nobis deos qui pr\ae cedant nos~: Moyses enim hic, qui eduxit nos de terra \AE gypti, nescimus quid factum sit ei.
${}^{41}$~Et vitulum fecerunt in diebus illis, et obtulerunt hostiam simulacro, et l\ae tabantur in operibus manuum suarum.
${}^{42}$~Convertit autem Deus, et tradidit eos servire militi\ae\ c\ae li, sicut scriptum est in libro prophetarum~: \begin{flushleft}\begin{verse}Numquid victimas et hostias obtulistis mihi\\ annis quadraginta in deserto, domus Isra\"el~?\\
${}^{43}$~Et suscepistis tabernaculum Moloch,\\ et sidus dei vestri Rempham,\\ figuras quas fecistis adorare eas~:\\ et transferam vos trans Babylonem.\end{verse}\end{flushleft}


${}^{44}$~Tabernaculum testimonii fuit cum patribus nostris in deserto, sicut disposuit illis Deus loquens ad Moysen, ut faceret illud secundum formam quam viderat.
${}^{45}$~Quod et induxerunt, suscipientes patres nostri cum Jesu in possessionem gentium quas expulit Deus a facie patrum nostrorum, usque in diebus David,
${}^{46}$~qui invenit gratiam ante Deum, et petiit ut inveniret tabernaculum Deo Jacob.
${}^{47}$~Salomon autem \ae dificavit illi domum.
${}^{48}$~Sed non Excelsus in manufactis habitat, sicut propheta dicit~:
\begin{flushleft}\begin{verse}${}^{49}$~C\ae lum mihi sedes est~:\\ terra autem scabellum pedum meorum.\\ Quam domum \ae dificabitis mihi~? dicit Dominus~:\\ aut quis locus requietionis me\ae\ est~?\\
${}^{50}$~Nonne manus mea fecit h\ae c omnia~?\end{verse}\end{flushleft}


${}^{51}$~Dura cervice, et incircumcisis cordibus et auribus, vos semper Spiritui Sancto resistitis~: sicut patres vestri, ita et vos.
${}^{52}$~Quem prophetarum non sunt persecuti patres vestri~? et occiderunt eos qui pr\ae nuntiabant de adventu Justi, cujus vos nunc proditores et homicid\ae\ fuistis~:
${}^{53}$~qui accepistis legem in dispositione angelorum, et non custodistis.


${}^{54}$~Audientes autem h\ae c, dissecabantur cordibus suis, et stridebant dentibus in eum.
${}^{55}$~Cum autem esset plenus Spiritu Sancto, intendens in c\ae lum, vidit gloriam Dei, et Jesum stantem a dextris Dei. Et ait~: Ecce video c\ae los apertos, et Filium hominis stantem a dextris Dei.
${}^{56}$~Exclamantes autem voce magna continuerunt aures suas, et impetum fecerunt unanimiter in eum.
${}^{57}$~Et ejicientes eum extra civitatem, lapidabant~: et testes deposuerunt vestimenta sua secus pedes adolescentis qui vocabatur Saulus.
${}^{58}$~Et lapidabant Stephanum invocantem, et dicentem~: Domine Jesu, suscipe spiritum meum.
${}^{59}$~Positis autem genibus, clamavit voce magna, dicens~: Domine, ne statuas illis hoc peccatum. Et cum hoc dixisset, obdormivit in Domino. Saulus autem erat consentiens neci ejus.

\bchapter{8}
\lettrine[lines=3,image=true,loversize=0.05,lraise=-0.03]{F}{}acta est autem in illa die persecutio magna in ecclesia qu\ae\ erat Jerosolymis, et omnes dispersi sunt per regiones Jud\ae \ae\ et Samari\ae\ pr\ae ter Apostolos.
${}^{2}$~Curaverunt autem Stephanum viri timorati, et fecerunt planctum magnum super eum.
${}^{3}$~Saulus autem devastabat ecclesiam per domos intrans, et trahens viros ac mulieres, tradebat in custodiam.


${}^{4}$~Igitur qui dispersi erant pertransibant, evangelizantes verbum Dei.
${}^{5}$~Philippus autem descendens in civitatem Samari\ae , pr\ae dicabat illis Christum.
${}^{6}$~Intendebant autem turb\ae\ his qu\ae\ a Philippo dicebantur, unanimiter audientes, et videntes signa qu\ae\ faciebat.
${}^{7}$~Multi enim eorum qui habebant spiritus immundos, clamantes voce magna exibant. Multi autem paralytici et claudi curati sunt.
${}^{8}$~Factum est ergo gaudium magnum in illa civitate.
${}^{9}$~Vir autem quidam nomine Simon, qui ante fuerat in civitate magus, seducens gentem Samari\ae , dicens se esse aliquem magnum~:
${}^{10}$~cui auscultabant omnes a minimo usque ad maximum, dicentes~: Hic est virtus Dei, qu\ae\ vocatur magna.
${}^{11}$~Attendebant autem eum~: propter quod multo tempore magiis suis dementasset eos.
${}^{12}$~Cum vero credidissent Philippo evangelizanti de regno Dei, in nomine Jesu Christi baptizabantur viri ac mulieres.
${}^{13}$~Tunc Simon et ipse credidit~: et cum baptizatus esset, adh\ae rebat Philippo. Videns etiam signa et virtutes maximas fieri, stupens admirabatur.


${}^{14}$~Cum autem audissent Apostoli qui erant Jerosolymis, quod recepisset Samaria verbum Dei, miserunt ad eos Petrum et Joannem.
${}^{15}$~Qui cum venissent, oraverunt pro ipsis ut acciperent Spiritum Sanctum~:
${}^{16}$~nondum enim in quemquam illorum venerat, sed baptizati tantum erant in nomine Domini Jesu.
${}^{17}$~Tunc imponebant manus super illos, et accipiebant Spiritum Sanctum.
${}^{18}$~Cum vidisset autem Simon quia per impositionem manus Apostolorum daretur Spiritus Sanctus, obtulit eis pecuniam,
${}^{19}$~dicens~: Date et mihi hanc potestatem, ut cuicumque imposuero manus, accipiat Spiritum Sanctum. Petrus autem dixit ad eum~:
${}^{20}$~Pecunia tua tecum sit in perditionem~: quoniam donum Dei existimasti pecunia possideri.
${}^{21}$~Non est tibi pars neque sors in sermone isto~: cor enim tuum non est rectum coram Deo.
${}^{22}$~Pœnitentiam itaque age ab hac nequitia tua~: et roga Deum, si forte remittatur tibi h\ae c cogitatio cordis tui.
${}^{23}$~In felle enim amaritudinis, et obligatione iniquitatis, video te esse.
${}^{24}$~Respondens autem Simon, dixit~: Precamini vos pro me ad Dominum, ut nihil veniat super me horum qu\ae\ dixistis.
${}^{25}$~Et illi quidem testificati, et locuti verbum Domini, redibant Jerosolymam, et multis regionibus Samaritanorum evangelizabant.


${}^{26}$~Angelus autem Domini locutus est ad Philippum, dicens~: Surge, et vade contra meridianum, ad viam qu\ae\ descendit ab Jerusalem in Gazam~: h\ae c est deserta.
${}^{27}$~Et surgens abiit. Et ecce vir \AE thiops, eunuchus, potens Candacis regin\ae\ \AE thiopum, qui erat super omnes gazas ejus, venerat adorare in Jerusalem~:
${}^{28}$~et revertebatur sedens super currum suum, legensque Isaiam prophetam.
${}^{29}$~Dixit autem Spiritus Philippo~: Accede, et adjunge te ad currum istum.
${}^{30}$~Accurrens autem Philippus, audivit eum legentem Isaiam prophetam, et dixit~: Putasne intelligis qu\ae\ legis~?
${}^{31}$~Qui ait~: Et quomodo possum, si non aliquis ostenderit mihi~? Rogavitque Philippum ut ascenderet, et sederet secum.
${}^{32}$~Locus autem Scriptur\ae\ quem legebat, erat hic~: \begin{flushleft}\begin{verse}Tamquam ovis ad occisionem ductus est~:\\ et sicut agnus coram tondente se, sine voce,\\ sic non aperuit os suum.\\
${}^{33}$~In humilitate judicium ejus sublatum est.\\ Generationem ejus quis enarrabit~?\\ quoniam tolletur de terra vita ejus.\end{verse}\end{flushleft}


${}^{34}$~Respondens autem eunuchus Philippo, dixit~: Obsecro te, de quo propheta dicit hoc~? de se, an de alio aliquo~?
${}^{35}$~Aperiens autem Philippus os suum, et incipiens a Scriptura ista, evangelizavit illi Jesum.
${}^{36}$~Et dum irent per viam, venerunt ad quamdam aquam~: et ait eunuchus~: Ecce aqua~: quid prohibet me baptizari~?
${}^{37}$~Dixit autem Philippus~: Si credis ex toto corde, licet. Et respondens ait~: Credo Filium Dei esse Jesum Christum.
${}^{38}$~Et jussit stare currum~: et descenderunt uterque in aquam, Philippus et eunuchus, et baptizavit eum.
${}^{39}$~Cum autem ascendissent de aqua, Spiritus Domini rapuit Philippum, et amplius non vidit eum eunuchus. Ibat autem per viam suam gaudens.
${}^{40}$~Philippus autem inventus est in Azoto, et pertransiens evangelizabat civitatibus cunctis, donec veniret C\ae saream.

\bchapter{9}
\lettrine[lines=3,image=true,loversize=0.05,lraise=-0.03]{S}{}aulus autem adhuc spirans minarum et c\ae dis in discipulos Domini, accessit ad principem sacerdotum,
${}^{2}$~et petiit ab eo epistolas in Damascum ad synagogas~: ut si quos invenisset hujus vi\ae\ viros ac mulieres, vinctos perduceret in Jerusalem.


${}^{3}$~Et cum iter faceret, contigit ut appropinquaret Damasco~: et subito circumfulsit eum lux de c\ae lo.
${}^{4}$~Et cadens in terram audivit vocem dicentem sibi~: Saule, Saule, quid me persequeris~?
${}^{5}$~Qui dixit~: Quis es, domine~? Et ille~: Ego sum Jesus, quem tu persequeris~: durum est tibi contra stimulum calcitrare.
${}^{6}$~Et tremens ac stupens dixit~: Domine, quid me vis facere~?
${}^{7}$~Et Dominus ad eum~: Surge, et ingredere civitatem, et ibi dicetur tibi quid te oporteat facere. Viri autem illi qui comitabantur cum eo, stabant stupefacti, audientes quidem vocem, neminem autem videntes.
${}^{8}$~Surrexit autem Saulus de terra, apertisque oculis nihil videbat. Ad manus autem illum trahentes, introduxerunt Damascum.
${}^{9}$~Et erat ibi tribus diebus non videns, et non manducavit, neque bibit.


${}^{10}$~Erat autem quidam discipulus Damasci, nomine Ananias~: et dixit ad illum in visu Dominus~: Anania. At ille ait~: Ecce ego, Domine.
${}^{11}$~Et Dominus ad eum~: Surge, et vade in vicum qui vocatur Rectus~: et qu\ae re in domo Jud\ae\ Saulum nomine Tarsensem~: ecce enim orat.
${}^{12}$~(Et vidit virum Ananiam nomine, intro\"euntem, et imponentem sibi manus ut visum recipiat.)
${}^{13}$~Respondit autem Ananias~: Domine, audivi a multis de viro hoc, quanta mala fecerit sanctis tuis in Jerusalem~:
${}^{14}$~et hic habet potestatem a principibus sacerdotum alligandi omnes qui invocant nomen tuum.
${}^{15}$~Dixit autem ad eum Dominus~: Vade, quoniam vas electionis est mihi iste, ut portet nomen meum coram gentibus, et regibus, et filiis Isra\"el.
${}^{16}$~Ego enim ostendam illi quanta oporteat eum pro nomine meo pati.


${}^{17}$~Et abiit Ananias, et introivit in domum~: et imponens ei manus, dixit~: Saule frater, Dominus misit me Jesus, qui apparuit tibi in via qua veniebas, ut videas, et implearis Spiritu Sancto.
${}^{18}$~Et confestim ceciderunt ab oculis ejus tamquam squam\ae , et visum recepit~: et surgens baptizatus est.
${}^{19}$~Et cum accepisset cibum, confortatus est.

 Fuit autem cum discipulis qui erant Damasci per dies aliquot.
${}^{20}$~Et continuo in synagogis pr\ae dicabat Jesum, quoniam hic est Filius Dei.
${}^{21}$~Stupebant autem omnes qui audiebant, et dicebant~: Nonne hic est qui expugnabat in Jerusalem eos qui invocabant nomen istud~: et huc ad hoc venit, ut vinctos illos duceret ad principes sacerdotum~?
${}^{22}$~Saulus autem multo magis convalescebat, et confundebat Jud\ae os qui habitabant Damasci, affirmans quoniam hic est Christus.
${}^{23}$~Cum autem implerentur dies multi, consilium fecerunt in unum Jud\ae i ut eum interficerent.
${}^{24}$~Not\ae\ autem fact\ae\ sunt Saulo insidi\ae\ eorum. Custodiebant autem et portas die ac nocte, ut eum interficerent.
${}^{25}$~Accipientes autem eum discipuli nocte, per murum dimiserunt eum, submittentes in sporta.


${}^{26}$~Cum autem venisset in Jerusalem, tentabat se jungere discipulis, et omnes timebant eum, non credentes quod esset discipulus.
${}^{27}$~Barnabas autem apprehensum illum duxit ad Apostolos~: et narravit illis quomodo in via vidisset Dominum, et quia locutus est ei, et quomodo in Damasco fiducialiter egerit in nomine Jesu.
${}^{28}$~Et erat cum illis intrans et exiens in Jerusalem, et fiducialiter agens in nomine Domini.
${}^{29}$~Loquebatur quoque gentibus, et disputabat cum Gr\ae cis~: illi autem qu\ae rebant occidere eum.


${}^{30}$~Quod cum cognovissent fratres, deduxerunt eum C\ae saream, et dimiserunt Tarsum.
${}^{31}$~Ecclesia quidem per totam Jud\ae am, et Galil\ae am, et Samariam habebat pacem, et \ae dificabatur ambulans in timore Domini, et consolatione Sancti Spiritus replebatur.


${}^{32}$~Factum est autem, ut Petrus dum pertransiret universos, deveniret ad sanctos qui habitabant Lydd\ae .
${}^{33}$~Invenit autem ibi hominem quemdam, nomine \AE neam, ab annis octo jacentem in grabato, qui erat paralyticus.
${}^{34}$~Et ait illi Petrus~: \AE nea, sanat te Dominus Jesus Christus~: surge, et sterne tibi. Et continuo surrexit.
${}^{35}$~Et viderunt eum omnes qui habitabant Lydd\ae\ et Saron\ae~: qui conversi sunt ad Dominum.


${}^{36}$~In Joppe autem fuit qu\ae dam discipula, nomine Tabitha, qu\ae\ interpretata dicitur Dorcas. H\ae c erat plena operibus bonis et eleemosynis quas faciebat.
${}^{37}$~Factum est autem in diebus illis ut infirmata moreretur. Quam cum lavissent, posuerunt eam in cœnaculo.
${}^{38}$~Cum autem prope esset Lydda ad Joppen, discipuli, audientes quia Petrus esset in ea, miserunt duos viros ad eum, rogantes~: Ne pigriteris venire ad nos.
${}^{39}$~Exsurgens autem Petrus, venit cum illis. Et cum advenisset, duxerunt illum in cœnaculum~: et circumsteterunt illum omnes vidu\ae\ flentes, et ostendentes ei tunicas et vestes quas faciebat illis Dorcas.
${}^{40}$~Ejectis autem omnibus foras, Petrus ponens genua oravit~: et conversus ad corpus, dixit~: Tabitha, surge. At illa aperuit oculos suos~: et viso Petro, resedit.
${}^{41}$~Dans autem illi manum, erexit eam. Et cum vocasset sanctos et viduas, assignavit eam vivam.
${}^{42}$~Notum autem factum est per universam Joppen~: et crediderunt multi in Domino.
${}^{43}$~Factum est autem ut dies multos moraretur in Joppe, apud Simonem quemdam coriarium.

\bchapter{10}
\lettrine[lines=3,image=true,loversize=0.05,lraise=-0.03]{V}{}ir autem quidam erat in C\ae sarea, nomine Cornelius, centurio cohortis qu\ae\ dicitur Italica,
${}^{2}$~religiosus, ac timens Deum cum omni domo sua, faciens eleemosynas multas plebi, et deprecans Deum semper.
${}^{3}$~Is vidit in visu manifeste, quasi hora diei nona, angelum Dei intro\"euntem ad se, et dicentem sibi~: Corneli.
${}^{4}$~At ille intuens eum, timore correptus, dixit~: Quid est, domine~? Dixit autem illi~: Orationes tu\ae\ et eleemosyn\ae\ tu\ae\ ascenderunt in memoriam in conspectu Dei.
${}^{5}$~Et nunc mitte viros in Joppen, et accersi Simonem quemdam, qui cognominatur Petrus~:
${}^{6}$~hic hospitatur apud Simonem quemdam coriarium, cujus est domus juxta mare~: hic dicet tibi quid te oporteat facere.
${}^{7}$~Et cum discessisset angelus qui loquebatur illi, vocavit duos domesticos suos, et militem metuentem Dominum ex his qui illi parebant.
${}^{8}$~Quibus cum narrasset omnia, misit illos in Joppen.


${}^{9}$~Postera autem die, iter illis facientibus, et appropinquantibus civitati, ascendit Petrus in superiora ut oraret circa horam sextam.
${}^{10}$~Et cum esuriret, voluit gustare. Parantibus autem illis, cecidit super eum mentis excessus~:
${}^{11}$~et vidit c\ae lum apertum, et descendens vas quoddam, velut linteum magnum, quatuor initiis submitti de c\ae lo in terram,
${}^{12}$~in quo erant omnia quadrupedia, et serpentia terr\ae , et volatilia c\ae li.
${}^{13}$~Et facta est vox ad eum~: Surge, Petre~: occide, et manduca.
${}^{14}$~Ait autem Petrus~: Absit Domine, quia numquam manducavi omne commune et immundum.
${}^{15}$~Et vox iterum secundo ad eum~: Quod Deus purificavit, tu commune ne dixeris.
${}^{16}$~Hoc autem factum est per ter~: et statim receptum est vas in c\ae lum.
${}^{17}$~Et dum intra se h\ae sitaret Petrus quidnam esset visio quam vidisset, ecce viri qui missi erant a Cornelio, inquirentes domum Simonis astiterunt ad januam.
${}^{18}$~Et cum vocassent, interrogabant, si Simon qui cognominatur Petrus illic haberet hospitium.
${}^{19}$~Petro autem cogitante de visione, dixit Spiritus ei~: Ecce viri tres qu\ae runt te.
${}^{20}$~Surge itaque, descende, et vade cum eis nihil dubitans~: quia ego misi illos.
${}^{21}$~Descendens autem Petrus ad viros, dixit~: Ecce ego sum, quem qu\ae ritis~: qu\ae\ causa est, propter quam venistis~?
${}^{22}$~Qui dixerunt~: Cornelius centurio, vir justus et timens Deum, et testimonium habens ab universa gente Jud\ae orum, responsum accepit ab angelo sancto accersire te in domum suam, et audire verba abs te.
${}^{23}$~Introducens ergo eos, recepit hospitio. Sequenti autem die, surgens profectus est cum illis, et quidam ex fratribus ab Joppe comitati sunt eum.


${}^{24}$~Altera autem die introivit C\ae saream. Cornelius vero exspectabat illos, convocatis cognatis suis et necessariis amicis.
${}^{25}$~Et factum est cum introisset Petrus, obvius venit ei Cornelius, et procidens ad pedes ejus adoravit.
${}^{26}$~Petrus vero elevavit eum, dicens~: Surge~: et ego ipse homo sum.
${}^{27}$~Et loquens cum illo intravit, et invenit multos qui convenerant~:
${}^{28}$~dixitque ad illos~: Vos scitis quomodo abominatum sit viro Jud\ae o conjungi aut accedere ad alienigenam~: sed mihi ostendit Deus neminem communem aut immundum dicere hominem.
${}^{29}$~Propter quod sine dubitatione veni accersitus. Interrogo ergo, quam ob causam accersistis me~?
${}^{30}$~Et Cornelius ait~: A nudiusquarta die usque ad hanc horam, orans eram hora nona in domo mea, et ecce vir stetit ante me in veste candida, et ait~:
${}^{31}$~Corneli, exaudita est oratio tua, et eleemosyn\ae\ tu\ae\ commemorat\ae\ sunt in conspectu Dei.
${}^{32}$~Mitte ergo in Joppen, et accersi Simonem qui cognominatur Petrus~: hic hospitatur in domo Simonis coriarii juxta mare.
${}^{33}$~Confestim ergo misi ad te~: et tu benefecisti veniendo. Nunc ergo omnes nos in conspectu tuo adsumus audire omnia qu\ae cumque tibi pr\ae cepta sunt a Domino.


${}^{34}$~Aperiens autem Petrus os suum, dixit~: In veritate comperi quia non est personarum acceptor Deus~;
${}^{35}$~sed in omni gente qui timet eum, et operatur justitiam, acceptus est illi.
${}^{36}$~Verbum misit Deus filiis Isra\"el, annuntians pacem per Jesum Christum (hic est omnium Dominus).
${}^{37}$~Vos scitis quod factum est verbum per universam Jud\ae am~: incipiens enim a Galil\ae a post baptismum quod pr\ae dicavit Joannes,
${}^{38}$~Jesum a Nazareth~: quomodo unxit eum Deus Spiritu Sancto, et virtute, qui pertransiit benefaciendo, et sanando omnes oppressos a diabolo, quoniam Deus erat cum illo.
${}^{39}$~Et nos testes sumus omnium qu\ae\ fecit in regione Jud\ae orum, et Jerusalem, quem occiderunt suspendentes in ligno.
${}^{40}$~Hunc Deus suscitavit tertia die, et dedit eum manifestum fieri,
${}^{41}$~non omni populo, sed testibus pr\ae ordinatis a Deo~: nobis, qui manducavimus et bibimus cum illo postquam resurrexit a mortuis.
${}^{42}$~Et pr\ae cepit nobis pr\ae dicare populo, et testificari, quia ipse est qui constitutus est a Deo judex vivorum et mortuorum.
${}^{43}$~Huic omnes prophet\ae\ testimonium perhibent remissionem peccatorum accipere per nomen ejus omnes qui credunt in eum.


${}^{44}$~Adhuc loquente Petro verba h\ae c, cecidit Spiritus Sanctus super omnes qui audiebant verbum.
${}^{45}$~Et obstupuerunt ex circumcisione fideles qui venerant cum Petro, quia et in nationes gratia Spiritus Sancti effusa est.
${}^{46}$~Audiebant enim illos loquentes linguis, et magnificantes Deum.
${}^{47}$~Tunc respondit Petrus~: Numquid aquam quis prohibere potest ut non baptizentur hi qui Spiritum Sanctum acceperunt sicut et nos~?
${}^{48}$~Et jussit eos baptizari in nomine Domini Jesu Christi. Tunc rogaverunt eum ut maneret apud eos aliquot diebus.

\bchapter{11}
\lettrine[lines=3,image=true,loversize=0.05,lraise=-0.03]{A}{}udierunt autem Apostoli et fratres qui erant in Jud\ae a, quoniam et gentes receperunt verbum Dei.
${}^{2}$~Cum autem ascendisset Petrus Jerosolymam, disceptabant adversus illum qui erant ex circumcisione,
${}^{3}$~dicentes~: Quare introisti ad viros pr\ae putium habentes, et manducasti cum illis~?
${}^{4}$~Incipiens autem Petrus exponebat illis ordinem, dicens~:
${}^{5}$~Ego eram in civitate Joppe orans, et vidi in excessu mentis visionem, descendens vas quoddam velut linteum magnum quatuor initiis summitti de c\ae lo, et venit usque ad me.
${}^{6}$~In quod intuens considerabam, et vidi quadrupedia terr\ae , et bestias, et reptilia, et volatilia c\ae li.
${}^{7}$~Audivi autem et vocem dicentem mihi~: Surge, Petre~: occide, et manduca.
${}^{8}$~Dixi autem~: Nequaquam Domine~: quia commune aut immundum numquam introivit in os meum.
${}^{9}$~Respondit autem vox secundo de c\ae lo~: Qu\ae\ Deus mundavit, tu ne commune dixeris.
${}^{10}$~Hoc autem factum est per ter~: et recepta sunt omnia rursum in c\ae lum.
${}^{11}$~Et ecce viri tres confestim astiterunt in domo in qua eram, missi a C\ae sarea ad me.
${}^{12}$~Dixit autem Spiritus mihi ut irem cum illis, nihil h\ae sitans. Venerunt autem mecum et sex fratres isti, et ingressi sumus in domum viri.
${}^{13}$~Narravit autem nobis quomodo vidisset angelum in domo sua, stantem et dicentem sibi~: Mitte in Joppen, et accersi Simonem qui cognominatur Petrus,
${}^{14}$~qui loquetur tibi verba in quibus salvus eris tu, et universa domus tua.
${}^{15}$~Cum autem cœpissem loqui, cecidit Spiritus Sanctus super eos, sicut et in nos in initio.
${}^{16}$~Recordatus sum autem verbi Domini, sicut dicebat~: Joannes quidem baptizavit aqua, vos autem baptizabimini Spiritu Sancto.
${}^{17}$~Si ergo eamdem gratiam dedit illis Deus, sicut et nobis qui credidimus in Dominum Jesum Christum~: ego quis eram, qui possem prohibere Deum~?
${}^{18}$~His auditis, tacuerunt~: et glorificaverunt Deum, dicentes~: Ergo et gentibus pœnitentiam dedit Deus ad vitam.


${}^{19}$~Et illi quidem qui dispersi fuerant a tribulatione qu\ae\ facta fuerat sub Stephano, perambulaverunt usque Phœnicen, et Cyprum, et Antiochiam, nemini loquentes verbum, nisi solis Jud\ae is.
${}^{20}$~Erant autem quidam ex eis viri Cyprii et Cyren\ae i, qui cum introissent Antiochiam, loquebantur et ad Gr\ae cos, annuntiantes Dominum Jesum.
${}^{21}$~Et erat manus Domini cum eis~: multusque numerus credentium conversus est ad Dominum.
${}^{22}$~Pervenit autem sermo ad aures ecclesi\ae\ qu\ae\ erat Jerosolymis super istis~: et miserunt Barnabam usque ad Antiochiam.
${}^{23}$~Qui cum pervenisset, et vidisset gratiam Dei, gavisus est~: et hortabatur omnes in proposito cordis permanere in Domino~:
${}^{24}$~quia erat vir bonus, et plenus Spiritu Sancto, et fide. Et apposita est multa turba Domino.


${}^{25}$~Profectus est autem Barnabas Tarsum, ut qu\ae reret Saulum~: quem cum invenisset, perduxit Antiochiam.
${}^{26}$~Et annum totum conversati sunt ibi in ecclesia~: et docuerunt turbam multam, ita ut cognominarentur primum Antiochi\ae\ discipuli, christiani.
${}^{27}$~In his autem diebus supervenerunt ab Jerosolymis prophet\ae\ Antiochiam~:
${}^{28}$~et surgens unus ex eis nomine Agabus, significabat per spiritum famem magnam futuram in universo orbe terrarum, qu\ae\ facta est sub Claudio.
${}^{29}$~Discipuli autem, prout quis habebat, proposuerunt singuli in ministerium mittere habitantibus in Jud\ae a fratribus~:
${}^{30}$~quod et fecerunt, mittentes ad seniores per manus Barnab\ae\ et Sauli.

\bchapter{12}
\lettrine[lines=3,image=true,loversize=0.05,lraise=-0.03]{E}{}odem autem tempore misit Herodes rex manus, ut affligeret quosdam de ecclesia.
${}^{2}$~Occidit autem Jacobum fratrem Joannis gladio.
${}^{3}$~Videns autem quia placeret Jud\ae is, apposuit ut apprehenderet et Petrum. Erant autem dies Azymorum.
${}^{4}$~Quem cum apprehendisset, misit in carcerem, tradens quatuor quaternionibus militum custodiendum, volens post Pascha producere eum populo.
${}^{5}$~Et Petrus quidem servabatur in carcere. Oratio autem fiebat sine intermissione ab ecclesia ad Deum pro eo.


${}^{6}$~Cum autem producturus eum esset Herodes, in ipsa nocte erat Petrus dormiens inter duos milites, vinctus catenis duabus~: et custodes ante ostium custodiebant carcerem.
${}^{7}$~Et ecce angelus Domini astitit, et lumen refulsit in habitaculo~: percussoque latere Petri, excitavit eum, dicens~: Surge velociter. Et ceciderunt caten\ae\ de manibus ejus.
${}^{8}$~Dixit autem angelus ad eum~: Pr\ae cingere, et calcea te caligas tuas. Et fecit sic. Et dixit illi~: Circumda tibi vestimentum tuum, et sequere me.
${}^{9}$~Et exiens sequebatur eum, et nesciebat quia verum est, quod fiebat per angelum~: existimabat autem se visum videre.
${}^{10}$~Transeuntes autem primam et secundam custodiam, venerunt ad portam ferream, qu\ae\ ducit ad civitatem~: qu\ae\ ultro aperta est eis. Et exeuntes processerunt vicum unum~: et continuo discessit angelus ab eo.


${}^{11}$~Et Petrus ad se reversus, dixit~: Nunc scio vere quia misit Dominus angelum suum, et eripuit me de manu Herodis, et de omni exspectatione plebis Jud\ae orum.
${}^{12}$~Consideransque venit ad domum Mari\ae\ matris Joannis, qui cognominatus est Marcus, ubi erant multi congregati, et orantes.
${}^{13}$~Pulsante autem eo ostium janu\ae , processit puella ad audiendum, nomine Rhode.
${}^{14}$~Et ut cognovit vocem Petri, pr\ae\ gaudio non aperuit januam, sed intro currens nuntiavit stare Petrum ante januam.
${}^{15}$~At illi dixerunt ad eam~: Insanis. Illa autem affirmabat sic se habere. Illi autem dicebant~: Angelus ejus est.
${}^{16}$~Petrus autem perseverabat pulsans. Cum autem aperuissent, viderunt eum, et obstupuerunt.
${}^{17}$~Annuens autem eis manu ut tacerent, narravit quomodo Dominus eduxisset eum de carcere, dixitque~: Nuntiate Jacobo et fratribus h\ae c. Et egressus abiit in alium locum.


${}^{18}$~Facta autem die, erat non parva turbatio inter milites, quidnam factum esset de Petro.
${}^{19}$~Herodes autem cum requisisset eum et non invenisset, inquisitione facta de custodibus, jussit eos duci~: descendensque a Jud\ae a in C\ae saream, ibi commoratus est.
${}^{20}$~Erat autem iratus Tyriis et Sidoniis. At illi unanimes venerunt ad eum, et persuaso Blasto, qui erat super cubiculum regis, postulabant pacem, eo quod alerentur regiones eorum ab illo.
${}^{21}$~Statuto autem die Herodes vestitus veste regia sedit pro tribunali, et concionabatur ad eos.
${}^{22}$~Populus autem acclamabat~: Dei voces, et non hominis.
${}^{23}$~Confestim autem percussit eum angelus Domini, eo quod non dedisset honorem Deo~: et consumptus a vermibus, expiravit.
${}^{24}$~Verbum autem Domini crescebat, et multiplicabatur.
${}^{25}$~Barnabas autem et Saulus reversi sunt ab Jerosolymis expleto ministerio assumpto Joanne, qui cognominatus est Marcus.

\bchapter{13}
\lettrine[lines=3,image=true,loversize=0.05,lraise=-0.03]{E}{}rant autem in ecclesia qu\ae\ erat Antiochi\ae , prophet\ae\ et doctores, in quibus Barnabas, et Simon qui vocabatur Niger, et Lucius Cyrenensis, et Manahen, qui erat Herodis Tetrarch\ae\ collactaneus, et Saulus.
${}^{2}$~Ministrantibus autem illis Domino, et jejunantibus, dixit illis Spiritus Sanctus~: Segregate mihi Saulum et Barnabam in opus ad quod assumpsi eos.
${}^{3}$~Tunc jejunantes et orantes, imponentesque eis manus, dimiserunt illos.


${}^{4}$~Et ipsi quidem missi a Spiritu Sancto abierunt Seleuciam~: et inde navigaverunt Cyprum.
${}^{5}$~Et cum venissent Salaminam, pr\ae dicabant verbum Dei in synagogis Jud\ae orum. Habebant autem et Joannem in ministerio.
${}^{6}$~Et cum perambulassent universam insulam usque Paphum, invenerunt quemdam virum magum pseudoprophetam, Jud\ae um, cui nomen erat Barjesu,
${}^{7}$~qui erat cum proconsule Sergio Paulo viro prudente. Hic, accersitis Barnaba et Saulo, desiderabat audire verbum Dei.
${}^{8}$~Resistebat autem illis Elymas magus (sic enim interpretatur nomen ejus), qu\ae rens avertere proconsulem a fide.
${}^{9}$~Saulus autem, qui et Paulus, repletus Spiritu Sancto, intuens in eum,
${}^{10}$~dixit~: O plene omni dolo et omni fallacia, fili diaboli, inimice omnis justiti\ae , non desinis subvertere vias Domini rectas.
${}^{11}$~Et nunc ecce manus Domini super te, et eris c\ae cus, non videns solem usque ad tempus. Et confestim cecidit in eum caligo et tenebr\ae~: et circuiens qu\ae rebat qui ei manum daret.
${}^{12}$~Tunc proconsul cum vidisset factum, credidit admirans super doctrina Domini.


${}^{13}$~Et cum a Papho navigassent Paulus et qui cum eo erant, venerunt Pergen Pamphyli\ae . Joannes autem discedens ab eis, reversus est Jerosolymam.
${}^{14}$~Illi vero pertranseuntes Pergen, venerunt Antiochiam Pisidi\ae~: et ingressi synagogam die sabbatorum, sederunt.
${}^{15}$~Post lectionem autem legis et prophetarum, miserunt principes synagog\ae\ ad eos, dicentes~: Viri fratres, si quis est in vobis sermo exhortationis ad plebem, dicite.


${}^{16}$~Surgens autem Paulus, et manu silentium indicens, ait~: Viri Isra\"elit\ae , et qui timetis Deum, audite~:
${}^{17}$~Deus plebis Isra\"el elegit patres nostros, et plebem exaltavit cum essent incol\ae\ in terra \AE gypti, et in brachio excelso eduxit eos ex ea,
${}^{18}$~et per quadraginta annorum tempus mores eorum sustinuit in deserto.
${}^{19}$~Et destruens gentes septem in terra Chanaan, sorte distribuit eis terram eorum,
${}^{20}$~quasi post quadringentos et quinquaginta annos~: et post h\ae c dedit judices, usque ad Samuel prophetam.
${}^{21}$~Et exinde postulaverunt regem~: et dedit illis Deus Saul filium Cis, virum de tribu Benjamin, annis quadraginta~:
${}^{22}$~et amoto illo, suscitavit illis David regem~: cui testimonium perhibens, dixit~: Inveni David filium Jesse, virum secundum cor meum, qui faciet omnes voluntates meas.
${}^{23}$~Hujus Deus ex semine secundum promissionem eduxit Isra\"el salvatorem Jesum,
${}^{24}$~pr\ae dicante Joanne ante faciem adventus ejus baptismum pœnitenti\ae\ omni populo Isra\"el.
${}^{25}$~Cum impleret autem Joannes cursum suum, dicebat~: Quem me arbitramini esse, non sum ego~: sed ecce venit post me, cujus non sum dignus calceamenta pedum solvere.
${}^{26}$~Viri fratres, filii generis Abraham, et qui in vobis timent Deum, vobis verbum salutis hujus missum est.
${}^{27}$~Qui enim habitabant Jerusalem, et principes ejus hunc ignorantes, et voces prophetarum qu\ae\ per omne sabbatum leguntur, judicantes impleverunt,
${}^{28}$~et nullam causam mortis invenientes in eo, petierunt a Pilato ut interficerent eum.
${}^{29}$~Cumque consummassent omnia qu\ae\ de eo scripta erant, deponentes eum de ligno, posuerunt eum in monumento.
${}^{30}$~Deus vero suscitavit eum a mortuis tertia die~: qui visus est per dies multos his
${}^{31}$~qui simul ascenderant cum eo de Galil\ae a in Jerusalem~: qui usque nunc sunt testes ejus ad plebem.
${}^{32}$~Et nos vobis annuntiamus eam, qu\ae\ ad patres nostros repromissio facta est~:
${}^{33}$~quoniam hanc Deus adimplevit filiis nostris resuscitans Jesum, sicut et in psalmo secundo scriptum est~: Filius meus es tu, ego hodie genui te.
${}^{34}$~Quod autem suscitavit eum a mortuis, amplius jam non reversurum in corruptionem, ita dixit~: Quia dabo vobis sancta David fidelia.
${}^{35}$~Ideoque et alias dicit~: Non dabis sanctum tuum videre corruptionem.
${}^{36}$~David enim in sua generatione cum administrasset, voluntati Dei dormivit~: et appositus est ad patres suos, et vidit corruptionem.
${}^{37}$~Quem vero Deus suscitavit a mortuis, non vidit corruptionem.
${}^{38}$~Notum igitur sit vobis, viri fratres, quia per hunc vobis remissio peccatorum annuntiatur, et ab omnibus quibus non potuistis in lege Moysi justificari,
${}^{39}$~in hoc omnis qui credit, justificatur.
${}^{40}$~Videte ergo ne superveniat vobis quod dictum est in prophetis~:
\begin{flushleft}\begin{verse}${}^{41}$~Videte contemptores, et admiramini, et disperdimini~:\\ quia opus operor ego in diebus vestris,\\ opus quod non credetis, si quis enarraverit vobis.\end{verse}\end{flushleft}


${}^{42}$~Exeuntibus autem illis rogabant ut sequenti sabbato loquerentur sibi verba h\ae c.
${}^{43}$~Cumque dimissa esset synagoga, secuti sunt multi Jud\ae orum, et colentium advenarum, Paulum et Barnabam~: qui loquentes suadebant eis ut permanerent in gratia Dei.


${}^{44}$~Sequenti vero sabbato pene universa civitas convenit audire verbum Dei.
${}^{45}$~Videntes autem turbas Jud\ae i, repleti sunt zelo, et contradicebant his qu\ae\ a Paulo dicebantur, blasphemantes.
${}^{46}$~Tunc constanter Paulus et Barnabas dixerunt~: Vobis oportebat primum loqui verbum Dei~: sed quoniam repellitis illud, et indignos vos judicatis \ae tern\ae\ vit\ae , ecce convertimur ad gentes.
${}^{47}$~Sic enim pr\ae cepit nobis Dominus~: Posui te in lucem gentium, ut sis in salutem usque ad extremum terr\ae .
${}^{48}$~Audientes autem gentes, gavis\ae\ sunt, et glorificabant verbum Domini~: et crediderunt quotquot erant pr\ae ordinati ad vitam \ae ternam.
${}^{49}$~Disseminabatur autem verbum Domini per universam regionem.
${}^{50}$~Jud\ae i autem concitaverunt mulieres religiosas et honestas, et primos civitatis, et excitaverunt persecutionem in Paulum et Barnabam~: et ejecerunt eos de finibus suis.
${}^{51}$~At illi excusso pulvere pedum in eos, venerunt Iconium.
${}^{52}$~Discipuli quoque replebantur gaudio, et Spiritu Sancto.

\bchapter{14}
\lettrine[lines=3,image=true,loversize=0.05,lraise=-0.03]{F}{}actum est autem Iconii, ut simul introirent in synagogam Jud\ae orum, et loquerentur, ita ut crederet Jud\ae orum et Gr\ae corum copiosa multitudo.
${}^{2}$~Qui vero increduli fuerunt Jud\ae i, suscitaverunt et ad iracundiam concitaverunt animas gentium adversus fratres.
${}^{3}$~Multo igitur tempore demorati sunt, fiducialiter agentes in Domino, testimonium perhibente verbo grati\ae\ su\ae , dante signa et prodigia fieri per manus eorum.
${}^{4}$~Divisa est autem multitudo civitatis~: et quidam quidem erant cum Jud\ae is, quidam vero cum Apostolis.


${}^{5}$~Cum autem factus esset impetus gentilium et Jud\ae orum cum principibus suis, ut contumeliis afficerent, et lapidarent eos,
${}^{6}$~intelligentes confugerunt ad civitates Lycaoni\ae\ Lystram et Derben, et universam in circuitu regionem, et ibi evangelizantes erant.
${}^{7}$~Et quidam vir Lystris infirmus pedibus sedebat, claudus ex utero matris su\ae , qui numquam ambulaverat.
${}^{8}$~Hic audivit Paulum loquentem. Qui intuitus eum, et videns quia fidem haberet ut salvus fieret,
${}^{9}$~dixit magna voce~: Surge super pedes tuos rectus. Et exilivit, et ambulabat.
${}^{10}$~Turb\ae\ autem cum vidissent quod fecerat Paulus, levaverunt vocem suam lycaonice, dicentes~: Dii similes facti hominibus descenderunt ad nos.
${}^{11}$~Et vocabant Barnabam Jovem, Paulum vero Mercurium~: quoniam ipse erat dux verbi.
${}^{12}$~Sacerdos quoque Jovis, qui erat ante civitatem, tauros et coronas ante januas afferens, cum populis volebat sacrificare.
${}^{13}$~Quod ubi audierunt Apostoli, Barnabas et Paulus, conscissis tunicis suis exilierunt in turbas, clamantes
${}^{14}$~et dicentes~: Viri, quid h\ae c facitis~? et nos mortales sumus, similes vobis homines, annuntiantes vobis ab his vanis converti ad Deum vivum, qui fecit c\ae lum, et terram, et mare, et omnia qu\ae\ in eis sunt~:
${}^{15}$~qui in pr\ae teritis generationibus dimisit omnes gentes ingredi vias suas.
${}^{16}$~Et quidem non sine testimonio semetipsum reliquit benefaciens de c\ae lo, dans pluvias et tempora fructifera, implens cibo et l\ae titia corda nostra.
${}^{17}$~Et h\ae c dicentes, vix sedaverunt turbas ne sibi immolarent.
${}^{18}$~Supervenerunt autem quidam ab Antiochia et Iconio Jud\ae i~: et persuasis turbis, lapidantesque Paulum, traxerunt extra civitatem, existimantes eum mortuum esse.
${}^{19}$~Circumdantibus autem eum discipulis, surgens intravit civitatem, et postera die profectus est cum Barnaba in Derben.


${}^{20}$~Cumque evangelizassent civitati illi, et docuissent multos, reversi sunt Lystram, et Iconium, et Antiochiam,
${}^{21}$~confirmantes animas discipulorum, exhortantesque ut permanerent in fide~: et quoniam per multas tribulationes oportet nos intrare in regnum Dei.
${}^{22}$~Et cum constituissent illis per singulas ecclesias presbyteros, et orassent cum jejunationibus, commendaverunt eos Domino, in quem crediderunt.
${}^{23}$~Transeuntesque Pisidiam, venerunt in Pamphyliam,
${}^{24}$~et loquentes verbum Domini in Perge, descenderunt in Attaliam~:
${}^{25}$~et inde navigaverunt Antiochiam, unde erant traditi grati\ae\ Dei in opus quod compleverunt.
${}^{26}$~Cum autem venissent, et congregassent ecclesiam, retulerunt quanta fecisset Deus cum illis, et quia aperuisset gentibus ostium fidei.
${}^{27}$~Morati sunt autem tempus non modicum cum discipulis.

\bchapter{15}
\lettrine[lines=3,image=true,loversize=0.05,lraise=-0.03]{E}{}t quidam descendentes de Jud\ae a docebant fratres~: Quia nisi circumcidamini secundum morem Moysi, non potestis salvari.
${}^{2}$~Facta ergo seditione non minima Paulo et Barnab\ae\ adversus illos, statuerunt ut ascenderent Paulus et Barnabas, et quidam alii ex aliis ad Apostolos et presbyteros in Jerusalem super hac qu\ae stione.
${}^{3}$~Illi ergo deducti ab ecclesia pertransibant Phœnicen et Samariam, narrantes conversionem gentium~: et faciebant gaudium magnum omnibus fratribus.
${}^{4}$~Cum autem venissent Jerosolymam, suscepti sunt ab ecclesia, et ab Apostolis et senioribus, annuntiantes quanta Deus fecisset cum illis.
${}^{5}$~Surrexerunt autem quidam de h\ae resi pharis\ae orum, qui crediderunt, dicentes quia oportet circumcidi eos, pr\ae cipere quoque servare legem Moysi.
${}^{6}$~Conveneruntque Apostoli et seniores videre de verbo hoc.
${}^{7}$~Cum autem magna conquisitio fieret, surgens Petrus dixit ad eos~: Viri fratres, vos scitis quoniam ab antiquis diebus Deus in nobis elegit, per os meum audire gentes verbum Evangelii et credere.
${}^{8}$~Et qui novit corda Deus, testimonium perhibuit, dans illis Spiritum Sanctum, sicut et nobis,
${}^{9}$~et nihil discrevit inter nos et illos, fide purificans corda eorum.
${}^{10}$~Nunc ergo quid tentatis Deum, imponere jugum super cervices discipulorum quod neque patres nostri, neque nos portare potuimus~?
${}^{11}$~sed per gratiam Domini Jesu Christi credimus salvari, quemadmodum et illi.
${}^{12}$~Tacuit autem omnis multitudo~: et audiebant Barnabam et Paulum narrantes quanta Deus fecisset signa et prodigia in gentibus per eos.


${}^{13}$~Et postquam tacuerunt, respondit Jacobus, dicens~: Viri fratres, audite me.
${}^{14}$~Simon narravit quemadmodum primum Deus visitavit sumere ex gentibus populum nomini suo.
${}^{15}$~Et huic concordant verba prophetarum~: sicut scriptum est~:
\begin{flushleft}\begin{verse}${}^{16}$~Post h\ae c revertar,\\ et re\ae dificabo tabernaculum David quod decidit~:\\ et diruta ejus re\ae dificabo,\\ et erigam illud~:\\
${}^{17}$~ut requirant ceteri hominum Dominum,\\ et omnes gentes super quas invocatum est nomen meum,\\ dicit Dominus faciens h\ae c.\\
${}^{18}$~Notum a s\ae culo est Domino opus suum.\end{verse}\end{flushleft}


${}^{19}$~Propter quod ego judico non inquietari eos qui ex gentibus convertuntur ad Deum,
${}^{20}$~sed scribere ad eos ut abstineant se a contaminationibus simulacrorum, et fornicatione, et suffocatis, et sanguine.
${}^{21}$~Moyses enim a temporibus antiquis habet in singulis civitatibus qui eum pr\ae dicent in synagogis, ubi per omne sabbatum legitur.


${}^{22}$~Tunc placuit Apostolis et senioribus cum omni ecclesia eligere viros ex eis, et mittere Antiochiam cum Paulo et Barnaba~: Judam, qui cognominabatur Barsabas, et Silam, viros primos in fratribus~:
${}^{23}$~scribentes per manus eorum~: Apostoli et seniores fratres, his qui sunt Antiochi\ae , et Syri\ae , et Cilici\ae , fratribus ex gentibus, salutem.
${}^{24}$~Quoniam audivimus quia quidam ex nobis exeuntes, turbaverunt vos verbis, evertentes animas vestras, quibus non mandavimus,
${}^{25}$~placuit nobis collectis in unum eligere viros, et mittere ad vos cum carissimis nostris Barnaba et Paulo,
${}^{26}$~hominibus qui tradiderunt animas suas pro nomine Domini nostri Jesu Christi.
${}^{27}$~Misimus ergo Judam et Silam, qui et ipsi vobis verbis referent eadem.
${}^{28}$~Visum est enim Spiritui Sancto et nobis nihil ultra imponere vobis oneris quam h\ae c necessaria~:
${}^{29}$~ut abstineatis vos ab immolatis simulacrorum, et sanguine, et suffocato, et fornicatione~: a quibus custodientes vos, bene agetis. Valete.
${}^{30}$~Illi ergo dimissi, descenderunt Antiochiam~: et congregata multitudine tradiderunt epistolam.
${}^{31}$~Quam cum legissent, gavisi sunt super consolatione.
${}^{32}$~Judas autem et Silas, et ipsi cum essent prophet\ae , verbo plurimo consolati sunt fratres, et confirmaverunt.
${}^{33}$~Facto autem ibi aliquanto tempore, dimissi sunt cum pace a fratribus ad eos qui miserant illos.
${}^{34}$~Visum est autem Sil\ae\ ibi remanere~: Judas autem solus abiit Jerusalem.


${}^{35}$~Paulus autem et Barnabas demorabantur Antiochi\ae , docentes et evangelizantes cum aliis pluribus verbum Domini.
${}^{36}$~Post aliquot autem dies, dixit ad Barnabam Paulus~: Revertentes visitemus fratres per universas civitates in quibus pr\ae dicavimus verbum Domini, quomodo se habeant.
${}^{37}$~Barnabas autem volebat secum assumere et Joannem, qui cognominabatur Marcus.
${}^{38}$~Paulus autem rogabat eum (ut qui discessisset ab eis de Pamphylia, et non isset cum eis in opus) non debere recipi.
${}^{39}$~Facta est autem dissensio, ita ut discederent ab invicem, et Barnabas quidem, assumpto Marco, navigaret Cyprum.


${}^{40}$~Paulus vero, electo Sila, profectus est, traditus grati\ae\ Dei a fratribus.
${}^{41}$~Perambulabat autem Syriam et Ciliciam, confirmans ecclesias~: pr\ae cipiens custodire pr\ae cepta Apostolorum et seniorum.

\bchapter{16}
\lettrine[lines=3,image=true,loversize=0.05,lraise=-0.03]{P}{}ervenit autem Derben et Lystram. Et ecce discipulus quidam erat ibi nomine Timotheus, filius mulieris Jud\ae \ae\ fidelis, patre gentili.
${}^{2}$~Huic testimonium bonum reddebant qui in Lystris erant et Iconio fratres.
${}^{3}$~Hunc voluit Paulus secum proficisci~: et assumens circumcidit eum propter Jud\ae os qui erant in illis locis. Sciebant enim omnes quod pater ejus erat gentilis.
${}^{4}$~Cum autem pertransirent civitates, tradebant eis custodiri dogmata qu\ae\ erant decreta ab Apostolis et senioribus qui erant Jerosolymis.
${}^{5}$~Et ecclesi\ae\ quidem confirmabantur fide, et abundabunt numero quotidie.
${}^{6}$~Transeuntes autem Phrygiam et Galati\ae\ regionem, vetati sunt a Spiritu Sancto loqui verbum Dei in Asia.
${}^{7}$~Cum venissent autem in Mysiam, tentabant ire in Bithyniam~: et non permisit eos Spiritus Jesu.
${}^{8}$~Cum autem pertransissent Mysiam, descenderunt Troadem~:
${}^{9}$~et visio per noctem Paulo ostensa est~: vir Macedo quidam erat stans et deprecans eum, et dicens~: Transiens in Macedoniam, adjuva nos.


${}^{10}$~Ut autem visum vidit, statim qu\ae sivimus proficisci in Macedoniam, certi facti quod vocasset nos Deus evangelizare eis.
${}^{11}$~Navigantes autem a Troade, recto cursu venimus Samothraciam, et sequenti die Neapolim~:
${}^{12}$~et inde Philippos, qu\ae\ est prima partis Macedoni\ae\ civitas, colonia. Eramus autem in hac urbe diebus aliquot, conferentes.
${}^{13}$~Die autem sabbatorum egressi sumus foras portam juxta flumen, ubi videbatur oratio esse~: et sedentes loquebamur mulieribus qu\ae\ convenerant.
${}^{14}$~Et qu\ae dam mulier nomine Lydia, purpuraria civitatis Thyatirenorum, colens Deum, audivit~: cujus Dominus aperuit cor intendere his qu\ae\ dicebantur a Paulo.
${}^{15}$~Cum autem baptizata esset, et domus ejus, deprecata est, dicens~: Si judicastis me fidelem Domino esse, introite in domum meam, et manete. Et co\"egit nos.
${}^{16}$~Factum est autem euntibus nobis ad orationem, puellam quamdam habentem spiritum pythonem obviare nobis, qu\ae\ qu\ae stum magnum pr\ae stabat dominis suis divinando.
${}^{17}$~H\ae c subsecuta Paulum et nos, clamabat dicens~: Isti homines servi Dei excelsi sunt, qui annuntiant vobis viam salutis.
${}^{18}$~Hoc autem faciebat multis diebus. Dolens autem Paulus, et conversus, spiritui dixit~: Pr\ae cipio tibi in nomine Jesu Christi exire ab ea. Et exiit eadem hora.
${}^{19}$~Videntes autem domini ejus quia exivit spes qu\ae stus eorum, apprehendentes Paulum et Silam, perduxerunt in forum ad principes~:
${}^{20}$~et offerentes eos magistratibus, dixerunt~: Hi homines conturbant civitatem nostram, cum sint Jud\ae i~:
${}^{21}$~et annuntiant morem quem non licet nobis suscipere neque facere, cum simus Romani.


${}^{22}$~Et cucurrit plebs adversus eos~: et magistratus, scissis tunicis eorum, jusserunt eos virgis c\ae di.
${}^{23}$~Et cum multas plagas eis imposuissent, miserunt eos in carcerem, pr\ae cipientes custodi ut diligenter custodiret eos.
${}^{24}$~Qui cum tale pr\ae ceptum accepisset, misit eos in interiorem carcerem, et pedes eorum strinxit ligno.
${}^{25}$~Media autem nocte Paulus et Silas orantes, laudabant Deum~: et audiebant eos qui in custodia erant.
${}^{26}$~Subito vero terr\ae motus factus est magnus, ita ut moverentur fundamenta carceris. Et statim aperta sunt omnia ostia~: et universorum vincula soluta sunt.
${}^{27}$~Expergefactus autem custos carceris, et videns januas apertas carceris, evaginato gladio volebat se interficere, \ae stimans fugisse vinctos.
${}^{28}$~Clamavit autem Paulus voce magna, dicens~: Nihil tibi mali feceris~: universi enim hic sumus.
${}^{29}$~Petitoque lumine, introgressus est~: et tremefactus procidit Paulo et Sil\ae\ ad pedes~:
${}^{30}$~et producens eos foras, ait~: Domini, quid me oportet facere, ut salvus fiam~?
${}^{31}$~At illi dixerunt~: Crede in Dominum Jesum, et salvus eris tu, et domus tua.
${}^{32}$~Et locuti sunt ei verbum Domini cum omnibus qui erant in domo ejus.
${}^{33}$~Et tollens eos in illa hora noctis, lavit plagas eorum~: et baptizatus est ipse, et omnis domus ejus continuo.
${}^{34}$~Cumque perduxisset eos in domum suam, apposuit eis mensam, et l\ae tatus est cum omni domo sua credens Deo.


${}^{35}$~Et cum dies factus esset, miserunt magistratus lictores, dicentes~: Dimitte homines illos.
${}^{36}$~Nuntiavit autem custos carceris verba h\ae c Paulo~: Quia miserunt magistratus ut dimittamini~: nunc igitur exeuntes, ite in pace.
${}^{37}$~Paulus autem dixit eis~: C\ae sos nos publice, indemnatos homines Romanos, miserunt in carcerem~: et nunc occulte nos ejiciunt~? Non ita~: sed veniant,
${}^{38}$~et ipsi nos ejiciant. Nuntiaverunt autem magistratibus lictores verba h\ae c. Timueruntque audito quod Romani essent~:
${}^{39}$~et venientes deprecati sunt eos, et educentes rogabant ut egrederentur de urbe.
${}^{40}$~Exeuntes autem de carcere, introierunt ad Lydiam~: et visis fratribus consolati sunt eos, et profecti sunt.

\bchapter{17}
\lettrine[lines=3,image=true,loversize=0.05,lraise=-0.03]{C}{}um autem perambulassent Amphipolim et Apolloniam, venerunt Thessalonicam, ubi erat synagoga Jud\ae orum.
${}^{2}$~Secundum consuetudinem autem Paulus introivit ad eos, et per sabbata tria disserebat eis de Scripturis,
${}^{3}$~adaperiens et insinuans quia Christum oportuit pati, et resurgere a mortuis~: et quia hic est Jesus Christus, quem ego annuntio vobis.
${}^{4}$~Et quidam ex eis crediderunt et adjuncti sunt Paulo et Sil\ae~: et de colentibus gentilibusque multitudo magna, et mulieres nobiles non pauc\ae .
${}^{5}$~Zelantes autem Jud\ae i, assumentesque de vulgo viros quosdam malos, et turba facta, concitaverunt civitatem~: et assistentes domui Jasonis qu\ae rebant eos producere in populum.
${}^{6}$~Et cum non invenissent eos, trahebant Jasonem et quosdam fratres ad principes civitatis, clamantes~: Quoniam hi qui urbem concitant, et huc venerunt,
${}^{7}$~quos suscepit Jason, et hi omnes contra decreta C\ae saris faciunt, regem alium dicentes esse, Jesum.
${}^{8}$~Concitaverunt autem plebem et principes civitatis audientes h\ae c.
${}^{9}$~Et accepta satisfactione a Jasone et a ceteris, dimiserunt eos.


${}^{10}$~Fratres vero confestim per noctem dimiserunt Paulum et Silam in Berœam. Qui cum venissent, in synagogam Jud\ae orum introierunt.
${}^{11}$~Hi autem erant nobiliores eorum qui sunt Thessalonic\ae , qui susceperunt verbum cum omni aviditate, quotidie scrutantes Scripturas, si h\ae c ita se haberent.
${}^{12}$~Et multi quidem crediderunt ex eis, et mulierum gentilium honestarum, et viri non pauci.
${}^{13}$~Cum autem cognovissent in Thessalonica Jud\ae i quia et Berœ\ae\ pr\ae dicatum est a Paulo verbum Dei, venerunt et illuc commoventes, et turbantes multitudinem.
${}^{14}$~Statimque tunc Paulum dimiserunt fratres, ut iret usque ad mare~: Silas autem et Timotheus remanserunt ibi.


${}^{15}$~Qui autem deducebant Paulum, perduxerunt eum usque Athenas, et accepto mandato ab eo ad Silam et Timotheum ut quam celeriter venirent ad illum, profecti sunt.
${}^{16}$~Paulus autem cum Athenis eos exspectaret, incitabatur spiritus ejus in ipso, videns idololatri\ae\ deditam civitatem.
${}^{17}$~Disputabat igitur in synagoga cum Jud\ae is et colentibus, et in foro, per omnes dies ad eos qui aderant.
${}^{18}$~Quidam autem epicurei et stoici philosophi disserebant cum eo, et quidam dicebant~: Quid vult seminiverbius hic dicere~? Alii vero~: Novorum d\ae moniorum videtur annuntiator esse~: quia Jesum et resurrectionem annuntiabat eis.


${}^{19}$~Et apprehensum eum ad Areopagum duxerunt, dicentes~: Possumus scire qu\ae\ est h\ae c nova, qu\ae\ a te dicitur, doctrina~?
${}^{20}$~nova enim qu\ae dam infers auribus nostris~: volumus ergo scire quidnam velint h\ae c esse.
${}^{21}$~(Athenienses autem omnes, et adven\ae\ hospites, ad nihil aliud vacabant nisi aut dicere aut audire aliquid novi.)
${}^{22}$~Stans autem Paulus in medio Areopagi, ait~: Viri Athenienses, per omnia quasi superstitiosiores vos video.
${}^{23}$~Pr\ae teriens enim, et videns simulacra vestra, inveni et aram in qua scriptum erat~: Ignoto Deo. Quod ergo ignorantes colitis, hoc ego annuntio vobis.
${}^{24}$~Deus, qui fecit mundum, et omnia qu\ae\ in eo sunt, hic c\ae li et terr\ae\ cum sit Dominus, non in manufactis templis habitat,
${}^{25}$~nec manibus humanis colitur indigens aliquo, cum ipse det omnibus vitam, et inspirationem, et omnia~:
${}^{26}$~fecitque ex uno omne genus hominum inhabitare super universam faciem terr\ae , definiens statuta tempora, et terminos habitationis eorum,
${}^{27}$~qu\ae rere Deum si forte attrectent eum, aut inveniant, quamvis non longe sit ab unoquoque nostrum.
${}^{28}$~In ipso enim vivimus, et movemur, et sumus~: sicut et quidam vestrorum po\"etarum dixerunt~: Ipsius enim et genus sumus.
${}^{29}$~Genus ergo cum simus Dei, non debemus \ae stimare auro, aut argento, aut lapidi, sculptur\ae\ artis, et cogitationis hominis, divinum esse simile.
${}^{30}$~Et tempora quidem hujus ignoranti\ae\ despiciens Deus, nunc annuntiat hominibus ut omnes ubique pœnitentiam agant,
${}^{31}$~eo quod statuit diem in quo judicaturus est orbem in \ae quitate, in viro in quo statuit, fidem pr\ae bens omnibus, suscitans eum a mortuis.
${}^{32}$~Cum audissent autem resurrectionem mortuorum, quidam quidem irridebant, quidam vero dixerunt~: Audiemus te de hoc iterum.
${}^{33}$~Sic Paulus exivit de medio eorum.
${}^{34}$~Quidam vero viri adh\ae rentes ei, crediderunt~: in quibus et Dionysius Areopagita, et mulier nomine Damaris, et alii cum eis.

\bchapter{18}
\lettrine[lines=3,image=true,loversize=0.05,lraise=-0.03]{P}{}ost h\ae c egressus ab Athenis, venit Corinthum~:
${}^{2}$~et inveniens quemdam Jud\ae um nomine Aquilam, Ponticum genere, qui nuper venerat ab Italia, et Priscillam uxorem ejus (eo quod pr\ae cepisset Claudius discedere omnes Jud\ae os a Roma), accessit ad eos.
${}^{3}$~Et quia ejusdem erat artis, manebat apud eos, et operabatur. (Erant autem scenofactori\ae\ artis.)
${}^{4}$~Et disputabat in synagoga per omne sabbatum, interponens nomen Domini Jesu~: suadebatque Jud\ae is et Gr\ae cis.
${}^{5}$~Cum venissent autem de Macedonia Silas et Timotheus, instabat verbo Paulus, testificans Jud\ae is esse Christum Jesum.
${}^{6}$~Contradicentibus autem eis, et blasphemantibus, excutiens vestimenta sua, dixit ad eos~: Sanguis vester super caput vestrum~: mundus ego~: ex hoc ad gentes vadam.
${}^{7}$~Et migrans inde, intravit in domum cujusdam, nomine Titi Justi, colentis Deum, cujus domus erat conjuncta synagog\ae .
${}^{8}$~Crispus autem archisynagogus credidit Domino cum omni domo sua~: et multi Corinthiorum audientes credebant, et baptizabantur.
${}^{9}$~Dixit autem Dominus nocte per visionem Paulo~: Noli timere, sed loquere, et ne taceas~:
${}^{10}$~propter quod ego sum tecum, et nemo apponetur tibi ut noceat te~: quoniam populus est mihi multus in hac civitate.
${}^{11}$~Sedit autem ibi annum et sex menses, docens apud eos verbum Dei.


${}^{12}$~Gallione autem proconsule Achai\ae , insurrexerunt uno animo Jud\ae i in Paulum, et adduxerunt eum ad tribunal,
${}^{13}$~dicentes~: Quia contra legem hic persuadet hominibus colere Deum.
${}^{14}$~Incipiente autem Paulo aperire os, dixit Gallio ad Jud\ae os~: Si quidem esset iniquum aliquid aut facinus pessimum, o viri Jud\ae i, recte vos sustinerem.
${}^{15}$~Si vero qu\ae stiones sunt de verbo, et nominibus, et lege vestra, vos ipsi videritis~: judex ego horum nolo esse.
${}^{16}$~Et minavit eos a tribunali.
${}^{17}$~Apprehendentes autem omnes Sosthenem principem synagog\ae , percutiebant eum ante tribunal~: et nihil eorum Gallioni cur\ae\ erat.


${}^{18}$~Paulus vero cum adhuc sustinuisset dies multos fratribus valefaciens, navigavit in Syriam (et cum eo Priscilla et Aquila), qui sibi totonderat in Cenchris caput~: habebat enim votum.
${}^{19}$~Devenitque Ephesum, et illos ibi reliquit. Ipse vero ingressus synagogam, disputabat cum Jud\ae is.
${}^{20}$~Rogantibus autem eis ut ampliori tempore maneret, non consensit,
${}^{21}$~sed valefaciens, et dicens~: Iterum revertar ad vos, Deo volente~: profectus est ab Epheso.
${}^{22}$~Et descendens C\ae saream, ascendit, et salutavit ecclesiam, et descendit Antiochiam.


${}^{23}$~Et facto ibi aliquanto tempore profectus est, perambulans ex ordine Galaticam regionem, et Phrygiam, confirmans omnes discipulos.
${}^{24}$~Jud\ae us autem quidam, Apollo nomine, Alexandrinus genere, vir eloquens, devenit Ephesum, potens in scripturis.
${}^{25}$~Hic erat edoctus viam Domini~: et fervens spiritu loquebatur, et docebat diligenter ea qu\ae\ sunt Jesu, sciens tantum baptisma Joannis.
${}^{26}$~Hic ergo cœpit fiducialiter agere in synagoga. Quem cum audissent Priscilla et Aquila, assumpserunt eum, et diligentius exposuerunt ei viam Domini.
${}^{27}$~Cum autem vellet ire Achaiam, exhortati fratres, scripserunt discipulis ut susciperent eum. Qui cum venisset, contulit multum his qui crediderant.
${}^{28}$~Vehementer enim Jud\ae os revincebat publice, ostendens per Scripturas esse Christum Jesum.

\bchapter{19}
\lettrine[lines=3,image=true,loversize=0.05,lraise=-0.03]{F}{}actum est autem cum Apollo esset Corinthi, ut Paulus peragratis superioribus partibus veniret Ephesum, et inveniret quosdam discipulos~:
${}^{2}$~dixitque ad eos~: Si Spiritum Sanctum accepistis credentes~? At illi dixerunt ad eum~: Sed neque si Spiritus Sanctus est, audivimus.
${}^{3}$~Ille vero ait~: In quo ergo baptizati estis~? Qui dixerunt~: In Joannis baptismate.
${}^{4}$~Dixit autem Paulus~: Joannes baptizavit baptismo pœnitenti\ae\ populum, dicens in eum qui venturus esset post ipsum ut crederent, hoc est, in Jesum.
${}^{5}$~His auditis, baptizati sunt in nomine Domini Jesu.
${}^{6}$~Et cum imposuisset illis manus Paulus, venit Spiritus Sanctus super eos, et loquebantur linguis, et prophetabant.
${}^{7}$~Erant autem omnes viri fere duodecim.


${}^{8}$~Introgressus autem synagogam, cum fiducia loquebatur per tres menses, disputans et suadens de regno Dei.
${}^{9}$~Cum autem quidam indurarentur, et non crederent, maledicentes viam Domini coram multitudine, discedens ab eis, segregavit discipulos, quotidie disputans in schola tyranni cujusdam.
${}^{10}$~Hoc autem factum est per biennium, ita ut omnes qui habitabant in Asia audirent verbum Domini, Jud\ae i atque gentiles.
${}^{11}$~Virtutesque non quaslibet faciebat Deus per manum Pauli,
${}^{12}$~ita ut etiam super languidos deferrentur a corpore ejus sudaria et semicinctia, et recedebant ab eis languores, et spiritus nequam egrediebantur.


${}^{13}$~Tentaverunt autem quidam et de circumeuntibus Jud\ae is exorcistis invocare super eos qui habebant spiritus malos nomen Domini Jesu, dicentes~: Adjuro vos per Jesum, quem Paulus pr\ae dicat.
${}^{14}$~Erant autem quidam Jud\ae i, Scev\ae\ principis sacerdotum septem filii, qui hoc faciebant.
${}^{15}$~Respondens autem spiritus nequam dixit eis~: Jesum novi, et Paulum scio~: vos autem qui estis~?
${}^{16}$~Et insiliens in eos homo, in quo erat d\ae monium pessimum, et dominatus amborum, invaluit contra eos, ita ut nudi et vulnerati effugerent de domo illa.
${}^{17}$~Hoc autem notum factum est omnibus Jud\ae is, atque gentilibus qui habitabant Ephesi~: et cecidit timor super omnes illos, et magnificabatur nomen Domini Jesu.
${}^{18}$~Multique credentium veniebant, confitentes et annuntiantes actus suos.
${}^{19}$~Multi autem ex eis, qui fuerant curiosa sectati, contulerunt libros, et combusserunt coram omnibus~: et computatis pretiis illorum, invenerunt pecuniam denariorum quinquaginta millium.
${}^{20}$~Ita fortiter crescebat verbum Dei, et confirmabatur.


${}^{21}$~His autem expletis, proposuit Paulus in Spiritu, transita Macedonia et Achaia, ire Jerosolymam, dicens~: Quoniam postquam fuero ibi, oportet me et Romam videre.
${}^{22}$~Mittens autem in Macedoniam duos ex ministrantibus sibi, Timotheum et Erastum, ipse remansit ad tempus in Asia.
${}^{23}$~Facta est autem illo tempore turbatio non minima de via Domini.
${}^{24}$~Demetrius enim quidam nomine, argentarius, faciens \ae des argenteas Dian\ae , pr\ae stabat artificibus non modicum qu\ae stum~:
${}^{25}$~quos convocans, et eos qui hujusmodi erant opifices, dixit~: Viri, scitis quia de hoc artificio est nobis acquisitio~:
${}^{26}$~et videtis et auditis quia non solum Ephesi, sed pene totius Asi\ae , Paulus hic suadens avertit multam turbam, dicens~: Quoniam non sunt dii, qui manibus fiunt.
${}^{27}$~Non solum autem h\ae c periclitabitur nobis pars in redargutionem venire, sed et magn\ae\ Dian\ae\ templum in nihilum reputabitur, sed et destrui incipiet majestas ejus, quam tota Asia et orbis colit.
${}^{28}$~His auditis, repleti sunt ira, et exclamaverunt dicentes~: Magna Diana Ephesiorum.
${}^{29}$~Et impleta est civitas confusione, et impetum fecerunt uno animo in theatrum, rapto Gajo et Aristarcho Macedonibus, comitibus Pauli.
${}^{30}$~Paulo autem volente intrare in populum, non permiserunt discipuli.
${}^{31}$~Quidam autem et de Asi\ae\ principibus, qui erant amici ejus, miserunt ad eum rogantes ne se daret in theatrum~:
${}^{32}$~alii autem aliud clamabant. Erat enim ecclesia confusa~: et plures nesciebant qua ex causa convenissent.
${}^{33}$~De turba autem detraxerunt Alexandrum, propellentibus eum Jud\ae is. Alexander autem manu silentio postulato, volebat reddere rationem populo.
${}^{34}$~Quem ut cognoverunt Jud\ae um esse, vox facta una est omnium, quasi per horas duas clamantium~: Magna Diana Ephesiorum.
${}^{35}$~Et cum sedasset scriba turbas, dixit~: Viri Ephesii, quis enim est hominum, qui nesciat Ephesiorum civitatem cultricem esse magn\ae\ Dian\ae , Jovisque prolis~?
${}^{36}$~Cum ergo his contradici non possit, oportet vos sedatos esse, et nihil temere agere.
${}^{37}$~Adduxistis enim homines istos, neque sacrilegos, neque blasphemantes deam vestram.
${}^{38}$~Quod si Demetrius et qui cum eo sunt artifices, habent adversus aliquem causam, conventus forenses aguntur, et proconsules sunt~: accusent invicem.
${}^{39}$~Si quid autem alterius rei qu\ae ritis, in legitima ecclesia poterit absolvi.
${}^{40}$~Nam et periclitamur argui seditionis hodiern\ae , cum nullus obnoxius sit de quo possimus reddere rationem concursus istius. Et cum h\ae c dixisset, dimisit ecclesiam.

\bchapter{20}
\lettrine[lines=3,image=true,loversize=0.05,lraise=-0.03]{P}{}ostquam autem cessavit tumultus, vocatis Paulus discipulis, et exhortatus eos, valedixit, et profectus est ut iret in Macedoniam.
${}^{2}$~Cum autem perambulasset partes illas, et exhortatus eos fuisset multo sermone, venit ad Gr\ae ciam~:
${}^{3}$~ubi cum fecisset menses tres, fact\ae\ sunt illi insidi\ae\ a Jud\ae is navigaturo in Syriam~: habuitque consilium ut reverteretur per Macedoniam.
${}^{4}$~Comitatus est autem eum Sopater Pyrrhi Berœensis, Thessalonicensium vero Aristarchus, et Secundus, et Gajus Derbeus, et Timotheus~: Asiani vero Tychicus et Trophimus.


${}^{5}$~Hi cum pr\ae cessissent, sustinuerunt nos Troade~:
${}^{6}$~nos vero navigavimus post dies azymorum a Philippis, et venimus ad eos Troadem in diebus quinque, ubi demorati sumus diebus septem.
${}^{7}$~Una autem sabbati cum convenissemus ad frangendum panem, Paulus disputabat cum eis profecturus in crastinum, protraxitque sermonem usque in mediam noctem.
${}^{8}$~Erant autem lampades copios\ae\ in cœnaculo, ubi eramus congregati.
${}^{9}$~Sedens autem quidam adolescens nomine Eutychus super fenestram, cum mergeretur somno gravi, disputante diu Paulo, ductus somno cecidit de tertio cœnaculo deorsum, et sublatus est mortuus.
${}^{10}$~Ad quem cum descendisset Paulus, incubuit super eum~: et complexus dixit~: Nolite turbari, anima enim ipsius in ipso est.
${}^{11}$~Ascendens autem, frangensque panem, et gustans, satisque allocutus usque in lucem, sic profectus est.
${}^{12}$~Adduxerunt autem puerum viventem, et consolati sunt non minime.


${}^{13}$~Nos autem ascendentes navem, navigavimus in Asson, inde suscepturi Paulum~: sic enim disposuerat ipse per terram iter facturus.
${}^{14}$~Cum autem convenisset nos in Asson, assumpto eo, venimus Mitylenen.
${}^{15}$~Et inde navigantes, sequenti die venimus contra Chium, et alia applicuimus Samum, et sequenti die venimus Miletum.
${}^{16}$~Proposuerat enim Paulus transnavigare Ephesum, ne qua mora illi fieret in Asia. Festinabat enim, si possibile sibi esset, ut diem Pentecostes faceret Jerosolymis.


${}^{17}$~A Mileto autem mittens Ephesum, vocavit majores natu ecclesi\ae .
${}^{18}$~Qui cum venissent ad eum, et simul essent, dixit eis~: Vos scitis a prima die qua ingressus sum in Asiam, qualiter vobiscum per omne tempus fuerim,
${}^{19}$~serviens Domino cum omni humilitate, et lacrimis, et tentationibus, qu\ae\ mihi acciderunt ex insidiis Jud\ae orum~:
${}^{20}$~quomodo nihil subtraxerim utilium, quominus annuntiarem vobis et docerem vos, publice et per domos,
${}^{21}$~testificans Jud\ae is atque gentilibus in Deum pœnitentiam, et fidem in Dominum nostrum Jesum Christum.
${}^{22}$~Et nunc ecce alligatus ego spiritu, vado in Jerusalem~: qu\ae\ in ea ventura sint mihi, ignorans~:
${}^{23}$~nisi quod Spiritus Sanctus per omnes civitates mihi protestatur, dicens quoniam vincula et tribulationes Jerosolymis me manent.
${}^{24}$~Sed nihil horum vereor~: nec facio animam meam pretiosiorem quam me, dummodo consummem cursum meum, et ministerium verbi quod accepi a Domino Jesu, testificari Evangelium grati\ae\ Dei.
${}^{25}$~Et nunc ecce ego scio quia amplius non videbitis faciem meam vos omnes, per quos transivi pr\ae dicans regnum Dei.
${}^{26}$~Quapropter contestor vos hodierna die, quia mundus sum a sanguine omnium.
${}^{27}$~Non enim subterfugi, quominus annuntiarem omne consilium Dei vobis.
${}^{28}$~Attendite vobis, et universo gregi, in quo vos Spiritus Sanctus posuit episcopos regere ecclesiam Dei, quam acquisivit sanguine suo.
${}^{29}$~Ego scio quoniam intrabunt post discessionem meam lupi rapaces in vos, non parcentes gregi.
${}^{30}$~Et ex vobisipsis exsurgent viri loquentes perversa, ut abducant discipulos post se.
${}^{31}$~Propter quod vigilate, memoria retinentes quoniam per triennium nocte et die non cessavi, cum lacrimis monens unumquemque vestrum.
${}^{32}$~Et nunc commendo vos Deo, et verbo grati\ae\ ipsius, qui potens est \ae dificare, et dare h\ae reditatem in sanctificatis omnibus.
${}^{33}$~Argentum, et aurum, aut vestem nullius concupivi, sicut
${}^{34}$~ipsi scitis~: quoniam ad ea qu\ae\ mihi opus erant, et his qui mecum sunt, ministraverunt manus ist\ae .
${}^{35}$~Omnia ostendi vobis, quoniam sic laborantes, oportet suscipere infirmos ac meminisse verbi Domini Jesu~: quoniam ipse dixit~: Beatius est magis dare, quam accipere.
${}^{36}$~Et cum h\ae c dixisset, positis genibus suis oravit cum omnibus illis.
${}^{37}$~Magnus autem fletus factus est omnium~: et procumbentes super collum Pauli, osculabantur eum,
${}^{38}$~dolentes maxime in verbo quod dixerat, quoniam amplius faciem ejus non essent visuri. Et deducebant eum ad navem.

\bchapter{21}
\lettrine[lines=3,image=true,loversize=0.05,lraise=-0.03]{C}{}um autem factum esset ut navigaremus abstracti ab eis, recto cursu venimus Coum, et sequenti die Rhodum, et inde Pataram.
${}^{2}$~Et cum invenissemus navem transfretantem in Phœnicen, ascendentes navigavimus.
${}^{3}$~Cum apparuissemus autem Cypro, relinquentes eam ad sinistram, navigavimus in Syriam, et venimus Tyrum~: ibi enim navis expositura erat onus.
${}^{4}$~Inventis autem discipulis, mansimus ibi diebus septem~: qui Paulo dicebant per Spiritum ne ascenderet Jerosolymam.
${}^{5}$~Et expletis diebus, profecti ibamus, deducentibus nos omnibus cum uxoribus et filiis usque foras civitatem~: et positis genibus in littore, oravimus.
${}^{6}$~Et cum valefecissemus invicem, ascendimus navem~: illi autem redierunt in sua.
${}^{7}$~Nos vero navigatione expleta a Tyro descendimus Ptolemaidam~: et salutatis fratribus, mansimus die una apud illos.
${}^{8}$~Alia autem die profecti, venimus C\ae saream. Et intrantes domum Philippi evangelist\ae , qui erat unus de septem, mansimus apud eum.
${}^{9}$~Huic autem erant quatuor fili\ae\ virgines prophetantes.


${}^{10}$~Et cum moraremur per dies aliquot, supervenit quidam a Jud\ae a propheta, nomine Agabus.
${}^{11}$~Is cum venisset ad nos, tulit zonam Pauli~: et alligans sibi pedes et manus, dixit~: H\ae c dicit Spiritus Sanctus~: Virum, cujus est zona h\ae c, sic alligabunt in Jerusalem Jud\ae i, et tradent in manus gentium.
${}^{12}$~Quod cum audissemus, rogabamus nos, et qui loci illius erant, ne ascenderet Jerosolymam.
${}^{13}$~Tunc respondit Paulus, et dixit~: Quid facitis flentes, et affligentes cor meum~? Ego enim non solum alligari, sed et mori in Jerusalem paratus sum propter nomen Domini Jesu.
${}^{14}$~Et cum ei suadere non possemus, quievimus, dicentes~: Domini voluntas fiat.


${}^{15}$~Post dies autem istos, pr\ae parati ascendebamus in Jerusalem.
${}^{16}$~Venerunt autem et ex discipulis a C\ae sarea nobiscum, adducentes secum apud quem hospitaremur Mnasonem quemdam Cyprium, antiquum discipulum.
${}^{17}$~Et cum venissemus Jerosolymam, libenter exceperunt nos fratres.


${}^{18}$~Sequenti autem die introibat Paulus nobiscum ad Jacobum, omnesque collecti sunt seniores.
${}^{19}$~Quos cum salutasset, narrabat per singula qu\ae\ Deus fecisset in gentibus per ministerium ipsius.
${}^{20}$~At illi cum audissent, magnificabant Deum, dixeruntque ei~: Vides, frater, quot millia sunt in Jud\ae is qui crediderunt, et omnes \ae mulatores sunt legis.
${}^{21}$~Audierunt autem de te quia discessionem doceas a Moyse eorum qui per gentes sunt Jud\ae orum, dicens non debere eos circumcidere filios suos, neque secundum consuetudinem ingredi.
${}^{22}$~Quid ergo est~? utique oportet convenire multitudinem~: audient enim te supervenisse.
${}^{23}$~Hoc ergo fac quod tibi dicimus. Sunt nobis viri quatuor, votum habentes super se.
${}^{24}$~His assumptis, sanctifica te cum illis, et impende in illis ut radant capita~: et scient omnes quia qu\ae\ de te audierunt, falsa sunt, sed ambulas et ipse custodiens legem.
${}^{25}$~De his autem qui crediderunt ex gentibus, nos scripsimus judicantes ut abstineant se ab idolis immolato, et sanguine, et suffocato, et fornicatione.
${}^{26}$~Tunc Paulus, assumptis viris, postera die purificatus cum illis intravit in templum, annuntians expletionem dierum purificationis, donec offerretur pro unoquoque eorum oblatio.


${}^{27}$~Dum autem septem dies consummarentur, hi qui de Asia erant Jud\ae i, cum vidissent eum in templo, concitaverunt omnem populum, et injecerunt ei manus, clamantes~:
${}^{28}$~Viri Isra\"elit\ae , adjuvate~: hic est homo qui adversus populum, et legem, et locum hunc, omnes ubique docens, insuper et gentiles induxit in templum, et violavit sanctum locum istum.
${}^{29}$~Viderant enim Trophimum Ephesium in civitate cum ipso, quem \ae stimaverunt quoniam in templum introduxisset Paulus.
${}^{30}$~Commotaque est civitas tota, et facta est concursio populi. Et apprehendentes Paulum, trahebant eum extra templum~: et statim claus\ae\ sunt janu\ae .
${}^{31}$~Qu\ae rentibus autem eum occidere, nuntiatum est tribuno cohortis quia tota confunditur Jerusalem.
${}^{32}$~Qui statim, assumptis militibus et centurionibus, decurrit ad illos. Qui cum vidissent tribunum et milites, cessaverunt percutere Paulum.
${}^{33}$~Tunc accedens tribunus apprehendit eum, et jussit eum alligari catenis duabus~: et interrogabat quis esset, et quid fecisset.
${}^{34}$~Alii autem aliud clamabant in turba. Et cum non posset certum cognoscere pr\ae\ tumultu, jussit duci eum in castra.
${}^{35}$~Et cum venisset ad gradus, contigit ut portaretur a militibus propter vim populi.
${}^{36}$~Sequebatur enim multitudo populi, clamans~: Tolle eum.
${}^{37}$~Et cum cœpisset induci in castra Paulus, dicit tribuno~: Si licet mihi loqui aliquid ad te~? Qui dixit~: Gr\ae ce nosti~?
${}^{38}$~nonne tu es \AE gyptius, qui ante hos dies tumultum concitasti, et eduxisti in desertum quatuor millia virorum sicariorum~?
${}^{39}$~Et dixit ad eum Paulus~: Ego homo sum quidem Jud\ae us a Tarso Cilici\ae , non ignot\ae\ civitatis municeps. Rogo autem te, permitte mihi loqui ad populum.
${}^{40}$~Et cum ille permisisset, Paulus stans in gradibus annuit manu ad plebem, et magno silentio facto, allocutus est lingua hebr\ae a, dicens~:

\bchapter{22}
\lettrine[lines=3,image=true,loversize=0.05,lraise=-0.03]{V}{}iri fratres, et patres, audite quam ad vos nunc reddo rationem.
${}^{2}$~Cum audissent autem quia hebr\ae a lingua loqueretur ad illos, magis pr\ae stiterunt silentium.
${}^{3}$~Et dicit~: Ego sum vir Jud\ae us, natus in Tarso Cilici\ae , nutritus autem in ista civitate, secus pedes Gamaliel eruditus juxta veritatem patern\ae\ legis, \ae mulator legis, sicut et vos omnes estis hodie~:
${}^{4}$~qui hanc viam persecutus sum usque ad mortem, alligans et tradens in custodias viros ac mulieres,
${}^{5}$~sicut princeps sacerdotum mihi testimonium reddit, et omnes majores natu~: a quibus et epistolas accipiens, ad fratres Damascum pergebam, ut adducerem inde vinctos in Jerusalem ut punirentur.
${}^{6}$~Factum est autem, eunte me, et appropinquante Damasco media die, subito de c\ae lo circumfulsit me lux copiosa~:
${}^{7}$~et decidens in terram, audivi vocem dicentem mihi~: Saule, Saule, quid me persequeris~?
${}^{8}$~Ego autem respondi~: Quis es, domine~? Dixitque ad me~: Ego sum Jesus Nazarenus, quem tu persequeris.
${}^{9}$~Et qui mecum erant, lumen quidem viderunt, vocem autem non audierunt ejus qui loquebatur mecum.
${}^{10}$~Et dixi~: Quid faciam, domine~? Dominus autem dixit ad me~: Surgens vade Damascum~: et ibi tibi dicetur de omnibus qu\ae\ te oporteat facere.
${}^{11}$~Et cum non viderem pr\ae\ claritate luminis illius, ad manum deductus a comitibus, veni Damascum.
${}^{12}$~Ananias autem quidam vir secundum legem, testimonium habens ab omnibus cohabitantibus Jud\ae is,
${}^{13}$~veniens ad me et astans, dixit mihi~: Saule frater, respice. Et ego eadem hora respexi in eum.
${}^{14}$~At ille dixit~: Deus patrum nostrorum pr\ae ordinavit te, ut cognosceres voluntatem ejus, et videres justum, et audires vocem ex ore ejus~:
${}^{15}$~quia eris testis illius ad omnes homines eorum qu\ae\ vidisti et audisti.
${}^{16}$~Et nunc quid moraris~? Exsurge, et baptizare, et ablue peccata tua, invocato nomine ipsius.
${}^{17}$~Factum est autem revertenti mihi in Jerusalem, et oranti in templo, fieri me in stupore mentis,
${}^{18}$~et videre illum dicentem mihi~: Festina, et exi velociter ex Jerusalem~: quoniam non recipient testimonium tuum de me.
${}^{19}$~Et ego dixi~: Domine, ipsi sciunt quia ego eram concludens in carcerem, et c\ae dens per synagogas eos qui credebant in te~:
${}^{20}$~et cum funderetur sanguis Stephani testis tui, ego astabam, et consentiebam, et custodiebam vestimenta interficientium illum.
${}^{21}$~Et dixit ad me~: Vade, quoniam ego in nationes longe mittam te.
${}^{22}$~Audiebant autem eum usque ad hoc verbum, et levaverunt vocem suam, dicentes~: Tolle de terra hujusmodi~: non enim fas est eum vivere.


${}^{23}$~Vociferantibus autem eis, et projicientibus vestimenta sua, et pulverem jactantibus in a\"erem,
${}^{24}$~jussit tribunus induci eum in castra, et flagellis c\ae di, et torqueri eum, ut sciret propter quam causam sic acclamarent ei.
${}^{25}$~Et cum astrinxissent eum loris, dicit astanti sibi centurioni Paulus~: Si hominem Romanum et indemnatum licet vobis flagellare~?
${}^{26}$~Quo audito, centurio accessit ad tribunum, et nuntiavit ei, dicens~: Quid acturus es~? hic enim homo civis Romanus est.
${}^{27}$~Accedens autem tribunus, dixit illi~: Dic mihi si tu Romanus es~? At ille dixit~: Etiam.
${}^{28}$~Et respondit tribunus~: Ego multa summa civilitatem hanc consecutus sum. Et Paulus ait~: Ego autem et natus sum.
${}^{29}$~Protinus ergo discesserunt ab illo qui eum torturi erant. Tribunus quoque timuit postquam rescivit, quia civis Romanus esset, et quia alligasset eum.
${}^{30}$~Postera autem die volens scire diligentius qua ex causa accusaretur a Jud\ae is, solvit eum, et jussit sacerdotes convenire, et omne concilium~: et producens Paulum, statuit inter illos.

\bchapter{23}
\lettrine[lines=3,image=true,loversize=0.05,lraise=-0.03]{I}{}ntendens autem in concilium Paulus, ait~: Viri fratres, ego omni conscientia bona conversatus sum ante Deum usque in hodiernum diem.
${}^{2}$~Princeps autem sacerdotum Ananias pr\ae cepit astantibus sibi percutere os ejus.
${}^{3}$~Tunc Paulus dixit ad eum~: Percutiet te Deus, paries dealbate. Et tu sedens judicas me secundum legem, et contra legem jubes me percuti~?
${}^{4}$~Et qui astabant dixerunt~: Summum sacerdotem Dei maledicis.
${}^{5}$~Dixit autem Paulus~: Nesciebam, fratres, quia princeps est sacerdotum. Scriptum est enim~: Principem populi tui non maledices.
${}^{6}$~Sciens autem Paulus quia una pars esset sadduc\ae orum, et altera pharis\ae orum, exclamavit in concilio~: Viri fratres, ego pharis\ae us sum, filius pharis\ae orum~: de spe et resurrectione mortuorum ego judicor.
${}^{7}$~Et cum h\ae c dixisset, facta est dissensio inter pharis\ae os et sadduc\ae os, et soluta est multitudo.
${}^{8}$~Sadduc\ae i enim dicunt non esse resurrectionem, neque angelum, neque spiritum~: pharis\ae i autem utraque confitentur.
${}^{9}$~Factus est autem clamor magnus. Et surgentes quidam pharis\ae orum, pugnabant, dicentes~: Nihil mali invenimus in homine isto~: quid si spiritus locutus est ei, aut angelus~?
${}^{10}$~Et cum magna dissensio facta esset, timens tribunus ne discerperetur Paulus ab ipsis, jussit milites descendere, et rapere eum de medio eorum, ac deducere eum in castra.


${}^{11}$~Sequenti autem nocte assistens ei Dominus, ait~: Constans esto~: sicut enim testificatus es de me in Jerusalem, sic te oportet et Rom\ae\ testificari.
${}^{12}$~Facta autem die collegerunt se quidam ex Jud\ae is, et devoverunt, se dicentes neque manducaturos, neque bibituros donec occiderent Paulum.
${}^{13}$~Erant autem plus quam quadraginta viri qui hanc conjurationem fecerant~:
${}^{14}$~qui accesserunt ad principes sacerdotum et seniores, et dixerunt~: Devotione devovimus nos nihil gustaturos, donec occidamus Paulum.
${}^{15}$~Nunc ergo vos notum facite tribuno cum concilio, ut producat illum ad vos, tamquam aliquid certius cognituri de eo. Nos vero priusquam appropiet, parati sumus interficere illum.
${}^{16}$~Quod cum audisset filius sororis Pauli insidias, venit, et intravit in castra, nuntiavitque Paulo.
${}^{17}$~Vocans autem Paulus ad se unum ex centurionibus, ait~: Adolescentem hunc perduc ad tribunum, habet enim aliquid indicare illi.
${}^{18}$~Et ille quidem assumens eum duxit ad tribunum, et ait~: Vinctus Paulus rogavit me hunc adolescentem perducere ad te, habentem aliquid loqui tibi.
${}^{19}$~Apprehendens autem tribunus manum illius, secessit cum eo seorsum, et interrogavit illum~: Quid est quod habes indicare mihi~?
${}^{20}$~Ille autem dixit~: Jud\ae is convenit rogare te ut crastina die producas Paulum in concilium, quasi aliquid certius inquisituri sint de illo~:
${}^{21}$~tu vero ne credideris illis~: insidiantur enim ei ex eis viri amplius quam quadraginta, qui se devoverunt non manducare, neque bibere donec interficiant eum~: et nunc parati sunt, exspectantes promissum tuum.
${}^{22}$~Tribunus igitur dimisit adolescentem, pr\ae cipiens ne cui loqueretur quoniam h\ae c nota sibi fecisset.


${}^{23}$~Et vocatis duobus centurionibus, dixit illis~: Parate milites ducentos ut eant usque C\ae saream, et equites septuaginta, et lancearios ducentos a tertia hora noctis,
${}^{24}$~et jumenta pr\ae parate ut imponentes Paulum, salvum perducerent ad Felicem pr\ae sidem.
${}^{25}$~(Timuit enim ne forte raperent eum Jud\ae i, et occiderent, et ipse postea calumniam sustineret, tamquam accepturus pecuniam.)
${}^{26}$~Scribens epistolam continentem h\ae c~: Claudius Lysias optimo pr\ae sidi Felici, salutem.
${}^{27}$~Virum hunc comprehensum a Jud\ae is, et incipientem interfici ab eis, superveniens cum exercitu eripui, cognito quia Romanus est.
${}^{28}$~Volensque scire causam quam objiciebant illi, deduxi eum in concilium eorum.
${}^{29}$~Quem inveni accusari de qu\ae stionibus legis ipsorum, nihil vero dignum morte aut vinculis habentem criminis.
${}^{30}$~Et cum mihi perlatum esset de insidiis quas paraverant illi, misi eum ad te, denuntians et accusatoribus ut dicant apud te. Vale.
${}^{31}$~Milites ergo secundum pr\ae ceptum sibi assumentes Paulum, duxerunt per noctem in Antipatridem.
${}^{32}$~Et postera die dimissis equitibus ut cum eo irent, reversi sunt ad castra.
${}^{33}$~Qui cum venissent C\ae saream, et tradidissent epistolam pr\ae sidi, statuerunt ante illum et Paulum.
${}^{34}$~Cum legisset autem, et interrogasset de qua provincia esset, et cognoscens quia de Cilicia~:
${}^{35}$~Audiam te, inquit, cum accusatores tui venerint. Jussitque in pr\ae torio Herodis custodiri eum.

\bchapter{24}
\lettrine[lines=3,image=true,loversize=0.05,lraise=-0.03]{P}{}ost quinque autem dies descendit princeps sacerdotum Ananias, cum senioribus quibusdam, et Tertullo quodam oratore, qui adierunt pr\ae sidem adversus Paulum.
${}^{2}$~Et citato Paulo cœpit accusare Tertullus, dicens~: Cum in multa pace agamus per te, et multa corrigantur per tuam providentiam,
${}^{3}$~semper et ubique suscipimus, optime Felix, cum omni gratiarum actione.
${}^{4}$~Ne diutius autem te protraham, oro, breviter audias nos pro tua clementia.
${}^{5}$~Invenimus hunc hominem pestiferum, et concitantem seditiones omnibus Jud\ae is in universo orbe, et auctorem seditionis sect\ae\ Nazarenorum~:
${}^{6}$~qui etiam templum violare conatus est, quem et apprehensum voluimus secundum legem nostram judicare.
${}^{7}$~Superveniens autem tribunus Lysias, cum vi magna eripuit eum de manibus nostris,
${}^{8}$~jubens accusatores ejus ad te venire~: a quo poteris ipse judicans, de omnibus istis cognoscere, de quibus nos accusamus eum.
${}^{9}$~Adjecerunt autem et Jud\ae i, dicentes h\ae c ita se habere.
${}^{10}$~Respondit autem Paulus (annuente sibi pr\ae side dicere)~: Ex multis annis te esse judicem genti huic sciens, bono animo pro me satisfaciam.
${}^{11}$~Potes enim cognoscere quia non plus sunt mihi dies quam duodecim, ex quo ascendi adorare in Jerusalem~:
${}^{12}$~et neque in templo invenerunt me cum aliquo disputantem, aut concursum facientem turb\ae , neque in synagogis, neque in civitate~:
${}^{13}$~neque probare possunt tibi de quibus nunc me accusant.
${}^{14}$~Confiteor autem hoc tibi, quod secundum sectam quam dicunt h\ae resim, sic deservio Patri et Deo meo, credens omnibus qu\ae\ in lege et prophetis scripta sunt~:
${}^{15}$~spem habens in Deum, quam et hi ipsi exspectant, resurrectionem futuram justorum et iniquorum.
${}^{16}$~In hoc et ipse studeo sine offendiculo conscientiam habere ad Deum et ad homines semper.
${}^{17}$~Post annos autem plures eleemosynas facturus in gentem meam, veni, et oblationes, et vota,
${}^{18}$~in quibus invenerunt me purificatum in templo~: non cum turba, neque cum tumultu.
${}^{19}$~Quidam autem ex Asia Jud\ae i, quos oportebat apud te pr\ae sto esse, et accusare si quid haberent adversum me~:
${}^{20}$~aut hi ipsi dicant si quid invenerunt in me iniquitatis cum stem in concilio,
${}^{21}$~nisi de una hac solummodo voce qua clamavi inter eos stans~: Quoniam de resurrectione mortuorum ego judicor hodie a vobis.


${}^{22}$~Distulit autem illos Felix, certissime sciens de via hac, dicens~: Cum tribunus Lysias descenderit, audiam vos.
${}^{23}$~Jussitque centurioni custodire eum, et habere requiem, nec quemquam de suis prohibere ministrare ei.
${}^{24}$~Post aliquot autem dies veniens Felix cum Drusilla uxore sua, qu\ae\ erat Jud\ae a, vocavit Paulum, et audivit ab eo fidem qu\ae\ est in Christum Jesum.
${}^{25}$~Disputante autem illo de justitia, et castitate, et de judicio futuro, tremefactus Felix, respondit~: Quod nunc attinet, vade~: tempore autem opportuno accersam te~:
${}^{26}$~simul et sperans quod pecunia ei daretur a Paulo, propter quod et frequenter accersens eum, loquebatur cum eo.
${}^{27}$~Biennio autem expleto, accepit successorem Felix Portium Festum. Volens autem gratiam pr\ae stare Jud\ae is Felix, reliquit Paulum vinctum.

\bchapter{25}
\lettrine[lines=3,image=true,loversize=0.05,lraise=-0.03]{F}{}estus ergo cum venisset in provinciam, post triduum ascendit Jerosolymam a C\ae sarea.
${}^{2}$~Adieruntque eum principes sacerdotum et primi Jud\ae orum adversus Paulum~: et rogabant eum,
${}^{3}$~postulantes gratiam adversus eum, ut juberet perduci eum in Jerusalem, insidias tendentes ut interficerent eum in via.
${}^{4}$~Festus autem respondit servari Paulum in C\ae sarea~: se autem maturius profecturum.
${}^{5}$~Qui ergo in vobis, ait, potentes sunt, descendentes simul, si quod est in viro crimen, accusent eum.
${}^{6}$~Demoratus autem inter eos dies non amplius quam octo aut decem, descendit C\ae saream, et altera die sedit pro tribunali, et jussit Paulum adduci.
${}^{7}$~Qui cum perductus esset, circumsteterunt eum, qui ab Jerosolyma descenderant Jud\ae i, multas et graves causas objicientes, quas non poterant probare~:
${}^{8}$~Paulo rationem reddente~: Quoniam neque in legem Jud\ae orum, neque in templum, neque in C\ae sarem quidquam peccavi.
${}^{9}$~Festus autem volens gratiam pr\ae stare Jud\ae is, respondens Paulo, dixit~: Vis Jerosolymam ascendere, et ibi de his judicari apud me~?
${}^{10}$~Dixit autem Paulus~: Ad tribunal C\ae saris sto~: ibi me oportet judicari~: Jud\ae is non nocui, sicut tu melius nosti.
${}^{11}$~Si enim nocui, aut dignum morte aliquid feci, non recuso mori~: si vero nihil est eorum qu\ae\ hi accusant me, nemo potest me illis donare. C\ae sarem appello.
${}^{12}$~Tunc Festus cum concilio locutus, respondit~: C\ae sarem appellasti~? ad C\ae sarem ibis.


${}^{13}$~Et cum dies aliquot transacti essent, Agrippa rex et Bernice descenderunt C\ae saream ad salutandum Festum.
${}^{14}$~Et cum dies plures ibi demorarentur, Festus regi indicavit de Paulo, dicens~: Vir quidam est derelictus a Felice vinctus,
${}^{15}$~de quo cum essem Jerosolymis, adierunt me principes sacerdotum et seniores Jud\ae orum, postulantes adversus illum damnationem.
${}^{16}$~Ad quos respondi~: Quia non est Romanis consuetudo damnare aliquem hominem priusquam is qui accusatur pr\ae sentes habeat accusatores, locumque defendendi accipiat ad abluenda crimina.
${}^{17}$~Cum ergo huc convenissent sine ulla dilatione, sequenti die sedens pro tribunali, jussi adduci virum.
${}^{18}$~De quo, cum stetissent accusatores, nullam causam deferebant, de quibus ego suspicabar malum.
${}^{19}$~Qu\ae stiones vero quasdam de sua superstitione habebant adversus eum, et de quodam Jesu defuncto, quem affirmabat Paulus vivere.
${}^{20}$~H\ae sitans autem ego de hujusmodi qu\ae stione, dicebam si vellet ire Jerosolymam, et ibi judicari de istis.
${}^{21}$~Paulo autem appellante ut servaretur ad Augusti cognitionem, jussi servari eum, donec mittam eum ad C\ae sarem.
${}^{22}$~Agrippa autem dixit ad Festum~: Volebam et ipse hominem audire. Cras, inquit, audies eum.
${}^{23}$~Altera autem die cum venisset Agrippa et Bernice cum multa ambitione, et introissent in auditorium cum tribunis et viris principalibus civitatis, jubente Festo, adductus est Paulus.
${}^{24}$~Et dicit Festus~: Agrippa rex, et omnes qui simul adestis nobiscum viri, videtis hunc de quo omnis multitudo Jud\ae orum interpellavit me Jerosolymis, petentes et acclamantes non oportere eum vivere amplius.
${}^{25}$~Ego vere comperi nihil dignum morte eum admisisse. Ipso autem hoc appellante ad Augustum, judicavi mittere.
${}^{26}$~De quo quid certum scribam domino, non habeo. Propter quod produxi eum ad vos, et maxime ad te, rex Agrippa, ut interrogatione facta habeam quid scribam.
${}^{27}$~Sine ratione enim mihi videtur mittere vinctum, et causas ejus non significare.

\bchapter{26}
\lettrine[lines=3,image=true,loversize=0.05,lraise=-0.03]{A}{}grippa vero ad Paulum ait~: Permittitur tibi loqui pro temetipso. Tunc Paulus extenta manu cœpit rationem reddere~:
${}^{2}$~De omnibus quibus accusor a Jud\ae is, rex Agrippa, \ae stimo me beatum apud te cum sim defensurus me hodie,
${}^{3}$~maxime te sciente omnia, et qu\ae\ apud Jud\ae os sunt consuetudines et qu\ae stiones~: propter quod obsecro patienter me audias.
${}^{4}$~Et quidem vitam meam a juventute, qu\ae\ ab initio fuit in gente mea in Jerosolymis, noverunt omnes Jud\ae i~:
${}^{5}$~pr\ae scientes me ab initio (si velint testimonium perhibere) quoniam secundum certissimam sectam nostr\ae\ religionis vixi pharis\ae us.
${}^{6}$~Et nunc, in spe qu\ae\ ad patres nostros repromissionis facta est a Deo, sto judicio subjectus~:
${}^{7}$~in quam duodecim tribus nostr\ae\ nocte ac die deservientes, sperant devenire. De qua spe accusor a Jud\ae is, rex.
${}^{8}$~Quid incredibile judicatur apud vos, si Deus mortuos suscitat~?
${}^{9}$~Et ego quidem existimaveram me adversus nomen Jesu Nazareni debere multa contraria agere,
${}^{10}$~quod et feci Jerosolymis, et multos sanctorum ego in carceribus inclusi, a principibus sacerdotum potestate accepta~: et cum occiderentur, detuli sententiam.
${}^{11}$~Et per omnes synagogas frequenter puniens eos, compellebam blasphemare~: et amplius insaniens in eos, persequebar usque in exteras civitates.
${}^{12}$~In quibus dum irem Damascum cum potestate et permissu principum sacerdotum,
${}^{13}$~die media in via vidi, rex, de c\ae lo supra splendorem solis circumfulsisse me lumen, et eos qui mecum simul erant.
${}^{14}$~Omnesque nos cum decidissemus in terram, audivi vocem loquentem mihi hebraica lingua~: Saule, Saule, quid me persequeris~? durum est tibi contra stimulum calcitrare.
${}^{15}$~Ego autem dixi~: Quis es, domine~? Dominus autem dixit~: Ego sum Jesus, quem tu persequeris.
${}^{16}$~Sed exsurge, et sta super pedes tuos~: ad hoc enim apparui tibi, ut constituam te ministrum, et testem eorum qu\ae\ vidisti, et eorum quibus apparebo tibi,
${}^{17}$~eripiens te de populo et gentibus, in quas nunc ego mitto te,
${}^{18}$~aperire oculos eorum, ut convertantur a tenebris ad lucem, et de potestate Satan\ae\ ad Deum, ut accipiant remissionem peccatorum, et sortem inter sanctos, per fidem qu\ae\ est in me.
${}^{19}$~Unde, rex Agrippa, non fui incredulus c\ae lesti visioni~:
${}^{20}$~sed his qui sunt Damasci primum, et Jerosolymis, et in omnem regionem Jud\ae \ae , et gentibus, annuntiabam, ut pœnitentiam agerent, et converterentur ad Deum, digna pœnitenti\ae\ opera facientes.
${}^{21}$~Hac ex causa me Jud\ae i, cum essem in templo, comprehensum tentabant interficere.
${}^{22}$~Auxilio autem adjutus Dei usque in hodiernum diem, sto, testificans minori atque majori, nihil extra dicens quam ea qu\ae\ prophet\ae\ locuti sunt futura esse, et Moyses,
${}^{23}$~si passibilis Christus, si primus ex resurrectione mortuorum, lumen annuntiaturus est populo et gentibus.
${}^{24}$~H\ae c loquente eo, et rationem reddente, Festus magna voce dixit~: Insanis, Paule~: mult\ae\ te litter\ae\ ad insaniam convertunt.
${}^{25}$~Et Paulus~: Non insanio, inquit, optime Feste, sed veritatis et sobrietatis verba loquor.
${}^{26}$~Scit enim de his rex, ad quem et constanter loquor~: latere enim eum nihil horum arbitror. Neque enim in angulo quidquam horum gestum est.
${}^{27}$~Credis, rex Agrippa, prophetis~? Scio quia credis.
${}^{28}$~Agrippa autem ad Paulum~: In modico suades me christianum fieri.
${}^{29}$~Et Paulus~: Opto apud Deum, et in modico et in magno, non tantum te, sed etiam omnes qui audiunt hodie fieri tales, qualis et ego sum, exceptis vinculis his.
${}^{30}$~Et exsurrexit rex, et pr\ae ses, et Bernice, et qui assidebant eis.
${}^{31}$~Et cum secessissent, loquebantur ad invicem, dicentes~: Quia nihil morte aut vinculis dignum quid fecit homo iste.
${}^{32}$~Agrippa autem Festo dixit~: Dimitti poterat homo hic, si non appellasset C\ae sarem.

\bchapter{27}
\lettrine[lines=3,image=true,loversize=0.05,lraise=-0.03]{U}{}t autem judicatum est navigare eum in Italiam, et tradi Paulum cum reliquis custodiis centurioni nomine Julio cohortis August\ae ,
${}^{2}$~ascendentes navem Adrumetinam, incipientes navigare circa Asi\ae\ loca, sustulimus, perseverante nobiscum Aristarcho Macedone Thessalonicensi.
${}^{3}$~Sequenti autem die devenimus Sidonem. Humane autem tractans Julius Paulum, permisit ad amicos ire, et curam sui agere.
${}^{4}$~Et inde cum sustulissemus, subnavigavimus Cyprum, propterea quod essent venti contrarii.
${}^{5}$~Et pelagus Cilici\ae\ et Pamphyli\ae\ navigantes, venimus Lystram, qu\ae\ est Lyci\ae~:
${}^{6}$~et ibi inveniens centurio navem Alexandrinam navigantem in Italiam, transposuit nos in eam.
${}^{7}$~Et cum multis diebus tarde navigaremus, et vix devenissemus contra Gnidum, prohibente nos vento, adnavigavimus Cret\ae\ juxta Salmonem~:
${}^{8}$~et vix juxta navigantes, venimus in locum quemdam qui vocatur Boniportus, cui juxta erat civitas Thalassa.
${}^{9}$~Multo autem tempore peracto, et cum jam non esset tuta navigatio eo quod et jejunium jam pr\ae teriisset, consolabatur eos Paulus,
${}^{10}$~dicens eis~: Viri, video quoniam cum injuria et multo damno non solum oneris, et navis, sed etiam animarum nostrarum incipit esse navigatio.
${}^{11}$~Centurio autem gubernatori et nauclero magis credebat, quam his qu\ae\ a Paulo dicebantur.
${}^{12}$~Et cum aptus portus non esset ad hiemandum, plurimi statuerunt consilium navigare inde, si quomodo possent, devenientes Phœnicen hiemare, portum Cret\ae\ respicientem ad Africum et ad Corum.
${}^{13}$~Aspirante autem austro, \ae stimantes propositum se tenere, cum sustulissent de Asson, legebant Cretam.


${}^{14}$~Non post multum autem misit se contra ipsam ventus typhonicus, qui vocatur Euroaquilo.
${}^{15}$~Cumque arrepta esset navis, et non posset conari in ventum, data nave flatibus, ferebamur.
${}^{16}$~In insulam autem quamdam decurrentes, qu\ae\ vocatur Cauda, potuimus vix obtinere scapham.
${}^{17}$~Qua sublata, adjutoriis utebantur, accingentes navem, timentes ne in Syrtim inciderent, summisso vase sic ferebantur.
${}^{18}$~Valida autem nobis tempestate jactatis, sequenti die jactum fecerunt~:
${}^{19}$~et tertia die suis manibus armamenta navis projecerunt.
${}^{20}$~Neque autem sole, neque sideribus apparentibus per plures dies, et tempestate non exigua imminente, jam ablata erat spes omnis salutis nostr\ae .
${}^{21}$~Et cum multa jejunatio fuisset, tunc stans Paulus in medio eorum, dixit~: Oportebat quidem, o viri, audito me, non tollere a Creta, lucrique facere injuriam hanc et jacturam.
${}^{22}$~Et nunc suadeo vobis bono animo esse~: amissio enim nullius anim\ae\ erit ex vobis, pr\ae terquam navis.
${}^{23}$~Astitit enim mihi hac nocte angelus Dei, cujus sum ego, et cui deservio,
${}^{24}$~dicens~: Ne timeas, Paule~: C\ae sari te oportet assistere~: et ecce donavit tibi Deus omnes qui navigant tecum.
${}^{25}$~Propter quod bono animo estote, viri~: credo enim Deo quia sic erit, quemadmodum dictum est mihi.
${}^{26}$~In insulam autem quamdam oportet nos devenire.
${}^{27}$~Sed posteaquam quartadecima nox supervenit, navigantibus nobis in Adria circa mediam noctem, suspicabantur naut\ae\ apparere sibi aliquam regionem.
${}^{28}$~Qui et summittentes bolidem, invenerunt passus viginti~: et pusillum inde separati, invenerunt passus quindecim.
${}^{29}$~Timentes autem ne in aspera loca incideremus, de puppi mittentes anchoras quatuor, optabant diem fieri.
${}^{30}$~Nautis vero qu\ae rentibus fugere de navi, cum misissent scapham in mare, sub obtentu quasi inciperent a prora anchoras extendere,
${}^{31}$~dixit Paulus centurioni et militibus~: Nisi hi in navi manserint, vos salvi fieri non potestis.
${}^{32}$~Tunc absciderunt milites funes scaph\ae , et passi sunt eam excidere.
${}^{33}$~Et cum lux inciperet fieri, rogabat Paulus omnes sumere cibum, dicens~: Quartadecima die hodie exspectantes jejuni permanetis, nihil accipientes.
${}^{34}$~Propter quod rogo vos accipere cibum pro salute vestra~: quia nullius vestrum capillus de capite peribit.
${}^{35}$~Et cum h\ae c dixisset, sumens panem, gratias egit Deo in conspectu omnium~: et cum fregisset, cœpit manducare.
${}^{36}$~Anim\ae quiores autem facti omnes, et ipsi sumpserunt cibum.
${}^{37}$~Eramus vero univers\ae\ anim\ae\ in navi ducent\ae\ septuaginta sex.
${}^{38}$~Et satiati cibo alleviabant navem, jactantes triticum in mare.
${}^{39}$~Cum autem dies factus esset, terram non agnoscebant~: sinum vero quemdam considerabant habentem littus, in quem cogitabant si possent ejicere navem.
${}^{40}$~Et cum anchoras sustulissent, committebant se mari, simul laxantes juncturas gubernaculorum~: et levato artemone secundum aur\ae\ flatum, tendebant ad littus.
${}^{41}$~Et cum incidissemus in locum dithalassum, impegerunt navem~: et prora quidem fixa manebat immobilis, puppis vero solvebatur a vi maris.
${}^{42}$~Militum autem consilium fuit ut custodias occiderent, ne quis cum enatasset, effugeret.
${}^{43}$~Centurio autem volens servare Paulum, prohibuit fieri~: jussitque eos qui possent natare, emittere se primos, et evadere, et ad terram exire~:
${}^{44}$~et ceteros, alios in tabulis ferebant, quosdam super ea qu\ae\ de navi erant. Et sic factum est, ut omnes anim\ae\ evaderent ad terram.

\bchapter{28}
\lettrine[lines=3,image=true,loversize=0.05,lraise=-0.03]{E}{}t cum evasissemus, tunc cognovimus quia Melita insula vocabatur. Barbari vero pr\ae stabant non modicam humanitatem nobis.
${}^{2}$~Accensa enim pyra, reficiebant nos omnes propter imbrem qui imminebat, et frigus.
${}^{3}$~Cum congregasset autem Paulus sarmentorum aliquantam multitudinem, et imposuisset super ignem, vipera a calore cum processisset, invasit manum ejus.
${}^{4}$~Ut vero viderunt barbari pendentem bestiam de manu ejus, ad invicem dicebant~: Utique homicida est homo hic, qui cum evaserit de mari, ultio non sinit eum vivere.
${}^{5}$~Et ille quidem excutiens bestiam in ignem, nihil mali passus est.
${}^{6}$~At illi existimabant eum in tumorem convertendum, et subito casurum et mori. Diu autem illis exspectantibus, et videntibus nihil mali in eo fieri, convertentes se, dicebant eum esse deum.
${}^{7}$~In locis autem illis erant pr\ae dia principis insul\ae , nomine Publii, qui nos suscipiens, triduo benigne exhibuit.
${}^{8}$~Contigit autem patrem Publii febribus et dysenteria vexatum jacere. Ad quem Paulus intravit~: et cum orasset, et imposuisset ei manus, salvavit eum.
${}^{9}$~Quo facto, omnes qui in insula habebant infirmitates, accedebant, et curabantur~:
${}^{10}$~qui etiam multis honoribus nos honoraverunt, et navigantibus imposuerunt qu\ae\ necessaria erant.


${}^{11}$~Post menses autem tres navigavimus in navi Alexandrina, qu\ae\ in insula hiemaverat, cui erat insigne Castorum.
${}^{12}$~Et cum venissemus Syracusam, mansimus ibi triduo.
${}^{13}$~Inde circumlegentes devenimus Rhegium~: et post unum diem, flante austro, secunda die venimus Puteolos~:
${}^{14}$~ubi inventis fratribus rogati sumus manere apud eos dies septem~: et sic venimus Romam.
${}^{15}$~Et inde cum audissent fratres, occurrerunt nobis usque ad Appii forum, ac tres Tabernas. Quos cum vidisset Paulus, gratias agens Deo, accepit fiduciam.
${}^{16}$~Cum autem venissemus Romam, permissum est Paulo manere sibimet cum custodiente se milite.


${}^{17}$~Post tertium autem diem convocavit primos Jud\ae orum. Cumque convenissent, dicebat eis~: Ego, viri fratres, nihil adversus plebem faciens, aut morem paternum, vinctus ab Jerosolymis traditus sum in manus Romanorum,
${}^{18}$~qui cum interrogationem de me habuissent, voluerunt me dimittere, eo quod nulla esset causa mortis in me.
${}^{19}$~Contradicentibus autem Jud\ae is, coactus sum appellare C\ae sarem, non quasi gentem meam habens aliquid accusare.
${}^{20}$~Propter hanc igitur causam rogavi vos videre, et alloqui. Propter spem enim Isra\"el catena hac circumdatus sum.
${}^{21}$~At illi dixerunt ad eum~: Nos neque litteras accepimus de te a Jud\ae a, neque adveniens aliquis fratrum nuntiavit, aut locutus est quid de te malum.
${}^{22}$~Rogamus autem a te audire qu\ae\ sentis~: nam de secta hac notum est nobis quia ubique ei contradicitur.
${}^{23}$~Cum constituissent autem illi diem, venerunt ad eum in hospitium plurimi, quibus exponebat testificans regnum Dei, suadensque eis de Jesu ex lege Moysi et prophetis a mane usque ad vesperam.
${}^{24}$~Et quidam credebant his qu\ae\ dicebantur~: quidam vero non credebant.
${}^{25}$~Cumque invicem non essent consentientes, discedebant, dicente Paulo unum verbum~: Quia bene Spiritus Sanctus locutus est per Isaiam prophetam ad patres nostros,
${}^{26}$~dicens~: Vade ad populum istum, et dic ad eos~: \begin{flushleft}\begin{verse}Aure audietis, et non intelligetis,\\ et videntes videbitis, et non perspicietis.\\
${}^{27}$~Incrassatum est enim cor populi hujus,\\ et auribus graviter audierunt,\\ et oculos suos compresserunt~:\\ ne forte videant oculis,\\ et auribus audiant,\\ et corde intelligant, et convertantur,\\ et sanem eos.\end{verse}\end{flushleft}


${}^{28}$~Notum ergo sit vobis, quoniam gentibus missum est hoc salutare Dei, et ipsi audient.
${}^{29}$~Et cum h\ae c dixisset, exierunt ab eo Jud\ae i, multam habentes inter se qu\ae stionem.


${}^{30}$~Mansit autem biennio toto in suo conducto~: et suscipiebat omnes qui ingrediebantur ad eum,
${}^{31}$~pr\ae dicans regnum Dei, et docens qu\ae\ sunt de Domino Jesu Christo cum omni fiducia, sine prohibitione.
