\clearpage
{\centering \section*{Canticum Canticorum Salomonis}}\thispagestyle{empty}
\addcontentsline{toc}{subsection}{Canticum Canticorum}
\fancyhead[C]{\textsc{Canticum Canticorum}}

\Needspace{2.5\baselineskip}\versal{1}\begin{flushleft}\begin{verse}\vspace{-11pt}\textsc{Sponsa.} Osculetur me osculo oris sui~;\\ quia meliora sunt ubera tua vino,\\
${}^{2}$~fragrantia unguentis optimis.\\ Oleum effusum nomen tuum~;\\ ideo adolescentul\ae\ dilexerunt te.\\
${}^{3}$~\textsc{Chorus Adolescentularum.} Trahe me, post te curremus\\ in odorem unguentorum tuorum.\\ Introduxit me rex in cellaria sua~;\\ exsultabimus et l\ae tabimur in te,\\ memores uberum tuorum super vinum.\\ Recti diligunt te.\\
${}^{4}$~\textsc{Sponsa.} Nigra sum, sed formosa, fili\ae\ Jerusalem,\\ sicut tabernacula Cedar, sicut pelles Salomonis.\\
${}^{5}$~Nolite me considerare quod fusca sim,\\ quia decoloravit me sol.\\ Filii matris me\ae\ pugnaverunt contra me~;\\ posuerunt me custodem in vineis~:\\ vineam meam non custodivi.\\
${}^{6}$~Indica mihi, quem diligit anima mea, ubi pascas,\\ ubi cubes in meridie,\\ ne vagari incipiam post greges sodalium tuorum.\\
${}^{7}$~\textsc{Sponsus.} Si ignoras te, o pulcherrima inter mulieres,\\ egredere, et abi post vestigia gregum,\\ et pasce h\ae dos tuos juxta tabernacula pastorum.\\
${}^{8}$~Equitatui meo in curribus Pharaonis\\ assimilavi te, amica mea.\\
${}^{9}$~Pulchr\ae\ sunt gen\ae\ tu\ae\ sicut turturis~;\\ collum tuum sicut monilia.\\
${}^{10}$~Murenulas aureas faciemus tibi,\\ vermiculatas argento.\\
${}^{11}$~\textsc{Sponsa.} Dum esset rex in accubitu suo,\\ nardus mea dedit odorem suum.\\
${}^{12}$~Fasciculus myrrh\ae\ dilectus meus mihi~;\\ inter ubera mea commorabitur.\\
${}^{13}$~Botrus cypri dilectus meus mihi\\ in vineis Engaddi.\\
${}^{14}$~\textsc{Sponsus.} Ecce tu pulchra es, amica mea~! ecce tu pulchra es~!\\ Oculi tui columbarum.\\
${}^{15}$~\textsc{Sponsa.} Ecce tu pulcher es, dilecte mi, et decorus~!\\ Lectulus noster floridus.\\
${}^{16}$~Tigna domorum nostrarum cedrina,\\ laquearia nostra cypressina.\end{verse}\end{flushleft}


\Needspace{2.5\baselineskip}\versal{2}\begin{flushleft}\begin{verse}\vspace{-19pt}Ego flos campi,\\ et lilium convallium.\\
${}^{2}$~\textsc{Sponsus.} Sicut lilium inter spinas,\\ sic amica mea inter filias.\\
${}^{3}$~\textsc{Sponsa.} Sicut malus inter ligna silvarum,\\ sic dilectus meus inter filios.\\ Sub umbra illius quem desideraveram sedi,\\ et fructus ejus dulcis gutturi meo.\\
${}^{4}$~Introduxit me in cellam vinariam~;\\ ordinavit in me caritatem.\\
${}^{5}$~Fulcite me floribus,\\ stipate me malis,\\ quia amore langueo.\\
${}^{6}$~L\ae va ejus sub capite meo,\\ et dextera illius amplexabitur me.\\
${}^{7}$~\textsc{Sponsus.} Adjuro vos, fili\ae\ Jerusalem,\\ per capreas cervosque camporum,\\ ne suscitetis, neque evigilare faciatis dilectam,\\ quoadusque ipsa velit.\\
${}^{8}$~\textsc{Sponsa.} Vox dilecti mei~; ecce iste venit,\\ saliens in montibus, transiliens colles.\\
${}^{9}$~Similis est dilectus meus capre\ae ,\\ hinnuloque cervorum.\\ En ipse stat post parietem nostrum,\\ respiciens per fenestras,\\ prospiciens per cancellos.\\
${}^{10}$~En dilectus meus loquitur mihi.\\ \textsc{Sponsus.} Surge, propera, amica mea,\\ columba mea, formosa mea, et veni~:\\
${}^{11}$~jam enim hiems transiit~;\\ imber abiit, et recessit.\\
${}^{12}$~Flores apparuerunt in terra nostra~;\\ tempus putationis advenit~:\\ vox turturis audita est in terra nostra~;\\
${}^{13}$~ficus protulit grossos suos~;\\ vine\ae\ florentes dederunt odorem suum.\\ Surge, amica mea, speciosa mea, et veni~:\\
${}^{14}$~columba mea, in foraminibus petr\ae , in caverna maceri\ae ,\\ ostende mihi faciem tuam,\\ sonet vox tua in auribus meis~:\\ vox enim tua dulcis, et facies tua decora.\\
${}^{15}$~\textsc{Sponsa.} Capite nobis vulpes parvulas\\ qu\ae\ demoliuntur vineas~:\\ nam vinea nostra floruit.\\
${}^{16}$~Dilectus meus mihi, et ego illi,\\ qui pascitur inter lilia,\\
${}^{17}$~donec aspiret dies, et inclinentur umbr\ae .\\ Revertere~; similis esto, dilecte mi, capre\ae ,\\ hinnuloque cervorum super montes Bether.\end{verse}\end{flushleft}


\Needspace{2.5\baselineskip}\versal{3}\begin{flushleft}\begin{verse}\vspace{-19pt}In lectulo meo, per noctes,\\ qu\ae sivi quem diligit anima mea~:\\ qu\ae sivi illum, et non inveni.\\
${}^{2}$~Surgam, et circuibo civitatem~:\\ per vicos et plateas\\ qu\ae ram quem diligit anima mea~:\\ qu\ae sivi illum, et non inveni.\\
${}^{3}$~Invenerunt me vigiles qui custodiunt civitatem~:\\ Num quem diligit anima mea vidistis~?\\
${}^{4}$~Paululum cum pertransissem eos,\\ inveni quem diligit anima mea~:\\ tenui eum, nec dimittam,\\ donec introducam illum in domum matris me\ae ,\\ et in cubiculum genetricis me\ae .\\
${}^{5}$~\textsc{Sponsus.} Adjuro vos, fili\ae\ Jerusalem,\\ per capreas cervosque camporum,\\ ne suscitetis, neque evigilare faciatis dilectam,\\ donec ipsa velit.\\
${}^{6}$~\textsc{Chorus.} Qu\ae\ est ista qu\ae\ ascendit per desertum\\ sicut virgula fumi ex aromatibus myrrh\ae ,\\ et thuris, et universi pulveris pigmentarii~?\\
${}^{7}$~En lectulum Salomonis sexaginta fortes ambiunt\\ ex fortissimis Isra\"el,\\
${}^{8}$~omnes tenentes gladios, et ad bella doctissimi~:\\ uniuscujusque ensis super femur suum\\ propter timores nocturnos.\\
${}^{9}$~Ferculum fecit sibi rex Salomon\\ de lignis Libani~;\\
${}^{10}$~columnas ejus fecit argenteas,\\ reclinatorium aureum, ascensum purpureum~;\\ media caritate constravit,\\ propter filias Jerusalem.\\
${}^{11}$~Egredimini et videte, fili\ae\ Sion,\\ regem Salomonem in diademate quo coronavit illum mater sua\\ in die desponsationis illius,\\ et in die l\ae titi\ae\ cordis ejus.\end{verse}\end{flushleft}


\Needspace{2.5\baselineskip}\versal{4}\begin{flushleft}\begin{verse}\vspace{-19pt}\textsc{Sponsus.} Quam pulchra es, amica mea~! quam pulchra es~!\\ Oculi tui columbarum,\\ absque eo quod intrinsecus latet.\\ Capilli tui sicut greges caprarum\\ qu\ae\ ascenderunt de monte Galaad.\\
${}^{2}$~Dentes tui sicut greges tonsarum\\ qu\ae\ ascenderunt de lavacro~;\\ omnes gemellis fœtibus,\\ et sterilis non est inter eas.\\
${}^{3}$~Sicut vitta coccinea labia tua,\\ et eloquium tuum dulce.\\ Sicut fragmen mali punici, ita gen\ae\ tu\ae ,\\ absque eo quod intrinsecus latet.\\
${}^{4}$~Sicut turris David collum tuum,\\ qu\ae\ \ae dificata est cum propugnaculis~;\\ mille clypei pendent ex ea,\\ omnis armatura fortium.\\
${}^{5}$~Duo ubera tua sicut duo hinnuli,\\ capre\ae\ gemelli, qui pascuntur in liliis.\\
${}^{6}$~Donec aspiret dies, et inclinentur umbr\ae ,\\ vadam ad montem myrrh\ae , et ad collem thuris.\\
${}^{7}$~Tota pulchra es, amica mea,\\ et macula non est in te.\\
${}^{8}$~Veni de Libano, sponsa mea~:\\ veni de Libano, veni, coronaberis~:\\ de capite Amana, de vertice Sanir et Hermon,\\ de cubilibus leonum, de montibus pardorum.\\
${}^{9}$~Vulnerasti cor meum, soror mea, sponsa~;\\ vulnerasti cor meum in uno oculorum tuorum,\\ et in uno crine colli tui.\\
${}^{10}$~Quam pulchr\ae\ sunt mamm\ae\ tu\ae , soror mea sponsa~!\\ pulchriora sunt ubera tua vino,\\ et odor unguentorum tuorum super omnia aromata.\\
${}^{11}$~Favus distillans labia tua, sponsa~;\\ mel et lac sub lingua tua~:\\ et odor vestimentorum tuorum sicut odor thuris.\\
${}^{12}$~Hortus conclusus soror mea, sponsa,\\ hortus conclusus, fons signatus.\\
${}^{13}$~Emissiones tu\ae\ paradisus malorum punicorum,\\ cum pomorum fructibus, cypri cum nardo.\\
${}^{14}$~Nardus et crocus, fistula et cinnamomum,\\ cum universis lignis Libani~;\\ myrrha et alo\"e, cum omnibus primis unguentis.\\
${}^{15}$~Fons hortorum, puteus aquarum viventium,\\ qu\ae\ fluunt impetu de Libano.\\
${}^{16}$~\textsc{Sponsa.} Surge, aquilo, et veni, auster~:\\ perfla hortum meum, et fluant aromata illius.\end{verse}\end{flushleft}


\Needspace{2.5\baselineskip}\versal{5}\begin{flushleft}\begin{verse}\vspace{-19pt}Veniat dilectus meus in hortum suum,\\ et comedat fructum pomorum suorum.\\ \textsc{Sponsus.} Veni in hortum meum, soror mea, sponsa~;\\ messui myrrham meam cum aromatibus meis~;\\ comedi favum cum melle meo~;\\ bibi vinum meum cum lacte meo~;\\ comedite, amici, et bibite,\\ et inebriamini, carissimi.\\
${}^{2}$~\textsc{Sponsa.} Ego dormio, et cor meum vigilat.\\ Vox dilecti mei pulsantis~:\\ \textsc{Sponsus.} Aperi mihi, soror mea, amica mea,\\ columba mea, immaculata mea,\\ quia caput meum plenum est rore,\\ et cincinni mei guttis noctium.\\
${}^{3}$~\textsc{Sponsa.} Expoliavi me tunica mea~: quomodo induar illa~?\\ lavi pedes meos~: quomodo inquinabo illos~?\\
${}^{4}$~Dilectus meus misit manum suam per foramen,\\ et venter meus intremuit ad tactum ejus.\\
${}^{5}$~Surrexi ut aperirem dilecto meo~;\\ manus me\ae\ stillaverunt myrrham,\\ et digiti mei pleni myrrha probatissima.\\
${}^{6}$~Pessulum ostii mei aperui dilecto meo,\\ at ille declinaverat, atque transierat.\\ Anima mea liquefacta est, ut locutus est~;\\ qu\ae sivi, et non inveni illum~;\\ vocavi, et non respondit mihi.\\
${}^{7}$~Invenerunt me custodes qui circumeunt civitatem~;\\ percusserunt me, et vulneraverunt me.\\ Tulerunt pallium meum mihi custodes murorum.\\
${}^{8}$~Adjuro vos, fili\ae\ Jerusalem,\\ si inveneritis dilectum meum,\\ ut nuntietis ei quia amore langueo.\\
${}^{9}$~\textsc{Chorus.} Qualis est dilectus tuus ex dilecto, o pulcherrima mulierum~?\\ qualis est dilectus tuus ex dilecto, quia sic adjurasti nos~?\\
${}^{10}$~\textsc{Sponsa.} Dilectus meus candidus et rubicundus~;\\ electus ex millibus.\\
${}^{11}$~Caput ejus aurum optimum~;\\ com\ae\ ejus sicut elat\ae\ palmarum, nigr\ae\ quasi corvus.\\
${}^{12}$~Oculi ejus sicut columb\ae\ super rivulos aquarum,\\ qu\ae\ lacte sunt lot\ae , et resident juxta fluenta plenissima.\\
${}^{13}$~Gen\ae\ illius sicut areol\ae\ aromatum,\\ consit\ae\ a pigmentariis.\\ Labia ejus lilia,\\ distillantia myrrham primam.\\
${}^{14}$~Manus illius tornatiles, aure\ae ,\\ plen\ae\ hyacinthis.\\ Venter ejus eburneus,\\ distinctus sapphiris.\\
${}^{15}$~Crura illius column\ae\ marmore\ae \\ qu\ae\ fundat\ae\ sunt super bases aureas.\\ Species ejus ut Libani,\\ electus ut cedri.\\
${}^{16}$~Guttur illius suavissimum,\\ et totus desiderabilis.\\ Talis est dilectus meus,\\ et ipse est amicus meus, fili\ae\ Jerusalem.\\
${}^{17}$~\textsc{Chorus.} Quo abiit dilectus tuus, o pulcherrima mulierum~?\\ quo declinavit dilectus tuus~?\\ et qu\ae remus eum tecum.\end{verse}\end{flushleft}


\Needspace{2.5\baselineskip}\versal{6}\begin{flushleft}\begin{verse}\vspace{-19pt}\textsc{Sponsa.} Dilectus meus descendit in hortum suum ad areolam aromatum,\\ ut pascatur in hortis, et lilia colligat.\\
${}^{2}$~Ego dilecto meo, et dilectus meus mihi,\\ qui pascitur inter lilia.\\
${}^{3}$~\textsc{Sponsus.} Pulchra es, amica mea~;\\ suavis, et decora sicut Jerusalem~;\\ terribilis ut castrorum acies ordinata.\\
${}^{4}$~Averte oculos tuos a me,\\ quia ipsi me avolare fecerunt.\\ Capilli tui sicut grex caprarum\\ qu\ae\ apparuerunt de Galaad.\\
${}^{5}$~Dentes tui sicut grex ovium\\ qu\ae\ ascenderunt de lavacro~:\\ omnes gemellis fœtibus,\\ et sterilis non est in eis.\\
${}^{6}$~Sicut cortex mali punici, sic gen\ae\ tu\ae ,\\ absque occultis tuis.\\
${}^{7}$~Sexaginta sunt regin\ae , et octoginta concubin\ae ,\\ et adolescentularum non est numerus.\\
${}^{8}$~Una est columba mea, perfecta mea,\\ una est matris su\ae , electa genetrici su\ae .\\ Viderunt eam fili\ae , et beatissimam pr\ae dicaverunt~;\\ regin\ae\ et concubin\ae , et laudaverunt eam.\\
${}^{9}$~Qu\ae\ est ista qu\ae\ progreditur quasi aurora consurgens,\\ pulchra ut luna, electa ut sol,\\ terribilis ut castrorum acies ordinata~?\\
${}^{10}$~\textsc{Sponsa.} Descendi in hortum nucum,\\ ut viderem poma convallium,\\ et inspicerem si floruisset vinea,\\ et germinassent mala punica.\\
${}^{11}$~Nescivi~: anima mea conturbavit me,\\ propter quadrigas Aminadab.\\
${}^{12}$~\textsc{Chorus.} Revertere, revertere, Sulamitis~!\\ revertere, revertere ut intueamur te.\end{verse}\end{flushleft}


\Needspace{2.5\baselineskip}\versal{7}\begin{flushleft}\begin{verse}\vspace{-19pt}\textsc{Sponsa.} Quid videbis in Sulamite, nisi choros castrorum~?\\ \textsc{Chorus.} Quam pulchri sunt gressus tui in calceamentis, filia principis~!\\ Junctur\ae\ femorum tuorum sicut monilia\\ qu\ae\ fabricata sunt manu artificis.\\
${}^{2}$~Umbilicus tuus crater tornatilis,\\ numquam indigens poculis.\\ Venter tuus sicut acervus tritici vallatus liliis.\\
${}^{3}$~Duo ubera tua sicut duo hinnuli,\\ gemelli capre\ae .\\
${}^{4}$~Collum tuum sicut turris eburnea~;\\ oculi tui sicut piscin\ae\ in Hesebon\\ qu\ae\ sunt in porta fili\ae\ multitudinis.\\ Nasus tuus sicut turris Libani,\\ qu\ae\ respicit contra Damascum.\\
${}^{5}$~Caput tuum ut Carmelus~;\\ et com\ae\ capitis tui sicut purpura regis\\ vincta canalibus.\\
${}^{6}$~\textsc{Sponsus.} Quam pulchra es, et quam decora,\\ carissima, in deliciis~!\\
${}^{7}$~Statura tua assimilata est palm\ae ,\\ et ubera tua botris.\\
${}^{8}$~Dixi~: Ascendam in palmam,\\ et apprehendam fructus ejus~;\\ et erunt ubera tua sicut botri vine\ae ,\\ et odor oris tui sicut malorum.\\
${}^{9}$~Guttur tuum sicut vinum optimum,\\ dignum dilecto meo ad potandum,\\ labiisque et dentibus illius ad ruminandum.\\
${}^{10}$~\textsc{Sponsa.} Ego dilecto meo,\\ et ad me conversio ejus.\\
${}^{11}$~Veni, dilecte mi, egrediamur in agrum,\\ commoremur in villis.\\
${}^{12}$~Mane surgamus ad vineas~:\\ videamus si floruit vinea,\\ si flores fructus parturiunt,\\ si floruerunt mala punica~;\\ ibi dabo tibi ubera mea.\\
${}^{13}$~Mandragor\ae\ dederunt odorem\\ in portis nostris omnia poma~:\\ nova et vetera, dilecte mi, servavi tibi.\end{verse}\end{flushleft}


\Needspace{2.5\baselineskip}\versal{8}\begin{flushleft}\begin{verse}\vspace{-19pt}Quis mihi det te fratrem meum,\\ sugentem ubera matris me\ae ,\\ ut inveniam te foris, et deosculer te,\\ et jam me nemo despiciat~?\\
${}^{2}$~Apprehendam te, et ducam in domum matris me\ae~:\\ ibi me docebis,\\ et dabo tibi poculum ex vino condito,\\ et mustum malorum granatorum meorum.\\
${}^{3}$~L\ae va ejus sub capite meo,\\ et dextera illius amplexabitur me.\\
${}^{4}$~\textsc{Sponsus.} Adjuro vos, fili\ae\ Jerusalem,\\ ne suscitetis, neque evigilare faciatis dilectam,\\ donec ipsa velit.\\
${}^{5}$~\textsc{Chorus.} Qu\ae\ est ista qu\ae\ ascendit de deserto, deliciis affluens,\\ innixa super dilectum suum~?\\ \textsc{Sponsus.} Sub arbore malo suscitavi te~;\\ ibi corrupta est mater tua,\\ ibi violata est genitrix tua.\\
${}^{6}$~\textsc{Sponsa.} Pone me ut signaculum super cor tuum,\\ ut signaculum super brachium tuum,\\ quia fortis est ut mors dilectio,\\ dura sicut infernus \ae mulatio~:\\ lampades ejus lampades ignis atque flammarum.\\
${}^{7}$~Aqu\ae\ mult\ae\ non potuerunt extinguere caritatem,\\ nec flumina obruent illam.\\ Si dederit homo omnem substantiam domus su\ae\ pro dilectione,\\ quasi nihil despiciet eam.\\
${}^{8}$~\textsc{Chorus Fratrum.} Soror nostra parva,\\ et ubera non habet~;\\ quid faciemus sorori nostr\ae \\ in die quando alloquenda est~?\\
${}^{9}$~Si murus est,\\ \ae dificemus super eum propugnacula argentea~;\\ si ostium est, compingamus illud tabulis cedrinis.\\
${}^{10}$~\textsc{Sponsa.} Ego murus, et ubera mea sicut turris,\\ ex quo facta sum coram eo, quasi pacem reperiens.\\
${}^{11}$~\textsc{Chorus Fratrum.} Vinea fuit pacifico in ea qu\ae\ habet populos~:\\ tradidit eam custodibus~;\\ vir affert pro fructu ejus mille argenteos.\\
${}^{12}$~\textsc{Sponsa.} Vinea mea coram me est.\\ Mille tui pacifici,\\ et ducenti his qui custodiunt fructus ejus.\\
${}^{13}$~\textsc{Sponsus.} Qu\ae\ habitas in hortis, amici auscultant~;\\ fac me audire vocem tuam.\\
${}^{14}$~\textsc{Sponsa.} Fuge, dilecte mi, et assimilare capre\ae ,\\ hinnuloque cervorum super montes aromatum.\end{verse}\end{flushleft}


