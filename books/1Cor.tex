\bbook{Epistola B. Pauli Apostoli ad Corinthios Prima}
{ad Corinthios I}{images/genese_heading}


\bchapter
\mylettrine{P}aulus vocatus Apostolus Jesu Christi per voluntatem Dei, et Sosthenes frater,
${}^{2}$~ecclesi\ae\ Dei, qu\ae\ est Corinthi, sanctificatis in Christo Jesu, vocatis sanctis, cum omnibus qui invocant nomen Domini nostri Jesu Christi, in omni loco ipsorum et nostro.
${}^{3}$~Gratia vobis, et pax a Deo Patre nostro, et Domino Jesu Christo.


${}^{4}$~Gratias ago Deo meo semper pro vobis in gratia Dei, qu\ae\ data est vobis in Christo Jesu~:
${}^{5}$~quod in omnibus divites facti estis in illo, in omni verbo, et in omni scientia.
${}^{6}$~Sicut testimonium Christi confirmatum est in vobis~:
${}^{7}$~ita ut nihil vobis desit in ulla gratia, exspectantibus revelationem Domini nostri Jesu Christi,
${}^{8}$~qui et confirmabit vos usque in finem sine crimine, in die adventus Domini nostri Jesu Christi.
${}^{9}$~Fidelis Deus~: per quem vocati estis in societatem filii ejus Jesu Christi Domini nostri.


${}^{10}$~Obsecro autem vos fratres per nomen Domini nostri Jesu Christi~: ut idipsum dicatis omnes, et non sint in vobis schismata~: sitis autem perfecti in eodem sensu, et in eadem sententia.
${}^{11}$~Significatum est enim mihi de vobis fratres mei ab iis, qui sunt Chlo\"es, quia contentiones sunt inter vos.
${}^{12}$~Hoc autem dico, quod unusquisque vestrum dicit~: Ego quidem sum Pauli~: ego autem Apollo~: ego vero Ceph\ae~: ego autem Christi.
${}^{13}$~Divisus est Christus~? numquid Paulus crucifixus est pro vobis~? aut in nomine Pauli baptizati estis~?
${}^{14}$~Gratias ago Deo, quod neminem vestrum baptizavi, nisi Crispum et Caium~:
${}^{15}$~ne quis dicat quod in nomine meo baptizati estis.
${}^{16}$~Baptizavi autem et Stephan\ae\ domum~: ceterum nescio si quem alium baptizaverim.
${}^{17}$~Non enim misit me Christus baptizare, sed evangelizare~: non in sapientia verbi, ut non evacuetur crux Christi.


${}^{18}$~Verbum enim crucis pereuntibus quidem stultitia est~: iis autem qui salvi fiunt, id est nobis, Dei virtus est.
${}^{19}$~Scriptum est enim~: Perdam sapientiam sapientium, et prudentiam prudentium reprobabo.
${}^{20}$~Ubi sapiens~? ubi scriba~? ubi conquisitor hujus s\ae culi~? Nonne stultam fecit Deus sapientiam hujus mundi~?
${}^{21}$~Nam quia in Dei sapientia non cognovit mundus per sapientiam Deum~: placuit Deo per stultitiam pr\ae dicationis salvos facere credentes.
${}^{22}$~Quoniam et Jud\ae i signa petunt, et Gr\ae ci sapientiam qu\ae runt~:
${}^{23}$~nos autem pr\ae dicamus Christum crucifixum~: Jud\ae is quidem scandalum, gentibus autem stultitiam,
${}^{24}$~ipsis autem vocatis Jud\ae is, atque Gr\ae cis Christum Dei virtutem, et Dei sapientiam~:
${}^{25}$~quia quod stultum est Dei, sapientius est hominibus~: et quod infirmum est Dei, fortius est hominibus.
${}^{26}$~Videte enim vocationem vestram, fratres, quia non multi sapientes secundum carnem, non multi potentes, non multi nobiles~:
${}^{27}$~sed qu\ae\ stulta sunt mundi elegit Deus, ut confundat sapientes~: et infirma mundi elegit Deus, ut confundat fortia~:
${}^{28}$~et ignobilia mundi, et contemptibilia elegit Deus, et ea qu\ae\ non sunt, ut ea qu\ae\ sunt destrueret~:
${}^{29}$~ut non glorietur omnis caro in conspectu ejus.
${}^{30}$~Ex ipso autem vos estis in Christo Jesu, qui factus est nobis sapientia a Deo, et justitia, et sanctificatio, et redemptio~:
${}^{31}$~ut quemadmodum scriptum est~: Qui gloriatur, in Domino glorietur.

\bchapter
\mylettrine{E}t ego, cum venissem ad vos, fratres, veni non in sublimitate sermonis, aut sapienti\ae , annuntians vobis testimonium Christi.
${}^{2}$~Non enim judicavi me scire aliquid inter vos, nisi Jesum Christum, et hunc crucifixum.
${}^{3}$~Et ego in infirmitate, et timore, et tremore multo fui apud vos~:
${}^{4}$~et sermo meus, et pr\ae dicatio mea non in persuasibilibus human\ae\ sapienti\ae\ verbis, sed in ostensione spiritus et virtutis~:
${}^{5}$~ut fides vestra non sit in sapientia hominum, sed in virtute Dei.


${}^{6}$~Sapientiam autem loquimur inter perfectos~: sapientiam vero non hujus s\ae culi, neque principum hujus s\ae culi, qui destruuntur~:
${}^{7}$~sed loquimur Dei sapientiam in mysterio, qu\ae\ abscondita est, quam pr\ae destinavit Deus ante s\ae cula in gloriam nostram,
${}^{8}$~quam nemo principum hujus s\ae culi cognovit~: si enim cognovissent, numquam Dominum glori\ae\ crucifixissent.
${}^{9}$~Sed sicut scriptum est~: Quod oculus non vidit, nec auris audivit, nec in cor hominis ascendit, qu\ae\ pr\ae paravit Deus iis qui diligunt illum~:
${}^{10}$~nobis autem revelavit Deus per Spiritum suum~: Spiritus enim omnia scrutatur, etiam profunda Dei.
${}^{11}$~Quis enim hominum scit qu\ae\ sunt hominis, nisi spiritus hominis, qui in ipso est~? ita et qu\ae\ Dei sunt, nemo cognovit, nisi Spiritus Dei.
${}^{12}$~Nos autem non spiritum hujus mundi accepimus, sed Spiritum qui ex Deo est, ut sciamus qu\ae\ a Deo donata sunt nobis~:
${}^{13}$~qu\ae\ et loquimur non in doctis human\ae\ sapienti\ae\ verbis, sed in doctrina Spiritus, spiritualibus spiritualia comparantes.
${}^{14}$~Animalis autem homo non percipit ea qu\ae\ sunt Spiritus Dei~: stultitia enim est illi, et non potest intelligere~: quia spiritualiter examinatur.
${}^{15}$~Spiritualis autem judicat omnia~: et ipse a nemine judicatur.
${}^{16}$~Quis enim cognovit sensum Domini, qui instruat eum~? nos autem sensum Christi habemus.

\bchapter
\mylettrine{E}t ego, fratres, non potui vobis loqui quasi spiritualibus, sed quasi carnalibus. Tamquam parvulis in Christo,
${}^{2}$~lac vobis potum dedi, non escam~: nondum enim poteratis~: sed nec nunc quidem potestis~: adhuc enim carnales estis.
${}^{3}$~Cum enim sit inter vos zelus, et contentio~: nonne carnales estis, et secundum hominem ambulatis~?
${}^{4}$~Cum enim quis dicat~: Ego quidem sum Pauli~; alius autem~: Ego Apollo~: nonne homines estis~?

 Quid igitur est Apollo~? quid vero Paulus~?
${}^{5}$~ministri ejus, cui credidistis, et unicuique sicut Dominus dedit.
${}^{6}$~Ego plantavi, Apollo rigavit~: sed Deus incrementum dedit.
${}^{7}$~Itaque neque qui plantat est aliquid, neque qui rigat~: sed qui incrementum dat, Deus.
${}^{8}$~Qui autem plantat, et qui rigat, unum sunt. Unusquisque autem propriam mercedem accipiet, secundum suum laborem.
${}^{9}$~Dei enim sumus adjutores~: Dei agricultura estis, Dei \ae dificatio estis.
${}^{10}$~Secundum gratiam Dei, qu\ae\ data est mihi, ut sapiens architectus fundamentum posui~: alius autem super\ae dificat. Unusquisque autem videat quomodo super\ae dificet.
${}^{11}$~Fundamentum enim aliud nemo potest ponere pr\ae ter id quod positum est, quod est Christus Jesus.
${}^{12}$~Si quis autem super\ae dificat super fundamentum hoc, aurum, argentum, lapides pretiosos, ligna, fœnum, stipulam,
${}^{13}$~uniuscujusque opus manifestum erit~: dies enim Domini declarabit, quia in igne revelabitur~: et uniuscujusque opus quale sit, ignis probabit.
${}^{14}$~Si cujus opus manserit quod super\ae dificavit, mercedem accipiet.
${}^{15}$~Si cujus opus arserit, detrimentum patietur~: ipse autem salvus erit, sic tamen quasi per ignem.
${}^{16}$~Nescitis quia templum Dei estis, et Spiritus Dei habitat in vobis~?
${}^{17}$~Si quis autem templum Dei violaverit, disperdet illum Deus. Templum enim Dei sanctum est, quod estis vos.


${}^{18}$~Nemo se seducat~: si quis videtur inter vos sapiens esse in hoc s\ae culo, stultus fiat ut sit sapiens.
${}^{19}$~Sapientia enim hujus mundi, stultitia est apud Deum. Scriptum est enim~: Comprehendam sapientes in astutia eorum.
${}^{20}$~Et iterum~: Dominus novit cogitationes sapientium quoniam van\ae\ sunt.
${}^{21}$~Nemo itaque glorietur in hominibus.
${}^{22}$~Omnia enim vestra sunt, sive Paulus, sive Apollo, sive Cephas, sive mundus, sive vita, sive mors, sive pr\ae sentia, sive futura~: omnia enim vestra sunt~:
${}^{23}$~vos autem Christi~: Christus autem Dei.

\bchapter
\mylettrine{S}ic nos existimet homo ut ministros Christi, et dispensatores mysteriorum Dei.
${}^{2}$~Hic jam qu\ae ritur inter dispensatores ut fidelis quis inveniatur.
${}^{3}$~Mihi autem pro minimo est ut a vobis judicer, aut ab humano die~: sed neque meipsum judico.
${}^{4}$~Nihil enim mihi conscius sum, sed non in hoc justificatus sum~: qui autem judicat me, Dominus est.
${}^{5}$~Itaque nolite ante tempus judicare, quoadusque veniat Dominus~: qui et illuminabit abscondita tenebrarum, et manifestabit consilia cordium~: et tunc laus erit unicuique a Deo.
${}^{6}$~H\ae c autem, fratres, transfiguravi in me et Apollo, propter vos~: ut in nobis discatis, ne supra quam scriptum est, unus adversus alterum infletur pro alio.
${}^{7}$~Quis enim te discernit~? quid autem habes quod non accepisti~? si autem accepisti, quid gloriaris quasi non acceperis~?


${}^{8}$~Jam saturati estis, jam divites facti estis~: sine nobis regnatis~: et utinam regnetis, ut et nos vobiscum regnemus.
${}^{9}$~Puto enim quod Deus nos Apostolos novissimos ostendit, tamquam morti destinatos~: quia spectaculum facti sumus mundo, et angelis, et hominibus.
${}^{10}$~Nos stulti propter Christum, vos autem prudentes in Christo~: nos infirmi, vos autem fortes~: vos nobiles, nos autem ignobiles.
${}^{11}$~Usque in hanc horam et esurimus, et sitimus, et nudi sumus, et colaphis c\ae dimur, et instabiles sumus,
${}^{12}$~et laboramus operantes manibus nostris~: maledicimur, et benedicimus~: persecutionem patimur, et sustinemus~:
${}^{13}$~blasphemamur, et obsecramus~: tamquam purgamenta hujus mundi facti sumus, omnium peripsema usque adhuc.


${}^{14}$~Non ut confundam vos, h\ae c scribo, sed ut filios meos carissimos moneo.
${}^{15}$~Nam si decem millia p\ae dagogorum habeatis in Christo, sed non multos patres. Nam in Christo Jesu per Evangelium ego vos genui.
${}^{16}$~Rogo ergo vos, imitatores mei estote, sicut et ego Christi.
${}^{17}$~Ideo misi ad vos Timotheum, qui est filius meus carissimus, et fidelis in Domino~: qui vos commonefaciet vias meas, qu\ae\ sunt in Christo Jesu, sicut ubique in omni ecclesia doceo.
${}^{18}$~Tamquam non venturus sim ad vos, sic inflati sunt quidam.
${}^{19}$~Veniam autem ad vos cito, si Dominus voluerit~: et cognoscam non sermonem eorum qui inflati sunt, sed virtutem.
${}^{20}$~Non enim in sermone est regnum Dei, sed in virtute.
${}^{21}$~Quid vultis~? in virga veniam ad vos, an in caritate, et spiritu mansuetudinis~?

\bchapter
\mylettrine{O}mnino auditur inter vos fornicatio, et talis fornicatio, qualis nec inter gentes, ita ut uxorem patris sui aliquis habeat.
${}^{2}$~Et vos inflati estis~: et non magis luctum habuistis ut tollatur de medio vestrum qui hoc opus fecit.
${}^{3}$~Ego quidem absens corpore, pr\ae sens autem spiritu, jam judicavi ut pr\ae sens eum, qui sic operatus est,
${}^{4}$~in nomine Domini nostri Jesu Christi, congregatis vobis et meo spiritu, cum virtute Domini nostri Jesu,
${}^{5}$~tradere hujusmodi Satan\ae\ in interitum carnis, ut spiritus salvus sit in die Domini nostri Jesu Christi.


${}^{6}$~Non est bona gloriatio vestra. Nescitis quia modicum fermentum totam massam corrumpit~?
${}^{7}$~Expurgate vetus fermentum, ut sitis nova conspersio, sicut estis azymi. Etenim Pascha nostrum immolatus est Christus.
${}^{8}$~Itaque epulemur~: non in fermento veteri, neque in fermento maliti\ae\ et nequiti\ae~: sed in azymis sinceritatis et veritatis.
${}^{9}$~Scripsi vobis in epistola~: Ne commisceamini fornicariis~:
${}^{10}$~non utique fornicariis hujus mundi, aut avaris, aut rapacibus, aut idolis servientibus~: alioquin debueratis de hoc mundo exiisse.
${}^{11}$~Nunc autem scripsi vobis non commisceri~: si is qui frater nominatur, est fornicator, aut avarus, aut idolis serviens, aut maledicus, aut ebriosus, aut rapax, cum ejusmodi nec cibum sumere.
${}^{12}$~Quid enim mihi de iis qui foris sunt, judicare~? nonne de iis qui intus sunt, vos judicatis~?
${}^{13}$~nam eos qui foris sunt, Deus judicabit. Auferte malum ex vobis ipsis.

\bchapter
\mylettrine{A}udet aliquis vestrum habens negotium adversus alterum, judicari apud iniquos, et non apud sanctos~?
${}^{2}$~an nescitis quoniam sancti de hoc mundo judicabunt~? et si in vobis judicabitur mundus, indigni estis qui de minimis judicetis~?
${}^{3}$~Nescitis quoniam angelos judicabimus~? quanto magis s\ae cularia~?
${}^{4}$~S\ae cularia igitur judicia si habueritis~: contemptibiles, qui sunt in ecclesia, illos constituite ad judicandum.
${}^{5}$~Ad verecundiam vestram dico. Sic non est inter vos sapiens quisquam, qui possit judicare inter fratrem suum~?
${}^{6}$~Sed frater cum fratre judicio contendit~: et hoc apud infideles~?
${}^{7}$~Jam quidem omnino delictum est in vobis, quod judicia habetis inter vos. Quare non magis injuriam accipitis~? quare non magis fraudem patimini~?
${}^{8}$~Sed vos injuriam facitis, et fraudatis~: et hoc fratribus.


${}^{9}$~An nescitis quia iniqui regnum Dei non possidebunt~? Nolite errare~: neque fornicarii, neque idolis servientes, neque adulteri,
${}^{10}$~neque molles, neque masculorum concubitores, neque fures, neque avari, neque ebriosi, neque maledici, neque rapaces regnum Dei possidebunt.
${}^{11}$~Et h\ae c quidam fuistis~: sed abluti estis, sed sanctificati estis, sed justificati estis in nomine Domini nostri Jesu Christi, et in Spiritu Dei nostri.
${}^{12}$~Omnia mihi licent, sed non omnia expediunt~: omnia mihi licent, sed ego sub nullis redigar potestate.
${}^{13}$~Esca ventri, et venter escis~: Deus autem et hunc et has destruet~: corpus autem non fornicationi, sed Domino~: et Dominus corpori.
${}^{14}$~Deus vero et Dominum suscitavit~: et nos suscitabit per virtutem suam.
${}^{15}$~Nescitis quoniam corpora vestra membra sunt Christi~? Tollens ergo membra Christi, faciam membra meretricis~? Absit.
${}^{16}$~An nescitis quoniam qui adh\ae ret meretrici, unum corpus efficitur~? Erunt enim (inquit) duo in carne una.
${}^{17}$~Qui autem adh\ae ret Domino, unus spiritus est.
${}^{18}$~Fugite fornicationem. Omne peccatum, quodcumque fecerit homo, extra corpus est~: qui autem fornicatur, in corpus suum peccat.
${}^{19}$~An nescitis quoniam membra vestra, templum sunt Spiritus Sancti, qui in vobis est, quem habetis a Deo, et non estis vestri~?
${}^{20}$~Empti enim estis pretio magno. Glorificate, et portate Deum in corpore vestro.

\bchapter
\mylettrine{D}e quibus autem scripsistis mihi~: Bonum est homini mulierem non tangere~:
${}^{2}$~propter fornicationem autem unusquisque suam uxorem habeat, et unaqu\ae que suum virum habeat.
${}^{3}$~Uxori vir debitum reddat~: similiter autem et uxor viro.
${}^{4}$~Mulier sui corporis potestatem non habet, sed vir. Similiter autem et vir sui corporis potestatem non habet, sed mulier.
${}^{5}$~Nolite fraudare invicem, nisi forte ex consensu ad tempus, ut vacetis orationi~: et iterum revertimini in idipsum, ne tentet vos Satanas propter incontinentiam vestram.
${}^{6}$~Hoc autem dico secundum indulgentiam, non secundum imperium.
${}^{7}$~Volo enim omnes vos esse sicut meipsum~: sed unusquisque proprium donum habet ex Deo~: alius quidem sic, alius vero sic.
${}^{8}$~Dico autem non nuptis, et viduis~: bonum est illis si sic permaneant, sicut et ego.
${}^{9}$~Quod si non se continent, nubant. Melius est enim nubere, quam uri.


${}^{10}$~Iis autem qui matrimonio juncti sunt, pr\ae cipio non ego, sed Dominus, uxorem a viro non discedere~:
${}^{11}$~quod si discesserit, manere innuptam, aut viro suo reconciliari. Et vir uxorem non dimittat.


${}^{12}$~Nam ceteris ego dico, non Dominus. Si quis frater uxorem habet infidelem, et h\ae c consentit habitare cum illo, non dimittat illam.
${}^{13}$~Et si qua mulier fidelis habet virum infidelem, et hic consentit habitare cum illa, non dimittat virum~:
${}^{14}$~sanctificatus est enim vir infidelis per mulierem fidelem, et sanctificata est mulier infidelis per virum fidelem~: alioquin filii vestri immundi essent, nunc autem sancti sunt.
${}^{15}$~Quod si infidelis discedit, discedat~: non enim servituti subjectus est frater, aut soror in hujusmodi~: in pace autem vocavit nos Deus.
${}^{16}$~Unde enim scis mulier, si virum salvum facies~? aut unde scis vir, si mulierem salvam facies~?


${}^{17}$~Nisi unicuique sicut divisit Dominus, unumquemque sicut vocavit Deus, ita ambulet, et sicut in omnibus ecclesiis doceo.
${}^{18}$~Circumcisus aliquis vocatus est~? non adducat pr\ae putium. In pr\ae putio aliquis vocatus est~? non circumcidatur.
${}^{19}$~Circumcisio nihil est, et pr\ae putium nihil est~: sed observatio mandatorum Dei.
${}^{20}$~Unusquisque in qua vocatione vocatus est, in ea permaneat.
${}^{21}$~Servus vocatus es~? non sit tibi cur\ae~: sed et si potes fieri liber, magis utere.
${}^{22}$~Qui enim in Domino vocatus est servus, libertus est Domini~: similiter qui liber vocatus est, servus est Christi.
${}^{23}$~Pretio empti estis~: nolite fieri servi hominum.
${}^{24}$~Unusquisque in quo vocatus est, fratres, in hoc permaneat apud Deum.


${}^{25}$~De virginibus autem pr\ae ceptum Domini non habeo~: consilium autem do, tamquam misericordiam consecutus a Domino, ut sim fidelis.
${}^{26}$~Existimo ergo hoc bonum esse propter instantem necessitatem, quoniam bonum est homini sic esse.
${}^{27}$~Alligatus es uxori~? noli qu\ae rere solutionem. Solutus es ab uxore~? noli qu\ae rere uxorem.
${}^{28}$~Si autem acceperis uxorem, non peccasti. Et si nupserit virgo, non peccavit~: tribulationem tamen carnis habebunt hujusmodi. Ego autem vobis parco.
${}^{29}$~Hoc itaque dico, fratres~: tempus breve est~: reliquum est, ut et qui habent uxores, tamquam non habentes sint~:
${}^{30}$~et qui flent, tamquam non flentes~: et qui gaudent, tamquam non gaudentes~: et qui emunt, tamquam non possidentes~:
${}^{31}$~et qui utuntur hoc mundo, tamquam non utantur~: pr\ae terit enim figura hujus mundi.
${}^{32}$~Volo autem vos sine sollicitudine esse. Qui sine uxore est, sollicitus est qu\ae\ Domini sunt, quomodo placeat Deo.
${}^{33}$~Qui autem cum uxore est, sollicitus est qu\ae\ sunt mundi, quomodo placeat uxori, et divisus est.
${}^{34}$~Et mulier innupta, et virgo, cogitat qu\ae\ Domini sunt, ut sit sancta corpore, et spiritu. Qu\ae\ autem nupta est, cogitat qu\ae\ sunt mundi, quomodo placeat viro.
${}^{35}$~Porro hoc ad utilitatem vestram dico~: non ut laqueum vobis injiciam, sed ad id, quod honestum est, et quod facultatem pr\ae beat sine impedimento Dominum obsecrandi.
${}^{36}$~Si quis autem turpem se videri existimat super virgine sua, quod sit superadulta, et ita oportet fieri~: quod vult faciat~: non peccat, si nubat.
${}^{37}$~Nam qui statuit in corde suo firmus, non habens necessitatem, potestatem autem habens su\ae\ voluntatis, et hoc judicavit in corde suo, servare virginem suam, bene facit.
${}^{38}$~Igitur et qui matrimonio jungit virginem suam, bene facit~: et qui non jungit, melius facit.
${}^{39}$~Mulier alligata est legi quanto tempore vir ejus vivit, quod si dormierit vir ejus, liberata est~: cui vult nubat, tantum in Domino.
${}^{40}$~Beatior autem erit si sic permanserit secundum meum consilium~: puto autem quod et ego Spiritum Dei habeam.

\bchapter
\mylettrine{D}e iis autem qu\ae\ idolis sacrificantur, scimus quia omnes scientiam habemus. Scientia inflat, caritas vero \ae dificat.
${}^{2}$~Si quis autem se existimat scire aliquid, nondum cognovit quemadmodum oporteat eum scire.
${}^{3}$~Si quis autem diligit Deum, hic cognitus est ab eo.
${}^{4}$~De escis autem qu\ae\ idolis immolantur, scimus quia nihil est idolum in mundo, et quod nullus est Deus, nisi unus.
${}^{5}$~Nam etsi sunt qui dicantur dii sive in c\ae lo, sive in terra (siquidem sunt dii multi, et domini multi)~:
${}^{6}$~nobis tamen unus est Deus, Pater, ex quo omnia, et nos in illum~: et unus Dominus Jesus Christus, per quem omnia, et nos per ipsum.
${}^{7}$~Sed non in omnibus est scientia. Quidam autem cum conscientia usque nunc idoli, quasi idolothytum manducant~: et conscientia ipsorum cum sit infirma, polluitur.
${}^{8}$~Esca autem nos non commendat Deo. Neque enim si manducaverimus, abundabimus~: neque si non manducaverimus, deficiemus.
${}^{9}$~Videte autem ne forte h\ae c licentia vestra offendiculum fiat infirmis.
${}^{10}$~Si enim quis viderit eum, qui habet scientiam, in idolio recumbentem~: nonne conscientia ejus, cum sit infirma, \ae dificabitur ad manducandum idolothyta~?
${}^{11}$~Et peribit infirmus in tua scientia, frater, propter quem Christus mortuus est~?
${}^{12}$~Sic autem peccantes in fratres, et percutientes conscientiam eorum infirmam, in Christum peccatis.
${}^{13}$~Quapropter si esca scandalizat fratrem meum, non manducabo carnem in \ae ternum, ne fratrem meum scandalizem.

\bchapter
\mylettrine{N}on sum liber~? non sum Apostolus~? nonne Christum Jesum Dominum nostrum vidi~? nonne opus meum vos estis in Domino~?
${}^{2}$~Et si aliis non sum Apostolus, sed tamen vobis sum~: nam signaculum apostolatus mei vos estis in Domino.
${}^{3}$~Mea defensio apud eos qui me interrogant, h\ae c est~:
${}^{4}$~Numquid non habemus potestatem manducandi et bibendi~?
${}^{5}$~numquid non habemus potestatem mulierem sororem circumducendi sicut et ceteri Apostoli, et fratres Domini, et Cephas~?
${}^{6}$~aut ego solus, et Barnabas, non habemus potestatem hoc operandi~?
${}^{7}$~Quis militat suis stipendiis umquam~? quis plantat vineam, et de fructu ejus non edit~? quis pascit gregem, et de lacte gregis non manducat~?
${}^{8}$~Numquid secundum hominem h\ae c dico~? an et lex h\ae c non dicit~?
${}^{9}$~Scriptum est enim in lege Moysi~: Non alligabis os bovi trituranti. Numquid de bobus cura est Deo~?
${}^{10}$~an propter nos utique hoc dicit~? Nam propter nos scripta sunt~: quoniam debet in spe qui arat, arare~: et qui triturat, in spe fructus percipiendi.
${}^{11}$~Si nos vobis spiritualia seminavimus, magnum est si nos carnalia vestra metamus~?
${}^{12}$~Si alii potestatis vestr\ae\ participes sunt, quare non potius nos~? Sed non usi sumus hac potestate~: sed omnia sustinemus, ne quod offendiculum demus Evangelio Christi.
${}^{13}$~Nescitis quoniam qui in sacrario operantur qu\ae\ de sacrario sunt, edunt~: et qui altari deserviunt, cum altari participant~?
${}^{14}$~Ita et Dominus ordinavit iis qui Evangelium annuntiant, de Evangelio vivere.
${}^{15}$~Ego autem nullo horum usus sum. Non autem scripsi h\ae c ut ita fiant in me~: bonum est enim mihi magis mori, quam ut gloriam meam quis evacuet.
${}^{16}$~Nam si evangelizavero, non est mihi gloria~: necessitas enim mihi incumbit~: v\ae\ enim mihi est, si non evangelizavero.
${}^{17}$~Si enim volens hoc ago, mercedem habeo~: si autem invitus, dispensatio mihi credita est.
${}^{18}$~Qu\ae\ est ergo merces mea~? ut Evangelium pr\ae dicans, sine sumptu ponam Evangelium, ut non abutar potestate mea in Evangelio.


${}^{19}$~Nam cum liber essem ex omnibus, omnium me servum feci, ut plures lucrifacerem.
${}^{20}$~Et factus sum Jud\ae is tamquam Jud\ae us, ut Jud\ae os lucrarer~:
${}^{21}$~iis qui sub lege sunt, quasi sub lege essem (cum ipse non essem sub lege) ut eos qui sub lege erant, lucrifacerem~: iis qui sine lege erant, tamquam sine lege essem (cum sine lege Dei non essem~: sed in lege essem Christi) ut lucrifacerem eos qui sine lege erant.
${}^{22}$~Factus sum infirmis infirmus, ut infirmos lucrifacerem. Omnibus omnia factus sum, ut omnes facerem salvos.
${}^{23}$~Omnia autem facio propter Evangelium~: ut particeps ejus efficiar.
${}^{24}$~Nescitis quod ii qui in stadio currunt, omnes quidem currunt, sed unus accipit bravium~? Sic currite ut comprehendatis.
${}^{25}$~Omnis autem qui in agone contendit, ab omnibus se abstinet, et illi quidem ut corruptibilem coronam accipiant~: nos autem incorruptam.
${}^{26}$~Ego igitur sic curro, non quasi in incertum~: sic pugno, non quasi a\"erem verberans~:
${}^{27}$~sed castigo corpus meum, et in servitutem redigo~: ne forte cum aliis pr\ae dicaverim, ipse reprobus efficiar.

\bchapter
\mylettrine{N}olo enim vos ignorare fratres, quoniam patres nostri omnes sub nube fuerunt, et omnes mare transierunt,
${}^{2}$~et omnes in Moyse baptizati sunt in nube, et in mari~:
${}^{3}$~et omnes eamdem escam spiritalem manducaverunt,
${}^{4}$~et omnes eumdem potum spiritalem biberunt (bibebant autem de spiritali, consequente eos, petra~: petra autem erat Christus)~:
${}^{5}$~sed non in pluribus eorum beneplacitum est Deo~: nam prostrati sunt in deserto.
${}^{6}$~H\ae c autem in figura facta sunt nostri, ut non simus concupiscentes malorum, sicut et illi concupierunt.
${}^{7}$~Neque idololatr\ae\ efficiamini, sicut quidam ex ipsis~: quemadmodum scriptum est~: Sedit populus manducare, et bibere, et surrexerunt ludere.
${}^{8}$~Neque fornicemur, sicut quidam ex ipsis fornicati sunt, et ceciderunt una die viginti tria millia.
${}^{9}$~Neque tentemus Christum, sicut quidam eorum tentaverunt, et a serpentibus perierunt.
${}^{10}$~Neque murmuraveritis, sicut quidam eorum murmuraverunt, et perierunt ab exterminatore.
${}^{11}$~H\ae c autem omnia in figura contingebant illis~: scripta sunt autem ad correptionem nostram, in quos fines s\ae culorum devenerunt.
${}^{12}$~Itaque qui se existimat stare, videat ne cadat.
${}^{13}$~Tentatio vos non apprehendat nisi humana~: fidelis autem Deus est, qui non patietur vos tentari supra id quod potestis, sed faciet etiam cum tentatione proventum ut possitis sustinere.


${}^{14}$~Propter quod, carissimi mihi, fugite ab idolorum cultura~:
${}^{15}$~ut prudentibus loquor, vos ipsi judicate quod dico.
${}^{16}$~Calix benedictionis, cui benedicimus, nonne communicatio sanguinis Christi est~? et panis quem frangimus, nonne participatio corporis Domini est~?
${}^{17}$~Quoniam unus panis, unum corpus multi sumus, omnes qui de uno pane participamus.
${}^{18}$~Videte Isra\"el secundum carnem~: nonne qui edunt hostias, participes sunt altaris~?
${}^{19}$~Quid ergo~? dico quod idolis immolatum sit aliquid~? aut quod idolum, sit aliquid~?
${}^{20}$~Sed qu\ae\ immolant gentes, d\ae moniis immolant, et non Deo. Nolo autem vos socios fieri d\ae moniorum~:
${}^{21}$~non potestis calicem Domini bibere, et calicem d\ae moniorum~; non potestis mens\ae\ Domini participes esse, et mens\ae\ d\ae moniorum.
${}^{22}$~An \ae mulamur Dominum~? numquid fortiores illo sumus~? Omnia mihi licent, sed non omnia expediunt.
${}^{23}$~Omnia mihi licent, sed non omnia \ae dificat.


${}^{24}$~Nemo quod suum est qu\ae rat, sed quod alterius.
${}^{25}$~Omne quod in macello venit, manducate, nihil interrogantes propter conscientiam.
${}^{26}$~Domini est terra, et plenitudo ejus.
${}^{27}$~Si quis vocat vos infidelium, et vultis ire~: omne quod vobis apponitur, manducate, nihil interrogantes propter conscientiam.
${}^{28}$~Si quis autem dixerit~: Hoc immolatum est idolis~: nolite manducare propter illum qui indicavit, et propter conscientiam~:
${}^{29}$~conscientiam autem dico non tuam, sed alterius. Ut quid enim libertas mea judicatur ab aliena conscientia~?
${}^{30}$~Si ego cum gratia participo, quid blasphemor pro eo quod gratias ago~?
${}^{31}$~Sive ergo manducatis, sive bibitis, sive aliud quid facitis~: omnia in gloriam Dei facite.
${}^{32}$~Sine offensione estote Jud\ae is, et gentibus, et ecclesi\ae\ Dei~:
${}^{33}$~sicut et ego per omnia omnibus placeo, non qu\ae rens quod mihi utile est, sed quod multis~: ut salvi fiant.

\bchapter
\mylettrine{I}mitatores mei estote, sicut et ego Christi.
${}^{2}$~Laudo autem vos fratres quod per omnia mei memores estis~: et sicut tradidi vobis, pr\ae cepta mea tenetis.
${}^{3}$~Volo autem vos scire quod omnis viri caput, Christus est~: caput autem mulieris, vir~: caput vero Christi, Deus.
${}^{4}$~Omnis vir orans, aut prophetans velato capite, deturpat caput suum.
${}^{5}$~Omnis autem mulier orans, aut prophetans non velato capite, deturpat caput suum~: unum enim est ac si decalvetur.
${}^{6}$~Nam si non velatur mulier, tondeatur. Si vero turpe est mulieri tonderi, aut decalvari, velet caput suum.
${}^{7}$~Vir quidem non debet velare caput suum~: quoniam imago et gloria Dei est, mulier autem gloria viri est.
${}^{8}$~Non enim vir ex muliere est, sed mulier ex viro.
${}^{9}$~Etenim non est creatus vir propter mulierem, sed mulier propter virum.
${}^{10}$~Ideo debet mulier potestatem habere supra caput propter angelos.
${}^{11}$~Verumtamen neque vir sine muliere~: neque mulier sine viro in Domino.
${}^{12}$~Nam sicut mulier de viro, ita et vir per mulierem~: omnia autem ex Deo.
${}^{13}$~Vos ipsi judicate~: decet mulierem non velatam orare Deum~?
${}^{14}$~Nec ipsa natura docet vos, quod vir quidem si comam nutriat, ignominia est illi~:
${}^{15}$~mulier vero si comam nutriat, gloria est illi~: quoniam capilli pro velamine ei dati sunt.
${}^{16}$~Si quis autem videtur contentiosus esse~: nos talem consuetudinem non habemus, neque ecclesia Dei.


${}^{17}$~Hoc autem pr\ae cipio~: non laudans quod non in melius, sed in deterius convenitis.
${}^{18}$~Primum quidem convenientibus vobis in ecclesiam, audio scissuras esse inter vos, et ex parte credo.
${}^{19}$~Nam oportet et h\ae reses esse, ut et qui probati sunt, manifesti fiant in vobis.
${}^{20}$~Convenientibus ergo vobis in unum, jam non est Dominicam cœnam manducare.
${}^{21}$~Unusquisque enim suam cœnam pr\ae sumit ad manducandum, et alius quidem esurit, alius autem ebrius est.
${}^{22}$~Numquid domos non habetis ad manducandum, et bibendum~? aut ecclesiam Dei contemnitis, et confunditis eos qui non habent~? Quid dicam vobis~? laudo vos~? in hoc non laudo.
${}^{23}$~Ego enim accepi a Domino quod et tradidi vobis, quoniam Dominus Jesus in qua nocte tradebatur, accepit panem,
${}^{24}$~et gratias agens fregit, et dixit~: Accipite, et manducate~: hoc est corpus meum, quod pro vobis tradetur~: hoc facite in meam commemorationem.
${}^{25}$~Similiter et calicem, postquam cœnavit, dicens~: Hic calix novum testamentum est in meo sanguine~: hoc facite quotiescumque bibetis, in meam commemorationem.
${}^{26}$~Quotiescumque enim manducabitis panem hunc, et calicem bibetis, mortem Domini annuntiabitis donec veniat.
${}^{27}$~Itaque quicumque manducaverit panem hunc, vel biberit calicem Domini indigne, reus erit corporis et sanguinis Domini.
${}^{28}$~Probet autem seipsum homo~: et sic de pane illo edat, et de calice bibat.
${}^{29}$~Qui enim manducat et bibit indigne, judicium sibi manducat et bibit, non dijudicans corpus Domini.
${}^{30}$~Ideo inter vos multi infirmi et imbecilles, et dormiunt multi.
${}^{31}$~Quod si nosmetipsos dijudicaremus, non utique judicaremur.
${}^{32}$~Dum judicamur autem, a Domino corripimur, ut non cum hoc mundo damnemur.
${}^{33}$~Itaque fratres mei, cum convenitis ad manducandum, invicem exspectate.
${}^{34}$~Si quis esurit, domi manducet, ut non in judicium conveniatis. Cetera autem, cum venero, disponam.

\bchapter
\mylettrine{D}e spiritualibus autem, nolo vos ignorare fratres.
${}^{2}$~Scitis quoniam cum gentes essetis, ad simulacra muta prout ducebamini euntes.
${}^{3}$~Ideo notum vobis facio, quod nemo in Spiritu Dei loquens, dicit anathema Jesu. Et nemo potest dicere, Dominus Jesus, nisi in Spiritu Sancto.
${}^{4}$~Divisiones vero gratiarum sunt, idem autem Spiritus~:
${}^{5}$~et divisiones ministrationum sunt, idem autem Dominus~:
${}^{6}$~et divisiones operationum sunt, idem vero Deus qui operatur omnia in omnibus.
${}^{7}$~Unicuique autem datur manifestatio Spiritus ad utilitatem.
${}^{8}$~Alii quidem per Spiritum datur sermo sapienti\ae~: alii autem sermo scienti\ae\ secundum eumdem Spiritum~:
${}^{9}$~alteri fides in eodem Spiritu~: alii gratia sanitatum in uno Spiritu~:
${}^{10}$~alii operatio virtutum, alii prophetia, alii discretio spirituum, alii genera linguarum, alii interpretatio sermonum.
${}^{11}$~H\ae c autem omnia operatur unus atque idem Spiritus, dividens singulis prout vult.


${}^{12}$~Sicut enim corpus unum est, et membra habet multa, omnia autem membra corporis cum sint multa, unum tamen corpus sunt~: ita et Christus.
${}^{13}$~Etenim in uno Spiritu omnes nos in unum corpus baptizati sumus, sive Jud\ae i, sive gentiles, sive servi, sive liberi~: et omnes in uno Spiritu potati sumus.
${}^{14}$~Nam et corpus non est unum membrum, sed multa.
${}^{15}$~Si dixerit pes~: Quoniam non sum manus, non sum de corpore~: num ideo non est de corpore~?
${}^{16}$~Et si dixerit auris~: Quoniam non sum oculus, non sum de corpore~: num ideo est de corpore~?
${}^{17}$~Si totum corpus oculus~: ubi auditus~? Si totum auditus~: ubi odoratus~?
${}^{18}$~Nunc autem posuit Deus membra, unumquodque eorum in corpore sicut voluit.
${}^{19}$~Quod si essent omnia unum membrum, ubi corpus~?
${}^{20}$~Nunc autem multa quidem membra, unum autem corpus.
${}^{21}$~Non potest autem oculus dicere manui~: Opera tua non indigeo~: aut iterum caput pedibus~: Non estis mihi necessarii.
${}^{22}$~Sed multo magis qu\ae\ videntur membra corporis infirmiora esse, necessariora sunt~:
${}^{23}$~et qu\ae\ putamus ignobiliora membra esse corporis, his honorem abundantiorem circumdamus~: et qu\ae\ inhonesta sunt nostra, abundantiorem honestatem habent.
${}^{24}$~Honesta autem nostra nullius egent~: sed Deus temperavit corpus, ei cui deerat, abundantiorem tribuendo honorem,
${}^{25}$~ut non sit schisma in corpore, sed idipsum pro invicem sollicita sint membra.
${}^{26}$~Et si quid patitur unum membrum, compatiuntur omnia membra~: sive gloriatur unum membrum, congaudent omnia membra.


${}^{27}$~Vos autem estis corpus Christi, et membra de membro.
${}^{28}$~Et quosdam quidem posuit Deus in ecclesia primum apostolos, secundo prophetas, exinde doctores, deinde virtutes, exinde gratias curationum, opitulationes, gubernationes, genera linguarum, interpretationes sermonum.
${}^{29}$~Numquid omnes apostoli~? numquid omnes prophet\ae~? numquid omnes doctores~?
${}^{30}$~numquid omnes virtutes~? numquid omnes gratiam habent curationum~? numquid omnes linguis loquuntur~? numquid omnes interpretantur~?
${}^{31}$~\AE mulamini autem charismata meliora. Et adhuc excellentiorem viam vobis demonstro.

\bchapter
\mylettrine{S}i linguis hominum loquar, et angelorum, caritatem autem non habeam, factus sum velut \ae s sonans, aut cymbalum tinniens.
${}^{2}$~Et si habuero prophetiam, et noverim mysteria omnia, et omnem scientiam~: et si habuero omnem fidem ita ut montes transferam, caritatem autem non habuero, nihil sum.
${}^{3}$~Et si distribuero in cibos pauperum omnes facultates meas, et si tradidero corpus meum ita ut ardeam, caritatem autem non habuero, nihil mihi prodest.


${}^{4}$~Caritas patiens est, benigna est. Caritas non \ae mulatur, non agit perperam, non inflatur,
${}^{5}$~non est ambitiosa, non qu\ae rit qu\ae\ sua sunt, non irritatur, non cogitat malum,
${}^{6}$~non gaudet super iniquitate, congaudet autem veritati~:
${}^{7}$~omnia suffert, omnia credit, omnia sperat, omnia sustinet.


${}^{8}$~Caritas numquam excidit~: sive propheti\ae\ evacuabuntur, sive lingu\ae\ cessabunt, sive scientia destruetur.
${}^{9}$~Ex parte enim cognoscimus, et ex parte prophetamus.
${}^{10}$~Cum autem venerit quod perfectum est, evacuabitur quod ex parte est.
${}^{11}$~Cum essem parvulus, loquebar ut parvulus, sapiebam ut parvulus, cogitabam ut parvulus. Quando autem factus sum vir, evacuavi qu\ae\ erant parvuli.
${}^{12}$~Videmus nunc per speculum in \ae nigmate~: tunc autem facie ad faciem. Nunc cognosco ex parte~: tunc autem cognoscam sicut et cognitus sum.
${}^{13}$~Nunc autem manent fides, spes, caritas, tria h\ae c~: major autem horum est caritas.

\bchapter
\mylettrine{S}ectamini caritatem, \ae mulamini spiritualia~: magis autem ut prophetetis.
${}^{2}$~Qui enim loquitur lingua, non hominibus loquitur, sed Deo~: nemo enim audit. Spiritu autem loquitur mysteria.
${}^{3}$~Nam qui prophetat, hominibus loquitur ad \ae dificationem, et exhortationem, et consolationem.
${}^{4}$~Qui loquitur lingua, semetipsum \ae dificat~: qui autem prophetat, ecclesiam Dei \ae dificat.
${}^{5}$~Volo autem omnes vos loqui linguis~: magis autem prophetare. Nam major est qui prophetat, quam qui loquitur linguis~; nisi forte interpretetur ut ecclesia \ae dificationem accipiat.
${}^{6}$~Nunc autem, fratres, si venero ad vos linguis loquens~: quid vobis prodero, nisi vobis loquar aut in revelatione, aut in scientia, aut in prophetia, aut in doctrina~?
${}^{7}$~Tamen qu\ae\ sine anima sunt vocem dantia, sive tibia, sive cithara~; nisi distinctionem sonituum dederint, quomodo scietur id quod canitur, aut quod citharizatur~?
${}^{8}$~Etenim si incertam vocem det tuba, quis parabit se ad bellum~?
${}^{9}$~Ita et vos per linguam nisi manifestum sermonem dederitis~: quomodo scietur id quod dicitur~? eritis enim in a\"era loquentes.
${}^{10}$~Tam multa, ut puta genera linguarum sunt in hoc mundo~: et nihil sine voce est.
${}^{11}$~Si ergo nesciero virtutem vocis, ero ei, cui loquor, barbarus~: et qui loquitur, mihi barbarus.
${}^{12}$~Sic et vos, quoniam \ae mulatores estis spirituum, ad \ae dificationem ecclesi\ae\ qu\ae rite ut abundetis.
${}^{13}$~Et ideo qui loquitur lingua, oret ut interpretetur.
${}^{14}$~Nam si orem lingua, spiritus meus orat, mens autem mea sine fructu est.
${}^{15}$~Quid ergo est~? Orabo spiritu, orabo et mente~: psallam spiritu, psallam et mente.
${}^{16}$~Ceterum si benedixeris spiritu, qui supplet locum idiot\ae , quomodo dicet~: Amen, super tuam benedictionem~? quoniam quid dicas, nescit.
${}^{17}$~Nam tu quidem bene gratias agis, sed alter non \ae dificatur.
${}^{18}$~Gratias ago Deo meo, quod omnium vestrum lingua loquor.
${}^{19}$~Sed in ecclesia volo quinque verba sensu meo loqui, ut et alios instruam~: quam decem millia verborum in lingua.
${}^{20}$~Fratres, nolite pueri effici sensibus, sed malitia parvuli estote~: sensibus autem perfecti estote.
${}^{21}$~In lege scriptum est~: Quoniam in aliis linguis et labiis aliis loquar populo huic~: et nec sic exaudient me, dicit Dominus.
${}^{22}$~Itaque lingu\ae\ in signum sunt non fidelibus, sed infidelibus~: propheti\ae\ autem non infidelibus, sed fidelibus.
${}^{23}$~Si ergo conveniat universa ecclesia in unum, et omnes linguis loquantur, intrent autem idiot\ae , aut infideles~: nonne dicent quod insanitis~?
${}^{24}$~Si autem omnes prophetent, intret autem quis infidelis, vel idiota, convincitur ab omnibus, dijudicatur ab omnibus~:
${}^{25}$~occulta cordis ejus manifesta fiunt~: et ita cadens in faciem adorabit Deum, pronuntians quod vere Deus in vobis sit.


${}^{26}$~Quid ergo est, fratres~? Cum convenitis, unusquisque vestrum psalmum habet, doctrinam habet, apocalypsim habet, linguam habet, interpretationem habet~: omnia ad \ae dificationem fiant.
${}^{27}$~Sive lingua quis loquitur, secundum duos, aut ut multum tres, et per partes, et unus interpretatur.
${}^{28}$~Si autem non fuerit interpres, taceat in ecclesia~: sibi autem loquatur, et Deo.
${}^{29}$~Prophet\ae\ autem duo, aut tres dicant, et ceteri dijudicent.
${}^{30}$~Quod si alii revelatum fuerit sedenti, prior taceat.
${}^{31}$~Potestis enim omnes per singulos prophetare~: ut omnes discant, et omnes exhortentur~:
${}^{32}$~et spiritus prophetarum prophetis subjecti sunt.
${}^{33}$~Non enim est dissensionis Deus, sed pacis~: sicut et in omnibus ecclesiis sanctorum doceo.
${}^{34}$~Mulieres in ecclesiis taceant, non enim permittitur eis loqui, sed subditas esse, sicut et lex dicit.
${}^{35}$~Si quid autem volunt discere, domi viros suos interrogent. Turpe est enim mulieri loqui in ecclesia.
${}^{36}$~An a vobis verbum Dei processit~? aut in vos solos pervenit~?
${}^{37}$~Si quis videtur propheta esse, aut spiritualis, cognoscat qu\ae\ scribo vobis, quia Domini sunt mandata.
${}^{38}$~Si quis autem ignorat, ignorabitur.
${}^{39}$~Itaque fratres \ae mulamini prophetare~: et loqui linguis nolite prohibere.
${}^{40}$~Omnia autem honeste, et secundum ordinem fiant.

\bchapter
\mylettrine{N}otum autem vobis facio, fratres, Evangelium, quod pr\ae dicavi vobis, quod et accepistis, in quo et statis,
${}^{2}$~per quod et salvamini~: qua ratione pr\ae dicaverim vobis, si tenetis, nisi frustra credidistis.
${}^{3}$~Tradidi enim vobis in primis quod et accepi~: quoniam Christus mortuus est pro peccatis nostris secundum Scripturas~:
${}^{4}$~et quia sepultus est, et quia resurrexit tertia die secundum Scripturas~:
${}^{5}$~et quia visus est Ceph\ae , et post hoc undecim~:
${}^{6}$~deinde visus est plus quam quingentis fratribus simul~: ex quibus multi manent usque adhuc, quidam autem dormierunt~:
${}^{7}$~deinde visus est Jacobo, deinde Apostolis omnibus~:
${}^{8}$~novissime autem omnium tamquam abortivo, visus est et mihi.
${}^{9}$~Ego enim sum minimus Apostolorum, qui non sum dignus vocari Apostolus, quoniam persecutus sum ecclesiam Dei.
${}^{10}$~Gratia autem Dei sum id quod sum, et gratia ejus in me vacua non fuit, sed abundantius illis omnibus laboravi~: non ego autem, sed gratia Dei mecum~:
${}^{11}$~sive enim ego, sive illi~: sic pr\ae dicamus, et sic credidistis.


${}^{12}$~Si autem Christus pr\ae dicatur quod resurrexit a mortuis, quomodo quidam dicunt in vobis, quoniam resurrectio mortuorum non est~?
${}^{13}$~Si autem resurrectio mortuorum non est~: neque Christus resurrexit.
${}^{14}$~Si autem Christus non resurrexit, inanis est ergo pr\ae dicatio nostra, inanis est et fides vestra~:
${}^{15}$~invenimur autem et falsi testes Dei~: quoniam testimonium diximus adversus Deum quod suscitaverit Christum, quem non suscitavit, si mortui non resurgunt.
${}^{16}$~Nam si mortui non resurgunt, neque Christus resurrexit.
${}^{17}$~Quod si Christus non resurrexit, vana est fides vestra~: adhuc enim estis in peccatis vestris.
${}^{18}$~Ergo et qui dormierunt in Christo, perierunt.
${}^{19}$~Si in hac vita tantum in Christo sperantes sumus, miserabiliores sumus omnibus hominibus.


${}^{20}$~Nunc autem Christus resurrexit a mortuis primiti\ae\ dormientium,
${}^{21}$~quoniam quidem per hominem mors, et per hominem resurrectio mortuorum.
${}^{22}$~Et sicut in Adam omnes moriuntur, ita et in Christo omnes vivificabuntur.
${}^{23}$~Unusquisque autem in suo ordine, primiti\ae\ Christus~: deinde ii qui sunt Christi, qui in adventu ejus crediderunt.
${}^{24}$~Deinde finis~: cum tradiderit regnum Deo et Patri, cum evacuaverit omnem principatum, et potestatem, et virtutem.
${}^{25}$~Oportet autem illum regnare donec ponat omnes inimicos sub pedibus ejus.
${}^{26}$~Novissima autem inimica destruetur mors~: omnia enim subjecit pedibus ejus. Cum autem dicat~:
${}^{27}$~Omnia subjecta sunt ei, sine dubio pr\ae ter eum qui subjecit ei omnia.
${}^{28}$~Cum autem subjecta fuerint illi omnia~: tunc et ipse Filius subjectus erit ei, qui subjecit sibi omnia, ut sit Deus omnia in omnibus.
${}^{29}$~Alioquin quid facient qui baptizantur pro mortuis, si omnino mortui non resurgunt~? ut quid et baptizantur pro illis~?
${}^{30}$~ut quid et nos periclitamur omni hora~?
${}^{31}$~Quotidie morior per vestram gloriam, fratres, quam habeo in Christo Jesu Domino nostro.
${}^{32}$~Si secundum hominem ad bestias pugnavi Ephesi, quid mihi prodest, si mortui non resurgunt~? Manducemus, et bibamus, cras enim moriemur.
${}^{33}$~Nolite seduci~: corrumpunt mores bonos colloquia mala.
${}^{34}$~Evigilate justi, et nolite peccare~: ignorantiam enim Dei quidam habent, ad reverentiam vobis loquor.


${}^{35}$~Sed dicet aliquis~: Quomodo resurgunt mortui~? qualive corpore venient~?
${}^{36}$~Insipiens, tu quod seminas non vivificatur, nisi prius moriatur~:
${}^{37}$~et quod seminas, non corpus, quod futurum est, seminas, sed nudum granum, ut puta tritici, aut alicujus ceterorum.
${}^{38}$~Deus autem dat illi corpus sicut vult~: ut unicuique seminum proprium corpus.
${}^{39}$~Non omnis caro, eadem caro~: sed alia quidem hominum, alia vero pecorum, alia volucrum, alia autem piscium.
${}^{40}$~Et corpora c\ae lestia, et corpora terrestria~: sed alia quidem c\ae lestium gloria, alia autem terrestrium.
${}^{41}$~Alia claritas solis, alia claritas lun\ae , et alia claritas stellarum. Stella enim a stella differt in claritate~:
${}^{42}$~sic et resurrectio mortuorum. Seminatur in corruptione, surget in incorruptione.
${}^{43}$~Seminatur in ignobilitate, surget in gloria~: seminatur in infirmitate, surget in virtute~:
${}^{44}$~seminatur corpus animale, surget corpus spiritale. Si est corpus animale, est et spiritale, sicut scriptum est~:
${}^{45}$~Factus est primus homo Adam in animam viventem, novissimus Adam in spiritum vivificantem.
${}^{46}$~Sed non prius quod spiritale est, sed quod animale~: deinde quod spiritale.
${}^{47}$~Primus homo de terra, terrenus~: secundus homo de c\ae lo, c\ae lestis.
${}^{48}$~Qualis terrenus, tales et terreni~: et qualis c\ae lestis, tales et c\ae lestes.
${}^{49}$~Igitur, sicut portavimus imaginem terreni, portemus et imaginem c\ae lestis.


${}^{50}$~Hoc autem dico, fratres~: quia caro et sanguis regnum Dei possidere non possunt~: neque corruptio incorruptelam possidebit.
${}^{51}$~Ecce mysterium vobis dico~: omnes quidem resurgemus, sed non omnes immutabimur.
${}^{52}$~In momento, in ictu oculi, in novissima tuba~: canet enim tuba, et mortui resurgent incorrupti~: et nos immutabimur.
${}^{53}$~Oportet enim corruptibile hoc induere incorruptionem~: et mortale hoc induere immortalitatem.
${}^{54}$~Cum autem mortale hoc induerit immortalitatem, tunc fiet sermo, qui scriptus est~: Absorpta est mors in victoria.
${}^{55}$~Ubi est mors victoria tua~? ubi est mors stimulus tuus~?
${}^{56}$~Stimulus autem mortis peccatum est~: virtus vero peccati lex.
${}^{57}$~Deo autem gratias, qui dedit nobis victoriam per Dominum nostrum Jesum Christum.
${}^{58}$~Itaque fratres mei dilecti, stabiles estote, et immobiles~: abundantes in opere Domini semper, scientes quod labor vester non est inanis in Domino.

\bchapter
\mylettrine{D}e collectis autem, qu\ae\ fiunt in sanctos, sicut ordinavi ecclesiis Galati\ae , ita et vos facite.
${}^{2}$~Per unam sabbati unusquisque vestrum apud se seponat, recondens quod ei bene placuerit~: ut non, cum venero, tunc collect\ae\ fiant.
${}^{3}$~Cum autem pr\ae sens fuero, quos probaveritis per epistolas, hos mittam perferre gratiam vestram in Jerusalem.
${}^{4}$~Quod si dignum fuerit ut et ego eam, mecum ibunt.


${}^{5}$~Veniam autem ad vos, cum Macedoniam pertransiero~: nam Macedoniam pertransibo.
${}^{6}$~Apud vos autem forsitan manebo, vel etiam hiemabo~: ut vos me deducatis quocumque iero.
${}^{7}$~Nolo enim vos modo in transitu videre, spero enim me aliquantulum temporis manere apud vos, si Dominus permiserit.
${}^{8}$~Permanebo autem Ephesi usque ad Pentecosten.
${}^{9}$~Ostium enim mihi apertum est magnum, et evidens~: et adversarii multi.


${}^{10}$~Si autem venerit Timotheus, videte ut sine timore sit apud vos~: opus enim Domini operatur, sicut et ego.
${}^{11}$~Ne quis ergo illum spernat~: deducite autem illum in pace, ut veniat ad me~: exspecto enim illum cum fratribus.
${}^{12}$~De Apollo autem fratre vobis notum facio, quoniam multum rogavi eum ut veniret ad vos cum fratribus~: et utique non fuit voluntas ut nunc veniret~: veniet autem, cum ei vacuum fuerit.
${}^{13}$~Vigilate, state in fide, viriliter agite, et confortamini.
${}^{14}$~Omnia vestra in caritate fiant.
${}^{15}$~Obsecro autem vos fratres, nostis domum Stephan\ae , et Fortunati, et Achaici~: quoniam sunt primiti\ae\ Achai\ae , et in ministerium sanctorum ordinaverunt seipsos~:
${}^{16}$~ut et vos subditi sitis ejusmodi, et omni cooperanti, et laboranti.
${}^{17}$~Gaudeo autem in pr\ae sentia Stephan\ae , et Fortunati, et Achaici~: quoniam id, quod vobis deerat, ipsi suppleverunt~:
${}^{18}$~refecerunt enim et meum spiritum, et vestrum. Cognoscite ergo qui hujusmodi sunt.


${}^{19}$~Salutant vos ecclesi\ae\ Asi\ae . Salutant vos in Domino multum, Aquila et Priscilla cum domestica sua ecclesia~: apud quos et hospitor.
${}^{20}$~Salutant vos omnes fratres. Salutate invicem in osculo sancto.
${}^{21}$~Salutatio, mea manu Pauli.
${}^{22}$~Si quis non amat Dominum nostrum Jesum Christum, sit anathema, Maran Atha.
${}^{23}$~Gratia Domini nostri Jesu Christi vobiscum.
${}^{24}$~Caritas mea cum omnibus vobis in Christo Jesu. Amen.
