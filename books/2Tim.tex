\bbook{Epistola B. Pauli Apostoli ad Timotheum Secunda}
{ad Timotheum II}{images/genese_heading}


\bchapter
\lettrine[lines=6,image=true,loversize=0.05,lraise=-0.03]{P}{}aulus Apostolus Jesu Christi per voluntatem Dei, secundum promissionem vit\ae , qu\ae\ est in Christo Jesu,
${}^{2}$~Timotheo carissimo filio~: gratia, misericordia, pax a Deo Patre, et Christo Jesu Domino nostro.


${}^{3}$~Gratias ago Deo, cui servio a progenitoribus in conscientia pura, quod sine intermissione habeam tui memoriam in orationibus meis, nocte ac die
${}^{4}$~desiderans te videre, memor lacrimarum tuarum, ut gaudio implear,
${}^{5}$~recordationem accipiens ejus fidei, qu\ae\ est in te non ficta, qu\ae\ et habitavit primum in avia tua Loide, et matre tua Eunice, certus sum autem quod et in te.


${}^{6}$~Propter quam causam admoneo te ut resuscites gratiam Dei, qu\ae\ est in te per impositionem manuum mearum.
${}^{7}$~Non enim dedit nobis Deus spiritum timoris~: sed virtutis, et dilectionis, et sobrietatis.
${}^{8}$~Noli itaque erubescere testimonium Domini nostri, neque me vinctum ejus~: sed collabora Evangelio secundum virtutem Dei~:
${}^{9}$~qui nos liberavit, et vocavit vocatione sua sancta, non secundum opera nostra, sed secundum propositum suum, et gratiam, qu\ae\ data est nobis in Christo Jesu ante tempora s\ae cularia.
${}^{10}$~Manifestata est autem nunc per illuminationem Salvatoris nostri Jesu Christi, qui destruxit quidem mortem, illuminavit autem vitam, et incorruptionem per Evangelium~:
${}^{11}$~in quo positus sum ego pr\ae dicator, et Apostolus, et magister gentium.
${}^{12}$~Ob quam causam etiam h\ae c patior, sed non confundor. Scio enim cui credidi, et certus sum quia potens est depositum meum servare in illum diem.
${}^{13}$~Formam habe sanorum verborum, qu\ae\ a me audisti in fide, et in dilectione in Christo Jesu.
${}^{14}$~Bonum depositum custodi per Spiritum Sanctum, qui habitat in nobis.
${}^{15}$~Scis hoc, quod aversi sunt a me omnes, qui in Asia sunt, ex quibus est Phigellus, et Hermogenes.
${}^{16}$~Det misericordiam Dominus Onesiphori domui~: quia s\ae pe me refrigeravit, et catenam meam non erubuit~:
${}^{17}$~sed cum Romam venisset, sollicite me qu\ae sivit, et invenit.
${}^{18}$~Det illi Dominus invenire misericordiam a Domino in illa die. Et quanta Ephesi ministravit mihi, tu melius nosti.

\bchapter
\lettrine[lines=3,image=true,loversize=0.05,lraise=-0.03]{T}{}u ergo fili mi, confortare in gratia, qu\ae\ est in Christo Jesu~:
${}^{2}$~et qu\ae\ audisti a me per multos testes, h\ae c commenda fidelibus hominibus, qui idonei erant et alios docere.
${}^{3}$~Labora sicut bonus miles Christi Jesu.
${}^{4}$~Nemo militans Deo implicat se negotiis s\ae cularibus~: ut ei placeat, cui se probavit.
${}^{5}$~Nam et qui certat in agone, non coronatur nisi legitime certaverit.
${}^{6}$~Laborantem agricolam oportet primum de fructibus percipere.
${}^{7}$~Intellige qu\ae\ dico~: dabit enim tibi Dominus in omnibus intellectum.
${}^{8}$~Memor esto Dominum Jesum Christum resurrexisse a mortuis ex semine David, secundum Evangelium meum,
${}^{9}$~in quo laboro usque ad vincula, quasi male operans~: sed verbum Dei non est alligatum.
${}^{10}$~Ideo omnia sustineo propter electos, ut et ipsi salutem consequantur, qu\ae\ est in Christo Jesu, cum gloria c\ae lesti.
${}^{11}$~Fidelis sermo~: nam si commortui sumus, et convivemus~:
${}^{12}$~si sustinebimus, et conregnabimus~: si negaverimus, et ille negabit nos~:
${}^{13}$~si non credimus, ille fidelis permanet, negare seipsum non potest.


${}^{14}$~H\ae c commone, testificans coram Domino. Noli contendere verbis~: ad nihil enim utile est, nisi ad subversionem audientium.
${}^{15}$~Sollicite cura teipsum probabilem exhibere Deo, operarium inconfusibilem, recte tractantem verbum veritatis.
${}^{16}$~Profana autem et vaniloquia devita~: multum enim proficiunt ad impietatem~:
${}^{17}$~et sermo eorum ut cancer serpit~: ex quibus est Hymen\ae us et Philetus,
${}^{18}$~qui a veritate exciderunt, dicentes resurrectionem esse jam factam, et subverterunt quorumdam fidem.
${}^{19}$~Sed firmum fundamentum Dei stat, habens signaculum hoc~: cognovit Dominus qui sunt ejus, et discedat ab iniquitate omnis qui nominat nomen Domini.
${}^{20}$~In magna autem domo non solum sunt vasa aurea, et argentea, sed et lignea, et fictilia~: et qu\ae dam quidem in honorem, qu\ae dam autem in contumeliam.
${}^{21}$~Si quis ergo emundaverit se ab istis, erit vas in honorem sanctificatum, et utile Domino ad omne opus bonum paratum.
${}^{22}$~Juvenilia autem desideria fuge, sectare vero justitiam, fidem, spem, caritatem, et pacem cum iis qui invocant Dominum de corde puro.
${}^{23}$~Stultas autem et sine disciplina qu\ae stiones devita~: sciens quia generant lites.
${}^{24}$~Servum autem Domini non oportet litigare~: sed mansuetum esse ad omnes, docibilem, patientem,
${}^{25}$~cum modestia corripientem eos qui resistunt veritati, nequando Deus det illis pœnitentiam ad cognoscendam veritatem,
${}^{26}$~et resipiscant a diaboli laqueis, a quo captivi tenentur ad ipsius voluntatem.

\bchapter
\lettrine[lines=3,image=true,loversize=0.05,lraise=-0.03]{H}{}oc autem scito, quod in novissimis diebus instabunt tempora periculosa~:
${}^{2}$~erunt homines seipsos amantes, cupidi, elati, superbi, blasphemi, parentibus non obedientes, ingrati, scelesti,
${}^{3}$~sine affectione, sine pace, criminatores, incontinentes, immites, sine benignitate,
${}^{4}$~proditores, protervi, tumidi, et voluptatum amatores magis quam Dei~:
${}^{5}$~habentes speciem quidem pietatis, virtutem autem ejus abnegantes. Et hos devita~:
${}^{6}$~ex his enim sunt qui penetrant domos, et captivas ducunt mulierculas oneratas peccatis, qu\ae\ ducuntur variis desideriis~:
${}^{7}$~semper discentes, et numquam ad scientiam veritatis pervenientes.
${}^{8}$~Quemadmodum autem Jannes et Mambres restiterunt Moysi~: ita et hi resistunt veritati, homines corrupti mente, reprobi circa fidem~;
${}^{9}$~sed ultra non proficient~: insipientia enim eorum manifesta erit omnibus, sicut et illorum fuit.
${}^{10}$~Tu autem assecutus es meam doctrinam, institutionem, propositum, fidem, longanimitatem, dilectionem, patientiam,
${}^{11}$~persecutiones, passiones~: qualia mihi facta sunt Antiochi\ae , Iconii, et Lystris~: quales persecutiones sustinui, et ex omnibus eripuit me Dominus.
${}^{12}$~Et omnes, qui pie volunt vivere in Christo Jesu, persecutionem patientur.
${}^{13}$~Mali autem homines et seductores proficient in pejus, errantes, et in errorem mittentes.
${}^{14}$~Tu vero permane in iis qu\ae\ didicisti, et credita sunt tibi~: sciens a quo didiceris~:
${}^{15}$~et quia ab infantia sacras litteras nosti, qu\ae\ te possunt instruere ad salutem, per fidem qu\ae\ est in Christo Jesu.
${}^{16}$~Omnis Scriptura divinitus inspirata utilis est ad docendum, ad arguendum, ad corripiendum, et erudiendum in justitia~:
${}^{17}$~ut perfectus sit homo Dei, ad omne opus bonum instructus.

\bchapter
\lettrine[lines=3,image=true,loversize=0.05,lraise=-0.03]{T}{}estificor coram Deo, et Jesu Christo, qui judicaturus est vivos et mortuos, per adventum ipsius, et regnum ejus~:
${}^{2}$~pr\ae dica verbum, insta opportune, importune~: argue, obsecra, increpa in omni patientia, et doctrina.
${}^{3}$~Erit enim tempus, cum sanam doctrinam non sustinebunt, sed ad sua desideria coacervabunt sibi magistros, prurientes auribus,
${}^{4}$~et a veritate quidem auditum avertent, ad fabulas autem convertentur.
${}^{5}$~Tu vero vigila, in omnibus labora, opus fac evangelist\ae , ministerium tuum imple. Sobrius esto.
${}^{6}$~Ego enim jam delibor, et tempus resolutionis me\ae\ instat.
${}^{7}$~Bonum certamen certavi, cursum consummavi, fidem servavi.
${}^{8}$~In reliquo reposita est mihi corona justiti\ae , quam reddet mihi Dominus in illa die, justus judex~: non solum autem mihi, sed et iis, qui diligunt adventum ejus.

 Festina ad me venire cito.
${}^{9}$~Demas enim me reliquit, diligens hoc s\ae culum, et abiit Thessalonicam~:
${}^{10}$~Crescens in Galatiam, Titus in Dalmatiam.
${}^{11}$~Lucas est mecum solus. Marcum assume, et adduc tecum~: est enim mihi utilis in ministerium.
${}^{12}$~Tychicum autem misi Ephesum.
${}^{13}$~Penulam, quam reliqui Troade apud Carpum, veniens affer tecum, et libros, maxime autem membranas.
${}^{14}$~Alexander \ae rarius multa mala mihi ostendit~: reddet illi Dominus secundum opera ejus~:
${}^{15}$~quem et tu devita~: valde enim restitit verbis nostris.
${}^{16}$~In prima mea defensione nemo mihi affuit, sed omnes me dereliquerunt~: non illis imputetur.
${}^{17}$~Dominus autem mihi astitit, et confortavit me, ut per me pr\ae dicatio impleatur, et audiant omnes gentes~: et liberatus sum de ore leonis.
${}^{18}$~Liberavit me Dominus ab omni opere malo~: et salvum faciet in regnum suum c\ae leste, cui gloria in s\ae cula s\ae culorum. Amen.
${}^{19}$~Saluta Priscam, et Aquilam, et Onesiphori domum.
${}^{20}$~Erastus remansit Corinthi. Trophimum autem reliqui infirmum Mileti.
${}^{21}$~Festina ante hiemem venire. Salutant te Eubulus, et Pudens, et Linus, et Claudia, et fratres omnes.
${}^{22}$~Dominus Jesus Christus cum spiritu tuo. Gratia vobiscum. Amen.
