\bbook{Epistola B. Pauli Apostoli ad Romanos}
{ad Romanos}{images/genese_heading}
\addcontentsline{toc}{subsection}{Ad Romanos}

\Needspace{2.5\baselineskip}\versal{1}~\lettrine[lines=10,image=true,loversize=0.05,lraise=-0.03]{P}{}aulus, servus Jesu Christi, vocatus Apostolus, segregatus in Evangelium Dei,
${}^{2}$~quod ante promiserat per prophetas suos in Scripturis sanctis
${}^{3}$~de Filio suo, qui factus est ei ex semine David secundum carnem,
${}^{4}$~qui pr\ae destinatus est Filius Dei in virtute secundum spiritum sanctificationis ex resurrectione mortuorum Jesu Christi Domini nostri~:
${}^{5}$~per quem accepimus gratiam, et apostolatum ad obediendum fidei in omnibus gentibus pro nomine ejus,
${}^{6}$~in quibus estis et vos vocati Jesu Christi~:
${}^{7}$~omnibus qui sunt Rom\ae , dilectis Dei, vocatis sanctis. Gratia vobis, et pax a Deo Patre nostro, et Domino Jesu Christo.


${}^{8}$~Primum quidem gratias ago Deo meo per Jesum Christum pro omnibus vobis~: quia fides vestra annuntiatur in universo mundo.
${}^{9}$~Testis enim mihi est Deus, cui servio in spiritu meo in Evangelio Filii ejus, quod sine intermissione memoriam vestri facio
${}^{10}$~semper in orationibus meis~: obsecrans, si quomodo tandem aliquando prosperum iter habeam in voluntate Dei veniendi ad vos.
${}^{11}$~Desidero enim videre vos, ut aliquid impertiar vobis grati\ae\ spiritualis ad confirmandos vos~:
${}^{12}$~id est, simul consolari in vobis per eam qu\ae\ invicem est, fidem vestram atque meam.
${}^{13}$~Nolo autem vos ignorare fratres~: quia s\ae pe proposui venire ad vos (et prohibitus sum usque adhuc) ut aliquem fructum habeam et in vobis, sicut et in ceteris gentibus.
${}^{14}$~Gr\ae cis ac barbaris, sapientibus, et insipientibus debitor sum~:
${}^{15}$~ita (quod in me) promptum est et vobis, qui Rom\ae\ estis, evangelizare.


${}^{16}$~Non enim erubesco Evangelium. Virtus enim Dei est in salutem omni credenti, Jud\ae o primum, et Gr\ae co.
${}^{17}$~Justitia enim Dei in eo revelatur ex fide in fidem~: sicut scriptum est~: Justus autem ex fide vivit.


${}^{18}$~Revelatur enim ira Dei de c\ae lo super omnem impietatem, et injustitiam hominum eorum, qui veritatem Dei in injustitia detinent~:
${}^{19}$~quia quod notum est Dei, manifestum est in illis. Deus enim illis manifestavit.
${}^{20}$~Invisibilia enim ipsius, a creatura mundi, per ea qu\ae\ facta sunt, intellecta, conspiciuntur~: sempiterna quoque ejus virtus, et divinitas~: ita ut sint inexcusabiles.
${}^{21}$~Quia cum cognovissent Deum, non sicut Deum glorificaverunt, aut gratias egerunt~: sed evanuerunt in cogitationibus suis, et obscuratum est insipiens cor eorum~:
${}^{22}$~dicentes enim se esse sapientes, stulti facti sunt.
${}^{23}$~Et mutaverunt gloriam incorruptibilis Dei in similitudinem imaginis corruptibilis hominis, et volucrum, et quadrupedum, et serpentium.


${}^{24}$~Propter quod tradidit illos Deus in desideria cordis eorum, in immunditiam, ut contumeliis afficiant corpora sua in semetipsis~:
${}^{25}$~qui commutaverunt veritatem Dei in mendacium~: et coluerunt, et servierunt creatur\ae\ potius quam Creatori, qui est benedictus in s\ae cula. Amen.
${}^{26}$~Propterea tradidit illos Deus in passiones ignomini\ae~: nam femin\ae\ eorum immutaverunt naturalem usum in eum usum qui est contra naturam.
${}^{27}$~Similiter autem et masculi, relicto naturali usu femin\ae , exarserunt in desideriis suis in invicem, masculi in masculos turpitudinem operantes, et mercedem, quam oportuit, erroris sui in semetipsis recipientes.
${}^{28}$~Et sicut non probaverunt Deum habere in notitia, tradidit illos Deus in reprobum sensum, ut faciant ea qu\ae\ non conveniunt,
${}^{29}$~repletos omni iniquitate, malitia, fornicatione, avaritia, nequitia, plenos invidia, homicidio, contentione, dolo, malignitate~: susurrones,
${}^{30}$~detractores, Deo odibiles, contumeliosos, superbos, elatos, inventores malorum, parentibus non obedientes,
${}^{31}$~insipientes, incompositos, sine affectione, absque fœdere, sine misericordia.
${}^{32}$~Qui cum justitiam Dei cognovissent, non intellexerunt quoniam qui talia agunt, digni sunt morte~: et non solum qui ea faciunt, sed etiam qui consentiunt facientibus.
\Needspace{2.5\baselineskip}\versal{2}~\lettrine[lines=10,image=true,loversize=0.05,lraise=-0.03]{P}{}ropter quod inexcusabilis es, o homo omnis qui judicas. In quo enim judicas alterum, teipsum condemnas~: eadem enim agis qu\ae\ judicas.
${}^{2}$~Scimus enim quoniam judicium Dei est secundum veritatem in eos qui talia agunt.
${}^{3}$~Existimas autem hoc, o homo, qui judicas eos qui talia agunt, et facis ea, quia tu effugies judicium Dei~?
${}^{4}$~an divitias bonitatis ejus, et patienti\ae , et longanimitatis contemnis~? ignoras quoniam benignitas Dei ad pœnitentiam te adducit~?
${}^{5}$~Secundum autem duritiam tuam, et impœnitens cor, thesaurizas tibi iram in die ir\ae , et revelationis justi judicii Dei,
${}^{6}$~qui reddet unicuique secundum opera ejus~:
${}^{7}$~iis quidem qui secundum patientiam boni operis, gloriam, et honorem, et incorruptionem qu\ae runt, vitam \ae ternam~:
${}^{8}$~iis autem qui sunt ex contentione, et qui non acquiescunt veritati, credunt autem iniquitati, ira et indignatio.
${}^{9}$~Tribulatio et angustia in omnem animam hominis operantis malum, Jud\ae i primum, et Gr\ae ci~:
${}^{10}$~gloria autem, et honor, et pax omni operanti bonum, Jud\ae o primum, et Gr\ae co~:
${}^{11}$~non enim est acceptio personarum apud Deum.


${}^{12}$~Quicumque enim sine lege peccaverunt, sine lege peribunt~: et quicumque in lege peccaverunt, per legem judicabuntur.
${}^{13}$~Non enim auditores legis justi sunt apud Deum, sed factores legis justificabuntur.
${}^{14}$~Cum autem gentes, qu\ae\ legem non habent, naturaliter ea, qu\ae\ legis sunt, faciunt, ejusmodi legem non habentes, ipsi sibi sunt lex~:
${}^{15}$~qui ostendunt opus legis scriptum in cordibus suis, testimonium reddente illis conscientia ipsorum, et inter se invicem cogitationibus accusantibus, aut etiam defendentibus,
${}^{16}$~in die, cum judicabit Deus occulta hominum, secundum Evangelium meum per Jesum Christum.


${}^{17}$~Si autem tu Jud\ae us cognominaris, et requiescis in lege, et gloriaris in Deo,
${}^{18}$~et nosti voluntatem ejus, et probas utiliora, instructus per legem,
${}^{19}$~confidis teipsum esse ducem c\ae corum, lumen eorum qui in tenebris sunt,
${}^{20}$~eruditorem insipientium, magistrum infantium, habentem formam scienti\ae , et veritatis in lege.
${}^{21}$~Qui ergo alium doces, teipsum non doces~: qui pr\ae dicas non furandum, furaris~:
${}^{22}$~qui dicis non mœchandum, mœcharis~: qui abominaris idola, sacrilegium facis~:
${}^{23}$~qui in lege gloriaris, per pr\ae varicationem legis Deum inhonoras.
${}^{24}$~(Nomen enim Dei per vos blasphematur inter gentes, sicut scriptum est.)


${}^{25}$~Circumcisio quidem prodest, si legem observes~: si autem pr\ae varicator legis sis, circumcisio tua pr\ae putium facta est.
${}^{26}$~Si igitur pr\ae putium justitias legis custodiat, nonne pr\ae putium illius in circumcisionem reputabitur~?
${}^{27}$~et judicabit id quod ex natura est pr\ae putium, legem consummans, te, qui per litteram et circumcisionem pr\ae varicator legis es~?
${}^{28}$~Non enim qui in manifesto, Jud\ae us est~: neque qu\ae\ in manifesto, in carne, est circumcisio~:
${}^{29}$~sed qui in abscondito, Jud\ae us est~: et circumcisio cordis in spiritu, non littera~: cujus laus non ex hominibus, sed ex Deo est.
\Needspace{2.5\baselineskip}\versal{3}~\lettrine[lines=10,image=true,loversize=0.05,lraise=-0.03]{Q}{}uid ergo amplius Jud\ae o est~? aut qu\ae\ utilitas circumcisionis~?
${}^{2}$~Multum per omnem modum. Primum quidem quia credita sunt illis eloquia Dei.
${}^{3}$~Quid enim si quidam illorum non crediderunt~? numquid incredulitas illorum fidem Dei evacuabit~? Absit.
${}^{4}$~Est autem Deus verax~: omnis autem homo mendax, sicut scriptum est~: \begin{flushleft}\begin{verse}Ut justificeris in sermonibus tuis~:\\ et vincas cum judicaris.\end{verse}\end{flushleft}


${}^{5}$~Si autem iniquitas nostra justitiam Dei commendat, quid dicemus~? Numquid iniquus est Deus, qui infert iram~?
${}^{6}$~secundum hominem dico. Absit. Alioquin quomodo judicabit Deus hunc mundum~?
${}^{7}$~Si enim veritas Dei in meo mendacio abundavit in gloriam ipsius~: quid adhuc et ego tamquam peccator judicor~?
${}^{8}$~et non (sicut blasphemamur, et sicut aiunt quidam nos dicere) faciamus mala ut veniant bona~: quorum damnatio justa est.


${}^{9}$~Quid ergo~? pr\ae cellimus eos~? Nequaquam. Causati enim sumus Jud\ae os et Gr\ae cos omnes sub peccato esse,
${}^{10}$~sicut scriptum est~: \begin{flushleft}\begin{verse}Quia non est justus quisquam~:\\
${}^{11}$~non est intelligens, non est requirens Deum.\\
${}^{12}$~Omnes declinaverunt, simul inutiles facti sunt~:\\ non est qui faciat bonum, non est usque ad unum.\\
${}^{13}$~Sepulchrum patens est guttur eorum,\\ linguis suis dolose agebant~:\\ venenum aspidum sub labiis eorum~:\\
${}^{14}$~quorum os maledictione, et amaritudine plenum est~:\\
${}^{15}$~veloces pedes eorum ad effundendum sanguinem~:\\
${}^{16}$~contritio et infelicitas in viis eorum~:\\
${}^{17}$~et viam pacis non cognoverunt~:\\
${}^{18}$~non est timor Dei ante oculos eorum.\end{verse}\end{flushleft}


${}^{19}$~Scimus autem quoniam qu\ae cumque lex loquitur, iis, qui in lege sunt, loquitur~: ut omne os obstruatur, et subditus fiat omnis mundus Deo~:
${}^{20}$~quia ex operibus legis non justificabitur omnis caro coram illo. Per legem enim cognitio peccati.


${}^{21}$~Nunc autem sine lege justitia Dei manifestata est~: testificata a lege et prophetis.
${}^{22}$~Justitia autem Dei per fidem Jesu Christi in omnes et super omnes qui credunt in eum~: non enim est distinctio~:
${}^{23}$~omnes enim peccaverunt, et egent gloria Dei.
${}^{24}$~Justificati gratis per gratiam ipsius, per redemptionem qu\ae\ est in Christo Jesu,
${}^{25}$~quem proposuit Deus propitiationem per fidem in sanguine ipsius, ad ostensionem justiti\ae\ su\ae\ propter remissionem pr\ae cedentium delictorum
${}^{26}$~in sustentatione Dei, ad ostensionem justiti\ae\ ejus in hoc tempore~: ut sit ipse justus, et justificans eum, qui est ex fide Jesu Christi.


${}^{27}$~Ubi est ergo gloriatio tua~? Exclusa est. Per quam legem~? Factorum~? Non~: sed per legem fidei.
${}^{28}$~Arbitramur enim justificari hominem per fidem sine operibus legis.
${}^{29}$~An Jud\ae orum Deus tantum~? nonne et gentium~? Immo et gentium~:
${}^{30}$~quoniam quidem unus est Deus, qui justificat circumcisionem ex fide, et pr\ae putium per fidem.
${}^{31}$~Legem ergo destruimus per fidem~? Absit~: sed legem statuimus.
\Needspace{2.5\baselineskip}\versal{4}~\lettrine[lines=10,image=true,loversize=0.05,lraise=-0.03]{Q}{}uid ergo dicemus invenisse Abraham patrem nostrum secundum carnem~?
${}^{2}$~Si enim Abraham ex operibus justificatus est, habet gloriam, sed non apud Deum.
${}^{3}$~Quid enim dicit Scriptura~? Credidit Abraham Deo, et reputatum est illi ad justitiam.
${}^{4}$~Ei autem qui operatur, merces non imputatur secundum gratiam, sed secundum debitum.
${}^{5}$~Ei vero qui non operatur, credenti autem in eum, qui justificat impium, reputatur fides ejus ad justitiam secundum propositum grati\ae\ Dei.
${}^{6}$~Sicut et David dicit beatitudinem hominis, cui Deus accepto fert justitiam sine operibus~:
\begin{flushleft}\begin{verse}${}^{7}$~Beati, quorum remiss\ae\ sunt iniquitates,\\ et quorum tecta sunt peccata.\\
${}^{8}$~Beatus vir, cui non imputavit Dominus peccatum.\end{verse}\end{flushleft}


${}^{9}$~Beatitudo ergo h\ae c in circumcisione tantum manet, an etiam in pr\ae putio~? Dicimus enim quia reputata est Abrah\ae\ fides ad justitiam.
${}^{10}$~Quomodo ergo reputata est~? in circumcisione, an in pr\ae putio~? Non in circumcisione, sed in pr\ae putio.
${}^{11}$~Et signum accepit circumcisionis, signaculum justiti\ae\ fidei, qu\ae\ est in pr\ae putio~: ut sit pater omnium credentium per pr\ae putium, ut reputetur et illis ad justitiam~:
${}^{12}$~et sit pater circumcisionis non iis tantum, qui sunt ex circumcisione, sed et iis qui sectantur vestigia fidei, qu\ae\ est in pr\ae putio patris nostri Abrah\ae .


${}^{13}$~Non enim per legem promissio Abrah\ae , aut semini ejus ut h\ae res esset mundi~: sed per justitiam fidei.
${}^{14}$~Si enim qui ex lege, h\ae redes sunt~: exinanita est fides, abolita est promissio.
${}^{15}$~Lex enim iram operatur. Ubi enim non est lex, nec pr\ae varicatio.
${}^{16}$~Ideo ex fide, ut secundum gratiam firma sit promissio omni semini, non ei qui ex lege est solum, sed et ei qui ex fide est Abrah\ae , qui pater est omnium nostrum
${}^{17}$~(sicut scriptum est~: Quia patrem multarum gentium posui te) ante Deum, cui credidit, qui vivificat mortuos, et vocat ea qu\ae\ non sunt, tamquam ea qu\ae\ sunt~:
${}^{18}$~qui contra spem in spem credidit, ut fieret pater multarum gentium secundum quod dictum est ei~: Sic erit semen tuum.
${}^{19}$~Et non infirmatus est fide, nec consideravit corpus suum emortuum, cum jam fere centum esset annorum, et emortuam vulvam Sar\ae .
${}^{20}$~In repromissione etiam Dei non h\ae sitavit diffidentia, sed confortatus est fide, dans gloriam Deo~:
${}^{21}$~plenissime sciens, quia qu\ae cumque promisit, potens est et facere.
${}^{22}$~Ideo et reputatum est illi ad justitiam.
${}^{23}$~Non est autem scriptum tantum propter ipsum quia reputatum est illi ad justitiam~:
${}^{24}$~sed et propter nos, quibus reputabitur credentibus in eum, qui suscitavit Jesum Christum Dominum nostrum a mortuis,
${}^{25}$~qui traditus est propter delicta nostra, et resurrexit propter justificationem nostram.
\Needspace{2.5\baselineskip}\versal{5}~\lettrine[lines=10,image=true,loversize=0.05,lraise=-0.03]{J}{}ustificati ergo ex fide, pacem habeamus ad Deum per Dominum nostrum Jesum Christum~:
${}^{2}$~per quem et habemus accessum per fidem in gratiam istam, in qua stamus, et gloriamur in spe glori\ae\ filiorum Dei.
${}^{3}$~Non solum autem, sed et gloriamur in tribulationibus~: scientes quod tribulatio patientiam operatur~:
${}^{4}$~patientia autem probationem, probatio vero spem,
${}^{5}$~spes autem non confundit~: quia caritas Dei diffusa est in cordibus nostris per Spiritum Sanctum, qui datus est nobis.
${}^{6}$~Ut quid enim Christus, cum adhuc infirmi essemus, secundum tempus, pro impiis mortuus est~?
${}^{7}$~vix enim pro justo quis moritur~: nam pro bono forsitan quis audeat mori.
${}^{8}$~Commendat autem caritatem suam Deus in nobis~: quoniam cum adhuc peccatores essemus, secundum tempus,
${}^{9}$~Christus pro nobis mortuus est~: multo igitur magis nunc justificati in sanguine ipsius, salvi erimus ab ira per ipsum.
${}^{10}$~Si enim cum inimici essemus, reconciliati sumus Deo per mortem filii ejus~: multo magis reconciliati, salvi erimus in vita ipsius.
${}^{11}$~Non solum autem~: sed et gloriamur in Deo per Dominum nostrum Jesum Christum, per quem nunc reconciliationem accepimus.


${}^{12}$~Propterea sicut per unum hominem peccatum in hunc mundum intravit, et per peccatum mors, et ita in omnes homines mors pertransiit, in quo omnes peccaverunt.
${}^{13}$~Usque ad legem enim peccatum erat in mundo~: peccatum autem non imputabatur, cum lex non esset.
${}^{14}$~Sed regnavit mors ab Adam usque ad Moysen etiam in eos qui non peccaverunt in similitudinem pr\ae varicationis Ad\ae , qui est forma futuri.
${}^{15}$~Sed non sicut delictum, ita et donum~: si enim unius delicto multi mortui sunt~: multo magis gratia Dei et donum in gratia unius hominis Jesu Christi in plures abundavit.
${}^{16}$~Et non sicut per unum peccatum, ita et donum. Nam judicium quidem ex uno in condemnationem~: gratia autem ex multis delictis in justificationem.
${}^{17}$~Si enim unius delicto mors regnavit per unum~: multo magis abundantiam grati\ae , et donationis, et justiti\ae\ accipientes, in vita regnabunt per unum Jesum Christum.
${}^{18}$~Igitur sicut per unius delictum in omnes homines in condemnationem~: sic et per unius justitiam in omnes homines in justificationem vit\ae .
${}^{19}$~Sicut enim per inobedientiam unius hominis, peccatores constituti sunt multi~: ita et per unius obeditionem, justi constituentur multi.
${}^{20}$~Lex autem subintravit ut abundaret delictum. Ubi autem abundavit delictum, superabundavit gratia~:
${}^{21}$~ut sicut regnavit peccatum in mortem~: ita et gratia regnet per justitiam in vitam \ae ternam, per Jesum Christum Dominum nostrum.
\Needspace{2.5\baselineskip}\versal{6}~\lettrine[lines=10,image=true,loversize=0.05,lraise=-0.03]{Q}{}uid ergo dicemus~? permanebimus in peccato ut gratia abundet~?
${}^{2}$~Absit. Qui enim mortui sumus peccato, quomodo adhuc vivemus in illo~?
${}^{3}$~an ignoratis quia quicumque baptizati sumus in Christo Jesu, in morte ipsius baptizati sumus~?
${}^{4}$~Consepulti enim sumus cum illo per baptismum in mortem~: ut quomodo Christus surrexit a mortuis per gloriam Patris, ita et nos in novitate vit\ae\ ambulemus.
${}^{5}$~Si enim complantati facti sumus similitudini mortis ejus~: simul et resurrectionis erimus.
${}^{6}$~Hoc scientes, quia vetus homo noster simul crucifixus est, ut destruatur corpus peccati, et ultra non serviamus peccato.
${}^{7}$~Qui enim mortuus est, justificatus est a peccato.
${}^{8}$~Si autem mortui sumus cum Christo, credimus quia simul etiam vivemus cum Christo,
${}^{9}$~scientes quod Christus resurgens ex mortuis jam non moritur~: mors illi ultra non dominabitur.
${}^{10}$~Quod enim mortuus est peccato, mortuus est semel~: quod autem vivit, vivit Deo.
${}^{11}$~Ita et vos existimate vos mortuos quidem esse peccato, viventes autem Deo, in Christo Jesu Domino nostro.


${}^{12}$~Non ergo regnet peccatum in vestro mortali corpore ut obediatis concupiscentiis ejus.
${}^{13}$~Sed neque exhibeatis membra vestra arma iniquitatis peccato~: sed exhibete vos Deo, tamquam ex mortuis viventes~: et membra vestra arma justiti\ae\ Deo.
${}^{14}$~Peccatum enim vobis non dominabitur~: non enim sub lege estis, sed sub gratia.
${}^{15}$~Quid ergo~? peccabimus, quoniam non sumus sub lege, sed sub gratia~? Absit.
${}^{16}$~Nescitis quoniam cui exhibetis vos servos ad obediendum, servi estis ejus, cui obeditis, sive peccati ad mortem, sive obeditionis ad justitiam~?
${}^{17}$~Gratias autem Deo quod fuistis servi peccati, obedistis autem ex corde in eam formam doctrin\ae , in quam traditi estis.
${}^{18}$~Liberati autem a peccato, servi facti estis justiti\ae .
${}^{19}$~Humanum dico, propter infirmitatem carnis vestr\ae~: sicut enim exhibuistis membra vestra servire immunditi\ae , et iniquitati ad iniquitatem, ita nunc exhibete membra vestra servire justiti\ae\ in sanctificationem.
${}^{20}$~Cum enim servi essetis peccati, liberi fuistis justiti\ae .
${}^{21}$~Quem ergo fructum habuistis tunc in illis, in quibus nunc erubescitis~? nam finis illorum mors est.
${}^{22}$~Nunc vero liberati a peccato, servi autem facti Deo, habetis fructum vestrum in sanctificationem, finem vero vitam \ae ternam.
${}^{23}$~Stipendia enim peccati, mors. Gratia autem Dei, vita \ae terna, in Christo Jesu Domino nostro.
\Needspace{2.5\baselineskip}\versal{7}~\lettrine[lines=10,image=true,loversize=0.05,lraise=-0.03]{A}{}n ignoratis, fratres (scientibus enim legem loquor), quia lex in homine dominatur quanto tempore vivit~?
${}^{2}$~Nam qu\ae\ sub viro est mulier, vivente viro, alligata est legi~: si autem mortuus fuerit vir ejus, soluta est a lege viri.
${}^{3}$~Igitur, vivente viro, vocabitur adultera si fuerit cum alio viro~: si autem mortuus fuerit vir ejus, liberata est a lege viri, ut non sit adultera si fuerit cum alio viro.
${}^{4}$~Itaque fratres mei, et vos mortificati estis legi per corpus Christi~: ut sitis alterius, qui ex mortuis resurrexit, ut fructificemus Deo.
${}^{5}$~Cum enim essemus in carne, passiones peccatorum, qu\ae\ per legem erant, operabantur in membris nostris, ut fructificarent morti.
${}^{6}$~Nunc autem soluti sumus a lege mortis, in qua detinebamur, ita ut serviamus in novitate spiritus, et non in vetustate litter\ae .


${}^{7}$~Quid ergo dicemus~? lex peccatum est~? Absit. Sed peccatum non cognovi, nisi per legem~: nam concupiscentiam nesciebam, nisi lex diceret~: Non concupisces.
${}^{8}$~Occasione autem accepta, peccatum per mandatum operatum est in me omnem concupiscentiam. Sine lege enim peccatum mortuum erat.
${}^{9}$~Ego autem vivebam sine lege aliquando~: sed cum venisset mandatum, peccatum revixit.
${}^{10}$~Ego autem mortuus sum~: et inventum est mihi mandatum, quod erat ad vitam, hoc esse ad mortem.
${}^{11}$~Nam peccatum occasione accepta per mandatum, seduxit me, et per illud occidit.
${}^{12}$~Itaque lex quidem sancta, et mandatum sanctum, et justum, et bonum.


${}^{13}$~Quod ergo bonum est, mihi factum est mors~? Absit. Sed peccatum, ut appareat peccatum, per bonum operatum est mihi mortem~: ut fiat supra modum peccans peccatum per mandatum.
${}^{14}$~Scimus enim quia lex spiritualis est~: ego autem carnalis sum, venundatus sub peccato.
${}^{15}$~Quod enim operor, non intelligo~: non enim quod volo bonum, hoc ago~: sed quod odi malum, illud facio.
${}^{16}$~Si autem quod nolo, illud facio~: consentio legi, quoniam bona est.
${}^{17}$~Nunc autem jam non ego operor illud, sed quod habitat in me peccatum.
${}^{18}$~Scio enim quia non habitat in me, hoc est in carne mea, bonum. Nam velle, adjacet mihi~: perficere autem bonum, non invenio.
${}^{19}$~Non enim quod volo bonum, hoc facio~: sed quod nolo malum, hoc ago.
${}^{20}$~Si autem quod nolo, illud facio~: jam non ego operor illud, sed quod habitat in me, peccatum.
${}^{21}$~Invenio igitur legem, volenti mihi facere bonum, quoniam mihi malum adjacet~:
${}^{22}$~condelector enim legi Dei secundum interiorem hominem~:
${}^{23}$~video autem aliam legem in membris meis, repugnantem legi mentis me\ae , et captivantem me in lege peccati, qu\ae\ est in membris meis.
${}^{24}$~Infelix ego homo, quis me liberabit de corpore mortis hujus~?
${}^{25}$~gratia Dei per Jesum Christum Dominum nostrum. Igitur ego ipse mente servio legi Dei~: carne autem, legi peccati.
\Needspace{2.5\baselineskip}\versal{8}~\lettrine[lines=10,image=true,loversize=0.05,lraise=-0.03]{N}{}ihil ergo nunc damnationis est iis qui sunt in Christo Jesu~: qui non secundum carnem ambulant.
${}^{2}$~Lex enim spiritus vit\ae\ in Christo Jesu liberavit me a lege peccati et mortis.
${}^{3}$~Nam quod impossibile erat legi, in quo infirmabatur per carnem~: Deus Filium suum mittens in similitudinem carnis peccati et de peccato, damnavit peccatum in carne,
${}^{4}$~ut justificatio legis impleretur in nobis, qui non secundum carnem ambulamus, sed secundum spiritum.
${}^{5}$~Qui enim secundum carnem sunt, qu\ae\ carnis sunt, sapiunt~: qui vero secundum spiritum sunt, qu\ae\ sunt spiritus, sentiunt.
${}^{6}$~Nam prudentia carnis, mors est~: prudentia autem spiritus, vita et pax~:
${}^{7}$~quoniam sapientia carnis inimica est Deo~: legi enim Dei non est subjecta, nec enim potest.
${}^{8}$~Qui autem in carne sunt, Deo placere non possunt.
${}^{9}$~Vos autem in carne non estis, sed in spiritu~: si tamen Spiritus Dei habitat in vobis. Si quis autem Spiritum Christi non habet, hic non est ejus.
${}^{10}$~Si autem Christus in vobis est, corpus quidem mortuum est propter peccatum, spiritus vero vivit propter justificationem.
${}^{11}$~Quod si Spiritus ejus, qui suscitavit Jesum a mortuis, habitat in vobis~: qui suscitavit Jesum Christum a mortuis, vivificabit et mortalia corpora vestra, propter inhabitantem Spiritum ejus in vobis.
${}^{12}$~Ergo fratres, debitores sumus non carni, ut secundum carnem vivamus.
${}^{13}$~Si enim secundum carnem vixeritis, moriemini~: si autem spiritu facta carnis mortificaveritis, vivetis.


${}^{14}$~Quicumque enim Spiritu Dei aguntur, ii sunt filii Dei.
${}^{15}$~Non enim accepistis spiritum servitutis iterum in timore, sed accepistis spiritum adoptionis filiorum, in quo clamamus~: Abba (Pater).
${}^{16}$~Ipse enim Spiritus testimonium reddit spiritui nostro quod sumus filii Dei.
${}^{17}$~Si autem filii, et h\ae redes~: h\ae redes, quidem Dei, coh\ae redes autem Christi~: si tamen compatimur ut et conglorificemur.


${}^{18}$~Existimo enim quod non sunt condign\ae\ passiones hujus temporis ad futuram gloriam, qu\ae\ revelabitur in nobis.
${}^{19}$~Nam exspectatio creatur\ae\ revelationem filiorum Dei exspectat.
${}^{20}$~Vanitati enim creatura subjecta est non volens, sed propter eum, qui subjecit eam in spe~:
${}^{21}$~quia et ipsa creatura liberabitur a servitute corruptionis in libertatem glori\ae\ filiorum Dei.
${}^{22}$~Scimus enim quod omnis creatura ingemiscit, et parturit usque adhuc.
${}^{23}$~Non solum autem illa, sed et nos ipsi primitias spiritus habentes~: et ipsi intra nos gemimus adoptionem filiorum Dei exspectantes, redemptionem corporis nostri.
${}^{24}$~Spe enim salvi facti sumus. Spes autem, qu\ae\ videtur, non est spes~: nam quod videt quis, quid sperat~?
${}^{25}$~Si autem quod non videmus, speramus~: per patientiam exspectamus.


${}^{26}$~Similiter autem et Spiritus adjuvat infirmitatem nostram~: nam quid oremus, sicut oportet, nescimus~: sed ipse Spiritus postulat pro nobis gemitibus inenarrabilibus.
${}^{27}$~Qui autem scrutatur corda, scit quid desideret Spiritus~: quia secundum Deum postulat pro sanctis.


${}^{28}$~Scimus autem quoniam diligentibus Deum omnia cooperantur in bonum, iis qui secundum propositum vocati sunt sancti.
${}^{29}$~Nam quos pr\ae scivit, et pr\ae destinavit conformes fieri imaginis Filii sui, ut sit ipse primogenitus in multis fratribus.
${}^{30}$~Quos autem pr\ae destinavit, hos et vocavit~: et quos vocavit, hos et justificavit~: quos autem justificavit, illos et glorificavit.
${}^{31}$~Quid ergo dicemus ad h\ae c~? si Deus pro nobis, quis contra nos~?
${}^{32}$~Qui etiam proprio Filio suo non pepercit, sed pro nobis omnibus tradidit illum~: quomodo non etiam cum illo omnia nobis donavit~?
${}^{33}$~Quis accusabit adversus electos Dei~? Deus qui justificat,
${}^{34}$~quis est qui condemnet~? Christus Jesus, qui mortuus est, immo qui et resurrexit, qui est ad dexteram Dei, qui etiam interpellat pro nobis.
${}^{35}$~Quis ergo nos separabit a caritate Christi~? tribulatio~? an angustia~? an fames~? an nuditas~? an periculum~? an persecutio~? an gladius~?
${}^{36}$~(Sicut scriptum est~: \begin{flushleft}\begin{verse}Quia propter te mortificamur tota die~:\\ \ae stimati sumus sicut oves occisionis.)\end{verse}\end{flushleft}


${}^{37}$~Sed in his omnibus superamus propter eum qui dilexit nos.
${}^{38}$~Certus sum enim quia neque mors, neque vita, neque angeli, neque principatus, neque virtutes, neque instantia, neque futura, neque fortitudo,
${}^{39}$~neque altitudo, neque profundum, neque creatura alia poterit nos separare a caritate Dei, qu\ae\ est in Christo Jesu Domino nostro.
\Needspace{2.5\baselineskip}\versal{9}~\lettrine[lines=10,image=true,loversize=0.05,lraise=-0.03]{V}{}eritatem dico in Christo, non mentior~: testimonium mihi perhibente conscientia mea in Spiritu Sancto~:
${}^{2}$~quoniam tristitia mihi magna est, et continuus dolor cordi meo.
${}^{3}$~Optabam enim ego ipse anathema esse a Christo pro fratribus meis, qui sunt cognati mei secundum carnem,
${}^{4}$~qui sunt Isra\"elit\ae , quorum adoptio est filiorum, et gloria, et testamentum, et legislatio, et obsequium, et promissa~:
${}^{5}$~quorum patres, et ex quibus est Christus secundum carnem, qui est super omnia Deus benedictus in s\ae cula. Amen.
${}^{6}$~Non autem quod exciderit verbum Dei. Non enim omnes qui ex Isra\"el sunt, ii sunt Isra\"elit\ae~:
${}^{7}$~neque qui semen sunt Abrah\ae , omnes filii~: sed in Isaac vocabitur tibi semen~:
${}^{8}$~id est, non qui filii carnis, hi filii Dei~: sed qui filii sunt promissionis, \ae stimantur in semine.
${}^{9}$~Promissionis enim verbum hoc est~: Secundum hoc tempus veniam~: et erit Sar\ae\ filius.
${}^{10}$~Non solum autem illa~: sed et Rebecca ex uno concubitu habens, Isaac patris nostri.
${}^{11}$~Cum enim nondum nati fuissent, aut aliquid boni egissent, aut mali (ut secundum electionem propositum Dei maneret),
${}^{12}$~non ex operibus, sed ex vocante dictum est ei quia major serviet minori,
${}^{13}$~sicut scriptum est~: Jacob dilexi, Esau autem odio habui.


${}^{14}$~Quid ergo dicemus~? numquid iniquitas apud Deum~? Absit.
${}^{15}$~Moysi enim dicit~: Miserebor cujus misereor~: et misericordiam pr\ae stabo cujus miserebor.
${}^{16}$~Igitur non volentis, neque currentis, sed miserentis est Dei.
${}^{17}$~Dicit enim Scriptura Pharaoni~: Quia in hoc ipsum excitavi te, ut ostendam in te virtutem meam~: et ut annuntietur nomen meum in universa terra.
${}^{18}$~Ergo cujus vult miseretur, et quem vult indurat.
${}^{19}$~Dicis itaque mihi~: Quid adhuc queritur~? voluntati enim ejus quis resistit~?
${}^{20}$~O homo, tu quis es, qui respondeas Deo~? numquid dicit figmentum ei qui se finxit~: Quid me fecisti sic~?
${}^{21}$~an non habet potestatem figulus luti ex eadem massa facere aliud quidem vas in honorem, aliud vero in contumeliam~?
${}^{22}$~Quod si Deus volens ostendere iram, et notum facere potentiam suam, sustinuit in multa patientia vasa ir\ae , apta in interitum,
${}^{23}$~ut ostenderet divitias glori\ae\ su\ae\ in vasa misericordi\ae , qu\ae\ pr\ae paravit in gloriam.
${}^{24}$~Quos et vocavit nos non solum ex Jud\ae is, sed etiam in gentibus,
${}^{25}$~sicut in Osee dicit~: Vocabo non plebem meam, plebem meam~: et non dilectam, dilectam~: et non misericordiam consecutam, misericordiam consecutam.
${}^{26}$~Et erit~: in loco, ubi dictum est eis~: Non plebs mea vos~: ibi vocabuntur filii Dei vivi.
${}^{27}$~Isaias autem clamat pro Isra\"el~: Si fuerit numerus filiorum Isra\"el tamquam arena maris, reliqui\ae\ salv\ae\ fient.
${}^{28}$~Verbum enim consummans, et abbrevians in \ae quitate~: quia verbum breviatum faciet Dominus super terram~:
${}^{29}$~et sicut pr\ae dixit Isaias~: Nisi Dominus Sabaoth reliquisset nobis semen, sicut Sodoma facti essemus, et sicut Gomorrha similes fuissemus.


${}^{30}$~Quid ergo dicemus~? Quod gentes, qu\ae\ non sectabantur justitiam, apprehenderunt justitiam~: justitiam autem, qu\ae\ ex fide est.
${}^{31}$~Isra\"el vero sectando legem justiti\ae , in legem justiti\ae\ non pervenit.
${}^{32}$~Quare~? Quia non ex fide, sed quasi ex operibus~: offenderunt enim in lapidem offensionis,
${}^{33}$~sicut scriptum est~: Ecce pono in Sion lapidem offensionis, et petram scandali~: et omnis qui credit in eum, non confundetur.
\Needspace{2.5\baselineskip}\versal{10}~\lettrine[lines=10,image=true,loversize=0.05,lraise=-0.03]{F}{}ratres, voluntas quidem cordis mei, et obsecratio ad Deum, fit pro illis in salutem.
${}^{2}$~Testimonium enim perhibeo illis quod \ae mulationem Dei habent, sed non secundum scientiam.
${}^{3}$~Ignorantes enim justitiam Dei, et suam qu\ae rentes statuere, justiti\ae\ Dei non sunt subjecti.
${}^{4}$~Finis enim legis, Christus, ad justitiam omni credenti.


${}^{5}$~Moyses enim scripsit, quoniam justitiam, qu\ae\ ex lege est, qui fecerit homo, vivet in ea.
${}^{6}$~Qu\ae\ autem ex fide est justitia, sic dicit~: Ne dixeris in corde tuo~: Quis ascendet in c\ae lum~? id est, Christum deducere~:
${}^{7}$~aut, Quis descendet in abyssum~? hoc est, Christum a mortuis revocare.
${}^{8}$~Sed quid dicit Scriptura~? Prope est verbum in ore tuo, et in corde tuo~: hoc est verbum fidei, quod pr\ae dicamus.
${}^{9}$~Quia si confitearis in ore tuo Dominum Jesum, et in corde tuo credideris quod Deus illum suscitavit a mortuis, salvus eris.
${}^{10}$~Corde enim creditur ad justitiam~: ore autem confessio fit ad salutem.
${}^{11}$~Dicit enim Scriptura~: Omnis qui credit in illum, non confundetur.
${}^{12}$~Non enim est distinctio Jud\ae i et Gr\ae ci~: nam idem Dominus omnium, dives in omnes qui invocant illum.
${}^{13}$~Omnis enim quicumque invocaverit nomen Domini, salvus erit.


${}^{14}$~Quomodo ergo invocabunt, in quem non crediderunt~? aut quomodo credent ei, quem non audierunt~? quomodo autem audient sine pr\ae dicante~?
${}^{15}$~quomodo vero pr\ae dicabunt nisi mittantur~? sicut scriptum est~: Quam speciosi pedes evangelizantium pacem, evangelizantium bona~!
${}^{16}$~Sed non omnes obediunt Evangelio. Isaias enim dicit~: Domine, quis credidit auditui nostro~?
${}^{17}$~Ergo fides ex auditu, auditus autem per verbum Christi.
${}^{18}$~Sed dico~: Numquid non audierunt~? Et quidem in omnem terram exivit sonus eorum, et in fines orbis terr\ae\ verba eorum.
${}^{19}$~Sed dico~: Numquid Isra\"el non cognovit~? Primus Moyses dicit~: Ego ad \ae mulationem vos adducam in non gentem~: in gentem insipientem, in iram vos mittam.
${}^{20}$~Isaias autem audet, et dicit~: Inventus sum a non qu\ae rentibus me~: palam apparui iis qui me non interrogabant.
${}^{21}$~Ad Isra\"el autem dicit~: Tota die expandi manus meas ad populum non credentem, et contradicentem.
\Needspace{2.5\baselineskip}\versal{11}~\lettrine[lines=10,image=true,loversize=0.05,lraise=-0.03]{D}{}ico ergo~: Numquid Deus repulit populum suum~? Absit. Nam et ego Isra\"elita sum ex semine Abraham, de tribu Benjamin~:
${}^{2}$~non repulit Deus plebem suam, quam pr\ae scivit. An nescitis in Elia quid dicit Scriptura~? quemadmodum interpellat Deum adversum Isra\"el~:
${}^{3}$~Domine, prophetas tuos occiderunt, altaria tua suffoderunt~: et ego relictus sum solus, et qu\ae runt animam meam.
${}^{4}$~Sed quid dicit illi divinum responsum~? Reliqui mihi septem millia virorum, qui non curvaverunt genua ante Baal.
${}^{5}$~Sic ergo et in hoc tempore reliqui\ae\ secundum electionem grati\ae\ salv\ae\ fact\ae\ sunt.
${}^{6}$~Si autem gratia, jam non ex operibus~: alioquin gratia jam non est gratia.
${}^{7}$~Quid ergo~? Quod qu\ae rebat Isra\"el, hoc non est consecutus~: electio autem consecuta est~: ceteri vero exc\ae cati sunt~:
${}^{8}$~sicut scriptum est~: Dedit illis Deus spiritum compunctionis~: oculos ut non videant, et aures ut non audiant, usque in hodiernum diem.
${}^{9}$~Et David dicit~: Fiat mensa eorum in laqueum, et in captionem, et in scandalum, et in retributionem illis.
${}^{10}$~Obscurentur oculi eorum ne videant~: et dorsum eorum semper incurva.


${}^{11}$~Dico ergo~: Numquid sic offenderunt ut caderent~? Absit. Sed illorum delicto, salus est gentibus ut illos \ae mulentur.
${}^{12}$~Quod si delictum illorum diviti\ae\ sunt mundi, et diminutio eorum diviti\ae\ gentium~: quanto magis plenitudo eorum~?
${}^{13}$~Vobis enim dico gentibus~: Quamdiu quidem ego sum gentium Apostolus, ministerium meum honorificabo,
${}^{14}$~si quomodo ad \ae mulandum provocem carnem meam, et salvos faciam aliquos ex illis.
${}^{15}$~Si enim amissio eorum, reconciliatio est mundi~: qu\ae\ assumptio, nisi vita ex mortuis~?
${}^{16}$~Quod si delibatio sancta est, et massa~: et si radix sancta, et rami.
${}^{17}$~Quod si aliqui ex ramis fracti sunt, tu autem cum oleaster esses, insertus es in illis, et socius radicis, et pinguedinis oliv\ae\ factus es,
${}^{18}$~noli gloriari adversus ramos. Quod si gloriaris~: non tu radicem portas, sed radix te.
${}^{19}$~Dices ergo~: Fracti sunt rami ut ego inserar.
${}^{20}$~Bene~: propter incredulitatem fracti sunt. Tu autem fide stas~: noli altum sapere, sed time.
${}^{21}$~Si enim Deus naturalibus ramis non pepercit~: ne forte nec tibi parcat.
${}^{22}$~Vide ergo bonitatem, et severitatem Dei~: in eos quidem qui ceciderunt, severitatem~: in te autem bonitatem Dei, si permanseris in bonitate, alioquin et tu excideris.
${}^{23}$~Sed et illi, si non permanserint in incredulitate, inserentur~: potens est enim Deus iterum inserere illos.
${}^{24}$~Nam si tu ex naturali excisus es oleastro, et contra naturam insertus es in bonam olivam~: quanto magis ii qui secundum naturam inserentur su\ae\ oliv\ae~?


${}^{25}$~Nolo enim vos ignorare, fratres, mysterium hoc (ut non sitis vobis ipsis sapientes), quia c\ae citas ex parte contigit in Isra\"el, donec plenitudo gentium intraret,
${}^{26}$~et sic omnis Isra\"el salvus fieret, sicut scriptum est~: Veniet ex Sion, qui eripiat, et avertat impietatem a Jacob.
${}^{27}$~Et hoc illis a me testamentum~: cum abstulero peccata eorum.
${}^{28}$~Secundum Evangelium quidem, inimici propter vos~: secundum electionem autem, carissimi propter patres.
${}^{29}$~Sine pœnitentia enim sunt dona et vocatio Dei.
${}^{30}$~Sicut enim aliquando et vos non credidistis Deo, nunc autem misericordiam consecuti estis propter incredulitatem illorum~:
${}^{31}$~ita et isti nunc non crediderunt in vestram misericordiam~: ut et ipsi misericordiam consequantur.
${}^{32}$~Conclusit enim Deus omnia in incredulitate, ut omnium misereatur.


${}^{33}$~O altitudo divitiarum sapienti\ae , et scienti\ae\ Dei~: quam incomprehensibilia sunt judicia ejus, et investigabiles vi\ae\ ejus~!
${}^{34}$~Quis enim cognovit sensum Domini~? aut quis consiliarius ejus fuit~?
${}^{35}$~aut quis prior dedit illi, et retribuetur ei~?
${}^{36}$~Quoniam ex ipso, et per ipsum, et in ipso sunt omnia~: ipsi gloria in s\ae cula. Amen.
\Needspace{2.5\baselineskip}\versal{12}~\lettrine[lines=10,image=true,loversize=0.05,lraise=-0.03]{O}{}bsecro itaque vos fratres per misericordiam Dei, ut exhibeatis corpora vestra hostiam viventem, sanctam, Deo placentem, rationabile obsequium vestrum.
${}^{2}$~Et nolite conformari huic s\ae culo, sed reformamini in novitate sensus vestri~: ut probetis qu\ae\ sit voluntas Dei bona, et beneplacens, et perfecta.
${}^{3}$~Dico enim per gratiam qu\ae\ data est mihi, omnibus qui sunt inter vos, non plus sapere quam oportet sapere, sed sapere ad sobrietatem~: et unicuique sicut Deus divisit mensuram fidei.
${}^{4}$~Sicut enim in uno corpore multa membra habemus, omnia autem membra non eumdem actum habent~:
${}^{5}$~ita multi unum corpus sumus in Christo, singuli autem alter alterius membra.
${}^{6}$~Habentes autem donationes secundum gratiam, qu\ae\ data est nobis, differentes~: sive prophetiam secundum rationem fidei,
${}^{7}$~sive ministerium in ministrando, sive qui docet in doctrina,
${}^{8}$~qui exhortatur in exhortando, qui tribuit in simplicitate, qui pr\ae est in sollicitudine, qui miseretur in hilaritate.


${}^{9}$~Dilectio sine simulatione~: odientes malum, adh\ae rentes bono~:
${}^{10}$~caritate fraternitatis invicem diligentes~: honore invicem pr\ae venientes~:
${}^{11}$~sollicitudine non pigri~: spiritu ferventes~: Domino servientes~:
${}^{12}$~spe gaudentes~: in tribulatione patientes~: orationi instantes~:
${}^{13}$~necessitatibus sanctorum communicantes~: hospitalitatem sectantes.
${}^{14}$~Benedicite persequentibus vos~: benedicite, et nolite maledicere.
${}^{15}$~Gaudere cum gaudentibus, flere cum flentibus~:
${}^{16}$~idipsum invicem sentientes~: non alta sapientes, sed humilibus consentientes. Nolite esse prudentes apud vosmetipsos~:
${}^{17}$~nulli malum pro malo reddentes~: providentes bona non tantum coram Deo, sed etiam coram omnibus hominibus.
${}^{18}$~Si fieri potest, quod ex vobis est, cum omnibus hominibus pacem habentes~:
${}^{19}$~non vosmetipsos defendentes carissimi, sed date locum ir\ae . Scriptum est enim~: Mihi vindicta~: ego retribuam, dicit Dominus.
${}^{20}$~Sed si esurierit inimicus tuus, ciba illum~: si sitit, potum da illi~: hoc enim faciens, carbones ignis congeres super caput ejus.
${}^{21}$~Noli vinci a malo, sed vince in bono malum.
\Needspace{2.5\baselineskip}\versal{13}~\lettrine[lines=10,image=true,loversize=0.05,lraise=-0.03]{O}{}mnis anima potestatibus sublimioribus subdita sit~: non est enim potestas nisi a Deo~: qu\ae\ autem sunt, a Deo ordinat\ae\ sunt.
${}^{2}$~Itaque qui resistit potestati, Dei ordinationi resistit. Qui autem resistunt, ipsi sibi damnationem acquirunt~:
${}^{3}$~nam principes non sunt timori boni operis, sed mali. Vis autem non timere potestatem~? Bonum fac~: et habebis laudem ex illa~:
${}^{4}$~Dei enim minister est tibi in bonum. Si autem malum feceris, time~: non enim sine causa gladium portat. Dei enim minister est~: vindex in iram ei qui malum agit.
${}^{5}$~Ideo necessitate subditi estote non solum propter iram, sed etiam propter conscientiam.
${}^{6}$~Ideo enim et tributa pr\ae statis~: ministri enim Dei sunt, in hoc ipsum servientes.
${}^{7}$~Reddite ergo omnibus debita~: cui tributum, tributum~: cui vectigal, vectigal~: cui timorem, timorem~: cui honorem, honorem.


${}^{8}$~Nemini quidquam debeatis, nisi ut invicem diligatis~: qui enim diligit proximum, legem implevit.
${}^{9}$~Nam~: Non adulterabis~: non occides~: non furaberis~: non falsum testimonium dices~: non concupisces~: et si quod est aliud mandatum, in hoc verbo instauratur~: diliges proximum tuum sicut teipsum.
${}^{10}$~Dilectio proximi malum non operatur. Plenitudo ergo legis est dilectio.


${}^{11}$~Et hoc scientes tempus~: quia hora est jam nos de somno surgere. Nunc enim propior est nostra salus, quam cum credidimus.
${}^{12}$~Nox pr\ae cessit, dies autem appropinquavit. Abjiciamus ergo opera tenebrarum, et induamur arma lucis.
${}^{13}$~Sicut in die honeste ambulemus~: non in comessationibus, et ebrietatibus, non in cubilibus, et impudicitiis, non in contentione, et \ae mulatione~:
${}^{14}$~sed induimini Dominum Jesum Christum, et carnis curam ne feceritis in desideriis.
\Needspace{2.5\baselineskip}\versal{14}~\lettrine[lines=10,image=true,loversize=0.05,lraise=-0.03]{I}{}nfirmum autem in fide assumite, non in disceptationibus cogitationum.
${}^{2}$~Alius enim credit se manducare omnia~: qui autem infirmus est, olus manducet.
${}^{3}$~Is qui manducat, non manducantem non spernat~: et qui non manducat, manducantem non judicet~: Deus enim illum assumpsit.
${}^{4}$~Tu quis es, qui judicas alienum servum~? domino suo stat, aut cadit~: stabit autem~: potens est enim Deus statuere illum.
${}^{5}$~Nam alius judicat diem inter diem~: alius autem judicat omnem diem~: unusquisque in suo sensu abundet.
${}^{6}$~Qui sapit diem, Domino sapit, et qui manducat, Domino manducat~: gratias enim agit Deo. Et qui non manducat, Domino non manducat, et gratias agit Deo.
${}^{7}$~Nemo enim nostrum sibi vivit, et nemo sibi moritur.
${}^{8}$~Sive enim vivemus, Domino vivimus~: sive morimur, Domino morimur. Sive ergo vivimus, sive morimur, Domini sumus.
${}^{9}$~In hoc enim Christus mortuus est, et resurrexit~: ut et mortuorum et vivorum dominetur.
${}^{10}$~Tu autem quid judicas fratrem tuum~? aut tu quare spernis fratrem tuum~? omnes enim stabimus ante tribunal Christi.
${}^{11}$~Scriptum est enim~: Vivo ego, dicit Dominus, quoniam mihi flectetur omne genu~: et omnis lingua confitebitur Deo.
${}^{12}$~Itaque unusquisque nostrum pro se rationem reddet Deo.


${}^{13}$~Non ergo amplius invicem judicemus~: sed hoc judicate magis, ne ponatis offendiculum fratri, vel scandalum.
${}^{14}$~Scio, et confido in Domino Jesu, quia nihil commune per ipsum, nisi ei qui existimat quid commune esset, illi commune est.
${}^{15}$~Si enim propter cibum frater tuus contristatur, jam non secundum caritatem ambulas. Noli cibo tuo illum perdere, pro quo Christus mortuus est.
${}^{16}$~Non ergo blasphemetur bonum nostrum.
${}^{17}$~Non est enim regnum Dei esca et potus~: sed justitia, et pax, et gaudium in Spiritu Sancto~:
${}^{18}$~qui enim in hoc servit Christo, placet Deo, et probatus est hominibus.
${}^{19}$~Itaque qu\ae\ pacis sunt, sectemur~: et qu\ae\ \ae dificationis sunt, in invicem custodiamus.
${}^{20}$~Noli propter escam destruere opus Dei, omnia quidem sunt munda~: sed malum est homini, qui per offendiculum manducat.
${}^{21}$~Bonum est non manducare carnem, et non bibere vinum, neque in quo frater tuus offenditur, aut scandalizatur, aut infirmatur.
${}^{22}$~Tu fidem habes~? penes temetipsum habe coram Deo. Beatus qui non judicat semetipsum in eo quod probat.
${}^{23}$~Qui autem discernit, si manducaverit, damnatus est~: quia non ex fide. Omne autem, quod non est ex fide, peccatum est.
\Needspace{2.5\baselineskip}\versal{15}~\lettrine[lines=10,image=true,loversize=0.05,lraise=-0.03]{D}{}ebemus autem nos firmiores imbecillitates infirmorum sustinere, et non nobis placere.
${}^{2}$~Unusquisque vestrum proximo suo placeat in bonum, ad \ae dificationem.
${}^{3}$~Etenim Christus non sibi placuit, sed sicut scriptum est~: Improperia improperantium tibi ceciderunt super me.
${}^{4}$~Qu\ae cumque enim scripta sunt, ad nostram doctrinam scripta sunt~: ut per patientiam, et consolationem Scripturarum, spem habeamus.
${}^{5}$~Deus autem patienti\ae\ et solatii det vobis idipsum sapere in alterutrum secundum Jesum Christum~:
${}^{6}$~ut unanimes, uno ore honorificetis Deum et patrem Domini nostri Jesu Christi.
${}^{7}$~Propter quod suscipite invicem, sicut et Christus suscepit vos in honorem Dei.
${}^{8}$~Dico enim Christum Jesum ministrum fuisse circumcisionis propter veritatem Dei, ad confirmandas promissiones patrum~:
${}^{9}$~gentes autem super misericordia honorare Deum, sicut scriptum est~: Propterea confitebor tibi in gentibus, Domine, et nomini tuo cantabo.
${}^{10}$~Et iterum dicit~: L\ae tamini gentes cum plebe ejus.
${}^{11}$~Et iterum~: Laudate omnes gentes Dominum~: et magnificate eum omnes populi.
${}^{12}$~Et rursus Isaias ait~: Erit radix Jesse, et qui exsurget regere gentes, in eum gentes sperabunt.
${}^{13}$~Deus autem spei repleat vos omni gaudio, et pace in credendo~: ut abundetis in spe, et virtute Spiritus Sancti.


${}^{14}$~Certus sum autem fratres mei et ego ipse de vobis, quoniam et ipsi pleni estis dilectione, repleti omni scientia, ita ut possitis alterutrum monere.
${}^{15}$~Audacius autem scripsi vobis fratres ex parte, tamquam in memoriam vos reducens~: propter gratiam, qu\ae\ data est mihi a Deo,
${}^{16}$~ut sim minister Christi Jesu in gentibus~: sanctificans Evangelium Dei, ut fiat oblatio gentium accepta, et sanctificata in Spiritu Sancto.
${}^{17}$~Habeo igitur gloriam in Christo Jesu ad Deum.
${}^{18}$~Non enim audeo aliquid loqui eorum, qu\ae\ per me non efficit Christus in obedientiam gentium, verbo et factis~:
${}^{19}$~in virtute signorum, et prodigiorum, in virtute Spiritus Sancti~: ita ut ab Jerusalem per circuitum usque ad Illyricum repleverim Evangelium Christi.
${}^{20}$~Sic autem pr\ae dicavi Evangelium hoc, non ubi nominatus est Christus, ne super alienum fundamentum \ae dificarem~:
${}^{21}$~sed sicut scriptum est~: Quibus non est annuntiatum de eo, videbunt~: et qui non audierunt, intelligent.


${}^{22}$~Propter quod et impediebar plurimum venire ad vos, et prohibitus sum usque adhuc.
${}^{23}$~Nunc vero ulterius locum non habens in his regionibus, cupiditatem autem habens veniendi ad vos ex multis jam pr\ae cedentibus annis~:
${}^{24}$~cum in Hispaniam proficisci cœpero, spero quod pr\ae teriens videam vos, et a vobis deducar illuc, si vobis primum ex parte fruitus fuero.
${}^{25}$~Nunc igitur proficiscar in Jerusalem ministrare sanctis.
${}^{26}$~Probaverunt enim Macedonia et Achaia collationem aliquam facere in pauperes sanctorum, qui sunt in Jerusalem.
${}^{27}$~Placuit enim eis~: et debitores sunt eorum. Nam si spiritualium eorum participes facti sunt gentiles, debent et in carnalibus ministrare illis.
${}^{28}$~Hoc igitur cum consummavero, et assignavero eis fructum hunc, per vos proficiscar in Hispaniam.
${}^{29}$~Scio autem quoniam veniens ad vos, in abundantia benedictionis Evangelii Christi veniam.
${}^{30}$~Obsecro ergo vos fratres per Dominum nostrum Jesum Christum, et per caritatem Sancti Spiritus, ut adjuvetis me in orationibus vestris pro me ad Deum,
${}^{31}$~ut liberer ab infidelibus, qui sunt in Jud\ae a, et obsequii mei oblatio accepta fiat in Jerusalem sanctis,
${}^{32}$~ut veniam ad vos in gaudio per voluntatem Dei, et refrigerer vobiscum.
${}^{33}$~Deus autem pacis sit cum omnibus vobis. Amen.
\Needspace{2.5\baselineskip}\versal{16}~\lettrine[lines=10,image=true,loversize=0.05,lraise=-0.03]{C}{}ommendo autem vobis Phœben sororem nostram, qu\ae\ est in ministerio ecclesi\ae , qu\ae\ est in Cenchris~:
${}^{2}$~ut eam suscipiatis in Domino digne sanctis~: et assistatis ei in quocumque negotio vestri indiguerit~: etenim ipsa quoque astitit multis, et mihi ipsi.
${}^{3}$~Salutate Priscam et Aquilam, adjutores meos in Christo Jesu
${}^{4}$~(qui pro anima mea suas cervices supposuerunt~: quibus non solus ego gratias ago, sed et cunct\ae\ ecclesi\ae\ gentium),
${}^{5}$~et domesticam ecclesiam eorum. Salutate Ep\ae netum dilectum mihi, qui est primitivus Asi\ae\ in Christo.
${}^{6}$~Salutate Mariam, qu\ae\ multum laboravit in vobis.
${}^{7}$~Salutate Andronicum et Juniam, cognatos, et concaptivos meos~: qui sunt nobiles in Apostolis, qui et ante me fuerunt in Christo.
${}^{8}$~Salutate Ampliatum dilectissimum mihi in Domino.
${}^{9}$~Salutate Urbanum adjutorem nostrum in Christo Jesu, et Stachyn dilectum meum.
${}^{10}$~Salutate Apellen probum in Christo.
${}^{11}$~Salutate eos qui sunt ex Aristoboli domo. Salutate Herodionem cognatum meum. Salutate eos qui sunt ex Narcisi domo, qui sunt in Domino.
${}^{12}$~Salutate Tryph\ae nam et Tryphosam, qu\ae\ laborant in Domino. Salutate Persidem carissimam, qu\ae\ multum laboravit in Domino.
${}^{13}$~Salutate Rufum electum in Domino, et matrem ejus, et meam.
${}^{14}$~Salutate Asyncritum, Phlegontem, Hermam, Patrobam, Hermen, et qui cum eis sunt, fratres.
${}^{15}$~Salutate Philologum et Juliam, Nereum, et sororem ejus, et Olympiadem, et omnes qui cum eis sunt, sanctos.
${}^{16}$~Salutate invicem in osculo sancto. Salutant vos omnes ecclesi\ae\ Christi.
${}^{17}$~Rogo autem vos fratres, ut observetis eos qui dissensiones et offendicula, pr\ae ter doctrinam, quam vos didicistis, faciunt, et declinate ab illis.
${}^{18}$~Hujuscemodi enim Christo Domino nostro non serviunt, sed suo ventri~: et per dulces sermones et benedictiones seducunt corda innocentium.
${}^{19}$~Vestra enim obedientia in omnem locum divulgata est. Gaudeo igitur in vobis. Sed volo vos sapientes esse in bono, et simplices in malo.
${}^{20}$~Deus autem pacis conterat Satanam sub pedibus vestris velociter. Gratia Domini nostri Jesu Christi vobiscum.
${}^{21}$~Salutat vos Timotheus adjutor meus, et Lucius, et Jason, et Sosipater cognati mei.
${}^{22}$~Saluto vos ego Tertius, qui scripsi epistolam, in Domino.
${}^{23}$~Salutat vos Cajus hospes meus, et universa ecclesia. Salutat vos Erastus arcarius civitatis, et Quartus, frater.
${}^{24}$~Gratia Domini nostri Jesu Christi cum omnibus vobis. Amen.


${}^{25}$~Ei autem, qui potens est vos confirmare juxta Evangelium meum, et pr\ae dicationem Jesu Christi, secundum revelationem mysterii temporibus \ae ternis taciti
${}^{26}$~(quod nunc patefactum est per Scripturas prophetarum secundum pr\ae ceptum \ae terni Dei, ad obeditionem fidei), in cunctis gentibus cogniti,
${}^{27}$~soli sapienti Deo, per Jesum Christum, cui honor et gloria in s\ae cula s\ae culorum. Amen.
