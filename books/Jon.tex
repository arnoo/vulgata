\clearpage
{\centering \section*{Prophetia Jonæ}}\thispagestyle{empty}
\addcontentsline{toc}{subsection}{Jonas}
\fancyhead[C]{\textsc{Jonas}}

\Needspace{2.5\baselineskip}\versal{1}~Et factum est verbum Domini ad Jonam, filium Amathi, dicens~:
${}^{2}$~Surge, et vade in Niniven, civitatem grandem, et pr\ae dica in ea, quia ascendit malitia ejus coram me.
${}^{3}$~Et surrexit Jonas, ut fugeret in Tharsis a facie Domini, et descendit in Joppen~: et invenit navem euntem in Tharsis, et dedit naulum ejus, et descendit in eam ut iret cum eis in Tharsis a facie Domini.
${}^{4}$~Dominus autem misit ventum magnum in mare~: et facta est tempestas magna in mari, et navis periclitabatur conteri.
${}^{5}$~Et timuerunt naut\ae , et clamaverunt viri ad deum suum, et miserunt vasa qu\ae\ erant in navi, in mare, ut alleviaretur ab eis~; et Jonas descendit ad interiora navis, et dormiebat sopore gravi.
${}^{6}$~Et accessit ad eum gubernator, et dixit ei~: Quid tu sopore deprimeris~? surge, invoca Deum tuum, si forte recogitet Deus de nobis, et non pereamus.
${}^{7}$~Et dixit vir ad collegam suum~: Venite et mittamus sortes, et sciamus quare hoc malum sit nobis. Et miserunt sortes, et cecidit sors super Jonam.
${}^{8}$~Et dixerunt ad eum~: Indica nobis cujus causa malum istud sit nobis~: quod est opus tuum~? qu\ae\ terra tua, et quo vadis~? vel ex quo populo es tu~?
${}^{9}$~Et dixit ad eos~: Hebr\ae us ego sum, et Dominum Deum c\ae li ego timeo, qui fecit mare et aridam.
${}^{10}$~Et timuerunt viri timore magno, et dixerunt ad eum~: Quid hoc fecisti~? cognoverunt enim viri quod a facie Domini fugeret, quia indicaverat eis.
${}^{11}$~Et dixerunt ad eum~: Quid faciemus tibi, et cessabit mare a nobis~? quia mare ibat, et intumescebat.
${}^{12}$~Et dixit ad eos~: Tollite me, et mittite in mare, et cessabit mare a vobis~: scio enim ego quoniam propter me tempestas h\ae c grandis venit super vos.
${}^{13}$~Et remigabant viri ut reverterentur ad aridam, et non valebant, quia mare ibat, et intumescebat super eos.
${}^{14}$~Et clamaverunt ad Dominum, et dixerunt~: Qu\ae sumus, Domine, ne pereamus in anima viri istius, et ne des super nos sanguinem innocentem~: quia tu, Domine, sicut voluisti, fecisti.
${}^{15}$~Et tulerunt Jonam, et miserunt in mare~: et stetit mare a fervore suo.
${}^{16}$~Et timuerunt viri timore magno Dominum~: et immolaverunt hostias Domino, et voverunt vota.
\Needspace{2.5\baselineskip}\versal{2}~Et pr\ae paravit Dominus piscem grandem ut deglutiret Jonam~: et erat Jonas in ventre piscis tribus diebus et tribus noctibus.
${}^{2}$~Et oravit Jonas ad Dominum Deum suum de ventre piscis,
${}^{3}$~et dixit~: \begin{flushleft}\begin{verse}Clamavi de tribulatione mea ad Dominum,\\ et exaudivit me~;\\ de ventre inferi clamavi,\\ et exaudisti vocem meam.\\
${}^{4}$~Et projecisti me in profundum in corde maris,\\ et flumen circumdedit me~:\\ omnes gurgites tui, et fluctus tui super me transierunt.\\
${}^{5}$~Et ego dixi~:\\ Abjectus sum a conspectu oculorum tuorum~;\\ verumtamen rursus videbo templum sanctum tuum.\\
${}^{6}$~Circumdederunt me aqu\ae\ usque ad animam~:\\ abyssus vallavit me,\\ pelagus operuit caput meum.\\
${}^{7}$~Ad extrema montium descendi~;\\ terr\ae\ vectes concluserunt me in \ae ternum~:\\ et sublevabis de corruptione vitam meam, Domine Deus meus.\\
${}^{8}$~Cum angustiaretur in me anima mea,\\ Domini recordatus sum~:\\ ut veniat ad te oratio mea,\\ ad templum sanctum tuum.\\
${}^{9}$~Qui custodiunt vanitates frustra,\\ misericordiam suam derelinquunt.\\
${}^{10}$~Ego autem in voce laudis immolabo tibi~:\\ qu\ae cumque vovi, reddam pro salute Domino.\end{verse}\end{flushleft}


${}^{11}$~Et dixit Dominus pisci, et evomuit Jonam in aridam.
\Needspace{2.5\baselineskip}\versal{3}~Et factum est verbum Domini ad Jonam secundo, dicens~:
${}^{2}$~Surge, et vade in Niniven, civitatem magnam, et pr\ae dica in ea pr\ae dicationem quam ego loquor ad te.
${}^{3}$~Et surrexit Jonas, et abiit in Niniven juxta verbum Domini~: et Ninive erat civitas magna, itinere trium dierum.
${}^{4}$~Et cœpit Jonas introire in civitatem itinere diei unius~: et clamavit, et dixit~: Adhuc quadraginta dies, et Ninive subvertetur.
${}^{5}$~Et crediderunt viri Ninivit\ae\ in Deum, et pr\ae dicaverunt jejunium, et vestiti sunt saccis, a majore usque ad minorem.
${}^{6}$~Et pervenit verbum ad regem Ninive~: et surrexit de solio suo, et abjecit vestimentum suum a se, et indutus est sacco, et sedit in cinere.
${}^{7}$~Et clamavit, et dixit in Ninive ex ore regis et principum ejus, dicens~: Homines, et jumenta, et boves, et pecora non gustent quidquam~: nec pascantur, et aquam non bibant.
${}^{8}$~Et operiantur saccis homines et jumenta, et clament ad Dominum in fortitudine~: et convertatur vir a via sua mala, et ab iniquitate qu\ae\ est in manibus eorum.
${}^{9}$~Quis scit si convertatur et ignoscat Deus, et revertatur a furore ir\ae\ su\ae , et non peribimus~?
${}^{10}$~Et vidit Deus opera eorum, quia conversi sunt de via sua mala~: et misertus est Deus super malitiam quam locutus fuerat ut faceret eis, et non fecit.
\Needspace{2.5\baselineskip}\versal{4}~Et afflictus est Jonas afflictione magna, et iratus est~:
${}^{2}$~et oravit ad Dominum, et dixit~: Obsecro, Domine, numquid non hoc est verbum meum cum adhuc essem in terra mea~? propter hoc pr\ae occupavi ut fugerem in Tharsis~: scio enim quia tu Deus clemens et misericors es, patiens et mult\ae\ miserationis, et ignoscens super malitia.
${}^{3}$~Et nunc, Domine, tolle, qu\ae so, animam meam a me, quia melior est mihi mors quam vita.
${}^{4}$~Et dixit Dominus~: Putasne bene irasceris tu~?
${}^{5}$~Et egressus est Jonas de civitate, et sedit contra orientem civitatis~: et fecit sibimet umbraculum ibi, et sedebat subter illud in umbra, donec videret quid accideret civitati.
${}^{6}$~Et pr\ae paravit Dominus Deus hederam, et ascendit super caput Jon\ae , ut esset umbra super caput ejus, et protegeret eum (laboraverat enim)~: et l\ae tatus est Jonas super hedera l\ae titia magna.
${}^{7}$~Et paravit Deus vermen ascensu diluculi in crastinum~: et percussit hederam, et exaruit.
${}^{8}$~Et cum ortus fuisset sol, pr\ae cepit Dominus vento calido et urenti~: et percussit sol super caput Jon\ae , et \ae stuabat~: et petivit anim\ae\ su\ae\ ut moreretur, et dixit~: Melius est mihi mori quam vivere.
${}^{9}$~Et dixit Dominus ad Jonam~: Putasne bene irasceris tu super hedera~? Et dixit~: Bene irascor ego usque ad mortem.
${}^{10}$~Et dixit Dominus~: Tu doles super hederam in qua non laborasti, neque fecisti ut cresceret~; qu\ae\ sub una nocte nata est, et sub una nocte periit~:
${}^{11}$~et ego non parcam Ninive, civitati magn\ae , in qua sunt plus quam centum viginti millia hominum qui nesciunt quid sit inter dexteram et sinistram suam, et jumenta multa~?
