{\centering \section*{Liber Tertius Regum}}\thispagestyle{empty}
\addcontentsline{toc}{subsection}{Regum III}
\fancyhead[C]{\textsc{Regum III}}

\Needspace{2.5\baselineskip}\versal{1}~\lettrine[lines=10,image=true,loversize=0.05,lraise=-0.03]{E}{}t rex David senuerat, habebatque \ae tatis plurimos dies~: cumque operiretur vestibus, non calefiebat.
${}^{2}$~Dixerunt ergo ei servi sui~: Qu\ae ramus domino nostro regi adolescentulam virginem, et stet coram rege, et foveat eum, dormiatque in sinu suo, et calefaciat dominum nostrum regem.
${}^{3}$~Qu\ae sierunt igitur adolescentulam speciosam in omnibus finibus Isra\"el, et invenerunt Abisag Sunamitidem, et adduxerunt eam ad regem.
${}^{4}$~Erat autem puella pulchra nimis, dormiebatque cum rege, et ministrabat ei~: rex vero non cognovit eam.


${}^{5}$~Adonias autem filius Haggith elevabatur, dicens~: Ego regnabo. Fecitque sibi currus et equites, et quinquaginta viros qui currerent ante eum.
${}^{6}$~Nec corripuit eum pater suus aliquando, dicens~: Quare hoc fecisti~? Erat autem et ipse pulcher valde, secundus natu post Absalom.
${}^{7}$~Et sermo ei cum Joab filio Sarvi\ae , et cum Abiathar sacerdote, qui adjuvabant partes Adoni\ae .
${}^{8}$~Sadoc vero sacerdos, et Banaias filius Jojad\ae , et Nathan propheta, et Semei et Rei, et robur exercitus David, non erat cum Adonia.
${}^{9}$~Immolatis ergo Adonias arietibus et vitulis, et universis pinguibus, juxta lapidem Zoheleth, qui erat vicinus fonti Rogel, vocavit universos fratres suos filios regis, et omnes viros Juda servos regis.
${}^{10}$~Nathan autem prophetam, et Banaiam, et robustos quosque, et Salomonem fratrem suum non vocavit.


${}^{11}$~Dixit itaque Nathan ad Bethsabee matrem Salomonis~: Num audisti quod regnaverit Adonias filius Haggith, et dominus noster David hoc ignorat~?
${}^{12}$~Nunc ergo veni, accipe consilium a me, et salva animam tuam, filiique tui Salomonis.
${}^{13}$~Vade, et ingredere ad regem David, et dic ei~: Nonne tu, domine mi rex, jurasti mihi ancill\ae\ tu\ae , dicens~: Salomon filius tuus regnabit post me, et ipse sedebit in solio meo~? quare ergo regnat Adonias~?
${}^{14}$~Et adhuc ibi te loquente cum rege, ego veniam post te, et complebo sermones tuos.
${}^{15}$~Ingressa est itaque Bethsabee ad regem in cubiculum~: rex autem senuerat nimis, et Abisag Sunamitis ministrabat ei.
${}^{16}$~Inclinavit se Bethsabee, et adoravit regem. Ad quam rex~: Quid tibi, inquit, vis~?
${}^{17}$~Qu\ae\ respondens, ait~: Domine mi, tu jurasti per Dominum Deum tuum ancill\ae\ tu\ae~: Salomon filius tuus regnabit post me, et ipse sedebit in solio meo.
${}^{18}$~Et ecce nunc Adonias regnat, te, domine mi rex, ignorante.
${}^{19}$~Mactavit boves, et pinguia qu\ae que, et arietes plurimos, et vocavit omnes filios regis, Abiathar quoque sacerdotem, et Joab principem militi\ae~: Salomonem autem servum tuum non vocavit.
${}^{20}$~Verumtamen, domine mi rex, in te oculi respiciunt totius Isra\"el, ut indices eis quis sedere debeat in solio tuo, domine mi rex, post te.
${}^{21}$~Eritque, cum dormierit dominus meus rex cum patribus suis, erimus ego et filius meus Salomon peccatores.


${}^{22}$~Adhuc illa loquente cum rege, Nathan propheta venit.
${}^{23}$~Et nuntiaverunt regi, dicentes~: Adest Nathan propheta. Cumque introisset in conspectu regis, et adorasset eum pronus in terram,
${}^{24}$~dixit Nathan~: Domine mi rex, tu dixisti~: Adonias regnet post me, et ipse sedeat super thronum meum~?
${}^{25}$~Quia descendit hodie, et immolavit boves, et pinguia, et arietes plurimos, et vocavit universos filios regis et principes exercitus, Abiathar quoque sacerdotem, illisque vescentibus et bibentibus coram eo, et dicentibus~: Vivat rex Adonias~:
${}^{26}$~me servum tuum, et Sadoc sacerdotem, et Banaiam filium Jojad\ae , et Salomonem famulum tuum non vocavit.
${}^{27}$~Numquid a domino meo rege exivit hoc verbum, et mihi non indicasti servo tuo quis sessurus esset super thronum domini mei regis post eum~?
${}^{28}$~Et respondit rex David, dicens~: Vocate ad me Bethsabee. Qu\ae\ cum fuisset ingressa coram rege, et stetisset ante eum,
${}^{29}$~juravit rex, et ait~: Vivit Dominus, qui eruit animam meam de omni angustia,
${}^{30}$~quia sicut juravi tibi per Dominum Deum Isra\"el, dicens~: Salomon filius tuus regnabit post me, et ipse sedebit super solium meum pro me~: sic faciam hodie.
${}^{31}$~Summissoque Bethsabee in terram vultu, adoravit regem, dicens~: Vivat dominus meus David in \ae ternum.


${}^{32}$~Dixit quoque rex David~: Vocate mihi Sadoc sacerdotem, et Nathan prophetam, et Banaiam filium Jojad\ae . Qui cum ingressi fuissent coram rege,
${}^{33}$~dixit ad eos~: Tollite vobiscum servos domini vestri, et imponite Salomonem filium meum super mulam meam, et ducite eum in Gihon.
${}^{34}$~Et ungat eum ibi Sadoc sacerdos et Nathan propheta in regem super Isra\"el~: et canetis buccina, atque dicetis~: Vivat rex Salomon.
${}^{35}$~Et ascendetis post eum, et veniet, et sedebit super solium meum, et ipse regnabit pro me~: illique pr\ae cipiam ut sit dux super Isra\"el et super Judam.
${}^{36}$~Et respondit Banaias filius Jojad\ae\ regi, dicens~: Amen~: sic loquatur Dominus Deus domini mei regis.
${}^{37}$~Quomodo fuit Dominus cum domino meo rege, sic sit cum Salomone, et sublimius faciat solium ejus a solio domini mei regis David.
${}^{38}$~Descendit ergo Sadoc sacerdos, et Nathan propheta, et Banaias filius Jojad\ae , et Cerethi, et Phelethi~: et imposuerunt Salomonem super mulam regis David, et adduxerunt eum in Gihon.
${}^{39}$~Sumpsitque Sadoc sacerdos cornu olei de tabernaculo, et unxit Salomonem~: et cecinerunt buccina, et dixit omnis populus~: Vivat rex Salomon.
${}^{40}$~Et ascendit universa multitudo post eum, et populus canentium tibiis, et l\ae tantium gaudio magno~: et insonuit terra a clamore eorum.


${}^{41}$~Audivit autem Adonias, et omnes qui invitati fuerant ab eo~: jamque convivium finitum erat~: sed et Joab, audita voce tub\ae , ait~: Quid sibi vult clamor civitatis tumultuantis~?
${}^{42}$~Adhuc illo loquente, Jonathas filius Abiathar sacerdotis venit~: cui dixit Adonias~: Ingredere, quia vir fortis es, et bona nuntians.
${}^{43}$~Responditque Jonathas Adoni\ae~: Nequaquam~: dominus enim noster rex David regem constituit Salomonem~:
${}^{44}$~misitque cum eo Sadoc sacerdotem, et Nathan prophetam, et Banaiam filium Jojad\ae , et Cerethi, et Phelethi, et imposuerunt eum super mulam regis.
${}^{45}$~Unxeruntque eum Sadoc sacerdos et Nathan propheta regem in Gihon~: et ascenderunt inde l\ae tantes, et insonuit civitas~: h\ae c est vox quam audistis.
${}^{46}$~Sed et Salomon sedet super solium regni.
${}^{47}$~Et ingressi servi regis benedixerunt domino nostro regi David, dicentes~: Amplificet Deus nomen Salomonis super nomen tuum, et magnificet thronus ejus super thronum tuum. Et adoravit rex in lectulo suo~:
${}^{48}$~et locutus est~: Benedictus Dominus Deus Isra\"el, qui dedit hodie sedentem in solio meo, videntibus oculis meis.


${}^{49}$~Territi sunt ergo, et surrexerunt omnes qui invitati fuerant ab Adonia, et ivit unusquisque in viam suam.
${}^{50}$~Adonias autem timens Salomonem, surrexit, et abiit, tenuitque cornu altaris.
${}^{51}$~Et nuntiaverunt Salomoni, dicentes~: Ecce Adonias timens regem Salomonem, tenuit cornu altaris, dicens~: Juret mihi rex Salomon hodie, quod non interficiat servum suum gladio.
${}^{52}$~Dixitque Salomon~: Si fuerit vir bonus, non cadet ne unus quidem capillus ejus in terram~: sin autem malum inventum fuerit in eo, morietur.
${}^{53}$~Misit ergo rex Salomon, et eduxit eum ab altari~: et ingressus adoravit regem Salomonem~: dixitque ei Salomon~: Vade in domum tuam.
\Needspace{2.5\baselineskip}\versal{2}~\lettrine[lines=10,image=true,loversize=0.05,lraise=-0.03]{A}{}ppropinquaverunt autem dies David ut moreretur~: pr\ae cepitque Salomoni filio suo, dicens~:
${}^{2}$~Ego ingredior viam univers\ae\ terr\ae~: confortare, et esto vir.
${}^{3}$~Et observa custodias Domini Dei tui, ut ambules in viis ejus~: ut custodias c\ae remonias ejus, et pr\ae cepta ejus, et judicia, et testimonia, sicut scriptum est in lege Moysi~: ut intelligas universa qu\ae\ facis, et quocumque te verteris~:
${}^{4}$~ut confirmet Dominus sermones suos quos locutus est de me, dicens~: Si custodierint filii tui vias suas, et ambulaverint coram me in veritate, in omni corde suo et in omni anima sua, non auferetur tibi vir de solio Isra\"el.
${}^{5}$~Tu quoque nosti qu\ae\ fecerit mihi Joab filius Sarvi\ae , qu\ae\ fecerit duobus principibus exercitus Isra\"el, Abner filio Ner, et Amas\ae\ filio Jether~: quos occidit, et effudit sanguinem belli in pace, et posuit cruorem pr\ae lii in balteo suo qui erat circa lumbos ejus, et in calceamento suo quod erat in pedibus ejus.
${}^{6}$~Facies ergo juxta sapientiam tuam, et non deduces canitiem ejus pacifice ad inferos.
${}^{7}$~Sed et filiis Berzellai Galaaditis reddes gratiam, eruntque comedentes in mensa tua~: occurrerunt enim mihi quando fugiebam a facie Absalom fratris tui.
${}^{8}$~Habes quoque apud te Semei filium Gera filii Jemini de Bahurim, qui maledixit mihi maledictione pessima quando ibam ad castra~: sed quia descendit mihi in occursum cum transirem Jordanem, et juravi ei per Dominum, dicens~: Non te interficiam gladio~:
${}^{9}$~tu noli pati eum esse innoxium. Vir autem sapiens es, ut scias qu\ae\ facies ei~: deducesque canos ejus cum sanguine ad inferos.
${}^{10}$~Dormivit igitur David cum patribus suis, et sepultus est in civitate David.
${}^{11}$~Dies autem quibus regnavit David super Isra\"el, quadraginta anni sunt~: in Hebron regnavit septem annis~; in Jerusalem, triginta tribus.


${}^{12}$~Salomon autem sedit super thronum David patris sui, et firmatum est regnum ejus nimis.
${}^{13}$~Et ingressus est Adonias filius Haggith ad Bethsabee matrem Salomonis. Qu\ae\ dixit ei~: Pacificusne est ingressus tuus~? Qui respondit~: Pacificus.
${}^{14}$~Addiditque~: Sermo mihi est ad te. Cui ait~: Loquere. Et ille~:
${}^{15}$~Tu, inquit, nosti, quia meum erat regnum, et me pr\ae posuerat omnis Isra\"el sibi in regem~: sed translatum est regnum, et factum est fratris mei~: a Domino enim constitutum est ei.
${}^{16}$~Nunc ergo petitionem unam precor a te~: ne confundas faciem meam. Qu\ae\ dixit ad eum~: Loquere.
${}^{17}$~Et ille ait~: Precor ut dicas Salomoni regi (neque enim negare tibi quidquam potest) ut det mihi Abisag Sunamitidem uxorem.
${}^{18}$~Et ait Bethsabee~: Bene~: ego loquar pro te regi.
${}^{19}$~Venit ergo Bethsabee ad regem Salomonem ut loqueretur ei pro Adonia~: et surrexit rex in occursum ejus, adoravitque eam, et sedit super thronum suum~: positusque est thronus matri regis, qu\ae\ sedit ad dexteram ejus.
${}^{20}$~Dixitque ei~: Petitionem unam parvulam ego deprecor a te~: ne confundas faciem meam. Et dixit ei rex~: Pete, mater mea~: neque enim fas est ut avertam faciem tuam.
${}^{21}$~Qu\ae\ ait~: Detur Abisag Sunamitis Adoni\ae\ fratri tuo uxor.
${}^{22}$~Responditque rex Salomon, et dixit matri su\ae~: Quare postulas Abisag Sunamitidem Adoni\ae~? postula ei et regnum~: ipse est enim frater meus major me, et habet Abiathar sacerdotem, et Joab filium Sarvi\ae .
${}^{23}$~Juravit itaque rex Salomon per Dominum, dicens~: H\ae c faciat mihi Deus, et h\ae c addat, quia contra animam suam locutus est Adonias verbum hoc.
${}^{24}$~Et nunc vivit Dominus, qui firmavit me, et collocavit me super solium David patris mei, et qui fecit mihi domum, sicut locutus est, quia hodie occidetur Adonias.
${}^{25}$~Misitque rex Salomon per manum Banai\ae\ filii Jojad\ae , qui interfecit eum, et mortuus est.
${}^{26}$~Abiathar quoque sacerdoti dixit rex~: Vade in Anathoth ad agrum tuum~: equidem vir mortis es~: sed hodie te non interficiam, quia portasti arcam Domini Dei coram David patre meo, et sustinuisti laborem in omnibus in quibus laboravit pater meus.
${}^{27}$~Ejecit ergo Salomon Abiathar ut non esset sacerdos Domini, ut impleretur sermo Domini quem locutus est super domum Heli in Silo.


${}^{28}$~Venit autem nuntius ad Joab, quod Joab declinasset post Adoniam, et post Salomonem non declinasset~: fugit ergo Joab in tabernaculum Domini, et apprehendit cornu altaris.
${}^{29}$~Nuntiatumque est regi Salomoni quod fugisset Joab in tabernaculum Domini, et esset juxta altare~: misitque Salomon Banaiam filium Jojad\ae , dicens~: Vade, interfice eum.
${}^{30}$~Et venit Banaias ad tabernaculum Domini, et dixit ei~: H\ae c dicit rex~: Egredere. Qui ait~: Non egrediar, sed hic moriar. Renuntiavit Banaias regi sermonem, dicens~: H\ae c locutus est Joab, et h\ae c respondit mihi.
${}^{31}$~Dixitque ei rex~: Fac sicut locutus est, et interfice eum, et sepeli~: et amovebis sanguinem innocentem qui effusus est a Joab, a me, et a domo patris mei.
${}^{32}$~Et reddet Dominus sanguinem ejus super caput ejus, quia interfecit duos viros justos, melioresque se~: et occidit eos gladio, patre meo David ignorante, Abner filium Ner principem militi\ae\ Isra\"el, et Amasam filium Jether principem exercitus Juda~:
${}^{33}$~et revertetur sanguis illorum in caput Joab, et in caput seminis ejus in sempiternum. David autem et semini ejus, et domui, et throno illius, sit pax usque in \ae ternum a Domino.
${}^{34}$~Ascendit itaque Banaias filius Jojad\ae , et aggressus eum interfecit~: sepultusque est in domo sua in deserto.
${}^{35}$~Et constituit rex Banaiam filium Jojad\ae\ pro eo super exercitum, et Sadoc sacerdotem posuit pro Abiathar.


${}^{36}$~Misit quoque rex, et vocavit Semei~: dixitque ei~: \AE difica tibi domum in Jerusalem, et habita ibi~: et non egredieris inde huc atque illuc.
${}^{37}$~Quacumque autem die egressus fueris, et transieris torrentem Cedron, scito te interficiendum~: sanguis tuus erit super caput tuum.
${}^{38}$~Dixitque Semei regi~: Bonus sermo~: sicut locutus est dominus meus rex, sic faciet servus tuus. Habitavit itaque Semei in Jerusalem diebus multis.
${}^{39}$~Factum est autem post annos tres ut fugerent servi Semei ad Achis filium Maacha regem Geth~: nuntiatumque est Semei quod servi ejus issent in Geth.
${}^{40}$~Et surrexit Semei, et stravit asinum suum, ivitque ad Achis in Geth ad requirendum servos suos, et adduxit eos de Geth.
${}^{41}$~Nuntiatum est autem Salomoni quod isset Semei in Geth de Jerusalem, et rediisset.
${}^{42}$~Et mittens vocavit eum, dixitque illi~: Nonne testificatus sum tibi per Dominum, et pr\ae dixi tibi~: Quacumque die egressus ieris huc et illuc, scito te esse moriturum~: et respondisti mihi~: Bonus sermo, quem audivi~?
${}^{43}$~quare ergo non custodisti jusjurandum Domini, et pr\ae ceptum quod pr\ae ceperam tibi~?
${}^{44}$~Dixitque rex ad Semei~: Tu nosti omne malum cujus tibi conscium est cor tuum, quod fecisti David patri meo~: reddidit Dominus malitiam tuam in caput tuum~:
${}^{45}$~et rex Salomon benedictus, et thronus David erit stabilis coram Domino usque in sempiternum.
${}^{46}$~Jussit itaque rex Banai\ae\ filio Jojad\ae , qui egressus, percussit eum, et mortuus est.
\Needspace{2.5\baselineskip}\versal{3}~\lettrine[lines=10,image=true,loversize=0.05,lraise=-0.03]{C}{}onfirmatum est igitur regnum in manu Salomonis, et affinitate conjunctus est Pharaoni regi \AE gypti~: accepit namque filiam ejus, et adduxit in civitatem David, donec compleret \ae dificans domum suam, et domum Domini, et murum Jerusalem per circuitum.
${}^{2}$~Attamen populus immolabat in excelsis~: non enim \ae dificatum erat templum nomini Domini usque in diem illum.
${}^{3}$~Dilexit autem Salomon Dominum, ambulans in pr\ae ceptis David patris sui, excepto quod in excelsis immolabat, et accendebat thymiama.
${}^{4}$~Abiit itaque in Gabaon, ut immolaret ibi~: illud quippe erat excelsum maximum~: mille hostias in holocaustum obtulit Salomon super altare illud in Gabaon.
${}^{5}$~Apparuit autem Dominus Salomoni per somnium nocte, dicens~: Postula quod vis ut dem tibi.
${}^{6}$~Et ait Salomon~: Tu fecisti cum servo tuo David patre meo misericordiam magnam, sicut ambulavit in conspectu tuo in veritate et justitia, et recto corde tecum~: custodisti ei misericordiam tuam grandem, et dedisti ei filium sedentem super thronum ejus, sicut est hodie.
${}^{7}$~Et nunc Domine Deus, tu regnare fecisti servum tuum pro David patre meo~: ego autem sum puer parvulus, et ignorans egressum et introitum meum.
${}^{8}$~Et servus tuus in medio est populi quem elegisti, populi infiniti, qui numerari et supputari non potest pr\ae\ multitudine.
${}^{9}$~Dabis ergo servo tuo cor docile, ut populum tuum judicare possit, et discernere inter bonum et malum. Quis enim poterit judicare populum istum, populum tuum hunc multum~?


${}^{10}$~Placuit ergo sermo coram Domino, quod Salomon postulasset hujuscemodi rem.
${}^{11}$~Et dixit Dominus Salomoni~: Quia postulasti verbum hoc, et non petisti tibi dies multos, nec divitias, aut animas inimicorum tuorum, sed postulasti tibi sapientiam ad discernendum judicium~:
${}^{12}$~ecce feci tibi secundum sermones tuos, et dedi tibi cor sapiens et intelligens, in tantum ut nullus ante te similis tui fuerit, nec post te surrecturus sit.
${}^{13}$~Sed et h\ae c qu\ae\ non postulasti, dedi tibi~: divitias scilicet, et gloriam, ut nemo fuerit similis tui in regibus cunctis retro diebus.
${}^{14}$~Si autem ambulaveris in viis meis, et custodieris pr\ae cepta mea et mandata mea, sicut ambulavit pater tuus, longos faciam dies tuos.
${}^{15}$~Igitur evigilavit Salomon, et intellexit quod esset somnium~: cumque venisset Jerusalem, stetit coram arca fœderis Domini, et obtulit holocausta, et fecit victimas pacificas, et grande convivium universis famulis suis.


${}^{16}$~Tunc venerunt du\ae\ mulieres meretrices ad regem, steteruntque coram eo~:
${}^{17}$~quarum una ait~: Obsecro, mi domine~: ego et mulier h\ae c habitabamus in domo una, et peperi apud eam in cubiculo.
${}^{18}$~Tertia autem die postquam ego peperi, peperit et h\ae c~: et eramus simul, nullusque alius nobiscum in domo, exceptis nobis duabus.
${}^{19}$~Mortuus est autem filius mulieris hujus nocte~: dormiens quippe oppressit eum.
${}^{20}$~Et consurgens intempest\ae\ noctis silentio, tulit filium meum de latere meo, ancill\ae\ tu\ae\ dormientis, et collocavit in sinu suo~: suum autem filium, qui erat mortuus, posuit in sinu meo.
${}^{21}$~Cumque surrexissem mane ut darem lac filio meo, apparuit mortuus~: quem diligentius intuens clara luce, deprehendi non esse meum quem genueram.
${}^{22}$~Responditque altera mulier~: Non est ita ut dicis, sed filius tuus mortuus est, meus autem vivit. E contrario illa dicebat~: Mentiris~: filius quippe meus vivit, et filius tuus mortuus est. Atque in hunc modum contendebant coram rege.
${}^{23}$~Tunc rex ait~: H\ae c dicit~: Filius meus vivit, et filius tuus mortuus est~: et ista respondit~: Non, sed filius tuus mortuus est, meus autem vivit.
${}^{24}$~Dixit ergo rex~: Afferte mihi gladium. Cumque attulissent gladium coram rege,
${}^{25}$~Dividite, inquit, infantem vivum in duas partes, et date dimidiam partem uni, et dimidiam partem alteri.
${}^{26}$~Dixit autem mulier, cujus filius erat vivus, ad regem (commota sunt quippe viscera ejus super filio suo)~: Obsecro, domine, date illi infantem vivum, et nolite interficere eum. E contrario illa dicebat~: Nec mihi nec tibi sit, sed dividatur.
${}^{27}$~Respondit rex, et ait~: Date huic infantem vivum, et non occidatur~: h\ae c est enim mater ejus.
${}^{28}$~Audivit itaque omnis Isra\"el judicium quod judicasset rex, et timuerunt regem, videntes sapientiam Dei esse in eo ad faciendum judicium.
\Needspace{2.5\baselineskip}\versal{4}~\lettrine[lines=10,image=true,loversize=0.05,lraise=-0.03]{E}{}rat autem rex Salomon regnans super omnem Isra\"el~:
${}^{2}$~et hi principes quos habebat~: Azarias filius Sadoc sacerdotis~:
${}^{3}$~Elihoreph et Ahia filii Sisa scrib\ae~: Josaphat filius Ahilud a commentariis~:
${}^{4}$~Banaias filius Jojad\ae\ super exercitum~: Sadoc autem et Abiathar sacerdotes~:
${}^{5}$~Azarias filius Nathan super eos qui assistebant regi~: Zabud filius Nathan sacerdos, amicus regis~:
${}^{6}$~et Ahisar pr\ae positus domus~: et Adoniram filius Abda super tributa.
${}^{7}$~Habebat autem Salomon duodecim pr\ae fectos super omnem Isra\"el, qui pr\ae bebant annonam regi et domui ejus~: per singulos enim menses in anno, singuli necessaria ministrabant.
${}^{8}$~Et h\ae c nomina eorum~: Benhur in monte Ephraim.
${}^{9}$~Bendecar in Macces, et in Salebim, et in Bethsames, et in Elon, et in Bethanan.
${}^{10}$~Benhesed in Aruboth~: ipsius erat Socho, et omnis terra Epher.
${}^{11}$~Benabinadab, cujus omnis Nephath Dor~: Tapheth filiam Salomonis habebat uxorem.
${}^{12}$~Bana filius Ahilud regebat Thanac et Mageddo, et universam Bethsan, qu\ae\ est juxta Sarthana subter Jezrahel, a Bethsan usque Abelmehula e regione Jecmaan.
${}^{13}$~Bengaber in Ramoth Galaad~: habebat Avothjair filii Manasse in Galaad~: ipse pr\ae erat in omni regione Argob, qu\ae\ est in Basan, sexaginta civitatibus magnis atque muratis qu\ae\ habebant seras \ae reas.
${}^{14}$~Ahinadab filius Addo pr\ae erat in Manaim.
${}^{15}$~Achimaas in Nephthali~: sed et ipse habebat Basemath filiam Salomonis in conjugio.
${}^{16}$~Baana filius Husi in Aser, et in Baloth.
${}^{17}$~Josaphat filius Pharue in Issachar.
${}^{18}$~Semei filius Ela in Benjamin.
${}^{19}$~Gaber filius Uri in terra Galaad, in terra Sehon regis Amorrh\ae i et Og regis Basan, super omnia qu\ae\ erant in illa terra.


${}^{20}$~Juda et Isra\"el innumerabiles, sicut arena maris in multitudine~: comedentes, et bibentes, atque l\ae tantes.
${}^{21}$~Salomon autem erat in ditione sua, habens omnia regna a flumine terr\ae\ Philisthiim usque ad terminum \AE gypti~: offerentium sibi munera, et servientium ei cunctis diebus vit\ae\ ejus.
${}^{22}$~Erat autem cibus Salomonis per dies singulos triginta cori simil\ae , et sexaginta cori farin\ae ,
${}^{23}$~decem boves pingues, et viginti boves pascuales, et centum arietes, excepta venatione cervorum, caprearum, atque bubalorum, et avium altilium.
${}^{24}$~Ipse enim obtinebat omnem regionem qu\ae\ erat trans flumen, a Thaphsa usque ad Gazan, et cunctos reges illarum regionum~: et habebat pacem ex omni parte in circuitu.
${}^{25}$~Habitabatque Juda et Isra\"el absque timore ullo, unusquisque sub vite sua et sub ficu sua, a Dan usque Bersabee, cunctis diebus Salomonis.
${}^{26}$~Et habebat Salomon quadraginta millia pr\ae sepia equorum currilium, et duodecim millia equestrium.
${}^{27}$~Nutriebantque eos supradicti regis pr\ae fecti~: sed et necessaria mens\ae\ regis Salomonis cum ingenti cura pr\ae bebant in tempore suo.
${}^{28}$~Hordeum quoque, et paleas equorum et jumentorum, deferebant in locum ubi erat rex, juxta constitutum sibi.


${}^{29}$~Dedit quoque Deus sapientiam Salomoni, et prudentiam multam nimis, et latitudinem cordis quasi arenam qu\ae\ est in littore maris.
${}^{30}$~Et pr\ae cedebat sapientia Salomonis sapientiam omnium Orientalium et \AE gyptiorum,
${}^{31}$~et erat sapientior cunctis hominibus~: sapientior Ethan Ezrahita, et Heman, et Chalcol, et Dorda filiis Mahol~: et erat nominatus in universis gentibus per circuitum.
${}^{32}$~Locutus est quoque Salomon tria millia parabolas~: et fuerunt carmina ejus quinque et mille.
${}^{33}$~Et disputavit super lignis a cedro qu\ae\ est in Libano, usque ad hyssopum qu\ae\ egreditur de pariete~: et disseruit de jumentis, et volucribus, et reptilibus, et piscibus.
${}^{34}$~Et veniebant de cunctis populis ad audiendam sapientiam Salomonis, et ab universis regibus terr\ae\ qui audiebant sapientiam ejus.
\Needspace{2.5\baselineskip}\versal{5}~\lettrine[lines=10,image=true,loversize=0.05,lraise=-0.03]{M}{}isit quoque Hiram rex Tyri servos suos ad Salomonem~: audivit enim quod ipsum unxissent regem pro patre ejus~: quia amicus fuerat Hiram David omni tempore.
${}^{2}$~Misit autem Salomon ad Hiram, dicens~:
${}^{3}$~Tu scis voluntatem David patris mei, et quia non potuerit \ae dificare domum nomini Domini Dei sui propter bella imminentia per circuitum, donec daret Dominus eos sub vestigio pedum ejus.
${}^{4}$~Nunc autem requiem dedit Dominus Deus meus mihi per circuitum, et non est satan, neque occursus malus.
${}^{5}$~Quam ob rem cogito \ae dificare templum nomini Domini Dei mei, sicut locutus est Dominus David patri meo, dicens~: Filius tuus, quem dabo pro te super solium tuum, ipse \ae dificabit domum nomini meo.
${}^{6}$~Pr\ae cipe igitur ut pr\ae cidant mihi servi tui cedros de Libano, et servi mei sint cum servis tuis~: mercedem autem servorum tuorum dabo tibi quamcumque petieris~: scis enim quomodo non est in populo meo vir qui noverit ligna c\ae dere sicut Sidonii.
${}^{7}$~Cum ergo audisset Hiram verba Salomonis, l\ae tatus est valde, et ait~: Benedictus Dominus Deus hodie, qui dedit David filium sapientissimum super populum hunc plurimum.
${}^{8}$~Et misit Hiram ad Salomonem, dicens~: Audivi qu\ae cumque mandasti mihi~: ego faciam omnem voluntatem tuam in lignis cedrinis et abiegnis.
${}^{9}$~Servi mei deponent ea de Libano ad mare, et ego componam ea in ratibus in mari usque ad locum quem significaveris mihi~: et applicabo ea ibi, et tu tolles ea~: pr\ae bebisque necessaria mihi, ut detur cibus domui me\ae .
${}^{10}$~Itaque Hiram dabat Salomoni ligna cedrina, et ligna abiegna, juxta omnem voluntatem ejus.
${}^{11}$~Salomon autem pr\ae bebat Hiram coros tritici viginti millia in cibum domui ejus, et viginti coros purissimi olei~: h\ae c tribuebat Salomon Hiram per singulos annos.
${}^{12}$~Dedit quoque Dominus sapientiam Salomoni, sicut locutus est ei~: et erat pax inter Hiram et Salomonem, et percusserunt ambo fœdus.


${}^{13}$~Elegitque rex Salomon operarios de omni Isra\"el, et erat indictio triginta millia virorum.
${}^{14}$~Mittebatque eos in Libanum, decem millia per menses singulos vicissim, ita ut duobus mensibus essent in domibus suis~: et Adoniram erat super hujuscemodi indictione.
${}^{15}$~Fueruntque Salomoni septuaginta millia eorum qui onera portabant, et octoginta millia latomorum in monte,
${}^{16}$~absque pr\ae positis qui pr\ae erant singulis operibus, numero trium millium et trecentorum, pr\ae cipientium populo et his qui faciebant opus.
${}^{17}$~Pr\ae cepitque rex ut tollerent lapides grandes, lapides pretiosos in fundamentum templi, et quadrarent eos~:
${}^{18}$~quos dolaverunt c\ae mentarii Salomonis et c\ae mentarii Hiram~: porro Giblii pr\ae paraverunt ligna et lapides ad \ae dificandam domum.
\Needspace{2.5\baselineskip}\versal{6}~\lettrine[lines=10,image=true,loversize=0.05,lraise=-0.03]{F}{}actum est ergo quadringentesimo et octogesimo anno egressionis filiorum Isra\"el de terra \AE gypti, in anno quarto, mense Zio (ipse est mensis secundus), regni Salomonis super Isra\"el, \ae dificari cœpit domus Domino.
${}^{2}$~Domus autem quam \ae dificabat rex Salomon Domino, habebat sexaginta cubitos in longitudine, et viginti cubitos in latitudine, et triginta cubitos in altitudine.
${}^{3}$~Et porticus erat ante templum viginti cubitorum longitudinis, juxta mensuram latitudinis templi~: et habebat decem cubitos latitudinis ante faciem templi.
${}^{4}$~Fecitque in templo fenestras obliquas.
${}^{5}$~Et \ae dificavit super parietem templi tabulata per gyrum, in parietibus domus per circuitum templi et oraculi, et fecit latera in circuitu.
${}^{6}$~Tabulatum quod subter erat, quinque cubitos habebat latitudinis, et medium tabulatum sex cubitorum latitudinis, et tertium tabulatum septem habens cubitos latitudinis. Trabes autem posuit in domo per circuitum forinsecus, ut non h\ae rerent muris templi.
${}^{7}$~Domus autem cum \ae dificaretur, de lapidibus dolatis atque perfectis \ae dificata est~: et malleus, et securis, et omne ferramentum non sunt audita in domo cum \ae dificaretur.
${}^{8}$~Ostium lateris medii in parte erat domus dextr\ae~: et per cochleam ascendebant in medium cœnaculum, et a medio in tertium.
${}^{9}$~Et \ae dificavit domum, et consummavit eam~: texit quoque domum laquearibus cedrinis.
${}^{10}$~Et \ae dificavit tabulatum super omnem domum quinque cubitis altitudinis, et operuit domum lignis cedrinis.
${}^{11}$~Et factus est sermo Domini ad Salomonem, dicens~:
${}^{12}$~Domus h\ae c, quam \ae dificas, si ambulaveris in pr\ae ceptis meis, et judicia mea feceris, et custodieris omnia mandata mea, gradiens per ea, firmabo sermonem meum tibi, quem locutus sum ad David patrem tuum~:
${}^{13}$~et habitabo in medio filiorum Isra\"el, et non derelinquam populum meum Isra\"el.


${}^{14}$~Igitur \ae dificavit Salomon domum, et consummavit eam.
${}^{15}$~Et \ae dificavit parietes domus intrinsecus tabulatis cedrinis~: a pavimento domus usque ad summitatem parietum, et usque ad laquearia, operuit lignis cedrinis intrinsecus~: et texit pavimentum domus tabulis abiegnis.
${}^{16}$~\AE dificavitque viginti cubitorum ad posteriorem partem templi tabulata cedrina, a pavimento usque ad superiora~: et fecit interiorem domum oraculi in Sanctum sanctorum.
${}^{17}$~Porro quadraginta cubitorum erat ipsum templum pro foribus oraculi.
${}^{18}$~Et cedro omnis domus intrinsecus vestiebatur, habens tornaturas et juncturas suas fabrefactas, et c\ae laturas eminentes~: omnia cedrinis tabulis vestiebantur~: nec omnino lapis apparere poterat in pariete.
${}^{19}$~Oraculum autem in medio domus, in interiori parte fecerat, ut poneret ibi arcam fœderis Domini.
${}^{20}$~Porro oraculum habebat viginti cubitos longitudinis, et viginti cubitos latitudinis, et viginti cubitos altitudinis~: et operuit illud atque vestivit auro purissimo~: sed et altare vestivit cedro.
${}^{21}$~Domum quoque ante oraculum operuit auro purissimo, et affixit laminas clavis aureis.
${}^{22}$~Nihilque erat in templo quod non auro tegeretur~: sed et totum altare oraculi texit auro.
${}^{23}$~Et fecit in oraculo duos cherubim de lignis olivarum, decem cubitorum altitudinis.
${}^{24}$~Quinque cubitorum ala cherub una, et quinque cubitorum ala cherub altera~: id est, decem cubitos habentes, a summitate al\ae\ unius usque ad al\ae\ alterius summitatem.
${}^{25}$~Decem quoque cubitorum erat cherub secundus~: in mensura pari, et opus unum erat in duobus cherubim,
${}^{26}$~id est, altitudinem habebat unus cherub decem cubitorum, et similiter cherub secundus.
${}^{27}$~Posuitque cherubim in medio templi interioris~: extendebant autem alas suas cherubim, et tangebat ala una parietem, et ala cherub secundi tangebat parietem alterum~: al\ae\ autem alter\ae\ in media parte templi se invicem contingebant.
${}^{28}$~Texit quoque cherubim auro.
${}^{29}$~Et omnes parietes templi per circuitum sculpsit variis c\ae laturis et torno~: et fecit in eis cherubim, et palmas, et picturas varias, quasi prominentes de pariete, et egredientes.
${}^{30}$~Sed et pavimentum domus texit auro intrinsecus et extrinsecus.
${}^{31}$~Et in ingressu oraculi fecit ostiola de lignis olivarum, postesque angulorum quinque.
${}^{32}$~Et duo ostia de lignis olivarum~: et sculpsit in eis picturam cherubim, et palmarum species, et anaglypha valde prominentia~: et texit ea auro, et operuit tam cherubim quam palmas, et cetera, auro.
${}^{33}$~Fecitque in introitu templi postes de lignis olivarum quadrangulatos,
${}^{34}$~et duo ostia de lignis abiegnis altrinsecus~: et utrumque ostium duplex erat, et se invicem tenens aperiebatur.
${}^{35}$~Et sculpsit cherubim, et palmas, et c\ae laturas valde eminentes~: operuitque omnia laminis aureis opere quadro ad regulam.
${}^{36}$~Et \ae dificavit atrium interius tribus ordinibus lapidum politorum, et uno ordine lignorum cedri.
${}^{37}$~Anno quarto fundata est domus Domini in mense Zio~:
${}^{38}$~et in anno undecimo, mense Bul (ipse est mensis octavus), perfecta est domus in omni opere suo, et in universis utensilibus suis~: \ae dificavitque eam annis septem.
\Needspace{2.5\baselineskip}\versal{7}~\lettrine[lines=10,image=true,loversize=0.05,lraise=-0.03]{D}{}omum autem suam \ae dificavit Salomon tredecim annis, et ad perfectum usque perduxit.
${}^{2}$~\AE dificavit quoque domum saltus Libani centum cubitorum longitudinis, et quinquaginta cubitorum latitudinis, et triginta cubitorum altitudinis~: et quatuor deambulacra inter columnas cedrinas~: ligna quippe cedrina exciderat in columnas.
${}^{3}$~Et tabulatis cedrinis vestivit totam cameram, qu\ae\ quadraginta quinque columnis sustentabatur. Unus autem ordo habebat columnas quindecim
${}^{4}$~contra se invicem positas,
${}^{5}$~et e regione se respicientes, \ae quali spatio inter columnas, et super columnas quadrangulata ligna in cunctis \ae qualia.
${}^{6}$~Et porticum columnarum fecit quinquaginta cubitorum longitudinis, et triginta cubitorum latitudinis~: et alteram porticum in facie majoris porticus~: et columnas, et epistylia super columnas.
${}^{7}$~Porticum quoque solii, in qua tribunal est, fecit~: et texit lignis cedrinis a pavimento usque ad summitatem.
${}^{8}$~Et domuncula, in qua sedebatur ad judicandum, erat in media porticu simili opere. Domum quoque fecit fili\ae\ Pharaonis (quam uxorem duxerat Salomon) tali opere, quali et hanc porticum.
${}^{9}$~Omnia lapidibus pretiosis, qui ad normam quamdam atque mensuram tam intrinsecus quam extrinsecus serrati erant~: a fundamento usque ad summitatem parietum, et extrinsecus usque ad atrium majus.
${}^{10}$~Fundamenta autem de lapidibus pretiosis, lapidibus magnis, decem sive octo cubitorum.
${}^{11}$~Et desuper lapides pretiosi \ae qualis mensur\ae\ secti erant, similiterque de cedro.
${}^{12}$~Et atrium majus rotundum trium ordinum de lapidibus sectis, et unius ordinis de dolata cedro~: necnon et in atrio domus Domini interiori, et in porticu domus.


${}^{13}$~Misit quoque rex Salomon, et tulit Hiram de Tyro,
${}^{14}$~filium mulieris vidu\ae\ de tribu Nephthali, patre Tyrio, artificem \ae rarium, et plenum sapientia, et intelligentia, et doctrina, ad faciendum omne opus ex \ae re. Qui cum venisset ad regem Salomonem, fecit omne opus ejus.
${}^{15}$~Et finxit duas columnas \ae reas, decem et octo cubitorum altitudinis columnam unam~: et linea duodecim cubitorum ambiebat columnam utramque.
${}^{16}$~Duo quoque capitella fecit, qu\ae\ ponerentur super capita columnarum, fusilia ex \ae re~: quinque cubitorum altitudinis capitellum unum, et quinque cubitorum altitudinis capitellum alterum~:
${}^{17}$~et quasi in modum retis, et catenarum sibi invicem miro opere contextarum. Utrumque capitellum columnarum fusile erat~: septena versuum retiacula in capitello uno, et septena retiacula in capitello altero.
${}^{18}$~Et perfecit columnas, et duos ordines per circuitum retiaculorum singulorum, ut tegerent capitella qu\ae\ erant super summitatem, malogranatorum~: eodem modo fecit et capitello secundo.
${}^{19}$~Capitella autem qu\ae\ erant super capita columnarum, quasi opere lilii fabricata erant in porticu quatuor cubitorum.
${}^{20}$~Et rursum alia capitella in summitate columnarum desuper juxta mensuram column\ae\ contra retiacula~: malogranatorum autem ducenti ordines erant in circuitu capitelli secundi.
${}^{21}$~Et statuit duas columnas in porticu templi~: cumque statuisset columnam dexteram, vocavit eam nomine Jachin~: similiter erexit columnam secundam, et vocavit nomen ejus Booz.
${}^{22}$~Et super capita columnarum opus in modum lilii posuit~: perfectumque est opus columnarum.
${}^{23}$~Fecit quoque mare fusile decem cubitorum a labio usque ad labium, rotundum in circuitu~: quinque cubitorum altitudo ejus, et resticula triginta cubitorum cingebat illud per circuitum.
${}^{24}$~Et sculptura subter labium circuibat illud decem cubitis ambiens mare~: duo ordines sculpturarum striatarum erant fusiles.
${}^{25}$~Et stabat super duodecim boves, e quibus tres respiciebant ad aquilonem, et tres ad occidentem, et tres ad meridiem, et tres ad orientem~: et mare super eos desuper erat~: quorum posteriora universa intrinsecus latitabant.
${}^{26}$~Grossitudo autem luteris, trium unciarum erat~: labiumque ejus quasi labium calicis, et folium repandi lilii~: duo millia batos capiebat.
${}^{27}$~Et fecit decem bases \ae neas, quatuor cubitorum longitudinis bases singulas, et quatuor cubitorum latitudinis, et trium cubitorum altitudinis.
${}^{28}$~Et ipsum opus basium, interrasile erat~: et sculptur\ae\ inter juncturas.
${}^{29}$~Et inter coronulas et plectas, leones et boves et cherubim, et in juncturis similiter desuper~: et subter leones et boves, quasi lora ex \ae re dependentia.
${}^{30}$~Et quatuor rot\ae\ per bases singulas, et axes \ae rei~: et per quatuor partes quasi humeruli subter luterem fusiles, contra se invicem respectantes.
${}^{31}$~Os quoque luteris intrinsecus erat in capitis summitate~: et quod forinsecus apparebat, unius cubiti erat totum rotundum, pariterque habebat unum cubitum et dimidium~: in angulis autem columnarum vari\ae\ c\ae latur\ae\ erant~: et media intercolumnia, quadrata non rotunda.
${}^{32}$~Quatuor quoque rot\ae\ qu\ae\ per quatuor angulos basis erant, coh\ae rebant sibi subter basim~: una rota habebat altitudinis cubitum et semis.
${}^{33}$~Tales autem rot\ae\ erant quales solent in curru fieri~: et axes earum, et radii, et canthi, et modioli, omnia fusilia.
${}^{34}$~Nam et humeruli illi quatuor per singulos angulos basis unius, ex ipsa basi fusiles et conjuncti erant.
${}^{35}$~In summitate autem basis erat qu\ae dam rotunditas dimidii cubiti, ita fabrefacta ut luter desuper posset imponi, habens c\ae laturas suas, variasque sculpturas ex semetipsa.
${}^{36}$~Sculpsit quoque in tabulatis illis qu\ae\ erant ex \ae re, et in angulis, cherubim, et leones, et palmas, quasi in similitudinem hominis stantis, ut non c\ae lata, sed apposita per circuitum viderentur.
${}^{37}$~In hunc modum fecit decem bases, fusura una, et mensura, sculpturaque consimili.
${}^{38}$~Fecit quoque decem luteres \ae neos~: quadraginta batos capiebat luter unus, eratque quatuor cubitorum~: singulos quoque luteres per singulas, id est, decem bases, posuit.
${}^{39}$~Et constituit decem bases, quinque ad dexteram partem templi, et quinque ad sinistram~: mare autem posuit ad dexteram partem templi contra orientem ad meridiem.
${}^{40}$~Fecit ergo Hiram lebetes, et scutras, et hamulas, et perfecit omne opus regis Salomonis in templo Domini.
${}^{41}$~Columnas duas, et funiculos capitellorum super capitella columnarum duos~: et retiacula duo, ut operirent duos funiculos qui erant super capita columnarum.
${}^{42}$~Et malogranata quadringenta in duobus retiaculis~: duos versus malogranatorum in retiaculis singulis, ad operiendos funiculos capitellorum qui erant super capita columnarum.
${}^{43}$~Et bases decem, et luteres decem super bases.
${}^{44}$~Et mare unum, et boves duodecim subter mare.
${}^{45}$~Et lebetes, et scutras, et hamulas, omnia vasa qu\ae\ fecit Hiram regi Salomoni in domo Domini, de auricalco erant.
${}^{46}$~In campestri regione Jordanis fudit ea rex in argillosa terra, inter Sochoth et Sarthan.
${}^{47}$~Et posuit Salomon omnia vasa~: propter multitudinem autem nimiam non erat pondus \ae ris.


${}^{48}$~Fecitque Salomon omnia vasa in domo Domini~: altare aureum, et mensam super quam ponerentur panes propositionis, auream~:
${}^{49}$~et candelabra aurea, quinque ad dexteram, et quinque ad sinistram contra oraculum, ex auro puro~: et quasi lilii flores, et lucernas desuper aureas~: et forcipes aureos,
${}^{50}$~et hydrias, et fuscinulas, et phialas, et mortariola, et thuribula, de auro purissimo~: et cardines ostiorum domus interioris Sancti sanctorum, et ostiorum domus templi, ex auro erant.
${}^{51}$~Et perfecit omne opus quod faciebat Salomon in domo Domini, et intulit qu\ae\ sanctificaverat David pater suus, argentum, et aurum, et vasa, reposuitque in thesauris domus Domini.
\Needspace{2.5\baselineskip}\versal{8}~\lettrine[lines=10,image=true,loversize=0.05,lraise=-0.03]{T}{}unc congregati sunt omnes majores natu Isra\"el cum principibus tribuum, et duces familiarum filiorum Isra\"el, ad regem Salomonem in Jerusalem, ut deferrent arcam fœderis Domini de civitate David, id est, de Sion.
${}^{2}$~Convenitque ad regem Salomonem universus Isra\"el in mense Ethanim, in solemni die~: ipse est mensis septimus.
${}^{3}$~Veneruntque cuncti senes de Isra\"el, et tulerunt arcam sacerdotes,
${}^{4}$~et portaverunt arcam Domini, et tabernaculum fœderis, et omnia vasa sanctuarii qu\ae\ erant in tabernaculo~: et ferebant ea sacerdotes et Levit\ae .
${}^{5}$~Rex autem Salomon, et omnis multitudo Isra\"el qu\ae\ convenerat ad eum, gradiebatur cum illo ante arcam, et immolabant oves et boves absque \ae stimatione et numero.
${}^{6}$~Et intulerunt sacerdotes arcam fœderis Domini in locum suum, in oraculum templi, in Sanctum sanctorum, subter alas cherubim.
${}^{7}$~Siquidem cherubim expandebant alas super locum arc\ae , et protegebant arcam, et vectes ejus desuper.
${}^{8}$~Cumque eminerent vectes, et apparerent summitates eorum foris sanctuarium ante oraculum, non apparebant ultra extrinsecus, qui et fuerunt ibi usque in pr\ae sentem diem.
${}^{9}$~In arca autem non erat aliud nisi du\ae\ tabul\ae\ lapide\ae\ quas posuerat in ea Moyses in Horeb, quando pepigit Dominus fœdus cum filiis Isra\"el, cum egrederentur de terra \AE gypti.
${}^{10}$~Factum est autem cum exissent sacerdotes de sanctuario, nebula implevit domum Domini,
${}^{11}$~et non poterant sacerdotes stare et ministrare propter nebulam~: impleverat enim gloria Domini domum Domini.
${}^{12}$~Tunc ait Salomon~: Dominus dixit ut habitaret in nebula.
${}^{13}$~\AE dificans \ae dificavi domum in habitaculum tuum~: firmissimum solium tuum in sempiternum.
${}^{14}$~Convertitque rex faciem suam, et benedixit omni ecclesi\ae\ Isra\"el~: omnia enim ecclesia Isra\"el stabat.
${}^{15}$~Et ait Salomon~: Benedictus Dominus Deus Isra\"el, qui locutus est ore suo ad David patrem meum, et in manibus ejus perfecit, dicens~:
${}^{16}$~A die qua eduxi populum meum Isra\"el de \AE gypto, non elegi civitatem de universis tribubus Isra\"el, ut \ae dificaretur domus, et esset nomen meum ibi~: sed elegi David ut esset super populum meum Isra\"el.
${}^{17}$~Voluitque David pater meus \ae dificare domum nomini Domini Dei Isra\"el~:
${}^{18}$~et ait Dominus ad David patrem meum~: Quod cogitasti in corde tuo \ae dificare domum nomini meo, bene fecisti, hoc ipsum mente tractans.
${}^{19}$~Verumtamen tu non \ae dificabis mihi domum, sed filius tuus, qui egredietur de renibus tuis, ipse \ae dificabit domum nomini meo.
${}^{20}$~Confirmavit Dominus sermonem suum quem locutus est~: stetique pro David patre meo, et sedi super thronum Isra\"el, sicut locutus est Dominus~: et \ae dificavi domum nomini Domini Dei Isra\"el.
${}^{21}$~Et constitui ibi locum arc\ae\ in qua fœdus Domini est, quod percussit cum patribus nostris quando egressi sunt de terra \AE gypti.


${}^{22}$~Stetit autem Salomon ante altare Domini in conspectu ecclesi\ae\ Isra\"el, et expandit manus suas in c\ae lum,
${}^{23}$~et ait~: Domine Deus Isra\"el, non est similis tui deus in c\ae lo desuper, et super terram deorsum~: qui custodis pactum et misericordiam servis tuis qui ambulant coram te in toto corde suo.
${}^{24}$~Qui custodisti servo tuo David patri meo qu\ae\ locutus es ei~: ore locutus es, et manibus perfecisti, ut h\ae c dies probat.
${}^{25}$~Nunc igitur Domine Deus Isra\"el, conserva famulo tuo David patri meo qu\ae\ locutus es ei, dicens~: Non auferetur de te vir coram me, qui sedeat super thronum Isra\"el~: ita tamen si custodierint filii tui viam suam, ut ambulent coram me sicut tu ambulasti in conspectu meo.
${}^{26}$~Et nunc Domine Deus Isra\"el, firmentur verba tua qu\ae\ locutus es servo tuo David patri meo.
${}^{27}$~Ergone putandum est quod vere Deus habitet super terram~? si enim c\ae lum, et c\ae li c\ae lorum, te capere non possunt, quanto magis domus h\ae c, quam \ae dificavi~?
${}^{28}$~Sed respice ad orationem servi tui, et ad preces ejus, Domine Deus meus~: audi hymnum et orationem quam servus tuus orat coram te hodie~:
${}^{29}$~ut sint oculi tui aperti super domum hanc nocte ac die~: super domum, de qua dixisti~: Erit nomen meum ibi~: ut exaudias orationem quam orat in loco isto ad te servus tuus~:
${}^{30}$~ut exaudias deprecationem servi tui et populi tui Isra\"el, quodcumque oraverint in loco isto, et exaudies in loco habitaculi tui in c\ae lo~: et cum exaudieris, propitius eris.


${}^{31}$~Si peccaverit homo in proximum suum, et habuerit aliquod juramentum quo teneatur astrictus, et venerit propter juramentum coram altari tuo in domum tuam,
${}^{32}$~tu exaudies in c\ae lo~: et facies, et judicabis servos tuos, condemnans impium, et reddens viam suam super caput ejus, justificansque justum, et retribuens ei secundum justitiam suam.
${}^{33}$~Si fugerit populus tuus Isra\"el inimicos suos (quia peccaturus est tibi), et agentes pœnitentiam, et confitentes nomini tuo, venerint, et oraverint, et deprecati te fuerint in domo hac~:
${}^{34}$~exaudi in c\ae lo, et dimitte peccatum populi tui Isra\"el, et reduces eos in terram quam dedisti patribus eorum.
${}^{35}$~Si clausum fuerit c\ae lum, et non pluerit propter peccata eorum, et orantes in loco isto, pœnitentiam egerint nomini tuo, et a peccatis suis conversi fuerint propter afflictionem suam~:
${}^{36}$~exaudi eos in c\ae lo, et dimitte peccata servorum tuorum, et populi tui Isra\"el~: et ostende eis viam bonam per quam ambulent, et da pluviam super terram tuam, quam dedisti populo tuo in possessionem.
${}^{37}$~Fames si oborta fuerit in terra, aut pestilentia, aut corruptus a\"er, aut \ae rugo, aut locusta, vel rubigo, et afflixerit eum inimicus ejus portas obsidens~: omnis plaga, universa infirmitas,
${}^{38}$~cuncta devotatio, et imprecatio qu\ae\ acciderit omni homini de populo tuo Isra\"el~: si quis cognoverit plagam cordis sui, et expanderit manus suas in domo hac,
${}^{39}$~tu exaudies in c\ae lo in loco habitationis tu\ae , et repropitiaberis, et facies ut des unicuique secundum omnes vias suas, sicut videris cor ejus (quia tu nosti solus cor omnium filiorum hominum),
${}^{40}$~ut timeant te cunctis diebus quibus vivunt super faciem terr\ae\ quam dedisti patribus nostris.


${}^{41}$~Insuper et alienigena, qui non est de populo tuo Isra\"el, cum venerit de terra longinqua propter nomen tuum (audietur enim nomen tuum magnum, et manus tua fortis, et brachium tuum
${}^{42}$~extentum ubique), cum venerit ergo, et oraverit in hoc loco,
${}^{43}$~tu exaudies in c\ae lo, in firmamento habitaculi tui, et facies omnia pro quibus invocaverit te alienigena~: ut discant universi populi terrarum nomen tuum timere, sicut populus tuus Isra\"el, et probent quia nomen tuum invocatum est super domum hanc quam \ae dificavi.
${}^{44}$~Si egressus fuerit populus tuus ad bellum contra inimicos suos per viam, quocumque miseris eos, orabunt te contra viam civitatis quam elegisti, et contra domum quam \ae dificavi nomini tuo,
${}^{45}$~et exaudies in c\ae lo orationes eorum et preces eorum, et facies judicium eorum.


${}^{46}$~Quod si peccaverint tibi (non est enim homo qui non peccet) et iratus tradideris eos inimicis suis, et captivi ducti fuerint in terram inimicorum longe vel prope,
${}^{47}$~et egerint pœnitentiam in corde suo in loco captivitatis, et conversi deprecati te fuerint in captivitate sua, dicentes~: Peccavimus~: inique egimus, impie gessimus~:
${}^{48}$~et reversi fuerint ad te in universo corde suo et tota anima sua in terra inimicorum suorum, ad quam captivi ducti fuerint~: et oraverint te contra viam terr\ae\ su\ae , quam dedisti patribus eorum, et civitatis quam elegisti, et templi quod \ae dificavi nomini tuo~:
${}^{49}$~exaudies in c\ae lo, in firmamento solii tui, orationes eorum et preces eorum, et facies judicium eorum~:
${}^{50}$~et propitiaberis populo tuo qui peccavit tibi, et omnibus iniquitatibus eorum quibus pr\ae varicati sunt in te~: et dabis misericordiam coram eis qui eos captivos habuerint, ut misereantur eis.
${}^{51}$~Populus enim tuus est, et h\ae reditas tua, quos eduxisti de terra \AE gypti, de medio fornacis ferre\ae .
${}^{52}$~Ut sint oculi tui aperti ad deprecationem servi tui, et populi tui Isra\"el, et exaudias eos in universis pro quibus invocaverint te.
${}^{53}$~Tu enim separasti eos tibi in h\ae reditatem de universis populis terr\ae , sicut locutus es per Moysen servum tuum quando eduxisti patres nostros de \AE gypto, Domine Deus.


${}^{54}$~Factum est autem, cum complesset Salomon orans Dominum omnem orationem et deprecationem hanc, surrexit de conspectu altaris Domini~: utrumque enim genu in terram fixerat, et manus expanderat in c\ae lum.
${}^{55}$~Stetit ergo, et benedixit omni ecclesi\ae\ Isra\"el voce magna, dicens~:
${}^{56}$~Benedictus Dominus, qui dedit requiem populo suo Isra\"el, juxta omnia qu\ae\ locutus est~: non cecidit ne unus quidem sermo ex omnibus bonis qu\ae\ locutus est per Moysen servum suum.
${}^{57}$~Sit Dominus Deus noster nobiscum, sicut fuit cum patribus nostris, non derelinquens nos, neque projiciens.
${}^{58}$~Sed inclinet corda nostra ad se, ut ambulemus in universis viis ejus, et custodiamus mandata ejus, et c\ae remonias ejus, et judicia qu\ae cumque mandavit patribus nostris.
${}^{59}$~Et sint sermones mei isti, quibus deprecatus sum coram Domino, appropinquantes Domino Deo nostro die ac nocte, ut faciat judicium servo suo, et populo suo Isra\"el per singulos dies~:
${}^{60}$~ut sciant omnes populi terr\ae\ quia Dominus ipse est Deus, et non est ultra absque eo.
${}^{61}$~Sit quoque cor nostrum perfectum cum Domino Deo nostro, ut ambulemus in decretis ejus, et custodiamus mandata ejus, sicut et hodie.


${}^{62}$~Igitur rex, et omnis Isra\"el cum eo, immolabant victimas coram Domino.
${}^{63}$~Mactavitque Salomon hostias pacificas, quas immolavit Domino, boum viginti duo millia, et ovium centum viginti millia~: et dedicaverunt templum Domini rex et filii Isra\"el.
${}^{64}$~In die illa sanctificavit rex medium atrii quod erat ante domum Domini~: fecit quippe holocaustum ibi, et sacrificium, et adipem pacificorum~: quoniam altare \ae reum quod erat coram Domino, minus erat, et capere non poterat holocaustum, et sacrificium, et adipem pacificorum.
${}^{65}$~Fecit ergo Salomon in tempore illo festivitatem celebrem, et omnis Isra\"el cum eo, multitudo magna ab introitu Emath usque ad rivum \AE gypti, coram Domino Deo nostro, septem diebus et septem diebus, id est, quatuordecim diebus.
${}^{66}$~Et in die octava dimisit populos~: qui benedicentes regi, profecti sunt in tabernacula sua l\ae tantes, et alacri corde super omnibus bonis qu\ae\ fecerat Dominus David servo suo, et Isra\"el populo suo.
\Needspace{2.5\baselineskip}\versal{9}~\lettrine[lines=10,image=true,loversize=0.05,lraise=-0.03]{F}{}actum est autem cum perfecisset Salomon \ae dificium domus Domini, et \ae dificium regis, et omne quod optaverat et voluerat facere,
${}^{2}$~apparuit ei Dominus secundo, sicut apparuerat ei in Gabaon.
${}^{3}$~Dixitque Dominus ad eum~: Exaudivi orationem tuam et deprecationem tuam, quam deprecatus es coram me~: sanctificavi domum hanc quam \ae dificasti, ut ponerem nomen meum ibi in sempiternum, et erunt oculi mei et cor meum ibi cunctis diebus.
${}^{4}$~Tu quoque si ambulaveris coram me sicut ambulavit pater tuus, in simplicitate cordis et in \ae quitate, et feceris omnia qu\ae\ pr\ae cepi tibi, et legitima mea et judicia mea servaveris,
${}^{5}$~ponam thronum regni tui super Isra\"el in sempiternum, sicut locutus sum David patri tuo, dicens~: Non auferetur vir de genere tuo de solio Isra\"el.
${}^{6}$~Si autem aversione aversi fueritis vos et filii vestri, non sequentes me, nec custodientes mandata mea et c\ae remonias meas quas proposui vobis, sed abieritis et colueritis deos alienos, et adoraveritis eos~:
${}^{7}$~auferam Isra\"el de superficie terr\ae\ quam dedi eis, et templum quod sanctificavi nomini meo, projiciam a conspectu meo~: eritque Isra\"el in proverbium, et in fabulam cunctis populis.
${}^{8}$~Et domus h\ae c erit in exemplum~: omnis qui transierit per eam, stupebit, et sibilabit, et dicet~: Quare fecit Dominus sic terr\ae\ huic, et domui huic~?
${}^{9}$~Et respondebunt~: Quia dereliquerunt Dominum Deum suum, qui eduxit patres eorum de terra \AE gypti, et secuti sunt deos alienos, et adoraverunt eos, et coluerunt eos~: idcirco induxit Dominus super eos omne malum hoc.


${}^{10}$~Expletis autem annis viginti postquam \ae dificaverat Salomon duas domos, id est, domum Domini, et domum regis
${}^{11}$~(Hiram rege Tyri pr\ae bente Salomoni ligna cedrina et abiegna, et aurum juxta omne quod opus habuerat), tunc dedit Salomon Hiram viginti oppida in terra Galil\ae \ae .
${}^{12}$~Et egressus est Hiram de Tyro ut videret oppida qu\ae\ dederat ei Salomon, et non placuerunt ei.
${}^{13}$~Et ait~: H\ae ccine sunt civitates quas dedisti mihi, frater~? Et appellavit eas terram Chabul, usque in diem hanc.
${}^{14}$~Misit quoque Hiram ad regem Salomonem centum viginti talenta auri.
${}^{15}$~H\ae c est summa expensarum quam obtulit rex Salomon ad \ae dificandam domum Domini et domum suam, et Mello, et murum Jerusalem, et Heser, et Mageddo, et Gazer.
${}^{16}$~Pharao rex \AE gypti ascendit, et cepit Gazar, succenditque eam igni, et Chanan\ae um, qui habitabat in civitate, interfecit~: et dedit eam in dotem fili\ae\ su\ae\ uxori Salomonis.
${}^{17}$~\AE dificavit ergo Salomon Gazer, et Bethoron inferiorem,
${}^{18}$~et Balaath, et Palmiram in terra solitudinis.
${}^{19}$~Et omnes vicos qui ad se pertinebant et erant absque muro, munivit, et civitates curruum et civitates equitum, et quodcumque ei placuit ut \ae dificaret in Jerusalem, et in Libano, et in omni terra potestatis su\ae .
${}^{20}$~Universum populum qui remanserat de Amorrh\ae is, et Heth\ae is, et Pherez\ae is, et Hev\ae is, et Jebus\ae is, qui non sunt de filiis Isra\"el~:
${}^{21}$~horum filios qui remanserant in terra, quos scilicet non potuerant filii Isra\"el exterminare, fecit Salomon tributarios usque in diem hanc.
${}^{22}$~De filiis autem Isra\"el non constituit Salomon servire quemquam, sed erant viri bellatores, et ministri ejus, et principes, et duces, et pr\ae fecti curruum et equorum.
${}^{23}$~Erant autem principes super omnia opera Salomonis pr\ae positi quingenti quinquaginta, qui habebant subjectum populum, et statutis operibus imperabant.
${}^{24}$~Filia autem Pharaonis ascendit de civitate David in domum suam, quam \ae dificaverat ei Salomon~: tunc \ae dificavit Mello.
${}^{25}$~Offerebat quoque Salomon, tribus vicibus per annos singulos, holocausta et pacificas victimas super altare quod \ae dificaverat Domino, et adolebat thymiama coram Domino~: perfectumque est templum.
${}^{26}$~Classem quoque fecit rex Salomon in Asiongaber, qu\ae\ est juxta Ailath in littore maris Rubri, in terra Idum\ae \ae .
${}^{27}$~Misitque Hiram in classe illa servos suos viros nauticos et gnaros maris, cum servis Salomonis.
${}^{28}$~Qui cum venissent in Ophir, sumptum inde aurum quadringentorum viginti talentorum, detulerunt ad regem Salomonem.
\Needspace{2.5\baselineskip}\versal{10}~\lettrine[lines=10,image=true,loversize=0.05,lraise=-0.03]{S}{}ed et regina Saba, audita fama Salomonis in nomine Domini, venit tentare eum in \ae nigmatibus.
${}^{2}$~Et ingressa Jerusalem multo cum comitatu et divitiis, camelis portantibus aromata, et aurum infinitum nimis, et gemmas pretiosas, venit ad regem Salomonem, et locuta est ei universa qu\ae\ habebat in corde suo.
${}^{3}$~Et docuit eam Salomon omnia verba qu\ae\ proposuerat~: non fuit sermo qui regem posset latere, et non responderet ei.
${}^{4}$~Videns autem regina Saba omnem sapientiam Salomonis, et domum quam \ae dificaverat,
${}^{5}$~et cibos mens\ae\ ejus, et habitacula servorum, et ordines ministrantium, vestesque eorum, et pincernas, et holocausta qu\ae\ offerebat in domo Domini~: non habebat ultra spiritum.
${}^{6}$~Dixitque ad regem~: Verus est sermo quem audivi in terra mea
${}^{7}$~super sermonibus tuis, et super sapientia tua~: et non credebam narrantibus mihi, donec ipsa veni, et vidi oculis meis, et probavi quod media pars mihi nuntiata non fuerit~: major est sapientia et opera tua, quam rumor quem audivi.
${}^{8}$~Beati viri tui, et beati servi tui, qui stant coram te semper, et audiunt sapientiam tuam.
${}^{9}$~Sit Dominus Deus tuus benedictus, cui complacuisti, et posuit te super thronum Isra\"el, eo quod dilexerit Dominus Isra\"el in sempiternum, et constituit te regem ut faceres judicium et justitiam.
${}^{10}$~Dedit ergo regi centum viginti talenta auri, et aromata multa nimis, et gemmas pretiosas~: non sunt allata ultra aromata tam multa, quam ea qu\ae\ dedit regina Saba regi Salomoni.


${}^{11}$~(Sed et classis Hiram, qu\ae\ portabat aurum de Ophir, attulit ex Ophir ligna thyina multa nimis, et gemmas pretiosas.
${}^{12}$~Fecitque rex de lignis thyinis fulcra domus Domini et domus regi\ae , et citharas lyrasque cantoribus~: non sunt allata hujuscemodi ligna thyina, neque visa usque in pr\ae sentem diem.)
${}^{13}$~Rex autem Salomon dedit regin\ae\ Saba omnia qu\ae\ voluit et petivit ab eo, exceptis his qu\ae\ ultro obtulerat ei munere regio. Qu\ae\ reversa est, et abiit in terram suam cum servis suis.
${}^{14}$~Erat autem pondus auri quod afferebatur Salomoni per annos singulos, sexcentorum sexaginta sex talentorum auri,
${}^{15}$~excepto eo quod afferebant viri qui super vectigalia erant, et negotiatores, universique scruta vendentes, et omnes reges Arabi\ae , ducesque terr\ae .
${}^{16}$~Fecit quoque rex Salomon ducenta scuta de auro purissimo~: sexcentos auri siclos dedit in laminas scuti unius.
${}^{17}$~Et trecentas peltas ex auro probato~: trecent\ae\ min\ae\ auri unam peltam vestiebant~: posuitque eas rex in domo saltus Libani.
${}^{18}$~Fecit etiam rex Salomon thronum de ebore grandem~: et vestivit eum auro fulvo nimis,
${}^{19}$~qui habebat sex gradus~: et summitas throni rotunda erat in parte posteriori~: et du\ae\ manus hinc atque inde tenentes sedile~: et duo leones stabant juxta manus singulas.
${}^{20}$~Et duodecim leunculi stantes super sex gradus hinc atque inde~: non est factum tale opus in universis regnis.
${}^{21}$~Sed et omnia vasa quibus potabat rex Salomon, erant aurea~: et universa supellex domus saltus Libani de auro purissimo~: non erat argentum, nec alicujus pretii putabatur in diebus Salomonis,
${}^{22}$~quia classis regis per mare cum classe Hiram semel per tres annos ibat in Tharsis, deferens inde aurum, et argentum, et dentes elephantorum, et simias, et pavos.


${}^{23}$~Magnificatus est ergo rex Salomon super omnes reges terr\ae\ divitiis et sapientia.
${}^{24}$~Et universa terra desiderabat vultum Salomonis, ut audiret sapientiam ejus, quam dederat Deus in corde ejus.
${}^{25}$~Et singuli deferebant ei munera, vasa argentea et aurea, vestes et arma bellica, aromata quoque, et equos et mulos per annos singulos.
${}^{26}$~Congregavitque Salomon currus et equites, et facti sunt ei mille quadringenti currus, et duodecim millia equitum~: et disposuit eos per civitates munitas, et cum rege in Jerusalem.
${}^{27}$~Fecitque ut tanta esset abundantia argenti in Jerusalem, quanta et lapidum~: et cedrorum pr\ae buit multitudinem quasi sycomoros qu\ae\ nascuntur in campestribus.
${}^{28}$~Et educebantur equi Salomoni de \AE gypto, et de Coa. Negotiatores enim regis emebant de Coa, et statuto pretio perducebant.
${}^{29}$~Egrediebatur autem quadriga ex \AE gypto sexcentis siclis argenti, et equus centum quinquaginta. Atque in hunc modum cuncti reges Heth\ae orum et Syri\ae\ equos venundabant.
\Needspace{2.5\baselineskip}\versal{11}~\lettrine[lines=10,image=true,loversize=0.05,lraise=-0.03]{R}{}ex autem Salomon adamavit mulieres alienigenas multas, filiam quoque Pharaonis, et Moabitidas, et Ammonitidas, Idum\ae as, et Sidonias, et Heth\ae as~:
${}^{2}$~de gentibus super quibus dixit Dominus filiis Isra\"el~: Non ingrediemini ad eas, neque de illis ingredientur ad vestras~: certissime enim avertent corda vestra ut sequamini deos earum. His itaque copulatus est Salomon ardentissimo amore.
${}^{3}$~Fueruntque ei uxores quasi regin\ae\ septingent\ae , et concubin\ae\ trecent\ae~: et averterunt mulieres cor ejus.
${}^{4}$~Cumque jam esset senex, depravatum est cor ejus per mulieres, ut sequeretur deos alienos~: nec erat cor ejus perfectum cum Domino Deo suo, sicut cor David patris ejus.
${}^{5}$~Sed colebat Salomon Astarthen deam Sidoniorum, et Moloch idolum Ammonitarum.
${}^{6}$~Fecitque Salomon quod non placuerat coram Domino, et non adimplevit ut sequeretur Dominum sicut David pater ejus.
${}^{7}$~Tunc \ae dificavit Salomon fanum Chamos idolo Moab in monte qui est contra Jerusalem, et Moloch idolo filiorum Ammon.
${}^{8}$~Atque in hunc modum fecit universis uxoribus suis alienigenis, qu\ae\ adolebant thura, et immolabant diis suis.
${}^{9}$~Igitur iratus est Dominus Salomoni, quod aversa esset mens ejus a Domino Deo Isra\"el, qui apparuerat ei secundo,
${}^{10}$~et pr\ae ceperat de verbo hoc ne sequeretur deos alienos~: et non custodivit qu\ae\ mandavit ei Dominus.
${}^{11}$~Dixit itaque Dominus Salomoni~: Quia habuisti hoc apud te, et non custodisti pactum meum, et pr\ae cepta mea qu\ae\ mandavi tibi, disrumpens scindam regnum tuum, et dabo illud servo tuo.
${}^{12}$~Verumtamen in diebus tuis non faciam propter David patrem tuum~: de manu filii tui scindam illud,
${}^{13}$~nec totum regnum auferam, sed tribum unam dabo filio tuo propter David servum meum, et Jerusalem, quam elegi.


${}^{14}$~Suscitavit autem Dominus adversarium Salomoni Adad Idum\ae um de semine regio, qui erat in Edom.
${}^{15}$~Cum enim esset David in Idum\ae a, et ascendisset Joab princeps militi\ae\ ad sepeliendum eos qui fuerant interfecti, et occidisset omnem masculinum in Idum\ae a
${}^{16}$~(sex enim mensibus ibi moratus est Joab, et omnis Isra\"el, donec interimeret omne masculinum in Idum\ae a),
${}^{17}$~fugit Adad ipse, et viri Idum\ae i de servis patris ejus cum eo, ut ingrederetur \AE gyptum~: erat autem Adad puer parvulus.
${}^{18}$~Cumque surrexissent de Madian, venerunt in Pharan, tuleruntque secum viros de Pharan, et introierunt \AE gyptum ad Pharaonem regem \AE gypti~: qui dedit ei domum, et cibos constituit, et terram delegavit.
${}^{19}$~Et invenit Adad gratiam coram Pharaone valde, in tantum ut daret ei uxorem sororem uxoris su\ae\ germanam Taphnes regin\ae .
${}^{20}$~Genuitque ei soror Taphnes Genubath filium, et nutrivit eum Taphnes in domo Pharaonis~: eratque Genubath habitans apud Pharaonem cum filiis ejus.
${}^{21}$~Cumque audisset Adad in \AE gypto dormisse David cum patribus suis, et mortuum esse Joab principem militi\ae , dixit Pharaoni~: Dimitte me, ut vadam in terram meam.
${}^{22}$~Dixitque ei Pharao~: Qua enim re apud me indiges, ut qu\ae ras ire ad terram tuam~? At ille respondit~: Nulla~: sed obsecro te ut dimittas me.


${}^{23}$~Suscitavit quoque ei Deus adversarium Razon filium Eliada, qui fugerat Adarezer regem Soba dominum suum~:
${}^{24}$~et congregavit contra eum viros, et factus est princeps latronum cum interficeret eos David~: abieruntque Damascum, et habitaverunt ibi, et constituerunt eum regem in Damasco~:
${}^{25}$~eratque adversarius Isra\"eli cunctis diebus Salomonis~: et hoc est malum Adad, et odium contra Isra\"el~: regnavitque in Syria.
${}^{26}$~Jeroboam quoque filius Nabat, Ephrath\ae us, de Sareda, servus Salomonis, cujus mater erat nomine Sarva, mulier vidua, levavit manum contra regem.
${}^{27}$~Et h\ae c est causa rebellionis adversus eum, quia Salomon \ae dificavit Mello, et co\ae quavit voraginem civitatis David patris sui.
${}^{28}$~Erat autem Jeroboam vir fortis et potens~: vidensque Salomon adolescentem bon\ae\ indolis et industrium, constituerat eum pr\ae fectum super tributa univers\ae\ domus Joseph.


${}^{29}$~Factum est igitur in tempore illo, ut Jeroboam egrederetur de Jerusalem, et inveniret eum Ahias Silonites propheta in via, opertus pallio novo~: erant autem duo tantum in agro.
${}^{30}$~Apprehendensque Ahias pallium suum novum quo coopertus erat, scidit in duodecim partes.
${}^{31}$~Et ait ad Jeroboam~: Tolle tibi decem scissuras~: h\ae c enim dicit Dominus Deus Isra\"el~: Ecce ego scindam regnum de manu Salomonis, et dabo tibi decem tribus.
${}^{32}$~Porro una tribus remanebit ei propter servum meum David, et Jerusalem civitatem, quam elegi ex omnibus tribubus Isra\"el~:
${}^{33}$~eo quod dereliquerit me, et adoraverit Astarthen deam Sidoniorum, et Chamos deum Moab, et Moloch deum filiorum Ammon~: et non ambulaverit in viis meis, ut faceret justitiam coram me, et pr\ae cepta mea et judicia, sicut David pater ejus.
${}^{34}$~Nec auferam omne regnum de manu ejus, sed ducem ponam eum cunctis diebus vit\ae\ su\ae , propter David servum meum quem elegi, qui custodivit mandata mea et pr\ae cepta mea.
${}^{35}$~Auferam autem regnum de manu filii ejus, et dabo tibi decem tribus~:
${}^{36}$~filio autem ejus dabo tribum unam, ut remaneat lucerna David servo meo cunctis diebus coram me in Jerusalem civitate, quam elegi ut esset nomen meum ibi.
${}^{37}$~Te autem assumam, et regnabis super omnia qu\ae\ desiderat anima tua, erisque rex super Isra\"el.
${}^{38}$~Si igitur audieris omnia qu\ae\ pr\ae cepero tibi, et ambulaveris in viis meis, et feceris quod rectum est coram me, custodiens mandata mea et pr\ae cepta mea, sicut fecit David servus meus~: ero tecum, et \ae dificabo tibi domum fidelem, quomodo \ae dificavi David domum~: et tradam tibi Isra\"el~:
${}^{39}$~et affligam semen David super hoc, verumtamen non cunctis diebus.
${}^{40}$~Voluit ergo Salomon interficere Jeroboam~: qui surrexit, et aufugit in \AE gyptum ad Sesac regem \AE gypti, et fuit in \AE gypto usque ad mortem Salomonis.


${}^{41}$~Reliquum autem verborum Salomonis, et omnia qu\ae\ fecit, et sapientia ejus, ecce universa scripta sunt in libro verborum dierum Salomonis.
${}^{42}$~Dies autem quos regnavit Salomon in Jerusalem super omnem Isra\"el, quadraginta anni sunt.
${}^{43}$~Dormivitque Salomon cum patribus suis, et sepultus est in civitate David patris sui~: regnavitque Roboam filius ejus pro eo.
\Needspace{2.5\baselineskip}\versal{12}~\lettrine[lines=10,image=true,loversize=0.05,lraise=-0.03]{V}{}enit autem Roboam in Sichem~: illuc enim congregatus erat omnis Isra\"el ad constituendum eum regem.
${}^{2}$~At vero Jeroboam filius Nabat, cum adhuc esset in \AE gypto profugus a facie regis Salomonis, audita morte ejus, reversus est de \AE gypto.
${}^{3}$~Miseruntque et vocaverunt eum~: venit ergo Jeroboam, et omnis multitudo Isra\"el, et locuti sunt ad Roboam, dicentes~:
${}^{4}$~Pater tuus durissimum jugum imposuit nobis~: tu itaque nunc imminue paululum de imperio patris tui durissimo, et de jugo gravissimo quod imposuit nobis, et serviemus tibi.
${}^{5}$~Qui ait eis~: Ite usque ad tertium diem, et revertimini ad me. Cumque abiisset populus,
${}^{6}$~iniit consilium rex Roboam cum senioribus qui assistebant coram Salomone patre ejus cum adhuc viveret, et ait~: Quod datis mihi consilium, ut respondeam populo huic~?
${}^{7}$~Qui dixerunt ei~: Si hodie obedieris populo huic, et servieris, et petitioni eorum cesseris, locutusque fueris ad eos verba lenia, erunt tibi servi cunctis diebus.
${}^{8}$~Qui dereliquit consilium senum, quod dederant ei, et adhibuit adolescentes, qui nutriti fuerant cum eo, et assistebant illi,
${}^{9}$~dixitque ad eos~: Quod mihi datis consilium, ut respondeam populo huic, qui dixerunt mihi~: Levius fac jugum quod imposuit pater tuus super nos~?
${}^{10}$~Et dixerunt ei juvenes qui nutriti fuerant cum eo~: Sic loqueris populo huic, qui locuti sunt ad te, dicentes~: Pater tuus aggravavit jugum nostrum~: tu releva nos. Sic loqueris ad eos~: Minimus digitus meus grossior est dorso patris mei.
${}^{11}$~Et nunc pater meus posuit super vos jugum grave, ego autem addam super jugum vestrum~: pater meus cecidit vos flagellis, ego autem c\ae dam vos scorpionibus.
${}^{12}$~Venit ergo Jeroboam et omnis populus ad Roboam die tertia, sicut locutus fuerat rex, dicens~: Revertimini ad me die tertia.
${}^{13}$~Responditque rex populo dura, derelicto consilio seniorum quod ei dederant,
${}^{14}$~et locutus est eis secundum consilium juvenum, dicens~: Pater meus aggravavit jugum vestrum, ego autem addam jugo vestro~: pater meus cecidit vos flagellis, ego autem c\ae dam vos scorpionibus.
${}^{15}$~Et non acquievit rex populo~: quoniam aversatus fuerat eum Dominus, ut suscitaret verbum suum quod locutus fuerat in manu Ahi\ae\ Silonit\ae , ad Jeroboam filium Nabat.


${}^{16}$~Videns itaque populus quod noluisset eos audire rex, respondit ei dicens~: Qu\ae\ nobis pars in David~? vel qu\ae\ h\ae reditas in filio Isai~? vade in tabernacula tua, Isra\"el~: nunc vide domum tuam, David. Et abiit Isra\"el in tabernacula sua.
${}^{17}$~Super filios autem Isra\"el, quicumque habitabant in civitatibus Juda, regnavit Roboam.
${}^{18}$~Misit ergo rex Roboam Aduram, qui erat super tributa~: et lapidavit eum omnis Isra\"el, et mortuus est. Porro rex Roboam festinus ascendit currum, et fugit in Jerusalem~:
${}^{19}$~recessitque Isra\"el a domo David usque in pr\ae sentem diem.
${}^{20}$~Factum est autem cum audisset omnis Isra\"el quod reversus esset Jeroboam, miserunt, et vocaverunt eum congregato cœtu, et constituerunt eum regem super omnem Isra\"el~: nec secutus est quisquam domum David pr\ae ter tribum Juda solam.
${}^{21}$~Venit autem Roboam Jerusalem, et congregavit universam domum Juda, et tribum Benjamin, centum octoginta millia electorum virorum bellatorum, ut pugnarent contra domum Isra\"el, et reducerent regnum Roboam filio Salomonis.
${}^{22}$~Factus est autem sermo Domini ad Semeiam virum Dei, dicens~:
${}^{23}$~Loquere ad Roboam filium Salomonis regem Juda, et ad omnem domum Juda, et Benjamin, et reliquos de populo, dicens~:
${}^{24}$~H\ae c dicit Dominus~: Non ascendetis, neque bellabitis contra fratres vestros filios Isra\"el~: revertatur vir in domum suam~: a me enim factum est verbum hoc. Audierunt sermonem Domini, et reversi sunt de itinere, sicut eis pr\ae ceperat Dominus.


${}^{25}$~\AE dificavit autem Jeroboam Sichem in monte Ephraim, et habitavit ibi~: et egressus inde \ae dificavit Phanuel.
${}^{26}$~Dixitque Jeroboam in corde suo~: Nunc revertetur regnum ad domum David,
${}^{27}$~si ascenderit populus iste ut faciat sacrificia in domo Domini in Jerusalem~: et convertetur cor populi hujus ad dominum suum Roboam regem Juda, interficientque me, et revertentur ad eum.
${}^{28}$~Et excogitato consilio fecit duos vitulos aureos, et dixit eis~: Nolite ultra ascendere in Jerusalem~: ecce dii tui Isra\"el, qui te eduxerunt de terra \AE gypti.
${}^{29}$~Posuitque unum in Bethel, et alterum in Dan~:
${}^{30}$~et factum est verbum hoc in peccatum~: ibat enim populus ad adorandum vitulum usque in Dan.
${}^{31}$~Et fecit fana in excelsis, et sacerdotes de extremis populi, qui non erant de filiis Levi.
${}^{32}$~Constituitque diem solemnem in mense octavo, quintadecima die mensis, in similitudinem solemnitatis qu\ae\ celebrabatur in Juda. Et ascendens altare, similiter fecit in Bethel, ut immolaret vitulis quos fabricatus fuerat~: constituitque in Bethel sacerdotes excelsorum qu\ae\ fecerat.
${}^{33}$~Et ascendit super altare quod exstruxerat in Bethel, quintadecima die mensis octavi, quem finxerat de corde suo~: et fecit solemnitatem filiis Isra\"el, et ascendit super altare, ut adoleret incensum.
\Needspace{2.5\baselineskip}\versal{13}~\lettrine[lines=10,image=true,loversize=0.05,lraise=-0.03]{E}{}t ecce vir Dei venit de Juda in sermone Domini in Bethel, Jeroboam stante super altare, et thus jaciente.
${}^{2}$~Et exclamavit contra altare in sermone Domini, et ait~: Altare, altare, h\ae c dicit Dominus~: Ecce filius nascetur domui David, Josias nomine, et immolabit super te sacerdotes excelsorum, qui nunc in te thura succendunt~: et ossa hominum super te incendet.
${}^{3}$~Deditque in illa die signum, dicens~: Hoc erit signum quod locutus est Dominus~: ecce altare scindetur, et effundetur cinis qui in eo est.
${}^{4}$~Cumque audisset rex sermonem hominis Dei quem inclamaverat contra altare in Bethel, extendit manum suam de altari, dicens~: Apprehendite eum. Et exaruit manus ejus quam extenderat contra eum, nec valuit retrahere eam ad se.
${}^{5}$~Altare quoque scissum est, et effusus est cinis de altari, juxta signum quod pr\ae dixerat vir Dei in sermone Domini.
${}^{6}$~Et ait rex ad virum Dei~: Deprecare faciem Domini Dei tui, et ora pro me, ut restituatur manus mea mihi. Oravitque vir Dei faciem Domini, et reversa est manus regis ad eum, et facta est sicut prius fuerat.
${}^{7}$~Locutus est autem rex ad virum Dei~: Veni mecum domum ut prandeas, et dabo tibi munera.
${}^{8}$~Responditque vir Dei ad regem~: Si dederis mihi mediam partem domus tu\ae , non veniam tecum, nec comedam panem, neque bibam aquam in loco isto~:
${}^{9}$~sic enim mandatum est mihi in sermone Domini pr\ae cipientis~: Non comedes panem, neque bibes aquam, nec reverteris per viam qua venisti.
${}^{10}$~Abiit ergo per aliam viam, et non est reversus per iter quo venerat in Bethel.


${}^{11}$~Prophetes autem quidam senex habitabat in Bethel~: ad quem venerunt filii sui, et narraverunt ei omnia opera qu\ae\ fecerat vir Dei illa die in Bethel~: et verba qu\ae\ locutus fuerat ad regem, narraverunt patri suo.
${}^{12}$~Et dixit eis pater eorum~: Per quam viam abiit~? Ostenderunt ei filii sui viam per quam abierat vir Dei, qui venerat de Juda.
${}^{13}$~Et ait filiis suis~: Sternite mihi asinum. Qui cum stravissent, ascendit,
${}^{14}$~et abiit post virum Dei, et invenit eum sedentem subtus terebinthum~: et ait illi~: Tune es vir Dei qui venisti de Juda~? Respondit ille~: Ego sum.
${}^{15}$~Dixitque ad eum~: Veni mecum domum, ut comedas panem.
${}^{16}$~Qui ait~: Non possum reverti, neque venire tecum~: nec comedam panem, neque bibam aquam in loco isto,
${}^{17}$~quia locutus est Dominus ad me in sermone Domini, dicens~: Non comedes panem, et non bibes aquam ibi, nec reverteris per viam qua ieris.
${}^{18}$~Qui ait illi~: Et ego propheta sum similis tui~: et angelus locutus est mihi in sermone Domini, dicens~: Reduc eum tecum in domum tuam, ut comedat panem, et bibat aquam. Fefellit eum,
${}^{19}$~et reduxit secum~: comedit ergo panem in domo ejus, et bibit aquam.
${}^{20}$~Cumque sederent ad mensam, factus est sermo Domini ad prophetam qui reduxerat eum.
${}^{21}$~Et exclamavit ad virum Dei qui venerat de Juda, dicens~: H\ae c dicit Dominus~: Quia non obediens fuisti ori Domini, et non custodisti mandatum quod pr\ae cepit tibi Dominus Deus tuus,
${}^{22}$~et reversus es, et comedisti panem, et bibisti aquam in loco in quo pr\ae cepit tibi ne comederes panem neque biberes aquam, non inferetur cadaver tuum in sepulchrum patrum tuorum.


${}^{23}$~Cumque comedisset et bibisset, stravit asinum suum prophet\ae\ quem reduxerat.
${}^{24}$~Qui cum abiisset, invenit eum leo in via, et occidit, et erat cadaver ejus projectum in itinere~: asinus autem stabat juxta illum, et leo stabat juxta cadaver.
${}^{25}$~Et ecce viri transeuntes viderunt cadaver projectum in via, et leonem stantem juxta cadaver. Et venerunt, et divulgaverunt in civitate in qua prophetes ille senex habitabat.
${}^{26}$~Quod cum audisset propheta ille qui reduxerat eum de via, ait~: Vir Dei est, qui inobediens fuit ori Domini, et tradidit eum Dominus leoni, et confregit eum, et occidit juxta verbum Domini quod locutus est ei.
${}^{27}$~Dixitque ad filios suos~: Sternite mihi asinum. Qui cum stravissent,
${}^{28}$~et ille abiisset, invenit cadaver ejus projectum in via, et asinum et leonem stantes juxta cadaver~: non comedit leo de cadavere, nec l\ae sit asinum.
${}^{29}$~Tulit ergo prophetes cadaver viri Dei, et posuit illud super asinum, et reversus intulit in civitatem prophet\ae\ senis ut plangeret eum.
${}^{30}$~Et posuit cadaver ejus in sepulchro suo, et planxerunt eum~: Heu, heu mi frater~!
${}^{31}$~Cumque planxissent eum, dixit ad filios suos~: Cum mortuus fuero, sepelite me in sepulchro in quo vir Dei sepultus est~: juxta ossa ejus ponite ossa mea.
${}^{32}$~Profecto enim veniet sermo quem pr\ae dixit in sermone Domini contra altare quod est in Bethel, et contra omnia fana excelsorum qu\ae\ sunt in urbibus Samari\ae .


${}^{33}$~Post verba h\ae c non est reversus Jeroboam de via sua pessima, sed e contrario fecit de novissimis populi sacerdotes excelsorum~: quicumque volebat, implebat manum suam, et fiebat sacerdos excelsorum.
${}^{34}$~Et propter hanc causam peccavit domus Jeroboam, et eversa est, et deleta de superficie terr\ae .
\Needspace{2.5\baselineskip}\versal{14}~\lettrine[lines=10,image=true,loversize=0.05,lraise=-0.03]{I}{}n tempore illo \ae grotavit Abia filius Jeroboam.
${}^{2}$~Dixitque Jeroboam uxori su\ae~: Surge, et commuta habitum, ne cognoscaris quod sis uxor Jeroboam, et vade in Silo, ubi est Ahias propheta, qui locutus est mihi quod regnaturus essem super populum hunc.
${}^{3}$~Tolle quoque in manu tua decem panes, et crustulam, et vas mellis, et vade ad illum~: ipse enim indicabit tibi quid eventurum sit puero huic.
${}^{4}$~Fecit ut dixerat, uxor Jeroboam~: et consurgens abiit in Silo, et venit in domum Ahi\ae~: at ille non poterat videre, quia caligaverant oculi ejus pr\ae\ senectute.
${}^{5}$~Dixit autem Dominus ad Ahiam~: Ecce uxor Jeroboam ingreditur ut consulat te super filio suo qui \ae grotat~: h\ae c et h\ae c loqueris ei. Cum ergo illa intraret, et dissimularet se esse qu\ae\ erat,
${}^{6}$~audivit Ahias sonitum pedum ejus intro\"euntis per ostium, et ait~: Ingredere, uxor Jeroboam~: quare aliam te esse simulas~? ego autem missus sum ad te durus nuntius.
${}^{7}$~Vade, et dic Jeroboam~: H\ae c dicit Dominus Deus Isra\"el~: Quia exaltavi te de medio populi, et dedi te ducem super populum meum Isra\"el,
${}^{8}$~et scidi regnum domus David, et dedi illud tibi, et non fuisti sicut servus meus David, qui custodivit mandata mea, et secutus est me in toto corde suo, faciens quod placitum esset in conspectu meo~:
${}^{9}$~sed operatus es mala super omnes qui fuerunt ante te, et fecisti tibi deos alienos et conflatiles, ut me ad iracundiam provocares, me autem projecisti post corpus tuum~:
${}^{10}$~idcirco ecce ego inducam mala super domum Jeroboam, et percutiam de Jeroboam mingentem ad parietem, et clausum, et novissimum in Isra\"el~: et mundabo reliquias domus Jeroboam, sicut mundari solet fimus usque ad purum.
${}^{11}$~Qui mortui fuerint de Jeroboam in civitate, comedent eos canes~: qui autem mortui fuerint in agro, vorabunt eos aves c\ae li~: quia Dominus locutus est.
${}^{12}$~Tu igitur surge, et vade in domum tuam~: et in ipso introitu pedum tuorum in urbem, morietur puer,
${}^{13}$~et planget eum omnis Isra\"el, et sepeliet~: iste enim solus inferetur de Jeroboam in sepulchrum, quia inventus est super eo sermo bonus a Domino Deo Isra\"el in domo Jeroboam.
${}^{14}$~Constituet autem sibi Dominus regem super Isra\"el, qui percutiet domum Jeroboam in hac die, et in hoc tempore~:
${}^{15}$~et percutiet Dominus Deus Isra\"el, sicut moveri solet arundo in aqua~: et evellet Isra\"el de terra bona hac, quam dedit patribus eorum, et ventilabit eos trans flumen~: quia fecerunt sibi lucos, ut irritarent Dominum.
${}^{16}$~Et tradet Dominus Isra\"el propter peccata Jeroboam, qui peccavit, et peccare fecit Isra\"el.
${}^{17}$~Surrexit itaque uxor Jeroboam, et abiit, et venit in Thersa~: cumque illa ingrederetur limen domus, puer mortuus est,
${}^{18}$~et sepelierunt eum. Et planxit eum omnis Isra\"el juxta sermonem Domini, quem locutus est in manu servi sui Ahi\ae\ prophet\ae .


${}^{19}$~Reliqua autem verborum Jeroboam, quomodo pugnaverit, et quomodo regnaverit, ecce scripta sunt in libro verborum dierum regum Isra\"el.
${}^{20}$~Dies autem quibus regnavit Jeroboam, viginti duo anni sunt~: et dormivit cum patribus suis, regnavitque Nadab filius ejus pro eo.


${}^{21}$~Porro Roboam filius Salomonis regnavit in Juda. Quadraginta et unius anni erat Roboam cum regnare cœpisset~: decem et septem annos regnavit in Jerusalem civitate, quam elegit Dominus ut poneret nomen suum ibi, ex omnibus tribubus Isra\"el. Nomen autem matris ejus Naama Ammanitis.
${}^{22}$~Et fecit Judas malum coram Domino, et irritaverunt eum super omnibus qu\ae\ fecerant patres eorum in peccatis suis qu\ae\ peccaverunt.
${}^{23}$~\AE dificaverunt enim et ipsi sibi aras, et statuas, et lucos super omnem collem excelsum, et subter omnem arborem frondosam~:
${}^{24}$~sed et effeminati fuerunt in terra, feceruntque omnes abominationes gentium quas attrivit Dominus ante faciem filiorum Isra\"el.
${}^{25}$~In quinto autem anno regni Roboam, ascendit Sesac rex \AE gypti in Jerusalem,
${}^{26}$~et tulit thesauros domus Domini, et thesauros regios, et universa diripuit~: scuta quoque aurea, qu\ae\ fecerat Salomon~:
${}^{27}$~pro quibus fecit rex Roboam scuta \ae rea, et tradidit ea in manum ducum scutariorum, et eorum qui excubabant ante ostium domus regis.
${}^{28}$~Cumque ingrederetur rex in domum Domini, portabant ea qui pr\ae eundi habebant officium~: et postea reportabant ad armamentarium scutariorum.
${}^{29}$~Reliqua autem sermonum Roboam, et omnia qu\ae\ fecit, ecce scripta sunt in libro sermonum dierum regum Juda.
${}^{30}$~Fuitque bellum inter Roboam et Jeroboam cunctis diebus.
${}^{31}$~Dormivitque Roboam cum patribus suis, et sepultus est cum eis in civitate David~: nomen autem matris ejus Naama Ammanitis~: et regnavit Abiam filius ejus pro eo.
\Needspace{2.5\baselineskip}\versal{15}~\lettrine[lines=10,image=true,loversize=0.05,lraise=-0.03]{I}{}gitur in octavodecimo anno regni Jeroboam filii Nabat, regnavit Abiam super Judam.
${}^{2}$~Tribus annis regnavit in Jerusalem~: nomen matris ejus Maacha filia Abessalom.
${}^{3}$~Ambulavitque in omnibus peccatis patris sui, qu\ae\ fecerat ante eum~: nec erat cor ejus perfectum cum Domino Deo suo, sicut cor David patris ejus.
${}^{4}$~Sed propter David dedit ei Dominus Deus suus lucernam in Jerusalem, ut suscitaret filium ejus post eum, et statueret Jerusalem~:
${}^{5}$~eo quod fecisset David rectum in oculis Domini, et non declinasset ab omnibus qu\ae\ pr\ae ceperat ei cunctis diebus vit\ae\ su\ae , excepto sermone Uri\ae\ Heth\ae i.
${}^{6}$~Attamen bellum fuit inter Roboam et Jeroboam omni tempore vit\ae\ ejus.
${}^{7}$~Reliqua autem sermonum Abiam, et omnia qu\ae\ fecit, nonne h\ae c scripta sunt in libro verborum dierum regum Juda~? Fuitque pr\ae lium inter Abiam et inter Jeroboam.
${}^{8}$~Et dormivit Abiam cum patribus suis, et sepelierunt eum in civitate David, regnavitque Asa filius ejus pro eo.


${}^{9}$~In anno ergo vigesimo Jeroboam regis Isra\"el regnavit Asa rex Juda,
${}^{10}$~et quadraginta et uno anno regnavit in Jerusalem. Nomen matris ejus Maacha filia Abessalom.
${}^{11}$~Et fecit Asa rectum ante conspectum Domini, sicut David pater ejus~:
${}^{12}$~et abstulit effeminatos de terra, purgavitque universas sordes idolorum qu\ae\ fecerant patres ejus.
${}^{13}$~Insuper et Maacham matrem suam amovit, ne esset princeps in sacris Priapi, et in luco ejus quem consecraverat~: subvertitque specum ejus, et confregit simulacrum turpissimum, et combussit in torrente Cedron~:
${}^{14}$~excelsa autem non abstulit. Verumtamen cor Asa perfectum erat cum Domino cunctis diebus suis~:
${}^{15}$~et intulit ea qu\ae\ sanctificaverat pater suus, et voverat, in domum Domini, argentum, et aurum, et vasa.


${}^{16}$~Bellum autem erat inter Asa, et Baasa regem Isra\"el cunctis diebus eorum.
${}^{17}$~Ascendit quoque Baasa rex Isra\"el in Judam, et \ae dificavit Rama, ut non posset quispiam egredi vel ingredi de parte Asa regis Juda.
${}^{18}$~Tollens itaque Asa omne argentum et aurum quod remanserat in thesauris domus Domini, et in thesauris domus regi\ae , et dedit illud in manus servorum suorum~: et misit ad Benadad filium Tabremon filii Hezion, regem Syri\ae , qui habitabat in Damasco, dicens~:
${}^{19}$~Fœdus est inter me et te, et inter patrem meum et patrem tuum~: ideo misi tibi munera, argentum et aurum~: et peto ut venias, et irritum facias fœdus quod habes cum Baasa rege Isra\"el, et recedat a me.
${}^{20}$~Acquiescens Benadad regi Asa, misit principes exercitus sui in civitates Isra\"el, et percusserunt Ahion, et Dan, et Abeldomum Maacha, et universam Cenneroth, omnem scilicet terram Nephthali.
${}^{21}$~Quod cum audisset Baasa, intermisit \ae dificare Rama, et reversus est in Thersa.
${}^{22}$~Rex autem Asa nuntium misit in omnem Judam, dicens~: Nemo sit excusatus. Et tulerunt lapides de Rama, et ligna ejus, quibus \ae dificaverat Baasa, et exstruxit de eis rex Asa Gabaa Benjamin, et Maspha.
${}^{23}$~Reliqua autem omnium sermonum Asa, et univers\ae\ fortitudines ejus, et cuncta qu\ae\ fecit, et civitates quas exstruxit, nonne h\ae c scripta sunt in libro verborum dierum regum Juda~? Verumtamen in tempore senectutis su\ae\ doluit pedes.
${}^{24}$~Et dormivit cum patribus suis, et sepultus est cum eis in civitate David patris sui. Regnavitque Josaphat filius ejus pro eo.


${}^{25}$~Nadab vero filius Jeroboam regnavit super Isra\"el anno secundo Asa regis Juda~: regnavitque super Isra\"el duobus annis.
${}^{26}$~Et fecit quod malum est in conspectu Domini, et ambulavit in viis patris sui, et in peccatis ejus quibus peccare fecit Isra\"el.
${}^{27}$~Insidiatus est autem ei Baasa filius Ahi\ae\ de domo Issachar, et percussit eum in Gebbethon, qu\ae\ est urbs Philisthinorum~: siquidem Nadab et omnis Isra\"el obsidebant Gebbethon.
${}^{28}$~Interfecit ergo illum Baasa in anno tertio Asa regis Juda, et regnavit pro eo.
${}^{29}$~Cumque regnasset, percussit omnem domum Jeroboam~: non dimisit ne unam quidem animam de semine ejus donec deleret eum, juxta verbum Domini quod locutus fuerat in manu servi sui Ahi\ae\ Silonitis,
${}^{30}$~propter peccata Jeroboam, qu\ae\ peccaverat, et quibus peccare fecerat Isra\"el~: et propter delictum quo irritaverat Dominum Deum Isra\"el.
${}^{31}$~Reliqua autem sermonum Nadab, et omnia qu\ae\ operatus est, nonne h\ae c scripta sunt in libro verborum dierum regum Isra\"el~?
${}^{32}$~Fuitque bellum inter Asa, et Baasa regem Isra\"el, cunctis diebus eorum.


${}^{33}$~Anno tertio Asa regis Juda, regnavit Baasa filius Ahi\ae\ super omnem Isra\"el in Thersa, viginti quatuor annis.
${}^{34}$~Et fecit malum coram Domino, ambulavitque in via Jeroboam, et in peccatis ejus quibus peccare fecit Isra\"el.
\Needspace{2.5\baselineskip}\versal{16}~\lettrine[lines=10,image=true,loversize=0.05,lraise=-0.03]{F}{}actus est autem sermo Domini ad Jehu filium Hanani contra Baasa, dicens~:
${}^{2}$~Pro eo quod exaltavi te de pulvere, et posui te ducem super populum meum Isra\"el, tu autem ambulasti in via Jeroboam, et peccare fecisti populum meum Isra\"el, ut me irritares in peccatis eorum~:
${}^{3}$~ecce ego demetam posteriora Baasa, et posteriora domus ejus, et faciam domum tuam sicut domum Jeroboam filii Nabat.
${}^{4}$~Qui mortuus fuerit de Baasa in civitate, comedent eum canes~: et qui mortuus fuerit ex eo in regione, comedent eum volucres c\ae li.
${}^{5}$~Reliqua autem sermonum Baasa, et qu\ae cumque fecit, et pr\ae lia ejus, nonne h\ae c scripta sunt in libro verborum dierum regum Isra\"el~?
${}^{6}$~Dormivit ergo Baasa cum patribus suis, sepultusque est in Thersa~: et regnavit Ela filius ejus pro eo.
${}^{7}$~Cum autem in manu Jehu filii Hanani prophet\ae\ verbum Domini factum esset contra Baasa, et contra domum ejus, et contra omne malum quod fecerat coram Domino, ad irritandum eum in operibus manuum suarum, ut fieret sicut domus Jeroboam~: ob hanc causam occidit eum, hoc est, Jehu filium Hanani prophetam.


${}^{8}$~Anno vigesimo sexto Asa regis Juda, regnavit Ela filius Baasa super Isra\"el in Thersa, duobus annis.
${}^{9}$~Et rebellavit contra eum servus suus Zambri, dux medi\ae\ partis equitum~: erat autem Ela in Thersa bibens, et temulentus in domo Arsa pr\ae fecti Thersa.
${}^{10}$~Irruens ergo Zambri, percussit et occidit eum, anno vigesimo septimo Asa regis Juda, et regnavit pro eo.
${}^{11}$~Cumque regnasset, et sedisset super solium ejus, percussit omnem domum Baasa, et non dereliquit ex ea mingentem ad parietem~: et propinquos et amicos ejus.
${}^{12}$~Delevitque Zambri omnem domum Baasa, juxta verbum Domini quod locutus fuerat ad Baasa in manu Jehu prophet\ae ,
${}^{13}$~propter universa peccata Baasa, et peccata Ela filii ejus, qui peccaverunt, et peccare fecerunt Isra\"el, provocantes Dominum Deum Isra\"el in vanitatibus suis.
${}^{14}$~Reliqua autem sermonum Ela, et omnia qu\ae\ fecit, nonne h\ae c scripta sunt in libro verborum dierum regum Isra\"el~?


${}^{15}$~Anno vigesimo septimo Asa regis Juda, regnavit Zambri septem diebus in Thersa~: porro exercitus obsidebat Gebbethon urbem Philisthinorum.
${}^{16}$~Cumque audisset rebellasse Zambri, et occidisse regem, fecit sibi regem omnis Isra\"el Amri, qui erat princeps militi\ae\ super Isra\"el in die illa in castris.
${}^{17}$~Ascendit ergo Amri, et omnis Isra\"el cum eo, de Gebbethon, et obsidebant Thersa.
${}^{18}$~Videns autem Zambri quod expugnanda esset civitas, ingressus est palatium, et succendit se cum domo regia~: et mortuus est
${}^{19}$~in peccatis suis qu\ae\ peccaverat, faciens malum coram Domino, et ambulans in via Jeroboam, et in peccato ejus, quo fecit peccare Isra\"el.
${}^{20}$~Reliqua autem sermonum Zambri, et insidiarum ejus, et tyrannidis, nonne h\ae c scripta sunt in libro verborum dierum regum Isra\"el~?
${}^{21}$~Tunc divisus est populus Isra\"el in duas partes~: media pars populi sequebatur Thebni filium Gineth, ut constitueret eum regem~: et media pars Amri.
${}^{22}$~Pr\ae valuit autem populus qui erat cum Amri, populo qui sequebatur Thebni filium Gineth~: mortuusque est Thebni, et regnavit Amri.


${}^{23}$~Anno trigesimo primo Asa regis Juda, regnavit Amri super Isra\"el, duodecim annis~: in Thersa regnavit sex annis.
${}^{24}$~Emitque montem Samari\ae\ a Somer duobus talentis argenti~: et \ae dificavit eum, et vocavit nomen civitatis quam exstruxerat, nomine Semer domini montis, Samariam.
${}^{25}$~Fecit autem Amri malum in conspectu Domini, et operatus est nequiter, super omnes qui fuerunt ante eum.
${}^{26}$~Ambulavitque in omni via Jeroboam filii Nabat, et in peccatis ejus quibus peccare fecerat Isra\"el, ut irritaret Dominum Deum Isra\"el in vanitatibus suis.
${}^{27}$~Reliqua autem sermonum Amri, et pr\ae lia ejus qu\ae\ gessit, nonne h\ae c scripta sunt in libro verborum dierum regum Isra\"el~?
${}^{28}$~Dormivitque Amri cum patribus suis, et sepultus est in Samaria~: regnavitque Achab filius ejus pro eo.


${}^{29}$~Achab vero filius Amri regnavit super Isra\"el anno trigesimo octavo Asa regis Juda. Et regnavit Achab filius Amri super Isra\"el in Samaria viginti et duobus annis.
${}^{30}$~Et fecit Achab filius Amri malum in conspectu Domini super omnes qui fuerunt ante eum.
${}^{31}$~Nec suffecit ei ut ambularet in peccatis Jeroboam filii Nabat~: insuper duxit uxorem Jezabel filiam Ethbaal regis Sidoniorum. Et abiit, et servivit Baal, et adoravit eum.
${}^{32}$~Et posuit aram Baal in templo Baal, quod \ae dificaverat in Samaria,
${}^{33}$~et plantavit lucum~: et addidit Achab in opere suo, irritans Dominum Deum Isra\"el super omnes reges Isra\"el qui fuerunt ante eum.
${}^{34}$~In diebus ejus \ae dificavit Hiel de Bethel Jericho~: in Abiram primitivo suo fundavit eam, et in Segub novissimo suo posuit portas ejus, juxta verbum Domini quod locutus fuerat in manu Josue filii Nun.
\Needspace{2.5\baselineskip}\versal{17}~\lettrine[lines=10,image=true,loversize=0.05,lraise=-0.03]{E}{}t dixit Elias Thesbites de habitatoribus Galaad ad Achab~: Vivit Dominus Deus Isra\"el, in cujus conspectu sto, si erit annis his ros et pluvia, nisi juxta oris mei verba.
${}^{2}$~Et factum est verbum Domini ad eum, dicens~:
${}^{3}$~Recede hinc, et vade contra orientem, et abscondere in torrente Carith, qui est contra Jordanem,
${}^{4}$~et ibi de torrente bibes~: corvisque pr\ae cepi ut pascant te ibi.
${}^{5}$~Abiit ergo, et fecit juxta verbum Domini~: cumque abiisset, sedit in torrente Carith, qui est contra Jordanem.
${}^{6}$~Corvi quoque deferebant ei panem et carnes mane, similiter panem et carnes vesperi, et bibebat de torrente.
${}^{7}$~Post dies autem siccatus est torrens~: non enim pluerat super terram.


${}^{8}$~Factus est ergo sermo Domini ad eum, dicens~:
${}^{9}$~Surge, et vade in Sarephta Sidoniorum, et manebis ibi~: pr\ae cepi enim ibi mulieri vidu\ae\ ut pascat te.
${}^{10}$~Surrexit, et abiit in Sarephta. Cumque venisset ad portam civitatis, apparuit ei mulier vidua colligens ligna, et vocavit eam, dixitque ei~: Da mihi paululum aqu\ae\ in vase ut bibam.
${}^{11}$~Cumque illa pergeret ut afferret, clamavit post tergum ejus, dicens~: Affer mihi, obsecro, et buccellam panis in manu tua.
${}^{12}$~Qu\ae\ respondit~: Vivit Dominus Deus tuus, quia non habeo panem, nisi quantum pugillus capere potest farin\ae\ in hydria, et paululum olei in lecytho~: en colligo duo ligna ut ingrediar et faciam illum mihi et filio meo, ut comedamus, et moriamur.
${}^{13}$~Ad quam Elias ait~: Noli timere, sed vade, et fac sicut dixisti~: verumtamen mihi primum fac de ipsa farinula subcinericium panem parvulum, et affer ad me~: tibi autem et filio tuo facies postea.
${}^{14}$~H\ae c autem dicit Dominus Deus Isra\"el~: Hydria farin\ae\ non deficiet, nec lecythus olei minuetur, usque ad diem in qua Dominus daturus est pluviam super faciem terr\ae .
${}^{15}$~Qu\ae\ abiit, et fecit juxta verbum Eli\ae~: et comedit ipse, et illa, et domus ejus~: et ex illa die
${}^{16}$~hydria farin\ae\ non defecit, et lecythus olei non est imminutus, juxta verbum Domini quod locutus fuerat in manu Eli\ae .
${}^{17}$~Factum est autem post h\ae c, \ae grotavit filius mulieris matrisfamilias, et erat languor fortissimus, ita ut non remaneret in eo halitus.
${}^{18}$~Dixit ergo ad Eliam~: Quid mihi et tibi, vir Dei~? ingressus es ad me, ut rememorarentur iniquitates me\ae , et interficeres filium meum~?
${}^{19}$~Et ait ad eam Elias~: Da mihi filium tuum. Tulitque eum de sinu ejus, et portavit in cœnaculum ubi ipse manebat, et posuit super lectulum suum.
${}^{20}$~Et clamavit ad Dominum, et dixit~: Domine Deus meus, etiam ne viduam apud quam ego utcumque sustentor, afflixisti ut interficeres filium ejus~?
${}^{21}$~Et expandit se, atque mensus est super puerum tribus vicibus, et clamavit ad Dominum, et ait~: Domine Deus meus, revertatur, obsecro, anima pueri hujus in viscera ejus.
${}^{22}$~Et exaudivit Dominus vocem Eli\ae~: et reversa est anima pueri intra eum, et revixit.
${}^{23}$~Tulitque Elias puerum, et deposuit eum de cœnaculo in inferiorem domum, et tradidit matri su\ae , et ait illi~: En vivit filius tuus.
${}^{24}$~Dixitque mulier ad Eliam~: Nunc in isto cognovi quoniam vir Dei es tu, et verbum Domini in ore tuo verum est.
\Needspace{2.5\baselineskip}\versal{18}~\lettrine[lines=10,image=true,loversize=0.05,lraise=-0.03]{P}{}ost dies multos factum est verbum Domini ad Eliam, in anno tertio, dicens~: Vade, et ostende te Achab, ut dem pluviam super faciem terr\ae .
${}^{2}$~Ivit ergo Elias, ut ostenderet se Achab~: erat autem fames vehemens in Samaria.
${}^{3}$~Vocavitque Achab Abdiam dispensatorem domus su\ae~: Abdias autem timebat Dominum valde.
${}^{4}$~Nam cum interficeret Jezabel prophetas Domini, tulit ille centum prophetas, et abscondit eos quinquagenos et quinquagenos in speluncis, et pavit eos pane et aqua.
${}^{5}$~Dixit ergo Achab ad Abdiam~: Vade in terram ad universos fontes aquarum, et in cunctas valles, si forte possimus invenire herbam, et salvare equos et mulos, et non penitus jumenta intereant.
${}^{6}$~Diviseruntque sibi regiones ut circuirent eas~: Achab ibat per viam unam, et Abdias per viam alteram seorsum.
${}^{7}$~Cumque esset Abdias in via, Elias occurrit ei~: qui cum cognovisset eum, cecidit super faciem suam, et ait~: Num tu es, domine mi, Elias~?
${}^{8}$~Cui ille respondit~: Ego. Vade, et dic domino tuo~: Adest Elias.
${}^{9}$~Et ille~: Quid peccavi, inquit, quoniam tradis me servum tuum in manu Achab, ut interficiat me~?
${}^{10}$~Vivit Dominus Deus tuus, quia non est gens aut regnum quo non miserit dominus meus te requirens~: et respondentibus cunctis~: Non est hic~: adjuravit regna singula et gentes, eo quod minime reperireris.
${}^{11}$~Et nunc tu dicis mihi~: Vade, et dic domino tuo~: Adest Elias.
${}^{12}$~Cumque recessero a te, spiritus Domini asportabit te in locum quem ego ignoro~: et ingressus nuntiabo Achab, et non inveniens te, interficiet me~: servus autem tuus timet Dominum ab infantia sua.
${}^{13}$~Numquid non indicatum est tibi domino meo quid fecerim cum interficeret Jezabel prophetas Domini, quod absconderim de prophetis Domini centum viros, quinquagenos et quinquagenos, in speluncis, et paverim eos pane et aqua~?
${}^{14}$~et nunc tu dicis~: Vade, et dic domino tuo~: Adest Elias~: ut interficiat me~?
${}^{15}$~Et dixit Elias~: Vivit Dominus exercituum, ante cujus vultum sto, quia hodie apparebo ei.
${}^{16}$~Abiit ergo Abdias in occursum Achab, et indicavit ei~: venitque Achab in occursum Eli\ae .
${}^{17}$~Et cum vidisset eum, ait~: Tune es ille, qui conturbas Isra\"el~?
${}^{18}$~Et ille ait~: Non ego turbavi Isra\"el, sed tu, et domus patris tui, qui dereliquistis mandata Domini, et secuti estis Baalim.


${}^{19}$~Verumtamen nunc mitte, et congrega ad me universum Isra\"el in monte Carmeli, et prophetas Baal quadringentos quinquaginta, prophetasque lucorum quadringentos, qui comedunt de mensa Jezabel.
${}^{20}$~Misit Achab ad omnes filios Isra\"el, et congregavit prophetas in monte Carmeli.
${}^{21}$~Accedens autem Elias ad omnem populum, ait~: Usquequo claudicatis in duas partes~? si Dominus est Deus, sequimini eum~: si autem Baal, sequimini illum. Et non respondit ei populus verbum.
${}^{22}$~Et ait rursus Elias ad populum~: Ego remansi propheta Domini solus~: prophet\ae\ autem Baal quadringenti et quinquaginta viri sunt.
${}^{23}$~Dentur nobis duo boves, et illi eligant sibi bovem unum, et in frusta c\ae dentes ponant super ligna, ignem autem non supponant~: et ego faciam bovem alterum, et imponam super ligna, ignem autem non supponam.
${}^{24}$~Invocate nomina deorum vestrorum, et ego invocabo nomen Domini mei~: et Deus qui exaudierit per ignem, ipse sit Deus. Respondens omnis populus ait~: Optima propositio.
${}^{25}$~Dixit ergo Elias prophetis Baal~: Eligite vobis bovem unum, et facite primi, quia vos plures estis~: et invocate nomina deorum vestrorum, ignemque non supponatis.
${}^{26}$~Qui cum tulissent bovem quem dederat eis, fecerunt~: et invocabant nomen Baal de mane usque ad meridiem, dicentes~: Baal, exaudi nos. Et non erat vox, nec qui responderet~: transiliebantque altare quod fecerant.
${}^{27}$~Cumque esset jam meridies, illudebat illis Elias, dicens~: Clamate voce majore~: deus enim est, et forsitan loquitur, aut in diversorio est, aut in itinere, aut certe dormit, ut excitetur.
${}^{28}$~Clamabant ergo voce magna, et incidebant se juxta ritum suum cultris et lanceolis, donec perfunderentur sanguine.
${}^{29}$~Postquam autem transiit meridies, et illis prophetantibus venerat tempus quo sacrificium offerri solet, nec audiebatur vox, nec aliquis respondebat, nec attendebat orantes,
${}^{30}$~dixit Elias omni populo~: Venite ad me.

 Et accedente ad se populo, curavit altare Domini quod destructum fuerat.
${}^{31}$~Et tulit duodecim lapides juxta numerum tribuum filiorum Jacob, ad quem factus est sermo Domini, dicens~: Isra\"el erit nomen tuum.
${}^{32}$~Et \ae dificavit de lapidibus altare in nomine Domini~: fecitque aqu\ae ductum, quasi per duas aratiunculas in circuitu altaris,
${}^{33}$~et composuit ligna~: divisitque per membra bovem, et posuit super ligna,
${}^{34}$~et ait~: Implete quatuor hydrias aqua, et fundite super holocaustum et super ligna. Rursumque dixit~: Etiam secundo hoc facite. Qui cum fecissent secundo, ait~: Etiam tertio idipsum facite. Feceruntque tertio,
${}^{35}$~et currebant aqu\ae\ circum altare, et fossa aqu\ae ductus repleta est.
${}^{36}$~Cumque jam tempus esset ut offerretur holocaustum, accedens Elias propheta ait~: Domine Deus Abraham, et Isaac, et Isra\"el, ostende hodie quia tu es Deus Isra\"el, et ego servus tuus, et juxta pr\ae ceptum tuum feci omnia verba h\ae c.
${}^{37}$~Exaudi me, Domine, exaudi me~: ut discat populus iste quia tu es Dominus Deus, et tu convertisti cor eorum iterum.
${}^{38}$~Cecidit autem ignis Domini, et voravit holocaustum, et ligna, et lapides, pulverem quoque, et aquam qu\ae\ erat in aqu\ae ductu lambens.
${}^{39}$~Quod cum vidisset omnis populus, cecidit in faciem suam, et ait~: Dominus ipse est Deus, Dominus ipse est Deus.
${}^{40}$~Dixitque Elias ad eos~: Apprehendite prophetas Baal, et ne unus quidem effugiat ex eis. Quos cum apprehendissent, duxit eos Elias ad torrentem Cison, et interfecit eos ibi.


${}^{41}$~Et ait Elias ad Achab~: Ascende, comede, et bibe, quia sonus mult\ae\ pluvi\ae\ est.
${}^{42}$~Ascendit Achab ut comederet et biberet~: Elias autem ascendit in verticem Carmeli, et pronus in terram posuit faciem suam inter genua sua,
${}^{43}$~et dixit ad puerum suum~: Ascende, et prospice contra mare. Qui cum ascendisset, et contemplatus esset, ait~: Non est quidquam. Et rursum ait illi~: Revertere septem vicibus.
${}^{44}$~In septima autem vice, ecce nubecula parva quasi vestigium hominis ascendebat de mari. Qui ait~: Ascende, et dic Achab~: Junge currum tuum et descende, ne occupet te pluvia.
${}^{45}$~Cumque se verteret huc atque illuc, ecce c\ae li contenebrati sunt, et nubes, et ventus, et facta est pluvia grandis. Ascendens itaque Achab, abiit in Jezrahel~:
${}^{46}$~et manus Domini facta est super Eliam, accinctisque lumbis currebat ante Achab, donec veniret in Jezrahel.
\Needspace{2.5\baselineskip}\versal{19}~\lettrine[lines=10,image=true,loversize=0.05,lraise=-0.03]{N}{}untiavit autem Achab Jezabel omnia qu\ae\ fecerat Elias, et quomodo occidisset universos prophetas gladio.
${}^{2}$~Misitque Jezabel nuntium ad Eliam, dicens~: H\ae c mihi faciant dii, et h\ae c addant, nisi hac hora cras posuero animam tuam sicut animam unius ex illis.
${}^{3}$~Timuit ergo Elias, et surgens abiit quocumque eum ferebat voluntas~: venitque in Bersabee Juda, et dimisit ibi puerum suum,
${}^{4}$~et perrexit in desertum, viam unius diei. Cumque venisset, et sederet subter unam juniperum, petivit anim\ae\ su\ae\ ut moreretur, et ait~: Sufficit mihi, Domine~: tolle animam meam~: neque enim melior sum quam patres mei.
${}^{5}$~Projecitque se, et obdormivit in umbra juniperi~: et ecce angelus Domini tetigit eum, et dixit illi~: Surge, et comede.
${}^{6}$~Respexit, et ecce ad caput suum subcinericius panis, et vas aqu\ae~: comedit ergo, et bibit, et rursum obdormivit.
${}^{7}$~Reversusque est angelus Domini secundo, et tetigit eum, dixitque illi~: Surge, comede~: grandis enim tibi restat via.
${}^{8}$~Qui cum surrexisset, comedit et bibit, et ambulavit in fortitudine cibi illius quadraginta diebus et quadraginta noctibus usque ad montem Dei Horeb.


${}^{9}$~Cumque venisset illuc, mansit in spelunca~: et ecce sermo Domini ad eum, dixitque illi~: Quid hic agis, Elia~?
${}^{10}$~At ille respondit~: Zelo zelatus sum pro Domino Deo exercituum, quia dereliquerunt pactum tuum filii Isra\"el~: altaria tua destruxerunt, prophetas tuos occiderunt gladio, derelictus sum ego solus, et qu\ae runt animam meam ut auferant eam.
${}^{11}$~Et ait ei~: Egredere, et sta in monte coram Domino~: et ecce Dominus transit. Et spiritus grandis et fortis subvertens montes, et conterens petras, ante Dominum~: non in spiritu Dominus. Et post spiritum commotio~: non in commotione Dominus.
${}^{12}$~Et post commotionem ignis~: non in igne Dominus. Et post ignem sibilus aur\ae\ tenuis.
${}^{13}$~Quod cum audisset Elias, operuit vultum suum pallio, et egressus stetit in ostio spelunc\ae . Et ecce vox ad eum dicens~: Quid hic agis, Elia~? Et ille respondit~:
${}^{14}$~Zelo zelatus sum pro Domino Deo exercituum, quia dereliquerunt pactum tuum filii Isra\"el~: altaria tua destruxerunt, prophetas tuos occiderunt gladio, derelictus sum ego solus, et qu\ae runt animam meam ut auferant eam.
${}^{15}$~Et ait Dominus ad eum~: Vade, et revertere in viam tuam per desertum in Damascum~: cumque perveneris illuc, unges Haza\"el regem super Syriam,
${}^{16}$~et Jehu filium Namsi unges regem super Isra\"el~: Eliseum autem filium Saphat, qui est de Abelmehula, unges prophetam pro te.
${}^{17}$~Et erit~: quicumque fugerit gladium Haza\"el, occidet eum Jehu~: et quicumque fugerit gladium Jehu, interficiet eum Eliseus.
${}^{18}$~Et derelinquam mihi in Isra\"el septem millia virorum, quorum genua non sunt incurvata ante Baal, et omne os quod non adoravit eum osculans manus.


${}^{19}$~Profectus ergo inde Elias, reperit Eliseum filium Saphat, arantem in duodecim jugis boum. Et ipse in duodecim jugis boum arantibus unus erat~: cumque venisset Elias ad eum, misit pallium suum super illum.
${}^{20}$~Qui statim relictis bobus cucurrit post Eliam, et ait~: Osculer, oro, patrem meum, et matrem meam, et sic sequar te. Dixitque ei~: Vade, et revertere~: quod enim meum erat, feci tibi.
${}^{21}$~Reversus autem ab eo, tulit par boum, et mactavit illud, et in aratro boum coxit carnes, et dedit populo, et comederunt~: consurgensque abiit, et secutus est Eliam, et ministrabat ei.
\Needspace{2.5\baselineskip}\versal{20}~\lettrine[lines=10,image=true,loversize=0.05,lraise=-0.03]{P}{}orro Benadad rex Syri\ae\ congregavit omnem exercitum suum, et triginta duos reges secum, et equos, et currus~: et ascendens pugnabat contra Samariam, et obsidebat eam.
${}^{2}$~Mittensque nuntios ad Achab regem Isra\"el in civitatem,
${}^{3}$~ait~: H\ae c dicit Benadad~: Argentum tuum, et aurum tuum meum est~: et uxores tu\ae , et filii tui optimi, mei sunt.
${}^{4}$~Responditque rex Isra\"el~: Juxta verbum tuum, domine mi rex, tuus sum ego, et omnia mea.
${}^{5}$~Revertentesque nuntii, dixerunt~: H\ae c dicit Benadad, qui misit nos ad te~: Argentum tuum, et aurum tuum, et uxores tuas, et filios tuos, dabis mihi.
${}^{6}$~Cras igitur hac eadem hora mittam servos meos ad te, et scrutabuntur domum tuam, et domum servorum tuorum~: et omne quod eis placuerit, ponent in manibus suis, et auferent.
${}^{7}$~Vocavit autem rex Isra\"el omnes seniores terr\ae , et ait~: Animadvertite, et videte quoniam insidietur nobis~: misit enim ad me pro uxoribus meis, et filiis, et pro argento et auro~: et non abnui.
${}^{8}$~Dixeruntque omnes majores natu, et universus populus, ad eum~: Non audias, neque acquiescas illi.
${}^{9}$~Respondit itaque nuntiis Benadad~: Dicite domino meo regi~: Omnia propter qu\ae\ misisti ad me servum tuum in initio, faciam~: hanc autem rem facere non possum.
${}^{10}$~Reversique nuntii retulerunt ei. Qui remisit, et ait~: H\ae c faciant mihi dii, et h\ae c addant, si suffecerit pulvis Samari\ae\ pugillis omnis populi qui sequitur me.
${}^{11}$~Et respondens rex Isra\"el, ait~: Dicite ei~: Ne glorietur, accinctus \ae que ut discinctus.
${}^{12}$~Factum est autem cum audisset Benadad verbum istud, bibebat ipse et reges in umbraculis~: et ait servis suis~: Circumdate civitatem. Et circumdederunt eam.


${}^{13}$~Et ecce propheta unus accedens ad Achab regem Isra\"el, ait ei~: H\ae c dicit Dominus~: Certe vidisti omnem multitudinem hanc nimiam~? ecce ego tradam eam in manu tua hodie, ut scias quia ego sum Dominus.
${}^{14}$~Et ait Achab~: Per quem~? Dixitque ei~: H\ae c dicit Dominus~: Per pedissequos principum provinciarum. Et ait~: Quis incipiet pr\ae liari~? Et ille dixit~: Tu.
${}^{15}$~Recensuit ergo pueros principum provinciarum, et reperit numerum ducentorum triginta duorum~: et recensuit post eos populum, omnes filios Isra\"el, septem millia.
${}^{16}$~Et egressi sunt meridie. Benadad autem bibebat temulentus in umbraculo suo, et reges triginta duo cum eo, qui ad auxilium ejus venerant.
${}^{17}$~Egressi sunt autem pueri principum provinciarum in prima fronte. Misit itaque Benadad~: qui nuntiaverunt ei, dicentes~: Viri egressi sunt de Samaria.
${}^{18}$~Et ille ait~: Sive pro pace veniunt, apprehendite eos vivos~: sive ut pr\ae lientur, vivos eos capite.
${}^{19}$~Egressi sunt ergo pueri principum provinciarum, ac reliquus exercitus sequebatur~:
${}^{20}$~et percussit unusquisque virum qui contra se veniebat~: fugeruntque Syri, et persecutus est eos Isra\"el. Fugit quoque Benadad rex Syri\ae\ in equo cum equitibus suis.
${}^{21}$~Necnon egressus rex Isra\"el percussit equos et currus, et percussit Syriam plaga magna.
${}^{22}$~Accedens autem propheta ad regem Isra\"el, dixit ei~: Vade, et confortare, et scito, et vide quid facias~: sequenti enim anno rex Syri\ae\ ascendet contra te.


${}^{23}$~Servi vero regis Syri\ae\ dixerunt ei~: Dii montium sunt dii eorum, ideo superaverunt nos~: sed melius est ut pugnemus contra eos in campestribus, et obtinebimus eos.
${}^{24}$~Tu ergo verbum hoc fac~: amove reges singulos ab exercitu tuo, et pone principes pro eis~:
${}^{25}$~et instaura numerum militum qui ceciderunt de tuis, et equos secundum equos pristinos, et currus secundum currus quos ante habuisti~: et pugnabimus contra eos in campestribus, et videbis quod obtinebimus eos. Credidit consilio eorum, et fecit ita.
${}^{26}$~Igitur postquam annus transierat, recensuit Benadad Syros, et ascendit in Aphec ut pugnaret contra Isra\"el.
${}^{27}$~Porro filii Isra\"el recensiti sunt, et acceptis cibariis profecti ex adverso, castraque metati sunt contra eos, quasi duo parvi greges caprarum~: Syri autem repleverunt terram.
${}^{28}$~(Et accedens unus vir Dei, dixit ad regem Isra\"el~: H\ae c dicit Dominus~: Quia dixerunt Syri~: Deus montium est Dominus, et non est Deus vallium~: dabo omnem multitudinem hanc grandem in manu tua, et scietis quia ego sum Dominus.)
${}^{29}$~Dirigebantque septem diebus ex adverso hi atque illi acies, septima autem die commissum est bellum~: percusseruntque filii Isra\"el de Syris centum millia peditum in die una.
${}^{30}$~Fugerunt autem qui remanserant in Aphec, in civitatem~: et cecidit murus super viginti septem millia hominum qui remanserant. Porro Benadad fugiens ingressus est civitatem, in cubiculum quod erat intra cubiculum.
${}^{31}$~Dixeruntque ei servi sui~: Ecce, audivimus quod reges domus Isra\"el clementes sint~: ponamus itaque saccos in lumbis nostris, et funiculos in capitibus nostris, et egrediamur ad regem Isra\"el~: forsitan salvabit animas nostras.
${}^{32}$~Accinxerunt saccis lumbos suos, et posuerunt funiculos in capitibus suis, veneruntque ad regem Isra\"el, et dixerunt ei~: Servus tuus Benadad dicit~: Vivat, oro te, anima mea. Et ille ait~: Si adhuc vivit, frater meus est.
${}^{33}$~Quod acceperunt viri pro omine~: et festinantes rapuerunt verbum ex ore ejus, atque dixerunt~: Frater tuus Benadad. Et dixit eis~: Ite, et adducite eum ad me. Egressus est ergo ad eum Benadad, et levavit eum in currum suum.
${}^{34}$~Qui dixit ei~: Civitates quas tulit pater meus a patre tuo, reddam~: et plateas fac tibi in Damasco, sicut fecit pater meus in Samaria, et ego fœderatus recedam a te. Pepigit ergo fœdus, et dimisit eum.


${}^{35}$~Tunc vir quidam de filiis prophetarum dixit ad socium suum in sermone Domini~: Percute me. At ille noluit percutere.
${}^{36}$~Cui ait~: Quia noluisti audire vocem Domini, ecce recedes a me, et percutiet te leo. Cumque paululum recessisset ab eo, invenit eum leo, atque percussit.
${}^{37}$~Sed et alterum inveniens virum, dixit ad eum~: Percute me. Qui percussit eum, et vulneravit.
${}^{38}$~Abiit ergo propheta, et occurrit regi in via, et mutavit aspersione pulveris os et oculos suos.
${}^{39}$~Cumque rex transisset, clamavit ad regem, et ait~: Servus tuus egressus est ad pr\ae liandum cominus~: cumque fugisset vir unus, adduxit eum quidam ad me, et ait~: Custodi virum istum~: qui si lapsus fuerit, erit anima tua pro anima ejus, aut talentum argenti appendes.
${}^{40}$~Dum autem ego turbatus huc illucque me verterem, subito non comparuit. Et ait rex Isra\"el ad eum~: Hoc est judicium tuum, quod ipse decrevisti.
${}^{41}$~At ille statim abstersit pulverem de facie sua, et cognovit eum rex Isra\"el, quod esset de prophetis.
${}^{42}$~Qui ait ad eum~: H\ae c dicit Dominus~: Quia dimisisti virum dignum morte de manu tua, erit anima tua pro anima ejus, et populus tuus pro populo ejus.
${}^{43}$~Reversus est igitur rex Isra\"el in domum suam, audire contemnens, et furibundus venit in Samariam.
\Needspace{2.5\baselineskip}\versal{21}~\lettrine[lines=10,image=true,loversize=0.05,lraise=-0.03]{P}{}ost verba autem h\ae c, tempore illo vinea erat Naboth Jezrahelit\ae , qu\ae\ erat in Jezrahel, juxta palatium Achab regis Samari\ae .
${}^{2}$~Locutus est ergo Achab ad Naboth, dicens~: Da mihi vineam tuam, ut faciam mihi hortum olerum, quia vicina est, et prope domum meam~: daboque tibi pro ea vineam meliorem, aut si commodius tibi putas, argenti pretium, quanto digna est.
${}^{3}$~Cui respondit Naboth~: Propitius sit mihi Dominus, ne dem h\ae reditatem patrum meorum tibi.
${}^{4}$~Venit ergo Achab in domum suam indignans, et frendens super verbo quod locutus fuerat ad eum Naboth Jezrahelites, dicens~: Non dabo tibi h\ae reditatem patrum meorum. Et projiciens se in lectulum suum, avertit faciem suam ad parietem, et non comedit panem.
${}^{5}$~Ingressa est autem ad eum Jezabel uxor sua, dixitque ei~: Quid est hoc, unde anima tua contristata est~? et quare non comedis panem~?
${}^{6}$~Qui respondit ei~: Locutus sum Naboth Jezrahelit\ae , et dixi ei~: Da mihi vineam tuam, accepta pecunia~: aut, si tibi placet, dabo tibi vineam meliorem pro ea. Et ille ait~: Non dabo tibi vineam meam.
${}^{7}$~Dixit ergo ad eum Jezabel uxor ejus~: Grandis auctoritatis es, et bene regis regnum Isra\"el. Surge, et comede panem, et \ae quo animo esto~: ego dabo tibi vineam Naboth Jezrahelit\ae .


${}^{8}$~Scripsit itaque litteras ex nomine Achab, et signavit eas annulo ejus, et misit ad majores natu, et optimates, qui erant in civitate ejus, et habitabant cum Naboth.
${}^{9}$~Litterarum autem h\ae c erat sententia~: Pr\ae dicate jejunium, et sedere facite Naboth inter primos populi~:
${}^{10}$~et submittite duos viros filios Belial contra eum, et falsum testimonium dicant~: Benedixit Deum et regem~: et educite eum, et lapidate, sicque moriatur.
${}^{11}$~Fecerunt ergo cives ejus majores natu et optimates, qui habitabant cum eo in urbe, sicut pr\ae ceperat eis Jezabel, et sicut scriptum erat in litteris quas miserat ad eos~:
${}^{12}$~pr\ae dicaverunt jejunium, et sedere fecerunt Naboth inter primos populi.
${}^{13}$~Et adductis duobus viris filiis diaboli, fecerunt eos sedere contra eum~: at illi, scilicet ut viri diabolici, dixerunt contra eum testimonium coram multitudine~: Benedixit Naboth Deum et regem~: quam ob rem eduxerunt eum extra civitatem, et lapidibus interfecerunt.
${}^{14}$~Miseruntque ad Jezabel, dicentes~: Lapidatus est Naboth, et mortuus est.
${}^{15}$~Factum est autem, cum audisset Jezabel lapidatum Naboth et mortuum, locuta est ad Achab~: Surge, et posside vineam Naboth Jezrahelit\ae , qui noluit tibi acquiescere, et dare eam accepta pecunia~: non enim vivit Naboth, sed mortuus est.
${}^{16}$~Quod cum audisset Achab, mortuum videlicet Naboth, surrexit, et descendebat in vineam Naboth Jezrahelit\ae , ut possideret eam.


${}^{17}$~Factus est igitur sermo Domini ad Eliam Thesbiten, dicens~:
${}^{18}$~Surge, et descende in occursum Achab regis Isra\"el, qui est in Samaria~: ecce ad vineam Naboth descendit, ut possideat eam.
${}^{19}$~Et loqueris ad eum, dicens~: H\ae c dicit Dominus~: Occidisti, insuper et possedisti. Et post h\ae c addes~: H\ae c dicit Dominus~: In loco hoc, in quo linxerunt canes sanguinem Naboth, lambent quoque sanguinem tuum.
${}^{20}$~Et ait Achab ad Eliam~: Num invenisti me inimicum tibi~? Qui dixit~: Inveni, eo quod venundatus sis, ut faceres malum in conspectu Domini.
${}^{21}$~Ecce ego inducam super te malum, et demetam posteriora tua, et interficiam de Achab mingentem ad parietem, et clausum et ultimum in Isra\"el.
${}^{22}$~Et dabo domum tuam sicut domum Jeroboam filii Nabat, et sicut domum Baasa filii Ahia~: quia egisti ut me ad iracundiam provocares, et peccare fecisti Isra\"el.
${}^{23}$~Sed et de Jezabel locutus est Dominus, dicens~: Canes comedent Jezabel in agro Jezrahel.
${}^{24}$~Si mortuus fuerit Achab in civitate, comedent eum canes~: si autem mortuus fuerit in agro, comedent eum volucres c\ae li.
${}^{25}$~Igitur non fuit alter talis sicut Achab, qui venundatus est ut faceret malum in conspectu Domini~: concitavit enim eum Jezabel uxor sua,
${}^{26}$~et abominabilis factus est, in tantum ut sequeretur idola qu\ae\ fecerant Amorrh\ae i, quos consumpsit Dominus a facie filiorum Isra\"el.
${}^{27}$~Itaque cum audisset Achab sermones istos, scidit vestimenta sua, et operuit cilicio carnem suam, jejunavitque et dormivit in sacco, et ambulavit demisso capite.
${}^{28}$~Et factus est sermo Domini ad Eliam Thesbiten, dicens~:
${}^{29}$~Nonne vidisti humiliatum Achab coram me~? quia igitur humiliatus est mei causa, non inducam malum in diebus ejus, sed in diebus filii sui inferam malum domui ejus.
\Needspace{2.5\baselineskip}\versal{22}~\lettrine[lines=10,image=true,loversize=0.05,lraise=-0.03]{T}{}ransierunt igitur tres anni absque bello inter Syriam et Isra\"el.
${}^{2}$~In anno autem tertio, descendit Josaphat rex Juda ad regem Isra\"el.
${}^{3}$~(Dixitque rex Isra\"el ad servos suos~: Ignoratis quod nostra sit Ramoth Galaad, et negligimus tollere eam de manu regis Syri\ae~?)
${}^{4}$~Et ait ad Josaphat~: Veniesne mecum ad pr\ae liandum in Ramoth Galaad~?
${}^{5}$~Dixitque Josaphat ad regem Isra\"el~: Sicut ego sum, ita et tu~: populus meus et populus tuus unum sunt~: et equites mei, equites tui. Dixitque Josaphat ad regem Isra\"el~: Qu\ae re, oro te, hodie sermonem Domini.
${}^{6}$~Congregavit ergo rex Isra\"el prophetas, quadringentos circiter viros, et ait ad eos~: Ire debeo in Ramoth Galaad ad bellandum, an quiescere~? Qui responderunt~: Ascende, et dabit eam Dominus in manu regis.
${}^{7}$~Dixit autem Josaphat~: Non est hic propheta Domini quispiam, ut interrogemus per eum~?
${}^{8}$~Et ait rex Isra\"el ad Josaphat~: Remansit vir unus per quem possumus interrogare Dominum~: sed ego odi eum, quia non prophetat mihi bonum, sed malum~: Mich\ae as filius Jemla. Cui Josaphat ait~: Ne loquaris ita, rex.
${}^{9}$~Vocavit ergo rex Isra\"el eunuchum quemdam, et dixit ei~: Festina adducere Mich\ae am filium Jemla.


${}^{10}$~Rex autem Isra\"el, et Josaphat rex Juda, sedebant unusquisque in solio suo, vestiti cultu regio, in area juxta ostium port\ae\ Samari\ae~: et universi prophet\ae\ prophetabant in conspectu eorum.
${}^{11}$~Fecit quoque sibi Sedecias filius Chanaana cornua ferrea, et ait~: H\ae c dicit Dominus~: His ventilabis Syriam, donec deleas eam.
${}^{12}$~Omnesque prophet\ae\ similiter prophetabant, dicentes~: Ascende in Ramoth Galaad, et vade prospere, et tradet Dominus in manus regis.
${}^{13}$~Nuntius vero qui ierat ut vocaret Mich\ae am, locutus est ad eum, dicens~: Ecce sermones prophetarum ore uno regi bona pr\ae dicant~: sit ergo sermo tuus similis eorum, et loquere bona.
${}^{14}$~Cui Mich\ae as ait~: Vivit Dominus, quia quodcumque dixerit mihi Dominus, hoc loquar.
${}^{15}$~Venit itaque ad regem, et ait illi rex~: Mich\ae a, ire debemus in Ramoth Galaad ad pr\ae liandum, an cessare~? Cui ille respondit~: Ascende, et vade prospere, et tradet eam Dominus in manus regis.
${}^{16}$~Dixit autem rex ad eum~: Iterum atque iterum adjuro te, ut non loquaris mihi nisi quod verum est, in nomine Domini.
${}^{17}$~Et ille ait~: Vidi cunctum Isra\"el dispersum in montibus, quasi oves non habentes pastorem. Et ait Dominus~: Non habent isti dominum~: revertatur unusquisque in domum suam in pace.
${}^{18}$~(Dixit ergo rex Isra\"el ad Josaphat~: Numquid non dixi tibi, quia non prophetat mihi bonum, sed semper malum~?)
${}^{19}$~Ille vero addens, ait~: Propterea audi sermonem Domini~: vidi Dominum sedentem super solium suum, et omnem exercitum c\ae li assistentem ei a dextris et a sinistris~:
${}^{20}$~et ait Dominus~: Quis decipiet Achab regem Isra\"el, ut ascendat, et cadat in Ramoth Galaad~? Et dixit unus verba hujuscemodi, et alius aliter.
${}^{21}$~Egressus est autem spiritus, et stetit coram Domino, et ait~: Ego decipiam illum. Cui locutus est Dominus~: In quo~?
${}^{22}$~Et ille ait~: Egrediar, et ero spiritus mendax in ore omnium prophetarum ejus. Et dixit Dominus~: Decipies, et pr\ae valebis~: egredere, et fac ita.
${}^{23}$~Nunc igitur ecce dedit Dominus spiritum mendacii in ore omnium prophetarum tuorum, qui hic sunt, et Dominus locutus est contra te malum.
${}^{24}$~Accessit autem Sedecias filius Chanaana, et percussit Mich\ae am in maxillam, et dixit~: Mene ergo dimisit spiritus Domini, et locutus est tibi~?
${}^{25}$~Et ait Mich\ae as~: Visurus es in die illa quando ingredieris cubiculum intra cubiculum ut abscondaris.
${}^{26}$~Et ait rex Isra\"el~: Tollite Mich\ae am, et maneat apud Amon principem civitatis, et apud Joas filium Amelech,
${}^{27}$~et dicite eis~: H\ae c dicit rex~: Mittite virum istum in carcerem, et sustentate eum pane tribulationis, et aqua angusti\ae , donec revertar in pace.
${}^{28}$~Dixitque Mich\ae as~: Si reversus fueris in pace, non est locutus in me Dominus. Et ait~: Audite, populi omnes.
${}^{29}$~Ascendit itaque rex Isra\"el, et Josaphat rex Juda, in Ramoth Galaad.
${}^{30}$~Dixit itaque rex Isra\"el ad Josaphat~: Sume arma, et ingredere pr\ae lium, et induere vestibus tuis. Porro rex Isra\"el mutavit habitum suum, et ingressus est bellum.
${}^{31}$~Rex autem Syri\ae\ pr\ae ceperat principibus curruum triginta duobus, dicens~: Non pugnabitis contra minorem et majorem quempiam, nisi contra regem Isra\"el solum.
${}^{32}$~Cum ergo vidissent principes curruum Josaphat, suspicati sunt quod ipse esset rex Isra\"el, et impetu facto pugnabant contra eum~: et exclamavit Josaphat.
${}^{33}$~Intellexeruntque principes curruum quod non esset rex Isra\"el, et cessaverunt ab eo.
${}^{34}$~Vir autem quidam tetendit arcum, in incertum sagittam dirigens, et casu percussit regem Isra\"el inter pulmonem et stomachum. At ille dixit aurig\ae\ suo~: Verte manum tuam, et ejice me de exercitu, quia graviter vulneratus sum.
${}^{35}$~Commissum est ergo pr\ae lium in die illa, et rex Isra\"el stabat in curru suo contra Syros, et mortuus est vespere~: fluebat autem sanguis plag\ae\ in sinum currus,
${}^{36}$~et pr\ae co insonuit in universo exercitu antequam sol occumberet, dicens~: Unusquisque revertatur in civitatem, et in terram suam.
${}^{37}$~Mortuus est autem rex, et perlatus est in Samariam~: sepelieruntque regem in Samaria,
${}^{38}$~et laverunt currum ejus in piscina Samari\ae~: et linxerunt canes sanguinem ejus, et habenas laverunt, juxta verbum Domini quod locutus fuerat.
${}^{39}$~Reliqua autem sermonum Achab, et universa qu\ae\ fecit, et domus eburnea quam \ae dificavit, cunctarumque urbium quas exstruxit, nonne h\ae c scripta sunt in libro sermonum dierum regum Isra\"el~?
${}^{40}$~Dormivit ergo Achab cum patribus suis, et regnavit Ochozias filius ejus pro eo.


${}^{41}$~Josaphat vero filius Asa regnare cœperat super Judam anno quarto Achab regis Isra\"el.
${}^{42}$~Triginta quinque annorum erat cum regnare cœpisset, et viginti quinque annis regnavit in Jerusalem~: nomen matris ejus Azuba filia Salai.
${}^{43}$~Et ambulavit in omni via Asa patris sui, et non declinavit ex ea~: fecitque quod rectum erat in conspectu Domini.
${}^{44}$~Verumtamen excelsa non abstulit~: adhuc enim populus sacrificabat, et adolebat incensum in excelsis.
${}^{45}$~Pacemque habuit Josaphat cum rege Isra\"el.
${}^{46}$~Reliqua autem verborum Josaphat, et opera ejus qu\ae\ gessit, et pr\ae lia, nonne h\ae c scripta sunt in libro verborum dierum regum Juda~?
${}^{47}$~Sed et reliquias effeminatorum qui remanserant in diebus Asa patris ejus, abstulit de terra.
${}^{48}$~Nec erat tunc rex constitutus in Edom.
${}^{49}$~Rex vero Josaphat fecerat classes in mari, qu\ae\ navigarent in Ophir propter aurum~: et ire non potuerunt, quia confract\ae\ sunt in Asiongaber.
${}^{50}$~Tunc ait Ochozias filius Achab ad Josaphat~: Vadant servi mei cum servis tuis in navibus. Et noluit Josaphat.
${}^{51}$~Dormivitque Josaphat cum patribus suis, et sepultus est cum eis in civitate David patris sui~: regnavitque Joram filius ejus pro eo.


${}^{52}$~Ochozias autem filius Achab regnare cœperat super Isra\"el in Samaria, anno septimodecimo Josaphat regis Juda~: regnavitque super Isra\"el duobus annis.
${}^{53}$~Et fecit malum in conspectu Domini, et ambulavit in via patris sui et matris su\ae , et in via Jeroboam filii Nabat, qui peccare fecit Isra\"el.
${}^{54}$~Servivit quoque Baal, et adoravit eum, et irritavit Dominum Deum Isra\"el, juxta omnia qu\ae\ fecerat pater ejus.
