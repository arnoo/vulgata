\clearpage
{\centering \section*{Liber Proverbiorum}}\thispagestyle{empty}
\addcontentsline{toc}{subsection}{Proverbia}
\fancyhead[C]{\textsc{Proverbia}}

\Needspace{2.5\baselineskip}\versal{1}\begin{flushleft}\begin{verse}\vspace{-11pt}Parabol\ae\ Salomonis, filii David, regis Isra\"el,\\
${}^{2}$~ad sciendam sapientiam et disciplinam~;\\
${}^{3}$~ad intelligenda verba prudenti\ae ,\\ et suscipiendam eruditionem doctrin\ae ,\\ justitiam, et judicium, et \ae quitatem~:\\
${}^{4}$~ut detur parvulis astutia,\\ adolescenti scientia et intellectus.\\
${}^{5}$~Audiens sapiens, sapientior erit,\\ et intelligens gubernacula possidebit.\\
${}^{6}$~Animadvertet parabolam et interpretationem,\\ verba sapientum et \ae nigmata eorum.\\
${}^{7}$~Timor Domini principium sapienti\ae~;\\ sapientiam atque doctrinam stulti despiciunt.\end{verse}\end{flushleft}


\begin{flushleft}\begin{verse}${}^{8}$~Audi, fili mi, disciplinam patris tui,\\ et ne dimittas legem matris tu\ae~:\\
${}^{9}$~ut addatur gratia capiti tuo,\\ et torques collo tuo.\\
${}^{10}$~Fili mi, si te lactaverint peccatores,\\ ne acquiescas eis.\\
${}^{11}$~Si dixerint~: Veni nobiscum, insidiemur sanguini~;\\ abscondamus tendiculas contra insontem frustra~;\\
${}^{12}$~deglutiamus eum sicut infernus viventem,\\ et integrum quasi descendentem in lacum~;\\
${}^{13}$~omnem pretiosam substantiam reperiemus~;\\ implebimus domos nostras spoliis~:\\
${}^{14}$~sortem mitte nobiscum,\\ marsupium unum sit omnium nostrum~:\\
${}^{15}$~fili mi, ne ambules cum eis~;\\ prohibe pedem tuum a semitis eorum~:\\
${}^{16}$~pedes enim illorum ad malum currunt,\\ et festinant ut effundant sanguinem.\\
${}^{17}$~Frustra autem jacitur rete\\ ante oculos pennatorum.\\
${}^{18}$~Ipsi quoque contra sanguinem suum insidiantur,\\ et moliuntur fraudes contra animas suas.\\
${}^{19}$~Sic semit\ae\ omnis avari~:\\ animas possidentium rapiunt.\end{verse}\end{flushleft}


\begin{flushleft}\begin{verse}${}^{20}$~Sapientia foris pr\ae dicat~;\\ in plateis dat vocem suam~:\\
${}^{21}$~in capite turbarum clamitat~;\\ in foribus portarum urbis profert verba sua, dicens~:\\
${}^{22}$~Usquequo, parvuli, diligitis infantiam,\\ et stulti ea qu\ae\ sibi sunt noxia cupient,\\ et imprudentes odibunt scientiam~?\\
${}^{23}$~convertimini ad correptionem meam.\\ En proferam vobis spiritum meum,\\ et ostendam vobis verba mea.\\
${}^{24}$~Quia vocavi, et renuistis~;\\ extendi manum meam, et non fuit qui aspiceret~:\\
${}^{25}$~despexistis omne consilium meum,\\ et increpationes meas neglexistis.\\
${}^{26}$~Ego quoque in interitu vestro ridebo,\\ et subsannabo cum vobis id quod timebatis advenerit.\\
${}^{27}$~Cum irruerit repentina calamitas,\\ et interitus quasi tempestas ingruerit~;\\ quando venerit super vos tribulatio et angustia~:\\
${}^{28}$~tunc invocabunt me, et non exaudiam~;\\ mane consurgent, et non invenient me~:\\
${}^{29}$~eo quod exosam habuerint disciplinam,\\ et timorem Domini non susceperint,\\
${}^{30}$~nec acquieverint consilio meo,\\ et detraxerint univers\ae\ correptioni me\ae .\\
${}^{31}$~Comedent igitur fructus vi\ae\ su\ae ,\\ suisque consiliis saturabuntur.\\
${}^{32}$~Aversio parvulorum interficiet eos,\\ et prosperitas stultorum perdet illos.\\
${}^{33}$~Qui autem me audierit, absque terrore requiescet,\\ et abundantia perfruetur, timore malorum sublato.\end{verse}\end{flushleft}


\Needspace{2.5\baselineskip}\versal{2}\begin{flushleft}\begin{verse}\vspace{-19pt}Fili mi, si susceperis sermones meos,\\ et mandata mea absconderis penes te~:\\
${}^{2}$~ut audiat sapientiam auris tua,\\ inclina cor tuum ad cognoscendam prudentiam.\\
${}^{3}$~Si enim sapientiam invocaveris,\\ et inclinaveris cor tuum prudenti\ae~;\\
${}^{4}$~si qu\ae sieris eam quasi pecuniam,\\ et sicut thesauros effoderis illam~:\\
${}^{5}$~tunc intelliges timorem Domini,\\ et scientiam Dei invenies,\\
${}^{6}$~quia Dominus dat sapientiam,\\ et ex ore ejus prudentia et scientia.\\
${}^{7}$~Custodiet rectorum salutem,\\ et proteget gradientes simpliciter,\\
${}^{8}$~servans semitas justiti\ae ,\\ et vias sanctorum custodiens.\\
${}^{9}$~Tunc intelliges justitiam, et judicium,\\ et \ae quitatem, et omnem semitam bonam.\end{verse}\end{flushleft}


\begin{flushleft}\begin{verse}${}^{10}$~Si intraverit sapientia cor tuum,\\ et scientia anim\ae\ tu\ae\ placuerit,\\
${}^{11}$~consilium custodiet te,\\ et prudentia servabit te~:\\
${}^{12}$~ut eruaris a via mala,\\ et ab homine qui perversa loquitur~;\\
${}^{13}$~qui relinquunt iter rectum,\\ et ambulant per vias tenebrosas~;\\
${}^{14}$~qui l\ae tantur cum malefecerint,\\ et exsultant in rebus pessimis~;\\
${}^{15}$~quorum vi\ae\ pervers\ae\ sunt,\\ et infames gressus eorum.\\
${}^{16}$~Ut eruaris a muliere aliena,\\ et ab extranea qu\ae\ mollit sermones suos,\\
${}^{17}$~et relinquit ducem pubertatis su\ae ,\\
${}^{18}$~et pacti Dei sui oblita est.\\ Inclinata est enim ad mortem domus ejus,\\ et ad inferos semit\ae\ ipsius.\\
${}^{19}$~Omnes qui ingrediuntur ad eam non revertentur,\\ nec apprehendent semitas vit\ae .\\
${}^{20}$~Ut ambules in via bona,\\ et calles justorum custodias~:\\
${}^{21}$~qui enim recti sunt habitabunt in terra,\\ et simplices permanebunt in ea~;\\
${}^{22}$~impii vero de terra perdentur,\\ et qui inique agunt auferentur ex ea.\end{verse}\end{flushleft}


\Needspace{2.5\baselineskip}\versal{3}\begin{flushleft}\begin{verse}\vspace{-19pt}Fili mi, ne obliviscaris legis me\ae ,\\ et pr\ae cepta mea cor tuum custodiat~:\\
${}^{2}$~longitudinem enim dierum, et annos vit\ae , et pacem,\\ apponent tibi.\\
${}^{3}$~Misericordia et veritas te non deserant~;\\ circumda eas gutturi tuo,\\ et describe in tabulis cordis tui~:\\
${}^{4}$~et invenies gratiam, et disciplinam bonam,\\ coram Deo et hominibus.\\
${}^{5}$~Habe fiduciam in Domino ex toto corde tuo,\\ et ne innitaris prudenti\ae\ tu\ae .\\
${}^{6}$~In omnibus viis tuis cogita illum,\\ et ipse diriget gressus tuos.\\
${}^{7}$~Ne sis sapiens apud temetipsum~;\\ time Deum, et recede a malo~:\\
${}^{8}$~sanitas quippe erit umbilico tuo,\\ et irrigatio ossium tuorum.\\
${}^{9}$~Honora Dominum de tua substantia,\\ et de primitiis omnium frugum tuarum da ei~:\\
${}^{10}$~et implebuntur horrea tua saturitate,\\ et vino torcularia tua redundabunt.\end{verse}\end{flushleft}


\begin{flushleft}\begin{verse}${}^{11}$~Disciplinam Domini, fili mi, ne abjicias,\\ nec deficias cum ab eo corriperis~:\\
${}^{12}$~quem enim diligit Dominus, corripit,\\ et quasi pater in filio complacet sibi.\\
${}^{13}$~Beatus homo qui invenit sapientiam,\\ et qui affluit prudentia.\\
${}^{14}$~Melior est acquisitio ejus negotiatione argenti,\\ et auri primi et purissimi fructus ejus.\\
${}^{15}$~Pretiosior est cunctis opibus,\\ et omnia qu\ae\ desiderantur huic non valent comparari.\\
${}^{16}$~Longitudo dierum in dextera ejus,\\ et in sinistra illius diviti\ae\ et gloria.\\
${}^{17}$~Vi\ae\ ejus vi\ae\ pulchr\ae ,\\ et omnes semit\ae\ illius pacific\ae .\\
${}^{18}$~Lignum vit\ae\ est his qui apprehenderint eam,\\ et qui tenuerit eam beatus.\\
${}^{19}$~Dominus sapientia fundavit terram~;\\ stabilivit c\ae los prudentia.\\
${}^{20}$~Sapientia illius eruperunt abyssi,\\ et nubes rore concrescunt.\end{verse}\end{flushleft}


\begin{flushleft}\begin{verse}${}^{21}$~Fili mi, ne effluant h\ae c ab oculis tuis.\\ Custodi legem atque consilium,\\
${}^{22}$~et erit vita anim\ae\ tu\ae ,\\ et gratia faucibus tuis.\\
${}^{23}$~Tunc ambulabis fiducialiter in via tua,\\ et pes tuus non impinget.\\
${}^{24}$~Si dormieris, non timebis~;\\ quiesces, et suavis erit somnus tuus.\\
${}^{25}$~Ne paveas repentino terrore,\\ et irruentes tibi potentias impiorum.\\
${}^{26}$~Dominus enim erit in latere tuo,\\ et custodiet pedem tuum, ne capiaris.\end{verse}\end{flushleft}


\begin{flushleft}\begin{verse}${}^{27}$~Noli prohibere benefacere eum qui potest~:\\ si vales, et ipse benefac.\\
${}^{28}$~Ne dicas amico tuo~: Vade, et revertere~: cras dabo tibi~:\\ cum statim possis dare.\\
${}^{29}$~Ne moliaris amico tuo malum,\\ cum ille in te habeat fiduciam.\\
${}^{30}$~Ne contendas adversus hominem frustra,\\ cum ipse tibi nihil mali fecerit.\\
${}^{31}$~Ne \ae muleris hominem injustum,\\ nec imiteris vias ejus~:\\
${}^{32}$~quia abominatio Domini est omnis illusor,\\ et cum simplicibus sermocinatio ejus.\\
${}^{33}$~Egestas a Domino in domo impii~;\\ habitacula autem justorum benedicentur.\\
${}^{34}$~Ipse deludet illusores,\\ et mansuetis dabit gratiam.\\
${}^{35}$~Gloriam sapientes possidebunt~;\\ stultorum exaltatio ignominia.\end{verse}\end{flushleft}


\Needspace{2.5\baselineskip}\versal{4}\begin{flushleft}\begin{verse}\vspace{-19pt}Audite, filii, disciplinam patris,\\ et attendite ut sciatis prudentiam.\\
${}^{2}$~Donum bonum tribuam vobis~:\\ legem meam ne derelinquatis.\\
${}^{3}$~Nam et ego filius fui patris mei,\\ tenellus et unigenitus coram matre mea.\\
${}^{4}$~Et docebat me, atque dicebat~:\\ Suscipiat verba mea cor tuum~;\\ custodi pr\ae cepta mea, et vives.\\
${}^{5}$~Posside sapientiam, posside prudentiam~:\\ ne obliviscaris, neque declines a verbis oris mei.\\
${}^{6}$~Ne dimittas eam, et custodiet te~:\\ dilige eam, et conservabit te.\\
${}^{7}$~Principium sapienti\ae~: posside sapientiam,\\ et in omni possessione tua acquire prudentiam.\\
${}^{8}$~Arripe illam, et exaltabit te~;\\ glorificaberis ab ea cum eam fueris amplexatus.\\
${}^{9}$~Dabit capiti tuo augmenta gratiarum,\\ et corona inclyta proteget te.\end{verse}\end{flushleft}


\begin{flushleft}\begin{verse}${}^{10}$~Audi, fili mi, et suscipe verba mea,\\ ut multiplicentur tibi anni vit\ae .\\
${}^{11}$~Viam sapienti\ae\ monstrabo tibi~;\\ ducam te per semitas \ae quitatis~:\\
${}^{12}$~quas cum ingressus fueris, non arctabuntur gressus tui,\\ et currens non habebis offendiculum.\\
${}^{13}$~Tene disciplinam, ne dimittas eam~;\\ custodi illam, quia ipsa est vita tua.\\
${}^{14}$~Ne delecteris in semitis impiorum,\\ nec tibi placeat malorum via.\\
${}^{15}$~Fuge ab ea, nec transeas per illam~;\\ declina, et desere eam.\\
${}^{16}$~Non enim dormiunt nisi malefecerint,\\ et rapitur somnus ab eis nisi supplantaverint.\\
${}^{17}$~Comedunt panem impietatis,\\ et vinum iniquitatis bibunt.\\
${}^{18}$~Justorum autem semita quasi lux splendens procedit,\\ et crescit usque ad perfectam diem.\\
${}^{19}$~Via impiorum tenebrosa~;\\ nesciunt ubi corruant.\\
${}^{20}$~Fili mi, ausculta sermones meos,\\ et ad eloquia mea inclina aurem tuam.\\
${}^{21}$~Ne recedant ab oculis tuis~:\\ custodi ea in medio cordis tui~:\\
${}^{22}$~vita enim sunt invenientibus ea,\\ et univers\ae\ carni sanitas.\\
${}^{23}$~Omni custodia serva cor tuum,\\ quia ex ipso vita procedit.\\
${}^{24}$~Remove a te os pravum,\\ et detrahentia labia sint procul a te.\\
${}^{25}$~Oculi tui recta videant,\\ et palpebr\ae\ tu\ae\ pr\ae cedant gressus tuos.\\
${}^{26}$~Dirige semitam pedibus tuis,\\ et omnes vi\ae\ tu\ae\ stabilientur.\\
${}^{27}$~Ne declines ad dexteram neque ad sinistram~;\\ averte pedem tuum a malo~:\\ vias enim qu\ae\ a dextris sunt novit Dominus~:\\ pervers\ae\ vero sunt qu\ae\ a sinistris sunt.\\ Ipse autem rectos faciet cursus tuos,\\ itinera autem tua in pace producet.\end{verse}\end{flushleft}


\Needspace{2.5\baselineskip}\versal{5}\begin{flushleft}\begin{verse}\vspace{-19pt}Fili mi, attende ad sapientiam meam,\\ et prudenti\ae\ me\ae\ inclina aurem tuam~:\\
${}^{2}$~ut custodias cogitationes, et disciplinam labia tua conservent.\\ Ne attendas fallaci\ae\ mulieris~;\\
${}^{3}$~favus enim distillans labia meretricis,\\ et nitidius oleo guttur ejus~:\\
${}^{4}$~novissima autem illius amara quasi absinthium,\\ et acuta quasi gladius biceps.\\
${}^{5}$~Pedes ejus descendunt in mortem,\\ et ad inferos gressus illius penetrant.\\
${}^{6}$~Per semitam vit\ae\ non ambulant~;\\ vagi sunt gressus ejus et investigabiles.\\
${}^{7}$~Nunc ergo fili mi, audi me,\\ et ne recedas a verbis oris mei.\\
${}^{8}$~Longe fac ab ea viam tuam,\\ et ne appropinques foribus domus ejus.\\
${}^{9}$~Ne des alienis honorem tuum,\\ et annos tuos crudeli~:\\
${}^{10}$~ne forte impleantur extranei viribus tuis,\\ et labores tui sint in domo aliena,\\
${}^{11}$~et gemas in novissimis,\\ quando consumpseris carnes tuas et corpus tuum,\\ et dicas~:
${}^{12}$~Cur detestatus sum disciplinam,\\ et increpationibus non acquievit cor meum,\\
${}^{13}$~nec audivi vocem docentium me,\\ et magistris non inclinavi aurem meam~?\\
${}^{14}$~pene fui in omni malo,\\ in medio ecclesi\ae\ et synagog\ae .\\
${}^{15}$~Bibe aquam de cisterna tua,\\ et fluenta putei tui~;\\
${}^{16}$~deriventur fontes tui foras,\\ et in plateis aquas tuas divide.\\
${}^{17}$~Habeto eas solus,\\ nec sint alieni participes tui.\\
${}^{18}$~Sit vena tua benedicta,\\ et l\ae tare cum muliere adolescenti\ae\ tu\ae .\\
${}^{19}$~Cerva carissima, et gratissimus hinnulus~:\\ ubera ejus inebrient te in omni tempore~;\\ in amore ejus delectare jugiter.\\
${}^{20}$~Quare seduceris, fili mi, ab aliena,\\ et foveris in sinu alterius~?\\
${}^{21}$~Respicit Dominus vias hominis,\\ et omnes gressus ejus considerat.\\
${}^{22}$~Iniquitates suas capiunt impium,\\ et funibus peccatorum suorum constringitur.\\
${}^{23}$~Ipse morietur, quia non habuit disciplinam,\\ et in multitudine stultiti\ae\ su\ae\ decipietur.\end{verse}\end{flushleft}


\Needspace{2.5\baselineskip}\versal{6}\begin{flushleft}\begin{verse}\vspace{-19pt}Fili mi, si spoponderis pro amico tuo,\\ defixisti apud extraneum manum tuam~:\\
${}^{2}$~illaqueatus es verbis oris tui,\\ et captus propriis sermonibus.\\
${}^{3}$~Fac ergo quod dico, fili mi, et temetipsum libera,\\ quia incidisti in manum proximi tui.\\ Discurre, festina, suscita amicum tuum.\\
${}^{4}$~Ne dederis somnum oculis tuis,\\ nec dormitent palpebr\ae\ tu\ae .\\
${}^{5}$~Eruere quasi damula de manu,\\ et quasi avis de manu aucupis.\end{verse}\end{flushleft}


\begin{flushleft}\begin{verse}${}^{6}$~Vade ad formicam, o piger,\\ et considera vias ejus, et disce sapientiam.\\
${}^{7}$~Qu\ae\ cum non habeat ducem,\\ nec pr\ae ceptorem, nec principem,\\
${}^{8}$~parat in \ae state cibum sibi,\\ et congregat in messe quod comedat.\\
${}^{9}$~Usquequo, piger, dormies~?\\ quando consurges e somno tuo~?\\
${}^{10}$~Paululum dormies, paululum dormitabis,\\ paululum conseres manus ut dormias~;\\
${}^{11}$~et veniet tibi quasi viator egestas,\\ et pauperies quasi vir armatus.\\ Si vero impiger fueris, veniet ut fons messis tua,\\ et egestas longe fugiet a te.\end{verse}\end{flushleft}


\begin{flushleft}\begin{verse}${}^{12}$~Homo apostata, vir inutilis, graditur ore perverso~;\\
${}^{13}$~annuit oculis, terit pede, digito loquitur,\\
${}^{14}$~pravo corde machinatur malum,\\ et omni tempore jurgia seminat.\\
${}^{15}$~Huic extemplo veniet perditio sua,\\ et subito conteretur, nec habebit ultra medicinam.\end{verse}\end{flushleft}


\begin{flushleft}\begin{verse}${}^{16}$~Sex sunt qu\ae\ odit Dominus,\\ et septimum detestatur anima ejus~:\\
${}^{17}$~oculos sublimes, linguam mendacem,\\ manus effundentes innoxium sanguinem,\\
${}^{18}$~cor machinans cogitationes pessimas,\\ pedes veloces ad currendum in malum,\\
${}^{19}$~proferentem mendacia testem fallacem,\\ et eum qui seminat inter fratres discordias.\end{verse}\end{flushleft}


\begin{flushleft}\begin{verse}${}^{20}$~Conserva, fili mi, pr\ae cepta patris tui,\\ et ne dimittas legem matris tu\ae .\\
${}^{21}$~Liga ea in corde tuo jugiter,\\ et circumda gutturi tuo.\\
${}^{22}$~Cum ambulaveris, gradiantur tecum~;\\ cum dormieris, custodiant te~:\\ et evigilans loquere cum eis.\\
${}^{23}$~Quia mandatum lucerna est, et lex lux,\\ et via vit\ae\ increpatio disciplin\ae~:\\
${}^{24}$~ut custodiant te a muliere mala,\\ et a blanda lingua extrane\ae .\\
${}^{25}$~Non concupiscat pulchritudinem ejus cor tuum,\\ nec capiaris nutibus illius~:\\
${}^{26}$~pretium enim scorti vix est unius panis,\\ mulier autem viri pretiosam animam capit.\\
${}^{27}$~Numquid potest homo abscondere ignem in sinu suo,\\ ut vestimenta illius non ardeant~?\\
${}^{28}$~aut ambulare super prunas,\\ ut non comburantur plant\ae\ ejus~?\\
${}^{29}$~sic qui ingreditur ad mulierem proximi sui,\\ non erit mundus cum tetigerit eam.\\
${}^{30}$~Non grandis est culpa cum quis furatus fuerit~:\\ furatur enim ut esurientem impleat animam~;\\
${}^{31}$~deprehensus quoque reddet septuplum,\\ et omnem substantiam domus su\ae\ tradet.\\
${}^{32}$~Qui autem adulter est,\\ propter cordis inopiam perdet animam suam~;\\
${}^{33}$~turpitudinem et ignominiam congregat sibi,\\ et opprobrium illius non delebitur~:\\
${}^{34}$~quia zelus et furor viri\\ non parcet in die vindict\ae ,\\
${}^{35}$~nec acquiescet cujusquam precibus,\\ nec suscipiet pro redemptione dona plurima.\end{verse}\end{flushleft}


\Needspace{2.5\baselineskip}\versal{7}\begin{flushleft}\begin{verse}\vspace{-19pt}Fili mi, custodi sermones meos,\\ et pr\ae cepta mea reconde tibi.\\ Fili,
${}^{2}$~serva mandata mea, et vives~;\\ et legem meam quasi pupillam oculi tui~:\\
${}^{3}$~liga eam in digitis tuis,\\ scribe illam in tabulis cordis tui.\\
${}^{4}$~Dic sapienti\ae~: Soror mea es,\\ et prudentiam voca amicam tuam~:\\
${}^{5}$~ut custodiant te a muliere extranea,\\ et ab aliena qu\ae\ verba sua dulcia facit.\\
${}^{6}$~De fenestra enim domus me\ae \\ per cancellos prospexi,\\
${}^{7}$~et video parvulos~;\\ considero vecordem juvenem,\\
${}^{8}$~qui transit per plateam juxta angulum\\ et prope viam domus illius graditur~:\\
${}^{9}$~in obscuro, advesperascente die,\\ in noctis tenebris et caligine.\\
${}^{10}$~Et ecce occurrit illi mulier ornatu meretricio,\\ pr\ae parata ad capiendas animas~:\\ garrula et vaga,\\
${}^{11}$~quietis impatiens,\\ nec valens in domo consistere pedibus suis~;\\
${}^{12}$~nunc foris, nunc in plateis,\\ nunc juxta angulos insidians.\\
${}^{13}$~Apprehensumque deosculatur juvenem,\\ et procaci vultu blanditur, dicens~:\\
${}^{14}$~Victimas pro salute vovi~;\\ hodie reddidi vota mea~:\\
${}^{15}$~idcirco egressa sum in occursum tuum,\\ desiderans te videre, et reperi.\\
${}^{16}$~Intexui funibus lectulum meum~;\\ stravi tapetibus pictis ex \AE gypto~:\\
${}^{17}$~aspersi cubile meum myrrha,\\ et alo\"e, et cinnamomo.\\
${}^{18}$~Veni, inebriemur uberibus,\\ et fruamur cupitis amplexibus donec illucescat dies.\\
${}^{19}$~Non est enim vir in domo sua~:\\ abiit via longissima~:\\
${}^{20}$~sacculum pecuni\ae\ secum tulit~;\\ in die plen\ae\ lun\ae\ reversurus est in domum suam.\\
${}^{21}$~Irretivit eum multis sermonibus,\\ et blanditiis labiorum protraxit illum.\\
${}^{22}$~Statim eam sequitur quasi bos ductus ad victimam,\\ et quasi agnus lasciviens,\\ et ignorans quod ad vincula stultus trahatur~:\\
${}^{23}$~donec transfigat sagitta jecur ejus,\\ velut si avis festinet ad laqueum,\\ et nescit quod de periculo anim\ae\ illius agitur.\\
${}^{24}$~Nunc ergo, fili mi, audi me,\\ et attende verbis oris mei.\\
${}^{25}$~Ne abstrahatur in viis illius mens tua,\\ neque decipiaris semitis ejus~;\\
${}^{26}$~multos enim vulneratos dejecit,\\ et fortissimi quique interfecti sunt ab ea.\\
${}^{27}$~Vi\ae\ inferi domus ejus,\\ penetrantes in interiora mortis.\end{verse}\end{flushleft}


\Needspace{2.5\baselineskip}\versal{8}\begin{flushleft}\begin{verse}\vspace{-19pt}Numquid non sapientia clamitat,\\ et prudentia dat vocem suam~?\\
${}^{2}$~In summis excelsisque verticibus supra viam,\\ in mediis semitis stans,\\
${}^{3}$~juxta portas civitatis,\\ in ipsis foribus loquitur, dicens~:\\
${}^{4}$~O viri, ad vos clamito,\\ et vox mea ad filios hominum.\\
${}^{5}$~Intelligite, parvuli, astutiam,\\ et insipientes, animadvertite.\\
${}^{6}$~Audite, quoniam de rebus magnis locutura sum,\\ et aperientur labia mea ut recta pr\ae dicent.\\
${}^{7}$~Veritatem meditabitur guttur meum,\\ et labia mea detestabuntur impium.\\
${}^{8}$~Justi sunt omnes sermones mei~:\\ non est in eis pravum quid, neque perversum~;\\
${}^{9}$~recti sunt intelligentibus,\\ et \ae qui invenientibus scientiam.\\
${}^{10}$~Accipite disciplinam meam, et non pecuniam~;\\ doctrinam magis quam aurum eligite~:\\
${}^{11}$~melior est enim sapientia cunctis pretiosissimis,\\ et omne desiderabile ei non potest comparari.\end{verse}\end{flushleft}


\begin{flushleft}\begin{verse}${}^{12}$~Ego sapientia, habito in consilio,\\ et eruditis intersum cogitationibus.\\
${}^{13}$~Timor Domini odit malum~:\\ arrogantiam, et superbiam,\\ et viam pravam, et os bilingue, detestor.\\
${}^{14}$~Meum est consilium et \ae quitas~;\\ mea est prudentia, mea est fortitudo.\\
${}^{15}$~Per me reges regnant,\\ et legum conditores justa decernunt~;\\
${}^{16}$~per me principes imperant,\\ et potentes decernunt justitiam.\\
${}^{17}$~Ego diligentes me diligo,\\ et qui mane vigilant ad me, invenient me.\\
${}^{18}$~Mecum sunt diviti\ae\ et gloria,\\ opes superb\ae\ et justitia.\\
${}^{19}$~Melior est enim fructus meus auro et lapide pretioso,\\ et genimina mea argento electo.\\
${}^{20}$~In viis justiti\ae\ ambulo,\\ in medio semitarum judicii~:\\
${}^{21}$~ut ditem diligentes me,\\ et thesauros eorum repleam.\end{verse}\end{flushleft}


\begin{flushleft}\begin{verse}${}^{22}$~Dominus possedit me in initio viarum suarum\\ antequam quidquam faceret a principio.\\
${}^{23}$~Ab \ae terno ordinata sum,\\ et ex antiquis antequam terra fieret.\\
${}^{24}$~Nondum erant abyssi, et ego jam concepta eram~:\\ necdum fontes aquarum eruperant,\\
${}^{25}$~necdum montes gravi mole constiterant~:\\ ante colles ego parturiebar.\\
${}^{26}$~Adhuc terram non fecerat, et flumina,\\ et cardines orbis terr\ae .\\
${}^{27}$~Quando pr\ae parabat c\ae los, aderam~;\\ quando certa lege et gyro vallabat abyssos~;\\
${}^{28}$~quando \ae thera firmabat sursum,\\ et librabat fontes aquarum~;\\
${}^{29}$~quando circumdabat mari terminum suum,\\ et legem ponebat aquis, ne transirent fines suos~;\\ quando appendebat fundamenta terr\ae~:\\
${}^{30}$~cum eo eram, cuncta componens.\\ Et delectabar per singulos dies,\\ ludens coram eo omni tempore,\\
${}^{31}$~ludens in orbe terrarum~;\\ et delici\ae\ me\ae\ esse cum filiis hominum.\\
${}^{32}$~Nunc ergo, filii, audite me~:\\ beati qui custodiunt vias meas.\\
${}^{33}$~Audite disciplinam, et estote sapientes,\\ et nolite abjicere eam.\\
${}^{34}$~Beatus homo qui audit me,\\ et qui vigilat ad fores meas quotidie,\\ et observat ad postes ostii mei.\\
${}^{35}$~Qui me invenerit, inveniet vitam,\\ et hauriet salutem a Domino.\\
${}^{36}$~Qui autem in me peccaverit, l\ae det animam suam~;\\ omnes qui me oderunt diligunt mortem.\end{verse}\end{flushleft}


\Needspace{2.5\baselineskip}\versal{9}\begin{flushleft}\begin{verse}\vspace{-19pt}Sapientia \ae dificavit sibi domum~:\\ excidit columnas septem.\\
${}^{2}$~Immolavit victimas suas, miscuit vinum,\\ et proposuit mensam suam.\\
${}^{3}$~Misit ancillas suas ut vocarent\\ ad arcem et ad mœnia civitatis.\\
${}^{4}$~Si quis est parvulus, veniat ad me.\\ Et insipientibus locuta est~:\\
${}^{5}$~Venite, comedite panem meum,\\ et bibite vinum quod miscui vobis.\\
${}^{6}$~Relinquite infantiam, et vivite,\\ et ambulate per vias prudenti\ae .\end{verse}\end{flushleft}


\begin{flushleft}\begin{verse}${}^{7}$~Qui erudit derisorem, ipse injuriam sibi facit,\\ et qui arguit impium, sibi maculam generat.\\
${}^{8}$~Noli arguere derisorem, ne oderit te~:\\ argue sapientem, et diliget te.\\
${}^{9}$~Da sapienti occasionem, et addetur ei sapientia~;\\ doce justum, et festinabit accipere.\\
${}^{10}$~Principium sapienti\ae\ timor Domini,\\ et scientia sanctorum prudentia.\\
${}^{11}$~Per me enim multiplicabuntur dies tui,\\ et addentur tibi anni vit\ae .\\
${}^{12}$~Si sapiens fueris, tibimetipsi eris~;\\ si autem illusor, solus portabis malum.\end{verse}\end{flushleft}


\begin{flushleft}\begin{verse}${}^{13}$~Mulier stulta et clamosa,\\ plenaque illecebris, et nihil omnino sciens,\\
${}^{14}$~sedit in foribus domus su\ae ,\\ super sellam in excelso urbis loco,\\
${}^{15}$~ut vocaret transeuntes per viam,\\ et pergentes itinere suo~:\\
${}^{16}$~Qui est parvulus declinet ad me.\\ Et vecordi locuta est~:\\
${}^{17}$~Aqu\ae\ furtiv\ae\ dulciores sunt,\\ et panis absconditus suavior.\\
${}^{18}$~Et ignoravit quod ibi sint gigantes,\\ et in profundis inferni conviv\ae\ ejus.\end{verse}\end{flushleft}


\Needspace{2.5\baselineskip}\versal{10}\begin{flushleft}\begin{verse}\vspace{-19pt}\hspace{6pt}Filius sapiens l\ae tificat patrem,\\\hspace{6pt} filius vero stultus mœstitia est matris su\ae .\\
${}^{2}$~Nil proderunt thesauri impietatis,\\ justitia vero liberabit a morte.\\
${}^{3}$~Non affliget Dominus fame animam justi,\\ et insidias impiorum subvertet.\\
${}^{4}$~Egestatem operata est manus remissa~;\\ manus autem fortium divitias parat.\\ Qui nititur mendaciis, hic pascit ventos~;\\ idem autem ipse sequitur aves volantes.\\
${}^{5}$~Qui congregat in messe, filius sapiens est~;\\ qui autem stertit \ae state, filius confusionis.\\
${}^{6}$~Benedictio Domini super caput justi~;\\ os autem impiorum operit iniquitas.\\
${}^{7}$~Memoria justi cum laudibus,\\ et nomen impiorum putrescet.\\
${}^{8}$~Sapiens corde pr\ae cepta suscipit~;\\ stultus c\ae ditur labiis.\\
${}^{9}$~Qui ambulat simpliciter ambulat confidenter~;\\ qui autem depravat vias suas manifestus erit.\\
${}^{10}$~Qui annuit oculo dabit dolorem~;\\ et stultus labiis verberabitur.\end{verse}\end{flushleft}


\begin{flushleft}\begin{verse}${}^{11}$~Vena vit\ae\ os justi,\\ et os impiorum operit iniquitatem.\\
${}^{12}$~Odium suscitat rixas,\\ et universa delicta operit caritas.\\
${}^{13}$~In labiis sapientis invenitur sapientia,\\ et virga in dorso ejus qui indiget corde.\\
${}^{14}$~Sapientes abscondunt scientiam~;\\ os autem stulti confusioni proximum est.\\
${}^{15}$~Substantia divitis, urbs fortitudinis ejus~;\\ pavor pauperum egestas eorum.\\
${}^{16}$~Opus justi ad vitam,\\ fructus autem impii ad peccatum.\\
${}^{17}$~Via vit\ae\ custodienti disciplinam~;\\ qui autem increpationes relinquit, errat.\\
${}^{18}$~Abscondunt odium labia mendacia~;\\ qui profert contumeliam, insipiens est.\\
${}^{19}$~In multiloquio non deerit peccatum,\\ qui autem moderatur labia sua prudentissimus est.\\
${}^{20}$~Argentum electum lingua justi~;\\ cor autem impiorum pro nihilo.\\
${}^{21}$~Labia justi erudiunt plurimos~;\\ qui autem indocti sunt in cordis egestate morientur.\end{verse}\end{flushleft}


\begin{flushleft}\begin{verse}${}^{22}$~Benedictio Domini divites facit,\\ nec sociabitur eis afflictio.\\
${}^{23}$~Quasi per risum stultus operatur scelus,\\ sapientia autem est viro prudentia.\\
${}^{24}$~Quod timet impius veniet super eum~;\\ desiderium suum justus dabitur.\\
${}^{25}$~Quasi tempestas transiens non erit impius~;\\ justus autem quasi fundamentum sempiternum.\\
${}^{26}$~Sicut acetum dentibus, et fumus oculis,\\ sic piger his qui miserunt eum.\\
${}^{27}$~Timor Domini apponet dies,\\ et anni impiorum breviabuntur.\\
${}^{28}$~Exspectatio justorum l\ae titia,\\ spes autem impiorum peribit.\\
${}^{29}$~Fortitudo simplicis via Domini,\\ et pavor his qui operantur malum.\\
${}^{30}$~Justus in \ae ternum non commovebitur,\\ impii autem non habitabunt super terram.\\
${}^{31}$~Os justi parturiet sapientiam~;\\ lingua pravorum peribit.\\
${}^{32}$~Labia justi considerant placita,\\ et os impiorum perversa.\end{verse}\end{flushleft}


\Needspace{2.5\baselineskip}\versal{11}\begin{flushleft}\begin{verse}\vspace{-19pt}\hspace{6pt}Statera dolosa abominatio est apud Dominum,\\\hspace{6pt} et pondus \ae quum voluntas ejus.\\
${}^{2}$~Ubi fuerit superbia, ibi erit et contumelia~;\\ ubi autem est humilitas, ibi et sapientia.\\
${}^{3}$~Simplicitas justorum diriget eos,\\ et supplantatio perversorum vastabit illos.\\
${}^{4}$~Non proderunt diviti\ae\ in die ultionis~;\\ justitia autem liberabit a morte.\\
${}^{5}$~Justitia simplicis diriget viam ejus,\\ et in impietate sua corruet impius.\\
${}^{6}$~Justitia rectorum liberabit eos,\\ et in insidiis suis capientur iniqui.\\
${}^{7}$~Mortuo homine impio, nulla erit ultra spes,\\ et exspectatio sollicitorum peribit.\\
${}^{8}$~Justus de angustia liberatus est,\\ et tradetur impius pro eo.\end{verse}\end{flushleft}


\begin{flushleft}\begin{verse}${}^{9}$~Simulator ore decipit amicum suum~;\\ justi autem liberabuntur scientia.\\
${}^{10}$~In bonis justorum exsultabit civitas,\\ et in perditione impiorum erit laudatio.\\
${}^{11}$~Benedictione justorum exaltabitur civitas,\\ et ore impiorum subvertetur.\\
${}^{12}$~Qui despicit amicum suum indigens corde est~;\\ vir autem prudens tacebit.\\
${}^{13}$~Qui ambulat fraudulenter, revelat arcana~;\\ qui autem fidelis est animi, celat amici commissum.\\
${}^{14}$~Ubi non est gubernator, populus corruet~;\\ salus autem, ubi multa consilia.\\
${}^{15}$~Affligetur malo qui fidem facit pro extraneo~;\\ qui autem cavet laqueos securus erit.\\
${}^{16}$~Mulier gratiosa inveniet gloriam,\\ et robusti habebunt divitias.\end{verse}\end{flushleft}


\begin{flushleft}\begin{verse}${}^{17}$~Benefacit anim\ae\ su\ae\ vir misericors~;\\ qui autem crudelis est, etiam propinquos abjicit.\\
${}^{18}$~Impius facit opus instabile,\\ seminanti autem justitiam merces fidelis.\\
${}^{19}$~Clementia pr\ae parat vitam,\\ et sectatio malorum mortem.\\
${}^{20}$~Abominabile Domino cor pravum,\\ et voluntas ejus in iis qui simpliciter ambulant.\\
${}^{21}$~Manus in manu non erit innocens malus~;\\ semen autem justorum salvabitur.\\
${}^{22}$~Circulus aureus in naribus suis,\\ mulier pulchra et fatua.\\
${}^{23}$~Desiderium justorum omne bonum est~;\\ pr\ae stolatio impiorum furor.\\
${}^{24}$~Alii dividunt propria, et ditiores fiunt~;\\ alii rapiunt non sua, et semper in egestate sunt.\\
${}^{25}$~Anima qu\ae\ benedicit impinguabitur,\\ et qui inebriat, ipse quoque inebriabitur.\\
${}^{26}$~Qui abscondit frumenta maledicetur in populis~;\\ benedictio autem super caput vendentium.\\
${}^{27}$~Bene consurgit diluculo qui qu\ae rit bona~;\\ qui autem investigator malorum est, opprimetur ab eis.\\
${}^{28}$~Qui confidit in divitiis suis corruet~:\\ justi autem quasi virens folium germinabunt.\\
${}^{29}$~Qui conturbat domum suam possidebit ventos,\\ et qui stultus est serviet sapienti.\\
${}^{30}$~Fructus justi lignum vit\ae ,\\ et qui suscipit animas sapiens est.\\
${}^{31}$~Si justus in terra recipit,\\ quanto magis impius et peccator~!\end{verse}\end{flushleft}


\Needspace{2.5\baselineskip}\versal{12}\begin{flushleft}\begin{verse}\vspace{-19pt}\hspace{6pt}Qui diligit disciplinam diligit scientiam~;\\\hspace{6pt} qui autem odit increpationes insipiens est.\\
${}^{2}$~Qui bonus est hauriet gratiam a Domino~;\\ qui autem confidit in cogitationibus suis impie agit.\\
${}^{3}$~Non roborabitur homo ex impietate,\\ et radix justorum non commovebitur.\\
${}^{4}$~Mulier diligens corona est viro suo~;\\ et putredo in ossibus ejus, qu\ae\ confusione res dignas gerit.\\
${}^{5}$~Cogitationes justorum judicia,\\ et consilia impiorum fraudulenta.\\
${}^{6}$~Verba impiorum insidiantur sanguini~;\\ os justorum liberabit eos.\\
${}^{7}$~Verte impios, et non erunt~;\\ domus autem justorum permanebit.\\
${}^{8}$~Doctrina sua noscetur vir~;\\ qui autem vanus et excors est patebit contemptui.\\
${}^{9}$~Melior est pauper et sufficiens sibi\\ quam gloriosus et indigens pane.\\
${}^{10}$~Novit justus jumentorum suorum animas~;\\ viscera autem impiorum crudelia.\\
${}^{11}$~Qui operatur terram suam satiabitur panibus~;\\ qui autem sectatur otium stultissimus est.\\ Qui suavis est in vini demorationibus,\\ in suis munitionibus relinquit contumeliam.\\
${}^{12}$~Desiderium impii munimentum est pessimorum~;\\ radix autem justorum proficiet.\end{verse}\end{flushleft}


\begin{flushleft}\begin{verse}${}^{13}$~Propter peccata labiorum ruina proximat malo~;\\ effugiet autem justus de angustia.\\
${}^{14}$~De fructu oris sui unusquisque replebitur bonis,\\ et juxta opera manuum suarum retribuetur ei.\\
${}^{15}$~Via stulti recta in oculis ejus~;\\ qui autem sapiens est audit consilia.\\
${}^{16}$~Fatuus statim indicat iram suam~;\\ qui autem dissimulat injuriam callidus est.\\
${}^{17}$~Qui quod novit loquitur, index justiti\ae\ est~;\\ qui autem mentitur, testis est fraudulentus.\\
${}^{18}$~Est qui promittit, et quasi gladio pungitur conscienti\ae~:\\ lingua autem sapientium sanitas est.\\
${}^{19}$~Labium veritatis firmum erit in perpetuum~;\\ qui autem testis est repentinus, concinnat linguam mendacii.\\
${}^{20}$~Dolus in corde cogitantium mala~;\\ qui autem pacis ineunt consilia, sequitur eos gaudium.\\
${}^{21}$~Non contristabit justum quidquid ei acciderit~:\\ impii autem replebuntur malo.\\
${}^{22}$~Abominatio est Domino labia mendacia~;\\ qui autem fideliter agunt placent ei.\\
${}^{23}$~Homo versatus celat scientiam,\\ et cor insipientium provocat stultitiam.\end{verse}\end{flushleft}


\begin{flushleft}\begin{verse}${}^{24}$~Manus fortium dominabitur~;\\ qu\ae\ autem remissa est, tributis serviet.\\
${}^{25}$~Mœror in corde viri humiliabit illum,\\ et sermone bono l\ae tificabitur.\\
${}^{26}$~Qui negligit damnum propter amicum, justus est~;\\ iter autem impiorum decipiet eos.\\
${}^{27}$~Non inveniet fraudulentus lucrum,\\ et substantia hominis erit auri pretium.\\
${}^{28}$~In semita justiti\ae\ vita~;\\ iter autem devium ducit ad mortem.\end{verse}\end{flushleft}


\Needspace{2.5\baselineskip}\versal{13}\begin{flushleft}\begin{verse}\vspace{-19pt}\hspace{6pt}Filius sapiens doctrina patris~;\\\hspace{6pt} qui autem illusor est non audit cum arguitur.\\
${}^{2}$~De fructu oris sui homo satiabitur bonis~:\\ anima autem pr\ae varicatorum iniqua.\\
${}^{3}$~Qui custodit os suum custodit animam suam~;\\ qui autem inconsideratus est ad loquendum, sentiet mala.\\
${}^{4}$~Vult et non vult piger~;\\ anima autem operantium impinguabitur.\\
${}^{5}$~Verbum mendax justus detestabitur~;\\ impius autem confundit, et confundetur.\\
${}^{6}$~Justitia custodit innocentis viam,\\ impietas autem peccatorem supplantat.\end{verse}\end{flushleft}


\begin{flushleft}\begin{verse}${}^{7}$~Est quasi dives, cum nihil habeat,\\ et est quasi pauper, cum in multis divitiis sit.\\
${}^{8}$~Redemptio anim\ae\ viri diviti\ae\ su\ae~;\\ qui autem pauper est, increpationem non sustinet.\\
${}^{9}$~Lux justorum l\ae tificat~:\\ lucerna autem impiorum extinguetur.\\
${}^{10}$~Inter superbos semper jurgia sunt~;\\ qui autem agunt omnia cum consilio, reguntur sapientia.\\
${}^{11}$~Substantia festinata minuetur~;\\ qu\ae\ autem paulatim colligitur manu, multiplicabitur.\\
${}^{12}$~Spes qu\ae\ differtur affligit animam~;\\ lignum vit\ae\ desiderium veniens.\end{verse}\end{flushleft}


\begin{flushleft}\begin{verse}${}^{13}$~Qui detrahit alicui rei, ipse se in futurum obligat~;\\ qui autem timet pr\ae ceptum, in pace versabitur.\\ Anim\ae\ dolos\ae\ errant in peccatis~:\\ justi autem misericordes sunt, et miserantur.\\
${}^{14}$~Lex sapientis fons vit\ae ,\\ ut declinet a ruina mortis.\\
${}^{15}$~Doctrina bona dabit gratiam~;\\ in itinere contemptorum vorago.\\
${}^{16}$~Astutus omnia agit cum consilio~;\\ qui autem fatuus est aperit stultitiam.\\
${}^{17}$~Nuntius impii cadet in malum~;\\ legatus autem fidelis, sanitas.\\
${}^{18}$~Egestas et ignominia ei qui deserit disciplinam~;\\ qui autem acquiescit arguenti glorificabitur.\\
${}^{19}$~Desiderium si compleatur delectat animam~;\\ detestantur stulti eos qui fugiunt mala.\\
${}^{20}$~Qui cum sapientibus graditur sapiens erit~;\\ amicus stultorum similis efficietur.\end{verse}\end{flushleft}


\begin{flushleft}\begin{verse}${}^{21}$~Peccatores persequitur malum,\\ et justis retribuentur bona.\\
${}^{22}$~Bonus reliquit h\ae redes filios et nepotes,\\ et custoditur justo substantia peccatoris.\\
${}^{23}$~Multi cibi in novalibus patrum,\\ et aliis congregantur absque judicio.\\
${}^{24}$~Qui parcit virg\ae\ odit filium suum~;\\ qui autem diligit illum instanter erudit.\\
${}^{25}$~Justus comedit et replet animam suam~;\\ venter autem impiorum insaturabilis.\end{verse}\end{flushleft}


\Needspace{2.5\baselineskip}\versal{14}\begin{flushleft}\begin{verse}\vspace{-19pt}\hspace{6pt}Sapiens mulier \ae dificat domum suam~;\\\hspace{6pt} insipiens exstructam quoque manibus destruet.\\
${}^{2}$~Ambulans recto itinere, et timens Deum,\\ despicitur ab eo qui infami graditur via.\\
${}^{3}$~In ore stulti virga superbi\ae~;\\ labia autem sapientium custodiunt eos.\\
${}^{4}$~Ubi non sunt boves, pr\ae sepe vacuum est~;\\ ubi autem plurim\ae\ segetes, ibi manifesta est fortitudo bovis.\\
${}^{5}$~Testis fidelis non mentitur~;\\ profert autem mendacium dolosus testis.\\
${}^{6}$~Qu\ae rit derisor sapientiam, et non invenit~;\\ doctrina prudentium facilis.\\
${}^{7}$~Vade contra virum stultum,\\ et nescit labia prudenti\ae .\\
${}^{8}$~Sapientia callidi est intelligere viam suam,\\ et imprudentia stultorum errans.\\
${}^{9}$~Stultus illudet peccatum,\\ et inter justos morabitur gratia.\\
${}^{10}$~Cor quod novit amaritudinem anim\ae\ su\ae ,\\ in gaudio ejus non miscebitur extraneus.\\
${}^{11}$~Domus impiorum delebitur~:\\ tabernacula vero justorum germinabunt.\\
${}^{12}$~Est via qu\ae\ videtur homini justa,\\ novissima autem ejus deducunt ad mortem.\\
${}^{13}$~Risus dolore miscebitur,\\ et extrema gaudii luctus occupat.\\
${}^{14}$~Viis suis replebitur stultus,\\ et super eum erit vir bonus.\end{verse}\end{flushleft}


\begin{flushleft}\begin{verse}${}^{15}$~Innocens credit omni verbo~;\\ astutus considerat gressus suos.\\ Filio doloso nihil erit boni~;\\ servo autem sapienti prosperi erunt actus,\\ et dirigetur via ejus.\\
${}^{16}$~Sapiens timet, et declinat a malo~;\\ stultus transilit, et confidit.\\
${}^{17}$~Impatiens operabitur stultitiam,\\ et vir versutus odiosus est.\\
${}^{18}$~Possidebunt parvuli stultitiam,\\ et exspectabunt astuti scientiam.\\
${}^{19}$~Jacebunt mali ante bonos,\\ et impii ante portas justorum.\\
${}^{20}$~Etiam proximo suo pauper odiosus erit~:\\ amici vero divitum multi.\\
${}^{21}$~Qui despicit proximum suum peccat~;\\ qui autem miseretur pauperis beatus erit.\\ Qui credit in Domino misericordiam diligit.\\
${}^{22}$~Errant qui operantur malum~;\\ misericordia et veritas pr\ae parant bona.\\
${}^{23}$~In omni opere erit abundantia~;\\ ubi autem verba sunt plurima, ibi frequenter egestas.\\
${}^{24}$~Corona sapientium diviti\ae\ eorum~;\\ fatuitas stultorum imprudentia.\\
${}^{25}$~Liberat animas testis fidelis,\\ et profert mendacia versipellis.\end{verse}\end{flushleft}


\begin{flushleft}\begin{verse}${}^{26}$~In timore Domini fiducia fortitudinis,\\ et filiis ejus erit spes.\\
${}^{27}$~Timor Domini fons vit\ae ,\\ ut declinent a ruina mortis.\\
${}^{28}$~In multitudine populi dignitas regis,\\ et in paucitate plebis ignominia principis.\\
${}^{29}$~Qui patiens est multa gubernatur prudentia~;\\ qui autem impatiens est exaltat stultitiam suam.\\
${}^{30}$~Vita carnium sanitas cordis~;\\ putredo ossium invidia.\\
${}^{31}$~Qui calumniatur egentem exprobrat factori ejus~;\\ honorat autem eum qui miseretur pauperis.\\
${}^{32}$~In malitia sua expelletur impius~:\\ sperat autem justus in morte sua.\\
${}^{33}$~In corde prudentis requiescit sapientia,\\ et indoctos quosque erudiet.\\
${}^{34}$~Justitia elevat gentem~;\\ miseros autem facit populos peccatum.\\
${}^{35}$~Acceptus est regi minister intelligens~;\\ iracundiam ejus inutilis sustinebit.\end{verse}\end{flushleft}


\Needspace{2.5\baselineskip}\versal{15}\begin{flushleft}\begin{verse}\vspace{-19pt}\hspace{6pt}Responsio mollis frangit iram~;\\\hspace{6pt} sermo durus suscitat furorem.\\
${}^{2}$~Lingua sapientium ornat scientiam~;\\ os fatuorum ebullit stultitiam.\\
${}^{3}$~In omni loco, oculi Domini\\ contemplantur bonos et malos.\\
${}^{4}$~Lingua placabilis lignum vit\ae~;\\ qu\ae\ autem immoderata est conteret spiritum.\\
${}^{5}$~Stultus irridet disciplinam patris sui~;\\ qui autem custodit increpationes astutior fiet.\\ In abundanti justitia virtus maxima est~:\\ cogitationes autem impiorum eradicabuntur.\\
${}^{6}$~Domus justi plurima fortitudo,\\ et in fructibus impii conturbatio.\\
${}^{7}$~Labia sapientium disseminabunt scientiam~;\\ cor stultorum dissimile erit.\\
${}^{8}$~Victim\ae\ impiorum abominabiles Domino~;\\ vota justorum placabilia.\\
${}^{9}$~Abominatio est Domino via impii~;\\ qui sequitur justitiam diligitur ab eo.\\
${}^{10}$~Doctrina mala deserenti viam vit\ae~;\\ qui increpationes odit, morietur.\\
${}^{11}$~Infernus et perditio coram Domino~;\\ quanto magis corda filiorum hominum~!\\
${}^{12}$~Non amat pestilens eum qui se corripit,\\ nec ad sapientes graditur.\end{verse}\end{flushleft}


\begin{flushleft}\begin{verse}${}^{13}$~Cor gaudens exhilarat faciem~;\\ in mœrore animi dejicitur spiritus.\\
${}^{14}$~Cor sapientis qu\ae rit doctrinam,\\ et os stultorum pascitur imperitia.\\
${}^{15}$~Omnes dies pauperis, mali~;\\ secura mens quasi juge convivium.\\
${}^{16}$~Melius est parum cum timore Domini,\\ quam thesauri magni et insatiabiles.\\
${}^{17}$~Melius est vocari ad olera cum caritate,\\ quam ad vitulum saginatum cum odio.\\
${}^{18}$~Vir iracundus provocat rixas~;\\ qui patiens est mitigat suscitatas.\\
${}^{19}$~Iter pigrorum quasi sepes spinarum~;\\ via justorum absque offendiculo.\\
${}^{20}$~Filius sapiens l\ae tificat patrem,\\ et stultus homo despicit matrem suam.\\
${}^{21}$~Stultitia gaudium stulto,\\ et vir prudens dirigit gressus suos.\\
${}^{22}$~Dissipantur cogitationes ubi non est consilium~;\\ ubi vero sunt plures consiliarii, confirmantur.\\
${}^{23}$~L\ae tatur homo in sententia oris sui,\\ et sermo opportunus est optimus.\\
${}^{24}$~Semita vit\ae\ super eruditum,\\ ut declinet de inferno novissimo.\end{verse}\end{flushleft}


\begin{flushleft}\begin{verse}${}^{25}$~Domum superborum demolietur Dominus,\\ et firmos faciet terminos vidu\ae .\\
${}^{26}$~Abominatio Domini cogitationes mal\ae ,\\ et purus sermo pulcherrimus firmabitur ab eo.\\
${}^{27}$~Conturbat domum suam qui sectatur avaritiam~;\\ qui autem odit munera, vivet.\\ Per misericordiam et fidem purgantur peccata~:\\ per timorem autem Domini declinat omnis a malo.\\
${}^{28}$~Mens justi meditatur obedientiam~;\\ os impiorum redundat malis.\\
${}^{29}$~Longe est Dominus ab impiis,\\ et orationes justorum exaudiet.\\
${}^{30}$~Lux oculorum l\ae tificat animam~;\\ fama bona impinguat ossa.\\
${}^{31}$~Auris qu\ae\ audit increpationes vit\ae \\ in medio sapientium commorabitur.\\
${}^{32}$~Qui abjicit disciplinam despicit animam suam~;\\ qui autem acquiescit increpationibus possessor est cordis.\\
${}^{33}$~Timor Domini disciplina sapienti\ae ,\\ et gloriam pr\ae cedit humilitas.\end{verse}\end{flushleft}


\Needspace{2.5\baselineskip}\versal{16}\begin{flushleft}\begin{verse}\vspace{-19pt}\hspace{6pt}Hominis est animam pr\ae parare,\\\hspace{6pt} et Domini gubernare linguam.\\
${}^{2}$~Omnes vi\ae\ hominis patent oculis ejus~;\\ spirituum ponderator est Dominus.\\
${}^{3}$~Revela Domino opera tua,\\ et dirigentur cogitationes tu\ae .\\
${}^{4}$~Universa propter semetipsum operatus est Dominus~;\\ impium quoque ad diem malum.\\
${}^{5}$~Abominatio Domini est omnis arrogans~;\\ etiamsi manus ad manum fuerit, non est innocens.\\ Initium vi\ae\ bon\ae\ facere justitiam~;\\ accepta est autem apud Deum magis quam immolare hostias.\\
${}^{6}$~Misericordia et veritate redimitur iniquitas,\\ et in timore Domini declinatur a malo.\\
${}^{7}$~Cum placuerint Domino vi\ae\ hominis,\\ inimicos quoque ejus convertet ad pacem.\\
${}^{8}$~Melius est parum cum justitia\\ quam multi fructus cum iniquitate.\\
${}^{9}$~Cor hominis disponit viam suam,\\ sed Domini est dirigere gressus ejus.\end{verse}\end{flushleft}


\begin{flushleft}\begin{verse}${}^{10}$~Divinatio in labiis regis~;\\ in judicio non errabit os ejus.\\
${}^{11}$~Pondus et statera judicia Domini sunt,\\ et opera ejus omnes lapides sacculi.\\
${}^{12}$~Abominabiles regi qui agunt impie,\\ quoniam justitia firmatur solium.\\
${}^{13}$~Voluntas regum labia justa~;\\ qui recta loquitur diligetur.\\
${}^{14}$~Indignatio regis nuntii mortis,\\ et vir sapiens placabit eam.\\
${}^{15}$~In hilaritate vultus regis vita,\\ et clementia ejus quasi imber serotinus.\end{verse}\end{flushleft}


\begin{flushleft}\begin{verse}${}^{16}$~Posside sapientiam, quia auro melior est,\\ et acquire prudentiam, quia pretiosior est argento.\\
${}^{17}$~Semita justorum declinat mala~;\\ custos anim\ae\ su\ae\ servat viam suam.\\
${}^{18}$~Contritionem pr\ae cedit superbia,\\ et ante ruinam exaltatur spiritus.\\
${}^{19}$~Melius est humiliari cum mitibus\\ quam dividere spolia cum superbis.\\
${}^{20}$~Eruditus in verbo reperiet bona,\\ et qui sperat in Domino beatus est.\\
${}^{21}$~Qui sapiens est corde appellabitur prudens,\\ et qui dulcis eloquio majora percipiet.\\
${}^{22}$~Fons vit\ae\ eruditio possidentis~;\\ doctrina stultorum fatuitas.\end{verse}\end{flushleft}


\begin{flushleft}\begin{verse}${}^{23}$~Cor sapientis erudiet os ejus,\\ et labiis ejus addet gratiam.\\
${}^{24}$~Favus mellis composita verba~;\\ dulcedo anim\ae\ sanitas ossium.\\
${}^{25}$~Est via qu\ae\ videtur homini recta,\\ et novissima ejus ducunt ad mortem.\\
${}^{26}$~Anima laborantis laborat sibi,\\ quia compulit eum os suum.\\
${}^{27}$~Vir impius fodit malum,\\ et in labiis ejus ignis ardescit.\\
${}^{28}$~Homo perversus suscitat lites,\\ et verbosus separat principes.\\
${}^{29}$~Vir iniquus lactat amicum suum,\\ et ducit eum per viam non bonam.\\
${}^{30}$~Qui attonitis oculis cogitat prava,\\ mordens labia sua perficit malum.\\
${}^{31}$~Corona dignitatis senectus,\\ qu\ae\ in viis justiti\ae\ reperietur.\\
${}^{32}$~Melior est patiens viro forti,\\ et qui dominatur animo suo expugnatore urbium.\\
${}^{33}$~Sortes mittuntur in sinum,\\ sed a Domino temperantur.\end{verse}\end{flushleft}


\Needspace{2.5\baselineskip}\versal{17}\begin{flushleft}\begin{verse}\vspace{-19pt}\hspace{6pt}Melior est buccella sicca cum gaudio\\\hspace{6pt} quam domus plena victimis cum jurgio.\\
${}^{2}$~Servus sapiens dominabitur filiis stultis,\\ et inter fratres h\ae reditatem dividet.\\
${}^{3}$~Sicut igne probatur argentum et aurum camino,\\ ita corda probat Dominus.\\
${}^{4}$~Malus obedit lingu\ae\ iniqu\ae ,\\ et fallax obtemperat labiis mendacibus.\\
${}^{5}$~Qui despicit pauperem exprobrat factori ejus,\\ et qui ruina l\ae tatur alterius non erit impunitus.\\
${}^{6}$~Corona senum filii filiorum,\\ et gloria filiorum patres eorum.\\
${}^{7}$~Non decent stultum verba composita,\\ nec principem labium mentiens.\\
${}^{8}$~Gemma gratissima exspectatio pr\ae stolantis~;\\ quocumque se vertit, prudenter intelligit.\\
${}^{9}$~Qui celat delictum qu\ae rit amicitias~;\\ qui altero sermone repetit, separat fœderatos.\\
${}^{10}$~Plus proficit correptio apud prudentem,\\ quam centum plag\ae\ apud stultum.\\
${}^{11}$~Semper jurgia qu\ae rit malus~:\\ angelus autem crudelis mittetur contra eum.\\
${}^{12}$~Expedit magis urs\ae\ occurrere raptis fœtibus,\\ quam fatuo confidenti in stultitia sua.\\
${}^{13}$~Qui reddit mala pro bonis,\\ non recedet malum de domo ejus.\\
${}^{14}$~Qui dimittit aquam caput est jurgiorum,\\ et antequam patiatur contumeliam judicium deserit.\end{verse}\end{flushleft}


\begin{flushleft}\begin{verse}${}^{15}$~Qui justificat impium, et qui condemnat justum,\\ abominabilis est uterque apud Deum.\\
${}^{16}$~Quid prodest stulto habere divitias,\\ cum sapientiam emere non possit~?\\ Qui altum facit domum suam qu\ae rit ruinam,\\ et qui evitat discere incidet in mala.\\
${}^{17}$~Omni tempore diligit qui amicus est,\\ et frater in angustiis comprobatur.\\
${}^{18}$~Stultus homo plaudet manibus,\\ cum spoponderit pro amico suo.\\
${}^{19}$~Qui meditatur discordias diligit rixas,\\ et qui exaltat ostium qu\ae rit ruinam.\\
${}^{20}$~Qui perversi cordis est non inveniet bonum,\\ et qui vertit linguam incidet in malum.\\
${}^{21}$~Natus est stultus in ignominiam suam~;\\ sed nec pater in fatuo l\ae tabitur.\\
${}^{22}$~Animus gaudens \ae tatem floridam facit~;\\ spiritus tristis exsiccat ossa.\\
${}^{23}$~Munera de sinu impius accipit,\\ ut pervertat semitas judicii.\\
${}^{24}$~In facie prudentis lucet sapientia~;\\ oculi stultorum in finibus terr\ae .\\
${}^{25}$~Ira patris filius stultus,\\ et dolor matris qu\ae\ genuit eum.\\
${}^{26}$~Non est bonum damnum inferre justo,\\ nec percutere principem qui recta judicat.\\
${}^{27}$~Qui moderatur sermones suos doctus et prudens est,\\ et pretiosi spiritus vir eruditus.\\
${}^{28}$~Stultus quoque, si tacuerit, sapiens reputabitur,\\ et si compresserit labia sua, intelligens.\end{verse}\end{flushleft}


\Needspace{2.5\baselineskip}\versal{18}\begin{flushleft}\begin{verse}\vspace{-19pt}\hspace{6pt}Occasiones qu\ae rit qui vult recedere ab amico~:\\\hspace{6pt} omni tempore erit exprobrabilis.\\
${}^{2}$~Non recipit stultus verba prudenti\ae ,\\ nisi ea dixeris qu\ae\ versantur in corde ejus.\\
${}^{3}$~Impius, cum in profundum venerit peccatorum, contemnit~;\\ sed sequitur eum ignominia et opprobrium.\\
${}^{4}$~Aqua profunda verba ex ore viri,\\ et torrens redundans fons sapienti\ae .\\
${}^{5}$~Accipere personam impii non est bonum,\\ ut declines a veritate judicii.\end{verse}\end{flushleft}


\begin{flushleft}\begin{verse}${}^{6}$~Labia stulti miscent se rixis,\\ et os ejus jurgia provocat.\\
${}^{7}$~Os stulti contritio ejus,\\ et labia ipsius ruina anim\ae\ ejus.\\
${}^{8}$~Verba bilinguis quasi simplicia,\\ et ipsa perveniunt usque ad interiora ventris.\\ Pigrum dejicit timor~;\\ anim\ae\ autem effeminatorum esurient.\\
${}^{9}$~Qui mollis et dissolutus est in opere suo\\ frater est sua opera dissipantis.\\
${}^{10}$~Turris fortissima nomen Domini~;\\ ad ipsum currit justus, et exaltabitur.\\
${}^{11}$~Substantia divitis urbs roboris ejus,\\ et quasi murus validus circumdans eum.\\
${}^{12}$~Antequam conteratur, exaltatur cor hominis,\\ et antequam glorificetur, humiliatur.\\
${}^{13}$~Qui prius respondet quam audiat,\\ stultum se esse demonstrat, et confusione dignum.\\
${}^{14}$~Spiritus viri sustentat imbecillitatem suam~;\\ spiritum vero ad irascendum facilem quis poterit sustinere~?\\
${}^{15}$~Cor prudens possidebit scientiam,\\ et auris sapientium qu\ae rit doctrinam.\end{verse}\end{flushleft}


\begin{flushleft}\begin{verse}${}^{16}$~Donum hominis dilatat viam ejus,\\ et ante principes spatium ei facit.\\
${}^{17}$~Justus prior est accusator sui~:\\ venit amicus ejus, et investigabit eum.\\
${}^{18}$~Contradictiones comprimit sors,\\ et inter potentes quoque dijudicat.\\
${}^{19}$~Frater qui adjuvatur a fratre quasi civitas firma,\\ et judicia quasi vectes urbium.\\
${}^{20}$~De fructu oris viri replebitur venter ejus,\\ et genimina labiorum ipsius saturabunt eum.\\
${}^{21}$~Mors et vita in manu lingu\ae~;\\ qui diligunt eam comedent fructus ejus.\\
${}^{22}$~Qui invenit mulierem bonam invenit bonum,\\ et hauriet jucunditatem a Domino.\\ Qui expellit mulierem bonam expellit bonum~;\\ qui autem tenet adulteram stultus est et impius.\\
${}^{23}$~Cum obsecrationibus loquetur pauper,\\ et dives effabitur rigide.\\
${}^{24}$~Vir amabilis ad societatem\\ magis amicus erit quam frater.\end{verse}\end{flushleft}


\Needspace{2.5\baselineskip}\versal{19}\begin{flushleft}\begin{verse}\vspace{-19pt}\hspace{6pt}Melior est pauper qui ambulat in simplicitate sua\\\hspace{6pt} quam dives torquens labia sua, et insipiens.\\
${}^{2}$~Ubi non est scientia anim\ae , non est bonum,\\ et qui festinus est pedibus offendet.\\
${}^{3}$~Stultitia hominis supplantat gressus ejus,\\ et contra Deum fervet animo suo.\\
${}^{4}$~Diviti\ae\ addunt amicos plurimos~;\\ a paupere autem et hi quos habuit separantur.\\
${}^{5}$~Testis falsus non erit impunitus,\\ et qui mendacia loquitur non effugiet.\\
${}^{6}$~Multi colunt personam potentis,\\ et amici sunt dona tribuentis.\\
${}^{7}$~Fratres hominis pauperis oderunt eum~;\\ insuper et amici procul recesserunt ab eo.\end{verse}\end{flushleft}

 \begin{flushleft}\begin{verse}Qui tantum verba sectatur nihil habebit~;\\
${}^{8}$~qui autem possessor est mentis diligit animam suam,\\ et custos prudenti\ae\ inveniet bona.\\
${}^{9}$~Falsus testis non erit impunitus,\\ et qui loquitur mendacia peribit.\\
${}^{10}$~Non decent stultum delici\ae ,\\ nec servum dominari principibus.\\
${}^{11}$~Doctrina viri per patientiam noscitur,\\ et gloria ejus est iniqua pr\ae tergredi.\\
${}^{12}$~Sicut fremitus leonis, ita et regis ira,\\ et sicut ros super herbam, ita et hilaritas ejus.\\
${}^{13}$~Dolor patris filius stultus,\\ et tecta jugiter perstillantia litigiosa mulier.\\
${}^{14}$~Domus et diviti\ae\ dantur a parentibus~;\\ a Domino autem proprie uxor prudens.\\
${}^{15}$~Pigredo immittit soporem,\\ et anima dissoluta esuriet.\\
${}^{16}$~Qui custodit mandatum custodit animam suam~;\\ qui autem negligit viam suam mortificabitur.\\
${}^{17}$~Fœneratur Domino qui miseretur pauperis,\\ et vicissitudinem suam reddet ei.\\
${}^{18}$~Erudi filium tuum~; ne desperes~:\\ ad interfectionem autem ejus ne ponas animam tuam.\\
${}^{19}$~Qui impatiens est sustinebit damnum,\\ et cum rapuerit, aliud apponet.\\
${}^{20}$~Audi consilium, et suscipe disciplinam,\\ ut sis sapiens in novissimis tuis.\\
${}^{21}$~Mult\ae\ cogitationes in corde viri~;\\ voluntas autem Domini permanebit.\\
${}^{22}$~Homo indigens misericors est,\\ et melior est pauper quam vir mendax.\\
${}^{23}$~Timor Domini ad vitam,\\ et in plenitudine commorabitur absque visitatione pessima.\end{verse}\end{flushleft}


\begin{flushleft}\begin{verse}${}^{24}$~Abscondit piger manum suam sub ascella,\\ nec ad os suum applicat eam.\\
${}^{25}$~Pestilente flagellato stultus sapientior erit~;\\ si autem corripueris sapientem, intelliget disciplinam.\\
${}^{26}$~Qui affligit patrem, et fugat matrem,\\ ignominiosus est et infelix.\\
${}^{27}$~Non cesses, fili, audire doctrinam,\\ nec ignores sermones scienti\ae .\\
${}^{28}$~Testis iniquus deridet judicium,\\ et os impiorum devorat iniquitatem.\\
${}^{29}$~Parata sunt derisoribus judicia,\\ et mallei percutientes stultorum corporibus.\end{verse}\end{flushleft}


\Needspace{2.5\baselineskip}\versal{20}\begin{flushleft}\begin{verse}\vspace{-19pt}\hspace{6pt}Luxuriosa res vinum, et tumultuosa ebrietas~:\\\hspace{6pt} quicumque his delectatur non erit sapiens.\\
${}^{2}$~Sicut rugitus leonis, ita et terror regis~:\\ qui provocat eum peccat in animam suam.\\
${}^{3}$~Honor est homini qui separat se a contentionibus~;\\ omnes autem stulti miscentur contumeliis.\\
${}^{4}$~Propter frigus piger arare noluit~;\\ mendicabit ergo \ae state, et non dabitur illi.\\
${}^{5}$~Sicut aqua profunda, sic consilium in corde viri~;\\ sed homo sapiens exhauriet illud.\\
${}^{6}$~Multi homines misericordes vocantur~;\\ virum autem fidelem quis inveniet~?\end{verse}\end{flushleft}


\begin{flushleft}\begin{verse}${}^{7}$~Justus qui ambulat in simplicitate sua\\ beatos post se filios derelinquet.\\
${}^{8}$~Rex qui sedet in solio judicii\\ dissipat omne malum intuitu suo.\\
${}^{9}$~Quis potest dicere~: Mundum est cor meum~;\\ purus sum a peccato~?\\
${}^{10}$~Pondus et pondus, mensura et mensura~:\\ utrumque abominabile est apud Deum.\\
${}^{11}$~Ex studiis suis intelligitur puer,\\ si munda et recta sint opera ejus.\\
${}^{12}$~Aurem audientem, et oculum videntem~:\\ Dominus fecit utrumque.\\
${}^{13}$~Noli diligere somnum, ne te egestas opprimat~:\\ aperi oculos tuos, et saturare panibus.\\
${}^{14}$~Malum est, malum est, dicit omnis emptor~;\\ et cum recesserit, tunc gloriabitur.\\
${}^{15}$~Est aurum et multitudo gemmarum,\\ et vas pretiosum labia scienti\ae .\end{verse}\end{flushleft}


\begin{flushleft}\begin{verse}${}^{16}$~Tolle vestimentum ejus qui fidejussor extitit alieni,\\ et pro extraneis aufer pignus ab eo.\\
${}^{17}$~Suavis est homini panis mendacii,\\ et postea implebitur os ejus calculo.\\
${}^{18}$~Cogitationes consiliis roborantur,\\ et gubernaculis tractanda sunt bella.\\
${}^{19}$~Ei qui revelat mysteria, et ambulat fraudulenter,\\ et dilatat labia sua, ne commiscearis.\\
${}^{20}$~Qui maledicit patri suo et matri,\\ extinguetur lucerna ejus in mediis tenebris~:\\
${}^{21}$~h\ae reditas ad quam festinatur in principio,\\ in novissimo benedictione carebit.\\
${}^{22}$~Ne dicas~: Reddam malum~:\\ exspecta Dominum, et liberabit te.\\
${}^{23}$~Abominatio est apud Dominum pondus et pondus~;\\ statera dolosa non est bona.\\
${}^{24}$~A Domino diriguntur gressus viri~:\\ quis autem hominum intelligere potest viam suam~?\\
${}^{25}$~Ruina est homini devorare sanctos,\\ et post vota retractare.\end{verse}\end{flushleft}


\begin{flushleft}\begin{verse}${}^{26}$~Dissipat impios rex sapiens,\\ et incurvat super eos fornicem.\\
${}^{27}$~Lucerna Domini spiraculum hominis,\\ qu\ae\ investigat omnia secreta ventris.\\
${}^{28}$~Misericordia et veritas custodiunt regem,\\ et roboratur clementia thronus ejus.\\
${}^{29}$~Exsultatio juvenum fortitudo eorum,\\ et dignitas senum canities.\\
${}^{30}$~Livor vulneris absterget mala,\\ et plag\ae\ in secretioribus ventris.\end{verse}\end{flushleft}


\Needspace{2.5\baselineskip}\versal{21}\begin{flushleft}\begin{verse}\vspace{-19pt}\hspace{6pt}Sicut divisiones aquarum, ita cor regis in manu Domini~:\\\hspace{6pt} quocumque voluerit, inclinabit illud.\\
${}^{2}$~Omnis via viri recta sibi videtur~:\\ appendit autem corda Dominus.\\
${}^{3}$~Facere misericordiam et judicium\\ magis placet Domino quam victim\ae .\\
${}^{4}$~Exaltatio oculorum est dilatatio cordis~;\\ lucerna impiorum peccatum.\\
${}^{5}$~Cogitationes robusti semper in abundantia~;\\ omnis autem piger semper in egestate est.\end{verse}\end{flushleft}


\begin{flushleft}\begin{verse}${}^{6}$~Qui congregat thesauros lingua mendacii vanus et excors est,\\ et impingetur ad laqueos mortis.\\
${}^{7}$~Rapin\ae\ impiorum detrahent eos,\\ quia noluerunt facere judicium.\\
${}^{8}$~Perversa via viri aliena est~;\\ qui autem mundus est, rectum opus ejus.\\
${}^{9}$~Melius est sedere in angulo domatis,\\ quam cum muliere litigiosa, et in domo communi.\\
${}^{10}$~Anima impii desiderat malum~:\\ non miserebitur proximo suo.\\
${}^{11}$~Mulctato pestilente, sapientior erit parvulus,\\ et si sectetur sapientem, sumet scientiam.\\
${}^{12}$~Excogitat justus de domo impii,\\ ut detrahat impios a malo.\end{verse}\end{flushleft}


\begin{flushleft}\begin{verse}${}^{13}$~Qui obturat aurem suam ad clamorem pauperis,\\ et ipse clamabit, et non exaudietur.\\
${}^{14}$~Munus absconditum extinguit iras,\\ et donum in sinu indignationem maximam.\\
${}^{15}$~Gaudium justo est facere judicium,\\ et pavor operantibus iniquitatem.\\
${}^{16}$~Vir qui erraverit a via doctrin\ae \\ in cœtu gigantum commorabitur.\\
${}^{17}$~Qui diligit epulas in egestate erit~;\\ qui amat vinum et pinguia non ditabitur.\\
${}^{18}$~Pro justo datur impius,\\ et pro rectis iniquus.\\
${}^{19}$~Melius est habitare in terra deserta\\ quam cum muliere rixosa et iracunda.\\
${}^{20}$~Thesaurus desiderabilis, et oleum in habitaculo justi~:\\ et imprudens homo dissipabit illud.\\
${}^{21}$~Qui sequitur justitiam et misericordiam\\ inveniet vitam, justitiam, et gloriam.\\
${}^{22}$~Civitatem fortium ascendit sapiens,\\ et destruxit robur fiduci\ae\ ejus.\\
${}^{23}$~Qui custodit os suum et linguam suam\\ custodit ab angustiis animam suam.\\
${}^{24}$~Superbus et arrogans vocatur indoctus,\\ qui in ira operatur superbiam.\\
${}^{25}$~Desideria occidunt pigrum~:\\ noluerunt enim quidquam manus ejus operari.\\
${}^{26}$~Tota die concupiscit et desiderat~;\\ qui autem justus est, tribuet, et non cessabit.\\
${}^{27}$~Hosti\ae\ impiorum abominabiles,\\ quia offeruntur ex scelere.\\
${}^{28}$~Testis mendax peribit~;\\ vir obediens loquetur victoriam.\\
${}^{29}$~Vir impius procaciter obfirmat vultum suum~;\\ qui autem rectus est corrigit viam suam.\\
${}^{30}$~Non est sapientia, non est prudentia,\\ non est consilium contra Dominum.\\
${}^{31}$~Equus paratur ad diem belli~;\\ Dominus autem salutem tribuit.\end{verse}\end{flushleft}


\Needspace{2.5\baselineskip}\versal{22}\begin{flushleft}\begin{verse}\vspace{-19pt}\hspace{6pt}Melius est nomen bonum quam diviti\ae\ mult\ae~;\\\hspace{6pt} super argentum et aurum gratia bona.\\
${}^{2}$~Dives et pauper obviaverunt sibi~:\\ utriusque operator est Dominus.\\
${}^{3}$~Callidus vidit malum, et abscondit se~;\\ innocens pertransiit, et afflictus est damno.\\
${}^{4}$~Finis modesti\ae\ timor Domini,\\ diviti\ae , et gloria, et vita.\\
${}^{5}$~Arma et gladii in via perversi~;\\ custos autem anim\ae\ su\ae\ longe recedit ab eis.\\
${}^{6}$~Proverbium est~: adolescens juxta viam suam~;\\ etiam cum senuerit, non recedet ab ea.\\
${}^{7}$~Dives pauperibus imperat,\\ et qui accipit mutuum servus est fœnerantis.\\
${}^{8}$~Qui seminat iniquitatem metet mala,\\ et virga ir\ae\ su\ae\ consummabitur.\\
${}^{9}$~Qui pronus est ad misericordiam benedicetur~:\\ de panibus enim suis dedit pauperi.\\ Victoriam et honorem acquiret qui dat munera~;\\ animam autem aufert accipientium.\\
${}^{10}$~Ejice derisorem, et exibit cum eo jurgium,\\ cessabuntque caus\ae\ et contumeli\ae .\\
${}^{11}$~Qui diligit cordis munditiam,\\ propter gratiam labiorum suorum habebit amicum regem.\\
${}^{12}$~Oculi Domini custodiunt scientiam,\\ et supplantantur verba iniqui.\\
${}^{13}$~Dicit piger~: Leo est foris~;\\ in medio platearum occidendus sum.\\
${}^{14}$~Fovea profunda os alien\ae~:\\ cui iratus est Dominus, incidet in eam.\\
${}^{15}$~Stultitia colligata est in corde pueri,\\ et virga disciplin\ae\ fugabit eam.\\
${}^{16}$~Qui calumniatur pauperem ut augeat divitias suas,\\ dabit ipse ditiori, et egebit.\end{verse}\end{flushleft}


\begin{flushleft}\begin{verse}${}^{17}$~Inclina aurem tuam, et audi verba sapientium~:\\ appone autem cor ad doctrinam meam,\\
${}^{18}$~qu\ae\ pulchra erit tibi cum servaveris eam in ventre tuo,\\ et redundabit in labiis tuis~:\\
${}^{19}$~ut sit in Domino fiducia tua,\\ unde et ostendi eam tibi hodie.\\
${}^{20}$~Ecce descripsi eam tibi tripliciter,\\ in cogitationibus et scientia~:\\
${}^{21}$~ut ostenderem tibi firmitatem et eloquia veritatis,\\ respondere ex his illis qui miserunt te.\\
${}^{22}$~Non facias violentiam pauperi quia pauper est,\\ neque conteras egenum in porta~:\\
${}^{23}$~quia judicabit Dominus causam ejus,\\ et configet eos qui confixerunt animam ejus.\\
${}^{24}$~Noli esse amicus homini iracundo,\\ neque ambules cum viro furioso~:\\
${}^{25}$~ne forte discas semitas ejus,\\ et sumas scandalum anim\ae\ tu\ae .\\
${}^{26}$~Noli esse cum his qui defigunt manus suas,\\ et qui vades se offerunt pro debitis~:\\
${}^{27}$~si enim non habes unde restituas,\\ quid caus\ae\ est ut tollat operimentum de cubili tuo~?\\
${}^{28}$~Ne transgrediaris terminos antiquos,\\ quos posuerunt patres tui.\\
${}^{29}$~Vidisti virum velocem in opere suo~?\\ coram regibus stabit, nec erit ante ignobiles.\end{verse}\end{flushleft}


\Needspace{2.5\baselineskip}\versal{23}\begin{flushleft}\begin{verse}\vspace{-19pt}\hspace{6pt}Quando sederis ut comedas cum principe,\\\hspace{6pt} diligenter attende qu\ae\ apposita sunt ante faciem tuam.\\
${}^{2}$~Et statue cultrum in gutture tuo~:\\ si tamen habes in potestate animam tuam.\\
${}^{3}$~Ne desideres de cibis ejus,\\ in quo est panis mendacii.\\
${}^{4}$~Noli laborare ut diteris,\\ sed prudenti\ae\ tu\ae\ pone modum.\\
${}^{5}$~Ne erigas oculos tuos ad opes quas non potes habere,\\ quia facient sibi pennas quasi aquil\ae , et volabunt in c\ae lum.\\
${}^{6}$~Ne comedas cum homine invido,\\ et ne desideres cibos ejus~:\\
${}^{7}$~quoniam in similitudinem arioli et conjectoris\\ \ae stimat quod ignorat.\\ Comede et bibe, dicet tibi~;\\ et mens ejus non est tecum.\\
${}^{8}$~Cibos quos comederas evomes,\\ et perdes pulchros sermones tuos.\\
${}^{9}$~In auribus insipientium ne loquaris,\\ qui despicient doctrinam eloquii tui.\\
${}^{10}$~Ne attingas parvulorum terminos,\\ et agrum pupillorum ne intro\"eas~:\\
${}^{11}$~propinquus enim illorum fortis est,\\ et ipse judicabit contra te causam illorum.\end{verse}\end{flushleft}


\begin{flushleft}\begin{verse}${}^{12}$~Ingrediatur ad doctrinam cor tuum,\\ et aures tu\ae\ ad verba scienti\ae .\\
${}^{13}$~Noli subtrahere a puero disciplinam~:\\ si enim percusseris eum virga, non morietur.\\
${}^{14}$~Tu virga percuties eum,\\ et animam ejus de inferno liberabis.\\
${}^{15}$~Fili mi, si sapiens fuerit animus tuus,\\ gaudebit tecum cor meum~:\\
${}^{16}$~et exsultabunt renes mei,\\ cum locuta fuerint rectum labia tua.\\
${}^{17}$~Non \ae muletur cor tuum peccatores,\\ sed in timore Domini esto tota die~:\\
${}^{18}$~quia habebis spem in novissimo,\\ et pr\ae stolatio tua non auferetur.\\
${}^{19}$~Audi, fili mi, et esto sapiens,\\ et dirige in via animum tuum.\\
${}^{20}$~Noli esse in conviviis potatorum,\\ nec in comessationibus eorum qui carnes ad vescendum conferunt~:\\
${}^{21}$~quia vacantes potibus et dantes symbola consumentur,\\ et vestietur pannis dormitatio.\\
${}^{22}$~Audi patrem tuum, qui genuit te,\\ et ne contemnas cum senuerit mater tua.\\
${}^{23}$~Veritatem eme, et noli vendere sapientiam,\\ et doctrinam, et intelligentiam.\\
${}^{24}$~Exsultat gaudio pater justi~;\\ qui sapientem genuit, l\ae tabitur in eo.\\
${}^{25}$~Gaudeat pater tuus et mater tua,\\ et exsultet qu\ae\ genuit te.\\
${}^{26}$~Pr\ae be, fili mi, cor tuum mihi,\\ et oculi tui vias meas custodiant.\\
${}^{27}$~Fovea enim profunda est meretrix,\\ et puteus angustus aliena.\\
${}^{28}$~Insidiatur in via quasi latro,\\ et quos incautos viderit, interficiet.\end{verse}\end{flushleft}


\begin{flushleft}\begin{verse}${}^{29}$~Cui v\ae~? cujus patri v\ae~?\\ cui rix\ae~? cui fove\ae~?\\ cui sine causa vulnera~? cui suffusio oculorum~?\\
${}^{30}$~nonne his qui commorantur in vino,\\ et student calicibus epotandis~?\\
${}^{31}$~Ne intuearis vinum quando flavescit,\\ cum splenduerit in vitro color ejus~:\\ ingreditur blande,\\
${}^{32}$~sed in novissimo mordebit ut coluber,\\ et sicut regulus venena diffundet.\\
${}^{33}$~Oculi tui videbunt extraneas,\\ et cor tuum loquetur perversa.\\
${}^{34}$~Et eris sicut dormiens in medio mari,\\ et quasi sopitus gubernator, amisso clavo.\\
${}^{35}$~Et dices~: Verberaverunt me, sed non dolui~;\\ traxerunt me, et ego non sensi.\\ Quando evigilabo, et rursus vina reperiam~?\end{verse}\end{flushleft}


\Needspace{2.5\baselineskip}\versal{24}\begin{flushleft}\begin{verse}\vspace{-19pt}\hspace{6pt}Ne \ae muleris viros malos,\\\hspace{6pt} nec desideres esse cum eis~:\\
${}^{2}$~quia rapinas meditatur mens eorum,\\ et fraudes labia eorum loquuntur.\\
${}^{3}$~Sapientia \ae dificabitur domus,\\ et prudentia roborabitur.\\
${}^{4}$~In doctrina replebuntur cellaria,\\ universa substantia pretiosa et pulcherrima.\\
${}^{5}$~Vir sapiens fortis est,\\ et vir doctus robustus et validus~:\\
${}^{6}$~quia cum dispositione initur bellum,\\ et erit salus ubi multa consilia sunt.\\
${}^{7}$~Excelsa stulto sapientia~;\\ in porta non aperiet os suum.\\
${}^{8}$~Qui cogitat mala facere stultus vocabitur~:\\
${}^{9}$~cogitatio stulti peccatum est,\\ et abominatio hominum detractor.\\
${}^{10}$~Si desperaveris lassus in die angusti\ae ,\\ imminuetur fortitudo tua.\end{verse}\end{flushleft}


\begin{flushleft}\begin{verse}${}^{11}$~Erue eos qui ducuntur ad mortem,\\ et qui trahuntur ad interitum, liberare ne cesses.\\
${}^{12}$~Si dixeris~: Vires non suppetunt~;\\ qui inspector est cordis ipse intelligit~:\\ et servatorem anim\ae\ tu\ae\ nihil fallit,\\ reddetque homini juxta opera sua.\\
${}^{13}$~Comede, fili mi, mel, quia bonum est,\\ et favum dulcissimum gutturi tuo.\\
${}^{14}$~Sic et doctrina sapienti\ae\ anim\ae\ tu\ae~:\\ quam cum inveneris, habebis in novissimis spem,\\ et spes tua non peribit.\\
${}^{15}$~Ne insidieris, et qu\ae ras impietatem in domo justi,\\ neque vastes requiem ejus.\\
${}^{16}$~Septies enim cadet justus, et resurget~:\\ impii autem corruent in malum.\\
${}^{17}$~Cum ceciderit inimicus tuus ne gaudeas,\\ et in ruina ejus ne exsultet cor tuum~:\\
${}^{18}$~ne forte videat Dominus, et displiceat ei,\\ et auferat ab eo iram suam.\\
${}^{19}$~Ne contendas cum pessimis,\\ nec \ae muleris impios~:\\
${}^{20}$~quoniam non habent futurorum spem mali,\\ et lucerna impiorum extinguetur.\\
${}^{21}$~Time Dominum, fili mi, et regem,\\ et cum detractoribus non commiscearis~:\\
${}^{22}$~quoniam repente consurget perditio eorum,\\ et ruinam utriusque quis novit~?\end{verse}\end{flushleft}


${}^{23}$~H\ae c quoque sapientibus. \begin{flushleft}\begin{verse}Cognoscere personam in judicio non est bonum.\\
${}^{24}$~Qui dicunt impio~: Justus es~: maledicent eis populi,\\ et detestabuntur eos tribus.\\
${}^{25}$~Qui arguunt eum laudabuntur,\\ et super ipsos veniet benedictio.\\
${}^{26}$~Labia deosculabitur\\ qui recta verba respondet.\\
${}^{27}$~Pr\ae para foris opus tuum,\\ et diligenter exerce agrum tuum,\\ ut postea \ae difices domum tuam.\\
${}^{28}$~Ne sis testis frustra contra proximum tuum,\\ nec lactes quemquam labiis tuis.\\
${}^{29}$~Ne dicas~: Quomodo fecit mihi, sic faciam ei~;\\ reddam unicuique secundum opus suum.\end{verse}\end{flushleft}


\begin{flushleft}\begin{verse}${}^{30}$~Per agrum hominis pigri transivi,\\ et per vineam viri stulti~:\\
${}^{31}$~et ecce totum repleverant urtic\ae ,\\ et operuerant superficiem ejus spin\ae ,\\ et maceria lapidum destructa erat.\\
${}^{32}$~Quod cum vidissem, posui in corde meo,\\ et exemplo didici disciplinam.\\
${}^{33}$~Parum, inquam, dormies, modicum dormitabis~;\\ pauxillum manus conseres ut quiescas~:\\
${}^{34}$~et veniet tibi quasi cursor egestas,\\ et mendicitas quasi vir armatus.\end{verse}\end{flushleft}


\Needspace{2.5\baselineskip}\versal{25}~H\ae\ quoque parabol\ae\ Salomonis, quas transtulerunt viri Ezechi\ae\ regis Juda.
\begin{flushleft}\begin{verse}\vspace{6pt}${}^{2}$~Gloria Dei est celare verbum,\\ et gloria regum investigare sermonem.\\
${}^{3}$~C\ae lum sursum, et terra deorsum,\\ et cor regum inscrutabile.\\
${}^{4}$~Aufer rubiginem de argento,\\ et egredietur vas purissimum.\\
${}^{5}$~Aufer impietatem de vultu regis,\\ et firmabitur justitia thronus ejus.\\
${}^{6}$~Ne gloriosus appareas coram rege,\\ et in loco magnorum ne steteris.\\
${}^{7}$~Melius est enim ut dicatur tibi~: Ascende huc,\\ quam ut humilieris coram principe.\end{verse}\end{flushleft}


\begin{flushleft}\begin{verse}${}^{8}$~Qu\ae\ viderunt oculi tui ne proferas in jurgio cito,\\ ne postea emendare non possis,\\ cum dehonestaveris amicum tuum.\\
${}^{9}$~Causam tuam tracta cum amico tuo,\\ et secretum extraneo ne reveles~:\\
${}^{10}$~ne forte insultet tibi cum audierit,\\ et exprobrare non cesset.\\ Gratia et amicitia liberant~:\\ quas tibi serva, ne exprobrabilis fias.\\
${}^{11}$~Mala aurea in lectis argenteis,\\ qui loquitur verbum in tempore suo.\\
${}^{12}$~Inauris aurea, et margaritum fulgens,\\ qui arguit sapientem et aurem obedientem.\\
${}^{13}$~Sicut frigus nivis in die messis,\\ ita legatus fidelis ei qui misit eum~:\\ animam ipsius requiescere facit.\\
${}^{14}$~Nubes, et ventus, et pluvi\ae\ non sequentes,\\ vir gloriosus et promissa non complens.\\
${}^{15}$~Patientia lenietur princeps,\\ et lingua mollis confringet duritiam.\end{verse}\end{flushleft}


\begin{flushleft}\begin{verse}${}^{16}$~Mel invenisti~: comede quod sufficit tibi,\\ ne forte satiatus evomas illud.\\
${}^{17}$~Subtrahe pedem tuum de domo proximi tui,\\ nequando satiatus oderit te.\\
${}^{18}$~Jaculum, et gladius, et sagitta acuta,\\ homo qui loquitur contra proximum suum falsum testimonium.\\
${}^{19}$~Dens putridus, et pes lassus,\\ qui sperat super infideli in die angusti\ae ,\\
${}^{20}$~et amittit pallium in die frigoris.\\ Acetum in nitro,\\ qui cantat carmina cordi pessimo.\\ Sicut tinea vestimento, et vermis ligno,\\ ita tristitia viri nocet cordi.\\
${}^{21}$~Si esurierit inimicus tuus, ciba illum~;\\ si sitierit, da ei aquam bibere~:\\
${}^{22}$~prunas enim congregabis super caput ejus,\\ et Dominus reddet tibi.\\
${}^{23}$~Ventus aquilo dissipat pluvias,\\ et facies tristis linguam detrahentem.\\
${}^{24}$~Melius est sedere in angulo domatis\\ quam cum muliere litigiosa et in domo communi.\\
${}^{25}$~Aqua frigida anim\ae\ sitienti,\\ et nuntius bonus de terra longinqua.\\
${}^{26}$~Fons turbatus pede et vena corrupta,\\ justus cadens coram impio.\\
${}^{27}$~Sicut qui mel multum comedit non est ei bonum,\\ sic qui scrutator est majestatis opprimetur a gloria.\\
${}^{28}$~Sicut urbs patens et absque murorum ambitu,\\ ita vir qui non potest in loquendo cohibere spiritum suum.\end{verse}\end{flushleft}


\Needspace{2.5\baselineskip}\versal{26}\begin{flushleft}\begin{verse}\vspace{-19pt}\hspace{6pt}Quomodo nix in \ae state, et pluvi\ae\ in messe,\\\hspace{6pt} sic indecens est stulto gloria.\\
${}^{2}$~Sicut avis ad alia transvolans, et passer quolibet vadens,\\ sic maledictum frustra prolatum in quempiam superveniet.\\
${}^{3}$~Flagellum equo, et camus asino,\\ et virga in dorso imprudentium.\\
${}^{4}$~Ne respondeas stulto juxta stultitiam suam,\\ ne efficiaris ei similis.\\
${}^{5}$~Responde stulto juxta stultitiam suam,\\ ne sibi sapiens esse videatur.\\
${}^{6}$~Claudus pedibus, et iniquitatem bibens,\\ qui mittit verba per nuntium stultum.\\
${}^{7}$~Quomodo pulchras frustra habet claudus tibias,\\ sic indecens est in ore stultorum parabola.\\
${}^{8}$~Sicut qui mittit lapidem in acervum Mercurii,\\ ita qui tribuit insipienti honorem.\\
${}^{9}$~Quomodo si spina nascatur in manu temulenti,\\ sic parabola in ore stultorum.\\
${}^{10}$~Judicium determinat causas,\\ et qui imponit stulto silentium iras mitigat.\\
${}^{11}$~Sicut canis qui revertitur ad vomitum suum,\\ sic imprudens qui iterat stultitiam suam.\\
${}^{12}$~Vidisti hominem sapientem sibi videri~?\\ magis illo spem habebit insipiens.\end{verse}\end{flushleft}


\begin{flushleft}\begin{verse}${}^{13}$~Dicit piger~: Leo est in via,\\ et le\ae na in itineribus.\\
${}^{14}$~Sicut ostium vertitur in cardine suo,\\ ita piger in lectulo suo.\\
${}^{15}$~Abscondit piger manum sub ascella sua,\\ et laborat si ad os suum eam converterit.\\
${}^{16}$~Sapientior sibi piger videtur\\ septem viris loquentibus sententias.\end{verse}\end{flushleft}


\begin{flushleft}\begin{verse}${}^{17}$~Sicut qui apprehendit auribus canem,\\ sic qui transit impatiens et commiscetur rix\ae\ alterius.\\
${}^{18}$~Sicut noxius est qui mittit sagittas et lanceas in mortem,\\
${}^{19}$~ita vir fraudulenter nocet amico suo,\\ et cum fuerit deprehensus dicit~: Ludens feci.\\
${}^{20}$~Cum defecerint ligna extinguetur ignis,\\ et susurrone subtracto, jurgia conquiescent.\\
${}^{21}$~Sicut carbones ad prunas, et ligna ad ignem,\\ sic homo iracundus suscitat rixas.\\
${}^{22}$~Verba susurronis quasi simplicia,\\ et ipsa perveniunt ad intima ventris.\\
${}^{23}$~Quomodo si argento sordido ornare velis vas fictile,\\ sic labia tumentia cum pessimo corde sociata.\\
${}^{24}$~Labiis suis intelligitur inimicus,\\ cum in corde tractaverit dolos.\\
${}^{25}$~Quando submiserit vocem suam, ne credideris ei,\\ quoniam septem nequiti\ae\ sunt in corde illius.\\
${}^{26}$~Qui operit odium fraudulenter,\\ revelabitur malitia ejus in consilio.\\
${}^{27}$~Qui fodit foveam incidet in eam,\\ et qui volvit lapidem revertetur ad eum.\\
${}^{28}$~Lingua fallax non amat veritatem,\\ et os lubricum operatur ruinas.\end{verse}\end{flushleft}


\Needspace{2.5\baselineskip}\versal{27}\begin{flushleft}\begin{verse}\vspace{-19pt}\hspace{6pt}Ne glorieris in crastinum,\\\hspace{6pt} ignorans quid superventura pariat dies.\\
${}^{2}$~Laudet te alienus, et non os tuum~;\\ extraneus, et non labia tua.\\
${}^{3}$~Grave est saxum, et onerosa arena,\\ sed ira stulti utroque gravior.\\
${}^{4}$~Ira non habet misericordiam nec erumpens furor,\\ et impetum concitati ferre quis poterit~?\\
${}^{5}$~Melior est manifesta correptio\\ quam amor absconditus.\\
${}^{6}$~Meliora sunt vulnera diligentis\\ quam fraudulenta oscula odientis.\\
${}^{7}$~Anima saturata calcabit favum,\\ et anima esuriens etiam amarum pro dulci sumet.\\
${}^{8}$~Sicut avis transmigrans de nido suo,\\ sic vir qui derelinquit locum suum.\\
${}^{9}$~Unguento et variis odoribus delectatur cor,\\ et bonis amici consiliis anima dulcoratur.\end{verse}\end{flushleft}


\begin{flushleft}\begin{verse}${}^{10}$~Amicum tuum et amicum patris tui ne dimiseris,\\ et domum fratris tui ne ingrediaris in die afflictionis tu\ae .\\ Melior est vicinus juxta\\ quam frater procul.\\
${}^{11}$~Stude sapienti\ae , fili mi, et l\ae tifica cor meum,\\ ut possis exprobranti respondere sermonem.\\
${}^{12}$~Astutus videns malum, absconditus est~:\\ parvuli transeuntes sustinuerunt dispendia.\\
${}^{13}$~Tolle vestimentum ejus qui spopondit pro extraneo,\\ et pro alienis aufer ei pignus.\\
${}^{14}$~Qui benedicit proximo suo voce grandi,\\ de nocte consurgens maledicenti similis erit.\\
${}^{15}$~Tecta perstillantia in die frigoris\\ et litigiosa mulier comparantur.\\
${}^{16}$~Qui retinet eam quasi qui ventum teneat,\\ et oleum dexter\ae\ su\ae\ vocabit.\\
${}^{17}$~Ferrum ferro exacuitur,\\ et homo exacuit faciem amici sui.\\
${}^{18}$~Qui servat ficum comedet fructus ejus,\\ et qui custos est domini sui glorificabitur.\\
${}^{19}$~Quomodo in aquis resplendent vultus prospicientium,\\ sic corda hominum manifesta sunt prudentibus.\\
${}^{20}$~Infernus et perditio numquam implentur~:\\ similiter et oculi hominum insatiabiles.\\
${}^{21}$~Quomodo probatur in conflatorio argentum et in fornace aurum,\\ sic probatur homo ore laudantis.\\ Cor iniqui inquirit mala,\\ cor autem rectum inquirit scientiam.\\
${}^{22}$~Si contuderis stultum in pila\\ quasi ptisanas feriente desuper pilo,\\ non auferetur ab eo stultitia ejus.\end{verse}\end{flushleft}


\begin{flushleft}\begin{verse}${}^{23}$~Diligenter agnosce vultum pecoris tui,\\ tuosque greges considera~:\\
${}^{24}$~non enim habebis jugiter potestatem,\\ sed corona tribuetur in generationem et generationem.\\
${}^{25}$~Aperta sunt prata, et apparuerunt herb\ae\ virentes,\\ et collecta sunt fœna de montibus.\\
${}^{26}$~Agni ad vestimentum tuum,\\ et h\ae di ad agri pretium.\\
${}^{27}$~Sufficiat tibi lac caprarum in cibos tuos,\\ et in necessaria domus tu\ae , et ad victum ancillis tuis.\end{verse}\end{flushleft}


\Needspace{2.5\baselineskip}\versal{28}\begin{flushleft}\begin{verse}\vspace{-19pt}\hspace{6pt}Fugit impius nemine persequente~;\\\hspace{6pt} justus autem, quasi leo confidens, absque terrore erit.\\
${}^{2}$~Propter peccata terr\ae\ multi principes ejus~;\\ et propter hominis sapientiam, et horum scientiam qu\ae\ dicuntur,\\ vita ducis longior erit.\\
${}^{3}$~Vir pauper calumnians pauperes\\ similis est imbri vehementi in quo paratur fames.\end{verse}\end{flushleft}


\begin{flushleft}\begin{verse}${}^{4}$~Qui derelinquunt legem laudant impium~;\\ qui custodiunt, succenduntur contra eum.\\
${}^{5}$~Viri mali non cogitant judicium~;\\ qui autem inquirunt Dominum animadvertunt omnia.\\
${}^{6}$~Melior est pauper ambulans in simplicitate sua\\ quam dives in pravis itineribus.\\
${}^{7}$~Qui custodit legem filius sapiens est~;\\ qui autem comessatores pascit confundit patrem suum.\\
${}^{8}$~Qui coacervat divitias usuris et fœnore,\\ liberali in pauperes congregat eas.\\
${}^{9}$~Qui declinat aures suas ne audiat legem,\\ oratio ejus erit execrabilis.\\
${}^{10}$~Qui decipit justos in via mala, in interitu suo corruet,\\ et simplices possidebunt bona ejus.\\
${}^{11}$~Sapiens sibi videtur vir dives~;\\ pauper autem prudens scrutabitur eum.\\
${}^{12}$~In exsultatione justorum multa gloria est~;\\ regnantibus impiis, ruin\ae\ hominum.\\
${}^{13}$~Qui abscondit scelera sua non dirigetur~;\\ qui autem confessus fuerit et reliquerit ea, misericordiam consequetur.\\
${}^{14}$~Beatus homo qui semper est pavidus~;\\ qui vero mentis est dur\ae\ corruet in malum.\\
${}^{15}$~Leo rugiens et ursus esuriens,\\ princeps impius super populum pauperem.\\
${}^{16}$~Dux indigens prudentia multos opprimet per calumniam~;\\ qui autem odit avaritiam, longi fient dies ejus.\\
${}^{17}$~Hominem qui calumniatur anim\ae\ sanguinem,\\ si usque ad lacum fugerit, nemo sustinet.\\
${}^{18}$~Qui ambulat simpliciter salvus erit~;\\ qui perversis graditur viis concidet semel.\\
${}^{19}$~Qui operatur terram suam satiabitur panibus~;\\ qui autem sectatur otium replebitur egestate.\end{verse}\end{flushleft}


\begin{flushleft}\begin{verse}${}^{20}$~Vir fidelis multum laudabitur~;\\ qui autem festinat ditari non erit innocens.\\
${}^{21}$~Qui cognoscit in judicio faciem non bene facit~;\\ iste et pro buccella panis deserit veritatem.\\
${}^{22}$~Vir qui festinat ditari, et aliis invidet,\\ ignorat quod egestas superveniet ei.\\
${}^{23}$~Qui corripit hominem gratiam postea inveniet apud eum,\\ magis quam ille qui per lingu\ae\ blandimenta decipit.\\
${}^{24}$~Qui subtrahit aliquid a patre suo et a matre,\\ et dicit hoc non esse peccatum,\\ particeps homicid\ae\ est.\\
${}^{25}$~Qui se jactat et dilatat, jurgia concitat~;\\ qui vero sperat in Domino sanabitur.\\
${}^{26}$~Qui confidit in corde suo stultus est~;\\ qui autem graditur sapienter, ipse salvabitur.\\
${}^{27}$~Qui dat pauperi non indigebit~;\\ qui despicit deprecantem sustinebit penuriam.\\
${}^{28}$~Cum surrexerint impii, abscondentur homines~;\\ cum illi perierint, multiplicabuntur justi.\end{verse}\end{flushleft}


\Needspace{2.5\baselineskip}\versal{29}\begin{flushleft}\begin{verse}\vspace{-19pt}\hspace{6pt}Viro qui corripientem dura cervice contemnit,\\\hspace{6pt} repentinus ei superveniet interitus,\\ et eum sanitas non sequetur.\\
${}^{2}$~In multiplicatione justorum l\ae tabitur vulgus~;\\ cum impii sumpserint principatum, gemet populus.\\
${}^{3}$~Vir qui amat sapientiam l\ae tificat patrem suum~;\\ qui autem nutrit scorta perdet substantiam.\\
${}^{4}$~Rex justus erigit terram~;\\ vir avarus destruet eam.\\
${}^{5}$~Homo qui blandis fictisque sermonibus loquitur amico suo\\ rete expandit gressibus ejus.\\
${}^{6}$~Peccantem virum iniquum involvet laqueus,\\ et justus laudabit atque gaudebit.\\
${}^{7}$~Novit justus causam pauperum~;\\ impius ignorat scientiam.\\
${}^{8}$~Homines pestilentes dissipant civitatem~;\\ sapientes vero avertunt furorem.\\
${}^{9}$~Vir sapiens si cum stulto contenderit,\\ sive irascatur, sive rideat, non inveniet requiem.\\
${}^{10}$~Viri sanguinum oderunt simplicem~;\\ justi autem qu\ae runt animam ejus.\\
${}^{11}$~Totum spiritum suum profert stultus~;\\ sapiens differt, et reservat in posterum.\\
${}^{12}$~Princeps qui libenter audit verba mendacii,\\ omnes ministros habet impios.\\
${}^{13}$~Pauper et creditor obviaverunt sibi~:\\ utriusque illuminator est Dominus.\\
${}^{14}$~Rex qui judicat in veritate pauperes,\\ thronus ejus in \ae ternum firmabitur.\end{verse}\end{flushleft}


\begin{flushleft}\begin{verse}${}^{15}$~Virga atque correptio tribuit sapientiam~;\\ puer autem qui dimittitur voluntati su\ae\ confundit matrem suam.\\
${}^{16}$~In multiplicatione impiorum multiplicabuntur scelera,\\ et justi ruinas eorum videbunt.\\
${}^{17}$~Erudi filium tuum, et refrigerabit te,\\ et dabit delicias anim\ae\ tu\ae .\\
${}^{18}$~Cum prophetia defecerit, dissipabitur populus~;\\ qui vero custodit legem beatus est.\\
${}^{19}$~Servus verbis non potest erudiri,\\ quia quod dicis intelligit, et respondere contemnit.\\
${}^{20}$~Vidisti hominem velocem ad loquendum~?\\ stultitia magis speranda est quam illius correptio.\\
${}^{21}$~Qui delicate a pueritia nutrit servum suum\\ postea sentiet eum contumacem.\end{verse}\end{flushleft}


\begin{flushleft}\begin{verse}${}^{22}$~Vir iracundus provocat rixas,\\ et qui ad indignandum facilis est erit ad peccandum proclivior.\\
${}^{23}$~Superbum sequitur humilitas,\\ et humilem spiritu suscipiet gloria.\\
${}^{24}$~Qui cum fure participat odit animam suam~;\\ adjurantem audit, et non indicat.\\
${}^{25}$~Qui timet hominem cito corruet~;\\ qui sperat in Domino sublevabitur.\\
${}^{26}$~Multi requirunt faciem principis,\\ et judicium a Domino egreditur singulorum.\\
${}^{27}$~Abominantur justi virum impium,\\ et abominantur impii eos qui in recta sunt via.\\ Verbum custodiens filius\\ extra perditionem erit.\end{verse}\end{flushleft}


\Needspace{2.5\baselineskip}\versal{30}~Verba Congregantis, filii Vomentis. Visio quam locutus est vir cum quo est Deus, et qui Deo secum morante confortatus, ait~:
\begin{flushleft}\begin{verse}\vspace{6pt}${}^{2}$~Stultissimus sum virorum,\\ et sapientia hominum non est mecum.\\
${}^{3}$~Non didici sapientiam,\\ et non novi scientiam sanctorum.\\
${}^{4}$~Quis ascendit in c\ae lum, atque descendit~?\\ quis continuit spiritum in manibus suis~?\\ quis colligavit aquas quasi in vestimento~?\\ quis suscitavit omnes terminos terr\ae~?\\ quod nomen est ejus, et quod nomen filii ejus, si nosti~?\\
${}^{5}$~Omnis sermo Dei ignitus~:\\ clypeus est sperantibus in se.\\
${}^{6}$~Ne addas quidquam verbis illius,\\ et arguaris, inveniarisque mendax.\\
${}^{7}$~Duo rogavi te~:\\ ne deneges mihi antequam moriar~:\\
${}^{8}$~vanitatem et verba mendacia longe fac a me~;\\ mendicitatem et divitias ne dederis mihi~:\\ tribue tantum victui meo necessaria,\\
${}^{9}$~ne forte satiatus illiciar ad negandum,\\ et dicam~: Quis est Dominus~?\\ aut egestate compulsus, furer,\\ et perjurem nomen Dei mei.\\
${}^{10}$~Ne accuses servum ad dominum suum,\\ ne forte maledicat tibi, et corruas.\end{verse}\end{flushleft}


\begin{flushleft}\begin{verse}${}^{11}$~Generatio qu\ae\ patri suo maledicit,\\ et qu\ae\ matri su\ae\ non benedicit~;\\
${}^{12}$~generatio qu\ae\ sibi munda videtur,\\ et tamen non est lota a sordibus suis~;\\
${}^{13}$~generatio cujus excelsi sunt oculi,\\ et palpebr\ae\ ejus in alta surrect\ae~;\\
${}^{14}$~generatio qu\ae\ pro dentibus gladios habet,\\ et commandit molaribus suis,\\ ut comedat inopes de terra,\\ et pauperes ex hominibus.\\
${}^{15}$~Sanguisug\ae\ du\ae\ sunt fili\ae ,\\ dicentes~: Affer, affer.\\ Tria sunt insaturabilia,\\ et quartum quod numquam dicit~: Sufficit.\\
${}^{16}$~Infernus, et os vulv\ae ,\\ et terra qu\ae\ non satiatur aqua~:\\ ignis vero numquam dicit~: Sufficit.\\
${}^{17}$~Oculum qui subsannat patrem,\\ et qui despicit partum matris su\ae ,\\ effodiant eum corvi de torrentibus,\\ et comedant eum filii aquil\ae~!\end{verse}\end{flushleft}


\begin{flushleft}\begin{verse}${}^{18}$~Tria sunt difficilia mihi,\\ et quartum penitus ignoro~:\\
${}^{19}$~viam aquil\ae\ in c\ae lo,\\ viam colubri super petram,\\ viam navis in medio mari,\\ et viam viri in adolescentia.\\
${}^{20}$~Talis est et via mulieris adulter\ae ,\\ qu\ae\ comedit, et tergens os suum\\ dicit~: Non sum operata malum.\\
${}^{21}$~Per tria movetur terra,\\ et quartum non potest sustinere~:\\
${}^{22}$~per servum, cum regnaverit~;\\ per stultum, cum saturatus fuerit cibo~;\\
${}^{23}$~per odiosam mulierem, cum in matrimonio fuerit assumpta~;\\ et per ancillam, cum fuerit h\ae res domin\ae\ su\ae .\end{verse}\end{flushleft}


\begin{flushleft}\begin{verse}${}^{24}$~Quatuor sunt minima terr\ae ,\\ et ipsa sunt sapientiora sapientibus~:\\
${}^{25}$~formic\ae , populus infirmus,\\ qui pr\ae parat in messe cibum sibi~;\\
${}^{26}$~lepusculus, plebs invalida,\\ qui collocat in petra cubile suum~;\\
${}^{27}$~regem locusta non habet,\\ et egreditur universa per turmas suas~;\\
${}^{28}$~stellio manibus nititur,\\ et moratur in \ae dibus regis.\\
${}^{29}$~Tria sunt qu\ae\ bene gradiuntur,\\ et quartum quod incedit feliciter~:\\
${}^{30}$~leo, fortissimus bestiarum,\\ ad nullius pavebit occursum~;\\
${}^{31}$~gallus succinctus lumbos~;\\ et aries~; nec est rex, qui resistat ei.\\
${}^{32}$~Est qui stultus apparuit postquam elevatus est in sublime~;\\ si enim intellexisset, ori suo imposuisset manum.\\
${}^{33}$~Qui autem fortiter premit ubera ad eliciendum lac exprimit butyrum~;\\ et qui vehementer emungit elicit sanguinem~;\\ et qui provocat iras producit discordias.\end{verse}\end{flushleft}


\Needspace{2.5\baselineskip}\versal{31}~Verba Lamuelis regis. Visio qua erudivit eum mater sua.
\begin{flushleft}\begin{verse}\vspace{6pt}${}^{2}$~Quid, dilecte mi~? quid, dilecte uteri mei~?\\ quid, dilecte votorum meorum~?\\
${}^{3}$~Ne dederis mulieribus substantiam tuam,\\ et divitias tuas ad delendos reges.\\
${}^{4}$~Noli regibus, o Lamuel, noli regibus dare vinum,\\ quia nullum secretum est ubi regnat ebrietas~;\\
${}^{5}$~et ne forte bibant, et obliviscantur judiciorum,\\ et mutent causam filiorum pauperis.\\
${}^{6}$~Date siceram mœrentibus,\\ et vinum his qui amaro sunt animo.\\
${}^{7}$~Bibant, et obliviscantur egestatis su\ae ,\\ et doloris sui non recordentur amplius.\\
${}^{8}$~Aperi os tuum muto,\\ et causis omnium filiorum qui pertranseunt.\\
${}^{9}$~Aperi os tuum, decerne quod justum est,\\ et judica inopem et pauperem.\end{verse}\end{flushleft}


\begin{flushleft}\begin{verse}${}^{10}$~Mulierem fortem quis inveniet~?\\ procul et de ultimis finibus pretium ejus.\\
${}^{11}$~Confidit in ea cor viri sui,\\ et spoliis non indigebit.\\
${}^{12}$~Reddet ei bonum, et non malum,\\ omnibus diebus vit\ae\ su\ae .\\
${}^{13}$~Qu\ae sivit lanam et linum,\\ et operata est consilia manuum suarum.\\
${}^{14}$~Facta est quasi navis institoris,\\ de longe portans panem suum.\\
${}^{15}$~Et de nocte surrexit,\\ deditque pr\ae dam domesticis suis,\\ et cibaria ancillis suis.\\
${}^{16}$~Consideravit agrum, et emit eum~;\\ de fructu manuum suarum plantavit vineam.\\
${}^{17}$~Accinxit fortitudine lumbos suos,\\ et roboravit brachium suum.\\
${}^{18}$~Gustavit, et vidit quia bona est negotiatio ejus~;\\ non extinguetur in nocte lucerna ejus.\\
${}^{19}$~Manum suam misit ad fortia,\\ et digiti ejus apprehenderunt fusum.\\
${}^{20}$~Manum suam aperuit inopi,\\ et palmas suas extendit ad pauperem.\\
${}^{21}$~Non timebit domui su\ae\ a frigoribus nivis~;\\ omnes enim domestici ejus vestiti sunt duplicibus.\\
${}^{22}$~Stragulatam vestem fecit sibi~;\\ byssus et purpura indumentum ejus.\\
${}^{23}$~Nobilis in portis vir ejus,\\ quando sederit cum senatoribus terr\ae .\\
${}^{24}$~Sindonem fecit, et vendidit,\\ et cingulum tradidit Chanan\ae o.\\
${}^{25}$~Fortitudo et decor indumentum ejus,\\ et ridebit in die novissimo.\\
${}^{26}$~Os suum aperuit sapienti\ae ,\\ et lex clementi\ae\ in lingua ejus.\\
${}^{27}$~Consideravit semitas domus su\ae ,\\ et panem otiosa non comedit.\\
${}^{28}$~Surrexerunt filii ejus, et beatissimam pr\ae dicaverunt~;\\ vir ejus, et laudavit eam.\\
${}^{29}$~Mult\ae\ fili\ae\ congregaverunt divitias~;\\ tu supergressa es universas.\\
${}^{30}$~Fallax gratia, et vana est pulchritudo~:\\ mulier timens Dominum, ipsa laudabitur.\\
${}^{31}$~Date ei de fructu manuum suarum,\\ et laudent eam in portis opera ejus.\end{verse}\end{flushleft}


