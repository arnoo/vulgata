\bbook{Ecclesiasticus Jesu, filii Sirach}
{Ecclesiasticus}{images/genese_heading}

\begin{center}\textsc{Prologus}\end{center}\vspace{-6pt} Multorum nobis et magnorum per legem, et prophetas, aliosque qui secuti sunt illos, sapientia demonstrata est, in quibus oportet laudare Isra\"el doctrin\ae\ et sapienti\ae\ causa, quia non solum ipsos loquentes necesse est esse peritos, sed etiam extraneos posse et dicentes et scribentes doctissimos fieri. Avus meus Jesus, postquam se amplius dedit ad diligentiam lectionis legis, et prophetarum, et aliorum librorum qui nobis a parentibus nostris traditi sunt, voluit et ipse scribere aliquid horum qu\ae\ ad doctrinam et sapientiam pertinent, ut desiderantes discere, et illorum periti facti, magis magisque attendant animo, et confirmentur ad legitimam vitam. Hortor itaque venire vos cum benevolentia, et attentiori studio lectionem facere, et veniam habere in illis, in quibus videmur, sequentes imaginem sapienti\ae , deficere in verborum compositione. Nam deficiunt verba hebraica, quando fuerint translata ad alteram linguam~: non autem solum h\ae c, sed et ipsa lex, et prophet\ae , ceteraque aliorum librorum non parvam habent differentiam quando inter se dicuntur. Nam in octavo et trigesimo anno temporibus Ptolem\ae i Evergetis regis, postquam perveni in \AE gyptum, et cum multum temporis ibi fuissem, inveni ibi libros relictos, non parv\ae\ neque contemnend\ae\ doctrin\ae . Itaque bonum et necessarium putavi et ipse aliquam addere diligentiam et laborem interpretandi librum istum~: et multa vigilia attuli doctrinam in spatio temporis, ad illa qu\ae\ ad finem ducunt, librum istum dare, et illis qui volunt animum intendere, et discere quemadmodum oporteat instituere mores, qui secundum legem Domini proposuerint vitam agere. 

\Needspace{2.5\baselineskip}\versal{1}\begin{flushleft}\begin{verse}\vspace{-19pt}Omnis sapientia a Domino Deo est~:\\ et cum illo fuit semper, et est ante \ae vum.\\
${}^{2}$~Arenam maris, et pluvi\ae\ guttas,\\ et dies s\ae culi, quis dinumeravit~?\\ altitudinem c\ae li, et latitudinem terr\ae ,\\ et profundum abyssi, quis dimensus est~?\\
${}^{3}$~sapientiam Dei pr\ae cedentem omnia, quis investigavit~?\\
${}^{4}$~Prior omnium creata est sapientia,\\ et intellectus prudenti\ae\ ab \ae vo.\\
${}^{5}$~Fons sapienti\ae\ verbum Dei in excelsis,\\ et ingressus illius mandata \ae terna.\\
${}^{6}$~Radix sapienti\ae\ cui revelata est~?\\ et astutias illius quis agnovit~?\\
${}^{7}$~disciplina sapienti\ae\ cui revelata est et manifestata~?\\ et multiplicationem ingressus illius quis intellexit~?\\
${}^{8}$~Unus est altissimus, Creator omnipotens,\\ et rex potens et metuendus nimis,\\ sedens super thronum illius, et dominans Deus.\\
${}^{9}$~Ipse creavit illam in Spiritu Sancto,\\ et vidit, et dinumeravit, et mensus est~:\\
${}^{10}$~et effudit illam super omnia opera sua,\\ et super omnem carnem, secundum datum suum,\\ et pr\ae buit illam diligentibus se.\end{verse}\end{flushleft}


\begin{flushleft}\begin{verse}${}^{11}$~Timor Domini gloria, et gloriatio,\\ et l\ae titia, et corona exsultationis.\\
${}^{12}$~Timor Domini delectabit cor,\\ et dabit l\ae titiam, et gaudium, et longitudinem dierum.\\
${}^{13}$~Timenti Dominum bene erit in extremis,\\ et in die defunctionis su\ae\ benedicetur.\\
${}^{14}$~Dilectio Dei honorabilis sapientia~:\\
${}^{15}$~quibus autem apparuerit in visu diligunt eam in visione,\\ et in agnitione magnalium suorum.\\
${}^{16}$~Initium sapienti\ae\ timor Domini~:\\ et cum fidelibus in vulva concreatus est~:\\ cum electis feminis graditur,\\ et cum justis et fidelibus agnoscitur.\\
${}^{17}$~Timor Domini scienti\ae\ religiositas~:\\
${}^{18}$~religiositas custodiet et justificabit cor~;\\ jucunditatem atque gaudium dabit.\\
${}^{19}$~Timenti Dominum bene erit,\\ et in diebus consummationis illius benedicetur.\\
${}^{20}$~Plenitudo sapienti\ae\ est timere Deum,\\ et plenitudo a fructibus illius.\\
${}^{21}$~Omnem domum illius implebit a generationibus,\\ et receptacula a thesauris illius.\\
${}^{22}$~Corona sapienti\ae\ timor Domini,\\ replens pacem et salutis fructum~:\\
${}^{23}$~et vidit, et dinumeravit eam~:\\ utraque autem sunt dona Dei.\\
${}^{24}$~Scientiam et intellectum prudenti\ae\ sapientia compartietur,\\ et gloriam tenentium se exaltat.\\
${}^{25}$~Radix sapienti\ae\ est timere Dominum,\\ et rami illius long\ae vi.\\
${}^{26}$~In thesauris sapienti\ae\ intellectus et scienti\ae\ religiositas~:\\ execratio autem peccatoribus sapientia.\\
${}^{27}$~Timor Domini expellit peccatum~:\\
${}^{28}$~nam qui sine timore est non poterit justificari~:\\ iracundia enim animositatis illius subversio illius est.\\
${}^{29}$~Usque in tempus sustinebit patiens,\\ et postea redditio jucunditatis.\\
${}^{30}$~Bonus sensus usque in tempus abscondet verba illius,\\ et labia multorum enarrabunt sensum illius.\\
${}^{31}$~In thesauris sapienti\ae\ significatio disciplin\ae~:\\
${}^{32}$~execratio autem peccatori cultura Dei.\\
${}^{33}$~Fili, concupiscens sapientiam, conserva justitiam,\\ et Deus pr\ae bebit illam tibi.\\
${}^{34}$~Sapientia enim et disciplina timor Domini~:\\ et quod beneplacitum est illi,\\
${}^{35}$~fides et mansuetudo,\\ et adimplebit thesauros illius.\\
${}^{36}$~Ne sis incredibilis timori Domini,\\ et ne accesseris ad illum duplici corde.\\
${}^{37}$~Ne fueris hypocrita in conspectu hominum,\\ et non scandalizeris in labiis tuis.\\
${}^{38}$~Attende in illis, ne forte cadas,\\ et adducas anim\ae\ tu\ae\ inhonorationem~:\\
${}^{39}$~et revelet Deus absconsa tua,\\ et in medio synagog\ae\ elidat te~:\\
${}^{40}$~quoniam accessisti maligne ad Dominum,\\ et cor tuum plenum est dolo et fallacia.\end{verse}\end{flushleft}


\Needspace{2.5\baselineskip}\versal{2}\begin{flushleft}\begin{verse}\vspace{-19pt}Fili, accedens ad servitutem Dei\\ sta in justitia et timore,\\ et pr\ae para animam tuam ad tentationem.\\
${}^{2}$~Deprime cor tuum, et sustine~:\\ inclina aurem tuam, et suscipe verba intellectus~:\\ et ne festines in tempore obductionis.\\
${}^{3}$~Sustine sustentationes Dei~:\\ conjungere Deo, et sustine,\\ ut crescat in novissimo vita tua.\\
${}^{4}$~Omne quod tibi applicitum fuerit accipe~:\\ et in dolore sustine,\\ et in humilitate tua patientiam habe~:\\
${}^{5}$~quoniam in igne probatur aurum et argentum,\\ homines vero receptibiles in camino humiliationis.\\
${}^{6}$~Crede Deo, et recuperabit te~:\\ et dirige viam tuam, et spera in illum~:\\ serva timorem illius, et in illo veterasce.\end{verse}\end{flushleft}


\begin{flushleft}\begin{verse}${}^{7}$~Metuentes Dominum, sustinete misericordiam ejus~:\\ et non deflectatis ab illo, ne cadatis.\\
${}^{8}$~Qui timetis Dominum, credite illi,\\ et non evacuabitur merces vestra.\\
${}^{9}$~Qui timetis Dominum, sperate in illum,\\ et in oblectationem veniet vobis misericordia.\\
${}^{10}$~Qui timetis Dominum, diligite illum,\\ et illuminabuntur corda vestra.\\
${}^{11}$~Respicite, filii, nationes hominum~:\\ et scitote quia nullus speravit in Domino et confusus est.\\
${}^{12}$~Quis enim permansit in mandatis ejus, et derelictus est~?\\ aut quis invocavit eum, et despexit illum~?\\
${}^{13}$~Quoniam pius et misericors est Deus,\\ et remittet in die tribulationis peccata,\\ et protector est omnibus exquirentibus se in veritate.\end{verse}\end{flushleft}


\begin{flushleft}\begin{verse}${}^{14}$~V\ae\ duplici corde, et labiis scelestis,\\ et manibus malefacientibus,\\ et peccatori terram ingredienti duabus viis~!\\
${}^{15}$~V\ae\ dissolutis corde, qui non credunt Deo,\\ et ideo non protegentur ab eo~!\\
${}^{16}$~V\ae\ his qui perdiderunt sustinentiam,\\ et qui dereliquerunt vias rectas,\\ et diverterunt in vias pravas~!\\
${}^{17}$~Et quid facient cum inspicere cœperit Dominus~?\\
${}^{18}$~Qui timent Dominum non erunt incredibiles verbo illius~:\\ et qui diligunt illum conservabunt viam illius.\\
${}^{19}$~Qui timent Dominum inquirent qu\ae\ beneplacita sunt ei,\\ et qui diligunt eum replebuntur lege ipsius.\\
${}^{20}$~Qui timent Dominum pr\ae parabunt corda sua,\\ et in conspectu illius sanctificabunt animas suas.\\
${}^{21}$~Qui timent Dominum custodiunt mandata illius,\\ et patientiam habebunt usque ad inspectionem illius,\\
${}^{22}$~dicentes~: Si pœnitentiam non egerimus,\\ incidemus in manus Domini, et non in manus hominum.\\
${}^{23}$~Secundum enim magnitudinem ipsius,\\ sic et misericordia illius cum ipso est.\end{verse}\end{flushleft}


\Needspace{2.5\baselineskip}\versal{3}\begin{flushleft}\begin{verse}\vspace{-19pt}Filii sapienti\ae\ ecclesia justorum,\\ et natio illorum obedientia et dilectio.\\
${}^{2}$~Judicium patris audite, filii,\\ et sic facite, ut salvi sitis.\\
${}^{3}$~Deus enim honoravit patrem in filiis~:\\ et judicium matris exquirens, firmavit in filios.\\
${}^{4}$~Qui diligit Deum exorabit pro peccatis,\\ et continebit se ab illis,\\ et in oratione dierum exaudietur.\\
${}^{5}$~Et sicut qui thesaurizat,\\ ita et qui honorificat matrem suam.\\
${}^{6}$~Qui honorat patrem suum jucundabitur in filiis,\\ et in die orationis su\ae\ exaudietur.\\
${}^{7}$~Qui honorat patrem suum vita vivet longiore,\\ et qui obedit patri refrigerabit matrem.\\
${}^{8}$~Qui timet Dominum honorat parentes,\\ et quasi dominis serviet his qui se genuerunt.\\
${}^{9}$~In opere, et sermone, et omni patientia, honora patrem tuum,\\
${}^{10}$~ut superveniat tibi benedictio ab eo,\\ et benedictio illius in novissimo maneat.\\
${}^{11}$~Benedictio patris firmat domos filiorum~:\\ maledictio autem matris eradicat fundamenta.\\
${}^{12}$~Ne glorieris in contumelia patris tui~:\\ non enim est tibi gloria ejus confusio.\\
${}^{13}$~Gloria enim hominis ex honore patris sui,\\ et dedecus filii pater sine honore.\\
${}^{14}$~Fili, suscipe senectam patris tui,\\ et non contristes eum in vita illius~:\\
${}^{15}$~et si defecerit sensu, veniam da,\\ et ne spernas eum in virtute tua~:\\ eleemosyna enim patris non erit in oblivione.\\
${}^{16}$~Nam pro peccato matris restituetur tibi bonum~:\\
${}^{17}$~et in justitia \ae dificabitur tibi,\\ et in die tribulationis commemorabitur tui,\\ et sicut in sereno glacies, solventur peccata tua.\\
${}^{18}$~Quam mal\ae\ fam\ae\ est qui derelinquit patrem,\\ et est maledictus a Deo qui exasperat matrem~!\end{verse}\end{flushleft}


\begin{flushleft}\begin{verse}${}^{19}$~Fili, in mansuetudine opera tua perfice,\\ et super hominum gloriam diligeris.\\
${}^{20}$~Quanto magnus es, humilia te in omnibus,\\ et coram Deo invenies gratiam~:\\
${}^{21}$~quoniam magna potentia Dei solius,\\ et ab humilibus honoratur.\\
${}^{22}$~Altiora te ne qu\ae sieris,\\ et fortiora te ne scrutatus fueris~:\\ sed qu\ae\ pr\ae cepit tibi Deus, illa cogita semper,\\ et in pluribus operibus ejus ne fueris curiosus.\\
${}^{23}$~Non est enim tibi necessarium\\ ea, qu\ae\ abscondita sunt, videre oculis tuis.\\
${}^{24}$~In supervacuis rebus noli scrutari multipliciter,\\ et in pluribus operibus ejus non eris curiosus.\\
${}^{25}$~Plurima enim super sensum hominum ostensa sunt tibi~:\\
${}^{26}$~multos quoque supplantavit suspicio illorum,\\ et in vanitate detinuit sensus illorum.\\
${}^{27}$~Cor durum habebit male in novissimo,\\ et qui amat periculum in illo peribit.\\
${}^{28}$~Cor ingrediens duas vias non habebit successus,\\ et pravus corde in illis scandalizabitur.\\
${}^{29}$~Cor nequam gravabitur in doloribus,\\ et peccator adjiciet ad peccandum.\\
${}^{30}$~Synagog\ae\ superborum non erit sanitas,\\ frutex enim peccati radicabitur in illis, et non intelligetur.\\
${}^{31}$~Cor sapientis intelligitur in sapientia,\\ et auris bona audiet cum omni concupiscentia sapientiam.\\
${}^{32}$~Sapiens cor et intelligibile abstinebit se a peccatis,\\ et in operibus justiti\ae\ successus habebit.\end{verse}\end{flushleft}


\begin{flushleft}\begin{verse}${}^{33}$~Ignem ardentem exstinguit aqua,\\ et eleemosyna resistit peccatis~:\\
${}^{34}$~et Deus prospector est ejus qui reddit gratiam~:\\ meminit ejus in posterum,\\ et in tempore casus sui inveniet firmamentum.\end{verse}\end{flushleft}


\Needspace{2.5\baselineskip}\versal{4}\begin{flushleft}\begin{verse}\vspace{-19pt}Fili, eleemosynam pauperis ne defraudes,\\ et oculos tuos ne transvertas a paupere.\\
${}^{2}$~Animam esurientem ne despexeris,\\ et non exasperes pauperem in inopia sua.\\
${}^{3}$~Cor inopis ne afflixeris,\\ et non protrahas datum angustianti.\\
${}^{4}$~Rogationem contribulati ne abjicias,\\ et non avertas faciem tuam ab egeno.\\
${}^{5}$~Ab inope ne avertas oculos tuos propter iram~:\\ et non relinquas qu\ae rentibus tibi retro maledicere.\\
${}^{6}$~Maledicentis enim tibi in amaritudine anim\ae ,\\ exaudietur deprecatio illius~:\\ exaudiet autem eum qui fecit illum.\\
${}^{7}$~Congregationi pauperum affabilem te facito~:\\ et presbytero humilia animam tuam,\\ et magnato humilia caput tuum.\\
${}^{8}$~Declina pauperi sine tristitia aurem tuam,\\ et redde debitum tuum,\\ et responde illi pacifica in mansuetudine.\\
${}^{9}$~Libera eum qui injuriam patitur de manu superbi,\\ et non acide feras in anima tua.\\
${}^{10}$~In judicando esto pupillis misericors ut pater,\\ et pro viro matri illorum~:\\
${}^{11}$~et eris tu velut filius Altissimi obediens,\\ et miserebitur tui magis quam mater.\end{verse}\end{flushleft}


\begin{flushleft}\begin{verse}${}^{12}$~Sapientia filiis suis vitam inspirat~:\\ et suscipit inquirentes se,\\ et pr\ae ibit in via justiti\ae .\\
${}^{13}$~Et qui illam diligit, diligit vitam,\\ et qui vigilaverint ad illam complectentur placorem ejus.\\
${}^{14}$~Qui tenuerint illam, vitam h\ae reditabunt~:\\ et quo introibit benedicet Deus.\\
${}^{15}$~Qui serviunt ei obsequentes erunt sancto~:\\ et eos qui diligunt illam, diligit Deus.\\
${}^{16}$~Qui audit illam judicabit gentes~:\\ et qui intuetur illam permanebit confidens.\\
${}^{17}$~Si crediderit ei, h\ae reditabit illam,\\ et erunt in confirmatione creatur\ae\ illius~:\\
${}^{18}$~quoniam in tentatione ambulat cum eo,\\ et in primis eligit eum.\\
${}^{19}$~Timorem, et metum, et probationem inducet super illum~:\\ et cruciabit illum in tribulatione doctrin\ae\ su\ae ,\\ donec tentet eum in cogitationibus suis,\\ et credat anim\ae\ illius.\\
${}^{20}$~Et firmabit illum, et iter adducet directum ad illum,\\ et l\ae tificabit illum~:\\
${}^{21}$~et denudabit absconsa sua illi,\\ et thesaurizabit super illum scientiam et intellectum justiti\ae .\\
${}^{22}$~Si autem oberraverit, derelinquet eum,\\ et tradet eum in manus inimici sui.\end{verse}\end{flushleft}


\begin{flushleft}\begin{verse}${}^{23}$~Fili, conserva tempus,\\ et devita a malo.\\
${}^{24}$~Pro anima tua ne confundaris dicere verum~:\\
${}^{25}$~est enim confusio adducens peccatum,\\ et est confusio adducens gloriam et gratiam.\\
${}^{26}$~Ne accipias faciem adversus faciem tuam,\\ nec adversus animam tuam mendacium.\\
${}^{27}$~Ne reverearis proximum tuum in casu suo,\\
${}^{28}$~nec retineas verbum in tempore salutis.\\ Non abscondas sapientiam tuam in decore suo~:\\
${}^{29}$~in lingua enim sapientia dignoscitur~:\\ et sensus, et scientia, et doctrina in verbo sensati,\\ et firmamentum in operibus justiti\ae .\\
${}^{30}$~Non contradicas verbo veritatis ullo modo,\\ et de mendacio ineruditionis tu\ae\ confundere.\\
${}^{31}$~Non confundaris confiteri peccata tua,\\ et ne subjicias te omni homini pro peccato.\\
${}^{32}$~Noli resistere contra faciem potentis,\\ nec coneris contra ictum fluvii.\\
${}^{33}$~Pro justitia agonizare pro anima tua,\\ et usque ad mortem certa pro justitia~:\\ et Deus expugnabit pro te inimicos tuos.\\
${}^{34}$~Noli citatus esse in lingua tua,\\ et inutilis, et remissus in operibus tuis.\\
${}^{35}$~Noli esse sicut leo in domo tua,\\ evertens domesticos tuos, et opprimens subjectos tibi.\\
${}^{36}$~Non sit porrecta manus tua ad accipiendum,\\ et ad dandum collecta.\end{verse}\end{flushleft}


\Needspace{2.5\baselineskip}\versal{5}\begin{flushleft}\begin{verse}\vspace{-19pt}Noli attendere ad possessiones iniquas,\\ et ne dixeris~: Est mihi sufficiens vita~:\\ nihil enim proderit in tempore vindict\ae\ et obductionis.\\
${}^{2}$~Ne sequaris in fortitudine tua concupiscentiam cordis tui,\\
${}^{3}$~et ne dixeris~: Quomodo potui~?\\ aut, Quis me subjiciet propter facta mea~?\\ Deus enim vindicans vindicabit.\\
${}^{4}$~Ne dixeris~: Peccavi~: et quid mihi accidit triste~?\\ Altissimus enim est patiens redditor.\\
${}^{5}$~De propitiatio peccato noli esse sine metu,\\ neque adjicias peccatum super peccatum.\\
${}^{6}$~Et ne dicas~: Miseratio Domini magna est,\\ multitudinis peccatorum meorum miserebitur~:\\
${}^{7}$~misericordia enim et ira ab illo cito proximant,\\ et in peccatores respicit ira illius.\\
${}^{8}$~Non tardes converti ad Dominum,\\ et ne differas de die in diem~:\\
${}^{9}$~subito enim veniet ira illius,\\ et in tempore vindict\ae\ disperdet te.\\
${}^{10}$~Noli anxius esse in divitiis injustis~:\\ non enim proderunt tibi in die obductionis et vindict\ae .\end{verse}\end{flushleft}


\begin{flushleft}\begin{verse}${}^{11}$~Non ventiles te in omnem ventum,\\ et non eas in omnem viam~:\\ sic enim omnis peccator probatur in duplici lingua.\\
${}^{12}$~Esto firmus in via Domini,\\ et in veritate sensus tui et scientia~:\\ et prosequatur te verbum pacis et justiti\ae .\\
${}^{13}$~Esto mansuetus ad audiendum verbum, ut intelligas,\\ et cum sapientia proferas responsum verum.\\
${}^{14}$~Si est tibi intellectus, responde proximo~:\\ sin autem, sit manus tua super os tuum,\\ ne capiaris in verbo indisciplinato, et confundaris.\\
${}^{15}$~Honor et gloria in sermone sensati~:\\ lingua vero imprudentis subversio est ipsius.\\
${}^{16}$~Non appelleris susurro,\\ et lingua tua ne capiaris et confundaris~:\\
${}^{17}$~super furem enim est confusio et pœnitentia,\\ et denotatio pessima super bilinguem~:\\ susurratori autem odium, et inimicitia, et contumelia.\\
${}^{18}$~Justifica pusillum\\ et magnum similiter.\end{verse}\end{flushleft}


\Needspace{2.5\baselineskip}\versal{6}\begin{flushleft}\begin{verse}\vspace{-19pt}Noli fieri pro amico inimicus proximo~:\\ improperium enim et contumeliam malus h\ae reditabit~:\\ et omnis peccator invidus et bilinguis.\end{verse}\end{flushleft}


\begin{flushleft}\begin{verse}\vspace{6pt}${}^{2}$~Non te extollas in cogitatione anim\ae\ tu\ae\ velut taurus,\\ ne forte elidatur virtus tua per stultitiam~:\\
${}^{3}$~et folia tua comedat, et fructus tuos perdat,\\ et relinquaris velut lignum aridum in eremo.\\
${}^{4}$~Anima enim nequam disperdet qui se habet,\\ et in gaudium inimicis dat illum,\\ et deducet in sortem impiorum.\end{verse}\end{flushleft}


\begin{flushleft}\begin{verse}${}^{5}$~Verbum dulce multiplicat amicos et mitigat inimicos,\\ et lingua eucharis in bono homine abundat.\\
${}^{6}$~Multi pacifici sint tibi~:\\ et consiliarius sit tibi unus de mille.\\
${}^{7}$~Si possides amicum, in tentatione posside eum,\\ et ne facile credas ei.\\
${}^{8}$~Est enim amicus secundum tempus suum,\\ et non permanebit in die tribulationis.\\
${}^{9}$~Et est amicus qui convertitur ad inimicitiam,\\ et est amicus qui odium et rixam et convitia denudabit.\\
${}^{10}$~Est autem amicus socius mens\ae ,\\ et non permanebit in die necessitatis.\\
${}^{11}$~Amicus si permanserit fixus, erit tibi quasi co\ae qualis,\\ et in domesticis tuis fiducialiter aget.\\
${}^{12}$~Si humiliaverit se contra te,\\ et a facie tua absconderit se,\\ unanimem habebis amicitiam bonam.\\
${}^{13}$~Ab inimicis tuis separare,\\ et ab amicis tuis attende.\\
${}^{14}$~Amicus fidelis protectio fortis~:\\ qui autem invenit illum, invenit thesaurum.\\
${}^{15}$~Amico fideli nulla est comparatio,\\ et non est digna ponderatio auri et argenti contra bonitatem fidei illius.\\
${}^{16}$~Amicus fidelis medicamentum vit\ae\ et immortalitatis~:\\ et qui metuunt Dominum, invenient illum.\\
${}^{17}$~Qui timet Deum \ae que habebit amicitiam bonam,\\ quoniam secundum illum erit amicus illius.\end{verse}\end{flushleft}


\begin{flushleft}\begin{verse}${}^{18}$~Fili, a juventute tua excipe doctrinam,\\ et usque ad canos invenies sapientiam.\\
${}^{19}$~Quasi is qui arat et seminat accede ad eam,\\ et sustine bonos fructus illius.\\
${}^{20}$~In opere enim ipsius exiguum laborabis,\\ et cito edes de generationibus illius.\\
${}^{21}$~Quam aspera est nimium sapientia indoctis hominibus~!\\ et non permanebit in illa excors.\\
${}^{22}$~Quasi lapidis virtus probatio erit in illis~:\\ et non demorabuntur projicere illam.\\
${}^{23}$~Sapientia enim doctrin\ae\ secundum nomen est ejus,\\ et non est multis manifestata~:\\ quibus autem cognita est,\\ permanet usque ad conspectum Dei.\\
${}^{24}$~Audi, fili, et accipe consilium intellectus,\\ et ne abjicias consilium meum.\\
${}^{25}$~Injice pedem tuum in compedes illius,\\ et in torques illius collum tuum.\\
${}^{26}$~Subjice humerum tuum, et porta illam,\\ et ne acedieris vinculis ejus.\\
${}^{27}$~In omni animo tuo accede ad illam,\\ et in omni virtute tua conserva vias ejus.\\
${}^{28}$~Investiga illam, et manifestabitur tibi~:\\ et continens factus, ne derelinquas eam~:\\
${}^{29}$~in novissimis enim invenies requiem in ea,\\ et convertetur tibi in oblectationem.\\
${}^{30}$~Et erunt tibi compedes ejus in protectionem fortitudinis et bases virtutis,\\ et torques illius in stolam glori\ae~:\\
${}^{31}$~decor enim vit\ae\ est in illa,\\ et vincula illius alligatura salutaris.\\
${}^{32}$~Stolam glori\ae\ indues eam,\\ et coronam gratulationis superpones tibi.\\
${}^{33}$~Fili, si attenderis mihi, disces~:\\ et si accommodaveris animum tuum, sapiens eris.\\
${}^{34}$~Si inclinaveris aurem tuam, excipies doctrinam~:\\ et si dilexeris audire, sapiens eris.\\
${}^{35}$~In multitudine presbyterorum prudentium sta,\\ et sapienti\ae\ illorum ex corde conjungere,\\ ut omnem narrationem Dei possis audire,\\ et proverbia laudis non effugiant a te.\\
${}^{36}$~Et si videris sensatum, evigila ad eum,\\ et gradus ostiorum illius exterat pes tuus.\\
${}^{37}$~Cogitatum tuum habe in pr\ae ceptis Dei,\\ et in mandatis illius maxime assiduus esto~:\\ et ipse dabit tibi cor,\\ et concupiscentia sapienti\ae\ dabitur tibi.\end{verse}\end{flushleft}


\Needspace{2.5\baselineskip}\versal{7}\begin{flushleft}\begin{verse}\vspace{-19pt}Noli facere mala, et non te apprehendent~:\\
${}^{2}$~discede ab iniquo, et deficient mala abs te.\\
${}^{3}$~Fili, non semines mala in sulcis injustiti\ae ,\\ et non metes ea in septuplum.\\
${}^{4}$~Noli qu\ae rere a Domino ducatum,\\ neque a rege cathedram honoris.\\
${}^{5}$~Non te justifices ante Deum,\\ quoniam agnitor cordis ipse est~:\\ et penes regem noli velle videri sapiens.\\
${}^{6}$~Noli qu\ae rere fieri judex,\\ nisi valeas virtute irrumpere iniquitates~:\\ ne forte extimescas faciem potentis,\\ et ponas scandalum in \ae quitate tua.\\
${}^{7}$~Non pecces in multitudinem civitatis,\\ nec te immittas in populum~:\\
${}^{8}$~neque alliges duplicia peccata,\\ nec enim in uno eris immunis.\\
${}^{9}$~Noli esse pusillanimis in animo tuo~:\\
${}^{10}$~exorare et facere eleemosynam ne despicias.\\
${}^{11}$~Ne dicas~: In multitudine munerum meorum respiciet Deus,\\ et offerente me Deo altissimo, munera mea suscipiet.\\
${}^{12}$~Non irrideas hominem in amaritudine anim\ae~:\\ est enim qui humiliat et exaltat circumspector Deus.\\
${}^{13}$~Noli amare mendacium adversus fratrem tuum,\\ neque in amicum similiter facias.\\
${}^{14}$~Noli velle mentiri omne mendacium~:\\ assiduitas enim illius non est bona.\\
${}^{15}$~Noli verbosus esse in multitudine presbyterorum,\\ et non iteres verbum in oratione tua.\\
${}^{16}$~Non oderis laboriosa opera,\\ et rusticationem creatam ab Altissimo.\\
${}^{17}$~Non te reputes in multitudine indisciplinatorum.\\
${}^{18}$~Memento ir\ae , quoniam non tardabit.\\
${}^{19}$~Humilia valde spiritum tuum,\\ quoniam vindicta carnis impii ignis et vermis.\end{verse}\end{flushleft}


\begin{flushleft}\begin{verse}${}^{20}$~Noli pr\ae varicari in amicum pecuniam differentem,\\ neque fratrem carissimum auro spreveris.\\
${}^{21}$~Noli discedere a muliere sensata et bona,\\ quam sortitus es in timore Domini~:\\ gratia enim verecundi\ae\ illius super aurum.\\
${}^{22}$~Non l\ae das servum in veritate operantem,\\ neque mercenarium dantem animam suam.\\
${}^{23}$~Servus sensatus sit tibi dilectus quasi anima tua~:\\ non defraudes illum libertate,\\ neque inopem derelinquas illum.\\
${}^{24}$~Pecora tibi sunt, attende illis~:\\ et si sunt utilia, perseverent apud te.\\
${}^{25}$~Filii tibi sunt~? erudi illos,\\ et curva illos a pueritia illorum.\\
${}^{26}$~Fili\ae\ tibi sunt~? serva corpus illarum,\\ et non ostendas hilarem faciem tuam ad illas.\\
${}^{27}$~Trade filiam, et grande opus feceris~:\\ et homini sensato da illam.\\
${}^{28}$~Mulier si est tibi secundum animam tuam, non projicias illam~:\\ et odibili non credas te.\\ In toto corde tuo\\
${}^{29}$~honora patrem tuum,\\ et gemitus matris tu\ae\ ne obliviscaris~:\\
${}^{30}$~memento quoniam nisi per illos natus non fuisses~:\\ et retribue illis, quomodo et illi tibi.\end{verse}\end{flushleft}


\begin{flushleft}\begin{verse}${}^{31}$~In tota anima tua time Dominum,\\ et sacerdotes illius sanctifica.\\
${}^{32}$~In omni virtute tua dilige eum qui te fecit,\\ et ministros ejus ne derelinquas.\\
${}^{33}$~Honora Deum ex tota anima tua,\\ et honorifica sacerdotes,\\ et propurga te cum brachiis.\\
${}^{34}$~Da illis partem, sicut mandatum est tibi, primitiarum et purgationis,\\ et de negligentia tua purga te cum paucis.\\
${}^{35}$~Datum brachiorum tuorum,\\ et sacrificium sanctificationis offeres Domino,\\ et initia sanctorum.\\
${}^{36}$~Et pauperi porrige manum tuam,\\ ut perficiatur propitiatio et benedictio tua.\\
${}^{37}$~Gratia dati in conspectu omnis viventis,\\ et mortuo non prohibeas gratiam.\\
${}^{38}$~Non desis plorantibus in consolatione,\\ et cum lugentibus ambula.\\
${}^{39}$~Non te pigeat visitare infirmum~:\\ ex his enim in dilectione firmaberis.\\
${}^{40}$~In omnibus operibus tuis memorare novissima tua,\\ et in \ae ternum non peccabis.\end{verse}\end{flushleft}


\Needspace{2.5\baselineskip}\versal{8}\begin{flushleft}\begin{verse}\vspace{-19pt}Non litiges cum homine potente,\\ ne forte incidas in manus illius.\\
${}^{2}$~Non contendas cum viro locuplete,\\ ne forte contra te constituat litem tibi~:\\
${}^{3}$~multos enim perdidit aurum et argentum,\\ et usque ad cor regum extendit et convertit.\\
${}^{4}$~Non litiges cum homine linguato,\\ et non strues in ignem illius ligna.\\
${}^{5}$~Non communices homini indocto,\\ ne male de progenie tua loquatur.\\
${}^{6}$~Ne despicias hominem avertentem se a peccato,\\ neque improperes ei~:\\ memento quoniam omnes in correptione sumus.\\
${}^{7}$~Ne spernas hominem in sua senectute,\\ etenim ex nobis senescunt.\\
${}^{8}$~Noli de mortuo inimico tuo gaudere~:\\ sciens quoniam omnes morimur,\\ et in gaudium nolumus venire.\\
${}^{9}$~Ne despicias narrationem presbyterorum sapientium,\\ et in proverbiis eorum conversare~:\\
${}^{10}$~ab ipsis enim disces sapientiam et doctrinam intellectus,\\ et servire magnatis sine querela.\\
${}^{11}$~Non te pr\ae tereat narratio seniorum,\\ ipsi enim didicerunt a patribus suis~:\\
${}^{12}$~quoniam ab ipsis disces intellectum,\\ et in tempore necessitatis dare responsum.\\
${}^{13}$~Non incendas carbones peccatorum arguens eos,\\ et ne incendaris flamma ignis peccatorum illorum.\\
${}^{14}$~Ne contra faciem stes contumeliosi,\\ ne sedeat quasi insidiator ori tuo.\\
${}^{15}$~Noli fœnerari homini fortiori te~:\\ quod si fœneraveris, quasi perditum habe.\\
${}^{16}$~Non spondeas super virtutem tuam~:\\ quod si spoponderis, quasi restituens cogita.\\
${}^{17}$~Non judices contra judicem,\\ quoniam secundum quod justum est judicat.\\
${}^{18}$~Cum audace non eas in via,\\ ne forte gravet mala sua in te~:\\ ipse enim secundum voluntatem suam vadit,\\ et simul cum stultitia illius peries.\\
${}^{19}$~Cum iracundo non facias rixam,\\ et cum audace non eas in desertum~:\\ quoniam quasi nihil est ante illum sanguis,\\ et ubi non est adjutorium, elidet te.\\
${}^{20}$~Cum fatuis consilium non habeas~:\\ non enim poterunt diligere nisi qu\ae\ eis placent.\\
${}^{21}$~Coram extraneo ne facias consilium~:\\ nescis enim quid pariet.\\
${}^{22}$~Non omni homini cor tuum manifestes,\\ ne forte inferat tibi gratiam falsam, et convicietur tibi.\end{verse}\end{flushleft}


\Needspace{2.5\baselineskip}\versal{9}\begin{flushleft}\begin{verse}\vspace{-19pt}Non zeles mulierem sinus tui,\\ ne ostendat super te malitiam doctrin\ae\ nequam.\\
${}^{2}$~Non des mulieri potestatem anim\ae\ tu\ae ,\\ ne ingrediatur in virtutem tuam, et confundaris.\\
${}^{3}$~Ne respicias mulierem multivolam,\\ ne forte incidas in laqueos illius.\\
${}^{4}$~Cum saltatrice ne assiduus sis,\\ nec audias illam, ne forte pereas in efficacia illius.\\
${}^{5}$~Virginem ne conspicias,\\ ne forte scandalizeris in decore illius.\\
${}^{6}$~Ne des fornicariis animam tuam in ullo,\\ ne perdas te et h\ae reditatem tuam.\\
${}^{7}$~Noli circumspicere in vicis civitatis,\\ nec oberraveris in plateis illius.\\
${}^{8}$~Averte faciem tuam a muliere compta,\\ et ne circumspicias speciem alienam.\\
${}^{9}$~Propter speciem mulieris multi perierunt~:\\ et ex hoc concupiscentia quasi ignis exardescit.\\
${}^{10}$~Omnis mulier qu\ae\ est fornicaria,\\ quasi stercus in via conculcabitur.\\
${}^{11}$~Speciem mulieris alien\ae\ multi admirati, reprobi facti sunt~:\\ colloquium enim illius quasi ignis exardescit.\\
${}^{12}$~Cum aliena muliere ne sedeas omnino,\\ nec accumbas cum ea super cubitum~:\\
${}^{13}$~et non alterceris cum illa in vino,\\ ne forte declinet cor tuum in illam,\\ et sanguine tuo labaris in perditionem.\end{verse}\end{flushleft}


\begin{flushleft}\begin{verse}${}^{14}$~Ne derelinquas amicum antiquum~:\\ novus enim non erit similis illi.\\
${}^{15}$~Vinum novum amicus novus~:\\ veterascet, et cum suavitate bibes illud.\\
${}^{16}$~Non zeles gloriam et opes peccatoris~:\\ non enim scis qu\ae\ futura sit illius subversio.\\
${}^{17}$~Non placeat tibi injuria injustorum,\\ sciens quoniam usque ad inferos non placebit impius.\\
${}^{18}$~Longe abesto ab homine potestatem habente occidendi,\\ et non suspicaberis timorem mortis.\\
${}^{19}$~Et si accesseris ad illum, noli aliquid committere,\\ ne forte auferat vitam tuam.\\
${}^{20}$~Communionem mortis scito,\\ quoniam in medio laqueorum ingredieris,\\ et super dolentium arma ambulabis.\\
${}^{21}$~Secundum virtutem tuam cave te a proximo tuo,\\ et cum sapientibus et prudentibus tracta.\\
${}^{22}$~Viri justi sint tibi conviv\ae ,\\ et in timore Dei sit tibi gloriatio~:\\
${}^{23}$~et in sensu sit tibi cogitatus Dei,\\ et omnis enarratio tua in pr\ae ceptis Altissimi.\\
${}^{24}$~In manu artificum opera laudabuntur,\\ et princeps populi in sapientia sermonis sui,\\ in sensu vero seniorum verbum.\\
${}^{25}$~Terribilis est in civitate sua homo linguosus~:\\ et temerarius in verbo suo odibilis erit.\end{verse}\end{flushleft}


\Needspace{2.5\baselineskip}\versal{10}\begin{flushleft}\begin{verse}\vspace{-19pt}\hspace{6pt}Judex sapiens judicabit populum suum,\\\hspace{6pt} et principatus sensati stabilis erit.\\
${}^{2}$~Secundum judicem populi, sic et ministri ejus~:\\ et qualis rector est civitatis, tales et inhabitantes in ea.\\
${}^{3}$~Rex insipiens perdet populum suum~:\\ et civitates inhabitabuntur per sensum potentium.\\
${}^{4}$~In manu Dei potestas terr\ae~:\\ et utilem rectorem suscitabit in tempus super illam.\\
${}^{5}$~In manu Dei prosperitas hominis,\\ et super faciem scrib\ae\ imponet honorem suum.\end{verse}\end{flushleft}


\begin{flushleft}\begin{verse}${}^{6}$~Omnis injuri\ae\ proximi ne memineris,\\ et nihil agas in operibus injuri\ae .\\
${}^{7}$~Odibilis coram Deo est et hominibus superbia,\\ et execrabilis omnis iniquitas gentium.\\
${}^{8}$~Regnum a gente in gentem transfertur propter injustitias,\\ et injurias, et contumelias, et diversos dolos.\\
${}^{9}$~Avaro autem nihil est scelestius.\\ Quid superbit terra et cinis~?\\
${}^{10}$~Nihil est iniquius quam amare pecuniam~:\\ hic enim et animam suam venalem habet,\\ quoniam in vita sua projecit intima sua.\\
${}^{11}$~Omnis potentatus brevis vita~;\\ languor prolixior gravat medicum.\\
${}^{12}$~Brevem languorem pr\ae cidit medicus~:\\ sic et rex hodie est, et cras morietur.\\
${}^{13}$~Cum enim morietur homo,\\ h\ae reditabit serpentes, et bestias, et vermes.\\
${}^{14}$~Initium superbi\ae\ hominis apostatare a Deo~:\\
${}^{15}$~quoniam ab eo qui fecit illum recessit cor ejus,\\ quoniam initium omnis peccati est superbia.\\ Qui tenuerit illam adimplebitur maledictis,\\ et subvertet eum in finem.\\
${}^{16}$~Propterea exhonoravit Dominus conventus malorum,\\ et destruxit eos usque in finem.\\
${}^{17}$~Sedes ducum superborum destruxit Deus,\\ et sedere fecit mites pro eis.\\
${}^{18}$~Radices gentium superbarum arefecit Deus,\\ et plantavit humiles ex ipsis gentibus.\\
${}^{19}$~Terras gentium evertit Dominus,\\ et perdidit eas usque ad fundamentum.\\
${}^{20}$~Arefecit ex ipsis, et disperdidit eos,\\ et cessare fecit memoriam eorum a terra.\\
${}^{21}$~Memoria superborum perdidit Deus,\\ et reliquit memoriam humilium sensu.\\
${}^{22}$~Non est creata hominibus superbia,\\ neque iracundia nationi mulierum.\end{verse}\end{flushleft}


\begin{flushleft}\begin{verse}${}^{23}$~Semen hominum honorabitur hoc, quod timet Deum~:\\ semen autem hoc exhonorabitur, quod pr\ae terit mandata Domini.\\
${}^{24}$~In medio fratrum rector illorum in honore~:\\ et qui timent Dominum erunt in oculis illius.\\
${}^{25}$~Gloria divitum, honoratorum, et pauperum,\\ timor Dei est.\\
${}^{26}$~Noli despicere hominem justum pauperem,\\ et noli magnificare virum peccatorem divitem.\\
${}^{27}$~Magnus, et judex, et potens est in honore~:\\ et non est major illo qui timet Deum.\\
${}^{28}$~Servo sensato liberi servient~:\\ et vir prudens et disciplinatus non murmurabit correptus,\\ et inscius non honorabitur.\\
${}^{29}$~Noli extollere te in faciendo opere tuo,\\ et noli cunctari in tempore angusti\ae .\\
${}^{30}$~Melior est qui operatur et abundat in omnibus,\\ quam qui gloriatur et eget pane.\\
${}^{31}$~Fili, in mansuetudine serva animam tuam,\\ et da illi honorem secundum meritum suum.\\
${}^{32}$~Peccantem in animam suam quis justificabit~?\\ et quis honorificabit exhonorantem animam suam~?\\
${}^{33}$~Pauper gloriatur per disciplinam et timorem suum~:\\ et est homo qui honorificatur propter substantiam suam.\\
${}^{34}$~Qui autem gloriatur in paupertate, quanto magis in substantia~!\\ et qui gloriatur in substantia, paupertatem vereatur.\end{verse}\end{flushleft}


\Needspace{2.5\baselineskip}\versal{11}\begin{flushleft}\begin{verse}\vspace{-19pt}\hspace{6pt}Sapientia humiliati exaltabit caput illius,\\\hspace{6pt} et in medio magnatorum consedere illum faciet.\\
${}^{2}$~Non laudes virum in specie sua,\\ neque spernas hominem in visu suo.\\
${}^{3}$~Brevis in volatilibus est apis,\\ et initium dulcoris habet fructus illius.\\
${}^{4}$~In vestitu ne glorieris umquam,\\ nec in die honoris tui extollaris~:\\ quoniam mirabilia opera Altissimi solius,\\ et gloriosa, et absconsa, et invisa opera illius.\\
${}^{5}$~Multi tyranni sederunt in throno~:\\ et insuspicabilis portavit diadema.\\
${}^{6}$~Multi potentes oppressi sunt valide,\\ et gloriosi traditi sunt in manus alterorum.\\
${}^{7}$~Priusquam interroges, ne vituperes quemquam~:\\ et cum interrogaveris, corripe juste.\end{verse}\end{flushleft}


\begin{flushleft}\begin{verse}${}^{8}$~Priusquam audias, ne respondeas verbum~:\\ et in medio sermonum ne adjicias loqui.\\
${}^{9}$~De ea re qu\ae\ te non molestat, ne certeris~:\\ et in judicio peccantium ne consistas.\\
${}^{10}$~Fili, ne in multis sint actus tui~:\\ et si dives fueris, non eris immunis a delicto.\\ Si enim secutus fueris, non apprehendes~:\\ et non effugies, si pr\ae cucurreris.\\
${}^{11}$~Est homo laborans et festinans, et dolens~:\\ impius, et tanto magis non abundabit.\\
${}^{12}$~Est homo marcidus egens recuperatione,\\ plus deficiens virtute, et abundans paupertate~:\\
${}^{13}$~et oculus Dei respexit illum in bono,\\ et erexit eum ab humilitate ipsius, et exaltavit caput ejus~:\\ et mirati sunt in illo multi, et honoraverunt Deum.\end{verse}\end{flushleft}


\begin{flushleft}\begin{verse}${}^{14}$~Bona et mala, vita et mors,\\ paupertas et honestas, a Deo sunt~:\\
${}^{15}$~sapientia, et disciplina, et scientia legis, apud Deum~:\\ dilectio, et vi\ae\ bonorum, apud ipsum.\\
${}^{16}$~Error et tenebr\ae\ peccatoribus concreata sunt~:\\ qui autem exsultant in malis consenescunt in malo.\\
${}^{17}$~Datio Dei permanet justis,\\ et profectus illius successus habebit in \ae ternum.\\
${}^{18}$~Est qui locupletatur parce agendo,\\ et h\ae c est pars mercedis illius.\\
${}^{19}$~In eo quod dicit~: Inveni requiem mihi,\\ et nunc manducabo de bonis meis solus~:\\
${}^{20}$~et nescit quod tempus pr\ae teriet, et mors appropinquet,\\ et relinquat omnia aliis, et morietur.\\
${}^{21}$~Sta in testamento tuo, et in illo colloquere,\\ et in opere mandatorum tuorum veterasce.\\
${}^{22}$~Ne manseris in operibus peccatorum~:\\ confide autem in Deo, et mane in loco tuo.\\
${}^{23}$~Facile est enim in oculis Dei\\ subito honestare pauperem.\\
${}^{24}$~Benedictio Dei in mercedem justi festinat,\\ et in hora veloci processus illius fructificat.\\
${}^{25}$~Ne dicas~: Quid est mihi opus~?\\ et qu\ae\ erunt mihi ex hoc bona~?\\
${}^{26}$~Ne dicas~: Sufficiens mihi sum~:\\ et quid ex hoc pessimabor~?\\
${}^{27}$~In die bonorum ne immemor sis malorum,\\ et in die malorum ne immemor sis bonorum~:\\
${}^{28}$~quoniam facile est coram Deo in die obitus\\ retribuere unicuique secundum vias suas.\\
${}^{29}$~Malitia hor\ae\ oblivionem facit luxuri\ae\ magn\ae ,\\ et in fine hominis denudatio operum illius.\\
${}^{30}$~Ante mortem ne laudes hominem quemquam~:\\ quoniam in filiis suis agnoscitur vir.\end{verse}\end{flushleft}


\begin{flushleft}\begin{verse}${}^{31}$~Non omnem hominem inducas in domum tuam~:\\ mult\ae\ enim sunt insidi\ae\ dolosi.\\
${}^{32}$~Sicut enim eructant pr\ae cordia fœtentium,\\ et sicut perdix inducitur in caveam, et ut caprea in laqueum~:\\ sic et cor superborum,\\ et sicut prospector videns casum proximi sui.\\
${}^{33}$~Bona enim in mala convertens insidiatur,\\ et in electis imponet maculam.\\
${}^{34}$~A scintilla una augetur ignis,\\ et ab uno doloso augetur sanguis~:\\ homo vero peccator sanguini insidiatur.\\
${}^{35}$~Attende tibi a pestifero, fabricat enim mala,\\ ne inducat super te subsannationem in perpetuum.\\
${}^{36}$~Admitte ad te alienigenam~:\\ et subvertet te in turbine,\\ et abalienabit te a tuis propriis.\end{verse}\end{flushleft}


\Needspace{2.5\baselineskip}\versal{12}\begin{flushleft}\begin{verse}\vspace{-19pt}\hspace{6pt}Si benefeceris, scito cui feceris,\\\hspace{6pt} et erit gratia in bonis tuis multa.\\
${}^{2}$~Benefac justo, et invenies retributionem magnam~:\\ et si non ab ipso, certe a Domino.\\
${}^{3}$~Non est enim ei bene qui assiduus est in malis,\\ et eleemosynas non danti~:\\ quoniam et Altissimus odio habet peccatores,\\ et misertus est pœnitentibus.\\
${}^{4}$~Da misericordi, et ne suscipias peccatorem~:\\ et impiis et peccatoribus reddet vindictam,\\ custodiens eos in diem vindict\ae .\\
${}^{5}$~Da bono, et non receperis peccatorem.\\
${}^{6}$~Benefac humili, et non dederis impio~:\\ prohibe panes illi dari, ne in ipsis potentior te sit~:\\
${}^{7}$~nam duplicia mala invenies in omnibus bonis qu\ae cumque feceris illi,\\ quoniam et Altissimus odio habet peccatores,\\ et impiis reddet vindictam.\end{verse}\end{flushleft}


\begin{flushleft}\begin{verse}${}^{8}$~Non agnoscetur in bonis amicus,\\ et non abscondetur in malis inimicus.\\
${}^{9}$~In bonis viri, inimici illius in tristitia~:\\ et in malitia illius, amicus agnitus est.\\
${}^{10}$~Non credas inimico tuo in \ae ternum~:\\ sicut enim \ae ramentum \ae ruginat nequitia illius~:\\
${}^{11}$~et si humiliatus vadat curvus,\\ adjice animum tuum, et custodi te ab illo.\\
${}^{12}$~Non statuas illum penes te,\\ nec sedeat ad dexteram tuam,\\ ne forte conversus in locum tuum, inquirat cathedram tuam,\\ et in novissimo agnosces verba mea,\\ et in sermonibus meis stimuleris.\\
${}^{13}$~Quis miserebitur incantatori a serpente percusso,\\ et omnibus qui appropiant bestiis~?\\ et sic qui comitatur cum viro iniquo,\\ et obvolutus est in peccatis ejus.\\
${}^{14}$~Una hora tecum permanebit~:\\ si autem declinaveris, non supportabit.\\
${}^{15}$~In labiis suis indulcat inimicus,\\ et in corde suo insidiatur ut subvertat te in foveam.\\
${}^{16}$~In oculis suis lacrimatur inimicus,\\ et si invenerit tempus, non satiabitur sanguine.\\
${}^{17}$~Et si incurrerint tibi mala,\\ invenies eum illic priorem.\\
${}^{18}$~In oculis suis lacrimatur inimicus,\\ et quasi adjuvans suffodiet plantas tuas.\\
${}^{19}$~Caput suum movebit, et plaudet manu,\\ et multa susurrans commutabit vultum suum.\end{verse}\end{flushleft}


\Needspace{2.5\baselineskip}\versal{13}\begin{flushleft}\begin{verse}\vspace{-19pt}\hspace{6pt}Qui tetigerit picem inquinabitur ab ea~:\\\hspace{6pt} et qui communicaverit superbo induet superbiam.\\
${}^{2}$~Pondus super se tollat qui honestiori se communicat,\\ et ditiori te ne socius fueris.\\
${}^{3}$~Quid communicabit cacabus ad ollam~?\\ quando enim se colliserint, confringetur.\\
${}^{4}$~Dives injuste egit, et fremet~:\\ pauper autem l\ae sus tacebit.\\
${}^{5}$~Si largitus fueris, assumet te~:\\ et si non habueris, derelinquet te.\\
${}^{6}$~Si habes, convivet tecum, et evacuabit te~:\\ et ipse non dolebit super te.\\
${}^{7}$~Si necessarius illi fueris, supplantabit te,\\ et subridens spem dabit, narrans tibi bona,\\ et dicet~: Quid opus est tibi~?\\
${}^{8}$~Et confundet te in cibis suis,\\ donec te exinaniat bis et ter~:\\ et in novissimo deridebit te,\\ et postea videns derelinquet te,\\ et caput suum movebit ad te.\\
${}^{9}$~Humiliare Deo, et exspecta manus ejus.\\
${}^{10}$~Attende ne seductus in stultitiam humilieris.\\
${}^{11}$~Noli esse humilis in sapientia tua,\\ ne humiliatus in stultitiam seducaris.\\
${}^{12}$~Advocatus a potentiore, discede~:\\ ex hoc enim magis te advocabit.\\
${}^{13}$~Ne improbus sis, ne impingaris~:\\ et ne longe sis ab eo, ne eas in oblivionem.\\
${}^{14}$~Ne retineas ex \ae quo loqui cum illo,\\ nec credas multis verbis illius~:\\ ex multa enim loquela tentabit te,\\ et subridens interrogabit te de absconditis tuis.\\
${}^{15}$~Immitis animus illius conservabit verba tua~:\\ et non parcet de malitia, et de vinculis.\\
${}^{16}$~Cave tibi, et attende diligenter auditui tuo,\\ quoniam cum subversione tua ambulas~:\\
${}^{17}$~audiens vero illa,\\ quasi in somnis vide, et vigilabis.\\
${}^{18}$~Omni vita tua dilige Deum,\\ et invoca illum in salute tua.\\
${}^{19}$~Omne animal diligit simile sibi,\\ sic et omnis homo proximum sibi.\\
${}^{20}$~Omnis caro ad similem sibi conjungetur,\\ et omnis homo simili sui sociabitur.\\
${}^{21}$~Si communicabit lupus agno aliquando,\\ sic peccator justo.\\
${}^{22}$~Qu\ae\ communicatio sancto homini ad canem~?\\ aut qu\ae\ pars diviti ad pauperem~?\\
${}^{23}$~Venatio leonis onager in eremo~:\\ sic et pascua divitum sunt pauperes.\\
${}^{24}$~Et sicut abominatio est superbo humilitas,\\ sic et execratio divitis pauper.\\
${}^{25}$~Dives commotus confirmatur ab amicis suis~:\\ humilis autem cum ceciderit, expelletur et a notis.\\
${}^{26}$~Diviti decepto multi recuperatores~:\\ locutus est superbia, et justificaverunt illum.\\
${}^{27}$~Humilis deceptus est, insuper et arguitur~:\\ locutus est sensate, et non est datus ei locus.\\
${}^{28}$~Dives locutus est, et omnes tacuerunt,\\ et verbum illius usque ad nubes perducent.\\
${}^{29}$~Pauper locutus est, et dicunt~: Quis est hic~?\\ et si offenderit, subvertent illum.\end{verse}\end{flushleft}


\begin{flushleft}\begin{verse}${}^{30}$~Bona est substantia cui non est peccatum in conscientia~:\\ et nequissima paupertas in ore impii.\\
${}^{31}$~Cor hominis immutat faciem illius,\\ sive in bona, sive in mala.\\
${}^{32}$~Vestigium cordis boni et faciem bonam\\ difficile invenies, et cum labore.\end{verse}\end{flushleft}


\Needspace{2.5\baselineskip}\versal{14}\begin{flushleft}\begin{verse}\vspace{-19pt}\hspace{6pt}Beatus vir qui non est lapsus verbo ex ore suo,\\\hspace{6pt} et non est stimulatus in tristitia delicti.\\
${}^{2}$~Felix qui non habuit animi sui tristitiam,\\ et non excidit a spe sua.\\
${}^{3}$~Viro cupido et tenaci sine ratione est substantia~:\\ et homini livido ad quid aurum~?\\
${}^{4}$~Qui acervat ex animo suo injuste, aliis congregat,\\ et in bonis illius alius luxuriabitur.\\
${}^{5}$~Qui sibi nequam est, cui alii bonus erit~?\\ et non jucundabitur in bonis suis.\\
${}^{6}$~Qui sibi invidet, nihil est illo nequius~:\\ et h\ae c redditio est maliti\ae\ illius.\\
${}^{7}$~Et si bene fecerit, ignoranter et non volens facit~:\\ et in novissimo manifestat malitiam suam.\\
${}^{8}$~Nequam est oculus lividi~:\\ et avertens faciem suam, et despiciens animam suam.\\
${}^{9}$~Insatiabilis oculus cupidi in parte iniquitatis~:\\ non satiabitur donec consumat arefaciens animam suam.\\
${}^{10}$~Oculus malus ad mala, et non saturabitur pane,\\ sed indigens et in tristitia erit super mensam suam.\\
${}^{11}$~Fili, si habes, benefac tecum,\\ et Deo dignas oblationes offer.\\
${}^{12}$~Memor esto quoniam mors non tardat,\\ et testamentum inferorum, quia demonstratum est tibi~:\\ testamentum enim hujus mundi morte morietur.\\
${}^{13}$~Ante mortem benefac amico tuo,\\ et secundum vires tuas exporrigens da pauperi.\\
${}^{14}$~Non defrauderis a die bono,\\ et particula boni doni non te pr\ae tereat.\\
${}^{15}$~Nonne aliis relinques dolores et labores tuos\\ in divisione sortis~?\\
${}^{16}$~Da et accipe,\\ et justifica animam tuam.\\
${}^{17}$~Ante obitum tuum operare justitiam,\\ quoniam non est apud inferos invenire cibum.\\
${}^{18}$~Omnis caro sicut fœnum veterascet,\\ et sicut folium fructificans in arbore viridi.\\
${}^{19}$~Alia generantur, et alia dejiciuntur~:\\ sic generatio carnis et sanguinis, alia finitur, et alia nascitur.\\
${}^{20}$~Omne opus corruptibile in fine deficiet,\\ et qui illud operatur ibit cum illo.\\
${}^{21}$~Et omne opus electum justificabitur,\\ et qui operatur illud honorabitur in illo.\end{verse}\end{flushleft}


\begin{flushleft}\begin{verse}${}^{22}$~Beatus vir qui in sapientia morabitur,\\ et qui in justitia sua meditabitur,\\ et in sensu cogitabit circumspectionem Dei~:\\
${}^{23}$~qui excogitat vias illius in corde suo,\\ et in absconditis suis intelligens,\\ vadens post illam quasi investigator,\\ et in viis illius consistens~:\\
${}^{24}$~qui respicit per fenestras illius,\\ et in januis illius audiens~:\\
${}^{25}$~qui requiescit juxta domum illius,\\ et in parietibus illius figens palum,\\ statuet casulam suam ad manus illius,\\ et requiescent in casula illius bona per \ae vum.\\
${}^{26}$~Statuet filios suos sub tegmine illius,\\ et sub ramis ejus morabitur.\\
${}^{27}$~Protegetur sub tegmine illius a fervore,\\ et in gloria ejus requiescet.\end{verse}\end{flushleft}


\Needspace{2.5\baselineskip}\versal{15}\begin{flushleft}\begin{verse}\vspace{-19pt}\hspace{6pt}Qui timet Deum faciet bona,\\\hspace{6pt} et qui continens est justiti\ae\ apprehendet illam~:\\
${}^{2}$~et obviabit illi quasi mater honorificata,\\ et quasi mulier a virginitate suscipiet illum.\\
${}^{3}$~Cibabit illum pane vit\ae\ et intellectus,\\ et aqua sapienti\ae\ salutaris potabit illum~:\\ et firmabitur in illo, et non flectetur~:\\
${}^{4}$~et continebit illum, et non confundetur~:\\ et exaltabit illum apud proximos suos,\\
${}^{5}$~et in medio ecclesi\ae\ aperiet os ejus,\\ et adimplebit illum spiritu sapienti\ae\ et intellectus,\\ et stola glori\ae\ vestiet illum.\\
${}^{6}$~Jucunditatem et exsultationem thesaurizabit super illum,\\ et nomine \ae terno h\ae reditabit illum.\\
${}^{7}$~Homines stulti non apprehendent illam,\\ et homines sensati obviabunt illi.\\ Homines stulti non videbunt eam~:\\ longe enim abest a superbia et dolo.\\
${}^{8}$~Viri mendaces non erunt illius memores~:\\ et viri veraces invenientur in illa,\\ et successum habebunt usque ad inspectionem Dei.\\
${}^{9}$~Non est speciosa laus in ore peccatoris,\\
${}^{10}$~quoniam a Deo profecta est sapientia.\\ Sapienti\ae\ enim Dei astabit laus,\\ et in ore fideli abundabit,\\ et Dominator dabit eam illi.\end{verse}\end{flushleft}


\begin{flushleft}\begin{verse}${}^{11}$~Non dixeris~: Per Deum abest~:\\ qu\ae\ enim odit ne feceris.\\
${}^{12}$~Non dicas~: Ille me implanavit~:\\ non enim necessarii sunt ei homines impii.\\
${}^{13}$~Omne execramentum erroris odit Dominus,\\ et non erit amabile timentibus eum.\\
${}^{14}$~Deus ab initio constituit hominem,\\ et reliquit illum in manu consilii sui~:\\
${}^{15}$~adjecit mandata et pr\ae cepta sua.\\
${}^{16}$~Si volueris mandata servare, conservabunt te,\\ et in perpetuum fidem placitam facere.\\
${}^{17}$~Apposuit tibi aquam et ignem,\\ ad quod volueris porrige manum tuam.\\
${}^{18}$~Ante hominem vita et mors, bonum et malum~:\\ quod placuerit ei dabitur illi~:\\
${}^{19}$~quoniam multa sapientia Dei, et fortis in potentia,\\ videns omnes sine intermissione.\\
${}^{20}$~Oculi Domini ad timentes eum,\\ et ipse agnoscit omnem operam hominis.\\
${}^{21}$~Nemini mandavit impie agere,\\ et nemini dedit spatium peccandi~:\\
${}^{22}$~non enim concupiscit multitudinem\\ filiorum infidelium et inutilium.\end{verse}\end{flushleft}


\Needspace{2.5\baselineskip}\versal{16}\begin{flushleft}\begin{verse}\vspace{-19pt}\hspace{6pt}Ne jucunderis in filiis impiis, si multiplicentur~:\\\hspace{6pt} nec oblecteris super ipsos, si non est timor Dei in illis.\\
${}^{2}$~Non credas vit\ae\ illorum,\\ et ne respexeris in labores eorum.\\
${}^{3}$~Melior est enim unus timens Deum,\\ quam mille filii impii~:\\
${}^{4}$~et utile est mori sine filiis,\\ quam relinquere filios impios.\\
${}^{5}$~Ab uno sensato inhabitabitur patria~:\\ tribus impiorum deseretur.\\
${}^{6}$~Multa talia vidit oculis meus,\\ et fortiora horum audivit auris mea.\\
${}^{7}$~In synagoga peccantium exardebit ignis,\\ et in gente incredibili exardescet ira.\\
${}^{8}$~Non exoraverunt pro peccatis suis antiqui gigantes,\\ qui destructi sunt confidentes su\ae\ virtuti.\\
${}^{9}$~Et non pepercit peregrinationi Lot,\\ et execratus est eos pr\ae\ superbia verbi illorum.\\
${}^{10}$~Non misertus est illis, gentem totam perdens,\\ et extollentem se in peccatis suis.\\
${}^{11}$~Et sicut sexcenta millia peditum,\\ qui congregati sunt in duritia cordis sui~:\\ et si unus fuisset cervicatus,\\ mirum si fuisset immunis.\\
${}^{12}$~Misericordia enim et ira est cum illo~:\\ potens exoratio, et effundens iram.\\
${}^{13}$~Secundum misericordiam suam,\\ sic correptio illius homines secundum opera sua judicat.\\
${}^{14}$~Non effugiet in rapina peccator,\\ et non retardabit sufferentia misericordiam facientis.\\
${}^{15}$~Omnis misericordia faciet locum unicuique,\\ secundum meritum operum suorum,\\ et secundum intellectum peregrinationis ipsius.\end{verse}\end{flushleft}


\begin{flushleft}\begin{verse}${}^{16}$~Non dicas~: A Deo abscondar~:\\ et ex summo, quis mei memorabitur~?\\
${}^{17}$~in populo magno non agnoscar~:\\ qu\ae\ est enim anima mea in tam immensa creatura~?\\
${}^{18}$~Ecce c\ae lum et c\ae li c\ae lorum,\\ abyssus, et universa terra, et qu\ae\ in eis sunt,\\ in conspectu illius commovebuntur.\\
${}^{19}$~Montes simul, et colles, et fundamenta terr\ae ,\\ cum conspexerit illa Deus, tremore concutientur.\\
${}^{20}$~Et in omnibus his insensatum est cor,\\ et omne cor intelligitur ab illo.\\
${}^{21}$~Et vias illius quis intelligit,\\ et procellam quam nec oculus videbit hominis~?\\
${}^{22}$~Nam plurima illius opera sunt in absconsis~:\\ sed opera justiti\ae\ ejus quis enuntiabit, aut quis sustinebit~?\\ longe enim est testamentum a quibusdam,\\ et interrogatio omnium in consummatione est.\\
${}^{23}$~Qui minoratur corde cogitat inania,\\ et vir imprudens et errans cogitat stulta.\end{verse}\end{flushleft}


\begin{flushleft}\begin{verse}${}^{24}$~Audi me, fili, et disce disciplinam sensus,\\ et in verbis meis attende in corde tuo~:\\
${}^{25}$~et dicam in \ae quitate disciplinam,\\ et scrutabor enarrare sapientiam~:\\ et in verbis meis attende in corde tuo,\\ et dico in \ae quitate spiritus virtutes\\ quas posuit Deus in opera sua ab initio,\\ et in veritate enuntio scientiam ejus.\\
${}^{26}$~In judicio Dei opera ejus ab initio,\\ et ab institutione ipsorum distinxit partes illorum,\\ et initia eorum in gentibus suis.\\
${}^{27}$~Ornavit in \ae ternum opera illorum~:\\ nec esurierunt, nec laboraverunt,\\ et non destiterunt ab operibus suis.\\
${}^{28}$~Unusquisque proximum sibi non angustiabit in \ae ternum~:\\
${}^{29}$~non sis incredibilis verbo illius.\\
${}^{30}$~Post h\ae c Deus in terram respexit,\\ et implevit illam bonis suis~:\\
${}^{31}$~anima omnis vitalis denuntiavit ante faciem ipsius,\\ et in ipsam iterum reversio illorum.\end{verse}\end{flushleft}


\Needspace{2.5\baselineskip}\versal{17}\begin{flushleft}\begin{verse}\vspace{-19pt}\hspace{6pt}Deus creavit de terra hominem,\\\hspace{6pt} et secundum imaginem suam fecit illum~:\\
${}^{2}$~et iterum convertit illum in ipsam,\\ et secundum se vestivit illum virtute.\\
${}^{3}$~Numerum dierum et tempus dedit illi,\\ et dedit illi potestatem eorum qu\ae\ sunt super terram.\\
${}^{4}$~Posuit timorem illius super omnem carnem,\\ et dominatus est bestiarum et volatilium.\\
${}^{5}$~Creavit ex ipso adjutorium simile sibi~:\\ consilium, et linguam, et oculos, et aures,\\ et cor dedit illis excogitandi,\\ et disciplina intellectus replevit illos.\\
${}^{6}$~Creavit illis scientiam spiritus,\\ sensu implevit cor illorum,\\ et mala et bona ostendit illis.\\
${}^{7}$~Posuit oculum suum super corda illorum,\\ ostendere illis magnalia operum suorum~:\\
${}^{8}$~ut nomen sanctificationis collaudent,\\ et gloriari in mirabilibus illius~;\\ ut magnalia enarrent operum ejus.\\
${}^{9}$~Addidit illis disciplinam,\\ et legem vit\ae\ h\ae reditavit illos.\\
${}^{10}$~Testamentum \ae ternum constituit cum illis,\\ et justitiam et judicia sua ostendit illis.\\
${}^{11}$~Et magnalia honoris ejus vidit oculus illorum,\\ et honorem vocis audierunt aures illorum.\\ Et dixit illis~: Attendite ab omni iniquo.\\
${}^{12}$~Et mandavit illis unicuique de proximo suo.\\
${}^{13}$~Vi\ae\ illorum coram ipso sunt semper~:\\ non sunt abscons\ae\ ab oculis ipsius.\\
${}^{14}$~In unamquamque gentem pr\ae posuit rectorem~:\\
${}^{15}$~et pars Dei Isra\"el facta est manifesta.\\
${}^{16}$~Et omnia opera illorum velut sol in conspectu Dei~:\\ et oculi ejus sine intermissione inspicientes in viis eorum.\\
${}^{17}$~Non sunt absconsa testamenta per iniquitatem illorum,\\ et omnes iniquitates eorum in conspectu Dei.\\
${}^{18}$~Eleemosyna viri quasi signaculum cum ipso,\\ et gratiam hominis quasi pupillam conservabit.\\
${}^{19}$~Et postea resurget,\\ et retribuet illis retributionem, unicuique in caput ipsorum,\\ et convertet in interiores partes terr\ae .\\
${}^{20}$~Pœnitentibus autem dedit viam justiti\ae ,\\ et confirmavit deficientes sustinere,\\ et destinavit illis sortem veritatis.\end{verse}\end{flushleft}


\begin{flushleft}\begin{verse}${}^{21}$~Convertere ad Dominum, et relinque peccata tua~:\\
${}^{22}$~precare ante faciem Domini, et minue offendicula.\\
${}^{23}$~Revertere ad Dominum, et avertere ab injustitia tua,\\ et nimis odito execrationem~:\\
${}^{24}$~et cognosce justitias et judicia Dei,\\ et sta in sorte propositionis, et orationis altissimi Dei.\\
${}^{25}$~In partes vade s\ae culi sancti,\\ cum vivis et dantibus confessionem Deo.\\
${}^{26}$~Non demoreris in errore impiorum~:\\ ante mortem confitere~:\\ a mortuo, quasi nihil, perit confessio.\\
${}^{27}$~Confiteberis vivens,\\ vivus et sanus confiteberis~:\\ et laudabis Deum,\\ et gloriaberis in miserationibus illius.\\
${}^{28}$~Quam magna misericordia Domini,\\ et propitiatio illius convertentibus ad se~!\\
${}^{29}$~Nec enim omnia possunt esse in hominibus,\\ quoniam non est immortalis filius hominis,\\ et in vanitate maliti\ae\ placuerunt.\\
${}^{30}$~Quid lucidius sole~?\\ et hic deficiet~;\\ aut quid nequius quam quod excogitavit caro et sanguis~?\\ et hoc arguetur.\\
${}^{31}$~Virtutem altitudinis c\ae li ipse conspicit~:\\ et omnes homines terra et cinis.\end{verse}\end{flushleft}


\Needspace{2.5\baselineskip}\versal{18}\begin{flushleft}\begin{verse}\vspace{-19pt}\hspace{6pt}Qui vivet in \ae ternum creavit omnia simul.\\\hspace{6pt} Deus solus justificabitur,\\ et manet invictus rex in \ae ternum.\\
${}^{2}$~Quis sufficit enarrare opera illius~?\\
${}^{3}$~quis enim investigabit magnalia ejus~?\\
${}^{4}$~virtutem autem magnitudinis ejus quis enuntiabit~?\\ aut quis adjiciet enarrare misericordiam ejus~?\\
${}^{5}$~Non est minuere neque adjicere,\\ nec est invenire magnalia Dei.\\
${}^{6}$~Cum consummaverit homo, tunc incipiet~:\\ et cum quieverit, aporiabitur.\\
${}^{7}$~Quid est homo~? et qu\ae\ est gratia illius~?\\ et quid bonum aut quid nequam illius~?\\
${}^{8}$~Numerus dierum hominum, ut multum centum anni,\\ quasi gutta aqu\ae\ maris deputati sunt~:\\ et sicut calculus aren\ae , sic exigui anni in die \ae vi.\\
${}^{9}$~Propter hoc patiens est Deus in illis,\\ et effundit super eos misericordiam suam.\\
${}^{10}$~Vidit pr\ae sumptionem cordis eorum, quoniam mala est~:\\ et cognovit subversionem illorum, quoniam nequam est.\\
${}^{11}$~Ideo adimplevit propitiationem suam in illis,\\ et ostendit eis viam \ae quitatis.\\
${}^{12}$~Miseratio hominis circa proximum suum~:\\ misericordia autem Dei super omnem carnem.\\
${}^{13}$~Qui misericordiam habet, docet et erudit\\ quasi pastor gregem suum.\\
${}^{14}$~Miseretur excipientis doctrinam miserationis,\\ et qui festinat in judiciis ejus.\end{verse}\end{flushleft}


\begin{flushleft}\begin{verse}${}^{15}$~Fili, in bonis non des querelam,\\ et in omni dato non des tristitiam verbi mali.\\
${}^{16}$~Nonne ardorem refrigerabit ros~?\\ sic et verbum melius quam datum.\\
${}^{17}$~Nonne ecce verbum super datum bonum~?\\ sed utraque cum homine justificato.\\
${}^{18}$~Stultus acriter improperabit~:\\ et datus indisciplinati tabescere facit oculos.\\
${}^{19}$~Ante judicium para justitiam tibi,\\ et antequam loquaris, disce.\\
${}^{20}$~Ante languorem adhibe medicinam~:\\ et ante judicium interroga teipsum,\\ et in conspectu Dei invenies propitiationem.\\
${}^{21}$~Ante languorem humilia te,\\ et in tempore infirmitatis ostende conversationem tuam.\\
${}^{22}$~Non impediaris orare semper,\\ et ne verearis usque ad mortem justificari,\\ quoniam merces Dei manet in \ae ternum.\\
${}^{23}$~Ante orationem pr\ae para animam tuam,\\ et noli esse quasi homo qui tentat Deum.\\
${}^{24}$~Memento ir\ae\ in die consummationis,\\ et tempus retributionis in conversatione faciei.\\
${}^{25}$~Memento paupertatis in tempore abundanti\ae ,\\ et necessitatum paupertatis in die divitiarum.\\
${}^{26}$~A mane usque ad vesperam immutabitur tempus,\\ et h\ae c omnia citata in oculis Dei.\\
${}^{27}$~Homo sapiens in omnibus metuet,\\ et in diebus delictorum attendet ab inertia.\\
${}^{28}$~Omnis astutus agnoscit sapientiam,\\ et invenienti eam dabit confessionem.\\
${}^{29}$~Sensati in verbis et ipsi sapienter egerunt,\\ et intellexerunt veritatem et justitiam,\\ et impluerunt proverbia et judicia.\end{verse}\end{flushleft}


\begin{flushleft}\begin{verse}${}^{30}$~Post concupiscentias tuas non eas,\\ et a voluntate tua avertere.\\
${}^{31}$~Si pr\ae stes anim\ae\ tu\ae\ concupiscentias ejus,\\ faciat te in gaudium inimicis tuis.\\
${}^{32}$~Ne oblecteris in turbis nec in modicis~:\\ assidua enim est commissio illorum.\\
${}^{33}$~Ne fueris mediocris in contentione ex fœnore,\\ et est tibi nihil in sacculo~:\\ eris enim invidus vit\ae\ tu\ae .\end{verse}\end{flushleft}


\Needspace{2.5\baselineskip}\versal{19}\begin{flushleft}\begin{verse}\vspace{-19pt}\hspace{6pt}Operarius ebriosus non locupletabitur~:\\\hspace{6pt} et qui spernit modica paulatim decidet.\\
${}^{2}$~Vinum et mulieres apostatare faciunt sapientes,\\ et arguent sensatos.\\
${}^{3}$~Et qui se jungit fornicariis erit nequam~:\\ putredo et vermes h\ae reditabunt illum~:\\ et extolletur in exemplum majus,\\ et tolletur de numero anima ejus.\end{verse}\end{flushleft}


\begin{flushleft}\begin{verse}${}^{4}$~Qui credit cito levis corde est, et minorabitur~:\\ et qui delinquit in animam suam, insuper habebitur.\\
${}^{5}$~Qui gaudet iniquitate, denotabitur~:\\ et qui odit correptionem, minuetur vita~:\\ et qui odit loquacitatem, extinguit malitiam.\\
${}^{6}$~Qui peccat in animam suam, pœnitebit~:\\ et qui jucundatur in malitia, denotabitur.\\
${}^{7}$~Ne iteres verbum nequam et durum,\\ et non minoraberis.\\
${}^{8}$~Amico et inimico noli narrare sensum tuum~:\\ et si est tibi delictum, noli denudare~:\\
${}^{9}$~audiet enim te, et custodiet te,\\ et quasi defendens peccatum, odiet te,\\ et sic aderit tibi semper.\\
${}^{10}$~Audisti verbum adversus proximum tuum~?\\ commoriatur in te, fidens quoniam non te dirumpet.\\
${}^{11}$~A facie verbi parturit fatuus,\\ tamquam gemitus partus infantis.\\
${}^{12}$~Sagitta infixa femori carnis,\\ sic verbum in corde stulti.\\
${}^{13}$~Corripe amicum,\\ ne forte non intellexerit, et dicat~: Non feci~:\\ aut, si fecerit, ne iterum addat facere.\\
${}^{14}$~Corripe proximum, ne forte non dixerit~:\\ et si dixerit, ne forte iteret.\\
${}^{15}$~Corripe amicum, s\ae pe enim fit commissio~:\\
${}^{16}$~et non omni verbo credas.\\ Est qui labitur lingua, sed non ex animo~:\\
${}^{17}$~quis est enim qui non deliquerit in lingua sua~?\end{verse}\end{flushleft}

 \begin{flushleft}\begin{verse}Corripe proximum antequam commineris,\\
${}^{18}$~et da locum timori Altissimi~:\\ quia omnis sapientia timor Dei, et in illa timere Deum,\\ et in omni sapientia dispositio legis.\\
${}^{19}$~Et non est sapientia nequiti\ae\ disciplina,\\ et non est cogitatus peccatorum prudentia.\\
${}^{20}$~Est nequitia, et in ipsa execratio,\\ et est insipiens qui minuitur sapientia.\\
${}^{21}$~Melior est homo qui minuitur sapientia,\\ et deficiens sensu, in timore,\\ quam qui abundat sensu,\\ et transgreditur legem Altissimi.\\
${}^{22}$~Est solertia certa, et ipsa iniqua~:\\
${}^{23}$~et est qui emittit verbum certum enarrans veritatem.\\ Est qui nequiter humiliat se,\\ et interiora ejus plena sunt dolo~:\\
${}^{24}$~et est qui se nimium submittit a multa humilitate~:\\ et est qui inclinat faciem suam,\\ et fingit se non videre quod ignoratum est~:\\
${}^{25}$~et si ab imbecillitate virium vetetur peccare,\\ si invenerit tempus malefaciendi, malefaciet.\\
${}^{26}$~Ex visu cognoscitur vir,\\ et ab occursu faciei cognoscitur sensatus.\\
${}^{27}$~Amictus corporis, et risus dentium,\\ et ingressus hominis, enuntiant de illo.\\
${}^{28}$~Est correptio mendax in ira contumeliosi,\\ et est judicium quod non probatur esse bonum~:\\ et est tacens, et ipse est prudens.\end{verse}\end{flushleft}


\Needspace{2.5\baselineskip}\versal{20}\begin{flushleft}\begin{verse}\vspace{-19pt}\hspace{6pt}Quam bonum est arguere, quam irasci,\\\hspace{6pt} et confitentem in oratione non prohibere~!\\
${}^{2}$~Concupiscentia spadonis devirginabit juvenculam~:\\
${}^{3}$~sic qui facit per vim judicium iniquum.\\
${}^{4}$~Quam bonum est correptum manifestare pœnitentiam~!\\ sic enim effugies voluntarium peccatum.\\
${}^{5}$~Est tacens qui invenitur sapiens~:\\ et est odibilis qui procax est ad loquendum.\\
${}^{6}$~Est tacens non habens sensum loquel\ae~:\\ et est tacens sciens tempus aptum.\\
${}^{7}$~Homo sapiens tacebit usque ad tempus~:\\ lascivus autem et imprudens non servabunt tempus.\\
${}^{8}$~Qui multis utitur verbis l\ae det animam suam~:\\ et qui potestatem sibi sumit injuste, odietur.\\
${}^{9}$~Est processio in malis viro indisciplinato,\\ et est inventio in detrimentum.\\
${}^{10}$~Est datum quod non est utile,\\ et est datum cujus retributio duplex.\\
${}^{11}$~Est propter gloriam minoratio,\\ et est qui ab humilitate levabit caput.\\
${}^{12}$~Est qui multa redimat modico pretio,\\ et restituens ea in septuplum.\\
${}^{13}$~Sapiens in verbis seipsum amabilem facit~:\\ grati\ae\ autem fatuorum effundentur.\\
${}^{14}$~Datus insipientis non erit utilis tibi~:\\ oculi enim illius septemplices sunt.\\
${}^{15}$~Exigua dabit, et multa improperabit~:\\ et apertio oris illius inflammatio est.\\
${}^{16}$~Hodie fœneratur quis, et cras expetit~:\\ odibilis est homo hujusmodi.\\
${}^{17}$~Fatuo non erit amicus,\\ et non erit gratia bonis illius~:\\
${}^{18}$~qui enim edunt panem illius, fals\ae\ lingu\ae\ sunt.\\ Quoties et quanti irridebunt eum~!\\
${}^{19}$~neque enim quod habendum erat directo sensu distribuit~;\\ similiter et quod non erat habendum.\\
${}^{20}$~Lapsus fals\ae\ lingu\ae\ quasi qui in pavimento cadens~:\\ sic casus malorum festinanter veniet.\\
${}^{21}$~Homo acharis quasi fabula vana,\\ in ore indisciplinatorum assidua erit.\\
${}^{22}$~Ex ore fatui reprobabitur parabola~:\\ non enim dicit illam in tempore suo.\end{verse}\end{flushleft}


\begin{flushleft}\begin{verse}${}^{23}$~Est qui vetatur peccare pr\ae\ inopia,\\ et in requie sua stimulabitur.\\
${}^{24}$~Est qui perdet animam suam pr\ae\ confusione,\\ et ab imprudenti persona perdet eam~:\\ person\ae\ autem acceptione perdet se.\\
${}^{25}$~Est qui pr\ae\ confusione promittit amico,\\ et lucratus est eum inimicum gratis.\\
${}^{26}$~Opprobrium nequam in homine mendacium~:\\ et in ore indisciplinatorum assidue erit.\\
${}^{27}$~Potior fur quam assiduitas viri mendacis~:\\ perditionem autem ambo h\ae reditabunt.\\
${}^{28}$~Mores hominum mendacium sine honore,\\ et confusio illorum cum ipsis sine intermissione.\\
${}^{29}$~Sapiens in verbis producet seipsum,\\ et homo prudens placebit magnatis.\\
${}^{30}$~Qui operatur terram suam inaltabit acervum frugum,\\ et qui operatur justitiam, ipse exaltabitur~:\\ qui vero placet magnatis effugiet iniquitatem.\\
${}^{31}$~Xenia et dona exc\ae cant oculos judicum,\\ et quasi mutus, in ore avertit correptiones eorum.\\
${}^{32}$~Sapientia absconsa, et thesaurus invisus,\\ qu\ae\ utilitas in utrisque~?\\
${}^{33}$~Melior est qui celat insipientiam suam,\\ quam homo qui abscondit sapientiam suam.\end{verse}\end{flushleft}


\Needspace{2.5\baselineskip}\versal{21}\begin{flushleft}\begin{verse}\vspace{-19pt}\hspace{6pt}Fili, peccasti, non adjicias iterum~:\\\hspace{6pt} sed et de pristinis deprecare, ut tibi dimittantur.\\
${}^{2}$~Quasi a facie colubri fuge peccata~:\\ et si accesseris ad illa, suscipient te.\\
${}^{3}$~Dentes leonis dentes ejus,\\ interficientes animas hominum.\\
${}^{4}$~Quasi rhomph\ae a bis acuta omnis iniquitas~:\\ plag\ae\ illius non est sanitas.\\
${}^{5}$~Objurgatio et injuri\ae\ annullabunt substantiam,\\ et domus qu\ae\ nimis locuples est annullabitur superbia~:\\ sic substantia superbi eradicabitur.\\
${}^{6}$~Deprecatio pauperis ex ore usque ad aures ejus perveniet,\\ et judicium festinato adveniet illi.\\
${}^{7}$~Qui odit correptionem vestigium est peccatoris,\\ et qui timet Deum convertetur ad cor suum.\\
${}^{8}$~Notus a longe potens lingua audaci,\\ et sensatus scit labi se ab ipso.\\
${}^{9}$~Qui \ae dificat domum suam impendiis alienis,\\ quasi qui colligit lapides suos in hieme.\\
${}^{10}$~Stupa collecta synagoga peccantium,\\ et consummatio illorum flamma ignis.\\
${}^{11}$~Via peccatorum complanata lapidibus~:\\ et in fine illorum inferi, et tenebr\ae , et pœn\ae .\end{verse}\end{flushleft}


\begin{flushleft}\begin{verse}${}^{12}$~Qui custodit justitiam, continebit sensum ejus.\\
${}^{13}$~Consummatio timoris Dei, sapientia et sensus.\\
${}^{14}$~Non erudietur\\ qui non est sapiens in bono.\\
${}^{15}$~Est autem sapientia qu\ae\ abundat in malo,\\ et non est sensus ubi est amaritudo.\\
${}^{16}$~Scientia sapientis tamquam inundatio abundabit,\\ et consilium illius sicut fons vit\ae\ permanet.\\
${}^{17}$~Cor fatui quasi vas confractum,\\ et omnem sapientiam non tenebit.\\
${}^{18}$~Verbum sapiens quodcumque audierit scius,\\ laudabit, et ad se adjiciet~:\\ audivit luxuriosus, et displicebit illi,\\ et projiciet illud post dorsum suum.\\
${}^{19}$~Narratio fatui quasi sarcina in via~:\\ nam in labiis sensati invenietur gratia.\\
${}^{20}$~Os prudentis qu\ae ritur in ecclesia,\\ et verba illius cogitabunt in cordibus suis.\\
${}^{21}$~Tamquam domus exterminata, sic fatuo sapientia~:\\ et scientia insensati inenarrabilia verba.\\
${}^{22}$~Compedes in pedibus, stulto doctrina~:\\ et quasi vincula manuum super manum dextram.\\
${}^{23}$~Fatuus in risu exaltat vocem suam~:\\ vir autem sapiens vix tacite ridebit.\\
${}^{24}$~Ornamentum aureum prudenti doctrina,\\ et quasi brachiale in brachio dextro.\\
${}^{25}$~Pes fatui facilis in domum proximi~:\\ et homo peritus confundetur a persona potentis.\\
${}^{26}$~Stultus a fenestra respiciet in domum~:\\ vir autem eruditus foris stabit.\\
${}^{27}$~Stultitia hominis auscultare per ostium~:\\ et prudens gravabitur contumelia.\\
${}^{28}$~Labia imprudentium stulta narrabunt~;\\ verba autem prudentium statera ponderabuntur.\\
${}^{29}$~In ore fatuorum cor illorum,\\ et in corde sapientium os illorum.\\
${}^{30}$~Dum maledicit impius diabolum,\\ maledicit ipse animam suam.\\
${}^{31}$~Susurro coinquinabit animam suam, et in omnibus odietur,\\ et qui cum eo manserit odiosus erit~:\\ tacitus et sensatus honorabitur.\end{verse}\end{flushleft}


\Needspace{2.5\baselineskip}\versal{22}\begin{flushleft}\begin{verse}\vspace{-19pt}\hspace{6pt}In lapide luteo lapidatus est piger~:\\\hspace{6pt} et omnes loquentur super aspernationem illius.\\
${}^{2}$~De stercore boum lapidatus est piger~:\\ et omnis qui tetigerit eum excutiet manus.\end{verse}\end{flushleft}


\begin{flushleft}\begin{verse}${}^{3}$~Confusio patris est de filio indisciplinato~:\\ filia autem in deminoratione fiet.\\
${}^{4}$~Filia prudens h\ae reditas viro suo~:\\ nam qu\ae\ confundit, in contumeliam fit genitoris.\\
${}^{5}$~Patrem et virum confundit audax,\\ et ab impiis non minorabitur~:\\ ab utrisque autem inhonorabitur.\\
${}^{6}$~Musica in luctu importuna narratio~:\\ flagella et doctrina in omni tempore sapientia.\end{verse}\end{flushleft}


\begin{flushleft}\begin{verse}${}^{7}$~Qui docet fatuum,\\ quasi qui conglutinat testam.\\
${}^{8}$~Qui narrat verbum non audienti,\\ quasi qui excitat dormientem de gravi somno.\\
${}^{9}$~Cum dormiente loquitur qui enarrat stulto sapientiam~:\\ et in fine narrationis dicit~: Quis est hic~?\\
${}^{10}$~Supra mortuum plora, defecit enim lux ejus~:\\ et supra fatuum plora, defecit enim sensus.\\
${}^{11}$~Modicum plora super mortuum, quoniam requievit~:\\
${}^{12}$~nequissimi enim nequissima vita super mortem fatui.\\
${}^{13}$~Luctus mortui septem dies~:\\ fatui autem et impii omnes dies vit\ae\ illorum.\\
${}^{14}$~Cum stulto ne multum loquaris,\\ et cum insensato ne abieris.\\
${}^{15}$~Serva te ab illo, ut non molestiam habeas,\\ et non coinquinaberis peccato illius.\\
${}^{16}$~Deflecte ab illo, et invenies requiem,\\ et non acediaberis in stultitia illius.\\
${}^{17}$~Super plumbum quid gravabitur~?\\ et quod illi aliud nomen quam fatuus~?\\
${}^{18}$~Arenam, et salem, et massam ferri facilius est ferre\\ quam hominem imprudentem, et fatuum, et impium.\end{verse}\end{flushleft}


\begin{flushleft}\begin{verse}${}^{19}$~Loramentum ligneum colligatum in fundamento \ae dificii non dissolvetur,\\ sic et cor confirmatum in cogitatione consilii.\\
${}^{20}$~Cogitatus sensati in omni tempore metu non depravabitur.\\
${}^{21}$~Sicut pali in excelsis, et c\ae menta sine impensa posita,\\ contra faciem venti non permanebunt~:\\
${}^{22}$~sic et cor timidum in cogitatione stulti\\ contra impetum timoris non resistet.\\
${}^{23}$~Sicut cor trepidum in cogitatione fatui omni tempore non metuet,\\ sic et qui in pr\ae ceptis Dei permanet semper.\end{verse}\end{flushleft}


\begin{flushleft}\begin{verse}${}^{24}$~Pungens oculum deducit lacrimas,\\ et qui pungit cor profert sensum.\\
${}^{25}$~Mittens lapidem in volatilia, dejiciet illa~:\\ sic et qui conviciatur amico, dissolvit amicitiam.\\
${}^{26}$~Ad amicum etsi produxeris gladium, non desperes~:\\ est enim regressus.\\ Ad amicum
${}^{27}$~si aperueris os triste, non timeas~:\\ est enim concordatio~:\\ excepto convitio, et improperio, et superbia,\\ et mysterii revelatione, et plaga dolosa~:\\ in his omnibus effugiet amicus.\\
${}^{28}$~Fidem posside cum amico in paupertate illius,\\ ut et in bonis illius l\ae teris.\\
${}^{29}$~In tempore tribulationis illius permane illi fidelis,\\ ut et in h\ae reditate illius coh\ae res sis.\\
${}^{30}$~Ante ignem camini vapor et fumus ignis inaltatur~:\\ sic et ante sanguinem maledicta, et contumeli\ae , et min\ae .\\
${}^{31}$~Amicum salutare non confundar,\\ a facie illius non me abscondam~:\\ et si mala mihi evenerint per illum, sustinebo.\\
${}^{32}$~Omnis qui audiet cavebit se ab eo.\end{verse}\end{flushleft}


\begin{flushleft}\begin{verse}${}^{33}$~Quis dabit ori meo custodiam,\\ et super labia mea signaculum certum,\\ ut non cadam ab ipsis,\\ et lingua mea perdat me~?\end{verse}\end{flushleft}


\Needspace{2.5\baselineskip}\versal{23}\begin{flushleft}\begin{verse}\vspace{-19pt}\hspace{6pt}Domine, pater et dominator vit\ae\ me\ae ,\\\hspace{6pt} ne derelinquas me in consilio eorum,\\ nec sinas me cadere in illis.\\
${}^{2}$~Quis superponet in cogitatu meo flagella,\\ et in corde meo doctrinam sapienti\ae ,\\ ut ignorationibus eorum non parcant mihi,\\ et non appareant delicta eorum,\\
${}^{3}$~et ne adincrescant ignoranti\ae\ me\ae ,\\ et multiplicentur delicta mea,\\ et peccata mea abundent,\\ et incidam in conspectu adversariorum meorum,\\ et gaudeat super me inimicus meus~?\\
${}^{4}$~Domine, pater et Deus vit\ae\ me\ae ,\\ ne derelinquas me in cogitatu illorum.\\
${}^{5}$~Extollentiam oculorum meorum ne dederis mihi,\\ et omne desiderium averte a me.\\
${}^{6}$~Aufer a me ventris concupiscentias,\\ et concubitus concupiscenti\ae\ ne apprehendant me,\\ et anim\ae\ irreverenti et infrunit\ae\ ne tradas me.\end{verse}\end{flushleft}


\begin{flushleft}\begin{verse}${}^{7}$~Doctrinam oris audite, filii~:\\ et qui custodierit illam non periet labiis,\\ nec scandalizabitur in operibus nequissimis.\\
${}^{8}$~In vanitate sua apprehenditur peccator~:\\ et superbus et maledicus scandalizabitur in illis.\\
${}^{9}$~Jurationi non assuescat os tuum~:\\ multi enim casus in illa.\\
${}^{10}$~Nominatio vero Dei non sit assidua in ore tuo,\\ et nominibus sanctorum non admiscearis,\\ quoniam non erit immunis ab eis.\\
${}^{11}$~Sicut enim servus interrogatus assidue a livore non minuitur,\\ sic omnis jurans et nominans in toto a peccato non purgabitur.\\
${}^{12}$~Vir multum jurans implebitur iniquitate,\\ et non discedet a domo illius plaga.\\
${}^{13}$~Et si frustraverit, delictum illius super ipsum erit~:\\ et si dissimulaverit, delinquit dupliciter~:\\
${}^{14}$~et si in vacuum juraverit, non justificabitur~:\\ replebitur enim retributione domus illius.\\
${}^{15}$~Est et alia loquela contraria morti~:\\ non inveniatur in h\ae reditate Jacob.\\
${}^{16}$~Etenim a misericordibus omnia h\ae c auferentur,\\ et in delictis non volutabuntur.\\
${}^{17}$~Indisciplinat\ae\ loquel\ae\ non assuescat os tuum~:\\ est enim in illa verbum peccati.\\
${}^{18}$~Memento patris et matris tu\ae~:\\ in medio enim magnatorum consistis~:\\
${}^{19}$~ne forte obliviscatur te Deus in conspectu illorum,\\ et assiduitate tua infatuatus, improperium patiaris,\\ et maluisses non nasci,\\ et diem nativitatis tu\ae\ maledicas.\\
${}^{20}$~Homo assuetus in verbis improperii\\ in omnibus diebus suis non erudietur.\end{verse}\end{flushleft}


\begin{flushleft}\begin{verse}${}^{21}$~Duo genera abundant in peccatis,\\ et tertium adducit iram et perditionem.\\
${}^{22}$~Anima calida quasi ignis ardens,\\ non extinguetur donec aliquid glutiat~:\\
${}^{23}$~et homo nequam in ore carnis su\ae \\ non desinet donec incendat ignem.\\
${}^{24}$~Homini fornicario omnis panis dulcis~:\\ non fatigabitur transgrediens usque ad finem.\\
${}^{25}$~Omnis homo qui transgreditur lectum suum,\\ contemnens in animam suam, et dicens~: Quis me videt~?\\
${}^{26}$~Tenebr\ae\ circumdant me, et parietes cooperiunt me,\\ et nemo circumspicit me~: quem vereor~?\\ delictorum meorum non memorabitur Altissimus.\\
${}^{27}$~Et non intelligit quoniam omnia videt oculus illius,\\ quoniam expellit a se timorem Dei hujusmodi hominis timor,\\ et oculi hominum timentes illum~:\\
${}^{28}$~et non cognovit quoniam oculi Domini\\ multo plus lucidiores sunt super solem,\\ circumspicientes omnes vias hominum,\\ et profundum abyssi, et hominum corda,\\ intuentes in absconditas partes.\\
${}^{29}$~Domino enim Deo antequam crearentur omnia sunt agnita~:\\ sic et post perfectum respicit omnia.\\
${}^{30}$~Hic in plateis civitatis vindicabitur,\\ et quasi pullus equinus fugabitur,\\ et ubi non speravit apprehendetur.\\
${}^{31}$~Et erit dedecus omnibus,\\ eo quod non intellexerit timorem Domini.\\
${}^{32}$~Sic et mulier omnis relinquens virum suum,\\ et statuens h\ae reditatem ex alieno matrimonio~:\\
${}^{33}$~primo enim in lege Altissimi incredibilis fuit~:\\ secundo in virum suum deliquit~:\\ tertio in adulterio fornicata est,\\ et ex alio viro filios statuit sibi.\\
${}^{34}$~H\ae c in ecclesiam adducetur,\\ et in filios ejus respicietur~:\\
${}^{35}$~non tradent filii ejus radices,\\ et rami ejus non dabunt fructum~:\\
${}^{36}$~derelinquet in maledictum memoriam ejus,\\ et dedecus illius non delebitur.\\
${}^{37}$~Et agnoscent qui derelicti sunt,\\ quoniam nihil melius est quam timor Dei,\\ et nihil dulcius quam respicere in mandatis Domini.\\
${}^{38}$~Gloria magna est sequi Dominum~:\\ longitudo enim dierum assumetur ab eo.\end{verse}\end{flushleft}


\Needspace{2.5\baselineskip}\versal{24}\begin{flushleft}\begin{verse}\vspace{-19pt}\hspace{6pt}Sapientia laudabit animam suam,\\\hspace{6pt} et in Deo honorabitur,\\ et in medio populi sui gloriabitur,\\
${}^{2}$~et in ecclesiis Altissimi aperiet os suum,\\ et in conspectu virtutis illius gloriabitur,\\
${}^{3}$~et in medio populi sui exaltabitur,\\ et in plenitudine sancta admirabitur,\\
${}^{4}$~et in multitudine electorum habebit laudem,\\ et inter benedictos benedicetur, dicens~:\\
${}^{5}$~Ego ex ore Altissimi prodivi,\\ primogenita ante omnem creaturam.\\
${}^{6}$~Ego feci in c\ae lis ut oriretur lumen indeficiens,\\ et sicut nebula texi omnem terram.\\
${}^{7}$~Ego in altissimis habitavi,\\ et thronus meus in columna nubis.\\
${}^{8}$~Gyrum c\ae li circuivi sola,\\ et profundum abyssi penetravi~:\\ in fluctibus maris ambulavi.\\
${}^{9}$~Et in omni terra steti~:\\ et in omni populo,\\
${}^{10}$~et in omni gente primatum habui~:\\
${}^{11}$~et omnium excellentium et humilium corda virtute calcavi.\\ Et in his omnibus requiem qu\ae sivi,\\ et in h\ae reditate Domini morabor.\\
${}^{12}$~Tunc pr\ae cepit, et dixit mihi Creator omnium~:\\ et qui creavit me, requievit in tabernaculo meo.\\
${}^{13}$~Et dixit mihi~: In Jacob inhabita,\\ et in Isra\"el h\ae reditare,\\ et in electis meis mitte radices.\end{verse}\end{flushleft}


\begin{flushleft}\begin{verse}${}^{14}$~Ab initio et ante s\ae cula creata sum,\\ et usque ad futurum s\ae culum non desinam~:\\ et in habitatione sancta coram ipso ministravi.\\
${}^{15}$~Et sic in Sion firmata sum,\\ et in civitate sanctificata similiter requievi,\\ et in Jerusalem potestas mea.\\
${}^{16}$~Et radicavi in populo honorificato,\\ et in parte Dei mei h\ae reditas illius,\\ et in plenitudine sanctorum detentio mea.\end{verse}\end{flushleft}


\begin{flushleft}\begin{verse}${}^{17}$~Quasi cedrus exaltata sum in Libano,\\ et quasi cypressus in monte Sion~:\\
${}^{18}$~quasi palma exaltata sum in Cades,\\ et quasi plantatio ros\ae\ in Jericho~:\\
${}^{19}$~quasi oliva speciosa in campis,\\ et quasi platanus exaltata sum juxta aquam in plateis.\\
${}^{20}$~Sicut cinnamomum et balsamum aromatizans odorem dedi~;\\ quasi myrrha electa dedi suavitatem odoris~:\\
${}^{21}$~et quasi storax, et galbanus, et ungula, et gutta,\\ et quasi Libanus non incisus vaporavi habitationem meam,\\ et quasi balsamum non mistum odor meus.\\
${}^{22}$~Ego quasi terebinthus extendi ramos meos,\\ et rami mei honoris et grati\ae .\\
${}^{23}$~Ego quasi vitis fructificavi suavitatem odoris~:\\ et flores mei fructus honoris et honestatis.\\
${}^{24}$~Ego mater pulchr\ae\ dilectionis, et timoris,\\ et agnitionis, et sanct\ae\ spei.\\
${}^{25}$~In me gratia omnis vi\ae\ et veritatis~:\\ in me omnis spes vit\ae\ et virtutis.\\
${}^{26}$~Transite ad me, omnes qui concupiscitis me,\\ et a generationibus meis implemini~:\\
${}^{27}$~spiritus enim meus super mel dulcis,\\ et h\ae reditas mea super mel et favum.\\
${}^{28}$~Memoria mea in generationes s\ae culorum.\\
${}^{29}$~Qui edunt me, adhuc esurient,\\ et qui bibunt me, adhuc sitient.\\
${}^{30}$~Qui audit me non confundetur,\\ et qui operantur in me non peccabunt~:\\
${}^{31}$~qui elucidant me, vitam \ae ternam habebunt.\end{verse}\end{flushleft}


\begin{flushleft}\begin{verse}${}^{32}$~H\ae c omnia liber vit\ae ,\\ et testamentum Altissimi, et agnitio veritatis.\\
${}^{33}$~Legem mandavit Moyses in pr\ae ceptis justitiarum,\\ et h\ae reditatem domui Jacob,\\ et Isra\"el promissiones.\\
${}^{34}$~Posuit David, puero suo,\\ excitare regem ex ipso fortissimum,\\ et in throno honoris sedentem in sempiternum.\\
${}^{35}$~Qui implet quasi Phison sapientiam,\\ et sicut Tigris in diebus novorum~:\\
${}^{36}$~qui adimplet quasi Euphrates sensum,\\ qui multiplicat quasi Jordanis in tempore messis~:\\
${}^{37}$~qui mittit disciplinam sicut lucem,\\ et assistens quasi Gehon in die vindemi\ae .\\
${}^{38}$~Qui perficit primus scire ipsam,\\ et infirmior non investigabit eam.\\
${}^{39}$~A mari enim abundavit cogitatio ejus,\\ et consilium illius ab abysso magna.\\
${}^{40}$~Ego sapientia effudi flumina~:\\
${}^{41}$~ego quasi trames aqu\ae\ immens\ae\ de fluvio~:\\ ego quasi fluvii dioryx,\\ et sicut aqu\ae ductus exivi de paradiso.\\
${}^{42}$~Dixi~: Rigabo hortum meum plantationum,\\ et inebriabo prati mei fructum.\\
${}^{43}$~Et ecce factus est mihi trames abundans,\\ et fluvius meus appropinquavit ad mare~:\\
${}^{44}$~quoniam doctrinam quasi antelucanum illumino omnibus,\\ et enarrabo illam usque ad longinquum.\\
${}^{45}$~Penetrabo omnes inferiores partes terr\ae ,\\ et inspiciam omnes dormientes,\\ et illuminabo omnes sperantes in Domino.\\
${}^{46}$~Adhuc doctrinam quasi prophetiam effundam,\\ et relinquam illam qu\ae rentibus sapientiam,\\ et non desinam in progenies illorum usque in \ae vum sanctum.\\
${}^{47}$~Videte quoniam non soli mihi laboravi,\\ sed omnibus exquirentibus veritatem.\end{verse}\end{flushleft}


\Needspace{2.5\baselineskip}\versal{25}\begin{flushleft}\begin{verse}\vspace{-19pt}\hspace{6pt}In tribus placitum est spiritui meo,\\\hspace{6pt} qu\ae\ sunt probata coram Deo et hominibus~:\\
${}^{2}$~concordia fratrum,\\ et amor proximorum,\\ et vir et mulier bene sibi consentientes.\\
${}^{3}$~Tres species odivit anima mea,\\ et aggravor valde anim\ae\ illorum~:\\
${}^{4}$~pauperem superbum, divitem mendacem,\\ senem fatuum et insensatum.\end{verse}\end{flushleft}


\begin{flushleft}\begin{verse}${}^{5}$~Qu\ae\ in juventute tua non congregasti,\\ quomodo in senectute tua invenies~?\\
${}^{6}$~Quam speciosum canitiei judicium,\\ et presbyteris cognoscere consilium~!\\
${}^{7}$~Quam speciosa veteranis sapientia,\\ et gloriosus intellectus et consilium~!\\
${}^{8}$~Corona senum multa peritia,\\ et gloria illorum timor Dei.\end{verse}\end{flushleft}


\begin{flushleft}\begin{verse}${}^{9}$~Novem insuspicabilia cordis magnificavi~:\\ et decimum dicam in lingua hominibus~:\\
${}^{10}$~homo qui jucundatur in filiis,\\ vivens et videns subversionem inimicorum suorum.\\
${}^{11}$~Beatus qui habitat cum muliere sensata,\\ et qui lingua sua non est lapsus,\\ et qui non servivit indignis se.\\
${}^{12}$~Beatus qui invenit amicum verum,\\ et qui enarrat justitiam auri audienti.\\
${}^{13}$~Quam magnus qui invenit sapientiam et scientiam~!\\ sed non est super timentem Dominum.\\
${}^{14}$~Timor Dei super omnia se superposuit.\\
${}^{15}$~Beatus homo cui donatum est habere timorem Dei~:\\ qui tenet illum, cui assimilabitur~?\\
${}^{16}$~Timor Dei initium dilectionis ejus~:\\ fidei autem initium agglutinandum est ei.\end{verse}\end{flushleft}


\begin{flushleft}\begin{verse}${}^{17}$~Omnis plaga tristitia cordis est,\\ et omnis malitia nequitia mulieris.\\
${}^{18}$~Et omnem plagam, et non plagam videbit cordis~:\\
${}^{19}$~et omnem nequitiam, et non nequitiam mulieris~:\\
${}^{20}$~et omnem obductum, et non obductum odientium~:\\
${}^{21}$~et omnem vindictam, et non vindictam inimicorum.\\
${}^{22}$~Non est caput nequius super caput colubri,\\
${}^{23}$~et non est ira super iram mulieris.\\ Commorari leoni et draconi placebit,\\ quam habitare cum muliere nequam.\\
${}^{24}$~Nequitia mulieris immutat faciem ejus~:\\ et obc\ae cat vultum suum tamquam ursus,\\ et quasi saccum ostendit.\\ In medio proximorum ejus\\
${}^{25}$~ingemuit vir ejus,\\ et audiens suspiravit modicum.\\
${}^{26}$~Brevis omnis malitia super malitiam mulieris~:\\ sors peccatorum cadat super illam.\\
${}^{27}$~Sicut ascensus arenosus in pedibus veterani,\\ sic mulier linguata homini quieto.\\
${}^{28}$~Ne respicias in mulieris speciem,\\ et non concupiscas mulierem in specie.\\
${}^{29}$~Mulieris ira, et irreverentia,\\ et confusio magna.\\
${}^{30}$~Mulier si primatum habeat,\\ contraria est viro suo.\\
${}^{31}$~Cor humile, et facies tristis,\\ et plaga cordis, mulier nequam.\\
${}^{32}$~Manus debiles et genua dissoluta,\\ mulier qu\ae\ non beatificat virum suum.\\
${}^{33}$~A muliere initium factum est peccati,\\ et per illam omnes morimur.\\
${}^{34}$~Non des aqu\ae\ tu\ae\ exitum, nec modicum~:\\ nec mulieri nequam veniam prodeundi.\\
${}^{35}$~Si non ambulaverit ad manum tuam,\\ confundet te in conspectu inimicorum.\\
${}^{36}$~A carnibus tuis abscinde illam,\\ ne semper te abutatur.\end{verse}\end{flushleft}


\Needspace{2.5\baselineskip}\versal{26}\begin{flushleft}\begin{verse}\vspace{-19pt}\hspace{6pt}Mulieris bon\ae\ beatus vir~:\\\hspace{6pt} numerus enim annorum illius duplex.\\
${}^{2}$~Mulier fortis oblectat virum suum,\\ et annos vit\ae\ illius in pace implebit.\\
${}^{3}$~Pars bona mulier bona,\\ in parte timentium Deum dabitur viro pro factis bonis~:\\
${}^{4}$~divitis autem et pauperis cor bonum,\\ in omni tempore vultus illorum hilaris.\\
${}^{5}$~A tribus timuit cor meum,\\ et in quarto facies mea metuit~:\\
${}^{6}$~delaturam civitatis, et collectionem populi~:\\
${}^{7}$~calumniam mendacem super mortem omnia gravia~:\\
${}^{8}$~dolor cordis et luctus, mulier zelotypa.\\
${}^{9}$~In muliere zelotypa flagellum lingu\ae ,\\ omnibus communicans.\\
${}^{10}$~Sicut boum jugum quod movetur, ita et mulier nequam~:\\ qui tenet illam quasi qui apprehendit scorpionem.\\
${}^{11}$~Mulier ebriosa ira magna, et contumelia~:\\ et turpitudo illius non tegetur.\\
${}^{12}$~Fornicatio mulieris in extollentia oculorum,\\ et in palpebris illius agnoscetur.\\
${}^{13}$~In filia non avertente se, firma custodiam,\\ ne inventa occasione utatur se.\\
${}^{14}$~Ab omni irreverentia oculorum ejus cave,\\ et ne mireris si te neglexerit.\\
${}^{15}$~Sicut viator sitiens ad fontem os aperiet,\\ et ab omni aqua proxima bibet,\\ et contra omnem palum sedebit,\\ et contra omnem sagittam aperiet pharetram donec deficiat.\\
${}^{16}$~Gratia mulieris sedul\ae\ delectabit virum suum,\\ et ossa illius impinguabit.\\
${}^{17}$~Disciplina illius datum Dei est.\\
${}^{18}$~Mulier sensata et tacita,\\ non est immutatio erudit\ae\ anim\ae .\\
${}^{19}$~Gratia super gratiam\\ mulier sancta et pudorata.\\
${}^{20}$~Omnis autem ponderatio non est digna\\ continentis anim\ae .\\
${}^{21}$~Sicut sol oriens mundo in altissimis Dei,\\ sic mulieris bon\ae\ species in ornamentum domus ejus.\\
${}^{22}$~Lucerna splendens super candelabrum sanctum,\\ et species faciei super \ae tatem stabilem.\\
${}^{23}$~Column\ae\ aure\ae\ super bases argenteas,\\ et pedes firmi super plantas stabilis mulieris.\\
${}^{24}$~Fundamenta \ae terna supra petram solidam,\\ et mandata Dei in corde mulieris sanct\ae .\end{verse}\end{flushleft}


\begin{flushleft}\begin{verse}${}^{25}$~In duobus contristatum est cor meum,\\ et in tertio iracundia mihi advenit~:\\
${}^{26}$~vir bellator deficiens per inopiam~;\\ et vir sensatus contemptus~;\\
${}^{27}$~et qui transgreditur a justitia ad peccatum~:\\ Deus paravit eum ad rhomph\ae am.\end{verse}\end{flushleft}


\begin{flushleft}\begin{verse}${}^{28}$~Du\ae\ species difficiles et periculos\ae\ mihi apparuerunt~:\\ difficile exuitur negotians a negligentia,\\ et non justificabitur caupo a peccatis labiorum.\end{verse}\end{flushleft}


\Needspace{2.5\baselineskip}\versal{27}\begin{flushleft}\begin{verse}\vspace{-19pt}\hspace{6pt}Propter inopiam multi deliquerunt~:\\\hspace{6pt} et qui qu\ae rit locupletari avertit oculum suum.\\
${}^{2}$~Sicut in medio compaginis lapidum palus figitur,\\ sic et inter medium venditionis et emptionis angustiabitur peccatum~:\\
${}^{3}$~conteretur cum delinquente delictum.\\
${}^{4}$~Si non in timore Domini tenueris te instanter,\\ cito subvertetur domus tua.\end{verse}\end{flushleft}


\begin{flushleft}\begin{verse}${}^{5}$~Sicut in percussura cribri remanebit pulvis,\\ sic aporia hominis in cogitatu illius.\\
${}^{6}$~Vasa figuli probat fornax,\\ et homines justos tentatio tribulationis.\\
${}^{7}$~Sicut rusticatio de ligno ostendit fructum illius,\\ sic verbum ex cogitatu cordis hominis.\\
${}^{8}$~Ante sermonem non laudes virum~:\\ h\ae c enim tentatio est hominum.\\
${}^{9}$~Si sequaris justitiam, apprehendes illam,\\ et indues quasi poderem honoris~:\\ et inhabitabis cum ea, et proteget te in sempiternum,\\ et in die agnitionis invenies firmamentum.\\
${}^{10}$~Volatilia ad sibi similia conveniunt~:\\ et veritas ad eos qui operantur illam revertetur.\\
${}^{11}$~Leo venationi insidiatur semper~:\\ sic peccata operantibus iniquitates.\\
${}^{12}$~Homo sanctus in sapientia manet sicut sol~:\\ nam stultus sicut luna mutatur.\\
${}^{13}$~In medio insensatorum serva verbum tempori~:\\ in medio autem cogitantium assiduus esto.\\
${}^{14}$~Narratio peccantium odiosa,\\ et risus illorum in deliciis peccati.\\
${}^{15}$~Loquela multum jurans horripilationem capiti statuet,\\ et irreverentia ipsius obturatio aurium.\\
${}^{16}$~Effusio sanguinis in rixa superborum,\\ et maledictio illorum auditus gravis.\\
${}^{17}$~Qui denudat arcana amici fidem perdit,\\ et non inveniet amicum ad animum suum.\\
${}^{18}$~Dilige proximum,\\ et conjungere fide cum illo.\\
${}^{19}$~Quod si denudaveris absconsa illius,\\ non persequeris post eum.\\
${}^{20}$~Sicut enim homo qui perdit amicum suum,\\ sic et qui perdit amicitiam proximi sui.\\
${}^{21}$~Et sicut qui dimittit avem de manu sua,\\ sic dereliquisti proximum tuum, et non eum capies.\\
${}^{22}$~Non illum sequaris, quoniam longe abest~:\\ effugit enim quasi caprea de laqueo, quoniam vulnerata est anima ejus~:\\
${}^{23}$~ultra eum non poteris colligare.\\ Et maledicti est concordatio~:\\
${}^{24}$~denudare autem amici mysteria, desperatio est anim\ae\ infelicis.\end{verse}\end{flushleft}


\begin{flushleft}\begin{verse}${}^{25}$~Annuens oculo fabricat iniqua,\\ et nemo eum abjiciet.\\
${}^{26}$~In conspectu oculorum tuorum condulcabit os suum,\\ et super sermones tuos admirabitur~:\\ novissime autem pervertet os suum,\\ et in verbis tuis dabit scandalum.\\
${}^{27}$~Multa odivi, et non co\ae quavi ei,\\ et Dominus odiet illum.\\
${}^{28}$~Qui in altum mittit lapidem, super caput ejus cadet~:\\ et plaga dolosa dolosi dividet vulnera.\\
${}^{29}$~Et qui foveam fodit incidet in eam~:\\ et qui statuit lapidem proximo offendet in eo~:\\ et qui laqueum alii ponit, peribit in illo.\\
${}^{30}$~Facienti nequissimum consilium, super ipsum devolvetur,\\ et non agnoscet unde adveniat illi.\\
${}^{31}$~Illusio et improperium superborum,\\ et vindicta sicut leo insidiabitur illi.\\
${}^{32}$~Laqueo peribunt qui oblectantur casu justorum,\\ dolor autem consumet illos antequam moriantur.\end{verse}\end{flushleft}


\begin{flushleft}\begin{verse}${}^{33}$~Ira et furor utraque execrabilia sunt,\\ et vir peccator continens erit illorum.\end{verse}\end{flushleft}


\Needspace{2.5\baselineskip}\versal{28}\begin{flushleft}\begin{verse}\vspace{-19pt}\hspace{6pt}Qui vindicari vult, a Domino inveniet vindictam,\\\hspace{6pt} et peccata illius servans servabit.\\
${}^{2}$~Relinque proximo tuo nocenti te,\\ et tunc deprecanti tibi peccata solventur.\\
${}^{3}$~Homo homini reservat iram,\\ et a Deo qu\ae rit medelam~:\\
${}^{4}$~in hominem similem sibi non habet misericordiam,\\ et de peccatis suis deprecatur.\\
${}^{5}$~Ipse cum caro sit reservat iram,\\ et propitiationem petit a Deo~:\\ quis exorabit pro delictis illius~?\\
${}^{6}$~Memento novissimorum, et desine inimicari~:\\
${}^{7}$~tabitudo enim et mors imminent in mandatis ejus.\\
${}^{8}$~Memorare timorem Dei,\\ et non irascaris proximo.\\
${}^{9}$~Memorare testamentum Altissimi,\\ et despice ignorantiam proximi.\\
${}^{10}$~Abstine te a lite,\\ et minues peccata.\\
${}^{11}$~Homo enim iracundus incendit litem,\\ et vir peccator turbabit amicos,\\ et in medio pacem habentium immittet inimicitiam.\\
${}^{12}$~Secundum enim ligna silv\ae\ sic ignis exardescit~:\\ et secundum virtutem hominis sic iracundia illius erit,\\ et secundum substantiam suam exaltabit iram suam.\\
${}^{13}$~Certamen festinatum incendit ignem,\\ et lis festinans effundit sanguinem~:\\ et lingua testificans adducit mortem.\\
${}^{14}$~Si sufflaveris in scintillam, quasi ignis exardebit~:\\ et si exspueris super illam, extinguetur~:\\ utraque ex ore proficiscuntur.\end{verse}\end{flushleft}


\begin{flushleft}\begin{verse}${}^{15}$~Susurro et bilinguis maledictus,\\ multos enim turbabit pacem habentes.\\
${}^{16}$~Lingua tertia multos commovit,\\ et dispersit illos de gente in gentem.\\
${}^{17}$~Civitates muratas divitum destruxit,\\ et domus magnatorum effodit.\\
${}^{18}$~Virtutes populorum concidit,\\ et gentes fortes dissolvit.\\
${}^{19}$~Lingua tertia mulieres viratas ejecit,\\ et privavit illas laboribus suis.\\
${}^{20}$~Qui respicit illam non habebit requiem,\\ nec habebit amicum in quo requiescat.\\
${}^{21}$~Flagelli plaga livorem facit~:\\ plaga autem lingu\ae\ comminuet ossa.\\
${}^{22}$~Multi ceciderunt in ore gladii~:\\ sed non sic quasi qui interierunt per linguam suam.\\
${}^{23}$~Beatus qui tectus est a lingua nequam,\\ qui in iracundiam illius non transivit,\\ et qui non attraxit jugum illius,\\ et in vinculis ejus non est ligatus~:\\
${}^{24}$~jugum enim illius jugum ferreum est,\\ et vinculum illius vinculum \ae reum est~;\\
${}^{25}$~mors illius mors nequissima~:\\ et utilis potius infernus quam illa.\\
${}^{26}$~Perseverantia illius non permanebit,\\ sed obtinebit vias injustorum,\\ et in flamma sua non comburet justos.\\
${}^{27}$~Qui relinquunt Deum incident in illam,\\ et exardebit in illis, et non extinguetur,\\ et immittetur in illos quasi leo,\\ et quasi pardus l\ae det illos.\\
${}^{28}$~Sepi aures tuas spinis~:\\ linguam nequam noli audire~:\\ et ori tuo facito ostia et seras.\\
${}^{29}$~Aurum tuum et argentum tuum confla,\\ et verbis tuis facito stateram,\\ et frenos ori tuo rectos~:\\
${}^{30}$~et attende ne forte labaris in lingua,\\ et cadas in conspectu inimicorum insidiantium tibi,\\ et sit casus tuus insanabilis in mortem.\end{verse}\end{flushleft}


\Needspace{2.5\baselineskip}\versal{29}\begin{flushleft}\begin{verse}\vspace{-19pt}\hspace{6pt}Qui facit misericordiam fœneratur proximo suo~:\\\hspace{6pt} et qui pr\ae valet manu mandata servat.\\
${}^{2}$~Fœnerare proximo tuo in tempore necessitatis illius~:\\ et iterum redde proximo in tempore suo.\\
${}^{3}$~Confirma verbum, et fideliter age cum illo~:\\ et in omni tempore invenies quod tibi necessarium est.\\
${}^{4}$~Multi quasi inventionem \ae stimaverunt fœnus,\\ et pr\ae stiterunt molestiam his qui se adjuverunt.\\
${}^{5}$~Donec accipiant, osculantur manus dantis,\\ et in promissionibus humiliant vocem suam~:\\
${}^{6}$~et in tempore redditionis postulabit tempus,\\ et loquetur verba t\ae dii et murmurationum,\\ et tempus causabitur.\\
${}^{7}$~Si autem potuerit reddere, adversabitur~:\\ solidi vix reddet dimidium,\\ et computabit illud quasi inventionem~:\\
${}^{8}$~sin autem, fraudabit illum pecunia sua,\\ et possidebit illum inimicum gratis~:\\
${}^{9}$~et convitia et maledicta reddet illi,\\ et pro honore et beneficio reddet illi contumeliam.\\
${}^{10}$~Multi non causa nequiti\ae\ non fœnerati sunt,\\ sed fraudari gratis timuerunt.\\
${}^{11}$~Verumtamen super humilem animo fortior esto,\\ et pro eleemosyna non trahas illum.\\
${}^{12}$~Propter mandatum assume pauperem,\\ et propter inopiam ejus ne dimittas eum vacuum.\\
${}^{13}$~Perde pecuniam propter fratrem et amicum tuum,\\ et non abscondas illam sub lapide in perditionem.\\
${}^{14}$~Pone thesaurum tuum in pr\ae ceptis Altissimi,\\ et proderit tibi magis quam aurum.\\
${}^{15}$~Conclude eleemosynam in corde pauperis,\\ et h\ae c pro te exorabit ab omni malo.\\
${}^{16}$~Super scutum potentis
${}^{17}$~et super lanceam\\
${}^{18}$~adversus inimicum tuum pugnabit.\end{verse}\end{flushleft}


\begin{flushleft}\begin{verse}${}^{19}$~Vir bonus fidem facit pro proximo suo~:\\ et qui perdiderit confusionem derelinquet sibi.\\
${}^{20}$~Gratiam fidejussoris ne obliviscaris~:\\ dedit enim pro te animam suam.\\
${}^{21}$~Repromissorem fugit peccator et immundus.\\
${}^{22}$~Bona repromissoris sibi ascribit peccator~:\\ et ingratus sensu derelinquet liberantem se.\\
${}^{23}$~Vir repromittit de proximo suo~:\\ et cum perdiderit reverentiam, derelinquetur ab eo.\\
${}^{24}$~Repromissio nequissima multos perdidit dirigentes,\\ et commovit illos quasi fluctus maris.\\
${}^{25}$~Viros potentes gyrans migrare fecit,\\ et vagati sunt in gentibus alienis.\\
${}^{26}$~Peccator transgrediens mandatum Domini incidet in promissionem nequam~:\\ et qui conatur multa agere incidet in judicium.\\
${}^{27}$~Recupera proximum secundum virtutem tuam,\\ et attende tibi ne incidas.\end{verse}\end{flushleft}


\begin{flushleft}\begin{verse}${}^{28}$~Initium vit\ae\ hominis, aqua et panis,\\ et vestimentum, et domus protegens turpitudinem.\\
${}^{29}$~Melior est victus pauperis sub tegmine asserum\\ quam epul\ae\ splendid\ae\ in peregre sine domicilio.\\
${}^{30}$~Minimum pro magno placeat tibi,\\ et improperium peregrinationis non audies.\\
${}^{31}$~Vita nequam hospitandi de domo in domum~:\\ et ubi hospitabitur non fiducialiter aget, nec aperiet os.\\
${}^{32}$~Hospitabitur, et pascet, et potabit ingratos,\\ et ad h\ae c amara audiet~:\\
${}^{33}$~transi, hospes, et orna mensam,\\ et qu\ae\ in manu habes ciba ceteros.\\
${}^{34}$~Exi a facie honoris amicorum meorum~:\\ necessitudine domus me\ae\ hospitio mihi factus est frater.\\
${}^{35}$~Gravia h\ae c homini habenti sensum~:\\ correptio domus, et improperium fœneratoris.\end{verse}\end{flushleft}


\Needspace{2.5\baselineskip}\versal{30}\begin{flushleft}\begin{verse}\vspace{-19pt}\hspace{6pt}Qui diligit filium suum assiduat illi flagella,\\\hspace{6pt} ut l\ae tetur in novissimo suo,\\ et non palpet proximorum ostia.\\
${}^{2}$~Qui docet filium suum laudabitur in illo,\\ et in medio domesticorum in illo gloriabitur.\\
${}^{3}$~Qui docet filium suum in zelum mittit inimicum,\\ et in medio amicorum gloriabitur in illo.\\
${}^{4}$~Mortuus est pater ejus, et quasi non est mortuus~:\\ similem enim reliquit sibi post se.\\
${}^{5}$~In vita sua vidit, et l\ae tatus est in illo~:\\ in obitu suo non est contristatus,\\ nec confusus est coram inimicis~:\\
${}^{6}$~reliquit enim defensorem domus contra inimicos,\\ et amicis reddentem gratiam.\\
${}^{7}$~Pro animabus filiorum colligabit vulnera sua,\\ et super omnem vocem turbabuntur viscera ejus.\\
${}^{8}$~Equus indomitus evadit durus,\\ et filius remissus evadet pr\ae ceps.\\
${}^{9}$~Lacta filium, et paventem te faciet~:\\ lude cum eo, et contristabit te.\\
${}^{10}$~Non corrideas illi, ne doleas,\\ et in novissimo obstupescent dentes tui.\\
${}^{11}$~Non des illi potestatem in juventute,\\ et ne despicias cogitatus illius.\\
${}^{12}$~Curva cervicem ejus in juventute,\\ et tunde latera ejus dum infans est,\\ ne forte induret, et non credat tibi,\\ et erit tibi dolor anim\ae .\\
${}^{13}$~Doce filium tuum, et operare in illo,\\ ne in turpitudinem illius offendas.\end{verse}\end{flushleft}


\begin{flushleft}\begin{verse}${}^{14}$~Melior est pauper sanus, et fortis viribus,\\ quam dives imbecillis et flagellatus malitia.\\
${}^{15}$~Salus anim\ae\ in sanctitate justiti\ae \\ melior est omni auro et argento~:\\ et corpus validum quam census immensus.\\
${}^{16}$~Non est census super censum salutis corporis,\\ et non est oblectamentum super cordis gaudium.\\
${}^{17}$~Melior est mors quam vita amara,\\ et requies \ae terna quam languor perseverans.\\
${}^{18}$~Bona abscondita in ore clauso,\\ quasi appositiones epularum circumposit\ae\ sepulchro.\\
${}^{19}$~Quid proderit libatio idolo~?\\ nec enim manducabit, nec odorabit.\\
${}^{20}$~Sic qui effugatur a Domino,\\ portans mercedes iniquitatis~:\\
${}^{21}$~videns oculis et ingemiscens,\\ sicut spado complectens virginem, et suspirans.\\
${}^{22}$~Tristitiam non des anim\ae\ tu\ae ,\\ et non affligas temetipsum in consilio tuo.\\
${}^{23}$~Jucunditas cordis, h\ae c est vita hominis,\\ et thesaurus sine defectione sanctitatis~:\\ et exsultatio viri est long\ae vitas.\\
${}^{24}$~Miserere anim\ae\ tu\ae\ placens Deo, et contine~:\\ congrega cor tuum in sanctitate ejus,\\ et tristitiam longe repelle a te.\\
${}^{25}$~Multos enim occidit tristitia,\\ et non est utilitas in illa.\\
${}^{26}$~Zelus et iracundia minuunt dies,\\ et ante tempus senectam adducet cogitatus.\\
${}^{27}$~Splendidum cor et bonum in epulis est~:\\ epul\ae\ enim illius diligenter fiunt.\end{verse}\end{flushleft}


\Needspace{2.5\baselineskip}\versal{31}\begin{flushleft}\begin{verse}\vspace{-19pt}\hspace{6pt}Vigilia honestatis tabefaciet carnes,\\\hspace{6pt} et cogitatus illius auferet somnum.\\
${}^{2}$~Cogitatus pr\ae scienti\ae\ avertit sensum,\\ et infirmitas gravis sobriam facit animam.\\
${}^{3}$~Laboravit dives in congregatione substanti\ae ,\\ et in requie sua replebitur bonis suis.\\
${}^{4}$~Laboravit pauper in diminutione victus,\\ et in fine inops fit.\\
${}^{5}$~Qui aurum diligit non justificabitur,\\ et qui insequitur consumptionem replebitur ex ea.\\
${}^{6}$~Multi dati sunt in auri casus,\\ et facta est in specie ipsius perditio illorum.\\
${}^{7}$~Lignum offensionis est aurum sacrificantium~:\\ v\ae\ illis qui sectantur illud~!\\ et omnis imprudens deperiet in illo.\\
${}^{8}$~Beatus dives qui inventus est sine macula,\\ et qui post aurum non abiit,\\ nec speravit in pecunia et thesauris.\\
${}^{9}$~Quis est hic~? et laudabimus eum~:\\ fecit enim mirabilia in vita sua.\\
${}^{10}$~Qui probatus est in illo, et perfectus est, erit illi gloria \ae terna~:\\ qui potuit transgredi, et non est transgressus~;\\ facere mala, et non fecit.\\
${}^{11}$~Ideo stabilita sunt bona illius in Domino,\\ et eleemosynas illius enarrabit omnis ecclesia sanctorum.\end{verse}\end{flushleft}


\begin{flushleft}\begin{verse}${}^{12}$~Supra mensam magnam sedisti~?\\ non aperias super illam faucem tuam prior.\\
${}^{13}$~Non dicas sic~: Multa sunt, qu\ae\ super illam sunt.\\
${}^{14}$~Memento quoniam malus est oculus nequam.\\
${}^{15}$~Nequius oculo quid creatum est~?\\ ideo ab omni facie sua lacrimabitur, cum viderit.\\
${}^{16}$~Ne extendas manum tuam prior,\\ et invidia contaminatus erubescas.\\
${}^{17}$~Ne comprimaris in convivio.\\
${}^{18}$~Intellige qu\ae\ sunt proximi tui ex teipso.\\
${}^{19}$~Utere quasi homo frugi his qu\ae\ tibi apponuntur~:\\ ne, cum manducas multum, odio habearis.\\
${}^{20}$~Cessa prior causa disciplin\ae~:\\ et noli nimius esse, ne forte offendas.\\
${}^{21}$~Et si in medio multorum sedisti,\\ prior illis ne extendas manum tuam,\\ nec prior poscas bibere.\\
${}^{22}$~Quam sufficiens est homini erudito vinum exiguum~!\\ et in dormiendo non laborabis ab illo,\\ et non senties dolorem.\\
${}^{23}$~Vigilia, cholera et tortura viro infrunito,\\
${}^{24}$~somnus sanitatis in homine parco~:\\ dormiet usque mane,\\ et anima illius cum ipso delectabitur.\\
${}^{25}$~Et si coactus fueris in edendo multum,\\ surge e medio, evome, et refrigerabit te,\\ et non adduces corpori tuo infirmitatem.\\
${}^{26}$~Audi me, fili, et ne spernas me,\\ et in novissimo invenies verba mea.\\
${}^{27}$~In omnibus operibus tuis esto velox,\\ et omnis infirmitas non occurret tibi.\\
${}^{28}$~Splendidum in panibus benedicent labia multorum,\\ et testimonium veritatis illius fidele.\\
${}^{29}$~Nequissimo in pane murmurabit civitas,\\ et testimonium nequiti\ae\ illius verum est.\\
${}^{30}$~Diligentes in vino noli provocare~:\\ multos enim exterminavit vinum.\\
${}^{31}$~Ignis probat ferrum durum~:\\ sic vinum corda superborum arguet in ebrietate potatum.\\
${}^{32}$~\AE qua vita hominibus vinum in sobrietate~:\\ si bibas illud moderate, eris sobrius.\\
${}^{33}$~Qu\ae\ vita est ei qui minuitur vino~?\\
${}^{34}$~Quid defraudat vitam~? mors.\\
${}^{35}$~Vinum in jucunditatem creatum est,\\ et non in ebrietatem ab initio.\\
${}^{36}$~Exsultatio anim\ae\ et cordis vinum moderate potatum.\\
${}^{37}$~Sanitas est anim\ae\ et corpori sobrius potus.\\
${}^{38}$~Vinum multum potatum irritationem,\\ et iram, et ruinas multas facit.\\
${}^{39}$~Amaritudo anim\ae \\ vinum multum potatum.\\
${}^{40}$~Ebrietatis animositas, imprudentis offensio,\\ minorans virtutem, et faciens vulnera.\\
${}^{41}$~In convivio vini non arguas proximum,\\ et non despicias eum in jucunditate illius.\\
${}^{42}$~Verba improperii non dicas illi,\\ et non premas illum repetendo.\end{verse}\end{flushleft}


\Needspace{2.5\baselineskip}\versal{32}\begin{flushleft}\begin{verse}\vspace{-19pt}\hspace{6pt}Rectorem te posuerunt~? noli extolli~:\\\hspace{6pt} esto in illis quasi unus ex ipsis.\\
${}^{2}$~Curam illorum habe, et sic conside,\\ et omni cura tua explicita recumbe~:\\
${}^{3}$~ut l\ae teris propter illos,\\ et ornamentum grati\ae\ accipias coronam,\\ et dignationem consequaris corrogationis.\\
${}^{4}$~Loquere major natu~:\\ decet enim te\\
${}^{5}$~primum verbum diligenti scientia,\\ et non impedias musicam.\\
${}^{6}$~Ubi auditus non est, non effundas sermonem,\\ et importune noli extolli in sapientia tua.\\
${}^{7}$~Gemmula carbunculi in ornamento auri,\\ et comparatio musicorum in convivio vini.\\
${}^{8}$~Sicut in fabricatione auri signum est smaragdi,\\ sic numerus musicorum in jucundo et moderato vino.\\
${}^{9}$~Audi tacens,\\ et pro reverentia accedet tibi bona gratia.\\
${}^{10}$~Adolescens, loquere in tua causa vix.\\
${}^{11}$~Si bis interrogatus fueris,\\ habeat caput responsum tuum.\\
${}^{12}$~In multis esto quasi inscius,\\ et audi tacens simul et qu\ae rens.\\
${}^{13}$~In medio magnatorum non pr\ae sumas~:\\ et ubi sunt senes non multum loquaris.\\
${}^{14}$~Ante grandinem pr\ae ibit coruscatio~:\\ et ante verecundiam pr\ae ibit gratia,\\ et pro reverentia accedet tibi bona gratia.\\
${}^{15}$~Et hora surgendi non te trices~:\\ pr\ae curre autem prior in domum tuam,\\ et illic avocare, et illic lude,\\
${}^{16}$~et age conceptiones tuas,\\ et non in delictis et verbo superbo~:\\
${}^{17}$~et super his omnibus benedicito Dominum, qui fecit,\\ et inebriantem te ab omnibus bonis suis.\end{verse}\end{flushleft}


\begin{flushleft}\begin{verse}${}^{18}$~Qui timet Dominum excipiet doctrinam ejus~:\\ et qui vigilaverint ad illum invenient benedictionem.\\
${}^{19}$~Qui qu\ae rit legem replebitur ab ea,\\ et qui insidiose agit scandalizabitur in ea.\\
${}^{20}$~Qui timent Dominum invenient judicium justum,\\ et justitias quasi lumen accendent.\\
${}^{21}$~Peccator homo vitabit correptionem,\\ et secundum voluntatem suam inveniet comparationem.\\
${}^{22}$~Vir consilii non disperdet intelligentiam~:\\ alienus et superbus non pertimescet timorem~:\\
${}^{23}$~etiam postquam fecit cum eo sine consilio,\\ et suis insectationibus arguetur.\\
${}^{24}$~Fili, sine consilio nihil facias,\\ et post factum non pœnitebis.\\
${}^{25}$~In via ruin\ae\ non eas, et non offendes in lapides~:\\ nec credas te vi\ae\ laborios\ae , ne ponas anim\ae\ tu\ae\ scandalum.\\
${}^{26}$~Et a filiis tuis cave,\\ et a domesticis tuis attende.\\
${}^{27}$~In omni opere tuo crede ex fide anim\ae\ tu\ae ,\\ hoc est enim conservatio mandatorum.\\
${}^{28}$~Qui credit Deo attendit mandatis~:\\ et qui confidit in illo non minorabitur.\end{verse}\end{flushleft}


\Needspace{2.5\baselineskip}\versal{33}\begin{flushleft}\begin{verse}\vspace{-19pt}\hspace{6pt}Timenti Dominum non occurrent mala~:\\\hspace{6pt} sed in tentatione Deus illum conservabit, et liberabit a malis.\\ Sapiens non odit mandata et justitias,\\
${}^{2}$~et non illidetur quasi in procella navis.\\
${}^{3}$~Homo sensatus credit legi Dei,\\ et lex illi fidelis.\end{verse}\end{flushleft}


\begin{flushleft}\begin{verse}${}^{4}$~Qui interrogationem manifestat parabit verbum,\\ et sic deprecatus exaudietur~:\\ et conservabit disciplinam, et tunc respondebit.\\
${}^{5}$~Pr\ae cordia fatui quasi rota carri,\\ et quasi axis versatilis cogitatus illius.\\
${}^{6}$~Equus emissarius, sic et amicus subsannator~:\\ sub omni supra sedente hinnit.\end{verse}\end{flushleft}


\begin{flushleft}\begin{verse}${}^{7}$~Quare dies diem superat, et iterum lux lucem,\\ et annus annum a sole~?\\
${}^{8}$~A Domini scientia separati sunt,\\ facto sole, et pr\ae ceptum custodiente.\\
${}^{9}$~Et immutavit tempora, et dies festos ipsorum,\\ et in illis dies festos celebraverunt ad horam.\\
${}^{10}$~Ex ipsis exaltavit et magnificavit Deus,\\ et ex ipsis posuit in numerum dierum~:\\ et omnes homines de solo et ex terra unde creatus est Adam.\\
${}^{11}$~In multitudine disciplin\ae\ Dominus separavit eos,\\ et immutavit vias eorum.\\
${}^{12}$~Ex ipsis benedixit et exaltavit,\\ et ex ipsis sanctificavit, et ad se applicavit,\\ et ex ipsis maledixit, et humiliavit,\\ et convertit illos a separatione ipsorum.\\
${}^{13}$~Quasi lutum figuli in manu ipsius,\\ plasmare illud et disponere.\\
${}^{14}$~Omnes vi\ae\ ejus secundum dispositionem ejus~:\\ sic homo in manu illius qui se fecit,\\ et reddet illi secundum judicium suum.\\
${}^{15}$~Contra malum bonum est, et contra mortem vita~:\\ sic et contra virum justum peccator,\\ et sic intuere in omnia opera Altissimi,\\ duo et duo, et unum contra unum.\end{verse}\end{flushleft}


\begin{flushleft}\begin{verse}${}^{16}$~Et ego novissimus evigilavi,\\ et quasi qui colligit acinos post vindemiatores.\\
${}^{17}$~In benedictione Dei et ipse speravi,\\ et quasi qui vindemiat replevi torcular.\\
${}^{18}$~Respicite quoniam non mihi soli laboravi,\\ sed omnibus exquirentibus disciplinam.\\
${}^{19}$~Audite me, magnates et omnes populi~:\\ et rectores ecclesi\ae , auribus percipite.\end{verse}\end{flushleft}


\begin{flushleft}\begin{verse}${}^{20}$~Filio et mulieri, fratri et amico,\\ non des potestatem super te in vita tua~:\\ et non dederis alii possessionem tuam,\\ ne forte pœniteat te, et depreceris pro illis.\\
${}^{21}$~Dum adhuc superes et aspiras,\\ non immutabit te omnis caro.\\
${}^{22}$~Melius est enim ut filii tui te rogent,\\ quam te respicere in manus filiorum tuorum.\\
${}^{23}$~In omnibus operibus tuis pr\ae cellens esto.\\
${}^{24}$~Ne dederis maculam in gloria tua.\\ In die consummationis dierum vit\ae\ tu\ae , et in tempore exitus tui,\\ distribue h\ae reditatem tuam.\end{verse}\end{flushleft}


\begin{flushleft}\begin{verse}${}^{25}$~Cibaria, et virga, et onus asino~:\\ panis, et disciplina, et opus servo.\\
${}^{26}$~Operatur in disciplina, et qu\ae rit requiescere~:\\ laxa manus illi, et qu\ae rit libertatem.\\
${}^{27}$~Jugum et lorum curvant collum durum,\\ et servum inclinant operationes assidu\ae .\\
${}^{28}$~Servo malevolo tortura et compedes~:\\ mitte illum in operationem, ne vacet~:\\
${}^{29}$~multam enim malitiam docuit otiositas.\\
${}^{30}$~In opera constitue eum~:\\ sic enim condecet illum.\\ Quod si non obaudierit, curva illum compedibus,\\ et non amplifices super omnem carnem~:\\ verum sine judicio nihil facias grave.\\
${}^{31}$~Si est tibi servus fidelis, sit tibi quasi anima tua~:\\ quasi fratrem sic eum tracta,\\ quoniam in sanguine anim\ae\ comparasti illum.\\
${}^{32}$~Si l\ae seris eum injuste,\\ in fugam convertetur~:\\
${}^{33}$~et si extollens discesserit,\\ quem qu\ae ras et in qua via qu\ae ras illum nescis.\end{verse}\end{flushleft}


\Needspace{2.5\baselineskip}\versal{34}\begin{flushleft}\begin{verse}\vspace{-19pt}\hspace{6pt}Vana spes et mendacium viro insensato~:\\\hspace{6pt} et somnia extollunt imprudentes.\\
${}^{2}$~Quasi qui apprehendit umbram et persequitur ventum,\\ sic et qui attendit ad visa mendacia.\\
${}^{3}$~Hoc secundum hoc visio somniorum,\\ ante faciem hominis similitudo hominis.\\
${}^{4}$~Ab immundo, quid mundabitur~?\\ et a mendace, quid verum dicetur~?\\
${}^{5}$~Divinatio erroris, et auguria mendacia,\\ et somnia malefacientium, vanitas est~:\\
${}^{6}$~et sicut parturientis, cor tuum phantasias patitur.\\ Nisi ab Altissimo fuerit emissa visitatio,\\ ne dederis in illis cor tuum~:\\
${}^{7}$~multos enim errare fecerunt somnia,\\ et exciderunt sperantes in illis.\\
${}^{8}$~Sine mendacio consummabitur verbum legis,\\ et sapientia in ore fidelis complanabitur.\end{verse}\end{flushleft}


\begin{flushleft}\begin{verse}${}^{9}$~Qui non est tentatus quid scit~?\\ vir in multis expertus cogitabit multa~:\\ et qui multa didicit enarrabit intellectum.\\
${}^{10}$~Qui non est expertus pauca recognoscit~:\\ qui autem in multis factus est, multiplicat malitiam.\\
${}^{11}$~Qui tentatus non est qualia scit~?\\ qui implanatus est abundabit nequitia.\\
${}^{12}$~Multa vidi errando,\\ et plurimas verborum consuetudines.\\
${}^{13}$~Aliquoties usque ad mortem periclitatus sum horum causa,\\ et liberatus sum gratia Dei.\end{verse}\end{flushleft}


\begin{flushleft}\begin{verse}${}^{14}$~Spiritus timentium Deum qu\ae ritur,\\ et in respectu illius benedicetur.\\
${}^{15}$~Spes enim illorum in salvantem illos,\\ et oculi Dei in diligentes se.\\
${}^{16}$~Qui timet Dominum nihil trepidabit~:\\ et non pavebit, quoniam ipse est spes ejus.\\
${}^{17}$~Timentis Dominum, beata est anima ejus.\\
${}^{18}$~Ad quem respicit, et quis est fortitudo ejus~?\\
${}^{19}$~Oculi Domini super timentes eum~:\\ protector potenti\ae , firmamentum virtutis,\\ tegimen ardoris, et umbraculum meridiani~:\\
${}^{20}$~deprecatio offensionis, et adjutorium casus~:\\ exaltans animam, et illuminans oculos,\\ dans sanitatem, et vitam, et benedictionem.\end{verse}\end{flushleft}


\begin{flushleft}\begin{verse}${}^{21}$~Immolantis ex iniquo oblatio est maculata,\\ et non sunt beneplacit\ae\ subsannationes injustorum.\\
${}^{22}$~Dominus solus sustinentibus se\\ in via veritatis et justiti\ae .\\
${}^{23}$~Dona iniquorum non probat Altissimus,\\ nec respicit in oblationes iniquorum,\\ nec in multitudine sacrificiorum eorum propitiabitur peccatis.\\
${}^{24}$~Qui offert sacrificium ex substantia pauperum,\\ quasi qui victimat filium in conspectu patris sui.\\
${}^{25}$~Panis egentium vita pauperum est~:\\ qui defraudat illum homo sanguinis est.\\
${}^{26}$~Qui aufert in sudore panem,\\ quasi qui occidit proximum suum.\\
${}^{27}$~Qui effundit sanguinem,\\ et qui fraudem facit mercenario, fratres sunt.\\
${}^{28}$~Unus \ae dificans, et unus destruens~:\\ quid prodest illis, nisi labor~?\\
${}^{29}$~Unus orans, et unus maledicens~:\\ cujus vocem exaudiet Deus~?\\
${}^{30}$~Qui baptizatur a mortuo, et iterum tangit eum,\\ quid proficit lavatio illius~?\\
${}^{31}$~Sic homo qui jejunat in peccatis suis,\\ et iterum eadem faciens~:\\ quid proficit humiliando se~?\\ orationem illius quis exaudiet~?\end{verse}\end{flushleft}


\Needspace{2.5\baselineskip}\versal{35}\begin{flushleft}\begin{verse}\vspace{-19pt}\hspace{6pt}Qui conservat legem\\\hspace{6pt} multiplicat oblationem.\\
${}^{2}$~Sacrificium salutare est attendere mandatis,\\ et discedere ab omni iniquitate.\\
${}^{3}$~Et propitiationem litare sacrificii super injustitias~:\\ et deprecatio pro peccatis, recedere ab injustitia.\\
${}^{4}$~Retribuet gratiam qui offert similaginem~:\\ et qui facit misericordiam offert sacrificium.\\
${}^{5}$~Beneplacitum est Domino recedere ab iniquitate~:\\ et deprecatio pro peccatis recedere ab injustitia.\\
${}^{6}$~Non apparebis ante conspectum Domini vacuus~:\\
${}^{7}$~h\ae c enim omnia propter mandatum Dei fiunt.\\
${}^{8}$~Oblatio justi impinguat altare,\\ et odor suavitatis est in conspectu Altissimi.\\
${}^{9}$~Sacrificium justi acceptum est,\\ et memoriam ejus non obliviscetur Dominus.\\
${}^{10}$~Bono animo gloriam redde Deo,\\ et non minuas primitias manuum tuarum.\\
${}^{11}$~In omni dato hilarem fac vultum tuum,\\ et in exsultatione sanctifica decimas tuas.\\
${}^{12}$~Da Altissimo secundum datum ejus,\\ et in bono oculo adinventionem facito manuum tuarum,\\
${}^{13}$~quoniam Dominus retribuens est,\\ et septies tantum reddet tibi.\\
${}^{14}$~Noli offerre munera prava,\\ non enim suscipiet illa.\\
${}^{15}$~Et noli inspicere sacrificum injustum,\\ quoniam Dominus judex est,\\ et non est apud illum gloria person\ae .\\
${}^{16}$~Non accipiet Dominus personam in pauperem,\\ et deprecationem l\ae si exaudiet.\\
${}^{17}$~Non despiciet preces pupilli,\\ nec viduam, si effundat loquelam gemitus.\\
${}^{18}$~Nonne lacrim\ae\ vidu\ae\ ad maxillam descendunt,\\ et exclamatio ejus super deducentem eas~?\\
${}^{19}$~A maxilla enim ascendunt usque ad c\ae lum,\\ et Dominus exauditor non delectabitur in illis.\\
${}^{20}$~Qui adorat Deum in oblectatione suscipietur,\\ et deprecatio illius usque ad nubes propinquabit.\end{verse}\end{flushleft}


\begin{flushleft}\begin{verse}${}^{21}$~Oratio humiliantis se nubes penetrabit,\\ et donec propinquet non consolabitur,\\ et non discedet donec Altissimus aspiciat.\\
${}^{22}$~Et Dominus non elongabit~:\\ et judicabit justos, et faciet judicium~:\\ et Fortissimus non habebit in illis patientiam,\\ ut contribulet dorsum ipsorum~:\\
${}^{23}$~et gentibus reddet vindictam,\\ donec tollat plenitudinem superborum,\\ et sceptra iniquorum contribulet~:\\
${}^{24}$~donec reddat hominibus secundum actus suos,\\ et secundum opera Ad\ae , et secundum pr\ae sumptionem illius~:\\
${}^{25}$~donec judicet judicium plebis su\ae ,\\ et oblectabit justos misericordia sua.\\
${}^{26}$~Speciosa misericordia Dei in tempore tribulationis,\\ quasi nubes pluvi\ae\ in tempore siccitatis.\end{verse}\end{flushleft}


\Needspace{2.5\baselineskip}\versal{36}\begin{flushleft}\begin{verse}\vspace{-19pt}\hspace{6pt}Miserere nostri, Deus omnium, et respice nos,\\\hspace{6pt} et ostende nobis lucem miserationum tuarum~:\\
${}^{2}$~et immitte timorem tuum super gentes qu\ae\ non exquisierunt te,\\ ut cognoscant quia non est deus nisi tu,\\ et enarrent magnalia tua.\\
${}^{3}$~Alleva manum tuam super gentes alienas,\\ ut videant potentiam tuam.\\
${}^{4}$~Sicut enim in conspectu eorum sanctificatus es in nobis,\\ sic in conspectu nostro magnificaberis in eis~:\\
${}^{5}$~ut cognoscant te, sicut et nos cognovimus\\ quoniam non est deus pr\ae ter te, Domine.\\
${}^{6}$~Innova signa, et immuta mirabilia.\\
${}^{7}$~Glorifica manum et brachium dextrum.\\
${}^{8}$~Excita furorem, et effunde iram.\\
${}^{9}$~Tolle adversarium, et afflige inimicum.\\
${}^{10}$~Festina tempus, et memento finis,\\ ut enarrent mirabilia tua.\\
${}^{11}$~In ira flamm\ae\ devoretur qui salvatur~:\\ et qui pessimant plebem tuam inveniant perditionem.\\
${}^{12}$~Contere caput principum inimicorum,\\ dicentium~: Non est alius pr\ae ter nos.\\
${}^{13}$~Congrega omnes tribus Jacob,\\ ut cognoscant quia non est deus nisi tu,\\ et enarrent magnalia tua,\\ et h\ae reditabis eos sicut ab initio.\\
${}^{14}$~Miserere plebi tu\ae , super quam invocatum est nomen tuum,\\ et Isra\"el quem co\ae quasti primogenito tuo.\\
${}^{15}$~Miserere civitati sanctificationis tu\ae ,\\ Jerusalem, civitati requiei tu\ae .\\
${}^{16}$~Reple Sion inenarrabilibus verbis tuis,\\ et gloria tua populum tuum.\\
${}^{17}$~Da testimonium his qui ab initio creatur\ae\ tu\ae\ sunt,\\ et suscita pr\ae dicationes quas locuti sunt in nomine tuo prophet\ae\ priores.\\
${}^{18}$~Da mercedem sustinentibus te,\\ ut prophet\ae\ tui fideles inveniantur~:\\ et exaudi orationes servorum tuorum,\\
${}^{19}$~secundum benedictionem Aaron de populo tuo~:\\ et dirige nos in viam justiti\ae ,\\ et sciant omnes qui habitant terram\\ quia tu es Deus conspector s\ae culorum.\end{verse}\end{flushleft}


\begin{flushleft}\begin{verse}${}^{20}$~Omnem escam manducabit venter~:\\ et est cibus cibo melior.\\
${}^{21}$~Fauces contingunt cibum fer\ae ,\\ et cor sensatum verba mendacia.\\
${}^{22}$~Cor pravum dabit tristitiam,\\ et homo peritus resistet illi.\\
${}^{23}$~Omnem masculum excipiet mulier~:\\ et est filia melior filia.\\
${}^{24}$~Species mulieris exhilarat faciem viri sui,\\ et super omnem concupiscentiam hominis superducit desiderium.\\
${}^{25}$~Si est lingua curationis,\\ est et mitigationis et misericordi\ae~:\\ non est vir illius secundum filios hominum.\\
${}^{26}$~Qui possidet mulierem bonam inchoat possessionem~:\\ adjutorium secundum illum est, et columna ut requies.\\
${}^{27}$~Ubi non est sepes, diripietur possessio~:\\ et ubi non est mulier, ingemiscit egens.\\
${}^{28}$~Quis credit ei qui non habet nidum,\\ et deflectens ubicumque obscuraverit,\\ quasi succinctus latro exiliens de civitate in civitatem~?\end{verse}\end{flushleft}


\Needspace{2.5\baselineskip}\versal{37}\begin{flushleft}\begin{verse}\vspace{-19pt}\hspace{6pt}Omnis amicus dicet~: Et ego amicitiam copulavi~:\\\hspace{6pt} sed est amicus solo nomine amicus.\\ Nonne tristitia inest usque ad mortem~?\\
${}^{2}$~sodalis autem et amicus ad inimicitiam convertentur.\\
${}^{3}$~O pr\ae sumptio nequissima, unde creata es\\ cooperire aridam malitia et dolositate illius~?\\
${}^{4}$~Sodalis amico conjucundatur in oblectationibus,\\ et in tempore tribulationis adversarius erit.\\
${}^{5}$~Sodalis amico condolet causa ventris,\\ et contra hostem accipiet scutum.\\
${}^{6}$~Non obliviscaris amici tui in animo tuo,\\ et non immemor sis illius in opibus tuis.\\
${}^{7}$~Noli consiliari cum eo qui tibi insidiatur,\\ et a zelantibus te absconde consilium.\end{verse}\end{flushleft}


\begin{flushleft}\begin{verse}${}^{8}$~Omnis consiliarius prodit consilium,\\ sed est consiliarius in semetipso.\\
${}^{9}$~A consiliario serva animam tuam~:\\ prius scito qu\ae\ sit illius necessitas~:\\ et ipse enim animo suo cogitabit~:\\
${}^{10}$~ne forte mittat sudem in terram,\\ et dicat tibi~:\\
${}^{11}$~Bona est via tua~:\\ et stet e contrario videre quid tibi eveniat.\\
${}^{12}$~Cum viro irreligioso tracta de sanctitate,\\ et cum injusto de justitia,\\ et cum muliere de ea qu\ae\ \ae mulatur,\\ cum timido de bello,\\ cum negotiatore de trajectione,\\ cum emptore de venditione,\\ cum viro livido de gratiis agendis,\\
${}^{13}$~cum impio de pietate,\\ cum inhonesto de honestate,\\ cum operario agrario de omni opere,\\
${}^{14}$~cum operario annuali de consummatione anni,\\ cum servo pigro de multa operatione.\\ Non attendas his in omni consilio~:\\
${}^{15}$~sed cum viro sancto assiduus esto,\\ quemcumque cognoveris observantem timorem Dei~:\\
${}^{16}$~cujus anima est secundum animam tuam,\\ et qui, cum titubaveris in tenebris, condolebit tibi.\\
${}^{17}$~Cor boni consilii statue tecum~:\\ non est enim tibi aliud pluris illo.\\
${}^{18}$~Anima viri sancti enuntiat aliquando vera,\\ quam septem circumspectores sedentes in excelso ad speculandum.\\
${}^{19}$~Et in his omnibus deprecare Altissimum,\\ ut dirigat in veritate viam tuam.\end{verse}\end{flushleft}


\begin{flushleft}\begin{verse}${}^{20}$~Ante omnia opera verbum verax pr\ae cedat te,\\ et ante omnem actum consilium stabile.\\
${}^{21}$~Verbum nequam immutabit cor~:\\ ex quo partes quatuor oriuntur~:\\ bonum et malum, vita et mors~:\\ et dominatrix illorum est assidua lingua.\\ Est vir astutus multorum eruditor,\\ et anim\ae\ su\ae\ inutilis est.\\
${}^{22}$~Vir peritus multos erudivit,\\ et anim\ae\ su\ae\ suavis est.\\
${}^{23}$~Qui sophistice loquitur odibilis est~:\\ in omni re defraudabitur.\\
${}^{24}$~Non est illi data a Domino gratia,\\ omni enim sapientia defraudatus est.\\
${}^{25}$~Est sapiens anim\ae\ su\ae\ sapiens,\\ et fructus sensus illius laudabilis.\\
${}^{26}$~Vir sapiens plebem suam erudit,\\ et fructus sensus illius fideles sunt.\\
${}^{27}$~Vir sapiens implebitur benedictionibus,\\ et videntes illum laudabunt.\\
${}^{28}$~Vita viri in numero dierum~:\\ dies autem Isra\"el innumerabiles sunt.\\
${}^{29}$~Sapiens in populo h\ae reditabit honorem,\\ et nomen illius erit vivens in \ae ternum.\end{verse}\end{flushleft}


\begin{flushleft}\begin{verse}${}^{30}$~Fili, in vita tua tenta animam tuam,\\ et si fuerit nequam non des illi potestatem~:\\
${}^{31}$~non enim omnia omnibus expediunt,\\ et non omni anim\ae\ omne genus placet.\\
${}^{32}$~Noli avidus esse in omni epulatione,\\ et non te effundas super omnem escam~:\\
${}^{33}$~in multis enim escis erit infirmitas,\\ et aviditas appropinquabit usque ad choleram.\\
${}^{34}$~Propter crapulam multi obierunt~:\\ qui autem abstinens est adjiciet vitam.\end{verse}\end{flushleft}


\Needspace{2.5\baselineskip}\versal{38}\begin{flushleft}\begin{verse}\vspace{-19pt}\hspace{6pt}Honora medicum propter necessitatem~:\\\hspace{6pt} etenim illum creavit Altissimus.\\
${}^{2}$~A Deo est enim omnis medela,\\ et a rege accipiet donationem.\\
${}^{3}$~Disciplina medici exaltabit caput illius,\\ et in conspectu magnatorum collaudabitur.\\
${}^{4}$~Altissimus creavit de terra medicamenta,\\ et vir prudens non abhorrebit illa.\\
${}^{5}$~Nonne a ligno indulcata est aqua amara~?\\
${}^{6}$~Ad agnitionem hominum virtus illorum~:\\ et dedit hominibus scientiam Altissimus,\\ honorari in mirabilibus suis.\\
${}^{7}$~In his curans mitigabit dolorem~:\\ et unguentarius faciet pigmenta suavitatis,\\ et unctiones conficiet sanitatis~:\\ et non consummabuntur opera ejus.\\
${}^{8}$~Pax enim Dei super faciem terr\ae .\\
${}^{9}$~Fili, in tua infirmitate ne despicias teipsum~:\\ sed ora Dominum, et ipse curabit te.\\
${}^{10}$~Averte a delicto, et dirige manus,\\ et ab omni delicto munda cor tuum.\\
${}^{11}$~Da suavitatem et memoriam similaginis,\\ et impingua oblationem, et da locum medico~:\\
${}^{12}$~etenim illum Dominus creavit, et non discedat a te,\\ quia opera ejus sunt necessaria.\\
${}^{13}$~Est enim tempus quando in manus illorum incurras~:\\
${}^{14}$~ipsi vero Dominum deprecabuntur, ut dirigat requiem eorum,\\ et sanitatem, propter conversationem illorum.\\
${}^{15}$~Qui delinquit in conspectu ejus qui fecit eum,\\ incidet in manus medici.\end{verse}\end{flushleft}


\begin{flushleft}\begin{verse}${}^{16}$~Fili, in mortuum produc lacrimas,\\ et quasi dira passus incipe plorare~:\\ et secundum judicium contege corpus illius,\\ et non despicias sepulturam illius.\\
${}^{17}$~Propter delaturam autem amare fer luctum illius uno die,\\ et consolare propter tristitiam~:\\
${}^{18}$~et fac luctum secundum meritum ejus\\ uno die, vel duobus, propter detractionem~:\\
${}^{19}$~a tristitia enim festinat mors, et cooperit virtutem,\\ et tristitia cordis flectit cervicem.\\
${}^{20}$~In abductione permanet tristitia,\\ et substantia inopis secundum cor ejus.\\
${}^{21}$~Ne dederis in tristitia cor tuum,\\ sed repelle eam a te, et memento novissimorum.\\
${}^{22}$~Noli oblivisci, neque enim est conversio~:\\ et huic nihil proderis, et teipsum pessimabis.\\
${}^{23}$~Memor esto judicii mei~: sic enim erit et tuum~:\\ mihi heri, et tibi hodie.\\
${}^{24}$~In requie mortui requiescere fac memoriam ejus,\\ et consolare illum in exitu spiritus sui.\end{verse}\end{flushleft}


\begin{flushleft}\begin{verse}${}^{25}$~Sapientia scrib\ae\ in tempore vacuitatis,\\ et qui minoratur actu sapientiam percipiet,\\ qua sapientia replebitur.\\
${}^{26}$~Qui tenet aratrum,\\ et qui gloriatur in jaculo, stimulo boves agitat,\\ et conversatur in operibus eorum,\\ et enarratio ejus in filiis taurorum.\\
${}^{27}$~Cor suum dabit ad versandos sulcos,\\ et vigilia ejus in sagina vaccarum.\\
${}^{28}$~Sic omnis faber et architectus,\\ qui noctem tamquam diem transigit~:\\ qui sculpit signacula sculptilia,\\ et assiduitas ejus variat picturam~:\\ cor suum dabit in similitudinem pictur\ae ,\\ et vigilia sua perficiet opus.\\
${}^{29}$~Sic faber ferrarius sedens juxta incudem,\\ et considerans opus ferri~:\\ vapor ignis uret carnes ejus,\\ et in calore fornacis concertatur.\\
${}^{30}$~Vox mallei innovat aurem ejus,\\ et contra similitudinem vasis oculus ejus.\\
${}^{31}$~Cor suum dabit in consummationem operum,\\ et vigilia sua ornabit in perfectionem.\\
${}^{32}$~Sic figulus sedens ad opus suum,\\ convertens pedibus suis rotam,\\ qui in sollicitudine positus est semper propter opus suum,\\ et in numero est omnis operatio ejus.\\
${}^{33}$~In brachio suo formabit lutum,\\ et ante pedes suos curvabit virtutem suam.\\
${}^{34}$~Cor suum dabit ut consummet linitionem,\\ et vigilia sua mundabit fornacem.\\
${}^{35}$~Omnes hi in manibus suis speraverunt,\\ et unusquisque in arte sua sapiens est.\\
${}^{36}$~Sine his omnibus non \ae dificatur civitas,\\
${}^{37}$~et non inhabitabunt, nec inambulabunt,\\ et in ecclesiam non transilient.\\
${}^{38}$~Super sellam judicis non sedebunt,\\ et testamentum judicii non intelligent,\\ neque palam facient disciplinam et judicium,\\ et in parabolis non invenientur~:\\
${}^{39}$~sed creaturam \ae vi confirmabunt~:\\ et deprecatio illorum in operatione artis,\\ accomodantes animam suam,\\ et conquirentes in lege Altissimi.\end{verse}\end{flushleft}


\Needspace{2.5\baselineskip}\versal{39}\begin{flushleft}\begin{verse}\vspace{-19pt}\hspace{6pt}Sapientiam omnium antiquorum exquiret sapiens,\\\hspace{6pt} et in prophetis vacabit.\\
${}^{2}$~Narrationem virorum nominatorum conservabit,\\ et in versutias parabolarum simul introibit.\\
${}^{3}$~Occulta proverbiorum exquiret,\\ et in absconditis parabolarum conversabitur.\\
${}^{4}$~In medio magnatorum ministrabit,\\ et in conspectu pr\ae sidis apparebit.\\
${}^{5}$~In terram alienigenarum gentium pertransiet~:\\ bona enim et mala in hominibus tentabit.\\
${}^{6}$~Cor suum tradet ad vigilandum diluculo ad Dominum, qui fecit illum,\\ et in conspectu Altissimi deprecabitur.\\
${}^{7}$~Aperiet os suum in oratione,\\ et pro delictis suis deprecabitur.\\
${}^{8}$~Si enim Dominus magnus voluerit,\\ spiritu intelligenti\ae\ replebit illum~:\\
${}^{9}$~et ipse tamquam imbres mittet eloquia sapienti\ae\ su\ae ,\\ et in oratione confitebitur Domino~:\\
${}^{10}$~et ipse diriget consilium ejus, et disciplinam,\\ et in absconditis suis consiliabitur.\\
${}^{11}$~Ipse palam faciet disciplinam doctrin\ae\ su\ae ,\\ et in lege testamenti Domini gloriabitur.\\
${}^{12}$~Collaudabunt multi sapientiam ejus,\\ et usque in s\ae culum non delebitur.\\
${}^{13}$~Non recedet memoria ejus,\\ et nomen ejus requiretur a generatione in generationem.\\
${}^{14}$~Sapientiam ejus enarrabunt gentes,\\ et laudem ejus enuntiabit ecclesia.\\
${}^{15}$~Si permanserit, nomen derelinquet plus quam mille~:\\ et si requieverit, proderit illi.\end{verse}\end{flushleft}


\begin{flushleft}\begin{verse}${}^{16}$~Adhuc consiliabor ut enarrem~:\\ ut furore enim repletus sum.\\
${}^{17}$~In voce dicit~: Obaudite me, divini fructus,\\ et quasi rosa plantata super rivos aquarum fructificate.\\
${}^{18}$~Quasi Libanus odorem suavitatis habete.\\
${}^{19}$~Florete flores quasi lilium~:\\ et date odorem, et frondete in gratiam~:\\ et collaudate canticum, et benedicite Dominum in operibus suis.\\
${}^{20}$~Date nomini ejus magnificentiam,\\ et confitemini illi in voce labiorum vestrorum,\\ et in canticis labiorum, et citharis~:\\ et sic dicetis in confessione~:\\
${}^{21}$~Opera Domini universa bona valde.\\
${}^{22}$~In verbo ejus stetit aqua sicut congeries~:\\ et in sermone oris illius sicut exceptoria aquarum~:\\
${}^{23}$~quoniam in pr\ae cepto ipsius placor fit,\\ et non est minoratio in salute ipsius.\\
${}^{24}$~Opera omnis carnis coram illo,\\ et non est quidquam absconditum ab oculis ejus.\\
${}^{25}$~A s\ae culo usque in s\ae culum respicit,\\ et nihil est mirabile in conspectu ejus.\\
${}^{26}$~Non est dicere~: Quid est hoc, aut quid est istud~?\\ omnia enim in tempore suo qu\ae rentur.\\
${}^{27}$~Benedictio illius quasi fluvius inundavit.\\
${}^{28}$~Quomodo cataclysmus aridam inebriavit,\\ sic ira ipsius gentes qu\ae\ non exquisierunt eum h\ae reditabit.\\
${}^{29}$~Quomodo convertit aquas in siccitatem, et siccata est terra,\\ et vi\ae\ illius viis illorum direct\ae\ sunt,\\ sic peccatoribus offensiones in ira ejus.\\
${}^{30}$~Bona bonis creata sunt ab initio~:\\ sic nequissimis bona et mala.\\
${}^{31}$~Initium necessari\ae\ rei vit\ae\ hominum, aqua, ignis, et ferrum,\\ sal, lac, et panis similagineus, et mel,\\ et botrus uv\ae , et oleum, et vestimentum.\\
${}^{32}$~H\ae c omnia sanctis in bona,\\ sic et impiis et peccatoribus in mala convertentur.\\
${}^{33}$~Sunt spiritus qui ad vindictam creati sunt,\\ et in furore suo confirmaverunt tormenta sua.\\
${}^{34}$~In tempore consummationis effundent virtutem,\\ et furorem ejus qui fecit illos placabunt.\\
${}^{35}$~Ignis, grando, fames, et mors,\\ omnia h\ae c ad vindictam creata sunt~:\\
${}^{36}$~bestiarum dentes, et scorpii, et serpentes,\\ et rhomph\ae a vindicans in exterminium impios.\\
${}^{37}$~In mandatis ejus epulabuntur~:\\ et super terram in necessitatem pr\ae parabuntur,\\ et in temporibus suis non pr\ae terient verbum.\\
${}^{38}$~Propterea ab initio confirmatus sum, et consiliatus sum,\\ et cogitavi, et scripta dimisi.\\
${}^{39}$~Omnia opera Domini bona,\\ et omne opus hora sua subministrabit.\\
${}^{40}$~Non est dicere~: Hoc illo nequius est~:\\ omnia enim in tempore suo comprobabuntur.\\
${}^{41}$~Et nunc in omni corde et ore collaudate,\\ et benedicite nomen Domini.\end{verse}\end{flushleft}


\Needspace{2.5\baselineskip}\versal{40}\begin{flushleft}\begin{verse}\vspace{-19pt}\hspace{6pt}Occupatio magna creata est omnibus hominibus,\\\hspace{6pt} et jugum grave super filios Adam,\\ a die exitus de ventre matris eorum\\ usque in diem sepultur\ae\ in matrem omnium.\\
${}^{2}$~Cogitationes eorum, et timores cordis,\\ adinventio exspectationis, et dies finitionis,\\
${}^{3}$~a residente super sedem gloriosam,\\ usque ad humiliatum in terra et cinere~:\\
${}^{4}$~ab eo qui utitur hyacintho et portat coronam,\\ usque ad eum qui operitur lino crudo~:\\ furor, zelus, tumultus, fluctuatio, et timor mortis,\\ iracundia perseverans, et contentio~:\\
${}^{5}$~et in tempore refectionis in cubili,\\ somnus noctis immutat scientiam ejus.\\
${}^{6}$~Modicum tamquam nihil in requie,\\ et ab eo in somnis, quasi in die respectus.\\
${}^{7}$~Conturbatus est in visu cordis sui,\\ tamquam qui evaserit in die belli~:\\ in tempore salutis su\ae\ exsurrexit,\\ et admirans ad nullum timorem~:\\
${}^{8}$~cum omni carne, ab homine usque ad pecus,\\ et super peccatores septuplum.\\
${}^{9}$~Ad h\ae c mors, sanguis, contentio, et rhomph\ae a,\\ oppressiones, fames, et contritio, et flagella~:\\
${}^{10}$~super iniquos creata sunt h\ae c omnia~:\\ et propter illos factus est cataclysmus.\end{verse}\end{flushleft}


\begin{flushleft}\begin{verse}${}^{11}$~Omnia qu\ae\ de terra sunt in terram convertentur,\\ et omnes aqu\ae\ in mare revertentur.\\
${}^{12}$~Omne munus et iniquitas delebitur,\\ et fides in s\ae culum stabit.\\
${}^{13}$~Substanti\ae\ injustorum sicut fluvius siccabuntur,\\ et sicut tonitruum magnum in pluvia personabunt.\\
${}^{14}$~In aperiendo manus suas l\ae tabitur~:\\ sic pr\ae varicatores in consummatione tabescent.\\
${}^{15}$~Nepotes impiorum non multiplicabunt ramos~:\\ et radices immund\ae\ super cacumen petr\ae\ sonant.\\
${}^{16}$~Super omnem aquam viriditas,\\ et ad oram fluminis ante omne fœnum evelletur.\end{verse}\end{flushleft}


\begin{flushleft}\begin{verse}${}^{17}$~Gratia sicut paradisus in benedictionibus,\\ et misericordia in s\ae culum permanet.\\
${}^{18}$~Vita sibi sufficientis operarii condulcabitur,\\ et in ea invenies thesaurum.\\
${}^{19}$~Filii et \ae dificatio civitatis confirmabit nomen~:\\ et super h\ae c mulier immaculata computabitur.\\
${}^{20}$~Vinum et musica l\ae tificant cor~:\\ et super utraque dilectio sapienti\ae .\\
${}^{21}$~Tibi\ae\ et psalterium suavem faciunt melodiam~:\\ et super utraque lingua suavis.\\
${}^{22}$~Gratiam et speciem desiderabit oculus tuus~:\\ et super h\ae c virides sationes.\\
${}^{23}$~Amicus et sodalis in tempore convenientes,\\ et super utrosque mulier cum viro.\\
${}^{24}$~Fratres in adjutorium in tempore tribulationis~:\\ et super eos misericordia liberabit.\\
${}^{25}$~Aurum et argentum est constitutio pedum~:\\ et super utrumque consilium beneplacitum.\\
${}^{26}$~Facultates et virtutes exaltant cor,\\ et super h\ae c timor Domini.\\
${}^{27}$~Non est in timore Domini minoratio~:\\ et non est in eo inquirere adjutorium.\\
${}^{28}$~Timor Domini sicut paradisus benedictionis,\\ et super omnem gloriam operuerunt illum.\end{verse}\end{flushleft}


\begin{flushleft}\begin{verse}${}^{29}$~Fili, in tempore vit\ae\ tu\ae\ ne indigeas~:\\ melius est enim mori quam indigere.\\
${}^{30}$~Vir respiciens in mensam alienam,\\ non est vita ejus in cogitatione victus~:\\ alit enim animam suam cibis alienis~:\\
${}^{31}$~vir autem disciplinatus et eruditus custodiet se.\\
${}^{32}$~In ore imprudentis condulcabitur inopia,\\ et in ventre ejus ignis ardebit.\end{verse}\end{flushleft}


\Needspace{2.5\baselineskip}\versal{41}\begin{flushleft}\begin{verse}\vspace{-19pt}\hspace{6pt}O mors, quam amara est memoria tua\\\hspace{6pt} homini pacem habenti in substantiis suis~:\\
${}^{2}$~viro quieto, et cujus vi\ae\ direct\ae\ sunt in omnibus,\\ et adhuc valenti accipere cibum~!\\
${}^{3}$~O mors, bonum est judicium tuum homini indigenti,\\ et qui minoratur viribus,\\
${}^{4}$~defecto \ae tate, et cui de omnibus cura est,\\ et incredibili, qui perdit patientiam~!\\
${}^{5}$~Noli metuere judicium mortis~:\\ memento qu\ae\ ante te fuerunt,\\ et qu\ae\ superventura sunt tibi~:\\ hoc judicium a Domino omni carni.\\
${}^{6}$~Et quid superveniet tibi in beneplacito Altissimi~?\\ sive decem, sive centum, sive mille anni~:\\
${}^{7}$~non est enim in inferno accusatio vit\ae .\end{verse}\end{flushleft}


\begin{flushleft}\begin{verse}${}^{8}$~Filii abominationum fiunt filii peccatorum,\\ et qui conversantur secus domos impiorum.\\
${}^{9}$~Filiorum peccatorum periet h\ae reditas,\\ et cum semine illorum assiduitas opprobrii.\\
${}^{10}$~De patre impio queruntur filii,\\ quoniam propter illum sunt in opprobrio.\\
${}^{11}$~V\ae\ vobis, viri impii,\\ qui dereliquistis legem Domini Altissimi~!\\
${}^{12}$~Et si nati fueritis, in maledictione nascemini~:\\ et si mortui fueritis, in maledictione erit pars vestra.\\
${}^{13}$~Omnia qu\ae\ de terra sunt in terram convertentur~:\\ sic impii a maledicto in perditionem.\\
${}^{14}$~Luctus hominum in corpore ipsorum~:\\ nomen autem impiorum delebitur.\\
${}^{15}$~Curam habe de bono nomine~:\\ hoc enim magis permanebit tibi\\ quam mille thesauri pretiosi et magni.\\
${}^{16}$~Bon\ae\ vit\ae\ numerus dierum~:\\ bonum autem nomen permanebit in \ae vum.\end{verse}\end{flushleft}


\begin{flushleft}\begin{verse}${}^{17}$~Disciplinam in pace conservate, filii~:\\ sapientia enim abscondita, et thesaurus invisus,\\ qu\ae\ utilitas in utrisque~?\\
${}^{18}$~Melior est homo qui abscondit stultitiam suam,\\ quam homo qui abscondit sapientiam suam.\\
${}^{19}$~Verumtamen reveremini in his qu\ae\ procedunt de ore meo~:\\
${}^{20}$~non est enim bonum omnem reverentiam observare,\\ et non omnia omnibus bene placent in fide.\\
${}^{21}$~Erubescite a patre et a matre de fornicatione~:\\ et a pr\ae sidente et a potente de mendacio~:\\
${}^{22}$~a principe et a judice de delicto~:\\ a synagoga et plebe de iniquitate~:\\
${}^{23}$~a socio et amico de injustitia,\\ et de loco in quo habitas~:\\
${}^{24}$~de furto, de veritate Dei, et testamento~:\\ de discubitu in panibus, et ab obfuscatione dati et accepti~:\\
${}^{25}$~a salutantibus de silentio,\\ a respectu mulieris fornicari\ae ,\\ et ab aversione vultus cognati.\\
${}^{26}$~Ne avertas faciem a proximo tuo,\\ et ab auferendo partem et non restituendo.\\
${}^{27}$~Ne respicias mulierem alieni viri,\\ et ne scruteris ancillam ejus,\\ neque steteris ad lectum ejus.\\
${}^{28}$~Ab amicis de sermonibus improperii~:\\ et cum dederis, ne improperes.\end{verse}\end{flushleft}


\Needspace{2.5\baselineskip}\versal{42}\begin{flushleft}\begin{verse}\vspace{-19pt}\hspace{6pt}Non duplices sermonem auditus de revelatione sermonis absconditi~:\\\hspace{6pt} et eris vere sine confusione,\\ et invenies gratiam in conspectu omnium hominum.\\ Ne pro his omnibus confundaris,\\ et ne accipias personam ut delinquas~:\\
${}^{2}$~de lege Altissimi, et testamento,\\ et de judicio justificare impium,\\
${}^{3}$~de verbo sociorum et viatorum,\\ et de datione h\ae reditatis amicorum,\\
${}^{4}$~de \ae qualitate stater\ae\ et ponderum,\\ de acquisitione multorum et paucorum,\\
${}^{5}$~de corruptione emptionis et negotiatorum,\\ et de multa disciplina filiorum,\\ et servo pessimo latus sanguinare.\\
${}^{6}$~Super mulierem nequam bonum est signum.\\
${}^{7}$~Ubi manus mult\ae\ sunt, claude~:\\ et quodcumque trades, numera et appende~:\\ datum vero et acceptum omne describe.\\
${}^{8}$~De disciplina insensati et fatui,\\ et de senioribus qui judicantur ab adolescentibus~:\\ et eris eruditus in omnibus,\\ et probabilis in conspectu omnium vivorum.\end{verse}\end{flushleft}


\begin{flushleft}\begin{verse}${}^{9}$~Filia patris abscondita est vigilia,\\ et sollicitudo ejus aufert somnum~:\\ ne forte in adolescentia sua adulta efficiatur,\\ et cum viro commorata odibilis fiat~:\\
${}^{10}$~nequando polluatur in virginitate sua,\\ et in paternis suis gravida inveniatur~:\\ ne forte cum viro commorata transgrediatur,\\ aut certe sterilis efficiatur.\\
${}^{11}$~Super filiam luxuriosam confirma custodiam,\\ nequando faciat te in opprobrium venire inimicis,\\ a detractione in civitate, et objectione plebis,\\ et confundat te in multitudine populi.\\
${}^{12}$~Omni homini noli intendere in specie,\\ et in medio mulierum noli commorari~:\\
${}^{13}$~de vestimentis enim procedit tinea,\\ et a muliere iniquitas viri.\\
${}^{14}$~Melior est enim iniquitas viri quam mulier benefaciens,\\ et mulier confundens in opprobrium.\end{verse}\end{flushleft}


\begin{flushleft}\begin{verse}${}^{15}$~Memor ero igitur operum Domini,\\ et qu\ae\ vidi annuntiabo.\\ In sermonibus Domini opera ejus.\\
${}^{16}$~Sol illuminans per omnia respexit,\\ et gloria Domini plenum est opus ejus.\\
${}^{17}$~Nonne Dominus fecit sanctos enarrare omnia mirabilia sua,\\ qu\ae\ confirmavit Dominus omnipotens stabiliri in gloria sua~?\\
${}^{18}$~Abyssum et cor hominum investigavit,\\ et in astutia eorum excogitavit.\\
${}^{19}$~Cognovit enim Dominus omnem scientiam,\\ et inspexit in signum \ae vi,\\ annuntians qu\ae\ pr\ae terierunt et qu\ae\ superventura sunt,\\ revelans vestigia occultorum.\\
${}^{20}$~Non pr\ae terit illum omnis cogitatus,\\ et non abscondit se ab eo ullus sermo.\\
${}^{21}$~Magnalia sapienti\ae\ su\ae\ decoravit,\\ qui est ante s\ae culum et usque in s\ae culum~:\\ neque adjectum est,\\
${}^{22}$~neque minuitur,\\ et non eget alicujus consilio.\\
${}^{23}$~Quam desiderabilia omnia opera ejus~!\\ et tamquam scintilla qu\ae\ est considerare~!\\
${}^{24}$~Omnia h\ae c vivunt, et manent in s\ae culum,\\ et in omni necessitate omnia obaudiunt ei.\\
${}^{25}$~Omnia duplicia, unum contra unum,\\ et non fecit quidquam deesse.\\
${}^{26}$~Uniuscujusque confirmavit bona~:\\ et quis satiabitur videns gloriam ejus~?\end{verse}\end{flushleft}


\Needspace{2.5\baselineskip}\versal{43}\begin{flushleft}\begin{verse}\vspace{-19pt}\hspace{6pt}Altitudinis firmamentum pulchritudo ejus est,\\\hspace{6pt} species c\ae li in visione glori\ae .\\
${}^{2}$~Sol in aspectu annuntians in exitu,\\ vas admirabile, opus Excelsi.\\
${}^{3}$~In meridiano exurit terram,\\ et in conspectu ardoris ejus quis poterit sustinere~?\\ fornacem custodiens in operibus ardoris~:\\
${}^{4}$~tripliciter sol exurens montes,\\ radios igneos exsufflans,\\ et refulgens radiis suis obc\ae cat oculos.\\
${}^{5}$~Magnus Dominus qui fecit illum,\\ et in sermonibus ejus festinavit iter.\end{verse}\end{flushleft}


\begin{flushleft}\begin{verse}${}^{6}$~Et luna in omnibus in tempore suo,\\ ostensio temporis, et signum \ae vi.\\
${}^{7}$~A luna signum diei festi~:\\ luminare quod minuitur in consummatione.\\
${}^{8}$~Mensis secundum nomen ejus est,\\ crescens mirabiliter in consummatione.\\
${}^{9}$~Vas castrorum in excelsis,\\ in firmamento c\ae li resplendens gloriose.\\
${}^{10}$~Species c\ae li gloria stellarum~:\\ mundum illuminans in excelsis Dominus.\\
${}^{11}$~In verbis Sancti stabunt ad judicium,\\ et non deficient in vigiliis suis.\end{verse}\end{flushleft}


\begin{flushleft}\begin{verse}${}^{12}$~Vide arcum, et benedic eum qui fecit illum~:\\ valde speciosus est in splendore suo.\\
${}^{13}$~Gyravit c\ae lum in circuitu glori\ae\ su\ae~:\\ manus Excelsi aperuerunt illum.\\
${}^{14}$~Imperio suo acceleravit nivem,\\ et accelerat coruscationes emittere judicii sui.\\
${}^{15}$~Propterea aperti sunt thesauri,\\ et evolaverunt nebul\ae\ sicut aves.\\
${}^{16}$~In magnitudine sua posuit nubes,\\ et confracti sunt lapides grandinis.\\
${}^{17}$~In conspectu ejus commovebuntur montes,\\ et in voluntate ejus aspirabit notus.\\
${}^{18}$~Vox tonitrui ejus verberavit terram,\\ tempestas aquilonis, et congregatio spiritus~:\\
${}^{19}$~et sicut avis deponens ad sedendum, aspergit nivem,\\ et sicut locusta demergens descensus ejus.\\
${}^{20}$~Pulchritudinem candoris ejus admirabitur oculus,\\ et super imbrem ejus expavescet cor.\\
${}^{21}$~Gelu sicut salem effundet super terram~:\\ et dum gelaverit, fiet tamquam cacumina tribuli.\\
${}^{22}$~Frigidus ventus aquilo flavit,\\ et gelavit crystallus ab aqua~:\\ super omnem congregationem aquarum requiescet,\\ et sicut lorica induet se aquis~:\\
${}^{23}$~et devorabit montes, et exuret desertum,\\ et extinguet viride, sicut igne.\\
${}^{24}$~Medicina omnium in festinatione nebul\ae~:\\ et ros obvians ab ardore venienti humilem efficiet eum.\\
${}^{25}$~In sermone ejus siluit ventus,\\ et cogitatione sua placavit abyssum~:\\ et plantavit in illa Dominus insulas.\\
${}^{26}$~Qui navigant mare enarrent pericula ejus,\\ et audientes auribus nostris admirabimur.\\
${}^{27}$~Illic pr\ae clara opera et mirabilia,\\ varia bestiarum genera, et omnium pecorum, et creatura belluarum.\\
${}^{28}$~Propter ipsum confirmatus est itineris finis,\\ et in sermone ejus composita sunt omnia.\end{verse}\end{flushleft}


\begin{flushleft}\begin{verse}${}^{29}$~Multa dicemus, et deficiemus in verbis~:\\ consummatio autem sermonum ipse est in omnibus.\\
${}^{30}$~Gloriantes ad quid valebimus~?\\ ipse enim omnipotens super omnia opera sua.\\
${}^{31}$~Terribilis Dominus, et magnus vehementer,\\ et mirabilis potentia ipsius.\\
${}^{32}$~Glorificantes Dominum quantumcumque potueritis,\\ supervalebit enim adhuc~: et admirabilis magnificentia ejus.\\
${}^{33}$~Benedicentes Dominum, exaltate illum quantum potestis~:\\ major enim est omni laude.\\
${}^{34}$~Exaltantes eum, replemini virtute, ne laboretis,\\ non enim comprehendetis.\\
${}^{35}$~Quis videbit eum et enarrabit~?\\ et quis magnificabit eum sicut est ab initio~?\\
${}^{36}$~Multa abscondita sunt majora his~:\\ pauca enim vidimus operum ejus.\\
${}^{37}$~Omnia autem Dominus fecit,\\ et pie agentibus dedit sapientiam.\end{verse}\end{flushleft}


\Needspace{2.5\baselineskip}\versal{44}\begin{flushleft}\begin{verse}\vspace{-19pt}\hspace{6pt}Laudemus viros gloriosos,\\\hspace{6pt} et parentes nostros in generatione sua.\\
${}^{2}$~Multam gloriam fecit Dominus~:\\ magnificentia sua a s\ae culo.\\
${}^{3}$~Dominantes in potestatibus suis,\\ homines magni virtute et prudentia sua pr\ae diti,\\ nuntiantes in prophetis dignitatem prophetarum~:\\
${}^{4}$~et imperantes in pr\ae senti populo,\\ et virtute prudenti\ae\ populis sanctissima verba~:\\
${}^{5}$~in peritia sua requirentes modos musicos,\\ et narrantes carmina Scripturarum~:\\
${}^{6}$~homines divites in virtute,\\ pulchritudinis studium habentes,\\ pacificantes in domibus suis.\\
${}^{7}$~Omnes isti in generationibus gentis su\ae\ gloriam adepti sunt,\\ et in diebus suis habentur in laudibus.\\
${}^{8}$~Qui de illis nati sunt reliquerunt nomen\\ narrandi laudes eorum.\\
${}^{9}$~Et sunt quorum non est memoria~:\\ perierunt quasi qui non fuerint~:\\ et nati sunt quasi non nati,\\ et filii ipsorum cum ipsis.\\
${}^{10}$~Sed illi viri misericordi\ae\ sunt,\\ quorum pietates non defuerunt.\\
${}^{11}$~Cum semine eorum permanent bona~:\\
${}^{12}$~h\ae reditas sancta nepotes eorum,\\ et in testamentis stetit semen eorum~:\\
${}^{13}$~et filii eorum propter illos usque in \ae ternum manent~:\\ semen eorum et gloria eorum non derelinquetur.\\
${}^{14}$~Corpora ipsorum in pace sepulta sunt,\\ et nomen eorum vivit in generationem et generationem.\\
${}^{15}$~Sapientiam ipsorum narrent populi,\\ et laudem eorum nuntiet ecclesia.\end{verse}\end{flushleft}


\begin{flushleft}\begin{verse}${}^{16}$~Enoch placuit Deo, et translatus est in paradisum,\\ ut det gentibus pœnitentiam.\\
${}^{17}$~No\"e inventus est perfectus, justus,\\ et in tempore iracundi\ae\ factus est reconciliatio.\\
${}^{18}$~Ideo dimissum est reliquum terr\ae ,\\ cum factum est diluvium.\\
${}^{19}$~Testamenta s\ae culi posita sunt apud illum,\\ ne deleri possit diluvio omnis caro.\end{verse}\end{flushleft}


\begin{flushleft}\begin{verse}${}^{20}$~Abraham magnus pater multitudinis gentium,\\ et non est inventus similis illi in gloria~:\\ qui conservavit legem Excelsi,\\ et fuit in testamento cum illo.\\
${}^{21}$~In carne ejus stare fecit testamentum,\\ et in tentatione inventus est fidelis.\\
${}^{22}$~Ideo jurejurando dedit illi gloriam in gente sua,\\ crescere illum quasi terr\ae\ cumulum,\\
${}^{23}$~et ut stellas exaltare semen ejus,\\ et h\ae reditare illos a mari usque ad mare,\\ et a flumine usque ad terminos terr\ae .\\
${}^{24}$~Et in Isaac eodem modo fecit,\\ propter Abraham patrem ejus.\\
${}^{25}$~Benedictionem omnium gentium dedit illi Dominus,\\ et testamentum confirmavit super caput Jacob.\\
${}^{26}$~Agnovit eum in benedictionibus suis,\\ et dedit illi h\ae reditatem,\\ et divisit illi partem in tribubus duodecim.\\
${}^{27}$~Et conservavit illi homines misericordi\ae ,\\ invenientes gratiam in oculis omnis carnis.\end{verse}\end{flushleft}


\Needspace{2.5\baselineskip}\versal{45}\begin{flushleft}\begin{verse}\vspace{-19pt}\hspace{6pt}Dilectus Deo et hominibus Moyses,\\\hspace{6pt} cujus memoria in benedictione est.\\
${}^{2}$~Similem illum fecit in gloria sanctorum,\\ et magnificavit eum in timore inimicorum,\\ et in verbis suis monstra placavit.\\
${}^{3}$~Glorificavit illum in conspectu regum,\\ et jussit illi coram populo suo,\\ et ostendit illi gloriam suam.\\
${}^{4}$~In fide et lenitate ipsius sanctum fecit illum,\\ et elegit eum ex omni carne.\\
${}^{5}$~Audivit enim eum, et vocem ipsius,\\ et induxit illum in nubem.\\
${}^{6}$~Et dedit illi coram pr\ae cepta,\\ et legem vit\ae\ et disciplin\ae ,\\ docere Jacob testamentum suum,\\ et judicia sua Isra\"el.\end{verse}\end{flushleft}


\begin{flushleft}\begin{verse}${}^{7}$~Excelsum fecit Aaron fratrem ejus,\\ et similem sibi, de tribu Levi.\\
${}^{8}$~Statuit ei testamentum \ae ternum,\\ et dedit illi sacerdotium gentis,\\ et beatificavit illum in gloria~:\\
${}^{9}$~et circumcinxit eum zona glori\ae ,\\ et induit eum stolam glori\ae ,\\ et coronavit eum in vasis virtutis.\\
${}^{10}$~Circumpedes, et femoralia, et humerale posuit ei~:\\ et cinxit illum tintinnabulis aureis plurimis in gyro~:\\
${}^{11}$~dare sonitum in incessu suo,\\ auditum facere sonitum in templo\\ in memoriam filiis gentis su\ae .\\
${}^{12}$~Stolam sanctam auro, et hyacintho, et purpura,\\ opus textile viri sapientis, judicio et veritate pr\ae diti~:\\
${}^{13}$~torto cocco opus artificis\\ gemmis pretiosis figuratis in ligatura auri,\\ et opere lapidarii sculptis,\\ in memoriam secundum numerum tribuum Isra\"el.\\
${}^{14}$~Corona aurea super mitram ejus\\ expressa signo sanctitatis, et gloria honoris~:\\ opus virtutis, et desideria oculorum ornata.\\
${}^{15}$~Sic pulchra ante ipsum non fuerunt talia\\ usque ad originem.\\
${}^{16}$~Non est indutus illa alienigena aliquis,\\ sed tantum filii ipsius soli,\\ et nepotes ejus per omne tempus.\\
${}^{17}$~Sacrificia ipsius\\ consumpta sunt igne quotidie.\\
${}^{18}$~Complevit Moyses manus ejus,\\ et unxit illum oleo sancto.\\
${}^{19}$~Factum est illi in testamentum \ae ternum,\\ et semini ejus, sicut dies c\ae li,\\ fungi sacerdotio, et habere laudem,\\ et glorificare populum suum in nomine ejus.\\
${}^{20}$~Ipsum elegit ab omni vivente,\\ offerre sacrificium Deo, incensum, et bonum odorem,\\ in memoriam placare pro populo suo~:\\
${}^{21}$~et dedit illi in pr\ae ceptis suis potestatem,\\ in testamentis judiciorum~:\\ docere Jacob testimonia,\\ et in lege sua lucem dare Isra\"el.\\
${}^{22}$~Quia contra illum steterunt alieni,\\ et propter invidiam circumdederunt illum homines in deserto,\\ qui erant cum Dathan et Abiron,\\ et congregatio Core in iracundia.\\
${}^{23}$~Vidit Dominus Deus, et non placuit illi,\\ et consumpti sunt in impetu iracundi\ae .\\
${}^{24}$~Fecit illis monstra,\\ et consumpsit illos in flamma ignis.\\
${}^{25}$~Et addidit Aaron gloriam,\\ et dedit illi h\ae reditatem,\\ et primitias frugum terr\ae\ divisit illi.\\
${}^{26}$~Panem ipsis in primis paravit in satietatem~:\\ nam et sacrificia Domini edent,\\ qu\ae\ dedit illi et semini ejus.\\
${}^{27}$~Ceterum in terra gentes non h\ae reditabit,\\ et pars non est illi in gente~:\\ ipse est enim pars ejus, et h\ae reditas.\end{verse}\end{flushleft}


\begin{flushleft}\begin{verse}${}^{28}$~Phinees, filius Eleazari, tertius in gloria est,\\ imitando eum in timore Domini,\\
${}^{29}$~et stare in reverentia gentis~:\\ in bonitate et alacritate anim\ae\ su\ae\ placuit Deo pro Isra\"el.\\
${}^{30}$~Ideo statuit illi testamentum pacis,\\ principem sanctorum et gentis su\ae ,\\ ut sit illi et semini ejus sacerdotii dignitas in \ae ternum.\\
${}^{31}$~Et testamentum David regi filio Jesse de tribu Juda,\\ h\ae reditas ipsi et semini ejus~:\\ ut daret sapientiam in cor nostrum,\\ judicare gentem suam in justitia,\\ ne abolerentur bona ipsorum~:\\ et gloriam ipsorum in gentem eorum \ae ternam fecit.\end{verse}\end{flushleft}


\Needspace{2.5\baselineskip}\versal{46}\begin{flushleft}\begin{verse}\vspace{-19pt}\hspace{6pt}Fortis in bello Jesus Nave, successor Moysi in prophetis,\\\hspace{6pt} qui fuit magnus secundum nomen suum,\\
${}^{2}$~maximus in salutem electorum Dei,\\ expugnare insurgentes hostes,\\ ut consequeretur h\ae reditatem Isra\"el.\\
${}^{3}$~Quam gloriam adeptus est in tollendo manus suas,\\ et jactando contra civitates rhomph\ae as~!\\
${}^{4}$~Quis ante illum sic restitit~?\\ nam hostes ipse Dominus perduxit.\\
${}^{5}$~An non in iracundia ejus impeditus est sol,\\ et una dies facta est quasi duo~?\\
${}^{6}$~Invocavit Altissimum potentem,\\ in oppugnando inimicos undique~:\\ et audivit illum magnus et sanctus Deus,\\ in saxis grandinis virtutis valde fortis.\\
${}^{7}$~Impetum fecit contra gentem hostilem,\\ et in descensu perdidit contrarios~:\\
${}^{8}$~ut cognoscant gentes potentiam ejus,\\ quia contra Deum pugnare non est facile.\\ Et secutus est a tergo potentis~:\\
${}^{9}$~et in diebus Moysi misericordiam fecit,\\ ipse, et Caleb filius Jephone,\\ stare contra hostem, et prohibere gentem a peccatis,\\ et perfringere murmur maliti\ae .\\
${}^{10}$~Et ipsi duo constituti a periculo liberati sunt\\ a numero sexcentorum millium peditum,\\ inducere illos in h\ae reditatem,\\ in terram qu\ae\ manat lac et mel.\\
${}^{11}$~Et dedit Dominus ipsi Caleb fortitudinem,\\ et usque in senectutem permansit illi virtus,\\ ut ascenderet in excelsum terr\ae\ locum,\\ et semen ipsius obtinuit h\ae reditatem,\\
${}^{12}$~ut viderent omnes filii Isra\"el\\ quia bonum est obsequi sancto Deo.\end{verse}\end{flushleft}


\begin{flushleft}\begin{verse}${}^{13}$~Et judices singuli suo nomine, quorum non est corruptum cor,\\ qui non aversi sunt a Domino,\\
${}^{14}$~ut sit memoria illorum in benedictione,\\ et ossa eorum pullulent de loco suo~:\\
${}^{15}$~et nomen eorum permaneat in \ae ternum,\\ permanens ad filios illorum, sanctorum virorum gloria.\end{verse}\end{flushleft}


\begin{flushleft}\begin{verse}${}^{16}$~Dilectus a Domino Deo suo Samuel, propheta Domini,\\ renovavit imperium,\\ et unxit principes in gente sua.\\
${}^{17}$~In lege Domini congregationem judicavit,\\ et vidit Deus Jacob~:\\ et in fide sua probatus est propheta,\\
${}^{18}$~et cognitus est in verbis suis fidelis,\\ quia vidit Deum lucis.\\
${}^{19}$~Et invocavit Dominum omnipotentem,\\ in oppugnando hostes circumstantes undique,\\ in oblatione agni inviolati.\\
${}^{20}$~Et intonuit de c\ae lo Dominus,\\ et in sonitu magno auditam fecit vocem suam~:\\
${}^{21}$~et contrivit principes Tyriorum,\\ et omnes duces Philisthiim~:\\
${}^{22}$~et ante tempus finis vit\ae\ su\ae\ et s\ae culi,\\ testimonium pr\ae buit in conspectu Domini et christi~:\\ pecunias et usque ad calceamenta ab omni carne non accepit,\\ et non accusavit illum homo.\\
${}^{23}$~Et post hoc dormivit~: et notum fecit regi,\\ et ostendit illi finem vit\ae\ su\ae~:\\ et exaltavit vocem suam de terra in prophetia,\\ delere impietatem gentis.\end{verse}\end{flushleft}


\Needspace{2.5\baselineskip}\versal{47}\begin{flushleft}\begin{verse}\vspace{-19pt}\hspace{6pt}Post h\ae c surrexit Nathan,\\\hspace{6pt} propheta in diebus David.\\
${}^{2}$~Et quasi adeps separatus a carne,\\ sic David a filiis Isra\"el.\\
${}^{3}$~Cum leonibus lusit quasi cum agnis,\\ et in ursis similiter fecit sicut in agnis ovium, in juventute sua.\\
${}^{4}$~Numquid non occidit gigantem,\\ et abstulit opprobrium de gente~?\\
${}^{5}$~In tollendo manum,\\ saxo fund\ae\ dejecit exsultationem Goli\ae~:\\
${}^{6}$~nam invocavit Dominum omnipotentem,\\ et dedit in dextera ejus tollere hominem fortem in bello,\\ et exaltare cornu gentis su\ae .\\
${}^{7}$~Sic in decem millibus glorificavit eum~:\\ et laudavit eum in benedictionibus Domini,\\ in offerendo illi coronam glori\ae~:\\
${}^{8}$~contrivit enim inimicos undique,\\ et extirpavit Philisthiim contrarios usque in hodiernum diem~:\\ contrivit cornu ipsorum usque in \ae ternum.\\
${}^{9}$~In omni opere dedit confessionem Sancto,\\ et Excelso in verbo glori\ae .\\
${}^{10}$~De omni corde suo laudavit Dominum~:\\ et dilexit Deum, qui fecit illum,\\ et dedit illi contra inimicos potentiam~:\\
${}^{11}$~et stare fecit cantores contra altare,\\ et in sono eorum dulces fecit modos.\\
${}^{12}$~Et dedit in celebrationibus decus,\\ et ornavit tempora usque ad consummationem vit\ae ,\\ ut laudarent nomen sanctum Domini,\\ et amplificarent mane Dei sanctitatem.\\
${}^{13}$~Dominus purgavit peccata ipsius,\\ et exaltavit in \ae ternum cornu ejus~:\\ et dedit illi testamentum regni,\\ et sedem glori\ae\ in Isra\"el.\end{verse}\end{flushleft}


\begin{flushleft}\begin{verse}${}^{14}$~Post ipsum surrexit filius sensatus,\\ et propter illum dejecit omnem potentiam inimicorum.\\
${}^{15}$~Salomon imperavit in diebus pacis,\\ cui subjecit Deus omnes hostes,\\ ut conderet domum in nomine suo,\\ et pararet sanctitatem in sempiternum.\\ Quemadmodum eruditus es in juventute tua,\\
${}^{16}$~et impletus es, quasi flumen, sapientia,\\ et terram retexit anima tua.\\
${}^{17}$~Et replesti in comparationibus \ae nigmata~:\\ ad insulas longe divulgatum est nomen tuum,\\ et dilectus es in pace tua.\\
${}^{18}$~In cantilenis, et proverbiis,\\ et comparationibus, et interpretationibus, mirat\ae\ sunt terr\ae~:\\
${}^{19}$~et in nomine Domini Dei,\\ cui est cognomen Deus Isra\"el.\\
${}^{20}$~Collegisti quasi auricalcum aurum,\\ et ut plumbum complesti argentum~:\\
${}^{21}$~et inclinasti femora tua mulieribus~:\\ potestatem habuisti in corpore tuo.\\
${}^{22}$~Dedisti maculam in gloria tua,\\ et profanasti semen tuum,\\ inducere iracundiam ad liberos tuos,\\ et incitari stultitiam tuam~:\\
${}^{23}$~ut faceres imperium bipartitum,\\ et ex Ephraim imperare imperium durum.\\
${}^{24}$~Deus autem non derelinquet misericordiam suam~:\\ et non corrumpet, nec delebit opera sua,\\ neque perdet a stirpe nepotes electi sui,\\ et semen ejus qui diligit Dominum non corrumpet.\\
${}^{25}$~Dedit autem reliquum Jacob,\\ et David de ipsa stirpe.\\
${}^{26}$~Et finem habuit Salomon cum patribus suis.\end{verse}\end{flushleft}


\begin{flushleft}\begin{verse}${}^{27}$~Et dereliquit post se de semine suo, gentis stultitiam,\\
${}^{28}$~et imminutum a prudentia, Roboam,\\ qui avertit gentem consilio suo~:\\
${}^{29}$~et Jeroboam filium Nabat, qui peccare fecit Isra\"el,\\ et dedit viam peccandi Ephraim~:\\ et plurima redundaverunt peccata ipsorum.\\
${}^{30}$~Valde averterunt illos a terra sua.\\
${}^{31}$~Et qu\ae sivit omnes nequitias,\\ usque dum perveniret ad illos defensio,\\ et ab omnibus peccatis liberavit eos.\end{verse}\end{flushleft}


\Needspace{2.5\baselineskip}\versal{48}\begin{flushleft}\begin{verse}\vspace{-19pt}\hspace{6pt}Et surrexit Elias propheta quasi ignis,\\\hspace{6pt} et verbum ipsius quasi facula ardebat.\\
${}^{2}$~Qui induxit in illos famem~:\\ et irritantes illum invidia sua pauci facti sunt~:\\ non enim poterant sustinere pr\ae cepta Domini.\\
${}^{3}$~Verbo Domini continuit c\ae lum,\\ et dejecit de c\ae lo ignem ter.\\
${}^{4}$~Sic amplificatus est Elias in mirabilibus suis.\\ Et quis potest similiter sic gloriari tibi~?\\
${}^{5}$~qui sustulisti mortuum ab inferis de sorte mortis,\\ in verbo Domini Dei~:\\
${}^{6}$~qui dejecisti reges ad pernicem,\\ et confregisti facile potentiam ipsorum,\\ et gloriosos de lecto suo~:\\
${}^{7}$~qui audis in Sina judicium,\\ et in Horeb judicia defensionis~:\\
${}^{8}$~qui ungis reges ad pœnitentiam,\\ et prophetas facis successores post te~:\\
${}^{9}$~qui receptus es in turbine ignis,\\ in curru equorum igneorum~:\\
${}^{10}$~qui scriptus es in judiciis temporum,\\ lenire iracundiam Domini,\\ conciliare cor patris ad filium,\\ et restituere tribus Jacob.\\
${}^{11}$~Beati sunt qui te viderunt,\\ et in amicitia tua decorati sunt.\\
${}^{12}$~Nam nos vita vivimus tantum~:\\ post mortem autem non erit tale nomen nostrum.\end{verse}\end{flushleft}


\begin{flushleft}\begin{verse}${}^{13}$~Elias quidem in turbine tectus est,\\ et in Eliseo completus est spiritus ejus~:\\ in diebus suis non pertimuit principem,\\ et potentia nemo vicit illum~:\\
${}^{14}$~nec superavit illum verbum aliquod,\\ et mortuum prophetavit corpus ejus.\\
${}^{15}$~In vita sua fecit monstra,\\ et in morte mirabilia operatus est.\end{verse}\end{flushleft}


\begin{flushleft}\begin{verse}${}^{16}$~In omnibus istis non pœnituit populus,\\ et non recesserunt a peccatis suis,\\ usque dum ejecti sunt de terra sua,\\ et dispersi sunt in omnem terram~:\\
${}^{17}$~et relicta est gens perpauca,\\ et princeps in domo David.\\
${}^{18}$~Quidam ipsorum fecerunt quod placeret Deo~:\\ alii autem multa commiserunt peccata.\\
${}^{19}$~Ezechias munivit civitatem suam,\\ et induxit in medium ipsius aquam~:\\ et fodit ferro rupem,\\ et \ae dificavit ad aquam puteum.\\
${}^{20}$~In diebus ipsius ascendit Sennacherib,\\ et misit Rabsacen, et sustulit manum suam contra illos~:\\ et extulit manum suam in Sion,\\ et superbus factus est potentia sua.\\
${}^{21}$~Tunc mota sunt corda et manus ipsorum~:\\ et doluerunt quasi parturientes mulieres.\\
${}^{22}$~Et invocaverunt Dominum misericordem,\\ et expandentes manus suas extulerunt ad c\ae lum~:\\ et Sanctus, Dominus Deus, audivit cito vocem ipsorum.\\
${}^{23}$~Non est commemoratus peccatorum illorum,\\ neque dedit illos inimicis suis~:\\ sed purgavit eos in manu Isai\ae\ sancti prophet\ae .\\
${}^{24}$~Dejecit castra Assyriorum,\\ et contrivit illos angelus Domini~:\\
${}^{25}$~nam fecit Ezechias quod placuit Deo,\\ et fortiter ivit in via David patris sui,\\ quam mandavit illi Isaias, propheta magnus,\\ et fidelis in conspectu Dei.\\
${}^{26}$~In diebus ipsius retro rediit sol,\\ et addidit regi vitam.\\
${}^{27}$~Spiritu magno vidit ultima,\\ et consolatus est lugentes in Sion\\ usque in sempiternum.\\
${}^{28}$~Ostendit futura,\\ et abscondita antequam evenirent.\end{verse}\end{flushleft}


\Needspace{2.5\baselineskip}\versal{49}\begin{flushleft}\begin{verse}\vspace{-19pt}\hspace{6pt}Memoria Josi\ae\ in compositionem odoris facta\\\hspace{6pt} opus pigmentarii.\\
${}^{2}$~In omni ore quasi mel indulcabitur ejus memoria,\\ et ut musica in convivio vini.\\
${}^{3}$~Ipse est directus divinitus in pœnitentiam gentis,\\ et tulit abominationes impietatis.\\
${}^{4}$~Et gubernavit ad Dominum cor ipsius,\\ et in diebus peccatorum corroboravit pietatem.\end{verse}\end{flushleft}


\begin{flushleft}\begin{verse}${}^{5}$~Pr\ae ter David et Ezechiam et Josiam,\\ omnes peccatum commiserunt~:\\
${}^{6}$~nam reliquerunt legem Altissimi reges Juda,\\ et contempserunt timorem Dei.\\
${}^{7}$~Dederunt enim regnum suum aliis,\\ et gloriam suam alienigen\ae\ genti.\\
${}^{8}$~Incenderunt electam sanctitatis civitatem,\\ et desertas fecerunt vias ipsius in manu Jeremi\ae .\\
${}^{9}$~Nam male tractaverunt illum\\ qui a ventre matris consecratus est propheta,\\ evertere, et eruere, et perdere,\\ et iterum \ae dificare, et renovare~:\\
${}^{10}$~Ezechiel, qui vidit conspectum glori\ae \\ quam ostendit illi in curru cherubim.\\
${}^{11}$~Nam commemoratus est inimicorum in imbre,\\ benefacere illis qui ostenderunt rectas vias.\\
${}^{12}$~Et duodecim prophetarum ossa pullulent de loco suo~:\\ nam corroboraverunt Jacob,\\ et redemerunt se in fide virtutis.\end{verse}\end{flushleft}


\begin{flushleft}\begin{verse}${}^{13}$~Quomodo amplificemus Zorobabel~?\\ nam et ipse quasi signum in dextera manu~:\\
${}^{14}$~sic et Jesum filium Josedec,\\ qui in diebus suis \ae dificaverunt domum,\\ et exaltaverunt templum sanctum Domino,\\ paratum in gloriam sempiternam.\\
${}^{15}$~Et Nehemias in memoriam multi temporis,\\ qui erexit nobis muros eversos,\\ et stare fecit portas et seras,\\ qui erexit domos nostras.\end{verse}\end{flushleft}


\begin{flushleft}\begin{verse}${}^{16}$~Nemo natus est in terra qualis Henoch,\\ nam et ipse receptus est a terra~:\\
${}^{17}$~neque ut Joseph, qui natus est homo princeps fratrum,\\ firmamentum gentis, rector fratrum, stabilimentum populi~:\\
${}^{18}$~et ossa ipsius visitata sunt,\\ et post mortem prophetaverunt.\\
${}^{19}$~Seth et Sem apud homines gloriam adepti sunt,\\ et super omnem animam in origine Adam.\end{verse}\end{flushleft}


\Needspace{2.5\baselineskip}\versal{50}\begin{flushleft}\begin{verse}\vspace{-19pt}\hspace{6pt}Simon, Oni\ae\ filius, sacerdos magnus,\\\hspace{6pt} qui in vita sua suffulsit domum,\\ et in diebus suis corroboravit templum.\\
${}^{2}$~Templi etiam altitudo ab ipso fundata est,\\ duplex \ae dificatio, et excelsi parietes templi.\\
${}^{3}$~In diebus ipsius emanaverunt putei aquarum,\\ et quasi mare adimpleti sunt supra modum.\\
${}^{4}$~Qui curavit gentem suam,\\ et liberavit eam a perditione~:\\
${}^{5}$~qui pr\ae valuit amplificare civitatem,\\ qui adeptus est gloriam in conversatione gentis,\\ et ingressum domus et atrii amplificavit.\\
${}^{6}$~Quasi stella matutina in medio nebul\ae ,\\ et quasi luna plena, in diebus suis lucet~:\\
${}^{7}$~et quasi sol refulgens,\\ sic ille effulsit in templo Dei.\\
${}^{8}$~Quasi arcus refulgens inter nebulas glori\ae ,\\ et quasi flos rosarum in diebus vernis,\\ et quasi lilia qu\ae\ sunt in transitu aqu\ae ,\\ et quasi thus redolens in diebus \ae statis~:\\
${}^{9}$~quasi ignis effulgens,\\ et thus ardens in igne~:\\
${}^{10}$~quasi vas auri solidum,\\ ornatum omni lapide pretioso~:\\
${}^{11}$~quasi oliva pullulans, et cypressus in altitudinem se extollens,\\ in accipiendo ipsum stolam glori\ae ,\\ et vestiri eum in consummationem virtutis.\\
${}^{12}$~In ascensu altaris sancti\\ gloriam dedit sanctitatis amictum.\\
${}^{13}$~In accipiendo autem partes de manu sacerdotum,\\ et ipse stans juxta aram~:\\ et circa illum corona fratrum~:\\ quasi plantatio cedri in monte Libano,\\
${}^{14}$~sic circa illum steterunt quasi rami palm\ae~:\\ et omnes filii Aaron in gloria sua.\\
${}^{15}$~Oblatio autem Domini in manibus ipsorum\\ coram omni synagoga Isra\"el~:\\ et consummatione fungens in ara,\\ amplificare oblationem excelsi Regis,\\
${}^{16}$~porrexit manum suam in libatione,\\ et libavit de sanguine uv\ae .\\
${}^{17}$~Effudit in fundamento altaris\\ odorem divinum excelso Principi.\\
${}^{18}$~Tunc exclamaverunt filii Aaron,\\ in tubis productilibus sonuerunt~:\\ et auditam fecerunt vocem magnam in memoriam coram Deo.\\
${}^{19}$~Tunc omnis populus simul properaverunt,\\ et ceciderunt in faciem super terram,\\ adorare Dominum Deum suum,\\ et dare preces omnipotenti Deo excelso.\\
${}^{20}$~Et amplificaverunt psallentes in vocibus suis,\\ et in magna domo auctus est sonus suavitatis plenus.\\
${}^{21}$~Et rogavit populus Dominum excelsum in prece,\\ usque dum perfectus est honor Domini,\\ et munus suum perfecerunt.\\
${}^{22}$~Tunc descendens, manus suas extulit\\ in omne congregationem filiorum Isra\"el,\\ dare gloriam Deo a labiis suis,\\ et in nomine ipsius gloriari~:\\
${}^{23}$~et iteravit orationem suam,\\ volens ostendere virtutem Dei.\\
${}^{24}$~Et nunc orate Deum omnium, qui magna fecit in omni terra,\\ qui auxit dies nostros a ventre matris nostr\ae ,\\ et fecit nobiscum secundum suam misericordiam~:\\
${}^{25}$~det nobis jucunditatem cordis,\\ et fieri pacem in diebus nostris in Isra\"el per dies sempiternos~:\\
${}^{26}$~credere Isra\"el nobiscum esse Dei misericordiam,\\ ut liberet nos in diebus suis.\end{verse}\end{flushleft}


\begin{flushleft}\begin{verse}${}^{27}$~Duas gentes odit anima mea~:\\ tertia autem non est gens quam oderim~:\\
${}^{28}$~qui sedent in monte Seir, et Philisthiim,\\ et stultus populus qui habitat in Sichimis.\end{verse}\end{flushleft}


\begin{flushleft}\begin{verse}${}^{29}$~Doctrinam sapienti\ae\ et disciplin\ae\ scripsit in codice isto\\ Jesus, filius Sirach, Jerosolymita,\\ qui renovavit sapientiam de corde suo.\\
${}^{30}$~Beatus qui in istis versatur bonis~:\\ qui ponit illa in corde suo, sapiens erit semper.\\
${}^{31}$~Si enim h\ae c fecerit, ad omnia valebit, quia lux Dei vestigium ejus est.\end{verse}\end{flushleft}



\bchapter
\lettrine[lines=3,image=true,loversize=0.05,lraise=-0.03]{O}{}ratio Jesu filii Sirach. \begin{flushleft}\begin{verse}\vspace{6pt}Confitebor tibi, Domine rex,\\ et collaudabo te Deum salvatorem meum.\\
${}^{2}$~Confitebor nomini tuo,\\ quoniam adjutor et protector factus es mihi,\\
${}^{3}$~et liberasti corpus meum a perditione~:\\ a laqueo lingu\ae\ iniqu\ae , et a labiis operantium mendacium~:\\ et in conspectu astantium factus es mihi adjutor.\\
${}^{4}$~Et liberasti me, secundum multitudinem misericordi\ae\ nominis tui,\\ a rugientibus pr\ae paratis ad escam~:\\
${}^{5}$~de manibus qu\ae rentium animam meam,\\ et de portis tribulationum qu\ae\ circumdederunt me~;\\
${}^{6}$~a pressura flamm\ae\ qu\ae\ circumdedit me,\\ et in medio ignis non sum \ae stuatus~;\\
${}^{7}$~de altitudine ventris inferi,\\ et a lingua coinquinata, et a verbo mendacii,\\ a rege iniquo, et a lingua injusta.\\
${}^{8}$~Laudabit usque ad mortem anima mea Dominum,\\
${}^{9}$~et vita mea appropinquans erat in inferno deorsum.\\
${}^{10}$~Circumdederunt me undique, et non erat qui adjuvaret~:\\ respiciens eram ad adjutorium hominum, et non erat.\\
${}^{11}$~Memoratus sum misericordi\ae\ tu\ae\ Domine,\\ et operationis tu\ae , qu\ae\ a s\ae culo sunt~:\\
${}^{12}$~quoniam eruis sustinentes te, Domine,\\ et liberas eos de manibus gentium.\\
${}^{13}$~Exaltasti super terram habitationem meam,\\ et pro morte defluente deprecatus sum.\\
${}^{14}$~Invocavi Dominum patrem Domini mei,\\ ut non derelinquat me in die tribulationis me\ae ,\\ et in tempore superborum, sine adjutorio.\\
${}^{15}$~Laudabo nomen tuum assidue,\\ et collaudabo illud in confessione~:\\ et exaudita est oratio mea,\\
${}^{16}$~et liberasti me de perditione,\\ et eripuisti me de tempore iniquo.\\
${}^{17}$~Propterea confitebor, et laudem dicam tibi,\\ et benedicam nomini Domini.\end{verse}\end{flushleft}


\begin{flushleft}\begin{verse}${}^{18}$~Cum adhuc junior essem, priusquam oberrarem,\\ qu\ae sivi sapientiam palam in oratione mea.\\
${}^{19}$~Ante templum postulabam pro illis,\\ et usque in novissimis inquiram eam~:\\ et effloruit tamquam pr\ae cox uva.\\
${}^{20}$~L\ae tatum est cor meum in ea~:\\ ambulavit pes meus iter rectum~:\\ a juventute mea investigabam eam.\\
${}^{21}$~Inclinavi modico aurem meam,\\ et excepi illam.\\
${}^{22}$~Multam inveni in meipso sapientiam,\\ et multum profeci in ea.\\
${}^{23}$~Danti mihi sapientiam dabo gloriam~:\\
${}^{24}$~consiliatus sum enim ut facerem illam.\\ Zelatus sum bonum, et non confundar.\\
${}^{25}$~Colluctata est anima mea in illa,\\ et in faciendo eam confirmatus sum.\\
${}^{26}$~Manus meas extendi in altum,\\ et insipientiam ejus luxi~;\\
${}^{27}$~animam meam direxi ad illam,\\ et in agnitione inveni eam.\\
${}^{28}$~Possedi cum ipsa cor ab initio~:\\ propter hoc, non derelinquar.\\
${}^{29}$~Venter meus conturbatus est qu\ae rendo illam~:\\ propterea bonam possidebo possessionem.\\
${}^{30}$~Dedit mihi Dominus linguam mercedem meam,\\ et in ipsa laudabo eum.\\
${}^{31}$~Appropiate ad me, indocti,\\ et congregate vos in domum disciplin\ae .\\
${}^{32}$~Quid adhuc retardatis~? et quid dicitis in his~?\\ anim\ae\ vestr\ae\ sitiunt vehementer.\\
${}^{33}$~Aperui os meum, et locutus sum~:\\ Comparate vobis sine argento,\\
${}^{34}$~et collum vestrum subjicite jugo~:\\ et suscipiat anima vestra disciplinam~:\\ in proximo est enim invenire eam.\\
${}^{35}$~Videte oculis vestris, quia modicum laboravi,\\ et inveni mihi multam requiem.\\
${}^{36}$~Assumite disciplinam in multo numero argenti,\\ et copiosum aurum possidete in ea.\\
${}^{37}$~L\ae tetur anima vestra in misericordia ejus,\\ et non confundemini in laude ipsius.\\
${}^{38}$~Operamini opus vestrum ante tempus,\\ et dabit vobis mercedem vestram in tempore suo.\end{verse}\end{flushleft}


