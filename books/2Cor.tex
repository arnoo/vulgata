\bbook{Epistola B. Pauli Apostoli ad Corinthios Secunda}
{ad Corinthios II}{images/genese_heading}


\bchapter{1}
\lettrine[lines=10,image=true,loversize=0.05,lraise=-0.03]{P}{}aulus, Apostolus Jesu Christi per voluntatem Dei, et Timotheus frater, ecclesi\ae\ Dei, qu\ae\ est Corinthi cum omnibus sanctis, qui sunt in universa Achaia.
${}^{2}$~Gratia vobis, et pax a Deo Patre nostro, et Domino Jesu Christo.


${}^{3}$~Benedictus Deus et Pater Domini nostri Jesu Christi, Pater misericordiarum, et Deus totius consolationis,
${}^{4}$~qui consolatur nos in omni tribulatione nostra~: ut possimus et ipsi consolari eos qui in omni pressura sunt, per exhortationem, qua exhortamur et ipsi a Deo.
${}^{5}$~Quoniam sicut abundant passiones Christi in nobis~: ita et per Christum abundat consolatio nostra.
${}^{6}$~Sive autem tribulamur pro vestra exhortatione et salute, sive consolamur pro vestra consolatione, sive exhortamur pro vestra exhortatione et salute, qu\ae\ operatur tolerantiam earumdem passionum, quas et nos patimur~:
${}^{7}$~ut spes nostra firma sit pro vobis~: scientes quod sicut socii passionum estis, sic eritis et consolationis.
${}^{8}$~Non enim volumus ignorare vos, fratres, de tribulatione nostra, qu\ae\ facta est in Asia, quoniam supra modum gravati sumus supra virtutem, ita ut t\ae deret nos etiam vivere.
${}^{9}$~Sed ipsi in nobismetipsis responsum mortis habuimus, ut non simus fidentes in nobis, sed in Deo, qui suscitat mortuos~:
${}^{10}$~qui de tantis periculis nos eripuit, et eruit~: in quem speramus quoniam et adhuc eripiet,
${}^{11}$~adjuvantibus et vobis in oratione pro nobis~: ut ex multorum personis, ejus qu\ae\ in nobis est donationis, per multos grati\ae\ agantur pro nobis.


${}^{12}$~Nam gloria nostra h\ae c est~: testimonium conscienti\ae\ nostr\ae , quod in simplicitate cordis et sinceritate Dei, et non in sapientia carnali, sed in gratia Dei, conversati sumus in hoc mundo~: abundantius autem ad vos.
${}^{13}$~Non enim alia scribimus vobis, quam qu\ae\ legistis, et cognovistis. Spero autem quod usque in finem cognoscetis,
${}^{14}$~sicut et cognovistis nos ex parte, quod gloria vestra sumus, sicut et vos nostra, in die Domini nostri Jesu Christi.
${}^{15}$~Et hac confidentia volui prius venire ad vos, ut secundam gratiam haberetis~:
${}^{16}$~et per vos transire in Macedoniam, et iterum a Macedonia venire ad vos, et a vobis deduci in Jud\ae am.
${}^{17}$~Cum ergo hoc voluissem, numquid levitate usus sum~? aut qu\ae\ cogito, secundum carnem cogito, ut sit apud me Est et Non~?
${}^{18}$~Fidelis autem Deus, quia sermo noster, qui fuit apud vos, non est in illo Est et Non.
${}^{19}$~Dei enim Filius Jesus Christus, qui in vobis per nos pr\ae dicatus est, per me, et Silvanum, et Timotheum, non fuit Est et Non, sed Est in illo fuit.
${}^{20}$~Quotquot enim promissiones Dei sunt, in illo Est~: ideo et per ipsum Amen Deo ad gloriam nostram.
${}^{21}$~Qui autem confirmat nos vobiscum in Christo, et qui unxit nos Deus~:
${}^{22}$~qui et signavit nos, et dedit pignus Spiritus in cordibus nostris.


${}^{23}$~Ego autem testem Deum invoco in animam meam, quod parcens vobis, non veni ultra Corinthum~: non quia dominamur fidei vest\ae , sed adjutores sumus gaudii vestri~: nam fide statis.

\bchapter{2}
\lettrine[lines=3,image=true,loversize=0.05,lraise=-0.03]{S}{}tatui autem hoc ipsum apud me, ne iterum in tristitia venirem ad vos.
${}^{2}$~Si enim ego contristo vos~: et quis est, qui me l\ae tificet, nisi qui contristatur ex me~?
${}^{3}$~Et hoc ipsum scripsi vobis, ut non cum venero, tristitiam super tristitiam habeam, de quibus oportuerat me gaudere~: confidens in omnibus vobis, quia meum gaudium, omnium vestrum est.
${}^{4}$~Nam ex multa tribulatione et angustia cordis scripsi vobis per multas lacrimas~: non ut contristemini, sed ut sciatis, quam caritatem habeam abundantius in vobis.


${}^{5}$~Si quis autem contristavit, non me contristavit~: sed ex parte, ut non onerem omnes vos.
${}^{6}$~Sufficit illi, qui ejusmodi est, objurgatio h\ae c, qu\ae\ fit a pluribus~:
${}^{7}$~ita ut e contrario magis donetis, et consolemini, ne forte abundantiori tristitia absorbeatur qui ejusmodi est.
${}^{8}$~Propter quod obsecro vos, ut confirmetis in illum caritatem.
${}^{9}$~Ideo enim et scripsi, ut cognoscam experimentum vestrum, an in omnibus obedientes sitis.
${}^{10}$~Cui autem aliquid donastis, et ego~: nam et ego quod donavi, si quid donavi, propter vos in persona Christi,
${}^{11}$~ut non circumveniamur a Satana~: non enim ignoramus cogitationes ejus.


${}^{12}$~Cum venissem autem Troadem propter Evangelium Christi, et ostium mihi apertum esset in Domino,
${}^{13}$~non habui requiem spiritui meo, eo quod non invenerim Titum fratrem meum, sed valefaciens eis, profectus sum in Macedoniam.
${}^{14}$~Deo autem gratias, qui semper triumphat nos in Christo Jesu, et odorem notiti\ae\ su\ae\ manifestat per nos in omni loco~:
${}^{15}$~quia Christi bonus odor sumus Deo in iis qui salvi fiunt, et in iis qui pereunt~:
${}^{16}$~aliis quidem odor mortis in mortem~: aliis autem odor vit\ae\ in vitam. Et ad h\ae c quis tam idoneus~?
${}^{17}$~non enim sumus sicut plurimi, adulterantes verbum Dei, sed ex sinceritate, sed sicut ex Deo, coram Deo, in Christo loquimur.

\bchapter{3}
\lettrine[lines=3,image=true,loversize=0.05,lraise=-0.03]{I}{}ncipimus iterum nosmetipsos commendare~? aut numquid egemus (sicut quidam) commendatitiis epistolis ad vos, aut ex vobis~?
${}^{2}$~Epistola nostra vos estis, scripta in cordibus nostris, qu\ae\ scitur, et legitur ab omnibus hominibus~:
${}^{3}$~manifestati quod epistola estis Christi, ministrata a nobis, et scripta non atramento, sed Spiritu Dei vivi~: non in tabulis lapideis, sed in tabulis cordis carnalibus.


${}^{4}$~Fiduciam autem talem habemus per Christum ad Deum~:
${}^{5}$~non quod sufficientes simus cogitare aliquid a nobis, quasi ex nobis~: sed sufficientia nostra ex Deo est~:
${}^{6}$~qui et idoneos nos fecit ministros novi testamenti~: non littera, sed Spiritu~: littera enim occidit, Spiritus autem vivificat.
${}^{7}$~Quod si ministratio mortis litteris deformata in lapidibus fuit in gloria, ita ut non possent intendere filii Isra\"el in faciem Moysi propter gloriam vultus ejus, qu\ae\ evacuatur~:
${}^{8}$~quomodo non magis ministratio Spiritus erit in gloria~?
${}^{9}$~Nam si ministratio damnationis gloria est~: multo magis abundat ministerium justiti\ae\ in gloria.
${}^{10}$~Nam nec glorificatum est, quod claruit in hac parte, propter excellentem gloriam.
${}^{11}$~Si enim quod evacuatur, per gloriam est~: multo magis quod manet, in gloria est.


${}^{12}$~Habentes igitur talem spem, multa fiducia utimur~:
${}^{13}$~et non sicut Moyses ponebat velamen super faciem suam, ut non intenderent filii Isra\"el in faciem ejus, quod evacuatur,
${}^{14}$~sed obtusi sunt sensus eorum. Usque in hodiernum enim diem, idipsum velamen in lectione veteris testamenti manet non revelatum (quoniam in Christo evacuatur),
${}^{15}$~sed usque in hodiernum diem, cum legitur Moyses, velamen positum est super cor eorum.
${}^{16}$~Cum autem conversus fuerit ad Dominum, auferetur velamen.
${}^{17}$~Dominus autem Spiritus est~: ubi autem Spiritus Domini, ibi libertas.
${}^{18}$~Nos vero omnes, revelata facie gloriam Domini speculantes, in eamdem imaginem transformamur a claritate in claritatem, tamquam a Domini Spiritu.

\bchapter{4}
\lettrine[lines=3,image=true,loversize=0.05,lraise=-0.03]{I}{}deo habentes administrationem, juxta quod misericordiam consecuti sumus, non deficimus,
${}^{2}$~sed abdicamus occulta dedecoris, non ambulantes in astutia, neque adulterantes verbum Dei, sed in manifestatione veritatis commendantes nosmetipsos ad omnem conscientiam hominum coram Deo.
${}^{3}$~Quod si etiam opertum est Evangelium nostrum, in iis, qui pereunt, est opertum~:
${}^{4}$~in quibus Deus hujus s\ae culi exc\ae cavit mentes infidelium, ut non fulgeat illis illuminatio Evangelii glori\ae\ Christi, qui est imago Dei.
${}^{5}$~Non enim nosmetipsos pr\ae dicamus, sed Jesum Christum Dominum nostrum~: nos autem servos vestros per Jesum~:
${}^{6}$~quoniam Deus, qui dixit de tenebris lucem splendescere, ipse illuxit in cordibus nostris ad illuminationem scienti\ae\ claritatis Dei, in facie Christi Jesu.


${}^{7}$~Habemus autem thesaurum istum in vasis fictilibus~: ut sublimitas sit virtutis Dei, et non ex nobis.
${}^{8}$~In omnibus tribulationem patimur, sed non angustiamur~: aporiamur, sed non destituimur~:
${}^{9}$~persecutionem patimur, sed non derelinquimur~: dejicimur, sed non perimus~:
${}^{10}$~semper mortificationem Jesu in corpore nostro circumferentes, ut et vita Jesu manifestetur in corporibus nostris.
${}^{11}$~Semper enim nos, qui vivimus, in mortem tradimur propter Jesum~: ut et vita Jesu manifestetur in carne nostra mortali.
${}^{12}$~Ergo mors in nobis operatur, vita autem in vobis.
${}^{13}$~Habentes autem eumdem spiritum fidei, sicut scriptum est~: Credidi, propter quod locutus sum~: et nos credimus, propter quod et loquimur~:
${}^{14}$~scientes quoniam qui suscitavit Jesum, et nos cum Jesu suscitabit, et constituet vobiscum.
${}^{15}$~Omnia enim propter vos~: ut gratia abundans, per multos in gratiarum actione, abundet in gloriam Dei.
${}^{16}$~Propter quod non deficimus~: sed licet is, qui foris est, noster homo corrumpatur, tamen is, qui intus est, renovatur de die in diem.
${}^{17}$~Id enim, quod in pr\ae senti est momentaneum et leve tribulationis nostr\ae , supra modum in sublimitate \ae ternum glori\ae\ pondus operatur in nobis,
${}^{18}$~non contemplantibus nobis qu\ae\ videntur, sed qu\ae\ non videntur. Qu\ae\ enim videntur, temporalia sunt~: qu\ae\ autem non videntur, \ae terna sunt.

\bchapter{5}
\lettrine[lines=3,image=true,loversize=0.05,lraise=-0.03]{S}{}cimus enim quoniam si terrestris domus nostra hujus habitationis dissolvatur, quod \ae dificationem ex Deo habemus, domum non manufactam, \ae ternam in c\ae lis.
${}^{2}$~Nam et in hoc ingemiscimus, habitationem nostram, qu\ae\ de c\ae lo est, superindui cupientes~:
${}^{3}$~si tamen vestiti, non nudi inveniamur.
${}^{4}$~Nam et qui sumus in hoc tabernaculo, ingemiscimus gravati~: eo quod nolumus expoliari, sed supervestiri, ut absorbeatur quod mortale est, a vita.
${}^{5}$~Qui autem efficit nos in hoc ipsum, Deus, qui dedit nobis pignus Spiritus.
${}^{6}$~Audentes igitur semper, scientes quoniam dum sumus in corpore, peregrinamur a Domino
${}^{7}$~(per fidem enim ambulamus, et non per speciem)~:
${}^{8}$~audemus autem, et bonam voluntatem habemus magis peregrinari a corpore, et pr\ae sentes esse ad Dominum.
${}^{9}$~Et ideo contendimus, sive absentes, sive pr\ae sentes, placere illi.
${}^{10}$~Omnes enim nos manifestari oportet ante tribunal Christi, ut referat unusquisque propria corporis, prout gessit, sive bonum, sive malum.


${}^{11}$~Scientes ergo timorem Domini, hominibus suademus, Deo autem manifesti sumus. Spero autem et in conscientiis vestris manifestos nos esse.
${}^{12}$~Non iterum commendamus nos vobis, sed occasionem damus vobis gloriandi pro nobis~: ut habeatis ad eos qui in facie gloriantur, et non in corde.
${}^{13}$~Sive enim mente excedimus Deo~: sive sobrii sumus, vobis.
${}^{14}$~Caritas enim Christi urget nos~: \ae stimantes hoc, quoniam si unus pro omnibus mortuus est, ergo omnes mortui sunt~:
${}^{15}$~et pro omnibus mortuus est Christus~: ut, et qui vivunt, jam non sibi vivant, sed ei qui pro ipsis mortuus est et resurrexit.
${}^{16}$~Itaque nos ex hoc neminem novimus secundum carnem. Et si cognovimus secundum carnem Christum, sed nunc jam non novimus.
${}^{17}$~Si qua ergo in Christo nova creatura, vetera transierunt~: ecce facta sunt omnia nova.
${}^{18}$~Omnia autem ex Deo, qui nos reconciliavit sibi per Christum~: et dedit nobis ministerium reconciliationis,
${}^{19}$~quoniam quidem Deus erat in Christo mundum reconcilians sibi, non reputans illis delicta ipsorum, et posuit in nobis verbum reconciliationis.
${}^{20}$~Pro Christo ergo legatione fungimur, tamquam Deo exhortante per nos. Obsecramus pro Christo, reconciliamini Deo.
${}^{21}$~Eum, qui non noverat peccatum, pro nobis peccatum fecit, ut nos efficeremur justitia Dei in ipso.

\bchapter{6}
\lettrine[lines=3,image=true,loversize=0.05,lraise=-0.03]{A}{}djuvantes autem exhortamur ne in vacuum gratiam Dei recipiatis.
${}^{2}$~Ait enim~: Tempore accepto exaudivi te, et in die salutis adjuvi te. Ecce nunc tempus acceptabile, ecce nunc dies salutis.
${}^{3}$~Nemini dantes ullam offensionem, ut non vituperetur ministerium nostrum~:
${}^{4}$~sed in omnibus exhibeamus nosmetipsos sicut Dei ministros in multa patientia, in tribulationibus, in necessitatibus, in angustiis,
${}^{5}$~in plagis, in carceribus, in seditionibus, in laboribus, in vigiliis, in jejuniis,
${}^{6}$~in castitate, in scientia, in longanimitate, in suavitate, in Spiritu Sancto, in caritate non ficta,
${}^{7}$~in verbo veritatis, in virtute Dei, per arma justiti\ae\ a dextris et a sinistris,
${}^{8}$~per gloriam, et ignobilitatem, per infamiam, et bonam famam~: ut seductores, et veraces, sicut qui ignoti, et cogniti~:
${}^{9}$~quasi morientes, et ecce vivimus~: ut castigati, et non mortificati~:
${}^{10}$~quasi tristes, semper autem gaudentes~: sicut egentes, multos autem locupletantes~: tamquam nihil habentes, et omnia possidentes.


${}^{11}$~Os nostrum patet ad vos, o Corinthii~; cor nostrum dilatatum est.
${}^{12}$~Non angustiamini in nobis~: angustiamini autem in visceribus vestris~:
${}^{13}$~eamdem autem habentes remunerationem, tamquam filiis dico, dilatamini et vos.


${}^{14}$~Nolite jugum ducere cum infidelibus. Qu\ae\ enim participatio justiti\ae\ cum iniquitate~? aut qu\ae\ societas luci ad tenebras~?
${}^{15}$~qu\ae\ autem conventio Christi ad Belial~? aut qu\ae\ pars fideli cum infideli~?
${}^{16}$~qui autem consensus templo Dei cum idolis~? vos enim estis templum Dei vivi, sicut dicit Deus~: \begin{flushleft}\begin{verse}Quoniam inhabitabo in illis, et inambulabo inter eos,\\ et ero illorum Deus, et ipsi erunt mihi populus.\\
${}^{17}$~Propter quod exite de medio eorum,\\ et separamini, dicit Dominus,\\ et immundum ne tetigeritis~:\\
${}^{18}$~et ego recipiam vos~:\\ et ero vobis in patrem,\\ et vos eritis mihi in filios et filias,\\ dicit Dominus omnipotens.\end{verse}\end{flushleft}



\bchapter{7}
\lettrine[lines=3,image=true,loversize=0.05,lraise=-0.03]{H}{}as ergo habentes promissiones, carissimi, mundemus nos ab omni inquinamento carnis et spiritus, perficientes sanctificationem in timore Dei.
${}^{2}$~Capite nos. Neminem l\ae simus, neminem corrupimus, neminem circumvenimus.
${}^{3}$~Non ad condemnationem vestram dico~: pr\ae diximus enim quod in cordibus nostris estis ad commoriendum et ad convivendum.
${}^{4}$~Multa mihi fiducia est apud vos, multa mihi gloriatio pro vobis~: repletus sum consolatione~; superabundo gaudio in omni tribulatione nostra.
${}^{5}$~Nam et cum venissemus in Macedoniam, nullam requiem habuit caro nostra, sed omnem tribulationem passi sumus~: foris pugn\ae , intus timores.
${}^{6}$~Sed qui consolatur humiles, consolatus est nos Deus in adventu Titi.
${}^{7}$~Non solum autem in adventu ejus, sed etiam in consolatione, qua consolatus est in vobis, referens nobis vestrum desiderium, vestrum fletum, vestram \ae mulationem pro me, ita ut magis gauderem.


${}^{8}$~Quoniam etsi contristavi vos in epistola, non me pœnitet~: etsi pœniteret, videns quod epistola illa (etsi ad horam) vos contristavit,
${}^{9}$~nunc gaudeo~: non quia contristati estis, sed quia contristati estis ad pœnitentiam. Contristati enim estis ad Deum, ut in nullo detrimentum patiamini ex nobis.
${}^{10}$~Qu\ae\ enim secundum Deum tristitia est, pœnitentiam in salutem stabilem operatur~: s\ae culi autem tristitia mortem operatur.
${}^{11}$~Ecce enim hoc ipsum, secundum Deum contristari vos, quantam in vobis operatur sollicitudinem~: sed defensionem, sed indignationem, sed timorem, sed desiderium, sed \ae mulationem, sed vindictam~: in omnibus exhibuistis vos incontaminatos esse negotio.
${}^{12}$~Igitur, etsi scripsi vobis, non propter eum qui fecit injuriam, nec propter eum qui passus est~: sed ad manifestandam sollicitudinem nostram, quam habemus pro vobis
${}^{13}$~coram Deo~: ideo consolati sumus. In consolatione autem nostra, abundantius magis gavisi sumus super gaudio Titi, quia refectus est spiritus ejus ab omnibus vobis~:
${}^{14}$~et si quid apud illum de vobis gloriatus sum, non sum confusus~: sed sicut omnia vobis in veritate locuti sumus, ita et gloriatio nostra, qu\ae\ fuit ad Titum, veritas facta est,
${}^{15}$~et viscera ejus abundantius in vobis sunt, reminiscentis omnium vestrum obedientiam~: quomodo cum timore et tremore excepistis illum.
${}^{16}$~Gaudeo quod in omnibus confido in vobis.

\bchapter{8}
\lettrine[lines=3,image=true,loversize=0.05,lraise=-0.03]{N}{}otam autem facimus vobis, fratres, gratiam Dei, qu\ae\ data est in ecclesiis Macedoni\ae~:
${}^{2}$~quod in multo experimento tribulationis abundantia gaudii ipsorum fuit, et altissima paupertas eorum, abundavit in divitias simplicitatis eorum~:
${}^{3}$~quia secundum virtutem testimonium illis reddo, et supra virtutem voluntarii fuerunt,
${}^{4}$~cum multa exhortatione obsecrantes nos gratiam, et communicationem ministerii, quod fit in sanctos.
${}^{5}$~Et non sicut speravimus, sed semetipsos dederunt primum Domino, deinde nobis per voluntatem Dei,
${}^{6}$~ita ut rogaremus Titum, ut quemadmodum cœpit, ita et perficiat in vobis etiam gratiam istam.
${}^{7}$~Sed sicut in omnibus abundatis fide, et sermone, et scientia, et omni sollicitudine, insuper et caritate vestra in nos, ut et in hac gratia abundetis.
${}^{8}$~Non quasi imperans dico~: sed per aliorum sollicitudinem, etiam vestr\ae\ caritatis ingenium bonum comprobans.
${}^{9}$~Scitis enim gratiam Domini nostri Jesu Christi, quoniam propter vos egenus factus est, cum esset dives, ut illius inopia vos divites essetis.
${}^{10}$~Et consilium in hoc do~: hoc enim vobis utile est, qui non solum facere, sed et velle cœpistis ab anno priore~:
${}^{11}$~nunc vero et facto perficite~: ut quemadmodum promptus est animus voluntatis, ita sit et perficiendi ex eo quod habetis.
${}^{12}$~Si enim voluntas prompta est, secundum id quod habet, accepta est, non secundum id quod non habet.
${}^{13}$~Non enim ut aliis sit remissio, vobis autem tribulatio, sed ex \ae qualitate.
${}^{14}$~In pr\ae senti tempore vestra abundantia illorum inopiam suppleat~: ut et illorum abundantia vestr\ae\ inopi\ae\ sit supplementum, ut fiat \ae qualitas, sicut scriptum est~:
${}^{15}$~Qui multum, non abundavit~: et qui modicum, non minoravit.


${}^{16}$~Gratias autem Deo, qui dedit eamdem sollicitudinem pro vobis in corde Titi,
${}^{17}$~quoniam exhortationem quidem suscepit~: sed cum sollicitior esset, sua voluntate profectus est ad vos.
${}^{18}$~Misimus etiam cum illo fratrem, cujus laus est in Evangelio per omnes ecclesias~:
${}^{19}$~non solum autem, sed et ordinatus est ab ecclesiis comes peregrinationis nostr\ae\ in hanc gratiam, qu\ae\ ministratur a nobis ad Domini gloriam, et destinatam voluntatem nostram~:
${}^{20}$~devitantes hoc, ne quis nos vituperet in hac plenitudine, qu\ae\ ministratur a nobis.
${}^{21}$~Providemus enim bona non solum coram Deo, sed etiam coram hominibus.
${}^{22}$~Misimus autem cum illis et fratrem nostrum, quem probavimus in multis s\ae pe sollicitum esse~: nunc autem multo sollicitiorem, confidentia multa in vos,
${}^{23}$~sive pro Tito, qui est socius meus, et in vos adjutor, sive fratres nostri, Apostoli ecclesiarum, gloria Christi.
${}^{24}$~Ostensionem ergo, qu\ae\ est caritatis vestr\ae , et nostr\ae\ glori\ae\ pro vobis, in illos ostendite in faciem ecclesiarum.

\bchapter{9}
\lettrine[lines=3,image=true,loversize=0.05,lraise=-0.03]{N}{}am de ministerio, quod fit in sanctos ex abundanti est mihi scribere vobis.
${}^{2}$~Scio enim promptum animum vestrum~: pro quo de vobis glorior apud Macedones. Quoniam et Achaia parata est ab anno pr\ae terito, et vestra \ae mulatio provocavit plurimos.
${}^{3}$~Misi autem fratres~: ut ne quod gloriamur de vobis, evacuetur in hac parte, ut (quemadmodum dixi) parati sitis~:
${}^{4}$~ne cum venerint Macedones mecum, et invenerint vos imparatos, erubescamus nos (ut non dicamus vos) in hac substantia.
${}^{5}$~Necessarium ergo existimavi rogare fratres, ut pr\ae veniant ad vos, et pr\ae parent repromissam benedictionem hanc paratam esse sic, quasi benedictionem, non tamquam avaritiam.


${}^{6}$~Hoc autem dico~: qui parce seminat, parce et metet~: et qui seminat in benedictionibus, de benedictionibus et metet.
${}^{7}$~Unusquisque, prout destinavit in corde suo, non ex tristitia, aut ex necessitate~: hilarem enim datorem diligit Deus.
${}^{8}$~Potens est autem Deus omnem gratiam abundare facere in vobis~: ut in omnibus semper omnem sufficientiam habentes, abundetis in omne opus bonum,
${}^{9}$~sicut scriptum est~: Dispersit, dedit pauperibus~: justitia ejus manet in s\ae culum s\ae culi.
${}^{10}$~Qui autem administrat semen seminanti~: et panem ad manducandum pr\ae stabit, et multiplicabit semen vestrum, et augebit incrementa frugum justiti\ae\ vestr\ae~:
${}^{11}$~ut in omnibus locupletati abundetis in omnem simplicitatem, qu\ae\ operatur per nos gratiarum actionem Deo.
${}^{12}$~Quoniam ministerium hujus officii non solum supplet ea qu\ae\ desunt sanctis, sed etiam abundat per multas gratiarum actiones in Domino,
${}^{13}$~per probationem ministerii hujus, glorificantes Deum in obedientia confessionis vestr\ae , in Evangelium Christi, et simplicitate communicationis in illos, et in omnes,
${}^{14}$~et in ipsorum obsecratione pro vobis, desiderantium vos propter eminentem gratiam Dei in vobis.
${}^{15}$~Gratias Deo super inenarrabili dono ejus.

\bchapter{10}
\lettrine[lines=3,image=true,loversize=0.05,lraise=-0.03]{I}{}pse autem ego Paulus obsecro vos per mansuetudinem et modestiam Christi, qui in facie quidem humilis sum inter vos, absens autem confido in vos.
${}^{2}$~Rogo autem vos ne pr\ae sens audeam per eam confidentiam, qua existimor audere in quosdam, qui arbitrantur nos tamquam secundum carnem ambulemus.
${}^{3}$~In carne enim ambulantes, non secundum carnem militamus.
${}^{4}$~Nam arma militi\ae\ nostr\ae\ non carnalia sunt, sed potentia Deo ad destructionem munitionum, consilia destruentes,
${}^{5}$~et omnem altitudinem extollentem se adversus scientiam Dei, et in captivitatem redigentes omnem intellectum in obsequium Christi,
${}^{6}$~et in promptu habentes ulcisci omnem inobedientiam, cum impleta fuerit vestra obedientia.


${}^{7}$~Qu\ae\ secundum faciem sunt, videte. Si quis confidit sibi Christi se esse, hoc cogitet iterum apud se~: quia sicut ipse Christi est, ita et nos.
${}^{8}$~Nam etsi amplius aliquid gloriatus fuero de potestate nostra, quam dedit nobis Dominus in \ae dificationem, et non in destructionem vestram, non erubescam.
${}^{9}$~Ut autem non existimer tamquam terrere vos per epistolas~:
${}^{10}$~quoniam quidem epistol\ae , inquiunt, graves sunt et fortes~: pr\ae sentia autem corporis infirma, et sermo contemptibilis~:
${}^{11}$~hoc cogitet qui ejusmodi est, quia quales sumus verbo per epistolas absentes, tales et pr\ae sentes in facto.


${}^{12}$~Non enim audemus inserere, aut comparare nos quibusdam, qui seipsos commendant~: sed ipsi in nobis nosmetipsos metientes, et comparantes nosmetipsos nobis.
${}^{13}$~Nos autem non in immensum gloriabimur, sed secundum mensuram regul\ae , qua mensus est nobis Deus, mensuram pertingendi usque ad vos.
${}^{14}$~Non enim quasi non pertingentes ad vos, superextendimus nos~: usque ad vos enim pervenimus in Evangelio Christi.
${}^{15}$~Non in immensum gloriantes in alienis laboribus~: spem autem habentes crescentis fidei vestr\ae , in vobis magnificari secundum regulam nostram in abundantiam,
${}^{16}$~etiam in illa, qu\ae\ ultra vos sunt, evangelizare, non in aliena regula in iis qu\ae\ pr\ae parata sunt gloriari.
${}^{17}$~Qui autem gloriatur, in Domino glorietur.
${}^{18}$~Non enim qui seipsum commendat, ille probatus est~: sed quem Deus commendat.

\bchapter{11}
\lettrine[lines=3,image=true,loversize=0.05,lraise=-0.03]{U}{}tinam sustineretis modicum quid insipienti\ae\ me\ae , sed et supportare me~:
${}^{2}$~\ae mulor enim vos Dei \ae mulatione. Despondi enim vos uni viro, virginem castam exhibere Christo.
${}^{3}$~Timeo autem ne sicut serpens Hevam seduxit astutia sua, ita corrumpantur sensus vestri, et excidant a simplicitate, qu\ae\ est in Christo.
${}^{4}$~Nam si is qui venit, alium Christum pr\ae dicat, quem non pr\ae dicavimus, aut alium spiritum accipitis, quem non accepistis~: aut aliud Evangelium, quod non recepistis~: recte pateremini.
${}^{5}$~Existimo enim nihil me minus fecisse a magnis Apostolis.
${}^{6}$~Nam etsi imperitus sermone, sed non scientia, in omnibus autem manifestati sumus vobis.


${}^{7}$~Aut numquid peccatum feci, meipsum humilians, ut vos exaltemini~? quoniam gratis Evangelium Dei evangelizavi vobis~?
${}^{8}$~Alias ecclesias expoliavi, accipiens stipendium ad ministerium vestrum.
${}^{9}$~Et cum essem apud vos, et egerem, nulli onerosus fui~: nam quod mihi deerat, suppleverunt fratres, qui venerunt a Macedonia~: et in omnibus sine onere me vobis servavi, et servabo.
${}^{10}$~Est veritas Christi in me, quoniam h\ae c gloriatio non infringetur in me in regionibus Achai\ae .
${}^{11}$~Quare~? quia non diligo vos~? Deus scit.
${}^{12}$~Quod autem facio, et faciam~: ut amputem occasionem eorum qui volunt occasionem, ut in quo gloriantur, inveniantur sicut et nos.
${}^{13}$~Nam ejusmodi pseudoapostoli sunt operarii subdoli, transfigurantes se in apostolos Christi.
${}^{14}$~Et non mirum~: ipse enim Satanas transfigurat se in angelum lucis.
${}^{15}$~Non est ergo magnum, si ministri ejus transfigurentur velut ministri justiti\ae~: quorum finis erit secundum opera ipsorum.


${}^{16}$~Iterum dico (ne quis me putet insipientem esse, alioquin velut insipientem accipite me, ut et ego modicum quid glorier),
${}^{17}$~quod loquor, non loquor secundum Deum, sed quasi in insipientia, in hac substantia glori\ae .
${}^{18}$~Quoniam multi gloriantur secundum carnem~: et ego gloriabor.
${}^{19}$~Libenter enim suffertis insipientes, cum sitis ipsi sapientes.
${}^{20}$~Sustinetis enim si quis vos in servitutem redigit, si quis devorat, si quis accipit, si quis extollitur, si quis in faciem vos c\ae dit.
${}^{21}$~Secundum ignobilitatem dico, quasi nos infirmi fuerimus in hac parte. In quo quis audet (in insipientia dico) audeo et ego~:
${}^{22}$~Hebr\ae i sunt, et ego~: Isra\"elit\ae\ sunt, et ego~: semen Abrah\ae\ sunt, et ego.


${}^{23}$~Ministri Christi sunt (ut minus sapiens dico), plus ego~: in laboribus plurimis, in carceribus abundantius, in plagis supra modum, in mortibus frequenter.
${}^{24}$~A Jud\ae is quinquies, quadragenas, una minus, accepi.
${}^{25}$~Ter virgis c\ae sus sum, semel lapidatus sum~: ter naufragium feci, nocte et die in profundo maris fui,
${}^{26}$~in itineribus s\ae pe, periculis fluminum, periculis latronum, periculis ex genere, periculis ex gentibus, periculis in civitate, periculis in solitudine, periculis in mari, periculis in falsis fratribus~:
${}^{27}$~in labore et \ae rumna, in vigiliis multis, in fame et siti, in jejuniis multis, in frigore et nuditate,
${}^{28}$~pr\ae ter illa qu\ae\ extrinsecus sunt, instantia mea quotidiana, sollicitudo omnium ecclesiarum.
${}^{29}$~Quis infirmatur, et ego non infirmor~? quis scandalizatur, et ego non uror~?
${}^{30}$~Si gloriari oportet, qu\ae\ infirmitatis me\ae\ sunt, gloriabor.
${}^{31}$~Deus et Pater Domini nostri Jesu Christi, qui est benedictus in s\ae cula, scit quod non mentior.
${}^{32}$~Damasci pr\ae positus gentis Aret\ae\ regis custodiebat civitatem Damascenorum ut me comprehenderet~:
${}^{33}$~et per fenestram in sporta dimissus sum per murum, et sic effugi manus ejus.

\bchapter{12}
\lettrine[lines=3,image=true,loversize=0.05,lraise=-0.03]{S}{}i gloriari oportet (non expedit quidem), veniam autem ad visiones et revelationes Domini.
${}^{2}$~Scio hominem in Christo ante annos quatuordecim, sive in corpore nescio, sive extra corpus nescio, Deus scit, raptum hujusmodi usque ad tertium c\ae lum.
${}^{3}$~Et scio hujusmodi hominem sive in corpore, sive extra corpus nescio, Deus scit~:
${}^{4}$~quoniam raptus est in paradisum~: et audivit arcana verba, qu\ae\ non licet homini loqui.
${}^{5}$~Pro hujusmodi gloriabor~: pro me autem nihil gloriabor nisi in infirmitatibus meis.
${}^{6}$~Nam etsi voluero gloriari, non ero insipiens~: veritatem enim dicam~: parco autem, ne quis me existimet supra id quod videt in me, aut aliquid audit ex me.
${}^{7}$~Et ne magnitudo revelationum extollat me, datus est mihi stimulus carnis me\ae\ angelus Satan\ae , qui me colaphizet.
${}^{8}$~Propter quod ter Dominum rogavi ut discederet a me~:
${}^{9}$~et dixit mihi~: Sufficit tibi gratia mea~: nam virtus in infirmitate perficitur. Libenter igitur gloriabor in infirmitatibus meis, ut inhabitet in me virtus Christi.
${}^{10}$~Propter quod placeo mihi in infirmitatibus meis, in contumeliis, in necessitatibus, in persecutionibus, in angustiis pro Christo~: cum enim infirmor, tunc potens sum.
${}^{11}$~Factus sum insipiens, vos me co\"egistis. Ego enim a vobis debui commendari~: nihil enim minus fui ab iis, qui sunt supra modum Apostoli~: tametsi nihil sum~:
${}^{12}$~signa tamen apostolatus mei facta sunt super vos in omni patientia, in signis, et prodigiis, et virtutibus.
${}^{13}$~Quid est enim, quod minus habuistis pr\ae\ ceteris ecclesiis, nisi quod ego ipse non gravavi vos~? donate mihi hanc injuriam.


${}^{14}$~Ecce tertio hoc paratus sum venire ad vos~: et non ero gravis vobis. Non enim qu\ae ro qu\ae\ vestra sunt, sed vos. Nec enim debent filii parentibus thesaurizare, sed parentes filiis.
${}^{15}$~Ego autem libentissime impendam, et super impendar ipse pro animabus vestris~: licet plus vos diligens, minus diligar.
${}^{16}$~Sed esto~: ego vos non gravavi~: sed cum essem astutus, dolo vos cepi.
${}^{17}$~Numquid per aliquem eorum, quod misi ad vos, circumveni vos~?
${}^{18}$~Rogavi Titum, et misi cum illo fratrem. Numquid Titus vos circumvenit~? nonne eodem spiritu ambulavimus~? nonne iisdem vestigiis~?


${}^{19}$~Olim putatis quod excusemus nos apud vos~? coram Deo in Christo loquimur~: omnia autem, carissimi, propter \ae dificationem vestram.
${}^{20}$~Timeo enim ne forte cum venero, non quales volo, inveniam vos~: et ego inveniar a vobis, qualem non vultis~: ne forte contentiones, \ae mulationes, animositates, dissensiones, detractiones, susurrationes, inflationes, seditiones sint inter vos~:
${}^{21}$~ne iterum cum venero, humiliet me Deus apud vos, et lugeam multos ex iis qui ante peccaverunt, et non egerunt pœnitentiam super immunditia, et fornicatione, et impudicitia, quam gesserunt.

\bchapter{13}
\lettrine[lines=3,image=true,loversize=0.05,lraise=-0.03]{E}{}cce tertio hoc venio ad vos~: in ore duorum vel trium testium stabit omne verbum.
${}^{2}$~Pr\ae dixi, et pr\ae dico, ut pr\ae sens, et nunc absens iis qui ante peccaverunt, et ceteris omnibus, quoniam si venero iterum, non parcam.
${}^{3}$~An experimentum qu\ae ritis ejus, qui in me loquitur Christus, qui in vobis non infirmatur, sed potens est in vobis~?
${}^{4}$~Nam etsi crucifixus est ex infirmitate~: sed vivit ex virtute Dei. Nam et nos infirmi sumus in illo~: sed vivemus cum eo ex virtute Dei in vobis.
${}^{5}$~Vosmetipsos tentate si estis in fide~: ipsi vos probate. An non cognoscitis vosmetipsos quia Christus Jesus in vobis est~? nisi forte reprobi estis.
${}^{6}$~Spero autem quod cognoscetis, quia nos non sumus reprobi.
${}^{7}$~Oramus autem Deum ut nihil mali faciatis, non ut nos probati appareamus, sed ut vos quod bonum est faciatis~: nos autem ut reprobi simus.
${}^{8}$~Non enim possumus aliquid adversus veritatem, sed pro veritate.
${}^{9}$~Gaudemus enim, quoniam nos infirmi sumus, vos autem potentes estis. Hoc et oramus, vestram consummationem.
${}^{10}$~Ideo h\ae c absens scribo, ut non pr\ae sens durius agam secundum potestatem, quam Dominus dedit mihi in \ae dificationem, et non in destructionem.


${}^{11}$~De cetero, fratres, gaudete, perfecti estote, exhortamini, idem sapite, pacem habete, et Deus pacis et dilectionis erit vobiscum.
${}^{12}$~Salutate invicem in osculo sancto. Salutant vos omnes sancti.
${}^{13}$~Gratia Domini nostri Jesu Christi, et caritas Dei, et communicatio Sancti Spiritus sit cum omnibus vobis. Amen.
