\bbook{Epistola B. Petri Apostoli Prima}
{Petri I}{images/genese_heading}


\bchapter{1}
\lettrine[lines=6,image=true,loversize=0.05,lraise=-0.03]{P}{}etrus Apostolus Jesu Christi, electis advenis dispersionis Ponti, Galati\ae , Cappadoci\ae , Asi\ae , et Bithyni\ae ,
${}^{2}$~secundum pr\ae scientiam Dei Patris, in sanctificationem Spiritus, in obedientiam, et aspersionem sanguinis Jesu Christi. Gratia vobis, et pax multiplicetur.


${}^{3}$~Benedictus Deus et Pater Domini nostri Jesu Christi, qui secundum misericordiam suam magnam regeneravit nos in spem vivam, per resurrectionem Jesu Christi ex mortuis,
${}^{4}$~in h\ae reditatem incorruptibilem, et incontaminatam, et immarcescibilem, conservatam in c\ae lis in vobis,
${}^{5}$~qui in virtute Dei custodimini per fidem in salutem, paratam revelari in tempore novissimo.
${}^{6}$~In quo exsultabis, modicum nunc si oportet contristari in variis tentationibus~:
${}^{7}$~ut probatio vestr\ae\ fidei multo pretiosior auro (quod per ignem probatur) inveniatur in laudem, et gloriam, et honorem in revelatione Jesu Christi~:
${}^{8}$~quem cum non videritis, diligitis~: in quem nunc quoque non videntes creditis~: credentes autem exsultabitis l\ae titia inenarrabili, et glorificata~:
${}^{9}$~reportantes finem fidei vestr\ae , salutem animarum.
${}^{10}$~De qua salute exquisierunt, atque scrutati sunt prophet\ae , qui de futura in vobis gratia prophetaverunt~:
${}^{11}$~scrutantes in quod vel quale tempus significaret in eis Spiritus Christi~: pr\ae nuntians eas qu\ae\ in Christo sunt passiones, et posteriores glorias~:
${}^{12}$~quibus revelatum est quia non sibimetipsis, vobis autem ministrabant ea qu\ae\ nunc nuntiata sunt vobis per eos qui evangelizaverunt vobis, Spiritu Sancto misso de c\ae lo, in quem desiderant angeli prospicere.


${}^{13}$~Propter quod succincti lumbos mentis vestr\ae , sobrii, perfecte sperate in eam, qu\ae\ offertur vobis, gratiam, in revelationem Jesu Christi~:
${}^{14}$~quasi filii obedienti\ae , non configurati prioribus ignoranti\ae\ vestr\ae\ desideriis~:
${}^{15}$~sed secundum eum qui vocavit vos, Sanctum~: et ipsi in omni conversatione sancti sitis~:
${}^{16}$~quoniam scriptum est~: Sancti eritis, quoniam ego sanctus sum.
${}^{17}$~Et si patrem invocatis eum, qui sine acceptione personarum judicat secundum uniuscujusque opus, in timore incolatus vestri tempore conversamini.
${}^{18}$~Scientes quod non corruptibilibus, auro vel argento, redempti estis de vana vestra conversatione patern\ae\ traditionis~:
${}^{19}$~sed pretioso sanguine quasi agni immaculati Christi, et incontaminati~:
${}^{20}$~pr\ae cogniti quidem ante mundi constitutionem, manifestati autem novissimis temporibus propter vos,
${}^{21}$~qui per ipsum fideles estis in Deo, qui suscitavit eum a mortuis, et dedit ei gloriam, ut fides vestra et spes esset in Deo~:
${}^{22}$~animas vestras castificantes in obedientia caritatis, in fraternitatis amore, simplici ex corde invicem diligite attentius~:
${}^{23}$~renati non ex semine corruptibili, sed incorruptibili per verbum Dei vivi, et permanentis in \ae ternum~:
${}^{24}$~quia omnis caro ut fœnum~: et omnis gloria ejus tamquam flos fœni~: exaruit fœnum, et flos ejus decidit.
${}^{25}$~Verbum autem Domini manet in \ae ternum~: hoc est autem verbum, quod evangelizatum est in vos.

\bchapter{2}
\lettrine[lines=3,image=true,loversize=0.05,lraise=-0.03]{D}{}eponentes igitur omnem malitiam, et omnem dolum, et simulationes, et invidias, et omnes detractiones,
${}^{2}$~sicut modo geniti infantes, rationabile, sine dolo lac concupiscite~: ut in eo crescatis in salutem~:
${}^{3}$~si tamen gustastis quoniam dulcis est Dominus.
${}^{4}$~Ad quem accedentes lapidem vivum, ab hominibus quidem reprobatum, a Deo autem electum, et honorificatum~:
${}^{5}$~et ipsi tamquam lapides vivi super\ae dificamini, domus spiritualis, sacerdotium sanctum, offerre spirituales hostias, acceptabiles Deo per Jesum Christum.
${}^{6}$~Propter quod continet Scriptura~: Ecce pono in Sion lapidem summum angularem, electum, pretiosum~: et qui crediderit in eum, non confundetur.
${}^{7}$~Vobis igitur honor credentibus~: non credentibus autem lapis, quem reprobaverunt \ae dificantes~: hic factus est in caput anguli,
${}^{8}$~et lapis offensionis, et petra scandali, his qui offendunt verbo, nec credunt in quo et positi sunt.
${}^{9}$~Vos autem genus electum, regale sacerdotium, gens sancta, populus acquisitionis~: ut virtutes annuntietis ejus qui de tenebris vos vocavit in admirabile lumen suum.
${}^{10}$~Qui aliquando non populus, nunc autem populus Dei~: qui non consecuti misericordiam, nunc autem misericordiam consecuti.


${}^{11}$~Carissimi, obsecro vos tamquam advenas et peregrinos abstinere vos a carnalibus desideriis, qu\ae\ militant adversus animam,
${}^{12}$~conversationem vestram inter gentes habentes bonam~: ut in eo quod detrectant de vobis tamquam de malefactoribus, ex bonis operibus vos considerantes, glorificent Deum in die visitationis.
${}^{13}$~Subjecti igitur estote omni human\ae\ creatur\ae\ propter Deum~: sive regi quasi pr\ae cellenti~:
${}^{14}$~sive ducibus tamquam ab eo missis ad vindictam malefactorum, laudem vero bonorum~:
${}^{15}$~quia sic est voluntas Dei, ut benefacientes obmutescere faciatis imprudentium hominum ignorantiam~:
${}^{16}$~quasi liberi, et non quasi velamen habentes maliti\ae\ libertatem, sed sicut servi Dei.
${}^{17}$~Omnes honorate~: fraternitatem diligite~: Deum timete~: regem honorificate.


${}^{18}$~Servi, subditi estote in omni timore dominis, non tantum bonis et modestis, sed etiam dyscolis.
${}^{19}$~H\ae c est enim gratia, si propter Dei conscientiam sustinet quis tristitias, patiens injuste.
${}^{20}$~Qu\ae\ enim est gloria, si peccantes, et colaphizati suffertis~? sed si bene facientes patienter sustinetis, h\ae c est gratia apud Deum.
${}^{21}$~In hoc enim vocati estis~: quia et Christus passus est pro nobis, vobis relinquens exemplum ut sequamini vestigia ejus~:
${}^{22}$~qui peccatum non fecit, nec inventus est dolus in ore ejus~:
${}^{23}$~qui cum malediceretur, non maledicebat~: cum pateretur, non comminabatur~: tradebat autem judicanti se injuste~:
${}^{24}$~qui peccata nostra ipse pertulit in corpore suo super lignum~; ut peccatis mortui, justiti\ae\ vivamus~: cujus livore sanati estis.
${}^{25}$~Eratis enim sicut oves errantes, sed conversi estis nunc ad pastorem, et episcopum animarum vestrarum.

\bchapter{3}
\lettrine[lines=3,image=true,loversize=0.05,lraise=-0.03]{S}{}imiliter et mulieres subdit\ae\ sint viris suis~: ut etsi qui non credunt verbo, per mulierem conversationem sine verbo lucrifiant~:
${}^{2}$~considerantes in timore castam conversationem vestram.
${}^{3}$~Quarum non sit extrinsecus capillatura, aut circumdatio auri, aut indumenti vestimentorum cultus~:
${}^{4}$~sed qui absconditus est cordis homo, in incorruptibilitate quieti, et modesti spiritus, qui est in conspectu Dei locuples.
${}^{5}$~Sic enim aliquando et sanct\ae\ mulieres, sperantes in Deo, ornabant se, subject\ae\ propriis viris.
${}^{6}$~Sicut Sara obediebat Abrah\ae , dominum eum vocans~: cujus estis fili\ae\ benefacientes, et non pertimentes ullam perturbationem.
${}^{7}$~Viri similiter cohabitantes secundum scientiam, quasi infirmiori vasculo muliebri impartientes honorem, tamquam et coh\ae redibus grati\ae\ vit\ae~: ut non impediantur orationes vestr\ae .


${}^{8}$~In fine autem omnes unanimes, compatientes fraternitatis amatores, misericordes, modesti, humiles~:
${}^{9}$~non reddentes malum pro malo, nec maledictum pro maledicto, sed e contrario benedicentes~: quia in hoc vocati estis, ut benedictionem h\ae reditate possideatis.
${}^{10}$~Qui enim vult vitam diligere, et dies videre bonos, co\"erceat linguam suam a malo, et labia ejus ne loquantur dolum.
${}^{11}$~Declinet a malo, et faciat bonum~: inquirat pacem, et sequatur eam~:
${}^{12}$~quia oculi Domini super justos, et aures ejus in preces eorum~: vultus autem Domini super facientes mala.
${}^{13}$~Et quis est qui vobis noceat, si boni \ae mulatores fueritis~?
${}^{14}$~Sed et si quid patimini propter justitiam, beati. Timorem autem eorum ne timueritis, et non conturbemini.
${}^{15}$~Dominum autem Christum sanctificate in cordibus vestris, parati semper ad satisfactionem omni poscenti vos rationem de ea, qu\ae\ in vobis est, spe.
${}^{16}$~Sed cum modestia, et timore, conscientiam habentes bonam~: ut in eo, quod detrahunt vobis, confundantur, qui calumniantur vestram bonam in Christo conversationem.
${}^{17}$~Melius est enim benefacientes (si voluntas Dei velit) pati, quam malefacientes.


${}^{18}$~Quia et Christus semel pro peccatis nostris mortuus est, justus pro injustis, ut nos offerret Deo, mortificatus quidem carne, vivificatus autem spiritu.
${}^{19}$~In quo et his, qui in carcere erant, spiritibus veniens pr\ae dicavit~:
${}^{20}$~qui increduli fuerant aliquando, quando exspectabant Dei patientiam in diebus No\"e, cum fabricaretur arca~: in qua pauci, id est octo anim\ae , salv\ae\ fact\ae\ sunt per aquam.
${}^{21}$~Quod et vos nunc similis form\ae\ salvos fecit baptisma~: non carnis depositio sordium, sed conscienti\ae\ bon\ae\ interrogatio in Deum per resurrectionem Jesu Christi.
${}^{22}$~Qui est in dextera Dei, deglutiens mortem ut vit\ae\ \ae tern\ae\ h\ae redes efficeremur~: profectus in c\ae lum subjectis sibi angelis, et potestatibus, et virtutibus.

\bchapter{4}
\lettrine[lines=3,image=true,loversize=0.05,lraise=-0.03]{C}{}hristo igitur passo in carne, et vos eadem cogitatione armamini~: quia qui passus est in carne, desiit a peccatis~:
${}^{2}$~ut jam non desideriis hominum, sed voluntati Dei, quod reliquum est in carne vivat temporis.
${}^{3}$~Sufficit enim pr\ae teritum tempus ad voluntatem gentium consummandam his qui ambulaverunt in luxuriis, desideriis, vinolentiis, comessationibus, potationibus, et illicitis idolorum cultibus.
${}^{4}$~In quo admirantur non concurrentibus vobis in eamdem luxuri\ae\ confusionem, blasphemantes.
${}^{5}$~Qui reddent rationem ei qui paratus est judicare vivos et mortuos.
${}^{6}$~Propter hoc enim et mortuis evangelizatum est~: ut judicentur quidem secundum homines in carne, vivant autem secundum Deum in spiritu.


${}^{7}$~Omnium autem finis appropinquavit. Estote itaque prudentes, et vigilate in orationibus.
${}^{8}$~Ante omnia autem, mutuam in vobismetipsis caritatem continuam habentes~: quia caritas operit multitudinem peccatorum.
${}^{9}$~Hospitales invicem sine murmuratione.
${}^{10}$~Unusquisque, sicut accepit gratiam, in alterutrum illam administrantes, sicut boni dispensatores multiformis grati\ae\ Dei.
${}^{11}$~Si quis loquitur, quasi sermones Dei~: si quis ministrat, tamquam ex virtute, quam administrat Deus~: ut in omnibus honorificetur Deus per Jesum Christum~: cui est gloria et imperium in s\ae cula s\ae culorum. Amen.


${}^{12}$~Carissimi, nolite peregrinari in fervore, qui ad tentationem vobis fit, quasi novi aliquid vobis contingat~:
${}^{13}$~sed communicantes Christi passionibus gaudete, ut et in revelatione glori\ae\ ejus gaudeatis exsultantes.
${}^{14}$~Si exprobramini in nomine Christi, beati eritis~: quoniam quod est honoris, glori\ae , et virtutis Dei, et qui est ejus Spiritus, super vos requiescit.
${}^{15}$~Nemo autem vestrum patiatur ut homicida, aut fur, aut maledicus, aut alienorum appetitor.
${}^{16}$~Si autem ut christianus, non erubescat~: glorificet autem Deum in isto nomine~:
${}^{17}$~quoniam tempus est ut incipiat judicium a domo Dei. Si autem primum a nobis, quis finis eorum, qui non credunt Dei Evangelio~?
${}^{18}$~et si justus vix salvabitur, impius et peccator ubi parebunt~?
${}^{19}$~Itaque et hi, qui patiuntur secundum voluntatem Dei, fideli Creatori commendent animas suas in benefactis.

\bchapter{5}
\lettrine[lines=3,image=true,loversize=0.05,lraise=-0.03]{S}{}eniores ergo, qui in vobis sunt, obsecro, consenior et testis Christi passionum~: qui et ejus, qu\ae\ in futuro revelanda est, glori\ae\ communicator~:
${}^{2}$~pascite qui in vobis est gregem Dei, providentes non coacte, sed spontanee secundum Deum~: neque turpis lucri gratia, sed voluntarie~:
${}^{3}$~neque ut dominantes in cleris, sed forma facti gregis ex animo.
${}^{4}$~Et cum apparuerit princeps pastorum, percipietis immarcescibilem glori\ae\ coronam.
${}^{5}$~Similiter adolescentes subditi estote senioribus.

 Omnes autem invicem humilitatem insinuate, quia Deus superbis resistit, humilibus autem dat gratiam.
${}^{6}$~Humiliamini igitur sub potenti manu Dei, ut vos exaltet in tempore visitationis~:
${}^{7}$~omnem sollicitudinem vestram projicientes in eum, quoniam ipsi cura est de vobis.
${}^{8}$~Sobrii estote, et vigilate~: quia adversarius vester diabolus tamquam leo rugiens circuit, qu\ae rens quem devoret~:
${}^{9}$~cui resistite fortes in fide~: scientes eamdem passionem ei qu\ae\ in mundo est vestr\ae\ fraternitati fieri.
${}^{10}$~Deus autem omnis grati\ae , qui vocavit nos in \ae ternam suam gloriam in Christo Jesu, modicum passos ipse perficiet, confirmabit, solidabitque.
${}^{11}$~Ipsi gloria, et imperium in s\ae cula s\ae culorum. Amen.


${}^{12}$~Per Silvanum fidelem fratrem vobis, ut arbitror, breviter scripsi~: obsecrans et contestans, hanc esse veram gratiam Dei, in qua statis.
${}^{13}$~Salutat vos ecclesia qu\ae\ est in Babylone co\"electa, et Marcus filius meus.
${}^{14}$~Salutate invicem in osculo sancto. Gratia vobis omnibus qui estis in Christo Jesu. Amen.
