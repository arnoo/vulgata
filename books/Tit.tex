{\centering \section*{Epistola B. Pauli Apostoli ad Titum}}\thispagestyle{empty}
\addcontentsline{toc}{subsection}{Ad Titum}
\fancyhead[C]{\textsc{ad Titum}}

\Needspace{2.5\baselineskip}\versal{1}~\lettrine[lines=10,image=true,loversize=0.05,lraise=-0.03]{P}{}aulus servus Dei, Apostolus autem Jesu Christi secundum fidem electorum Dei, et agnitionem veritatis, qu\ae\ secundum pietatem est
${}^{2}$~in spem vit\ae\ \ae tern\ae , quam promisit qui non mentitur, Deus, ante tempora s\ae cularia~:
${}^{3}$~manifestavit autem temporibus suis verbum suum in pr\ae dicatione, qu\ae\ credita est mihi secundum pr\ae ceptum Salvatoris nostri Dei~:
${}^{4}$~Tito dilecto filio secundum communem fidem, gratia, et pax a Deo Patre, et Christo Jesu Salvatore nostro.


${}^{5}$~Hujus rei gratia reliqui te Cret\ae , ut ea qu\ae\ desunt, corrigas, et constituas per civitates presbyteros, sicut et ego disposui tibi,
${}^{6}$~si quis sine crimine est, unius uxoris vir, filios habens fideles, non in accusatione luxuri\ae , aut non subditos.
${}^{7}$~Oportet enim episcopum sine crimine esse, sicut Dei dispensatorem~: non superbum, non iracundum, non vinolentum, non percussorem, non turpis lucri cupidum~:
${}^{8}$~sed hospitalem, benignum, sobrium, justum, sanctum, continentem,
${}^{9}$~amplectentem eum, qui secundum doctrinam est, fidelem sermonem~: ut potens sit exhortari in doctrina sana, et eos qui contradicunt, arguere.


${}^{10}$~Sunt enim multi etiam inobedientes, vaniloqui, et seductores~: maxime qui de circumcisione sunt~:
${}^{11}$~quos oportet redargui~: qui universas domos subvertunt, docentes qu\ae\ non oportet, turpis lucri gratia.
${}^{12}$~Dixit quidam ex illis, proprius ipsorum propheta~: Cretenses semper mendaces, mal\ae\ besti\ae , ventres pigri.
${}^{13}$~Testimonium hoc verum est. Quam ob causam increpa illos dure, ut sani sint in fide,
${}^{14}$~non intendentes judaicis fabulis, et mandatis hominum, aversantium se a veritate.
${}^{15}$~Omnia munda mundis~: coinquinatis autem et infidelibus, nihil est mundum, sed inquinat\ae\ sunt eorum et mens et conscientia.
${}^{16}$~Confitentur se nosse Deum, factis autem negant~: cum sint abominati, et incredibiles, et ad omne opus bonum reprobi.
\Needspace{2.5\baselineskip}\versal{2}~\lettrine[lines=10,image=true,loversize=0.05,lraise=-0.03]{T}{}u autem loquere qu\ae\ decent sanam doctrinam~:
${}^{2}$~senes ut sobrii sint, pudici, prudentes, sani in fide, in dilectione, in patientia~:
${}^{3}$~anus similiter in habitu sancto, non criminatrices, non multo vino servientes, bene docentes~:
${}^{4}$~ut prudentiam doceant adolescentulas, ut viros suos ament, filios suos diligant,
${}^{5}$~prudentes, castas, sobrias, domus curam habentes, benignas, subditas viris suis, ut non blasphemetur verbum Dei.
${}^{6}$~Juvenes similiter hortare ut sobrii sint.
${}^{7}$~In omnibus teipsum pr\ae be exemplum bonorum operum, in doctrina, in integritate, in gravitate,
${}^{8}$~verbum sanum, irreprehensibile~: ut is qui ex adverso est, vereatur, nihil habens malum dicere de nobis.
${}^{9}$~Servos dominis suis subditos esse, in omnibus placentes, non contradicentes,
${}^{10}$~non fraudantes, sed in omnibus fidem bonam ostendentes~: ut doctrinam Salvatoris nostri Dei ornent in omnibus.
${}^{11}$~Apparuit enim gratia Dei Salvatoris nostri omnibus hominibus,
${}^{12}$~erudiens nos, ut abnegantes impietatem, et s\ae cularia desideria, sobrie, et juste, et pie vivamus in hoc s\ae culo,
${}^{13}$~exspectantes beatam spem, et adventum glori\ae\ magni Dei, et Salvatoris nostri Jesu Christi~:
${}^{14}$~qui dedit semetipsum pro nobis, ut nos redimeret ab omni iniquitate, et mundaret sibi populum acceptabilem, sectatorem bonorum operum.
${}^{15}$~H\ae c loquere, et exhortare, et argue cum omni imperio. Nemo te contemnat.
\Needspace{2.5\baselineskip}\versal{3}~\lettrine[lines=10,image=true,loversize=0.05,lraise=-0.03]{A}{}dmone illos principibus, et potestatibus subditos esse, dicto obedire, ad omne opus bonum paratos esse~:
${}^{2}$~neminem blasphemare, non litigiosos esse, sed modestos, omnem ostendentes mansuetudinem ad omnes homines.
${}^{3}$~Eramus enim aliquando et nos insipientes, increduli, errantes, servientes desideriis, et voluptatibus variis, in malitia et invidia agentes, odibiles, odientes invicem.
${}^{4}$~Cum autem benignitas et humanitas apparuit Salvatoris nostri Dei,
${}^{5}$~non ex operibus justiti\ae , qu\ae\ fecimus nos, sed secundum suam misericordiam salvos nos fecit per lavacrum regenerationis et renovationis Spiritus Sancti,
${}^{6}$~quem effudit in nos abunde per Jesum Christum Salvatorem nostrum~:
${}^{7}$~ut justificati gratia ipsius, h\ae redes simus secundum spem vit\ae\ \ae tern\ae .
${}^{8}$~Fidelis sermo est~: et de his volo te confirmare~: ut curent bonis operibus pr\ae esse qui credunt Deo. H\ae c sunt bona, et utilia hominibus.
${}^{9}$~Stultas autem qu\ae stiones, et genealogias, et contentiones, et pugnas legis devita~: sunt enim inutiles, et van\ae .
${}^{10}$~H\ae reticum hominem post unam et secundam correptionem devita~:
${}^{11}$~sciens quia subversus est, qui ejusmodi est, et delinquit, cum sit proprio judicio condemnatus.


${}^{12}$~Cum misero ad te Artemam, aut Tychicum, festina ad me venire Nicopolim~: ibi enim statui hiemare.
${}^{13}$~Zenam legisperitum et Apollo sollicite pr\ae mitte, ut nihil illis desit.
${}^{14}$~Discant autem et nostri bonis operibus pr\ae esse ad usus necessarios~: ut non sint infructuosi.
${}^{15}$~Salutant te qui mecum sunt omnes~: saluta eos qui nos amant in fide. Gratia Dei cum omnibus vobis. Amen.
