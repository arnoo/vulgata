\bbook{Epistola Catholica B. Jacobi Apostoli}
{Jacobi}{images/genese_heading}


\bchapter
\mylettrine{J}acobus, Dei et Domini nostri Jesu Christi servus, duodecim tribubus, qu\ae\ sunt in dispersione, salutem.


${}^{2}$~Omne gaudium existimate fratres mei, cum in tentationes varias incideritis~:
${}^{3}$~scientes quod probatio fidei vestr\ae\ patientiam operatur.
${}^{4}$~Patientia autem opus perfectum habet~: ut sitis perfecti et integri in nullo deficientes.
${}^{5}$~Si quis autem vestrum indiget sapientia, postulet a Deo, qui dat omnibus affluenter, et non improperat~: et dabitur ei.
${}^{6}$~Postulet autem in fide nihil h\ae sitans~: qui enim h\ae sitat, similis est fluctui maris, qui a vento movetur et circumfertur~:
${}^{7}$~non ergo \ae stimet homo ille quod accipiat aliquid a Domino.
${}^{8}$~Vir duplex animo inconstans est in omnibus viis suis.
${}^{9}$~Glorietur autem frater humilis in exaltatione sua~:
${}^{10}$~dives autem in humilitate sua, quoniam sicut flos fœni transibit~;
${}^{11}$~exortus est enim sol cum ardore, et arefecit fœnum, et flos ejus decidit, et decor vultus ejus deperiit~: ita et dives in itineribus suis marcescet.
${}^{12}$~Beatus vir qui suffert tentationem~: quoniam cum probatus fuerit, accipiet coronam vit\ae , quam repromisit Deus diligentibus se.
${}^{13}$~Nemo cum tentatur, dicat quoniam a Deo tentatur~: Deus enim intentator malorum est~: ipse autem neminem tentat.
${}^{14}$~Unusquisque vero tentatur a concupiscentia sua abstractus, et illectus.
${}^{15}$~Deinde concupiscentia cum conceperit, parit peccatum~: peccatum vero cum consummatum fuerit, generat mortem.
${}^{16}$~Nolite itaque errare, fratres mei dilectissimi.
${}^{17}$~Omne datum optimum, et omne donum perfectum desursum est, descendens a Patre luminum, apud quem non est transmutatio, nec vicissitudinis obumbratio.
${}^{18}$~Voluntarie enim genuit nos verbo veritatis, ut simus initium aliquod creatur\ae\ ejus.


${}^{19}$~Scitis, fratres mei dilectissimi. Sit autem omnis homo velox ad audiendum~: tardus autem ad loquendum, et tardus ad iram.
${}^{20}$~Ira enim viri justitiam Dei non operatur.
${}^{21}$~Propter quod abjicientes omnem immunditiam, et abundantiam maliti\ae , in mansuetudine suscipite insitum verbum, quod potest salvare animas vestras.
${}^{22}$~Estote autem factores verbi, et non auditores tantum~: fallentes vosmetipsos.
${}^{23}$~Quia si quis auditor est verbi, et non factor, hic comparabitur viro consideranti vultum nativitatis su\ae\ in speculo~:
${}^{24}$~consideravit enim se, et abiit, et statim oblitus est qualis fuerit.
${}^{25}$~Qui autem perspexerit in legem perfectam libertatis, et permanserit in ea, non auditor obliviosus factus, sed factor operis~: hic beatus in facto suo erit.
${}^{26}$~Si quis autem putat se religiosum esse, non refrenans linguam suam, sed seducens cor suum, hujus vana est religio.
${}^{27}$~Religio munda et immaculata apud Deum et Patrem, h\ae c est~: visitare pupillos et viduas in tribulatione eorum, et immaculatum se custodire ab hoc s\ae culo.

\bchapter
\mylettrine{F}ratres mei, nolite in personarum acceptione habere fidem Domini nostri Jesu Christi glori\ae .
${}^{2}$~Etenim si introierit in conventum vestrum vir aureum annulum habens in veste candida, introierit autem et pauper in sordido habitu,
${}^{3}$~et intendatis in eum qui indutus est veste pr\ae clara, et dixeritis ei~: Tu sede hic bene~: pauperi autem dicatis~: Tu sta illic~; aut sede sub scabello pedum meorum~:
${}^{4}$~nonne judicatis apud vosmetipsos, et facti estis judices cogitationum iniquarum~?
${}^{5}$~Audite, fratres mei dilectissimi~: nonne Deus elegit pauperes in hoc mundo, divites in fide, et h\ae redes regni, quod repromisit Deus diligentibus se~?
${}^{6}$~vos autem exhonorastis pauperem. Nonne divites per potentiam opprimunt vos, et ipsi trahunt vos ad judicia~?
${}^{7}$~nonne ipsi blasphemant bonum nomen, quod invocatum est super vos~?
${}^{8}$~Si tamen legem perficitis regalem secundum Scripturas~: Diliges proximum tuum sicut teipsum~: bene facitis~:
${}^{9}$~si autem personas accipitis, peccatum operamini, redarguti a lege quasi transgressores.
${}^{10}$~Quicumque autem totam legem servaverit, offendat autem in uno, factus est omnium reus.
${}^{11}$~Qui enim dixit~: Non mœchaberis, dixit et~: Non occides. Quod si non mœchaberis, occides autem, factus es transgressor legis.
${}^{12}$~Sic loquimini, et sic facite sicut per legem libertatis incipientes judicari.
${}^{13}$~Judicium enim sine misericordia illi qui non fecit misericordiam~: superexaltat autem misericordia judicium.


${}^{14}$~Quid proderit, fratres mei, si fidem quis dicat se habere, opera autem non habeat~? numquid poterit fides salvare eum~?
${}^{15}$~Si autem frater et soror nudi sint, et indigeant victu quotidiano,
${}^{16}$~dicat autem aliquis ex vobis illis~: Ite in pace, calefacimini et saturamini~: non dederitis autem eis qu\ae\ necessaria sunt corpori, quid proderit~?
${}^{17}$~Sic et fides, si non habeat opera, mortua est in semetipsa.
${}^{18}$~Sed dicet quis~: Tu fidem habes, et ego opera habeo~: ostende mihi fidem tuam sine operibus~: et ego ostendam tibi ex operibus fidem meam.
${}^{19}$~Tu credis quoniam unus est Deus~: bene facis~: et d\ae mones credunt, et contremiscunt.
${}^{20}$~Vis autem scire, o homo inanis, quoniam fides sine operibus mortua est~?
${}^{21}$~Abraham pater noster nonne ex operibus justificatus est, offerens Isaac filium suum super altare~?
${}^{22}$~Vides quoniam fides cooperabatur operibus illius~: et ex operibus fides consummata est~?
${}^{23}$~Et suppleta est Scriptura, dicens~: Credidit Abraham Deo, et reputatum est illi ad justitiam, et amicus Dei appellatus est.
${}^{24}$~Videtis quoniam ex operibus justificatur homo, et non ex fide tantum~?
${}^{25}$~Similiter et Rahab meretrix, nonne ex operibus justificata est, suscipiens nuntios, et alia via ejiciens~?
${}^{26}$~Sicut enim corpus sine spiritu mortuum est, ita et fides sine operibus mortua est.

\bchapter
\mylettrine{N}olite plures magistri fieri fratres mei, scientes quoniam majus judicium sumitis.
${}^{2}$~In multis enim offendimus omnes. Si quis in verbo non offendit, hic perfectus est vir~: potest etiam freno circumducere totum corpus.
${}^{3}$~Si autem equis frena in ora mittimus ad consentiendum nobis, et omne corpus illorum circumferimus.
${}^{4}$~Ecce et naves, cum magn\ae\ sint, et a ventis validis minentur, circumferuntur a modico gubernaculo ubi impetus dirigentis voluerit.
${}^{5}$~Ita et lingua modicum quidem membrum est, et magna exaltat. Ecce quantus ignis quam magnam silvam incendit~!
${}^{6}$~Et lingua ignis est, universitas iniquitatis. Lingua constituitur in membris nostris, qu\ae\ maculat totum corpus, et inflammat rotam nativitatis nostr\ae\ inflammata a gehenna.
${}^{7}$~Omnis enim natura bestiarum, et volucrum, et serpentium, et ceterorum domantur, et domita sunt a natura humana~:
${}^{8}$~linguam autem nullus hominum domare potest~: inquietum malum, plena veneno mortifero.
${}^{9}$~In ipsa benedicimus Deum et Patrem~: et in ipsa maledicimus homines, qui ad similitudinem Dei facti sunt.
${}^{10}$~Ex ipso ore procedit benedictio et maledictio. Non oportet, fratres mei, h\ae c ita fieri.
${}^{11}$~Numquid fons de eodem foramine emanat dulcem et amaram aquam~?
${}^{12}$~Numquid potest, fratres mei, ficus uvas facere, aut vitis ficus~? Sic neque salsa dulcem potest facere aquam.


${}^{13}$~Quis sapiens et disciplinatus inter vos~? Ostendat ex bona conversatione operationem suam in mansuetudine sapienti\ae .
${}^{14}$~Quod si zelum amarum habetis, et contentiones sint in cordibus vestris~: nolite gloriari, et mendaces esse adversus veritatem~:
${}^{15}$~non est enim ista sapientia desursum descendens~: sed terrena, animalis, diabolica.
${}^{16}$~Ubi enim zelus et contentio, ibi inconstantia et omne opus pravum.
${}^{17}$~Qu\ae\ autem desursum est sapientia, primum quidem pudica est, deinde pacifica, modesta, suadibilis, bonis consentiens, plena misericordia et fructibus bonis, non judicans, sine simulatione.
${}^{18}$~Fructus autem justiti\ae , in pace seminatur, facientibus pacem.

\bchapter
\mylettrine{U}nde bella et lites in vobis~? nonne hinc~: ex concupiscentiis vestris, qu\ae\ militant in membris vestris~?
${}^{2}$~concupiscitis, et non habetis~: occiditis, et zelatis~: et non potestis adipisci~: litigatis, et belligeratis, et non habetis, propter quod non postulatis.
${}^{3}$~Petitis, et non accipitis~: eo quod male petatis~: ut in concupiscentiis vestris insumatis.
${}^{4}$~Adulteri, nescitis quia amicitia hujus mundi inimica est Dei~? quicumque ergo voluerit amicus esse s\ae culi hujus, inimicus Dei constituitur.
${}^{5}$~An putatis quia inaniter Scriptura dicat~: Ad invidiam concupiscit spiritus qui habitat in vobis~?
${}^{6}$~majorem autem dat gratiam. Propter quod dicit~: Deus superbis resistit, humilibus autem dat gratiam.
${}^{7}$~Subditi ergo estote Deo, resistite autem diabolo, et fugiet a vobis.
${}^{8}$~Appropinquate Deo, et appropinquabit vobis. Emundate manus, peccatores~: et purificate corda, duplices animo.
${}^{9}$~Miseri estote, et lugete, et plorate~: risus vester in luctum convertatur, et gaudium in mœrorem.
${}^{10}$~Humiliamini in conspectu Domini, et exaltabit vos.
${}^{11}$~Nolite detrahere alterutrum fratres. Qui detrahit fratri, aut qui judicat fratrem suum, detrahit legi, et judicat legem. Si autem judicas legem, non es factor legis, sed judex.
${}^{12}$~Unus est legislator et judex, qui potest perdere et liberare.


${}^{13}$~Tu autem quis es, qui judicas proximum~? Ecce nunc qui dicitis~: Hodie, aut crastino ibimus in illam civitatem, et faciemus ibi quidem annum, et mercabimur, et lucrum faciemus~:
${}^{14}$~qui ignoratis quid erit in crastino.
${}^{15}$~Qu\ae\ est enim vita vestra~? vapor est ad modicum parens, et deinceps exterminabitur~; pro eo ut dicatis~: Si Dominus voluerit. Et~: Si vixerimus, faciemus hoc, aut illud.
${}^{16}$~Nunc autem exsultatis in superbiis vestris. Omnis exsultatio talis, maligna est.
${}^{17}$~Scienti igitur bonum facere, et non facienti, peccatum est illi.

\bchapter
\mylettrine{A}gite nunc divites, plorate ululantes in miseriis vestris, qu\ae\ advenient vobis.
${}^{2}$~Diviti\ae\ vestr\ae\ putrefact\ae\ sunt, et vestimenta vestra a tineis comesta sunt.
${}^{3}$~Aurum et argentum vestrum \ae ruginavit~: et \ae rugo eorum in testimonium vobis erit, et manducabit carnes vestras sicut ignis. Thesaurizastis vobis iram in novissimis diebus.
${}^{4}$~Ecce merces operariorum, qui messuerunt regiones vestras, qu\ae\ fraudata est a vobis, clamat~: et clamor eorum in aures Domini sabbaoth introivit.
${}^{5}$~Epulati estis super terram, et in luxuriis enutristis corda vestra in die occisionis.
${}^{6}$~Addixistis, et occidistis justum, et non restitit vobis.


${}^{7}$~Patientes igitur estote, fratres, usque ad adventum Domini. Ecce agricola exspectat pretiosum fructum terr\ae , patienter ferens donec accipiat temporaneum et serotinum.
${}^{8}$~Patientes igitur estote et vos, et confirmate corda vestra~: quoniam adventus Domini appropinquavit.
${}^{9}$~Nolite ingemiscere, fratres, in alterutrum, ut non judicemini. Ecce judex ante januam assistit.
${}^{10}$~Exemplum accipite, fratres, exitus mali, laboris, et patienti\ae , prophetas qui locuti sunt in nomine Domini.
${}^{11}$~Ecce beatificamus eos qui sustinuerunt. Sufferentiam Job audistis, et finem Domini vidistis, quoniam misericors Dominus est, et miserator.


${}^{12}$~Ante omnia autem, fratres mei, nolite jurare, neque per c\ae lum, neque per terram, neque aliud quodcumque juramentum. Sit autem sermo vester~: Est, est~: Non, non~: ut non sub judicio decidatis.


${}^{13}$~Tristatur aliquis vestrum~? oret. \AE quo animo est~? psallat.
${}^{14}$~Infirmatur quis in vobis~? inducat presbyteros ecclesi\ae , et orent super eum, ungentes eum oleo in nomine Domini~:
${}^{15}$~et oratio fidei salvabit infirmum, et alleviabit eum Dominus~: et si in peccatis sit, remittentur ei.
${}^{16}$~Confitemini ergo alterutrum peccata vestra, et orate pro invicem ut salvemini~: multum enim valet deprecatio justi assidua.
${}^{17}$~Elias homo erat similis nobis passibilis~: et oratione oravit ut non plueret super terram, et non pluit annos tres, et menses sex.
${}^{18}$~Et rursum oravit~: et c\ae lum dedit pluviam, et terra dedit fructum suum.


${}^{19}$~Fratres mei, si quis ex vobis erraverit a veritate, et converterit quis eum~:
${}^{20}$~scire debet quoniam qui converti fecerit peccatorem ab errore vi\ae\ su\ae , salvabit animam ejus a morte, et operiet multitudinem peccatorum.
