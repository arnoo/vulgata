\bbook{Liber Genesis}
{Genesis}{images/genese_heading}


\bchapter
\mylettrine{I}n principio creavit Deus c\ae lum et terram.
${}^{2}$~Terra autem erat inanis et vacua, et tenebr\ae\ erant super faciem abyssi~: et spiritus Dei ferebatur super aquas.
${}^{3}$~Dixitque Deus~: Fiat lux. Et facta est lux.
${}^{4}$~Et vidit Deus lucem quod esset bona~: et divisit lucem a tenebris.
${}^{5}$~Appellavitque lucem Diem, et tenebras Noctem~: factumque est vespere et mane, dies unus.
${}^{6}$~Dixit quoque Deus~: Fiat firmamentum in medio aquarum~: et dividat aquas ab aquis.
${}^{7}$~Et fecit Deus firmamentum, divisitque aquas, qu\ae\ erant sub firmamento, ab his, qu\ae\ erant super firmamentum. Et factum est ita.
${}^{8}$~Vocavitque Deus firmamentum, C\ae lum~: et factum est vespere et mane, dies secundus.
${}^{9}$~Dixit vero Deus~: Congregentur aqu\ae , qu\ae\ sub c\ae lo sunt, in locum unum~: et appareat arida. Et factum est ita.
${}^{10}$~Et vocavit Deus aridam Terram, congregationesque aquarum appellavit Maria. Et vidit Deus quod esset bonum.
${}^{11}$~Et ait~: Germinet terra herbam virentem, et facientem semen, et lignum pomiferum faciens fructum juxta genus suum, cujus semen in semetipso sit super terram. Et factum est ita.
${}^{12}$~Et protulit terra herbam virentem, et facientem semen juxta genus suum, lignumque faciens fructum, et habens unumquodque sementem secundum speciem suam. Et vidit Deus quod esset bonum.
${}^{13}$~Et factum est vespere et mane, dies tertius.
${}^{14}$~Dixit autem Deus~: Fiant luminaria in firmamento c\ae li, et dividant diem ac noctem, et sint in signa et tempora, et dies et annos~:
${}^{15}$~ut luceant in firmamento c\ae li, et illuminent terram. Et factum est ita.
${}^{16}$~Fecitque Deus duo luminaria magna~: luminare majus, ut pr\ae esset diei~: et luminare minus, ut pr\ae esset nocti~: et stellas.
${}^{17}$~Et posuit eas in firmamento c\ae li, ut lucerent super terram,
${}^{18}$~et pr\ae essent diei ac nocti, et dividerent lucem ac tenebras. Et vidit Deus quod esset bonum.
${}^{19}$~Et factum est vespere et mane, dies quartus.
${}^{20}$~Dixit etiam Deus~: Producant aqu\ae\ reptile anim\ae\ viventis, et volatile super terram sub firmamento c\ae li.
${}^{21}$~Creavitque Deus cete grandia, et omnem animam viventem atque motabilem, quam produxerant aqu\ae\ in species suas, et omne volatile secundum genus suum. Et vidit Deus quod esset bonum.
${}^{22}$~Benedixitque eis, dicens~: Crescite, et multiplicamini, et replete aquas maris~: avesque multiplicentur super terram.
${}^{23}$~Et factum est vespere et mane, dies quintus.
${}^{24}$~Dixit quoque Deus~: Producat terra animam viventem in genere suo, jumenta, et reptilia, et bestias terr\ae\ secundum species suas. Factumque est ita.
${}^{25}$~Et fecit Deus bestias terr\ae\ juxta species suas, et jumenta, et omne reptile terr\ae\ in genere suo.

 Et vidit Deus quod esset bonum,
${}^{26}$~et ait~: Faciamus hominem ad imaginem et similitudinem nostram~: et pr\ae sit piscibus maris, et volatilibus c\ae li, et bestiis, univers\ae que terr\ae , omnique reptili, quod movetur in terra.
${}^{27}$~Et creavit Deus hominem ad imaginem suam~: ad imaginem Dei creavit illum, masculum et feminam creavit eos.
${}^{28}$~Benedixitque illis Deus, et ait~: Crescite et multiplicamini, et replete terram, et subjicite eam, et dominamini piscibus maris, et volatilibus c\ae li, et universis animantibus, qu\ae\ moventur super terram.
${}^{29}$~Dixitque Deus~: Ecce dedi vobis omnem herbam afferentem semen super terram, et universa ligna qu\ae\ habent in semetipsis sementem generis sui, ut sint vobis in escam~:
${}^{30}$~et cunctis animantibus terr\ae , omnique volucri c\ae li, et universis qu\ae\ moventur in terra, et in quibus est anima vivens, ut habeant ad vescendum. Et factum est ita.
${}^{31}$~Viditque Deus cuncta qu\ae\ fecerat, et erant valde bona. Et factum est vespere et mane, dies sextus.

\bchapter
\mylettrine{I}gitur perfecti sunt c\ae li et terra, et omnis ornatus eorum.
${}^{2}$~Complevitque Deus die septimo opus suum quod fecerat~: et requievit die septimo ab universo opere quod patrarat.
${}^{3}$~Et benedixit diei septimo, et sanctificavit illum, quia in ipso cessaverat ab omni opere suo quod creavit Deus ut faceret.


${}^{4}$~Ist\ae\ sunt generationes c\ae li et terr\ae , quando creata sunt, in die quo fecit Dominus Deus c\ae lum et terram,
${}^{5}$~et omne virgultum agri antequam oriretur in terra, omnemque herbam regionis priusquam germinaret~: non enim pluerat Dominus Deus super terram, et homo non erat qui operaretur terram~:
${}^{6}$~sed fons ascendebat e terra, irrigans universam superficiem terr\ae .
${}^{7}$~Formavit igitur Dominus Deus hominem de limo terr\ae , et inspiravit in faciem ejus spiraculum vit\ae , et factus est homo in animam viventem.
${}^{8}$~Plantaverat autem Dominus Deus paradisum voluptatis a principio, in quo posuit hominem quem formaverat.
${}^{9}$~Produxitque Dominus Deus de humo omne lignum pulchrum visu, et ad vescendum suave lignum etiam vit\ae\ in medio paradisi, lignumque scienti\ae\ boni et mali.
${}^{10}$~Et fluvius egrediebatur de loco voluptatis ad irrigandum paradisum, qui inde dividitur in quatuor capita.
${}^{11}$~Nomen uni Phison~: ipse est qui circuit omnem terram Hevilath, ubi nascitur aurum~:
${}^{12}$~et aurum terr\ae\ illius optimum est~; ibi invenitur bdellium, et lapis onychinus.
${}^{13}$~Et nomen fluvii secundi Gehon~; ipse est qui circumit omnem terram \AE thiopi\ae .
${}^{14}$~Nomen vero fluminis tertii, Tigris~: ipse vadit contra Assyrios. Fluvius autem quartus, ipse est Euphrates.
${}^{15}$~Tulit ergo Dominus Deus hominem, et posuit eum in paradiso voluptatis, ut operaretur, et custodiret illum~:
${}^{16}$~pr\ae cepitque ei, dicens~: Ex omni ligno paradisi comede~;
${}^{17}$~de ligno autem scienti\ae\ boni et mali ne comedas~: in quocumque enim die comederis ex eo, morte morieris.
${}^{18}$~Dixit quoque Dominus Deus~: Non est bonum esse hominem solum~: faciamus ei adjutorium simile sibi.
${}^{19}$~Formatis igitur Dominus Deus de humo cunctis animantibus terr\ae , et universis volatilibus c\ae li, adduxit ea ad Adam, ut videret quid vocaret ea~: omne enim quod vocavit Adam anim\ae\ viventis, ipsum est nomen ejus.
${}^{20}$~Appellavitque Adam nominibus suis cuncta animantia, et universa volatilia c\ae li, et omnes bestias terr\ae~: Ad\ae\ vero non inveniebatur adjutor similis ejus.
${}^{21}$~Immisit ergo Dominus Deus soporem in Adam~: cumque obdormisset, tulit unam de costis ejus, et replevit carnem pro ea.
${}^{22}$~Et \ae dificavit Dominus Deus costam, quam tulerat de Adam, in mulierem~: et adduxit eam ad Adam.
${}^{23}$~Dixitque Adam~: Hoc nunc os ex ossibus meis, et caro de carne mea~: h\ae c vocabitur Virago, quoniam de viro sumpta est.
${}^{24}$~Quam ob rem relinquet homo patrem suum, et matrem, et adh\ae rebit uxori su\ae~: et erunt duo in carne una.
${}^{25}$~Erat autem uterque nudus, Adam scilicet et uxor ejus~: et non erubescebant.

\bchapter
\mylettrine{S}ed et serpens erat callidior cunctis animantibus terr\ae\ qu\ae\ fecerat Dominus Deus. Qui dixit ad mulierem~: Cur pr\ae cepit vobis Deus ut non comederetis de omni ligno paradisi~?
${}^{2}$~Cui respondit mulier~: De fructu lignorum, qu\ae\ sunt in paradiso, vescimur~:
${}^{3}$~de fructu vero ligni quod est in medio paradisi, pr\ae cepit nobis Deus ne comederemus, et ne tangeremus illud, ne forte moriamur.
${}^{4}$~Dixit autem serpens ad mulierem~: Nequaquam morte moriemini.
${}^{5}$~Scit enim Deus quod in quocumque die comederitis ex eo, aperientur oculi vestri, et eritis sicut dii, scientes bonum et malum.
${}^{6}$~Vidit igitur mulier quod bonum esset lignum ad vescendum, et pulchrum oculis, aspectuque delectabile~: et tulit de fructu illius, et comedit~: deditque viro suo, qui comedit.
${}^{7}$~Et aperti sunt oculi amborum~; cumque cognovissent se esse nudos, consuerunt folia ficus, et fecerunt sibi perizomata.
${}^{8}$~Et cum audissent vocem Domini Dei deambulantis in paradiso ad auram post meridiem, abscondit se Adam et uxor ejus a facie Domini Dei in medio ligni paradisi.
${}^{9}$~Vocavitque Dominus Deus Adam, et dixit ei~: Ubi es~?
${}^{10}$~Qui ait~: Vocem tuam audivi in paradiso, et timui, eo quod nudus essem, et abscondi me.
${}^{11}$~Cui dixit~: Quis enim indicavit tibi quod nudus esses, nisi quod ex ligno de quo pr\ae ceperam tibi ne comederes, comedisti~?
${}^{12}$~Dixitque Adam~: Mulier, quam dedisti mihi sociam, dedit mihi de ligno, et comedi.
${}^{13}$~Et dixit Dominus Deus ad mulierem~: Quare hoc fecisti~? Qu\ae\ respondit~: Serpens decepit me, et comedi.


${}^{14}$~Et ait Dominus Deus ad serpentem~: \begin{flushleft}\begin{verse}Quia fecisti hoc,\\ maledictus es inter omnia animantia, et bestias terr\ae~:\\ super pectus tuum gradieris, et terram comedes cunctis diebus vit\ae\ tu\ae .\\
${}^{15}$~Inimicitias ponam inter te et mulierem,\\ et semen tuum et semen illius~:\\ ipsa conteret caput tuum,\\ et tu insidiaberis calcaneo ejus.\end{verse}\end{flushleft}


${}^{16}$~Mulieri quoque dixit~: Multiplicabo \ae rumnas tuas, et conceptus tuos~: in dolore paries filios, et sub viri potestate eris, et ipse dominabitur tui.
${}^{17}$~Ad\ae\ vero dixit~: Quia audisti vocem uxoris tu\ae , et comedisti de ligno, ex quo pr\ae ceperam tibi ne comederes, maledicta terra in opere tuo~: in laboribus comedes ex ea cunctis diebus vit\ae\ tu\ae .
${}^{18}$~Spinas et tribulos germinabit tibi, et comedes herbam terr\ae .
${}^{19}$~In sudore vultus tui vesceris pane, donec revertaris in terram de qua sumptus es~: quia pulvis es et in pulverem reverteris.
${}^{20}$~Et vocavit Adam nomen uxoris su\ae , Heva~: eo quod mater esset cunctorum viventium.
${}^{21}$~Fecit quoque Dominus Deus Ad\ae\ et uxori ejus tunicas pelliceas, et induit eos~:
${}^{22}$~et ait~: Ecce Adam quasi unus ex nobis factus est, sciens bonum et malum~: nunc ergo ne forte mittat manum suam, et sumat etiam de ligno vit\ae , et comedat, et vivat in \ae ternum.
${}^{23}$~Et emisit eum Dominus Deus de paradiso voluptatis, ut operaretur terram de qua sumptus est.
${}^{24}$~Ejecitque Adam~: et collocavit ante paradisum voluptatis cherubim, et flammeum gladium, atque versatilem, ad custodiendam viam ligni vit\ae .

\bchapter
\mylettrine{A}dam vero cognovit uxorem suam Hevam, qu\ae\ concepit et peperit Cain, dicens~: Possedi hominem per Deum.
${}^{2}$~Rursumque peperit fratrem ejus Abel. Fuit autem Abel pastor ovium, et Cain agricola.
${}^{3}$~Factum est autem post multos dies ut offerret Cain de fructibus terr\ae\ munera Domino.
${}^{4}$~Abel quoque obtulit de primogenitis gregis sui, et de adipibus eorum~: et respexit Dominus ad Abel, et ad munera ejus.
${}^{5}$~Ad Cain vero, et ad munera illius non respexit~: iratusque est Cain vehementer, et concidit vultus ejus.
${}^{6}$~Dixitque Dominus ad eum~: Quare iratus es~? et cur concidit facies tua~?
${}^{7}$~nonne si bene egeris, recipies~: sin autem male, statim in foribus peccatum aderit~? sed sub te erit appetitus ejus, et tu dominaberis illius.
${}^{8}$~Dixitque Cain ad Abel fratrem suum~: Egrediamur foras. Cumque essent in agro, consurrexit Cain adversus fratrem suum Abel, et interfecit eum.
${}^{9}$~Et ait Dominus ad Cain~: Ubi est Abel frater tuus~? Qui respondit~: Nescio~: num custos fratris mei sum ego~?
${}^{10}$~Dixitque ad eum~: Quid fecisti~? vox sanguinis fratris tui clamat ad me de terra.
${}^{11}$~Nunc igitur maledictus eris super terram, qu\ae\ aperuit os suum, et suscepit sanguinem fratris tui de manu tua.
${}^{12}$~Cum operatus fueris eam, non dabit tibi fructus suos~: vagus et profugus eris super terram.
${}^{13}$~Dixitque Cain ad Dominum~: Major est iniquitas mea, quam ut veniam merear.
${}^{14}$~Ecce ejicis me hodie a facie terr\ae , et a facie tua abscondar, et ero vagus et profugus in terra~: omnis igitur qui invenerit me, occidet me.
${}^{15}$~Dixitque ei Dominus~: Nequaquam ita fiet~: sed omnis qui occiderit Cain, septuplum punietur. Posuitque Dominus Cain signum, ut non interficeret eum omnis qui invenisset eum.
${}^{16}$~Egressusque Cain a facie Domini, habitavit profugus in terra ad orientalem plagam Eden.


${}^{17}$~Cognovit autem Cain uxorem suam, qu\ae\ concepit, et peperit Henoch~: et \ae dificavit civitatem, vocavitque nomen ejus ex nomine filii sui, Henoch.
${}^{18}$~Porro Henoch genuit Irad, et Irad genuit Mavia\"el, et Mavia\"el genuit Mathusa\"el, et Mathusa\"el genuit Lamech.
${}^{19}$~Qui accepit duas uxores, nomen uni Ada, et nomen alteri Sella.
${}^{20}$~Genuitque Ada Jabel, qui fuit pater habitantium in tentoriis, atque pastorum.
${}^{21}$~Et nomen fratris ejus Jubal~: ipse fuit pater canentium cithara et organo.
${}^{22}$~Sella quoque genuit Tubalcain, qui fuit malleator et faber in cuncta opera \ae ris et ferri. Soror vero Tubalcain, No\"ema.
${}^{23}$~Dixitque Lamech uxoribus suis Ad\ae\ et Sell\ae~: \begin{flushleft}\begin{verse}Audite vocem meam, uxores Lamech~;\\ auscultate sermonem meum~:\\ quoniam occidi virum in vulnus meum,\\ et adolescentulum in livorem meum.\\
${}^{24}$~Septuplum ultio dabitur de Cain~:\\ de Lamech vero septuagies septies.\end{verse}\end{flushleft}


${}^{25}$~Cognovit quoque adhuc Adam uxorem suam~: et peperit filium, vocavitque nomen ejus Seth, dicens~: Posuit mihi Deus semen aliud pro Abel, quem occidit Cain.
${}^{26}$~Sed et Seth natus est filius, quem vocavit Enos~: iste cœpit invocare nomen Domini.

\bchapter
\mylettrine{H}ic est liber generationis Adam. In die qua creavit Deus hominem, ad similitudinem Dei fecit illum.
${}^{2}$~Masculum et feminam creavit eos, et benedixit illis~: et vocavit nomen eorum Adam, in die quo creati sunt.
${}^{3}$~Vixit autem Adam centum triginta annis~: et genuit ad imaginem et similitudinem suam, vocavitque nomen ejus Seth.
${}^{4}$~Et facti sunt dies Adam, postquam genuit Seth, octingenti anni~: genuitque filios et filias.
${}^{5}$~Et factum est omne tempus quod vixit Adam, anni nongenti triginta, et mortuus est.
${}^{6}$~Vixit quoque Seth centum quinque annis, et genuit Enos.
${}^{7}$~Vixitque Seth, postquam genuit Enos, octingentis septem annis, genuitque filios et filias.
${}^{8}$~Et facti sunt omnes dies Seth nongentorum duodecim annorum, et mortuus est.
${}^{9}$~Vixit vero Enos nonaginta annis, et genuit Cainan.
${}^{10}$~Post cujus ortum vixit octingentis quindecim annis, et genuit filios et filias.
${}^{11}$~Factique sunt omnes dies Enos nongenti quinque anni, et mortuus est.
${}^{12}$~Vixit quoque Cainan septuaginta annis, et genuit Malaleel.
${}^{13}$~Et vixit Cainan, postquam genuit Malaleel, octingentis quadraginta annis, genuitque filios et filias.
${}^{14}$~Et facti sunt omnes dies Cainan nongenti decem anni, et mortuus est.
${}^{15}$~Vixit autem Malaleel sexaginta quinque annis, et genuit Jared.
${}^{16}$~Et vixit Malaleel, postquam genuit Jared, octingentis triginta annis, et genuit filios et filias.
${}^{17}$~Et facti sunt omnes dies Malaleel octingenti nonaginta quinque anni, et mortuus est.
${}^{18}$~Vixitque Jared centum sexaginta duobus annis, et genuit Henoch.
${}^{19}$~Et vixit Jared, postquam genuit Henoch, octingentis annis, et genuit filios et filias.
${}^{20}$~Et facti sunt omnes dies Jared nongenti sexaginta duo anni, et mortuus est.
${}^{21}$~Porro Henoch vixit sexaginta quinque annis, et genuit Mathusalam.
${}^{22}$~Et ambulavit Henoch cum Deo~: et vixit, postquam genuit Mathusalam, trecentis annis, et genuit filios et filias.
${}^{23}$~Et facti sunt omnes dies Henoch trecenti sexaginta quinque anni.
${}^{24}$~Ambulavitque cum Deo, et non apparuit~: quia tulit eum Deus.
${}^{25}$~Vixit quoque Mathusala centum octoginta septem annis, et genuit Lamech.
${}^{26}$~Et vixit Mathusala, postquam genuit Lamech, septingentis octoginta duobus annis, et genuit filios et filias.
${}^{27}$~Et facti sunt omnes dies Mathusala nongenti sexaginta novem anni, et mortuus est.
${}^{28}$~Vixit autem Lamech centum octoginta duobus annis, et genuit filium~:
${}^{29}$~vocavitque nomen ejus No\"e, dicens~: Iste consolabitur nos ab operibus et laboribus manuum nostrarum in terra, cui maledixit Dominus.
${}^{30}$~Vixitque Lamech, postquam genuit No\"e, quingentis nonaginta quinque annis, et genuit filios et filias.
${}^{31}$~Et facti sunt omnes dies Lamech septingenti septuaginta septem anni, et mortuus est. No\"e vero cum quingentorum esset annorum, genuit Sem, Cham et Japheth.

\bchapter
\mylettrine{C}umque cœpissent homines multiplicari super terram, et filias procreassent,
${}^{2}$~videntes filii Dei filias hominum quod essent pulchr\ae , acceperunt sibi uxores ex omnibus, quas elegerant.
${}^{3}$~Dixitque Deus~: Non permanebit spiritus meus in homine in \ae ternum, quia caro est~: eruntque dies illius centum viginti annorum.
${}^{4}$~Gigantes autem erant super terram in diebus illis~: postquam enim ingressi sunt filii Dei ad filias hominum, ill\ae que genuerunt, isti sunt potentes a s\ae culo viri famosi.
${}^{5}$~Videns autem Deus quod multa malitia hominum esset in terra, et cuncta cogitatio cordis intenta esset ad malum omni tempore,
${}^{6}$~pœnituit eum quod hominum fecisset in terra. Et tactus dolore cordis intrinsecus,
${}^{7}$~Delebo, inquit, hominem, quem creavi, a facie terr\ae , ab homine usque ad animantia, a reptili usque ad volucres c\ae li~: pœnitet enim me fecisse eos.


${}^{8}$~No\"e vero invenit gratiam coram Domino.
${}^{9}$~H\ae\ sunt generationes No\"e~: No\"e vir justus atque perfectus fuit in generationibus suis~; cum Deo ambulavit.
${}^{10}$~Et genuit tres filios, Sem, Cham et Japheth.
${}^{11}$~Corrupta est autem terra coram Deo, et repleta est iniquitate.
${}^{12}$~Cumque vidisset Deus terram esse corruptam (omnis quippe caro corruperat viam suam super terram),
${}^{13}$~dixit ad No\"e~: Finis univers\ae\ carnis venit coram me~: repleta est terra iniquitate a facie eorum, et ego disperdam eos cum terra.
${}^{14}$~Fac tibi arcam de lignis l\ae vigatis~; mansiunculas in arca facies, et bitumine linies intrinsecus et extrinsecus.
${}^{15}$~Et sic facies eam~: trecentorum cubitorum erit longitudo arc\ae , quinquaginta cubitorum latitudo, et triginta cubitorum altitudo illius.
${}^{16}$~Fenestram in arca facies, et in cubito consummabis summitatem ejus~: ostium autem arc\ae\ pones ex latere~; deorsum, cœnacula et tristega facies in ea.\\
${}^{17}$~Ecce ego adducam aquas diluvii super terram, ut interficiam omnem carnem, in qua spiritus vit\ae\ est subter c\ae lum~: universa qu\ae\ in terra sunt, consumentur.
${}^{18}$~Ponamque fœdus meum tecum~: et ingredieris arcam tu et filii tui, uxor tua, et uxores filiorum tuorum tecum.
${}^{19}$~Et ex cunctis animantibus univers\ae\ carnis bina induces in arcam, ut vivant tecum~: masculini sexus et feminini.
${}^{20}$~De volucribus juxta genus suum, et de jumentis in genere suo, et ex omni reptili terr\ae\ secundum genus suum~: bina de omnibus ingredientur tecum, ut possint vivere.
${}^{21}$~Tolles igitur tecum ex omnibus escis, qu\ae\ mandi possunt, et comportabis apud te~: et erunt tam tibi, quam illis in cibum.
${}^{22}$~Fecit igitur No\"e omnia qu\ae\ pr\ae ceperat illi Deus.

\bchapter
\mylettrine{D}ixitque Dominus ad eum~: Ingredere tu et omnis domus tua in arcam~: te enim vidi justum coram me in generatione hac.
${}^{2}$~Ex omnibus animantibus mundis tolle septena et septena, masculum et feminam~: de animantibus vero immundis duo et duo, masculum et feminam.
${}^{3}$~Sed et de volatilibus c\ae li septena et septena, masculum et feminam~: ut salvetur semen super faciem univers\ae\ terr\ae .
${}^{4}$~Adhuc enim, et post dies septem ego pluam super terram quadraginta diebus et quadraginta noctibus~: et delebo omnem substantiam, quam feci, de superficie terr\ae .
${}^{5}$~Fecit ergo No\"e omnia qu\ae\ mandaverat ei Dominus.
${}^{6}$~Eratque sexcentorum annorum quando diluvii aqu\ae\ inundaverunt super terram.
${}^{7}$~Et ingressus est No\"e et filii ejus, uxor ejus et uxores filiorum ejus cum eo in arcam propter aquas diluvii.
${}^{8}$~De animantibus quoque mundis et immundis, et de volucribus, et ex omni quod movetur super terram,
${}^{9}$~duo et duo ingressa sunt ad No\"e in arcam, masculus et femina, sicut pr\ae ceperat Dominus No\"e.
${}^{10}$~Cumque transissent septem dies, aqu\ae\ diluvii inundaverunt super terram.
${}^{11}$~Anno sexcentesimo vit\ae\ No\"e, mense secundo, septimodecimo die mensis, rupti sunt omnes fontes abyssi magn\ae , et cataract\ae\ c\ae li apert\ae\ sunt~:
${}^{12}$~et facta est pluvia super terram quadraginta diebus et quadraginta noctibus.
${}^{13}$~In articulo diei illius ingressus est No\"e, et Sem, et Cham, et Japheth filii ejus~; uxor illius, et tres uxores filiorum ejus cum eis in arcam~:
${}^{14}$~ipsi et omne animal secundum genus suum, universaque jumenta in genere suo, et omne quod movetur super terram in genere suo, cunctumque volatile secundum genus suum, univers\ae\ aves, omnesque volucres,
${}^{15}$~ingress\ae\ sunt ad No\"e in arcam, bina et bina ex omni carne, in qua erat spiritus vit\ae .
${}^{16}$~Et qu\ae\ ingressa sunt, masculus et femina ex omni carne introierunt, sicut pr\ae ceperat ei Deus~: et inclusit eum Dominus deforis.


${}^{17}$~Factumque est diluvium quadraginta diebus super terram~: et multiplicat\ae\ sunt aqu\ae , et elevaverunt arcam in sublime a terra.
${}^{18}$~Vehementer enim inundaverunt, et omnia repleverunt in superficie terr\ae~: porro arca ferebatur super aquas.
${}^{19}$~Et aqu\ae\ pr\ae valuerunt nimis super terram~: opertique sunt omnes montes excelsi sub universo c\ae lo.
${}^{20}$~Quindecim cubitis altior fuit aqua super montes, quos operuerat.
${}^{21}$~Consumptaque est omnis caro qu\ae\ movebatur super terram, volucrum, animantium, bestiarum, omniumque reptilium, qu\ae\ reptant super terram~: universi homines,
${}^{22}$~et cuncta, in quibus spiraculum vit\ae\ est in terra, mortua sunt.
${}^{23}$~Et delevit omnem substantiam qu\ae\ erat super terram, ab homine usque ad pecus, tam reptile quam volucres c\ae li~: et deleta sunt de terra. Remansit autem solus No\"e, et qui cum eo erant in arca.
${}^{24}$~Obtinueruntque aqu\ae\ terram centum quinquaginta diebus.

\bchapter
\mylettrine{R}ecordatus autem Deus No\"e, cunctorumque animantium, et omnium jumentorum, qu\ae\ erant cum eo in arca, adduxit spiritum super terram, et imminut\ae\ sunt aqu\ae .
${}^{2}$~Et clausi sunt fontes abyssi, et cataract\ae\ c\ae li~: et prohibit\ae\ sunt pluvi\ae\ de c\ae lo.
${}^{3}$~Revers\ae que sunt aqu\ae\ de terra euntes et redeuntes~: et cœperunt minui post centum quinquaginta dies.
${}^{4}$~Requievitque arca mense septimo, vigesimo septimo die mensis, super montes Armeni\ae .
${}^{5}$~At vero aqu\ae\ ibant et decrescebant usque ad decimum mensem~: decimo enim mense, primo die mensis, apparuerunt cacumina montium.
${}^{6}$~Cumque transissent quadraginta dies, aperiens No\"e fenestram arc\ae , quam fecerat, dimisit corvum,
${}^{7}$~qui egrediebatur, et non revertebatur, donec siccarentur aqu\ae\ super terram.
${}^{8}$~Emisit quoque columbam post eum, ut videret si jam cessassent aqu\ae\ super faciem terr\ae .
${}^{9}$~Qu\ae\ cum non invenisset ubi requiesceret pes ejus, reversa est ad eum in arcam~: aqu\ae\ enim erant super universam terram~: extenditque manum, et apprehensam intulit in arcam.
${}^{10}$~Expectatis autem ultra septem diebus aliis, rursum dimisit columbam ex arca.
${}^{11}$~At illa venit ad eum ad vesperam, portans ramum oliv\ae\ virentibus foliis in ore suo~: intellexit ergo No\"e quod cessassent aqu\ae\ super terram.
${}^{12}$~Expectavitque nihilominus septem alios dies~: et emisit columbam, qu\ae\ non est reversa ultra ad eum.
${}^{13}$~Igitur sexcentesimo primo anno, primo mense, prima die mensis, imminut\ae\ sunt aqu\ae\ super terram~: et aperiens No\"e tectum arc\ae , aspexit, viditque quod exsiccata esset superficies terr\ae .
${}^{14}$~Mense secundo, septimo et vigesimo die mensis arefacta est terra.


${}^{15}$~Locutus est autem Deus ad No\"e, dicens~:
${}^{16}$~Egredere de arca, tu et uxor tua, filii tui et uxores filiorum tuorum tecum.
${}^{17}$~Cuncta animantia, qu\ae\ sunt apud te, ex omni carne, tam in volatilibus quam in bestiis et universis reptilibus, qu\ae\ reptant super terram, educ tecum, et ingredimini super terram~: crescite et multiplicamini super eam.
${}^{18}$~Egressus est ergo No\"e, et filii ejus~: uxor illius, et uxores filiorum ejus cum eo.
${}^{19}$~Sed et omnia animantia, jumenta, et reptilia qu\ae\ reptant super terram, secundum genus suum, egressa sunt de arca.
${}^{20}$~\AE dificavit autem No\"e altare Domino~: et tollens de cunctis pecoribus et volucribus mundis, obtulit holocausta super altare.
${}^{21}$~Odoratusque est Dominus odorem suavitatis, et ait~: Nequaquam ultra maledicam terr\ae\ propter homines~: sensus enim et cogitatio humani cordis in malum prona sunt ab adolescentia sua~: non igitur ultra percutiam omnem animam viventem sicut feci.
${}^{22}$~Cunctis diebus terr\ae , sementis et messis, frigus et \ae stus, \ae stas et hiems, nox et dies non requiescent.

\bchapter
\mylettrine{B}enedixitque Deus No\"e et filiis ejus. Et dixit ad eos~: Crescite, et multiplicamini, et replete terram.
${}^{2}$~Et terror vester ac tremor sit super cuncta animalia terr\ae , et super omnes volucres c\ae li, cum universis qu\ae\ moventur super terram~: omnes pisces maris manui vestr\ae\ traditi sunt.
${}^{3}$~Et omne, quod movetur et vivit, erit vobis in cibum~: quasi olera virentia tradidi vobis omnia.
${}^{4}$~Excepto, quod carnem cum sanguine non comedetis.
${}^{5}$~Sanguinem enim animarum vestrarum requiram de manu cunctarum bestiarum~: et de manu hominis, de manu viri, et fratris ejus requiram animam hominis.
${}^{6}$~Quicumque effuderit humanum sanguinem, fundetur sanguis illius~: ad imaginem quippe Dei factus est homo.
${}^{7}$~Vos autem crescite et multiplicamini, et ingredimini super terram, et implete eam.


${}^{8}$~H\ae c quoque dixit Deus ad No\"e, et ad filios ejus cum eo~:
${}^{9}$~Ecce ego statuam pactum meum vobiscum, et cum semine vestro post vos~:
${}^{10}$~et ad omnem animam viventem, qu\ae\ est vobiscum, tam in volucribus quam in jumentis et pecudibus terr\ae\ cunctis, qu\ae\ egressa sunt de arca, et universis bestiis terr\ae .
${}^{11}$~Statuam pactum meum vobiscum, et nequaquam ultra interficietur omnis caro aquis diluvii, neque erit deinceps diluvium dissipans terram.
${}^{12}$~Dixitque Deus~: Hoc signum fœderis quod do inter me et vos, et ad omnem animam viventem, qu\ae\ est vobiscum in generationes sempiternas~:
${}^{13}$~arcum meum ponam in nubibus, et erit signum fœderis inter me et inter terram.
${}^{14}$~Cumque obduxero nubibus c\ae lum, apparebit arcus meus in nubibus~:
${}^{15}$~et recordabor fœderis mei vobiscum, et cum omni anima vivente qu\ae\ carnem vegetat~: et non erunt ultra aqu\ae\ diluvii ad delendum universam carnem.
${}^{16}$~Eritque arcus in nubibus, et videbo illum, et recordabor fœderis sempiterni quod pactum est inter Deum et omnem animam viventem univers\ae\ carnis qu\ae\ est super terram.
${}^{17}$~Dixitque Deus ad No\"e~: Hoc erit signum fœderis, quod constitui inter me et omnem carnem super terram.


${}^{18}$~Erant ergo filii No\"e, qui egressi sunt de arca, Sem, Cham et Japheth~: porro Cham ipse est pater Chanaan.
${}^{19}$~Tres isti filii sunt No\"e~: et ab his disseminatum est omne genus hominum super universam terram.
${}^{20}$~Cœpitque No\"e vir agricola exercere terram, et plantavit vineam.
${}^{21}$~Bibensque vinum inebriatus est, et nudatus in tabernaculo suo.
${}^{22}$~Quod cum vidisset Cham, pater Chanaan, verenda scilicet patris sui esse nudata, nuntiavit duobus fratribus suis foras.
${}^{23}$~At vero Sem et Japheth pallium imposuerunt humeris suis, et incedentes retrorsum, operuerunt verenda patris sui~: faciesque eorum avers\ae\ erant, et patris virilia non viderunt.
${}^{24}$~Evigilans autem No\"e ex vino, cum didicisset qu\ae\ fecerat ei filius suus minor,
${}^{25}$~ait~: \begin{flushleft}\begin{verse}Maledictus Chanaan,\\ servus servorum erit fratribus suis.\end{verse}\end{flushleft}


${}^{26}$~Dixitque~: \begin{flushleft}\begin{verse}Benedictus Dominus Deus Sem,\\ sit Chanaan servus ejus.\\
${}^{27}$~Dilatet Deus Japheth, et habitet in tabernaculis Sem,\\ sitque Chanaan servus ejus.\end{verse}\end{flushleft}


${}^{28}$~Vixit autem No\"e post diluvium trecentis quinquaginta annis.
${}^{29}$~Et impleti sunt omnes dies ejus nongentorum quinquaginta annorum~: et mortuus est.

\bchapter
\mylettrine{H}\ae\ sunt generationes filiorum No\"e, Sem, Cham et Japheth~: natique sunt eis filii post diluvium.
${}^{2}$~Filii Japheth~: Gomer, et Magog, et Madai, et Javan, et Thubal, et Mosoch, et Thiras.
${}^{3}$~Porro filii Gomer~: Ascenez et Riphath et Thogorma.
${}^{4}$~Filii autem Javan~: Elisa et Tharsis, Cetthim et Dodanim.
${}^{5}$~Ab his divis\ae\ sunt insul\ae\ gentium in regionibus suis, unusquisque secundum linguam suam et familias suas in nationibus suis.


${}^{6}$~Filii autem Cham~: Chus, et Mesraim, et Phuth, et Chanaan.
${}^{7}$~Filii Chus~: Saba, et Hevila, et Sabatha, et Regma, et Sabatacha. Filii Regma~: Saba et Dadan.
${}^{8}$~Porro Chus genuit Nemrod~: ipse cœpit esse potens in terra,
${}^{9}$~et erat robustus venator coram Domino. Ob hoc exivit proverbium~: Quasi Nemrod robustus venator coram Domino.
${}^{10}$~Fuit autem principium regni ejus Babylon, et Arach et Achad, et Chalanne, in terra Sennaar.
${}^{11}$~De terra illa egressus est Assur, et \ae dificavit Niniven, et plateas civitatis, et Chale.
${}^{12}$~Resen quoque inter Niniven et Chale~: h\ae c est civitas magna.
${}^{13}$~At vero Mesraim genuit Ludim, et Anamim et Laabim, Nephthuim,
${}^{14}$~et Phetrusim, et Chasluim~: de quibus egressi sunt Philisthiim et Caphtorim.
${}^{15}$~Chanaan autem genuit Sidonem primogenitum suum. Heth\ae um,
${}^{16}$~et Jebus\ae um, et Amorrh\ae um, Gerges\ae um,
${}^{17}$~Hev\ae um, et Arac\ae um~: Sin\ae um,
${}^{18}$~et Aradium, Samar\ae um, et Amath\ae um~: et post h\ae c disseminati sunt populi Chanan\ae orum.
${}^{19}$~Factique sunt termini Chanaan venientibus a Sidone Geraram usque Gazam, donec ingrediaris Sodomam et Gomorrham, et Adamam, et Seboim usque Lesa.
${}^{20}$~Hi sunt filii Cham in cognationibus, et linguis, et generationibus, terrisque et gentibus suis.


${}^{21}$~De Sem quoque nati sunt, patre omnium filiorum Heber, fratre Japheth majore.
${}^{22}$~Filii Sem~: \AE lam, et Assur, et Arphaxad, et Lud, et Aram.
${}^{23}$~Filii Aram~: Us, et Hul, et Gether, et Mes.
${}^{24}$~At vero Arphaxad genuit Sale, de quo ortus est Heber.
${}^{25}$~Natique sunt Heber filii duo~: nomen uni Phaleg, eo quod in diebus ejus divisa sit terra~: et nomen fratris ejus Jectan.
${}^{26}$~Qui Jectan genuit Elmodad, et Saleph, et Asarmoth, Jare,
${}^{27}$~et Aduram, et Uzal, et Decla,
${}^{28}$~et Ebal, et Abima\"el, Saba,
${}^{29}$~et Ophir, et Hevila, et Jobab~: omnes isti, filii Jectan.
${}^{30}$~Et facta est habitatio eorum de Messa pergentibus usque Sephar montem orientalem.
${}^{31}$~Isti filii Sem secundum cognationes, et linguas, et regiones in gentibus suis.
${}^{32}$~H\ae\ famili\ae\ No\"e juxta populos et nationes suas. Ab his divis\ae\ sunt gentes in terra post diluvium.

\bchapter
\mylettrine{E}rat autem terra labii unius, et sermonum eorumdem.
${}^{2}$~Cumque proficiscerentur de oriente, invenerunt campum in terra Sennaar, et habitaverunt in eo.
${}^{3}$~Dixitque alter ad proximum suum~: Venite, faciamus lateres, et coquamus eos igni. Habueruntque lateres pro saxis, et bitumen pro c\ae mento~:
${}^{4}$~et dixerunt~: Venite, faciamus nobis civitatem et turrim, cujus culmen pertingat ad c\ae lum~: et celebremus nomen nostrum antequam dividamur in universas terras.
${}^{5}$~Descendit autem Dominus ut videret civitatem et turrim, quam \ae dificabant filii Adam,
${}^{6}$~et dixit~: Ecce, unus est populus, et unum labium omnibus~: cœperuntque hoc facere, nec desistent a cogitationibus suis, donec eas opere compleant.
${}^{7}$~Venite igitur, descendamus, et confundamus ibi linguam eorum, ut non audiat unusquisque vocem proximi sui.
${}^{8}$~Atque ita divisit eos Dominus ex illo loco in universas terras, et cessaverunt \ae dificare civitatem.
${}^{9}$~Et idcirco vocatum est nomen ejus Babel, quia ibi confusum est labium univers\ae\ terr\ae~: et inde dispersit eos Dominus super faciem cunctarum regionum.


${}^{10}$~H\ae\ sunt generationes Sem~: Sem erat centum annorum quando genuit Arphaxad, biennio post diluvium.
${}^{11}$~Vixitque Sem, postquam genuit Arphaxad, quingentis annis~: et genuit filios et filias.
${}^{12}$~Porro Arphaxad vixit triginta quinque annis, et genuit Sale.
${}^{13}$~Vixitque Arphaxad, postquam genuit Sale, trecentis tribus annis~: et genuit filios et filias.
${}^{14}$~Sale quoque vixit triginta annis, et genuit Heber.
${}^{15}$~Vixitque Sale, postquam genuit Heber, quadringentis tribus annis~: et genuit filios et filias.
${}^{16}$~Vixit autem Heber triginta quatuor annis, et genuit Phaleg.
${}^{17}$~Et vixit Heber postquam genuit Phaleg, quadringentis triginta annis~: et genuit filios et filias.
${}^{18}$~Vixit quoque Phaleg triginta annis, et genuit Reu.
${}^{19}$~Vixitque Phaleg, postquam genuit Reu, ducentis novem annis~: et genuit filios et filias.
${}^{20}$~Vixit autem Reu triginta duobus annis, et genuit Sarug.
${}^{21}$~Vixit quoque Reu, postquam genuit Sarug, ducentis septem annis~: et genuit filios et filias.
${}^{22}$~Vixit vero Sarug triginta annis, et genuit Nachor.
${}^{23}$~Vixitque Sarug, postquam genuit Nachor, ducentis annis~: et genuit filios et filias.
${}^{24}$~Vixit autem Nachor viginti novem annis, et genuit Thare.
${}^{25}$~Vixitque Nachor, postquam genuit Thare, centum decem et novem annis~: et genuit filios et filias.
${}^{26}$~Vixitque Thare septuaginta annis, et genuit Abram, et Nachor, et Aran.


${}^{27}$~H\ae\ sunt autem generationes Thare~: Thare genuit Abram, Nachor et Aran. Porro Aran genuit Lot.
${}^{28}$~Mortuusque est Aran ante Thare patrem suum, in terra nativitatis su\ae , in Ur Chald\ae orum.
${}^{29}$~Duxerunt autem Abram et Nachor uxores~: nomen uxoris Abram, Sarai~: et nomen uxoris Nachor, Melcha filia Aran, patris Melch\ae , et patris Jesch\ae .
${}^{30}$~Erat autem Sarai sterilis, nec habebat liberos.
${}^{31}$~Tulit itaque Thare Abram filium suum, et Lot filium Aran, filium filii sui, et Sarai nurum suam, uxorem Abram filii sui, et eduxit eos de Ur Chald\ae orum, ut irent in terram Chanaan~: veneruntque usque Haran, et habitaverunt ibi.
${}^{32}$~Et facti sunt dies Thare ducentorum quinque annorum, et mortuus est in Haran.

\bchapter
\mylettrine{D}ixit autem Dominus ad Abram~: Egredere de terra tua, et de cognatione tua, et de domo patris tui, et veni in terram quam monstrabo tibi.
${}^{2}$~Faciamque te in gentem magnam, et benedicam tibi, et magnificabo nomen tuum, erisque benedictus.
${}^{3}$~Benedicam benedicentibus tibi, et maledicam maledicentibus tibi, atque in te benedicentur univers\ae\ cognationes terr\ae .
${}^{4}$~Egressus est itaque Abram sicut pr\ae ceperat ei Dominus, et ivit cum eo Lot~: septuaginta quinque annorum erat Abram cum egrederetur de Haran.
${}^{5}$~Tulitque Sarai uxorem suam, et Lot filium fratris sui, universamque substantiam quam possederant, et animas quas fecerant in Haran~: et egressi sunt ut irent in terram Chanaan. Cumque venissent in eam,
${}^{6}$~pertransivit Abram terram usque ad locum Sichem, usque ad convallem illustrem~: Chanan\ae us autem tunc erat in terra.
${}^{7}$~Apparuit autem Dominus Abram, et dixit ei~: Semini tuo dabo terram hanc. Qui \ae dificavit ibi altare Domino, qui apparuerat ei.
${}^{8}$~Et inde transgrediens ad montem, qui erat contra orientem Bethel, tetendit ibi tabernaculum suum, ab occidente habens Bethel, et ab oriente Hai~: \ae dificavit quoque ibi altare Domino, et invocavit nomen ejus.
${}^{9}$~Perrexitque Abram vadens, et ultra progrediens ad meridiem.


${}^{10}$~Facta est autem fames in terra~: descenditque Abram in \AE gyptum, ut peregrinaretur ibi~: pr\ae valuerat enim fames in terra.
${}^{11}$~Cumque prope esset ut ingrederetur \AE gyptum, dixit Sarai uxori su\ae~: Novi quod pulchra sis mulier~:
${}^{12}$~et quod cum viderint te \AE gyptii, dicturi sunt~: Uxor ipsius est~: et interficient me, et te reservabunt.
${}^{13}$~Dic ergo, obsecro te, quod soror mea sis~: ut bene sit mihi propter te, et vivat anima mea ob gratiam tui.
${}^{14}$~Cum itaque ingressus esset Abram \AE gyptum, viderunt \AE gyptii mulierem quod esset pulchra nimis.
${}^{15}$~Et nuntiaverunt principes Pharaoni, et laudaverunt eam apud illum~: et sublata est mulier in domum Pharaonis.
${}^{16}$~Abram vero bene usi sunt propter illam~: fueruntque ei oves et boves et asini, et servi et famul\ae , et asin\ae\ et cameli.
${}^{17}$~Flagellavit autem Dominus Pharaonem plagis maximis, et domum ejus, propter Sarai uxorem Abram.
${}^{18}$~Vocavitque Pharao Abram, et dixit ei~: Quidnam est hoc quod fecisti mihi~? quare non indicasti quod uxor tua esset~?
${}^{19}$~quam ob causam dixisti esse sororem tuam, ut tollerem eam mihi in uxorem~? Nunc igitur ecce conjux tua, accipe eam, et vade.
${}^{20}$~Pr\ae cepitque Pharao super Abram viris~: et deduxerunt eum, et uxorem illius, et omnia qu\ae\ habebat.

\bchapter
\mylettrine{A}scendit ergo Abram de \AE gypto, ipse et uxor ejus, et omnia qu\ae\ habebat, et Lot cum eo, ad australem plagam.
${}^{2}$~Erat autem dives valde in possessione auri et argenti.
${}^{3}$~Reversusque est per iter, quo venerat, a meridie in Bethel, usque ad locum ubi prius fixerat tabernaculum inter Bethel et Hai,
${}^{4}$~in loco altaris quod fecerat prius~: et invocavit ibi nomen Domini.
${}^{5}$~Sed et Lot qui erat cum Abram, fuerunt greges ovium, et armenta, et tabernacula.
${}^{6}$~Nec poterat eos capere terra, ut habitarent simul~: erat quippe substantia eorum multa, et nequibant habitare communiter.
${}^{7}$~Unde et facta est rixa inter pastores gregum Abram et Lot. Eo autem tempore Chanan\ae us et Pherez\ae us habitabant in terra illa.
${}^{8}$~Dixit ergo Abram ad Lot~: Ne qu\ae so sit jurgium inter me et te, et inter pastores meos et pastores tuos~: fratres enim sumus.
${}^{9}$~Ecce universa terra coram te est~: recede a me, obsecro~: si ad sinistram ieris, ego dexteram tenebo~: si tu dexteram elegeris, ego ad sinistram pergam.
${}^{10}$~Elevatis itaque Lot oculis, vidit omnem circa regionem Jordanis, qu\ae\ universa irrigabatur antequam subverteret Dominus Sodomam et Gomorrham, sicut paradisus Domini, et sicut \AE gyptus venientibus in Segor.
${}^{11}$~Elegitque sibi Lot regionem circa Jordanem, et recessit ab oriente~: divisique sunt alterutrum a fratre suo.
${}^{12}$~Abram habitavit in terra Chanaan~; Lot vero moratus est in oppidis, qu\ae\ erant circa Jordanem, et habitavit in Sodomis.
${}^{13}$~Homines autem Sodomit\ae\ pessimi erant, et peccatores coram Domino nimis.


${}^{14}$~Dixitque Dominus ad Abram, postquam divisus est ab eo Lot~: Leva oculos tuos et vide a loco, in quo nunc es, ad aquilonem et meridiem, ad orientem et occidentem.
${}^{15}$~Omnem terram, quam conspicis, tibi dabo, et semini tuo usque in sempiternum.
${}^{16}$~Faciamque semen tuum sicut pulverem terr\ae~: si quis potest hominum numerare pulverem terr\ae , semen quoque tuum numerare poterit.
${}^{17}$~Surge, et perambula terram in longitudine et in latitudine sua~: quia tibi daturus sum eam.
${}^{18}$~Movens igitur tabernaculum suum Abram, venit, et habitavit juxta convallem Mambre, qu\ae\ est in Hebron~: \ae dificavitque ibi altare Domino.

\bchapter
\mylettrine{F}actum est autem in illo tempore, ut Amraphel rex Sennaar, et Arioch rex Ponti, et Chodorlahomor rex Elamitarum, et Thadal rex gentium
${}^{2}$~inirent bellum contra Bara regem Sodomorum, et contra Bersa regem Gomorrh\ae , et contra Sennaab regem Adam\ae , et contra Semeber regem Seboim, contraque regem Bal\ae , ipsa est Segor.
${}^{3}$~Omnes hi convenerunt in vallem Silvestrem, qu\ae\ nunc est mare salis.
${}^{4}$~Duodecim enim annis servierunt Chodorlahomor, et tertiodecimo anno recesserunt ab eo.
${}^{5}$~Igitur quartodecimo anno venit Chodorlahomor, et reges qui erant cum eo~: percusseruntque Raphaim in Astarothcarnaim, et Zuzim cum eis, et Emim in Save Cariathaim,
${}^{6}$~et Chorr\ae os in montibus Seir, usque ad Campestria Pharan, qu\ae\ est in solitudine.
${}^{7}$~Reversique sunt, et venerunt ad fontem Misphat, ipsa est Cades~: et percusserunt omnem regionem Amalecitarum, et Amorrh\ae um, qui habitabat in Asasonthamar.
${}^{8}$~Et egressi sunt rex Sodomorum, et rex Gomorrh\ae , rexque Adam\ae , et rex Seboim, necnon et rex Bal\ae , qu\ae\ est Segor~: et direxerunt aciem contra eos in valle Silvestri~:
${}^{9}$~scilicet adversus Chodorlahomor regem Elamitarum, et Thadal regem Gentium, et Amraphel regem Sennaar, et Arioch regem Ponti~: quatuor reges adversus quinque.
${}^{10}$~Vallis autem Silvestris habebat puteos multos bituminis. Itaque rex Sodomorum, et Gomorrh\ae , terga verterunt, cecideruntque ibi~: et qui remanserant, fugerunt ad montem.
${}^{11}$~Tulerunt autem omnem substantiam Sodomorum et Gomorrh\ae , et universa qu\ae\ ad cibum pertinent, et abierunt~:
${}^{12}$~necnon et Lot, et substantiam ejus, filium fratris Abram, qui habitabat in Sodomis.


${}^{13}$~Et ecce unus, qui evaserat, nuntiavit Abram Hebr\ae o, qui habitabat in convalle Mambre Amorrh\ae i, fratris Escol, et fratris Aner~: hi enim pepigerant fœdus cum Abram.
${}^{14}$~Quod cum audisset Abram, captum videlicet Lot fratrem suum, numeravit expeditos vernaculos suos trecentos decem et octo~: et persecutus est usque Dan.
${}^{15}$~Et divisis sociis, irruit super eos nocte~: percussitque eos, et persecutus est eos usque Hoba, qu\ae\ est ad l\ae vam Damasci.
${}^{16}$~Reduxitque omnem substantiam, et Lot fratrem suum cum substantia illius, mulieres quoque et populum.
${}^{17}$~Egressus est autem rex Sodomorum in occursum ejus postquam reversus est a c\ae de Chodorlahomor, et regum qui cum eo erant in valle Save, qu\ae\ est vallis regis.


${}^{18}$~At vero Melchisedech rex Salem, proferens panem et vinum, erat enim sacerdos Dei altissimi,
${}^{19}$~benedixit ei, et ait~: Benedictus Abram Deo excelso, qui creavit c\ae lum et terram~:
${}^{20}$~et benedictus Deus excelsus, quo protegente, hostes in manibus tuis sunt. Et dedit ei decimas ex omnibus.
${}^{21}$~Dixit autem rex Sodomorum ad Abram~: Da mihi animas, cetera tolle tibi.
${}^{22}$~Qui respondit ei~: Levo manum meam ad Dominum Deum excelsum possessorem c\ae li et terr\ae ,
${}^{23}$~quod a filo subtegminis usque ad corigiam calig\ae , non accipiam ex omnibus qu\ae\ tua sunt, ne dicas~: Ego ditavi Abram~:
${}^{24}$~exceptis his, qu\ae\ comederunt juvenes, et partibus virorum, qui venerunt mecum, Aner, Escol et Mambre~: isti accipient partes suas.

\bchapter
\mylettrine{H}is itaque transactis, factus est sermo Domini ad Abram per visionem dicens~: Noli timere, Abram~: ego protector tuus sum, et merces tua magna nimis.
${}^{2}$~Dixitque Abram~: Domine Deus, quid dabis mihi~? ego vadam absque liberis, et filius procuratoris domus me\ae\ iste Damascus Eliezer.
${}^{3}$~Addiditque Abram~: Mihi autem non dedisti semen, et ecce vernaculus meus, h\ae res meus erit.
${}^{4}$~Statimque sermo Domini factus est ad eum, dicens~: Non erit hic h\ae res tuus, sed qui egredietur de utero tuo, ipsum habebis h\ae redem.
${}^{5}$~Eduxitque eum foras, et ait illi~: Suspice c\ae lum, et numera stellas, si potes. Et dixit ei~: Sic erit semen tuum.
${}^{6}$~Credidit Abram Deo, et reputatum est illi ad justitiam.
${}^{7}$~Dixitque ad eum~: Ego Dominus qui eduxi te de Ur Chald\ae orum ut darem tibi terram istam, et possideres eam.
${}^{8}$~At ille ait~: Domine Deus, unde scire possum quod possessurus sim eam~?
${}^{9}$~Et respondens Dominus~: Sume, inquit, mihi vaccam triennem, et capram trimam, et arietem annorum trium, turturem quoque et columbam.
${}^{10}$~Qui tollens universa h\ae c, divisit ea per medium, et utrasque partes contra se altrinsecus posuit~; aves autem non divisit.
${}^{11}$~Descenderuntque volucres super cadavera, et abigebat eas Abram.
${}^{12}$~Cumque sol occumberet, sopor irruit super Abram, et horror magnus et tenebrosus invasit eum.
${}^{13}$~Dictumque est ad eum~: Scito pr\ae noscens quod peregrinum futurum sit semen tuum in terra non sua, et subjicient eos servituti, et affligent quadringentis annis.
${}^{14}$~Verumtamen gentem, cui servituri sunt, ego judicabo~: et post h\ae c egredientur cum magna substantia.
${}^{15}$~Tu autem ibis ad patres tuos in pace, sepultus in senectute bona.
${}^{16}$~Generatione autem quarta revertentur huc~: necdum enim complet\ae\ sunt iniquitates Amorrh\ae orum usque ad pr\ae sens tempus.
${}^{17}$~Cum ergo occubuisset sol, facta est caligo tenebrosa, et apparuit clibanus fumans, et lampas ignis transiens inter divisiones illas.
${}^{18}$~In illo die pepigit Dominus fœdus cum Abram, dicens~: Semini tuo dabo terram hanc a fluvio \AE gypti usque ad fluvium magnum Euphraten,
${}^{19}$~Cin\ae os, et Cenez\ae os, Cedmon\ae os,
${}^{20}$~et Heth\ae os, et Pherez\ae os, Raphaim quoque,
${}^{21}$~et Amorrh\ae os, et Chanan\ae os, et Gerges\ae os, et Jebus\ae os.

\bchapter
\mylettrine{I}gitur Sarai, uxor Abram, non genuerat liberos~: sed habens ancillam \ae gyptiam nomine Agar,
${}^{2}$~dixit marito suo~: Ecce, conclusit me Dominus, ne parerem. Ingredere ad ancillam meam, si forte saltem ex illa suscipiam filios. Cumque ille acquiesceret deprecanti,
${}^{3}$~tulit Agar \ae gyptiam ancillam suam post annos decem quam habitare cœperant in terra Chanaan~: et dedit eam viro suo uxorem.
${}^{4}$~Qui ingressus est ad eam. At illa concepisse se videns, despexit dominam suam.
${}^{5}$~Dixitque Sarai ad Abram~: Inique agis contra me~: ego dedi ancillam meam in sinum tuum, qu\ae\ videns quod conceperit, despectui me habet~: judicet Dominus inter me et te.
${}^{6}$~Cui respondens Abram~: Ecce, ait, ancilla tua in manu tua est, utere ea ut libet. Affligente igitur eam Sarai, fugam iniit.
${}^{7}$~Cumque invenisset eam angelus Domini juxta fontem aqu\ae\ in solitudine, qui est in via Sur in deserto,
${}^{8}$~dixit ad illam~: Agar ancilla Sarai, unde venis~? et quo vadis~? Qu\ae\ respondit~: A facie Sarai domin\ae\ me\ae\ ego fugio.
${}^{9}$~Dixitque ei angelus Domini~: Revertere ad dominam tuam, et humiliare sub manu illius.
${}^{10}$~Et rursum~: Multiplicans, inquit, multiplicabo semen tuum, et non numerabitur pr\ae\ multitudine.
${}^{11}$~Ac deinceps~: Ecce, ait, concepisti, et paries filium~: vocabisque nomen ejus Isma\"el, eo quod audierit Dominus afflictionem tuam.
${}^{12}$~Hic erit ferus homo~: manus ejus contra omnes, et manus omnium contra eum~: et e regione universorum fratrum suorum figet tabernacula.
${}^{13}$~Vocavit autem nomen Domini qui loquebatur ad eam~: Tu Deus qui vidisti me. Dixit enim~: Profecto hic vidi posteriora videntis me.
${}^{14}$~Propterea appellavit puteum illum Puteum viventis et videntis me. Ipse est inter Cades et Barad.
${}^{15}$~Peperitque Agar Abr\ae\ filium~: qui vocavit nomen ejus Isma\"el.
${}^{16}$~Octoginta et sex annorum erat Abram quando peperit ei Agar Isma\"elem.

\bchapter
\mylettrine{P}ostquam vero nonaginta et novem annorum esse cœperat, apparuit ei Dominus, dixitque ad eum~: Ego Deus omnipotens~: ambula coram me, et esto perfectus.
${}^{2}$~Ponamque fœdus meum inter me et te, et multiplicabo te vehementer nimis.
${}^{3}$~Cecidit Abram pronus in faciem.
${}^{4}$~Dixitque ei Deus~: Ego sum, et pactum meum tecum, erisque pater multarum gentium.
${}^{5}$~Nec ultra vocabitur nomen tuum Abram, sed appellaberis Abraham~: quia patrem multarum gentium constitui te.
${}^{6}$~Faciamque te crescere vehementissime, et ponam te in gentibus, regesque ex te egredientur.
${}^{7}$~Et statuam pactum meum inter me et te, et inter semen tuum post te in generationibus suis, fœdere sempiterno~: ut sim Deus tuus, et seminis tui post te.
${}^{8}$~Daboque tibi et semini tuo terram peregrinationis tu\ae , omnem terram Chanaan in possessionem \ae ternam, eroque Deus eorum.


${}^{9}$~Dixit iterum Deus ad Abraham~: Et tu ergo custodies pactum meum, et semen tuum post te in generationibus suis.
${}^{10}$~Hoc est pactum meum quod observabitis inter me et vos, et semen tuum post te~: circumcidetur ex vobis omne masculinum~:
${}^{11}$~et circumcidetis carnem pr\ae putii vestri, ut sit in signum fœderis inter me et vos.
${}^{12}$~Infans octo dierum circumcidetur in vobis, omne masculinum in generationibus vestris~: tam vernaculus, quam emptitius circumcidetur, et quicumque non fuerit de stirpe vestra~:
${}^{13}$~eritque pactum meum in carne vestra in fœdus \ae ternum.
${}^{14}$~Masculus, cujus pr\ae putii caro circumcisa non fuerit, delebitur anima illa de populo suo~: quia pactum meum irritum fecit.


${}^{15}$~Dixit quoque Deus ad Abraham~: Sarai uxorem tuam non vocabis Sarai, sed Saram.
${}^{16}$~Et benedicam ei, et ex illa dabo tibi filium cui benedicturus sum~: eritque in nationes, et reges populorum orientur ex eo.
${}^{17}$~Cecidit Abraham in faciem suam, et risit, dicens in corde suo~: Putasne centenario nascetur filius~? et Sara nonagenaria pariet~?
${}^{18}$~Dixitque ad Deum~: Utinam Isma\"el vivat coram te.
${}^{19}$~Et ait Deus ad Abraham~: Sara uxor tua pariet tibi filium, vocabisque nomen ejus Isaac, et constituam pactum meum illi in fœdus sempiternum, et semini ejus post eum.
${}^{20}$~Super Isma\"el quoque exaudivi te~: ecce, benedicam ei, et augebo, et multiplicabo eum valde~: duodecim duces generabit, et faciam illum in gentem magnam.
${}^{21}$~Pactum vero meum statuam ad Isaac, quem pariet tibi Sara tempore isto in anno altero.
${}^{22}$~Cumque finitus esset sermo loquentis cum eo, ascendit Deus ab Abraham.


${}^{23}$~Tulit autem Abraham Isma\"el filium suum, et omnes vernaculos domus su\ae , universosque quos emerat, cunctos mares ex omnibus viris domus su\ae~: et circumcidit carnem pr\ae putii eorum statim in ipsa die, sicut pr\ae ceperat ei Deus.
${}^{24}$~Abraham nonaginta et novem erat annorum quando circumcidit carnem pr\ae putii sui.
${}^{25}$~Et Isma\"el filius tredecim annos impleverat tempore circumcisionis su\ae .
${}^{26}$~Eadem die circumcisus est Abraham et Isma\"el filius ejus~:
${}^{27}$~et omnes viri domus illius, tam vernaculi, quam emptitii et alienigen\ae\ pariter circumcisi sunt.

\bchapter
\mylettrine{A}pparuit autem ei Dominus in convalle Mambre sedenti in ostio tabernaculi sui in ipso fervore diei.
${}^{2}$~Cumque elevasset oculos, apparuerunt ei tres viri stantes prope eum~: quos cum vidisset, cucurrit in occursum eorum de ostio tabernaculi, et adoravit in terram.
${}^{3}$~Et dixit~: Domine, si inveni gratiam in oculis tuis, ne transeas servum tuum~:
${}^{4}$~sed afferam pauxillum aqu\ae , et lavate pedes vestros, et requiescite sub arbore.
${}^{5}$~Ponamque buccellam panis, et confortate cor vestrum~: postea transibitis~: idcirco enim declinastis ad servum vestrum. Qui dixerunt~: Fac ut locutus es.
${}^{6}$~Festinavit Abraham in tabernaculum ad Saram, dixitque ei~: Accelera, tria sata simil\ae\ commisce, et fac subcinericios panes.
${}^{7}$~Ipse vero ad armentum cucurrit, et tulit inde vitulum tenerrimum et optimum, deditque puero~: qui festinavit et coxit illum.
${}^{8}$~Tulit quoque butyrum et lac, et vitulum quem coxerat, et posuit coram eis~: ipse vero stabat juxta eos sub arbore.


${}^{9}$~Cumque comedissent, dixerunt ad eum~: Ubi est Sara uxor tua~? Ille respondit~: Ecce in tabernaculo est.
${}^{10}$~Cui dixit~: Revertens veniam ad te tempore isto, vita comite, et habebit filium Sara uxor tua. Quo audito, Sara risit post ostium tabernaculi.
${}^{11}$~Erant autem ambo senes, provect\ae que \ae tatis, et desierant Sar\ae\ fieri muliebria.
${}^{12}$~Qu\ae\ risit occulte dicens~: Postquam consenui, et dominus meus vetulus est, voluptati operam dabo~?
${}^{13}$~Dixit autem Dominus ad Abraham~: Quare risit Sara, dicens~: Num vere paritura sum anus~?
${}^{14}$~Numquid Deo quidquam est difficile~? juxta condictum revertar ad te hoc eodem tempore, vita comite, et habebit Sara filium.
${}^{15}$~Negavit Sara, dicens~: Non risi, timore perterrita. Dominus autem~: Non est, inquit, ita~: sed risisti.


${}^{16}$~Cum ergo surrexissent inde viri, direxerunt oculos contra Sodomam~: et Abraham simul gradiebatur, deducens eos.
${}^{17}$~Dixitque Dominus~: Num celare potero Abraham qu\ae\ gesturus sum~:
${}^{18}$~cum futurus sit in gentem magnam, ac robustissimam, et benedicend\ae\ sint in illo omnes nationes terr\ae~?
${}^{19}$~Scio enim quod pr\ae cepturus sit filiis suis, et domui su\ae\ post se ut custodiant viam Domini, et faciant judicium et justitiam~: ut adducat Dominus propter Abraham omnia qu\ae\ locutus est ad eum.
${}^{20}$~Dixit itaque Dominus~: Clamor Sodomorum et Gomorrh\ae\ multiplicatus est, et peccatum eorum aggravatum est nimis.
${}^{21}$~Descendam, et videbo utrum clamorem qui venit ad me, opere compleverint~; an non est ita, ut sciam.
${}^{22}$~Converteruntque se inde, et abierunt Sodomam~: Abraham vero adhuc stabat coram Domino.
${}^{23}$~Et appropinquans ait~: Numquid perdes justum cum impio~?
${}^{24}$~si fuerint quinquaginta justi in civitate, peribunt simul~? et non parces loco illi propter quinquaginta justos, si fuerint in eo~?
${}^{25}$~Absit a te ut rem hanc facias, et occidas justum cum impio, fiatque justus sicut impius, non est hoc tuum~: qui judicas omnem terram, nequaquam facies judicium hoc.
${}^{26}$~Dixitque Dominus ad eum~: Si invenero Sodomis quinquaginta justos in medio civitatis, dimittam omni loco propter eos.
${}^{27}$~Respondensque Abraham, ait~: Quia semel cœpi, loquar ad Dominum meum, cum sim pulvis et cinis.
${}^{28}$~Quid si minus quinquaginta justis quinque fuerint~? delebis, propter quadraginta quinque, universam urbem~? Et ait~: Non delebo, si invenero ibi quadraginta quinque.
${}^{29}$~Rursumque locutus est ad eum~: Sin autem quadraginta ibi inventi fuerint, quid facies~? Ait~: Non percutiam propter quadraginta.
${}^{30}$~Ne qu\ae so, inquit, indigneris, Domine, si loquar~: quid si ibi inventi fuerint triginta~? Respondit~: Non faciam, si invenero ibi triginta.
${}^{31}$~Quia semel, ait, cœpi loquar ad Dominum meum~: quid si ibi inventi fuerint viginti~? Ait~: Non interficiam propter viginti.
${}^{32}$~Obsecro, inquit, ne irascaris, Domine, si loquar adhuc semel~: quid si inventi fuerint ibi decem~? Et dixit~: Non delebo propter decem.
${}^{33}$~Abiitque Dominus, postquam cessavit loqui ad Abraham~: et ille reversus est in locum suum.

\bchapter
\mylettrine{V}eneruntque duo angeli Sodomam vespere, et sedente Lot in foribus civitatis. Qui cum vidisset eos, surrexit, et ivit obviam eis~: adoravitque pronus in terram,
${}^{2}$~et dixit~: Obsecro, domini, declinate in domum pueri vestri, et manete ibi~: lavate pedes vestros, et mane proficiscemini in viam vestram. Qui dixerunt~: Minime, sed in platea manebimus.
${}^{3}$~Compulit illos oppido ut diverterent ad eum~: ingressisque domum illius fecit convivium, et coxit azyma, et comederunt.
${}^{4}$~Prius autem quam irent cubitum, viri civitatis vallaverunt domum a puero usque ad senem, omnis populus simul.
${}^{5}$~Vocaveruntque Lot, et dixerunt ei~: Ubi sunt viri qui introierunt ad te nocte~? educ illos huc, ut cognoscamus eos.
${}^{6}$~Egressus ad eos Lot, post tergum occludens ostium, ait~:
${}^{7}$~Nolite, qu\ae so, fratres mei, nolite malum hoc facere.
${}^{8}$~Habeo duas filias, qu\ae\ necdum cognoverunt virum~: educam eas ad vos, et abutimini eis sicut vobis placuerit, dummodo viris istis nihil mali faciatis, quia ingressi sunt sub umbra culminis mei.
${}^{9}$~At illi dixerunt~: Recede illuc. Et rursus~: Ingressus es, inquiunt, ut advena~; numquid ut judices~? te ergo ipsum magis quam hos affligemus. Vimque faciebant Lot vehementissime~: jamque prope erat ut effringerent fores.
${}^{10}$~Et ecce miserunt manum viri, et introduxerunt ad se Lot, clauseruntque ostium~:
${}^{11}$~et eos, qui foris erant, percusserunt c\ae citate a minimo usque ad maximum, ita ut ostium invenire non possent.
${}^{12}$~Dixerunt autem ad Lot~: Habes hic quempiam tuorum~? generum, aut filios, aut filias, omnes, qui tui sunt, educ de urbe hac~:
${}^{13}$~delebimus enim locum istum, eo quod increverit clamor eorum coram Domino, qui misit nos ut perdamus illos.
${}^{14}$~Egressus itaque Lot, locutus est ad generos suos qui accepturi erant filias ejus, et dixit~: Surgite, egredimini de loco isto~: quia delebit Dominus civitatem hanc. Et visus est eis quasi ludens loqui.


${}^{15}$~Cumque esset mane, cogebant eum angeli, dicentes~: Surge, tolle uxorem tuam, et duas filias quas habes~: ne et tu pariter pereas in scelere civitatis.
${}^{16}$~Dissimulante illo, apprehenderunt manum ejus, et manum uxoris, ac duarum filiarum ejus, eo quod parceret Dominus illi.
${}^{17}$~Eduxeruntque eum, et posuerunt extra civitatem~: ibique locuti sunt ad eum, dicentes~: Salva animam tuam~: noli respicere post tergum, nec stes in omni circa regione~: sed in monte salvum te fac, ne et tu simul pereas.
${}^{18}$~Dixitque Lot ad eos~: Qu\ae so, domine mi,
${}^{19}$~quia invenit servus tuus gratiam coram te, et magnificasti misericordiam tuam quam fecisti mecum, ut salvares animam meam, nec possum in monte salvari, ne forte apprehendat me malum, et moriar~:
${}^{20}$~est civitas h\ae c juxta, ad quam possum fugere, parva, et salvabor in ea~: numquid non modica est, et vivet anima mea~?
${}^{21}$~Dixitque ad eum~: Ecce etiam in hoc suscepi preces tuas, ut non subvertam urbem pro qua locutus es.
${}^{22}$~Festina, et salvare ibi~: quia non potero facere quidquam donec ingrediaris illuc. Idcirco vocatum est nomen urbis illius Segor.
${}^{23}$~Sol egressus est super terram, et Lot ingressus est Segor.
${}^{24}$~Igitur Dominus pluit super Sodomam et Gomorrham sulphur et ignem a Domino de c\ae lo~:
${}^{25}$~et subvertit civitates has, et omnem circa regionem, universos habitatores urbium, et cuncta terr\ae\ virentia.
${}^{26}$~Respiciensque uxor ejus post se, versa est in statuam salis.
${}^{27}$~Abraham autem consurgens mane, ubi steterat prius cum Domino,
${}^{28}$~intuitus est Sodomam et Gomorrham, et universam terram regionis illius~: viditque ascendentem favillam de terra quasi fornacis fumum.
${}^{29}$~Cum enim subverteret Deus civitates regionis illius, recordatus Abrah\ae , liberavit Lot de subversione urbium in quibus habitaverat.


${}^{30}$~Ascenditque Lot de Segor, et mansit in monte, du\ae\ quoque fili\ae\ ejus cum eo (timuerat enim manere in Segor) et mansit in spelunca ipse, et du\ae\ fili\ae\ ejus cum eo.
${}^{31}$~Dixitque major ad minorem~: Pater noster senex est, et nullus virorum remansit in terra, qui possit ingredi ad nos juxta morem univers\ae\ terr\ae .
${}^{32}$~Veni, inebriemus eum vino, dormiamusque cum eo, ut servare possimus ex patre nostro semen.
${}^{33}$~Dederunt itaque patri suo bibere vinum nocte illa. Et ingressa est major, dormivitque cum patre~; at ille non sensit, nec quando accubuit filia, nec quando surrexit.
${}^{34}$~Altera quoque die dixit major ad minorem~: Ecce dormivi heri cum patre meo, demus ei bibere vinum etiam hac nocte, et dormies cum eo, ut salvemus semen de patre nostro.
${}^{35}$~Dederunt etiam et illa nocte patri suo bibere vinum, ingressaque minor filia, dormivit cum eo~: et ne tunc quidem sensit quando concubuerit, vel quando illa surrexerit.
${}^{36}$~Conceperunt ergo du\ae\ fili\ae\ Lot de patre suo.
${}^{37}$~Peperitque major filium, et vocavit nomen ejus Moab~: ipse est pater Moabitarum usque in pr\ae sentem diem.
${}^{38}$~Minor quoque peperit filium, et vocavit nomen ejus Ammon, id est, Filius populi mei~: ipse est pater Ammonitarum usque hodie.

\bchapter
\mylettrine{P}rofectus inde Abraham in terram australem, habitavit inter Cades et Sur~: et peregrinatus est in Geraris.
${}^{2}$~Dixitque de Sara uxore suo~: Soror mea est. Misit ergo Abimelech rex Gerar\ae , et tulit eam.
${}^{3}$~Venit autem Deus ad Abimelech per somnium nocte, et ait illi~: En morieris propter mulierem quam tulisti~: habet enim virum.
${}^{4}$~Abimelech vero non tetigerat eam, et ait~: Domine, num gentem ignorantem et justam interficies~?
${}^{5}$~nonne ipse dixit mihi~: Soror mea est~: et ipsa ait~: Frater meus est~? In simplicitate cordis mei, et munditia manuum mearum feci hoc.
${}^{6}$~Dixitque ad eum Deus~: Et ego scio quod simplici corde feceris~: et ideo custodivi te ne peccares in me, et non dimisi ut tangeres eam.
${}^{7}$~Nunc ergo redde viro suo uxorem, quia propheta est~: et orabit pro te, et vives~: si autem nolueris reddere, scito quod morte morieris tu, et omnia qu\ae\ tua sunt.
${}^{8}$~Statimque de nocte consurgens Abimelech, vocavit omnes servos suos~: et locutus est universa verba h\ae c in auribus eorum, timueruntque omnes viri valde.
${}^{9}$~Vocavit autem Abimelech etiam Abraham, et dixit ei~: Quid fecisti nobis~? quid peccavimus in te, quia induxisti super me et super regnum meum peccatum grande~? qu\ae\ non debuisti facere, fecisti nobis.
${}^{10}$~Rursumque expostulans, ait~: Quid vidisti, ut hoc faceres~?
${}^{11}$~Respondit Abraham~: Cogitavi mecum, dicens~: Forsitan non est timor Dei in loco isto~: et interficient me propter uxorem meam~:
${}^{12}$~alias autem et vere soror mea est, filia patris mei, et non filia matris me\ae , et duxi eam in uxorem.
${}^{13}$~Postquam autem eduxit me Deus de domo patris mei, dixi ad eam~: Hanc misericordiam facies mecum~: in omni loco, ad quem ingrediemur, dices quod frater tuus sim.
${}^{14}$~Tulit igitur Abimelech oves et boves, et servos et ancillas, et dedit Abraham~: reddiditque illi Saram uxorem suam,
${}^{15}$~et ait~: Terra coram vobis est, ubicumque tibi placuerit habita.
${}^{16}$~Sar\ae\ autem dixit~: Ecce mille argenteos dedi fratri tuo, hoc erit tibi in velamen oculorum ad omnes qui tecum sunt, et quocumque perrexeris~: mementoque te deprehensam.
${}^{17}$~Orante autem Abraham, sanavit Deus Abimelech et uxorem, ancillasque ejus, et pepererunt~:
${}^{18}$~concluserat enim Dominus omnem vulvam domus Abimelech propter Saram uxorem Abrah\ae .

\bchapter
\mylettrine{V}isitavit autem Dominus Saram, sicut promiserat~: et implevit qu\ae\ locutus est.
${}^{2}$~Concepitque et peperit filium in senectute sua, tempore quo pr\ae dixerat ei Deus.
${}^{3}$~Vocavitque Abraham nomen filii sui, quem genuit ei Sara, Isaac~:
${}^{4}$~et circumcidit eum octavo die, sicut pr\ae ceperat ei Deus,
${}^{5}$~cum centum esset annorum~: hac quippe \ae tate patris, natus est Isaac.
${}^{6}$~Dixitque Sara~: Risum fecit mihi Deus~: quicumque audierit, corridebit mihi.
${}^{7}$~Rursumque ait~: Quis auditurus crederet Abraham quod Sara lactaret filium, quem peperit ei jam seni~?
${}^{8}$~Crevit igitur puer, et ablactatus est~: fecitque Abraham grande convivium in die ablactationis ejus.


${}^{9}$~Cumque vidisset Sara filium Agar \AE gypti\ae\ ludentem cum Isaac filio suo, dixit ad Abraham~:
${}^{10}$~Ejice ancillam hanc, et filium ejus~: non enim erit h\ae res filius ancill\ae\ cum filio meo Isaac.
${}^{11}$~Dure accepit hoc Abraham pro filio suo.
${}^{12}$~Cui dixit Deus~: Non tibi videatur asperum super puero, et super ancilla tua~: omnia qu\ae\ dixerit tibi Sara, audi vocem ejus~: quia in Isaac vocabitur tibi semen.
${}^{13}$~Sed et filium ancill\ae\ faciam in gentem magnam, quia semen tuum est.
${}^{14}$~Surrexit itaque Abraham mane, et tollens panem et utrem aqu\ae , imposuit scapul\ae\ ejus, tradiditque puerum, et dimisit eam. Qu\ae\ cum abiisset, errabat in solitudine Bersabee.
${}^{15}$~Cumque consumpta esset aqua in utre, abjecit puerum subter unam arborum, qu\ae\ ibi erant.
${}^{16}$~Et abiit, seditque e regione procul quantum potest arcus jacere~: dixit enim~: Non videbo morientem puerum~: et sedens contra, levavit vocem suam et flevit.
${}^{17}$~Exaudivit autem Deus vocem pueri~: vocavitque angelus Dei Agar de c\ae lo, dicens~: Quid agis Agar~? noli timere~: exaudivit enim Deus vocem pueri de loco in quo est.
${}^{18}$~Surge, tolle puerum, et tene manum illius~: quia in gentem magnam faciam eum.
${}^{19}$~Aperuitque oculos ejus Deus~: qu\ae\ videns puteum aqu\ae , abiit, et implevit utrem, deditque puero bibere.
${}^{20}$~Et fuit cum eo~: qui crevit, et moratus est in solitudine, factusque est juvenis sagittarius.
${}^{21}$~Habitavitque in deserto Pharan, et accepit illi mater sua uxorem de terra \AE gypti.


${}^{22}$~Eodem tempore dixit Abimelech, et Phicol princeps exercitus ejus, ad Abraham~: Deus tecum est in universis qu\ae\ agis.
${}^{23}$~Jura ergo per Deum, ne noceas mihi, et posteris meis, stirpique me\ae~: sed juxta misericordiam, quam feci tibi, facies mihi, et terr\ae\ in qua versatus es advena.
${}^{24}$~Dixitque Abraham~: Ego jurabo.
${}^{25}$~Et increpavit Abimelech propter puteum aqu\ae\ quem vi abstulerunt servi ejus.
${}^{26}$~Responditque Abimelech~: Nescivi quis fecerit hanc rem~: sed et tu non indicasti mihi, et ego non audivi pr\ae ter hodie.
${}^{27}$~Tulit itaque Abraham oves et boves, et dedit Abimelech~: percusseruntque ambo fœdus.
${}^{28}$~Et statuit Abraham septem agnas gregis seorsum.
${}^{29}$~Cui dixit Abimelech~: Quid sibi volunt septem agn\ae\ ist\ae , quas stare fecisti seorsum~?
${}^{30}$~At ille~: Septem, inquit, agnas accipies de manu mea~: ut sint mihi in testimonium, quoniam ego fodi puteum istum.
${}^{31}$~Idcirco vocatus est locus ille Bersabee~: quia ibi uterque juravit.
${}^{32}$~Et inierunt fœdus pro puteo juramenti.
${}^{33}$~Surrexit autem Abimelech, et Phicol princeps exercitus ejus, reversique sunt in terram Pal\ae stinorum. Abraham vero plantavit nemus in Bersabee, et invocavit ibi nomen Domini Dei \ae terni.
${}^{34}$~Et fuit colonus terr\ae\ Pal\ae stinorum diebus multis.

\bchapter
\mylettrine{Q}u\ae\ postquam gesta sunt, tentavit Deus Abraham, et dixit ad eum~: Abraham, Abraham. At ille respondit~: Adsum.
${}^{2}$~Ait illi~: Tolle filium tuum unigenitum, quem diligis, Isaac, et vade in terram visionis, atque ibi offeres eum in holocaustum super unum montium quem monstravero tibi.
${}^{3}$~Igitur Abraham de nocte consurgens, stravit asinum suum, ducens secum duos juvenes, et Isaac filium suum~: cumque concidisset ligna in holocaustum, abiit ad locum quem pr\ae ceperat ei Deus.
${}^{4}$~Die autem tertio, elevatis oculis, vidit locum procul~:
${}^{5}$~dixitque ad pueros suos~: Expectate hic cum asino~: ego et puer illuc usque properantes, postquam adoraverimus, revertemur ad vos.
${}^{6}$~Tulit quoque ligna holocausti, et imposuit super Isaac filium suum~: ipse vero portabat in manibus ignem et gladium. Cumque duo pergerent simul,
${}^{7}$~dixit Isaac patri suo~: Pater mi. At ille respondit~: Quid vis, fili~? Ecce, inquit, ignis et ligna~: ubi est victima holocausti~?
${}^{8}$~Dixit autem Abraham~: Deus providebit sibi victimam holocausti, fili mi. Pergebant ergo pariter.
${}^{9}$~Et venerunt ad locum quem ostenderat ei Deus, in quo \ae dificavit altare, et desuper ligna composuit~; cumque alligasset Isaac filium suum, posuit eum in altare super struem lignorum.
${}^{10}$~Extenditque manum, et arripuit gladium, ut immolaret filium suum.
${}^{11}$~Et ecce angelus Domini de c\ae lo clamavit, dicens~: Abraham, Abraham. Qui respondit~: Adsum.
${}^{12}$~Dixitque ei~: Non extendas manum tuam super puerum, neque facias illi quidquam~: nunc cognovi quod times Deum, et non pepercisti unigenito filio tuo propter me.
${}^{13}$~Levavit Abraham oculos suos, viditque post tergum arietem inter vepres h\ae rentem cornibus, quem assumens obtulit holocaustum pro filio.
${}^{14}$~Appellavitque nomen loci illius, Dominus videt. Unde usque hodie dicitur~: In monte Dominus videbit.


${}^{15}$~Vocavit autem angelus Domini Abraham secundo de c\ae lo, dicens~:
${}^{16}$~Per memetipsum juravi, dicit Dominus~: quia fecisti hanc rem, et non pepercisti filio tuo unigenito propter me~:
${}^{17}$~benedicam tibi, et multiplicabo semen tuum sicut stellas c\ae li, et velut arenam qu\ae\ est in littore maris~: possidebit semen tuum portas inimicorum suorum,
${}^{18}$~et benedicentur in semine tuo omnes gentes terr\ae , quia obedisti voci me\ae .
${}^{19}$~Reversus est Abraham ad pueros suos, abieruntque Bersabee simul, et habitavit ibi.


${}^{20}$~His ita gestis, nuntiatum est Abrah\ae\ quod Melcha quoque genuisset filios Nachor fratri suo~:
${}^{21}$~Hus primogenitum, et Buz fratrem ejus, et Camuel patrem Syrorum,
${}^{22}$~et Cased, et Azau, Pheldas quoque et Jedlaph,
${}^{23}$~ac Bathuel, de quo nata est Rebecca~: octo istos genuit Melcha, Nachor fratri Abrah\ae .
${}^{24}$~Concubina vero illius, nomine Roma, peperit Tabee, et Gaham, et Thahas, et Maacha.

\bchapter
\mylettrine{V}ixit autem Sara centum viginti septem annis.
${}^{2}$~Et mortua est in civitate Arbee, qu\ae\ est Hebron, in terra Chanaan~: venitque Abraham ut plangeret et fleret eam.
${}^{3}$~Cumque surrexisset ab officio funeris, locutus est ad filios Heth, dicens~:
${}^{4}$~Advena sum et peregrinus apud vos~: date mihi jus sepulchri vobiscum, ut sepeliam mortuum meum.
${}^{5}$~Responderunt filii Heth, dicentes~:
${}^{6}$~Audi nos, domine~: princeps Dei es apud nos~: in electis sepulchris nostris sepeli mortuum tuum, nullusque te prohibere poterit quin in monumento ejus sepelias mortuum tuum.
${}^{7}$~Surrexit Abraham, et adoravit populum terr\ae , filios videlicet Heth~:
${}^{8}$~dixitque ad eos~: Si placet anim\ae\ vestr\ae\ ut sepeliam mortuum meum, audite me, et intercedite pro me apud Ephron filium Seor~:
${}^{9}$~ut det mihi speluncam duplicem, quam habet in extrema parte agri sui~: pecunia digna tradat eam mihi coram vobis in possessionem sepulchri.
${}^{10}$~Habitabat autem Ephron in medio filiorum Heth. Responditque Ephron ad Abraham, cunctis audientibus qui ingrediebantur portam civitatis illius, dicens~:
${}^{11}$~Nequaquam ita fiat, domine mi, sed tu magis ausculta quod loquor. Agrum trado tibi, et speluncam qu\ae\ in eo est, pr\ae sentibus filiis populi mei~; sepeli mortuum tuum.
${}^{12}$~Adoravit Abraham coram populo terr\ae .
${}^{13}$~Et locutus est ad Ephron circumstante plebe~: Qu\ae so ut audias me~: dabo pecuniam pro agro~: suscipe eam, et sic sepeliam mortuum meum in eo.
${}^{14}$~Responditque Ephron~:
${}^{15}$~Domine mi, audi me~: terra, quam postulas, quadringentis siclis argenti valet~: istud est pretium inter me et te~: sed quantum est hoc~? sepeli mortuum tuum.
${}^{16}$~Quod cum audisset Abraham, appendit pecuniam, quam Ephron postulaverat, audientibus filiis Heth, quadringentos siclos argenti probat\ae\ monet\ae\ public\ae .
${}^{17}$~Confirmatusque est ager quondam Ephronis, in quo erat spelunca duplex, respiciens Mambre, tam ipse, quam spelunca, et omnes arbores ejus in cunctis terminis ejus per circuitum,
${}^{18}$~Abrah\ae\ in possessionem, videntibus filiis Heth, et cunctis qui intrabant portam civitatis illius.
${}^{19}$~Atque ita sepelivit Abraham Saram uxorem suam in spelunca agri duplici, qu\ae\ respiciebat Mambre. H\ae c est Hebron in terra Chanaan.
${}^{20}$~Et confirmatus est ager, et antrum quod erat in eo, Abrah\ae\ in possessionem monumenti a filiis Heth.

\bchapter
\mylettrine{E}rat autem Abraham senex, dierumque multorum~: et Dominus in cunctis benedixerat ei.
${}^{2}$~Dixitque ad servum seniorem domus su\ae , qui pr\ae erat omnibus qu\ae\ habebat~: Pone manum tuam subter femur meum,
${}^{3}$~ut adjurem te per Dominum Deum c\ae li et terr\ae , ut non accipias uxorem filio meo de filiabus Chanan\ae orum, inter quos habito~:
${}^{4}$~sed ad terram et cognationem meam proficiscaris et inde accipias uxorem filio meo Isaac.
${}^{5}$~Respondit servus~: Si noluerit mulier venire mecum in terram hanc, numquid reducere debeo filium tuum ad locum, de quo egressus es~?
${}^{6}$~Dixitque Abraham~: Cave nequando reducas filium meum illuc.
${}^{7}$~Dominus Deus c\ae li, qui tulit me de domo patris mei, et de terra nativitatis me\ae , qui locutus est mihi, et juravit mihi, dicens~: Semini tuo dabo terram hanc~: ipse mittet angelum suum coram te, et accipies inde uxorem filio meo~:
${}^{8}$~sin autem mulier noluerit sequi te, non teneberis juramento~: filium meum tantum ne reducas illuc.
${}^{9}$~Posuit ergo servus manum sub femore Abraham domini sui, et juravit illi super sermone hoc.
${}^{10}$~Tulitque decem camelos de grege domini sui, et abiit, ex omnibus bonis ejus portans secum, profectusque perrexit in Mesopotamiam ad urbem Nachor.
${}^{11}$~Cumque camelos fecisset accumbere extra oppidum juxta puteum aqu\ae\ vespere, tempore quo solent mulieres egredi ad hauriendam aquam, dixit~:
${}^{12}$~Domine Deus domini mei Abraham, occurre, obsecro, mihi hodie, et fac misericordiam cum domino meo Abraham.
${}^{13}$~Ecce ego sto prope fontem aqu\ae , et fili\ae\ habitatorum hujus civitatis egredientur ad hauriendam aquam.
${}^{14}$~Igitur puella, cui ego dixero~: Inclina hydriam tuam ut bibam~: et illa responderit~: Bibe, quin et camelis tuis dabo potum~: ipsa est quam pr\ae parasti servo tuo Isaac~: et per hoc intelligam quod feceris misericordiam cum domino meo.
${}^{15}$~Necdum intra se verba compleverat, et ecce Rebecca egrediebatur, filia Bathuel, filii Melch\ae\ uxoris Nachor fratris Abraham, habens hydriam in scapula sua~:
${}^{16}$~puella decora nimis, virgoque pulcherrima, et incognita viro~: descenderat autem ad fontem, et impleverat hydriam, ac revertebatur.
${}^{17}$~Occurritque ei servus, et ait~: Pauxillum aqu\ae\ mihi ad bibendum pr\ae be de hydria tua.
${}^{18}$~Qu\ae\ respondit~: Bibe, domine mi~: celeriterque deposuit hydriam super ulnam suam, et dedit ei potum.
${}^{19}$~Cumque ille bibisset, adjecit~: Quin et camelis tuis hauriam aquam, donec cuncti bibant.
${}^{20}$~Effundensque hydriam in canalibus, recurrit ad puteum ut hauriret aquam~: et haustam omnibus camelis dedit.
${}^{21}$~Ipse autem contemplabatur eam tacitus, scire volens utrum prosperum iter suum fecisset Dominus, an non.
${}^{22}$~Postquam autem biberunt cameli, protulit vir inaures aureas, appendentes siclos duos, et armillas totidem pondo siclorum decem.
${}^{23}$~Dixitque ad eam~: Cujus es filia~? indica mihi, est in domo patris tui locus ad manendum~?
${}^{24}$~Qu\ae\ respondit~: Filia sum Bathuelis, filii Melch\ae , quem peperit ipsi Nachor.
${}^{25}$~Et addidit, dicens~: Palearum quoque et fœni plurimum est apud nos, et locus spatiosus ad manendum.
${}^{26}$~Inclinavit se homo, et adoravit Dominum,
${}^{27}$~dicens~: Benedictus Dominus Deus domini mei Abraham, qui non abstulit misericordiam et veritatem suam a domino meo, et recto itinere me perduxit in domum fratris domini mei.


${}^{28}$~Cucurrit itaque puella, et nuntiavit in domum matris su\ae\ omnia qu\ae\ audierat.
${}^{29}$~Habebat autem Rebecca fratrem nomine Laban, qui festinus egressus est ad hominem, ubi erat fons.
${}^{30}$~Cumque vidisset inaures et armillas in manibus sororis su\ae , et audisset cuncta verba referentis~: H\ae c locutus est mihi homo~: venit ad virum qui stabat juxta camelos, et prope fontem aqu\ae~:
${}^{31}$~dixitque ad eum~: Ingredere, benedicte Domini~: cur foris stas~? pr\ae paravi domum, et locum camelis.
${}^{32}$~Et introduxit eum in hospitium~: ac destravit camelos, deditque paleas et fœnum, et aquam ad lavandos pedes ejus, et virorum qui venerant cum eo.
${}^{33}$~Et appositus est in conspectu ejus panis. Qui ait~: Non comedam, donec loquar sermones meos. Respondit ei~: Loquere.
${}^{34}$~At ille~: Servus, inquit, Abraham sum~:
${}^{35}$~et Dominus benedixit domino meo valde, magnificatusque est~: et dedit ei oves et boves, argentum et aurum, servos et ancillas, camelos et asinos.
${}^{36}$~Et peperit Sara uxor domini mei filium domino meo in senectute sua, deditque illi omnia qu\ae\ habuerat.
${}^{37}$~Et adjuravit me dominus meus, dicens~: Non accipies uxorem filio meo de filiabus Chanan\ae orum, in quorum terra habito~:
${}^{38}$~sed ad domum patris mei perges, et de cognatione mea accipies uxorem filio meo.
${}^{39}$~Ego vero respondi domino meo~: Quid si noluerit venire mecum mulier~?
${}^{40}$~Dominus, ait, in cujus conspectu ambulo, mittet angelum suum tecum, et diriget viam tuam~: accipiesque uxorem filio meo de cognatione mea, et de domo patris mei.
${}^{41}$~Innocens eris a maledictione mea, cum veneris ad propinquos meos, et non dederint tibi.
${}^{42}$~Veni ergo hodie ad fontem aqu\ae , et dixi~: Domine Deus domini mei Abraham, si direxisti viam meam, in qua nunc ambulo,
${}^{43}$~ecce sto juxta fontem aqu\ae , et virgo, qu\ae\ egredietur ad hauriendam aquam, audierit a me~: Da mihi pauxillum aqu\ae\ ad bibendum ex hydria tua~:
${}^{44}$~et dixerit mihi~: Et tu bibe, et camelis tuis hauriam~: ipsa est mulier, quam pr\ae paravit Dominus filio domini mei.
${}^{45}$~Dumque h\ae c tacitus mecum volverem, apparuit Rebecca veniens cum hydria, quam portabat in scapula~: descenditque ad fontem, et hausit aquam. Et aio ad eam~: Da mihi paululum bibere.
${}^{46}$~Qu\ae\ festinans deposuit hydriam de humero, et dixit mihi~: Et tu bibe, et camelis tuis tribuam potum. Bibi, et adaquavit camelos.
${}^{47}$~Interrogavique eam, et dixi~: Cujus es filia~? Qu\ae\ respondit~: Filia Bathuelis sum, filii Nachor, quem peperit ei Melcha. Suspendi itaque inaures ad ornandam faciem ejus, et armillas posui in manibus ejus.
${}^{48}$~Pronusque adoravi Dominum, benedicens Domino Deo domini mei Abraham, qui perduxit me recto itinere, ut sumerem filiam fratris domini mei filio ejus.
${}^{49}$~Quam ob rem si facitis misericordiam et veritatem cum domino meo, indicate mihi~: sin autem aliud placet, et hoc dicite mihi, ut vadam ad dextram, sive ad sinistram.


${}^{50}$~Responderuntque Laban et Bathuel~: A Domino egressus est sermo~: non possumus extra placitum ejus quidquam aliud loqui tecum.
${}^{51}$~En Rebecca coram te est, tolle eam, et proficiscere, et sit uxor filii domini tui, sicut locutus est Dominus.
${}^{52}$~Quod cum audisset puer Abraham, procidens adoravit in terram Dominum.
${}^{53}$~Prolatisque vasis argenteis, et aureis, ac vestibus, dedit ea Rebecc\ae\ pro munere~: fratribus quoque ejus et matri dona obtulit.
${}^{54}$~Inito convivio, vescentes pariter et bibentes manserunt ibi. Surgens autem mane, locutus est puer~: Dimitte me, ut vadam ad dominum meum.
${}^{55}$~Responderuntque fratres ejus et mater~: Maneat puella saltem decem dies apud nos, et postea proficiscetur.
${}^{56}$~Nolite, ait, me retinere, quia Dominus direxit viam meam~: dimittite me ut pergam ad dominum meum.
${}^{57}$~Et dixerunt~: Vocemus puellam, et qu\ae ramus ipsius voluntatem.
${}^{58}$~Cumque vocata venisset, sciscitati sunt~: Vis ire cum homine isto~? Qu\ae\ ait~: Vadam.
${}^{59}$~Dimiserunt ergo eam, et nutricem illius, servumque Abraham, et comites ejus,
${}^{60}$~imprecantes prospera sorori su\ae , atque dicentes~: Soror nostra es, crescas in mille millia, et possideat semen tuum portas inimicorum suorum.
${}^{61}$~Igitur Rebecca et puell\ae\ illius, ascensis camelis, secut\ae\ sunt virum~: qui festinus revertebatur ad dominum suum.
${}^{62}$~Eo autem tempore deambulabat Isaac per viam qu\ae\ ducit ad puteum, cujus nomen est Viventis et videntis~: habitabat enim in terra australi~:
${}^{63}$~et egressus fuerat ad meditandum in agro, inclinata jam die~: cumque elevasset oculos, vidit camelos venientes procul.
${}^{64}$~Rebecca quoque, conspecto Isaac, descendit de camelo,
${}^{65}$~et ait ad puerum~: Quis est ille homo qui venit per agrum in occursum nobis~? Dixitque ei~: Ipse est dominus meus. At illa tollens cito pallium, operuit se.
${}^{66}$~Servus autem cuncta, qu\ae\ gesserat, narravit Isaac.
${}^{67}$~Qui introduxit eam in tabernaculum Sar\ae\ matris su\ae , et accepit eam uxorem~: et in tantum dilexit eam, ut dolorem, qui ex morte matris ejus acciderat, temperaret.

\bchapter
\mylettrine{A}braham vero aliam duxit uxorem nomine Ceturam~:
${}^{2}$~qu\ae\ peperit ei Zamran et Jecsan, et Madan, et Madian, et Jesboc, et Sue.
${}^{3}$~Jecsan quoque genuit Saba et Dadan. Filii Dadan fuerunt Assurim, et Latusim, et Loomin.
${}^{4}$~At vero ex Madian ortus est Epha, et Opher, et Henoch, et Abida, et Eldaa~: omnes hi filii Cetur\ae .
${}^{5}$~Deditque Abraham cuncta qu\ae\ possederat, Isaac~:
${}^{6}$~filiis autem concubinarum largitus est munera, et separavit eos ab Isaac filio suo, dum adhuc ipse viveret, ad plagam orientalem.
${}^{7}$~Fuerunt autem dies vit\ae\ Abrah\ae , centum septuaginta quinque anni.
${}^{8}$~Et deficiens mortuus est in senectute bona, provect\ae que \ae tatis et plenus dierum~: congregatusque est ad populum suum.
${}^{9}$~Et sepelierunt eum Isaac et Isma\"el filii sui in spelunca duplici, qu\ae\ sita est in agro Ephron filii Seor Heth\ae i, e regione Mambre,
${}^{10}$~quem emerat a filiis Heth~: ibi sepultus est ipse, et Sara uxor ejus.
${}^{11}$~Et post obitum illius benedixit Deus Isaac filio ejus, qui habitabat juxta puteum nomine Viventis et videntis.


${}^{12}$~H\ae\ sunt generationes Isma\"el filii Abrah\ae , quem peperit ei Agar \AE gyptia, famula Sar\ae~: et
${}^{13}$~h\ae c nomina filiorum ejus in vocabulis et generationibus suis. Primogenitus Isma\"elis Nabaioth, deinde Cedar, et Adbeel, et Mabsam,
${}^{14}$~Masma quoque, et Duma, et Massa,
${}^{15}$~Hadar, et Thema, et Jethur, et Naphis, et Cedma.
${}^{16}$~Isti sunt filii Isma\"elis~: et h\ae c nomina per castella et oppida eorum, duodecim principes tribuum suarum.
${}^{17}$~Et facti sunt anni vit\ae\ Isma\"elis centum triginta septem, deficiensque mortuus est, et appositus ad populum suum.
${}^{18}$~Habitavit autem ab Hevila usque Sur, qu\ae\ respicit \AE gyptum intro\"euntibus Assyrios~; coram cunctis fratribus suis obiit.


${}^{19}$~H\ae\ quoque sunt generationes Isaac filii Abraham~: Abraham genuit Isaac~:
${}^{20}$~qui cum quadraginta esset annorum, duxit uxorem Rebeccam filiam Bathuelis Syri de Mesopotamia, sororem Laban.
${}^{21}$~Deprecatusque est Isaac Dominum pro uxore sua, eo quod esset sterilis~: qui exaudivit eum, et dedit conceptum Rebecc\ae .
${}^{22}$~Sed collidebantur in utero ejus parvuli~; qu\ae\ ait~: Si sic mihi futurum erat, quid necesse fuit concipere~? perrexitque ut consuleret Dominum.
${}^{23}$~Qui respondens ait~: \begin{flushleft}\begin{verse}Du\ae\ gentes sunt in utero tuo,\\ et duo populi ex ventre tuo dividentur,\\ populusque populum superabit,\\ et major serviet minori.\end{verse}\end{flushleft}


${}^{24}$~Jam tempus pariendi advenerat, et ecce gemini in utero ejus reperti sunt.
${}^{25}$~Qui prior egressus est, rufus erat, et totus in morem pellis hispidus~: vocatumque est nomen ejus Esau. Protinus alter egrediens, plantam fratris tenebat manu~: et idcirco appellavit eum Jacob.
${}^{26}$~Sexagenarius erat Isaac quando nati sunt ei parvuli.
${}^{27}$~Quibus adultis, factus est Esau vir gnarus venandi, et homo agricola~: Jacob autem vir simplex habitabat in tabernaculis.
${}^{28}$~Isaac amabat Esau, eo quod de venationibus illius vesceretur~: et Rebecca diligebat Jacob.
${}^{29}$~Coxit autem Jacob pulmentum~: ad quem cum venisset Esau de agro lassus,
${}^{30}$~ait~: Da mihi de coctione hac rufa, quia oppido lassus sum. Quam ob causam vocatum est nomen ejus Edom.
${}^{31}$~Cui dixit Jacob~: Vende mihi primogenita tua.
${}^{32}$~Ille respondit~: En morior, quid mihi proderunt primogenita~?
${}^{33}$~Ait Jacob~: Jura ergo mihi. Juravit ei Esau et vendidit primogenita.
${}^{34}$~Et sic, accepto pane et lentis edulio, comedit et bibit, et abiit, parvipendens quod primogenita vendidisset.

\bchapter
\mylettrine{O}rta autem fame super terram post eam sterilitatem, qu\ae\ acciderat in diebus Abraham, abiit Isaac ad Abimelech regem Pal\ae stinorum in Gerara.
${}^{2}$~Apparuitque ei Dominus, et ait~: Ne descendas in \AE gyptum, sed quiesce in terra quam dixero tibi,
${}^{3}$~et peregrinare in ea~: eroque tecum, et benedicam tibi~: tibi enim et semini tuo dabo universas regiones has, complens juramentum quod spopondi Abraham patri tuo.
${}^{4}$~Et multiplicabo semen tuum sicut stellas c\ae li~: daboque posteris tuis universas regiones has~: et benedicentur in semine tuo omnes gentes terr\ae ,
${}^{5}$~eo quod obedierit Abraham voci me\ae , et custodierit pr\ae cepta et mandata mea, et c\ae remonias legesque servaverit.
${}^{6}$~Mansit itaque Isaac in Geraris.


${}^{7}$~Qui cum interrogaretur a viris loci illius super uxore sua, respondit~: Soror mea est~: timuerat enim confiteri quod sibi esset sociata conjugio, reputans ne forte interficerent eum propter illius pulchritudinem.
${}^{8}$~Cumque pertransissent dies plurimi, et ibidem moraretur, prospiciens Abimelech rex Pal\ae stinorum per fenestram, vidit eum jocantem cum Rebecca uxore sua.
${}^{9}$~Et accersito eo, ait~: Perspicuum est quod uxor tua sit~: cur mentitus es eam sororem tuam esse~? Respondit~: Timui ne morerer propter eam.
${}^{10}$~Dixitque Abimelech~: Quare imposuisti nobis~? potuit coire quispiam de populo cum uxore tua, et induxeras super nos grande peccatum. Pr\ae cepitque omni populo, dicens~:
${}^{11}$~Qui tetigerit hominis hujus uxorem, morte morietur.


${}^{12}$~Sevit autem Isaac in terra illa, et invenit in ipso anno centuplum~: benedixitque ei Dominus.
${}^{13}$~Et locupletatus est homo, et ibat proficiens atque succrescens, donec magnus vehementer effectus est~:
${}^{14}$~habuit quoque possessiones ovium et armentorum, et famili\ae\ plurimum. Ob hoc invidentes ei Pal\ae stini,
${}^{15}$~omnes puteos, quos foderant servi patris illius Abraham, illo tempore obstruxerunt, implentes humo~:
${}^{16}$~in tantum, ut ipse Abimelech diceret ad Isaac~: Recede a nobis, quoniam potentior nobis factus es valde.
${}^{17}$~Et ille discedens, ut veniret ad torrentem Gerar\ae , habitaretque ibi,
${}^{18}$~rursum fodit alios puteos, quos foderant servi patris sui Abraham, et quos, illo mortuo, olim obstruxerant Philisthiim~: appellavitque eos eisdem nominibus quibus ante pater vocaverat.
${}^{19}$~Foderuntque in torrente, et repererunt aquam vivam.
${}^{20}$~Sed et ibi jurgium fuit pastorum Gerar\ae\ adversus pastores Isaac, dicentium~: Nostra est aqua, quam ob rem nomen putei ex eo, quod acciderat, vocavit Calumniam.
${}^{21}$~Foderunt autem et alium~: et pro illo quoque rixati sunt, appellavitque eum Inimicitias.
${}^{22}$~Profectus inde fodit alium puteum, pro quo non contenderunt~: itaque vocavit nomen ejus Latitudo, dicens~: Nunc dilatavit nos Dominus, et fecit crescere super terram.


${}^{23}$~Ascendit autem ex illo loco in Bersabee,
${}^{24}$~ubi apparuit ei Dominus in ipsa nocte, dicens~: Ego sum Deus Abraham patris tui~: noli timere, quia ego tecum sum~: benedicam tibi, et multiplicabo semen tuum propter servum meum Abraham.
${}^{25}$~Itaque \ae dificavit ibi altare~: et invocato nomine Domini, extendit tabernaculum, pr\ae cepitque servis suis ut foderent puteum.
${}^{26}$~Ad quem locum cum venissent de Geraris Abimelech, et Ochozath amicus illius, et Phicol dux militum,
${}^{27}$~locutus est eis Isaac~: Quid venistis ad me, hominem quem odistis, et expulistis a vobis~?
${}^{28}$~Qui responderunt~: Vidimus tecum esse Dominum, et idcirco nos diximus~: Sit juramentum inter nos, et ineamus fœdus,
${}^{29}$~ut non facias nobis quidquam mali, sicut et nos nihil tuorum attigimus, nec fecimus quod te l\ae deret~: sed cum pace dimisimus auctum benedictione Domini.
${}^{30}$~Fecit ergo eis convivium, et post cibum et potum
${}^{31}$~surgentes mane, juraverunt sibi mutuo~: dimisitque eos Isaac pacifice in locum suum.
${}^{32}$~Ecce autem venerunt in ipso die servi Isaac annuntiantes ei de puteo, quem foderant, atque dicentes~: Invenimus aquam.
${}^{33}$~Unde appellavit eum Abundantiam~: et nomen urbi impositum est Bersabee, usque in pr\ae sentem diem.
${}^{34}$~Esau vero quadragenarius duxit uxores, Judith filiam Beeri Heth\ae i, et Basemath filiam Elon ejusdem loci~:
${}^{35}$~qu\ae\ amb\ae\ offenderant animum Isaac et Rebecc\ae .

\bchapter
\mylettrine{S}enuit autem Isaac, et caligaverunt oculi ejus, et videre non poterat~: vocavitque Esau filium suum majorem, et dixit ei~: Fili mi~? Qui respondit~: Adsum.
${}^{2}$~Cui pater~: Vides, inquit, quod senuerim, et ignorem diem mortis me\ae .
${}^{3}$~Sume arma tua, pharetram, et arcum, et egredere foras~: cumque venatu aliquid apprehenderis,
${}^{4}$~fac mihi inde pulmentum sicut velle me nosti, et affer ut comedam~: et benedicat tibi anima mea antequam moriar.
${}^{5}$~Quod cum audisset Rebecca, et ille abiisset in agrum ut jussionem patris impleret,
${}^{6}$~dixit filio suo Jacob~: Audivi patrem tuum loquentem cum Esau fratre tuo, et dicentem ei~:
${}^{7}$~Affer mihi de venatione tua, et fac cibos ut comedam, et benedicam tibi coram Domino antequam moriar.
${}^{8}$~Nunc ergo, fili mi, acquiesce consiliis meis~:
${}^{9}$~et pergens ad gregem, affer mihi duos h\ae dos optimos, ut faciam ex eis escas patri tuo, quibus libenter vescitur~:
${}^{10}$~quas cum intuleris, et comederit, benedicat tibi priusquam moriatur.
${}^{11}$~Cui ille respondit~: Nosti quod Esau frater meus homo pilosus sit, et ego lenis~:
${}^{12}$~si attrectaverit me pater meus, et senserit, timeo ne putet me sibi voluisse illudere, et inducam super me maledictionem pro benedictione.
${}^{13}$~Ad quem mater~: In me sit, ait, ista maledictio, fili mi~: tantum audi vocem meam, et pergens, affer qu\ae\ dixi.
${}^{14}$~Abiit, et attulit, deditque matri. Paravit illa cibos, sicut velle noverat patrem illius.
${}^{15}$~Et vestibus Esau valde bonis, quas apud se habebat domi, induit eum~:
${}^{16}$~pelliculasque h\ae dorum circumdedit manibus, et colli nuda protexit~:
${}^{17}$~deditque pulmentum, et panes, quos coxerat, tradidit.
${}^{18}$~Quibus illatis, dixit~: Pater mi~? At ille respondit~: Audio. Quis es tu, fili mi~?
${}^{19}$~Dixitque Jacob~: Ego sum primogenitus tuus Esau~: feci sicut pr\ae cepisti mihi~: surge, sede, et comede de venatione mea, ut benedicat mihi anima tua.
${}^{20}$~Rursumque Isaac ad filium suum~: Quomodo, inquit, tam cito invenire potuisti, fili mi~? Qui respondit~: Voluntas Dei fuit ut cito occurreret mihi quod volebam.
${}^{21}$~Dixitque Isaac~: Accede huc, ut tangam te, fili mi, et probem utrum tu sis filius meus Esau, an non.
${}^{22}$~Accessit ille ad patrem, et palpato eo, dixit Isaac~: Vox quidem, vox Jacob est~: sed manus, manus sunt Esau.
${}^{23}$~Et non cognovit eum, quia pilos\ae\ manus similitudinem majoris expresserant. Benedicens ergo illi,
${}^{24}$~ait~: Tu es filius meus Esau~? Respondit~: Ego sum.
${}^{25}$~At ille~: Affer mihi, inquit, cibos de venatione tua, fili mi, ut benedicat tibi anima mea. Quos cum oblatos comedisset, obtulit ei etiam vinum. Quo hausto,
${}^{26}$~dixit ad eum~: Accede ad me, et da mihi osculum, fili mi.
${}^{27}$~Accessit, et osculatus est eum. Statimque ut sensit vestimentorum illius fragrantiam, benedicens illi, ait~: \begin{flushleft}\begin{verse}Ecce odor filii mei\\ sicut odor agri pleni,\\ cui benedixit Dominus.\\
${}^{28}$~Det tibi Deus de rore c\ae li\\ et de pinguedine terr\ae \\ abundantiam frumenti et vini.\\
${}^{29}$~Et serviant tibi populi,\\ et adorent te tribus~:\\ esto dominus fratrum tuorum,\\ et incurventur ante te filii matris tu\ae~:\\ qui maledixerit tibi, sit ille maledictus,\\ et qui benedixerit tibi, benedictionibus repleatur.\end{verse}\end{flushleft}


${}^{30}$~Vix Isaac sermonem impleverat, et egresso Jacob foras, venit Esau,
${}^{31}$~coctosque de venatione cibos intulit patri, dicens~: Surge, pater mi, et comede de venatione filii tui, ut benedicat mihi anima tua.
${}^{32}$~Dixitque illi Isaac~: Quis enim es tu~? Qui respondit~: Ego sum filius tuus primogenitus Esau.
${}^{33}$~Expavit Isaac stupore vehementi~: et ultra quam credi potest admirans, ait~: Quis igitur ille est qui dudum captam venationem attulit mihi, et comedi ex omnibus priusquam tu venires~; benedixique ei, et erit benedictus~?
${}^{34}$~Auditis Esau sermonibus patris, irrugiit clamore magno~: et consternatus, ait~: Benedic etiam et mihi, pater mi.
${}^{35}$~Qui ait~: Venit germanus tuus fraudulenter, et accepit benedictionem tuam.
${}^{36}$~At ille subjunxit~: Juste vocatum est nomen ejus Jacob~: supplantavit enim me en altera vice~: primogenita mea ante tulit, et nunc secundo surripuit benedictionem meam. Rursumque ad patrem~: Numquid non reservasti, ait, et mihi benedictionem~?
${}^{37}$~Respondit Isaac~: Dominum tuum illum constitui, et omnes fratres ejus servituti illius subjugavi~; frumento et vino stabilivi eum~: et tibi post h\ae c, fili mi, ultra quid faciam~?
${}^{38}$~Cui Esau~: Num unam, inquit, tantum benedictionem habes, pater~? mihi quoque obsecro ut benedicas. Cumque ejulatu magno fleret,
${}^{39}$~motus Isaac, dixit ad eum~: \begin{flushleft}\begin{verse}In pinguedine terr\ae ,\\ et in rore c\ae li desuper,
${}^{40}$~erit benedictio tua.\\ Vives in gladio, et fratri tuo servies~:\\ tempusque veniet, cum excutias et solvas jugum ejus de cervicibus tuis.\end{verse}\end{flushleft}


${}^{41}$~Oderat ergo semper Esau Jacob pro benedictione qua benedixerat ei pater~: dixitque in corde suo~: Venient dies luctus patris mei, et occidam Jacob fratrem meum.
${}^{42}$~Nuntiata sunt h\ae c Rebecc\ae~: qu\ae\ mittens et vocans Jacob filium suum, dixit ad eum~: Ecce Esau frater tuus minatur ut occidat te.
${}^{43}$~Nunc ergo, fili mi, audi vocem meam, et consurgens fuge ad Laban fratrem meum in Haran~:
${}^{44}$~habitabisque cum eo dies paucos, donec requiescat furor fratris tui,
${}^{45}$~et cesset indignatio ejus, obliviscaturque eorum qu\ae\ fecisti in eum~: postea mittam, et adducam te inde huc~: cur utroque orbabor filio in uno die~?
${}^{46}$~Dixitque Rebecca ad Isaac~: T\ae det me vit\ae\ me\ae\ propter filias Heth~: si acceperit Jacob uxorem de stirpe hujus terr\ae , nolo vivere.

\bchapter
\mylettrine{V}ocavit itaque Isaac Jacob, et benedixit eum, pr\ae cepitque ei dicens~: Noli accipere conjugem de genere Chanaan~:
${}^{2}$~sed vade, et proficiscere in Mesopotamiam Syri\ae , ad domum Bathuel patris matris tu\ae , et accipe tibi inde uxorem de filiabus Laban avunculi tui.
${}^{3}$~Deus autem omnipotens benedicat tibi, et crescere te faciat, atque multiplicet, ut sis in turbas populorum.
${}^{4}$~Et det tibi benedictiones Abrah\ae , et semini tuo post te~: ut possideas terram peregrinationis tu\ae , quam pollicitus est avo tuo.
${}^{5}$~Cumque dimisisset eum Isaac, profectus venit in Mesopotamiam Syri\ae\ ad Laban filium Bathuel Syri, fratrem Rebecc\ae\ matris su\ae .
${}^{6}$~Videns autem Esau quod benedixisset pater suus Jacob, et misisset eum in Mesopotamiam Syri\ae , ut inde uxorem duceret~; et quod post benedictionem pr\ae cepisset ei, dicens~: Non accipies uxorem de filiabus Chanaan~:
${}^{7}$~quodque obediens Jacob parentibus suis isset in Syriam~:
${}^{8}$~probans quoque quod non libenter aspiceret filias Chanaan pater suus~:
${}^{9}$~ivit ad Isma\"elem, et duxit uxorem absque iis, quas prius habebat, Maheleth filiam Isma\"el filii Abraham, sororem Nabaioth.


${}^{10}$~Igitur egressus Jacob de Bersabee, pergebat Haran.
${}^{11}$~Cumque venisset ad quemdam locum, et vellet in eo requiescere post solis occubitum, tulit de lapidibus qui jacebant, et supponens capiti suo, dormivit in eodem loco.
${}^{12}$~Viditque in somnis scalam stantem super terram, et cacumen illius tangens c\ae lum~: angelos quoque Dei ascendentes et descendentes per eam,
${}^{13}$~et Dominum innixum scal\ae\ dicentem sibi~: Ego sum Dominus Deus Abraham patris tui, et Deus Isaac~: terram, in qua dormis, tibi dabo et semini tuo.
${}^{14}$~Eritque semen tuum quasi pulvis terr\ae~: dilataberis ad occidentem, et orientem, et septentrionem, et meridiem~: et benedicentur in te et in semine tuo cunct\ae\ tribus terr\ae .
${}^{15}$~Et ero custos tuus quocumque perrexeris, et reducam te in terram hanc~: nec dimittam nisi complevero universa qu\ae\ dixi.
${}^{16}$~Cumque evigilasset Jacob de somno, ait~: Vere Dominus est in loco isto, et ego nesciebam.
${}^{17}$~Pavensque, Quam terribilis est, inquit, locus iste~! non est hic aliud nisi domus Dei, et porta c\ae li.
${}^{18}$~Surgens ergo Jacob mane, tulit lapidem quem supposuerat capiti suo, et erexit in titulum, fundens oleum desuper.
${}^{19}$~Appellavitque nomen urbis Bethel, qu\ae\ prius Luza vocabatur.
${}^{20}$~Vovit etiam votum, dicens~: Si fuerit Deus mecum, et custodierit me in via, per quam ego ambulo, et dederit mihi panem ad vescendum, et vestimentum ad induendum,
${}^{21}$~reversusque fuero prospere ad domum patris mei~: erit mihi Dominus in Deum,
${}^{22}$~et lapis iste, quem erexi in titulum, vocabitur Domus Dei~: cunctorumque qu\ae\ dederis mihi, decimas offeram tibi.

\bchapter
\mylettrine{P}rofectus ergo Jacob venit in terram orientalem.
${}^{2}$~Et vidit puteum in agro, tres quoque greges ovium accubantes juxta eum~: nam ex illo adaquabantur pecora, et os ejus grandi lapide claudebatur.
${}^{3}$~Morisque erat ut cunctis ovibus congregatis devolverent lapidem, et refectis gregibus rursum super os putei ponerent.
${}^{4}$~Dixitque ad pastores~: Fratres, unde estis~? Qui responderunt~: De Haran.
${}^{5}$~Quos interrogans, Numquid, ait, nostis Laban filium Nachor~? Dixerunt~: Novimus.
${}^{6}$~Sanusne est~? inquit. Valet, inquiunt~: et ecce Rachel filia ejus venit cum grege suo.
${}^{7}$~Dixitque Jacob~: Adhuc multum diei superest, nec est tempus ut reducantur ad caulas greges~: date ante potum ovibus, et sic eas ad pastum reducite.
${}^{8}$~Qui responderunt~: Non possumus, donec omnia pecora congregentur, et amoveamus lapidem de ore putei, ut adaquemus greges.
${}^{9}$~Adhuc loquebantur, et ecce Rachel veniebat cum ovibus patris sui~: nam gregem ipsa pascebat.
${}^{10}$~Quam cum vidisset Jacob, et sciret consobrinam suam, ovesque Laban avunculi sui, amovit lapidem quo puteus claudebatur.
${}^{11}$~Et adaquato grege, osculatus est eam~: et elevata voce flevit,
${}^{12}$~et indicavit ei quod frater esset patris sui, et filius Rebecc\ae~: at illa festinans nuntiavit patri suo.
${}^{13}$~Qui cum audisset venisse Jacob filium sororis su\ae , cucurrit obviam ei~: complexusque eum, et in oscula ruens, duxit in domum suam. Auditis autem causis itineris,
${}^{14}$~respondit~: Os meum es, et caro mea. Et postquam impleti sunt dies mensis unius,
${}^{15}$~dixit ei~: Num quia frater meus es, gratis servies mihi~? dic quid mercedis accipias.


${}^{16}$~Habebat vero duas filias~: nomen majoris Lia, minor vero appellabatur Rachel.
${}^{17}$~Sed Lia lippis erat oculis~: Rachel decora facie, et venusto aspectu.
${}^{18}$~Quam diligens Jacob, ait~: Serviam tibi pro Rachel filia tua minore, septem annis.
${}^{19}$~Respondit Laban~: Melius est ut tibi eam dem quam alteri viro~: mane apud me.
${}^{20}$~Servivit ergo Jacob pro Rachel septem annis~: et videbantur illi pauci dies pr\ae\ amoris magnitudine.
${}^{21}$~Dixitque ad Laban~: Da mihi uxorem meam~: quia jam tempus impletum est, ut ingrediar ad illam.
${}^{22}$~Qui vocatis multis amicorum turbis ad convivium, fecit nuptias.
${}^{23}$~Et vespere Liam filiam suam introduxit ad eum,
${}^{24}$~dans ancillam fili\ae , Zelpham nomine. Ad quam cum ex more Jacob fuisset ingressus, facto mane vidit Liam~:
${}^{25}$~et dixit ad socerum suum~: Quid est quod facere voluisti~? nonne pro Rachel servivi tibi~? quare imposuisti mihi~?
${}^{26}$~Respondit Laban~: Non est in loco nostro consuetudinis, ut minores ante tradamus ad nuptias.
${}^{27}$~Imple hebdomadam dierum hujus copul\ae~: et hanc quoque dabo tibi pro opere quo serviturus es mihi septem annis aliis.
${}^{28}$~Acquievit placito~: et hebdomada transacta, Rachel duxit uxorem~:
${}^{29}$~cui pater servam Balam tradiderat.
${}^{30}$~Tandemque potitus optatis nuptiis, amorem sequentis priori pr\ae tulit, serviens apud eum septem annis aliis.


${}^{31}$~Videns autem Dominus quod despiceret Liam, aperuit vulvam ejus, sorore sterili permanente.
${}^{32}$~Qu\ae\ conceptum genuit filium, vocavitque nomen ejus Ruben, dicens~: Vidit Dominus humilitatem meam~: nunc amabit me vir meus.
${}^{33}$~Rursumque concepit et peperit filium, et ait~: Quoniam audivit me Dominus haberi contemptui, dedit etiam istum mihi~; vocavitque nomen ejus Simeon.
${}^{34}$~Concepitque tertio, et genuit alium filium~: dixitque~: Nunc quoque copulabitur mihi maritus meus~: eo quod pepererim ei tres filios~: et idcirco appellavit nomen ejus Levi.
${}^{35}$~Quarto concepit, et peperit filium, et ait~: Modo confitebor Domino, et ob hoc vocavit eum Judam~: cessavitque parere.

\bchapter
\mylettrine{C}ernens autem Rachel quod infecunda esset, invidit sorori su\ae , et ait marito suo~: Da mihi liberos, alioquin moriar.
${}^{2}$~Cui iratus respondit Jacob~: Num pro Deo ego sum, qui privavit te fructu ventris tui~?
${}^{3}$~At illa~: Habeo, inquit, famulam Balam~: ingredere ad illam, ut pariat super genua mea, et habeam ex illa filios.
${}^{4}$~Deditque illi Balam in conjugium~: qu\ae ,
${}^{5}$~ingresso ad se viro, concepit, et peperit filium.
${}^{6}$~Dixitque Rachel~: Judicavit mihi Dominus, et exaudivit vocem meam, dans mihi filium, et idcirco appellavit nomen ejus Dan.
${}^{7}$~Rursumque Bala concipiens, peperit alterum,
${}^{8}$~pro quo ait Rachel~: Comparavit me Deus cum sorore mea, et invalui~: vocavitque eum Nephthali.
${}^{9}$~Sentiens Lia quod parere desiisset, Zelpham ancillam suam marito tradidit.
${}^{10}$~Qua post conceptum edente filium,
${}^{11}$~dixit~: Feliciter, et idcirco vocavit nomen ejus Gad.
${}^{12}$~Peperit quoque Zelpha alterum.
${}^{13}$~Dixitque Lia~: Hoc pro beatitudine mea~: beatam quippe me dicent mulieres~: propterea appellavit eum Aser.
${}^{14}$~Egressus autem Ruben tempore messis tritice\ae\ in agrum, reperit mandragoras, quas matri Li\ae\ detulit. Dixitque Rachel~: Da mihi partem de mandragoris filii tui.
${}^{15}$~Illa respondit~: Parumne tibi videtur quod pr\ae ripueris maritum mihi, nisi etiam mandragoras filii mei tuleris~? Ait Rachel~: Dormiat tecum hac nocte pro mandragoris filii tui.
${}^{16}$~Redeuntique ad vesperam Jacob de agro, egressa est in occursum ejus Lia, et Ad me, inquit, intrabis~: quia mercede conduxi te pro mandragoris filii mei. Dormivitque cum ea nocte illa.
${}^{17}$~Et exaudivit Deus preces ejus, concepitque et peperit filium quintum,
${}^{18}$~et ait~: Dedit Deus mercedem mihi, quia dedi ancillam meam viro meo~: appellavitque nomen ejus Issachar.
${}^{19}$~Rursum Lia concipiens, peperit sextum filium,
${}^{20}$~et ait~: Dotavit me Deus dote bona~: etiam hac vice mecum erit maritus meus, eo quod genuerim ei sex filios~: et idcirco appellavit nomen ejus Zabulon.
${}^{21}$~Post quem peperit filiam, nomine Dinam.
${}^{22}$~Recordatus quoque Dominus Rachelis, exaudivit eam, et aperuit vulvam ejus.
${}^{23}$~Qu\ae\ concepit, et peperit filium, dicens~: Abstulit Deus opprobrium meum.
${}^{24}$~Et vocavit nomen ejus Joseph, dicens~: Addat mihi Dominus filium alterum.
${}^{25}$~Nato autem Joseph, dixit Jacob socero suo~: Dimitte me ut revertar in patriam, et ad terram meam.
${}^{26}$~Da mihi uxores, et liberos meos, pro quibus servivi tibi, ut abeam~: tu nosti servitutem qua servivi tibi.


${}^{27}$~Ait illi Laban~: Inveniam gratiam in conspectu tuo, experimento didici, quia benedixerit mihi Deus propter te~:
${}^{28}$~constitue mercedem tuam quam dem tibi.
${}^{29}$~At ille respondit~: Tu nosti quomodo servierim tibi, et quanta in manibus meis fuerit possessio tua.
${}^{30}$~Modicum habuisti antequam venirem ad te, et nunc dives effectus es~: benedixitque tibi Dominus ad introitum meum. Justum est igitur ut aliquando provideam etiam domui me\ae .
${}^{31}$~Dixitque Laban~: Quid tibi dabo~? At ille ait~: Nihil volo~: sed si feceris quod postulo, iterum pascam, et custodiam pecora tua.
${}^{32}$~Gyra omnes greges tuos, et separa cunctas oves varias, et sparso vellere~; quodcumque furvum, et maculosum, variumque fuerit, tam in ovibus quam in capris, erit merces mea.
${}^{33}$~Respondebitque mihi cras justitia mea, quando placiti tempus advenerit coram te~: et omnia qu\ae\ non fuerint varia, et maculosa, et furva, tam in ovibus quam in capris, furti me arguent.
${}^{34}$~Dixitque Laban~: Gratum habeo quod petis.
${}^{35}$~Et separavit in die illa capras, et oves, et hircos, et arietes varios, atque maculosos~: cunctum autem gregem unicolorem, id est albi et nigri velleris, tradidit in manu filiorum suorum.
${}^{36}$~Et posuit spatium itineris trium dierum inter se et generum, qui pascebat reliquos greges ejus.
${}^{37}$~Tollens ergo Jacob virgas populeas virides, et amygdalinas, et ex platanis, ex parte decorticavit eas~: detractisque corticibus, in his, qu\ae\ spoliata fuerant, candor apparuit~: illa vero qu\ae\ integra fuerant, viridia permanserunt~: atque in hunc modum color effectus est varius.
${}^{38}$~Posuitque eas in canalibus, ubi effundebatur aqua~: ut cum venissent greges ad bibendum, ante oculos haberent virgas, et in aspectu earum conciperent.
${}^{39}$~Factumque est ut in ipso calore coitus, oves intuerentur virgas, et parerent maculosa, et varia, et diverso colore respersa.
${}^{40}$~Divisitque gregem Jacob, et posuit virgas in canalibus ante oculos arietum~: erant autem alba et nigra qu\ae que, Laban~; cetera vero, Jacob, separatis inter se gregibus.
${}^{41}$~Igitur quando primo tempore ascendebantur oves, ponebat Jacob virgas in canalibus aquarum ante oculos arietum et ovium, ut in earum contemplatione conciperent~:
${}^{42}$~quando vero serotina admissura erat, et conceptus extremus, non ponebat eas. Factaque sunt ea qu\ae\ erant serotina, Laban~: et qu\ae\ primi temporis, Jacob.
${}^{43}$~Ditatusque est homo ultra modum, et habuit greges multos, ancillas et servos, camelos et asinos.

\bchapter
\mylettrine{P}ostquam autem audivit verba filiorum Laban dicentium~: Tulit Jacob omnia qu\ae\ fuerunt patris nostri, et de illius facultate ditatus, factus est inclytus~:
${}^{2}$~animadvertit quoque faciem Laban, quod non esset erga se sicut heri et nudiustertius,
${}^{3}$~maxime dicente sibi Domino~: Revertere in terram patrum tuorum, et ad generationem tuam, eroque tecum.
${}^{4}$~Misit, et vocavit Rachel et Liam in agrum, ubi pascebat greges,
${}^{5}$~dixitque eis~: Video faciem patris vestri quod non sit erga me sicut heri et nudiustertius~: Deus autem patris mei fuit mecum.
${}^{6}$~Et ips\ae\ nostis quod totis viribus meis servierim patri vestro.
${}^{7}$~Sed et pater vester circumvenit me et mutavit mercedem meam decem vicibus~: et tamen non dimisit eum Deus ut noceret mihi.
${}^{8}$~Si quando dixit~: Vari\ae\ erunt mercedes tu\ae~: pariebant omnes oves varios fœtus~; quando vero e contrario, ait~: Alba qu\ae que accipies pro mercede~: omnes greges alba pepererunt.
${}^{9}$~Tulitque Deus substantiam patris vestri, et dedit mihi.
${}^{10}$~Postquam enim conceptus ovium tempus advenerat, levavi oculos meos, et vidi in somnis ascendentes mares super feminas, varios et maculosos, et diversorum colorum.
${}^{11}$~Dixitque angelus Dei ad me in somnis~: Jacob~? Et ego respondi~: Adsum.
${}^{12}$~Qui ait~: Leva oculos tuos, et vide universos masculos ascendentes super feminas, varios, maculosos, atque respersos. Vidi enim omnia qu\ae\ fecit tibi Laban.
${}^{13}$~Ego sum Deus Bethel, ubi unxisti lapidem, et votum vovisti mihi. Nunc ergo surge, et egredere de terra hac, revertens in terram nativitatis tu\ae .
${}^{14}$~Responderuntque Rachel et Lia~: Numquid habemus residui quidquam in facultatibus et h\ae reditate domus patris nostri~?
${}^{15}$~nonne quasi alienas reputavit nos, et vendidit, comeditque pretium nostrum~?
${}^{16}$~Sed Deus tulit opes patris nostri, et eas tradidit nobis, ac filiis nostris~: unde omnia qu\ae\ pr\ae cepit tibi Deus, fac.
${}^{17}$~Surrexit itaque Jacob, et impositis liberis ac conjugibus suis super camelos, abiit.
${}^{18}$~Tulitque omnem substantiam suam, et greges, et quidquid in Mesopotamia acquisierat, pergens ad Isaac patrem suum in terram Chanaan.


${}^{19}$~Eo tempore ierat Laban ad tondendas oves, et Rachel furata est idola patris sui.
${}^{20}$~Noluitque Jacob confiteri socero suo quod fugeret.
${}^{21}$~Cumque abiisset tam ipse quam omnia qu\ae\ juris sui erant, et amne transmisso pergeret contra montem Galaad,
${}^{22}$~nuntiatum est Laban die tertio quod fugeret Jacob.
${}^{23}$~Qui, assumptis fratribus suis, persecutus est eum diebus septem~: et comprehendit eum in monte Galaad.
${}^{24}$~Viditque in somnis dicentem sibi Deum~: Cave ne quidquam aspere loquaris contra Jacob.
${}^{25}$~Jamque Jacob extenderat in monte tabernaculum~: cumque ille consecutus fuisset eum cum fratribus suis, in eodem monte Galaad fixit tentorium.
${}^{26}$~Et dixit ad Jacob~: Quare ita egisti, ut clam me abigeres filias meas quasi captivas gladio~?
${}^{27}$~cur ignorante me fugere voluisti, nec indicare mihi, ut prosequerer te cum gaudio, et canticis, et tympanis, et citharis~?
${}^{28}$~Non es passus ut oscularer filios meos et filias~: stulte operatus es~: et nunc quidem
${}^{29}$~valet manus mea reddere tibi malum~: sed Deus patris vestri heri dixit mihi~: Cave ne loquaris contra Jacob quidquam durius.
${}^{30}$~Esto, ad tuos ire cupiebas, et desiderio erat tibi domus patris tui~: cur furatus es deos meos~?
${}^{31}$~Respondit Jacob~: Quod inscio te profectus sum, timui ne violenter auferres filias tuas.
${}^{32}$~Quod autem furti me arguis~: apud quemcumque inveneris deos tuos, necetur coram fratribus nostris~: scrutare, quidquid tuorum apud me inveneris, et aufer. H\ae c dicens, ignorabat quod Rachel furata esset idola.
${}^{33}$~Ingressus itaque Laban tabernaculum Jacob, et Li\ae , et utriusque famul\ae , non invenit. Cumque intrasset tentorium Rachelis,
${}^{34}$~illa festinans abscondit idola subter stramenta cameli, et sedit desuper~: scrutantique omne tentorium, et nihil invenienti,
${}^{35}$~ait~: Ne irascatur dominus meus quod coram te assurgere nequeo~: quia juxta consuetudinem feminarum nunc accidit mihi~: sic delusa sollicitudo qu\ae rentis est.
${}^{36}$~Tumensque Jacob, cum jurgio ait~: Quam ob culpam meam, et ob quod peccatum meum sic exarsisti post me,
${}^{37}$~et scrutatus es omnem supellectilem meam~? quid invenisti de cuncta substantia domus tu\ae~? pone hic coram fratribus meis, et fratribus tuis, et judicent inter me et te.
${}^{38}$~Idcirco viginti annis fui tecum~? oves tu\ae\ et capr\ae\ steriles non fuerunt, arietes gregis tui non comedi~:
${}^{39}$~nec captum a bestia ostendi tibi, ego damnum omne reddebam~: quidquid furto peribat, a me exigebas~:
${}^{40}$~die noctuque \ae stu urebar, et gelu, fugiebatque somnus ab oculis meis.
${}^{41}$~Sicque per viginti annos in domo tua servivi tibi, quatuordecim pro filiabus, et sex pro gregibus tuis~: immutasti quoque mercedem meam decem vicibus.
${}^{42}$~Nisi Deus patris mei Abraham, et timor Isaac affuisset mihi, forsitan modo nudum me dimisisses~: afflictionem meam et laborem manuum mearum respexit Deus, et arguit te heri.


${}^{43}$~Respondit ei Laban~: Fili\ae\ me\ae\ et filii, et greges tui, et omnia qu\ae\ cernis, mea sunt~: quid possum facere filiis et nepotibus meis~?
${}^{44}$~Veni ergo, et ineamus fœdus, ut sit in testimonium inter me et te.
${}^{45}$~Tulit itaque Jacob lapidem, et erexit illum in titulum~:
${}^{46}$~dixitque fratribus suis~: Afferte lapides. Qui congregantes fecerunt tumulum, comederuntque super eum~:
${}^{47}$~quem vocavit Laban Tumulum testis~: et Jacob, Acervum testimonii, uterque juxta proprietatem lingu\ae\ su\ae .
${}^{48}$~Dixitque Laban~: Tumulus iste erit testis inter me et te hodie, et idcirco appellatum est nomen ejus Galaad, id est, Tumulus testis.
${}^{49}$~Intueatur et judicet Dominus inter nos quando recesserimus a nobis,
${}^{50}$~si afflixeris filias meas, et si introduxeris alias uxores super eas~: nullus sermonis nostri testis est absque Deo, qui pr\ae sens respicit.
${}^{51}$~Dixitque rursus ad Jacob~: En tumulus hic, et lapis quem erexi inter me et te,
${}^{52}$~testis erit~: tumulus, inquam, iste et lapis sint in testimonium, si aut ego transiero illum pergens ad te, aut tu pr\ae terieris, malum mihi cogitans.
${}^{53}$~Deus Abraham, et Deus Nachor, judicet inter nos, Deus patris eorum. Juravit ergo Jacob per timorem patris sui Isaac~:
${}^{54}$~immolatisque victimis in monte, vocavit fratres suos ut ederent panem. Qui cum comedissent, manserunt ibi~:
${}^{55}$~Laban vero de nocte consurgens, osculatus est filios, et filias suas, et benedixit illis~: reversusque est in locum suum.

\bchapter
\mylettrine{J}acob quoque abiit itinere quo cœperat~: fueruntque ei obviam angeli Dei.
${}^{2}$~Quos cum vidisset, ait~: Castra Dei sunt h\ae c~: et appellavit nomen loci illius Mahanaim, id est, Castra.
${}^{3}$~Misit autem et nuntios ante se ad Esau fratrem suum in terram Seir, in regionem Edom~:
${}^{4}$~pr\ae cepitque eis, dicens~: Sic loquimini domino meo Esau~: H\ae c dicit frater tuus Jacob~: Apud Laban peregrinatus sum, et fui usque in pr\ae sentem diem.
${}^{5}$~Habeo boves, et asinos, et oves, et servos, et ancillas~: mittoque nunc legationem ad dominum meum, ut inveniam gratiam in conspectu tuo.
${}^{6}$~Reversique sunt nuntii ad Jacob, dicentes~: Venimus ad Esau fratrem tuum, et ecce properat tibi in occursum cum quadringentis viris.
${}^{7}$~Timuit Jacob valde~: et perterritus divisit populum qui secum erat, greges quoque et oves, et boves, et camelos, in duas turmas,
${}^{8}$~dicens~: Si venerit Esau ad unam turmam, et percusserit eam, alia turma, qu\ae\ relicta est, salvabitur.
${}^{9}$~Dixitque Jacob~: Deus patris mei Abraham, et Deus patris mei Isaac~: Domine qui dixisti mihi~: Revertere in terram tuam, et in locum nativitatis tu\ae , et benefaciam tibi~:
${}^{10}$~minor sum cunctis miserationibus tuis, et veritate tua quam explevisti servo tuo. In baculo meo transivi Jordanem istum~: et nunc cum duabus turmis regredior.
${}^{11}$~Erue me de manu fratris mei Esau, quia valde eum timeo~: ne forte veniens percutiat matrem cum filiis.
${}^{12}$~Tu locutus es quod benefaceres mihi, et dilatares semen meum sicut arenam maris, qu\ae\ pr\ae\ multitudine numerari non potest.
${}^{13}$~Cumque dormisset ibi nocte illa, separavit de his qu\ae\ habebat, munera Esau fratri suo,
${}^{14}$~capras ducentas, hircos viginti, oves ducentas, et arietes viginti,
${}^{15}$~camelos fœtas cum pullis suis triginta, vaccas quadraginta, et tauros viginti, asinas viginti et pullos earum decem.
${}^{16}$~Et misit per manus servorum suorum singulos seorsum greges, dixitque pueris suis~: Antecedite me, et sit spatium inter gregem et gregem.
${}^{17}$~Et pr\ae cepit priori, dicens~: Si obvium habueris fratrem meum Esau, et interrogaverit te~: Cujus es~? aut, Quo vadis~? aut, Cujus sunt ista qu\ae\ sequeris~?
${}^{18}$~respondebis~: Servi tui Jacob, munera misit domino meo Esau, ipse quoque post nos venit.
${}^{19}$~Similiter dedit mandata secundo, et tertio, et cunctis qui sequebantur greges, dicens~: Iisdem verbis loquimini ad Esau cum inveneritis eum.
${}^{20}$~Et addetis~: Ipse quoque servus tuus Jacob iter nostrum insequitur. Dixit enim~: Placabo illum muneribus qu\ae\ pr\ae cedunt, et postea videbo illum~: forsitan propitiabitur mihi.
${}^{21}$~Pr\ae cesserunt itaque munera ante eum, ipse vero mansit nocte illa in castris.


${}^{22}$~Cumque mature surrexisset, tulit duas uxores suas, et totidem famulas cum undecim filiis, et transivit vadum Jaboc.
${}^{23}$~Traductisque omnibus qu\ae\ ad se pertinebant,
${}^{24}$~mansit solus~: et ecce vir luctabatur cum eo usque mane.
${}^{25}$~Qui cum videret quod eum superare non posset, tetigit nervum femoris ejus, et statim emarcuit.
${}^{26}$~Dixitque ad eum~: Dimitte me~: jam enim ascendit aurora. Respondit~: Non dimittam te, nisi benedixeris mihi.
${}^{27}$~Ait ergo~: Quod nomen est tibi~? Respondit~: Jacob.
${}^{28}$~At ille~: Nequaquam, inquit, Jacob appellabitur nomen tuum, sed Isra\"el~: quoniam si contra Deum fortis fuisti, quanto magis contra homines pr\ae valebis~?
${}^{29}$~Interrogavit eum Jacob~: Dic mihi, quo appellaris nomine~? Respondit~: Cur qu\ae ris nomen meum~? Et benedixit ei in eodem loco.
${}^{30}$~Vocavitque Jacob nomen loci illius Phanuel, dicens~: Vidi Deum facie ad faciem, et salva facta est anima mea.
${}^{31}$~Ortusque est ei statim sol, postquam transgressus est Phanuel~: ipse vero claudicabat pede.
${}^{32}$~Quam ob causam non comedunt nervum filii Isra\"el, qui emarcuit in femore Jacob, usque in pr\ae sentem diem~: eo quod tetigerit nervum femoris ejus, et obstupuerit.

\bchapter
\mylettrine{E}levans autem Jacob oculos suos, vidit venientem Esau, et cum eo quadringentos viros~: divisitque filios Li\ae\ et Rachel, ambarumque famularum~:
${}^{2}$~et posuit utramque ancillam, et liberos earum, in principio~: Liam vero, et filios ejus, in secundo loco~: Rachel autem et Joseph novissimos.
${}^{3}$~Et ipse progrediens adoravit pronus in terram septies, donec appropinquaret frater ejus.
${}^{4}$~Currens itaque Esau obviam fratri suo, amplexatus est eum~: stringensque collum ejus, et osculans flevit.
${}^{5}$~Levatisque oculis, vidit mulieres et parvulos earum, et ait~: Quid sibi volunt isti~? et si ad te pertinent~? Respondit~: Parvuli sunt quos donavit mihi Deus servo tuo.
${}^{6}$~Et appropinquantes ancill\ae\ et filii earum, incurvati sunt.
${}^{7}$~Accessit quoque Lia cum pueris suis~: et cum similiter adorassent, extremi Joseph et Rachel adoraverunt.
${}^{8}$~Dixitque Esau~: Qu\ae nam sunt ist\ae\ turm\ae\ quas obviam habui~? Respondit~: Ut invenirem gratiam coram domino meo.
${}^{9}$~At ille ait~: Habeo plurima, frater mi, sint tua tibi.
${}^{10}$~Dixitque Jacob~: Noli ita, obsecro~: sed si inveni gratiam in oculis tuis, accipe munusculum de manibus meis. Sic enim vidi faciem tuam, quasi viderim vultum Dei~: esto mihi propitius,
${}^{11}$~et suscipe benedictionem quam attuli tibi, et quam donavit mihi Deus tribuens omnia. Vix fratre compellente, suscipiens,
${}^{12}$~ait~: Gradiamur simul, eroque socius itineris tui.
${}^{13}$~Dixitque Jacob~: Nosti, domine mi, quod parvulos habeam teneros, et oves, et boves fœtas mecum~: quas si plus in ambulando fecero laborare, morientur una die cuncti greges.
${}^{14}$~Pr\ae cedat dominus meus ante servum suum~: et ego sequar paulatim vestigia ejus, sicut videro parvulos meos posse, donec veniam ad dominum meum in Seir.
${}^{15}$~Respondit Esau~: Oro te, ut de populo qui mecum est, saltem socii remaneant vi\ae\ tu\ae . Non est, inquit, necesse~: hoc uno tantum indigeo, ut inveniam gratiam in conspectu tuo, domine mi.
${}^{16}$~Reversus est itaque illo die Esau itinere quo venerat in Seir.
${}^{17}$~Et Jacob venit in Socoth~: ubi \ae dificata domo et fixis tentoriis appellavit nomen loci illius Socoth, id est, Tabernacula.
${}^{18}$~Transivitque in Salem urbem Sichimorum, qu\ae\ est in terra Chanaan, postquam reversus est de Mesopotamia Syri\ae~: et habitavit juxta oppidum.
${}^{19}$~Emitque partem agri, in qua fixerat tabernacula, a filiis Hemor patris Sichem centum agnis.
${}^{20}$~Et erecto ibi altari, invocavit super illud fortissimum Deum Isra\"el.

\bchapter
\mylettrine{E}gressa est autem Dina filia Li\ae\ ut videret mulieres regionis illius.
${}^{2}$~Quam cum vidisset Sichem filius Hemor Hev\ae i, princeps terr\ae\ illius, adamavit eam~: et rapuit, et dormivit cum illa, vi opprimens virginem.
${}^{3}$~Et conglutinata est anima ejus cum ea, tristemque delinivit blanditiis.
${}^{4}$~Et pergens ad Hemor patrem suum~: Accipe, inquit, mihi puellam hanc conjugem.
${}^{5}$~Quod cum audisset Jacob absentibus filiis, et in pastu pecorum occupatis, siluit donec redirent.


${}^{6}$~Egresso autem Hemor patre Sichem ut loqueretur ad Jacob,
${}^{7}$~ecce filii ejus veniebant de agro~: auditoque quod acciderat, irati sunt valde, eo quod fœdam rem operatus esset in Isra\"el et, violata filia Jacob, rem illicitam perpetrasset.
${}^{8}$~Locutus est itaque Hemor ad eos~: Sichem filii mei adh\ae sit anima fili\ae\ vestr\ae~: date eam illi uxorem~:
${}^{9}$~et jungamus vicissim connubia~: filias vestras tradite nobis, et filias nostras accipite,
${}^{10}$~et habitate nobiscum~: terra in potestate vestra est~: exercete, negotiamini, et possidete eam.
${}^{11}$~Sed et Sichem ad patrem et ad fratres ejus ait~: Inveniam gratiam coram vobis~: et qu\ae cumque statueritis, dabo~:
${}^{12}$~augete dotem, et munera postulate, et libenter tribuam quod petieritis~: tantum date mihi puellam hanc uxorem.
${}^{13}$~Responderunt filii Jacob Sichem et patri ejus in dolo, s\ae vientes ob stuprum sororis~:
${}^{14}$~Non possumus facere quod petitis, nec dare sororem nostram homini incircumciso~: quod illicitum et nefarium est apud nos.
${}^{15}$~Sed in hoc valebimus fœderari, si volueritis esse similes nostri, et circumcidatur in vobis omne masculini sexus~;
${}^{16}$~tunc dabimus et accipiemus mutuo filias vestras ac nostras~: et habitabimus vobiscum, erimusque unus populus.
${}^{17}$~Si autem circumcidi nolueritis, tollemus filiam nostram, et recedemus.
${}^{18}$~Placuit oblatio eorum Hemor, et Sichem filio ejus,
${}^{19}$~nec distulit adolescens quin statim quod petebatur expleret~: amabat enim puellam valde, et ipse erat inclytus in omni domo patris sui.
${}^{20}$~Ingressique portam urbis, locuti sunt ad populum~:
${}^{21}$~Viri isti pacifici sunt, et volunt habitare nobiscum~: negotientur in terra, et exerceant eam, qu\ae\ spatiosa et lata cultoribus indiget~: filias eorum accipiemus uxores, et nostras illis dabimus.
${}^{22}$~Unum est quo differtur tantum bonum~: si circumcidamus masculos nostros, ritum gentis imitantes.
${}^{23}$~Et substantia eorum, et pecora, et cuncta qu\ae\ possident, nostra erunt~: tantum in hoc acquiescamus, et habitantes simul, unum efficiemus populum.
${}^{24}$~Assensique sunt omnes, circumcisis cunctis maribus.


${}^{25}$~Et ecce, die tertio, quando gravissimus vulnerum dolor est~: arreptis duo filii Jacob, Simeon et Levi fratres Din\ae , gladiis, ingressi sunt urbem confidenter~: interfectisque omnibus masculis,
${}^{26}$~Hemor et Sichem pariter necaverunt, tollentes Dinam de domo Sichem sororem suam.
${}^{27}$~Quibus egressis, irruerunt super occisos ceteri filii Jacob~: et depopulati sunt urbem in ultionem stupri.
${}^{28}$~Oves eorum, et armenta, et asinos, cunctaque vastantes qu\ae\ in domibus et in agris erant,
${}^{29}$~parvulos quoque eorum et uxores duxerunt captivas.
${}^{30}$~Quibus patratis audacter, Jacob dixit ad Simeon et Levi~: Turbastis me, et odiosum fecistis me Chanan\ae is, et Pherez\ae is habitatoribus terr\ae\ hujus~: nos pauci sumus~; illi congregati percutient me, et delebor ego, et domus mea.
${}^{31}$~Responderunt~: Numquid ut scorto abuti debuere sorore nostra~?

\bchapter
\mylettrine{I}nterea locutus est Deus ad Jacob~: Surge, et ascende Bethel, et habita ibi, facque altare Deo qui apparuit tibi quando fugiebas Esau fratrem tuum.
${}^{2}$~Jacob vero convocata omni domo sua, ait~: Abjicite deos alienos qui in medio vestri sunt, et mundamini, ac mutate vestimenta vestra.
${}^{3}$~Surgite, et ascendamus in Bethel, ut faciamus ibi altare Deo~: qui exaudivit me in die tribulationis me\ae , et socius fuit itineris mei.
${}^{4}$~Dederunt ergo ei omnes deos alienos quos habebant, et inaures qu\ae\ erant in auribus eorum~: at ille infodit ea subter terebinthum, qu\ae\ est post urbem Sichem.
${}^{5}$~Cumque profecti essent, terror Dei invasit omnes per circuitum civitates, et non sunt ausi persequi recedentes.
${}^{6}$~Venit igitur Jacob Luzam, qu\ae\ est in terra Chanaan, cognomento Bethel~: ipse et omnis populus cum eo.
${}^{7}$~\AE dificavitque ibi altare, et appellavit nomen loci illius, Domus Dei~: ibi enim apparuit ei Deus cum fugeret fratrem suum.
${}^{8}$~Eodem tempore mortua est Debora nutrix Rebecc\ae , et sepulta est ad radices Bethel subter quercum~: vocatumque est nomen loci illius, Quercus fletus.
${}^{9}$~Apparuit autem iterum Deus Jacob postquam reversus est de Mesopotamia Syri\ae , benedixitque ei
${}^{10}$~dicens~: Non vocaberis ultra Jacob, sed Isra\"el erit nomen tuum. Et appellavit eum Isra\"el,
${}^{11}$~dixitque ei~: Ego Deus omnipotens~: cresce, et multiplicare~: gentes et populi nationum ex te erunt, reges de lumbis tuis egredientur,
${}^{12}$~terramque quam dedi Abraham et Isaac, dabo tibi et semini tuo post te.
${}^{13}$~Et recessit ab eo.
${}^{14}$~Ille vero erexit titulum lapideum in loco quo locutus fuerat ei Deus~: libans super eum libamina, et effundens oleum~:
${}^{15}$~vocansque nomen loci illius Bethel.


${}^{16}$~Egressus autem inde, venit verno tempore ad terram qu\ae\ ducit Ephratam~: in qua cum parturiret Rachel,
${}^{17}$~ob difficultatem partus periclitari cœpit. Dixitque ei obstetrix~: Noli timere, quia et hunc habebis filium.
${}^{18}$~Egrediente autem anima pr\ae\ dolore, et imminente jam morte, vocavit nomen filii sui Benomi, id est, Filius doloris mei~: pater vero appellavit eum Benjamin, id est, Filius dextr\ae .
${}^{19}$~Mortua est ergo Rachel, et sepulta est in via qu\ae\ ducit Ephratam, h\ae c est Bethlehem.
${}^{20}$~Erexitque Jacob titulum super sepulchrum ejus~: hic est titulus monumenti Rachel, usque in pr\ae sentem diem.
${}^{21}$~Egressus inde, fixit tabernaculum trans Turrem gregis.
${}^{22}$~Cumque habitaret in illa regione, abiit Ruben, et dormivit cum Bala concubina patris sui~: quod illum minime latuit. Erant autem filii Jacob duodecim.
${}^{23}$~Filii Li\ae~: primogenitus Ruben, et Simeon, et Levi, et Judas, et Issachar, et Zabulon.
${}^{24}$~Filii Rachel~: Joseph et Benjamin.
${}^{25}$~Filii Bal\ae\ ancill\ae\ Rachelis~: Dan et Nephthali.
${}^{26}$~Filii Zelph\ae\ ancill\ae\ Li\ae~: Gad et Aser~: hi sunt filii Jacob, qui nati sunt ei in Mesopotamia Syri\ae .


${}^{27}$~Venit etiam ad Isaac patrem suum in Mambre, civitatem Arbee, h\ae c est Hebron, in qua peregrinatus est Abraham et Isaac.
${}^{28}$~Et completi sunt dies Isaac centum octoginta annorum.
${}^{29}$~Consumptusque \ae tate mortuus est~: et appositus est populo suo senex et plenus dierum~: et sepelierunt eum Esau et Jacob filii sui.

\bchapter
\mylettrine{H}\ae\ sunt autem generationes Esau, ipse est Edom.
${}^{2}$~Esau accepit uxores de filiabus Chanaan~: Ada filiam Elon Heth\ae i, et Oolibama filiam An\ae\ fili\ae\ Sebeon Hev\ae i~:
${}^{3}$~Basemath quoque filiam Isma\"el sororem Nabaioth.
${}^{4}$~Peperit autem Ada Eliphaz~: Basemath genuit Rahuel~:
${}^{5}$~Oolibama genuit Jehus et Ihelon et Core. Hi filii Esau qui nati sunt ei in terra Chanaan.
${}^{6}$~Tulit autem Esau uxores suas et filios et filias, et omnem animam domus su\ae , et substantiam, et pecora, et cuncta qu\ae\ habere poterat in terra Chanaan~: et abiit in alteram regionem, recessitque a fratre suo Jacob.
${}^{7}$~Divites enim erant valde, et simul habitare non poterant~: nec sustinebat eos terra peregrinationis eorum pr\ae\ multitudine gregum.
${}^{8}$~Habitavitque Esau in monte Seir, ipse est Edom.
${}^{9}$~H\ae\ autem sunt generationes Esau patris Edom in monte Seir,
${}^{10}$~et h\ae c nomina filiorum ejus~: Eliphaz filius Ada uxoris Esau~: Rahuel quoque filius Basemath uxoris ejus.
${}^{11}$~Fueruntque Eliphaz filii~: Theman, Omar, Sepho, et Gatham, et Cenez.
${}^{12}$~Erat autem Thamna concubina Eliphaz filii Esau~: qu\ae\ peperit ei Amalech. Hi sunt filii Ada uxoris Esau.
${}^{13}$~Filii autem Rahuel~: Nahath et Zara, Samma et Meza~: hi filii Basemath uxoris Esau.
${}^{14}$~Isti quoque erant filii Oolibama fili\ae\ An\ae\ fili\ae\ Sebeon, uxoris Esau, quos genuit ei, Jehus et Ihelon et Core.
${}^{15}$~Hi duces filiorum Esau~: filii Eliphaz primogeniti Esau~: dux Theman, dux Omra, dux Sepho, dux Cenez,
${}^{16}$~dux Core, dux Gathan, dux Amalech. Hi filii Eliphaz in terra Edom, et hi filii Ada.
${}^{17}$~Hi quoque filii Rahuel filii Esau~: dux Nahath, dux Zara, dux Samma, dux Meza~: hi autem duces Rahuel in terra Edom~: isti filii Basemath uxoris Esau.
${}^{18}$~Hi autem filii Oolibama uxoris Esau~: dux Jehus, dux Ihelon, dux Core~: hi duces Oolibama fili\ae\ An\ae\ uxoris Esau.
${}^{19}$~Isti sunt filii Esau, et hi duces eorum~: ipse est Edom.
${}^{20}$~Isti sunt filii Seir Horr\ae i, habitatores terr\ae~: Lotan, et Sobal, et Sebeon, et Ana,
${}^{21}$~et Dison, et Eser, et Disan~: hi duces Horr\ae i, filii Seir in terra Edom.
${}^{22}$~Facti sunt autem filii Lotan~: Hori et Heman. Erat autem soror Lotan, Thamna.
${}^{23}$~Et isti filii Sobal~: Alvan et Manahat et Ebal, et Sepho et Onam.
${}^{24}$~Et hi filii Sebeon~: Aja et Ana. Iste est Ana qui invenit aquas calidas in solitudine, cum pasceret asinos Sebeon patris sui~:
${}^{25}$~habuitque filium Dison, et filiam Oolibama.
${}^{26}$~Et isti filii Dison~: Hamdan, et Eseban, et Jethram, et Charan.
${}^{27}$~Hi quoque filii Eser~: Balaan, et Zavan, et Acan.
${}^{28}$~Habuit autem filios Disan~: Hus et Aram.
${}^{29}$~Hi duces Horr\ae orum~: dux Lotan, dux Sobal, dux Sebeon, dux Ana,
${}^{30}$~dux Dison, dux Eser, dux Disan~: isti duces Horr\ae orum qui imperaverunt in terra Seir.
${}^{31}$~Reges autem qui regnaverunt in terra Edom antequam haberent regem filii Isra\"el, fuerunt hi~:
${}^{32}$~Bela filius Beor, nomenque urbis ejus Denaba.
${}^{33}$~Mortuus est autem Bela, et regnavit pro eo Jobab filius Zar\ae\ de Bosra.
${}^{34}$~Cumque mortuus esset Jobab, regnavit pro eo Husam de terra Themanorum.
${}^{35}$~Hoc quoque mortuo, regnavit pro eo Adad filius Badad, qui percussit Madian in regione Moab~: et nomen urbis ejus Avith.
${}^{36}$~Cumque mortuus esset Adad, regnavit pro eo Semla de Masreca.
${}^{37}$~Hoc quoque mortuo regnavit pro eo Saul de fluvio Rohoboth.
${}^{38}$~Cumque et hic obiisset, successit in regnum Balanan filius Achobor.
${}^{39}$~Isto quoque mortuo regnavit pro eo Adar, nomenque urbis ejus Phau~: et appellabatur uxor ejus Meetabel, filia Matred fili\ae\ Mezaab.
${}^{40}$~H\ae c ergo nomina ducum Esau in cognationibus, et locis, et vocabulis suis~: dux Thamna, dux Alva, dux Jetheth,
${}^{41}$~dux Oolibama, dux Ela, dux Phinon,
${}^{42}$~dux Cenez, dux Theman, dux Mabsar,
${}^{43}$~dux Magdiel, dux Hiram~: hi duces Edom habitantes in terra imperii sui, ipse est Esau pater Idum\ae orum.

\bchapter
\mylettrine{H}abitavit autem Jacob in terra Chanaan, in qua pater suus peregrinatus est.
${}^{2}$~Et h\ae\ sunt generationes ejus~: Joseph cum sedecim esset annorum, pascebat gregem cum fratribus suis adhuc puer~: et erat cum filiis Bal\ae\ et Zelph\ae\ uxorum patris sui~: accusavitque fratres suos apud patrem crimine pessimo.
${}^{3}$~Isra\"el autem diligebat Joseph super omnes filios suos, eo quod in senectute genuisset eum~: fecitque ei tunicam polymitam.
${}^{4}$~Videntes autem fratres ejus quod a patre plus cunctis filiis amaretur, oderant eum, nec poterant ei quidquam pacifice loqui.
${}^{5}$~Accidit quoque ut visum somnium referret fratribus suis~: qu\ae\ causa majoris odii seminarium fuit.
${}^{6}$~Dixitque ad eos~: Audite somnium meum quod vidi~:
${}^{7}$~putabam nos ligare manipulos in agro~: et quasi consurgere manipulum meum, et stare, vestrosque manipulos circumstantes adorare manipulum meum.
${}^{8}$~Responderunt fratres ejus~: Numquid rex noster eris~? aut subjiciemur ditioni tu\ae~? H\ae c ergo causa somniorum atque sermonum, invidi\ae\ et odii fomitem ministravit.
${}^{9}$~Aliud quoque vidit somnium, quod narrans fratribus, ait~: Vidi per somnium, quasi solem, et lunam, et stellas undecim adorare me.
${}^{10}$~Quod cum patri suo, et fratribus retulisset, increpavit eum pater suus, et dixit~: Quid sibi vult hoc somnium quod vidisti~? num ego et mater tua, et fratres tui adorabimus te super terram~?
${}^{11}$~Invidebant ei igitur fratres sui~: pater vero rem tacitus considerabat.


${}^{12}$~Cumque fratres illius in pascendis gregibus patris morarentur in Sichem,
${}^{13}$~dixit ad eum Isra\"el~: Fratres tui pascunt oves in Sichimis~: veni, mittam te ad eos. Quo respondente,
${}^{14}$~Pr\ae sto sum, ait ei~: Vade, et vide si cuncta prospera sint erga fratres tuos, et pecora~: et renuntia mihi quid agatur. Missus de valle Hebron, venit in Sichem~:
${}^{15}$~invenitque eum vir errantem in agro, et interrogavit quid qu\ae reret.
${}^{16}$~At ille respondit~: Fratres meos qu\ae ro~: indica mihi ubi pascant greges.
${}^{17}$~Dixitque ei vir~: Recesserunt de loco isto~: audivi autem eos dicentes~: Eamus in Dothain. Perrexit ergo Joseph post fratres suos, et invenit eos in Dothain.
${}^{18}$~Qui cum vidissent eum procul, antequam accederet ad eos, cogitaverunt illum occidere~:
${}^{19}$~et mutuo loquebantur~: Ecce somniator venit~:
${}^{20}$~venite, occidamus eum, et mittamus in cisternam veterem~: dicemusque~: Fera pessima devoravit eum~: et tunc apparebit quid illi prosint somnia sua.
${}^{21}$~Audiens autem hoc Ruben, nitebatur liberare eum de manibus eorum, et dicebat~:
${}^{22}$~Non interficiatis animam ejus, nec effundatis sanguinem~: sed projicite eum in cisternam hanc, qu\ae\ est in solitudine, manusque vestras servate innoxias~: hoc autem dicebat, volens eripere eum de manibus eorum, et reddere patri suo.
${}^{23}$~Confestim igitur ut pervenit ad fratres suos, nudaverunt eum tunica talari et polymita~:
${}^{24}$~miseruntque eum in cisternam veterem, qu\ae\ non habebat aquam.
${}^{25}$~Et sedentes ut comederent panem, viderunt Isma\"elitas viatores venire de Galaad, et camelos eorum portantes aromata, et resinam, et stacten in \AE gyptum.
${}^{26}$~Dixit ergo Judas fratribus suis~: Quid nobis prodest si occiderimus fratrem nostrum, et celaverimus sanguinem ipsius~?
${}^{27}$~melius est ut venundetur Isma\"elitis, et manus nostr\ae\ non polluantur~: frater enim et caro nostra est. Acquieverunt fratres sermonibus illius.
${}^{28}$~Et pr\ae tereuntibus Madianitis negotiatoribus, extrahentes eum de cisterna, vendiderunt eum Isma\"elitis, viginti argenteis~: qui duxerunt eum in \AE gyptum.
${}^{29}$~Reversusque Ruben ad cisternam, non invenit puerum~:
${}^{30}$~et scissis vestibus pergens ad fratres suos, ait~: Puer non comparet, et ego quo ibo~?
${}^{31}$~Tulerunt autem tunicam ejus, et in sanguine h\ae di, quem occiderant, tinxerunt~:
${}^{32}$~mittentes qui ferrent ad patrem, et dicerent~: Hanc invenimus~: vide utrum tunica filii tui sit, an non.
${}^{33}$~Quam cum agnovisset pater, ait~: Tunica filii mei est~: fera pessima comedit eum, bestia devoravit Joseph.
${}^{34}$~Scissisque vestibus, indutus est cilicio, lugens filium suum multo tempore.
${}^{35}$~Congregatis autem cunctis liberis ejus ut lenirent dolorem patris, noluit consolationem accipere, sed ait~: Descendam ad filium meum lugens in infernum. Et illo perseverante in fletu,
${}^{36}$~Madianit\ae\ vendiderunt Joseph in \AE gypto Putiphari eunucho Pharaonis, magistro militum.

\bchapter
\mylettrine{E}odem tempore, descendens Judas a fratribus suis, divertit ad virum Odollamitem, nomine Hiram.
${}^{2}$~Viditque ibi filiam hominis Chanan\ae i, vocabulo Sue~: et accepta uxore, ingressus est ad eam.
${}^{3}$~Qu\ae\ concepit, et peperit filium, et vocavit nomen ejus Her.
${}^{4}$~Rursumque concepto fœtu, natum filium vocavit Onan.
${}^{5}$~Tertium quoque peperit~: quem appellavit Sela~; quo nato, parere ultra cessavit.
${}^{6}$~Dedit autem Judas uxorem primogenito suo Her, nomine Thamar.
${}^{7}$~Fuit quoque Her primogenitus Jud\ae\ nequam in conspectu Domini~: et ab eo occisus est.
${}^{8}$~Dixit ergo Judas ad Onan filium suum~: Ingredere uxorem fratris tui, et sociare illi, ut suscites semen fratri tuo.
${}^{9}$~Ille sciens non sibi nasci filios, introiens ad uxorem fratris sui, semen fundebat in terram, ne liberi fratris nomine nascerentur.
${}^{10}$~Et idcirco percussit eum Dominus, quod rem detestabilem faceret.
${}^{11}$~Quam ob rem dixit Judas Thamar nurui su\ae~: Esto vidua in domo patris tui, donec crescat Sela filius meus~: timebat enim ne et ipse moreretur, sicut fratres ejus. Qu\ae\ abiit, et habitavit in domo patris sui.
${}^{12}$~Evolutis autem multis diebus, mortua est filia Sue uxor Jud\ae~: qui, post luctum consolatione suscepta, ascendebat ad tonsores ovium suarum, ipse et Hiras opilio gregis Odollamites, in Thamnas.
${}^{13}$~Nuntiatumque est Thamar quod socer illius ascenderet in Thamnas ad tondendas oves.
${}^{14}$~Qu\ae , depositis viduitatis vestibus, assumpsit theristrum~: et mutato habitu, sedit in bivio itineris, quod ducit Thamnam~: eo quod crevisset Sela, et non eum accepisset maritum.
${}^{15}$~Quam cum vidisset Judas, suspicatus est esse meretricem~: operuerat enim vultum suum, ne agnosceretur.
${}^{16}$~Ingrediensque ad eam, ait~: Dimitte me ut co\"eam tecum~: nesciebat enim quod nurus sua esset. Qua respondente~: Quid dabis mihi ut fruaris concubitu meo~?
${}^{17}$~dixit~: Mittam tibi h\ae dum de gregibus. Rursumque illa dicente~: Patiar quod vis, si dederis mihi arrhabonem, donec mittas quod polliceris.
${}^{18}$~Ait Judas~: Quid tibi vis pro arrhabone dari~? Respondit~: Annulum tuum, et armillam, et baculum quem manu tenes. Ad unum igitur coitum mulier concepit,
${}^{19}$~et surgens abiit~: depositoque habitu quem sumpserat, induta est viduitatis vestibus.
${}^{20}$~Misit autem Judas h\ae dum per pastorem suum Odollamitem, ut reciperet pignus quod dederat mulieri~: qui cum non invenisset eam,
${}^{21}$~interrogavit homines loci illius~: Ubi est mulier qu\ae\ sedebat in bivio~? Respondentibus cunctis~: Non fuit in loco ista meretrix.
${}^{22}$~Reversus est ad Judam, et dixit ei~: Non inveni eam~: sed et homines loci illius dixerunt mihi, numquam sedisse ibi scortum.
${}^{23}$~Ait Judas~: Habeat sibi, certe mendacii arguere nos non potest, ego misi h\ae dum quem promiseram~: et tu non invenisti eam.
${}^{24}$~Ecce autem post tres menses nuntiaverunt Jud\ae , dicentes~: Fornicata est Thamar nurus tua, et videtur uterus illius intumescere. Dixitque Judas~: Producite eam ut comburatur.
${}^{25}$~Qu\ae\ cum duceretur ad pœnam, misit ad socerum suum, dicens~: De viro, cujus h\ae c sunt, concepi~: cognosce cujus sit annulus, et armilla, et baculus.
${}^{26}$~Qui, agnitis muneribus, ait~: Justior me est~: quia non tradidi eam Sela filio meo. Attamen ultra non cognovit eam.
${}^{27}$~Instante autem partu, apparuerunt gemini in utero~: atque in ipsa effusione infantium unus protulit manum, in qua obstetrix ligavit coccinum, dicens~:
${}^{28}$~Iste egredietur prior.
${}^{29}$~Illo vero retrahente manum, egressus est alter~: dixitque mulier~: Quare divisa est propter te maceria~? et ob hanc causam, vocavit nomen ejus Phares.
${}^{30}$~Postea egressus est frater ejus, in cujus manu erat coccinum~: quem appellavit Zara.

\bchapter
\mylettrine{I}gitur Joseph ductus est in \AE gyptum, emitque eum Putiphar eunuchus Pharaonis, princeps exercitus, vir \ae gyptius, de manu Isma\"elitarum, a quibus perductus erat.
${}^{2}$~Fuitque Dominus cum eo, et erat vir in cunctis prospere agens~: habitavitque in domo domini sui,
${}^{3}$~qui optime noverat Dominum esse cum eo, et omnia, qu\ae\ gerebat, ab eo dirigi in manu illius.
${}^{4}$~Invenitque Joseph gratiam coram domino suo, et ministrabat ei~: a quo pr\ae positus omnibus gubernabat creditam sibi domum, et universa qu\ae\ ei tradita fuerant~:
${}^{5}$~benedixitque Dominus domui \AE gyptii propter Joseph, et multiplicavit tam in \ae dibus quam in agris cunctam ejus substantiam~:
${}^{6}$~nec quidquam aliud noverat, nisi panem quo vescebatur. Erat autem Joseph pulchra facie, et decorus aspectu.
${}^{7}$~Post multos itaque dies injecit domina sua oculos suos in Joseph, et ait~: Dormi mecum.
${}^{8}$~Qui nequaquam acquiescens operi nefario, dixit ad eam~: Ecce dominus meus, omnibus mihi traditis, ignorat quid habeat in domo sua~:
${}^{9}$~nec quidquam est quod non in mea sit potestate, vel non tradiderit mihi, pr\ae ter te, qu\ae\ uxor ejus es~: quomodo ergo possum hoc malum facere, et peccare in Deum meum~?
${}^{10}$~Hujuscemodi verbis per singulos dies, et mulier molesta erat adolescenti~: et ille recusabat stuprum.
${}^{11}$~Accidit autem quadam die ut intraret Joseph domum, et operis quippiam absque arbitris faceret~:
${}^{12}$~et illa, apprehensa lacinia vestimenti ejus, diceret~: Dormi mecum. Qui relicto in manu ejus pallio fugit, et egressus est foras.
${}^{13}$~Cumque vidisset mulier vestem in manibus suis, et se esse contemptam,
${}^{14}$~vocavit ad se homines domus su\ae , et ait ad eos~: En introduxit virum hebr\ae um, ut illuderet nobis~: ingressus est ad me, ut coiret mecum~: cumque ego succlamassem,
${}^{15}$~et audisset vocem meam, reliquit pallium quod tenebam, et fugit foras.
${}^{16}$~In argumentum ergo fidei retentum pallium ostendit marito revertenti domum,
${}^{17}$~et ait~: Ingressus est ad me servus hebr\ae us quem adduxisti, ut illuderet mihi~:
${}^{18}$~cumque audisset me clamare, reliquit pallium quod tenebam, et fugit foras.
${}^{19}$~His auditis dominus, et nimium credulus verbis conjugis, iratus est valde~:
${}^{20}$~tradiditque Joseph in carcerem, ubi vincti regis custodiebantur, et erat ibi clausus.


${}^{21}$~Fuit autem Dominus cum Joseph, et misertus illius dedit ei gratiam in conspectu principis carceris.
${}^{22}$~Qui tradidit in manu illius universos vinctos qui in custodia tenebantur~: et quidquid fiebat, sub ipso erat.
${}^{23}$~Nec noverat aliquid, cunctis ei creditis~: Dominus enim erat cum illo, et omnia opera ejus dirigebat.

\bchapter
\mylettrine{H}is ita gestis, accidit ut peccarent duo eunuchi, pincerna regis \AE gypti, et pistor, domino suo.
${}^{2}$~Iratusque contra eos Pharao (nam alter pincernis pr\ae erat, alter pistoribus),
${}^{3}$~misit eos in carcerem principis militum, in quo erat vinctus et Joseph.
${}^{4}$~At custos carceris tradidit eos Joseph, qui et ministrabat eis~: aliquantulum temporis fluxerat, et illi in custodia tenebantur.
${}^{5}$~Videruntque ambo somnium nocte una, juxta interpretationem congruam sibi~:
${}^{6}$~ad quos cum introisset Joseph mane, et vidisset eos tristes,
${}^{7}$~sciscitatus est eos, dicens~: Cur tristior est hodie solito facies vestra~?
${}^{8}$~Qui responderunt~: Somnium vidimus, et non est qui interpretetur nobis. Dixitque ad eos Joseph~: Numquid non Dei est interpretatio~? referte mihi quid videritis.
${}^{9}$~Narravit prior, pr\ae positus pincernarum, somnium suum~: Videbam coram me vitem,
${}^{10}$~in qua erant tres propagines, crescere paulatim in gemmas, et post flores uvas maturescere~:
${}^{11}$~calicemque Pharaonis in manu mea~: tuli ergo uvas, et expressi in calicem quem tenebam, et tradidi poculum Pharaoni.
${}^{12}$~Respondit Joseph~: H\ae c est interpretatio somnii~: tres propagines, tres adhuc dies sunt~:
${}^{13}$~post quos recordabitur Pharao ministerii tui, et restituet te in gradum pristinum~: dabisque ei calicem juxta officium tuum, sicut ante facere consueveras.
${}^{14}$~Tantum memento mei, cum bene tibi fuerit, et facias mecum misericordiam~: ut suggeras Pharaoni ut educat me de isto carcere~:
${}^{15}$~quia furto sublatus sum de terra Hebr\ae orum, et hic innocens in lacum missus sum.
${}^{16}$~Videns pistorum magister quod prudenter somnium dissolvisset, ait~: Et ego vidi somnium~: quod tria canistra farin\ae\ haberem super caput meum~:
${}^{17}$~et in uno canistro quod erat excelsius, portare me omnes cibos qui fiunt arte pistoria, avesque comedere ex eo.
${}^{18}$~Respondit Joseph~: H\ae c est interpretatio somnii~: tria canistra, tres adhuc dies sunt~:
${}^{19}$~post quos auferet Pharao caput tuum, ac suspendet te in cruce, et lacerabunt volucres carnes tuas.
${}^{20}$~Exinde dies tertius natalitius Pharaonis erat~: qui faciens grande convivium pueris suis, recordatus est inter epulas magistri pincernarum, et pistorum principis.
${}^{21}$~Restituitque alterum in locum suum, ut porrigeret ei poculum~:
${}^{22}$~alterum suspendit in patibulo, ut conjectoris veritas probaretur.
${}^{23}$~Et tamen succedentibus prosperis, pr\ae positus pincernarum oblitus est interpretis sui.

\bchapter
\mylettrine{P}ost duos annos vidit Pharao somnium. Putabat se stare super fluvium,
${}^{2}$~de quo ascendebant septem boves, pulchr\ae\ et crass\ae\ nimis~: et pascebantur in locis palustribus.
${}^{3}$~Ali\ae\ quoque septem emergebant de flumine, fœd\ae\ confect\ae que macie~: et pascebantur in ipsa amnis ripa in locis virentibus~:
${}^{4}$~devoraveruntque eas, quarum mira species et habitudo corporum erat. Expergefactus Pharao,
${}^{5}$~rursum dormivit, et vidit alterum somnium~: septem spic\ae\ pullulabant in culmo uno plen\ae\ atque formos\ae~:
${}^{6}$~ali\ae\ quoque totidem spic\ae\ tenues, et percuss\ae\ uredine oriebantur,
${}^{7}$~devorantes omnium priorum pulchritudinem. Evigilans Pharao post quietem,
${}^{8}$~et facto mane, pavore perterritus, misit ad omnes conjectores \AE gypti, cunctosque sapientes, et accersitis narravit somnium, nec erat qui interpretaretur.
${}^{9}$~Tunc demum reminiscens pincernarum magister, ait~: Confiteor peccatum meum~:
${}^{10}$~iratus rex servis suis, me et magistrum pistorum retrudi jussit in carcerem principis militum~:
${}^{11}$~ubi una nocte uterque vidimus somnium pr\ae sagum futurorum.
${}^{12}$~Erat ibi puer hebr\ae us, ejusdem ducis militum famulus~: cui narrantes somnia,
${}^{13}$~audivimus quidquid postea rei probavit eventus~; ego enim redditus sum officio meo, et ille suspensus est in cruce.
${}^{14}$~Protinus ad regis imperium eductum de carcere Joseph totonderunt~: ac veste mutata obtulerunt ei.
${}^{15}$~Cui ille ait~: Vidi somnia, nec est qui edisserat~: qu\ae\ audivi te sapientissime conjicere.
${}^{16}$~Respondit Joseph~: Absque me Deus respondebit prospera Pharaoni.
${}^{17}$~Narravit ergo Pharao quod viderat~: Putabam me stare super ripam fluminis,
${}^{18}$~et septem boves de amne conscendere, pulchras nimis, et obesis carnibus~: qu\ae\ in pastu paludis virecta carpebant.
${}^{19}$~Et ecce, has sequebantur ali\ae\ septem boves, in tantum deformes et macilent\ae , ut numquam tales in terra \AE gypti viderim~:
${}^{20}$~qu\ae , devoratis et consumptis prioribus,
${}^{21}$~nullum saturitatis dedere vestigium~: sed simili macie et squalore torpebant. Evigilans, rursus sopore depressus,
${}^{22}$~vidi somnium. Septem spic\ae\ pullulabant in culmo uno plen\ae\ atque pulcherrim\ae .
${}^{23}$~Ali\ae\ quoque septem tenues et percuss\ae\ uredine, oriebantur e stipula~:
${}^{24}$~qu\ae\ priorum pulchritudinem devoraverunt. Narravi conjectoribus somnium, et nemo est qui edisserat.
${}^{25}$~Respondit Joseph~: Somnium regis unum est~: qu\ae\ facturus est Deus, ostendit Pharaoni.
${}^{26}$~Septem boves pulchr\ae , et septem spic\ae\ plen\ae , septem ubertatis anni sunt~: eamdemque vim somnii comprehendunt.
${}^{27}$~Septem quoque boves tenues atque macilent\ae , qu\ae\ ascenderunt post eas, et septem spic\ae\ tenues, et vento urente percuss\ae , septem anni ventur\ae\ sunt famis.
${}^{28}$~Qui hoc ordine complebuntur~:
${}^{29}$~ecce septem anni venient fertilitatis magn\ae\ in universa terra \AE gypti,
${}^{30}$~quos sequentur septem anni alii tant\ae\ sterilitatis, ut oblivioni tradatur cuncta retro abundantia~: consumptura est enim fames omnem terram,
${}^{31}$~et ubertatis magnitudinem perditura est inopi\ae\ magnitudo.
${}^{32}$~Quod autem vidisti secundo ad eamdem rem pertinens somnium~: firmitatis indicium est, eo quod fiat sermo Dei, et velocius impleatur.
${}^{33}$~Nunc ergo provideat rex virum sapientem et industrium, et pr\ae ficiat eum terr\ae\ \AE gypti~:
${}^{34}$~qui constituat pr\ae positos per cunctas regiones~: et quintam partem fructuum per septem annos fertilitatis,
${}^{35}$~qui jam nunc futuri sunt, congreget in horrea~: et omne frumentum sub Pharaonis potestate condatur, serveturque in urbibus.
${}^{36}$~Et pr\ae paretur futur\ae\ septem annorum fami, qu\ae\ oppressura est \AE gyptum, et non consumetur terra inopia.
${}^{37}$~Placuit Pharaoni consilium et cunctis ministris ejus~:
${}^{38}$~locutusque est ad eos~: Num invenire poterimus talem virum, qui spiritu Dei plenus sit~?
${}^{39}$~Dixit ergo ad Joseph~: Quia ostendit tibi Deus omnia qu\ae\ locutus es, numquid sapientiorem et consimilem tui invenire potero~?


${}^{40}$~Tu eris super domum meam, et ad tui oris imperium cunctus populus obediet~: uno tantum regni solio te pr\ae cedam.
${}^{41}$~Dixitque rursus Pharao ad Joseph~: Ecce, constitui te super universam terram \AE gypti.
${}^{42}$~Tulitque annulum de manu sua, et dedit eum in manu ejus~: vestivitque eum stola byssina, et collo torquem auream circumposuit.
${}^{43}$~Fecitque eum ascendere super currum suum secundum, clamante pr\ae cone, ut omnes coram eo genu flecterent, et pr\ae positum esse scirent univers\ae\ terr\ae\ \AE gypti.
${}^{44}$~Dixit quoque rex ad Joseph~: Ego sum Pharao~: absque tuo imperio non movebit quisquam manum aut pedem in omni terra \AE gypti.
${}^{45}$~Vertitque nomen ejus, et vocavit eum, lingua \ae gyptiaca, Salvatorem mundi. Deditque illi uxorem Aseneth filiam Putiphare sacerdotis Heliopoleos. Egressus est itaque Joseph ad terram \AE gypti
${}^{46}$~(triginta autem annorum erat quando stetit in conspectu regis Pharaonis), et circuivit omnes regiones \AE gypti.
${}^{47}$~Venitque fertilitas septem annorum~: et in manipulos redact\ae\ segetes congregat\ae\ sunt in horrea \AE gypti.
${}^{48}$~Omnis etiam frugum abundantia in singulis urbibus condita est.
${}^{49}$~Tantaque fuit abundantia tritici, ut aren\ae\ maris co\ae quaretur, et copia mensuram excederet.
${}^{50}$~Nati sunt autem Joseph filii duo antequam veniret fames~: quos peperit ei Aseneth filia Putiphare sacerdotis Heliopoleos.
${}^{51}$~Vocavitque nomen primogeniti Manasses, dicens~: Oblivisci me fecit Deus omnium laborum meorum, et domus patris mei.
${}^{52}$~Nomen quoque secundi appellavit Ephraim, dicens~: Crescere me fecit Deus in terra paupertatis me\ae .
${}^{53}$~Igitur transactis septem ubertatis annis, qui fuerant in \AE gypto,
${}^{54}$~cœperunt venire septem anni inopi\ae , quos pr\ae dixerat Joseph~: et in universo orbe fames pr\ae valuit, in cuncta autem terra \AE gypti panis erat.
${}^{55}$~Qua esuriente, clamavit populus ad Pharaonem, alimenta petens. Quibus ille respondit~: Ite ad Joseph~: et quidquid ipse vobis dixerit, facite.
${}^{56}$~Crescebat autem quotidie fames in omni terra~: aperuitque Joseph universa horrea, et vendebat \AE gyptiis~: nam et illos oppresserat fames.
${}^{57}$~Omnesque provinci\ae\ veniebant in \AE gyptum, ut emerent escas, et malum inopi\ae\ temperarent.

\bchapter
\mylettrine{A}udiens autem Jacob quod alimenta venderentur in \AE gypto, dixit filiis suis~: Quare negligitis~?
${}^{2}$~audivi quod triticum venundetur in \AE gypto~: descendite, et emite nobis necessaria, ut possimus vivere, et non consumamur inopia.
${}^{3}$~Descendentes igitur fratres Joseph decem, ut emerent frumenta in \AE gypto,
${}^{4}$~Benjamin domi retento a Jacob, qui dixerat fratribus ejus~: Ne forte in itinere quidquam patiatur mali~:
${}^{5}$~ingressi sunt terram \AE gypti cum aliis qui pergebant ad emendum. Erat autem fames in terra Chanaan.
${}^{6}$~Et Joseph erat princeps in terra \AE gypti, atque ad ejus nutum frumenta populis vendebantur. Cumque adorassent eum fratres sui,
${}^{7}$~et agnovisset eos, quasi ad alienos durius loquebatur, interrogans eos~: Unde venistis~? Qui responderunt~: De terra Chanaan, ut emamus victui necessaria.
${}^{8}$~Et tamen fratres ipse cognoscens, non est cognitus ab eis.
${}^{9}$~Recordatusque somniorum, qu\ae\ aliquando viderat, ait ad eos~: Exploratores estis~: ut videatis infirmiora terr\ae\ venistis.
${}^{10}$~Qui dixerunt~: Non est ita, domine, sed servi tui venerunt ut emerent cibos.
${}^{11}$~Omnes filii unius viri sumus~: pacifici venimus, nec quidquam famuli tui machinantur mali.
${}^{12}$~Quibus ille respondit~: Aliter est~: immunita terr\ae\ hujus considerare venistis.
${}^{13}$~At illi~: Duodecim, inquiunt, servi tui, fratres sumus, filii viri unius in terra Chanaan~: minimus cum patre nostro est, alius non est super.
${}^{14}$~Hoc est, ait, quod locutus sum~: exploratores estis.
${}^{15}$~Jam nunc experimentum vestri capiam~: per salutem Pharaonis non egrediemini hinc, donec veniat frater vester minimus.
${}^{16}$~Mittite ex vobis unum, et adducat eum~: vos autem eritis in vinculis, donec probentur qu\ae\ dixistis utrum vera an falsa sint~: alioquin per salutem Pharaonis exploratores estis.
${}^{17}$~Tradidit ergo illos custodi\ae\ tribus diebus.
${}^{18}$~Die autem tertio eductis de carcere, ait~: Facite qu\ae\ dixi, et vivetis~: Deum enim timeo.
${}^{19}$~Si pacifici estis, frater vester unus ligetur in carcere~: vos autem abite, et ferte frumenta qu\ae\ emistis, in domos vestras,
${}^{20}$~et fratrem vestrum minimum ad me adducite, ut possim vestros probare sermones, et non moriamini. Fecerunt ut dixerat,
${}^{21}$~et locuti sunt ad invicem~: Merito h\ae c patimur, quia peccavimus in fratrem nostrum, videntes angustiam anim\ae\ illius, dum deprecaretur nos, et non audivimus~: idcirco venit super nos ista tribulatio.
${}^{22}$~E quibus unus Ruben, ait~: Numquid non dixi vobis~: Nolite peccare in puerum~: et non audistis me~? en sanguis ejus exquiritur.
${}^{23}$~Nesciebant autem quod intelligeret Joseph, eo quod per interpretem loqueretur ad eos.
${}^{24}$~Avertitque se parumper, et flevit~: et reversus locutus est ad eos.


${}^{25}$~Tollensque Simeon, et ligans illis pr\ae sentibus, jussit ministris ut implerent eorum saccos tritico, et reponerent pecunias singulorum in sacculis suis, datis supra cibariis in viam~: qui fecerunt ita.
${}^{26}$~At illi portantes frumenta in asinis suis, profecti sunt.
${}^{27}$~Apertoque unus sacco, ut daret jumento pabulum in diversorio, contemplatus pecuniam in ore sacculi,
${}^{28}$~dixit fratribus suis~: Reddita est mihi pecunia, en habetur in sacco. Et obstupefacti, turbatique, mutuo dixerunt~: Quidnam est hoc quod fecit nobis Deus~?
${}^{29}$~Veneruntque ad Jacob patrem suum in terram Chanaan, et narraverunt ei omnia qu\ae\ accidissent sibi, dicentes~:
${}^{30}$~Locutus est nobis dominus terr\ae\ dure, et putavit nos exploratores esse provinci\ae .
${}^{31}$~Cui respondimus~: Pacifici sumus, nec ullas molimur insidias.
${}^{32}$~Duodecim fratres uno patre geniti sumus~: unus non est super, minimus cum patre nostro est in terra Chanaan.
${}^{33}$~Qui ait nobis~: Sic probabo quod pacifici sitis~: fratrem vestrum unum dimittite apud me, et cibaria domibus vestris necessaria sumite, et abite,
${}^{34}$~fratremque vestrum minimum adducite ad me, ut sciam quod non sitis exploratores~: et istum, qui tenetur in vinculis, recipere possitis~: ac deinceps qu\ae\ vultis, emendi habeatis licentiam.
${}^{35}$~His dictis, cum frumenta effunderent, singuli repererunt in ore saccorum ligatas pecunias, exterritisque simul omnibus,
${}^{36}$~dixit pater Jacob~: Absque liberis me esse fecistis~: Joseph non est super, Simeon tenetur in vinculis, et Benjamin auferetis~: in me h\ae c omnia mala reciderunt.
${}^{37}$~Cui respondit Ruben~: Duos filios meos interfice, si non reduxero illum tibi~: trade illum in manu mea, et ego eum tibi restituam.
${}^{38}$~At ille~: Non descendet, inquit, filius meus vobiscum~: frater ejus mortuus est, et ipse solus remansit~: si quid ei adversi acciderit in terra ad quam pergitis, deducetis canos meos cum dolore ad inferos.

\bchapter
\mylettrine{I}nterim fames omnem terram vehementer premebat.
${}^{2}$~Consumptisque cibis quos ex \AE gypto detulerant, dixit Jacob ad filios suos~: Revertimini, et emite nobis pauxillum escarum.
${}^{3}$~Respondit Judas~: Denuntiavit nobis vir ille sub attestatione jurisjurandi, dicens~: Non videbitis faciem meam, nisi fratrem vestrum minimum adduxeritis vobiscum.
${}^{4}$~Si ergo vis eum mittere nobiscum, pergemus pariter, et ememus tibi necessaria~:
${}^{5}$~sin autem non vis, non ibimus~: vir enim, ut s\ae pe diximus, denuntiavit nobis, dicens~: Non videbitis faciem meam absque fratre vestro minimo.
${}^{6}$~Dixit eis Isra\"el~: In meam hoc fecistis miseriam, ut indicaretis ei et alium habere vos fratrem.
${}^{7}$~At illi responderunt~: Interrogavit nos homo per ordinem nostram progeniem~: si pater viveret~: si haberemus fratrem~: et nos respondimus ei consequenter juxta id quod fuerat sciscitatus~: numquid scire poteramus quod dicturus esset~: Adducite fratrem vestrum vobiscum~?
${}^{8}$~Judas quoque dixit patri suo~: Mitte puerum mecum, ut proficiscamur, et possimus vivere~: ne moriamur nos et parvuli nostri.
${}^{9}$~Ego suscipio puerum~: de manu mea require illum~: nisi reduxero, et reddidero eum tibi, ero peccati reus in te omni tempore.
${}^{10}$~Si non intercessisset dilatio, jam vice alter venissemus.
${}^{11}$~Igitur Isra\"el pater eorum dixit ad eos~: Si sic necesse est, facite quod vultis~: sumite de optimis terr\ae\ fructibus in vasis vestris, et deferte viro munera, modicum resin\ae , et mellis, et storacis, stactes, et terebinthi, et amygdalarum.
${}^{12}$~Pecuniam quoque duplicem ferte vobiscum~: et illam, quam invenistis in sacculis, reportate, ne forte errore factum sit~:
${}^{13}$~sed et fratrem vestrum tollite, et ite ad virum.
${}^{14}$~Deus autem meus omnipotens faciat vobis eum placabilem~: et remittat vobiscum fratrem vestrum quem tenet, et hunc Benjamin~: ego autem quasi orbatus absque liberis ero.


${}^{15}$~Tulerunt ergo viri munera, et pecuniam duplicem, et Benjamin~: descenderuntque in \AE gyptum, et steterunt coram Joseph.
${}^{16}$~Quos cum ille vidisset et Benjamin simul, pr\ae cepit dispensatori domus su\ae , dicens~: Introduc viros domum, et occide victimas, et instrue convivium~: quoniam mecum sunt comesturi meridie.
${}^{17}$~Fecit ille quod sibi fuerat imperatum, et introduxit viros domum.
${}^{18}$~Ibique exterriti, dixerunt mutuo~: Propter pecuniam, quam retulimus prius in saccis nostris, introducti sumus~: ut devolvat in nos calumniam, et violenter subjiciat servituti et nos, et asinos nostros.
${}^{19}$~Quam ob rem in ipsis foribus accedentes ad dispensatorem domus,
${}^{20}$~locuti sunt~: Oramus, domine, ut audias nos. Jam ante descendimus ut emeremus escas~:
${}^{21}$~quibus emptis, cum venissemus ad diversorium, aperuimus saccos nostros, et invenimus pecuniam in ore saccorum~: quam nunc eodem pondere reportavimus.
${}^{22}$~Sed et aliud attulimus argentum, ut emamus qu\ae\ nobis necessaria sunt~: non est in nostra conscientia quis posuerit eam in marsupiis nostris.
${}^{23}$~At ille respondit~: Pax vobiscum, nolite timere~: Deus vester, et Deus patris vestri, dedit vobis thesauros in saccis vestris~: nam pecuniam, quam dedistis mihi, probatam ego habeo. Eduxitque ad eos Simeon.
${}^{24}$~Et introductis domum, attulit aquam, et laverunt pedes suos, deditque pabulum asinis eorum.
${}^{25}$~Illi vero parabant munera, donec ingrederetur Joseph meridie~: audierant enim quod ibi comesturi essent panem.
${}^{26}$~Igitur ingressus est Joseph domum suam, obtuleruntque ei munera, tenentes in manibus suis~: et adoraverunt proni in terram.
${}^{27}$~At ille, clementer resalutatis eis, interrogavit eos, dicens~: Salvusne est pater vester senex, de quo dixeratis mihi~? adhuc vivit~?
${}^{28}$~Qui responderunt~: Sospes est servus tuus pater noster, adhuc vivit. Et incurvati, adoraverunt eum.
${}^{29}$~Attollens autem Joseph oculos, vidit Benjamin fratrem suum uterinum, et ait~: Iste est frater vester parvulus, de quo dixeratis mihi~? Et rursum~: Deus, inquit, misereatur tui, fili mi.
${}^{30}$~Festinavitque, quia commota fuerant viscera ejus super fratre suo, et erumpebant lacrim\ae~: et introiens cubiculum flevit.
${}^{31}$~Rursumque lota facie egressus, continuit se, et ait~: Ponite panes.
${}^{32}$~Quibus appositis, seorsum Joseph, et seorsum fratribus, \AE gyptiis quoque qui vescebantur simul, seorsum (illicitum est enim \AE gyptiis comedere cum Hebr\ae is, et profanum putant hujuscemodi convivium)
${}^{33}$~sederunt coram eo, primogenitus juxta primogenita sua, et minimus juxta \ae tatem suam. Et mirabantur nimis,
${}^{34}$~sumptis partibus quas ab eo acceperant~: majorque pars venit Benjamin, ita ut quinque partibus excederet. Biberuntque et inebriati sunt cum eo.

\bchapter
\mylettrine{P}r\ae cepit autem Joseph dispensatori domus su\ae , dicens~: Imple saccos eorum frumento, quantum possunt capere~: et pone pecuniam singulorum in summitate sacci.
${}^{2}$~Scyphum autem meum argenteum, et pretium quod dedit tritici, pone in ore sacci junioris. Factumque est ita.
${}^{3}$~Et orto mane, dimissi sunt cum asinis suis.
${}^{4}$~Jamque urbem exierant, et processerant paululum~: tunc Joseph accersito dispensatore domus, Surge, inquit, et persequere viros~: et apprehensis dicito~: Quare reddidistis malum pro bono~?
${}^{5}$~scyphus, quem furati estis, ipse est in quo bibit dominus meus, et in quo augurari solet~: pessimam rem fecistis.
${}^{6}$~Fecit ille ut jusserat~: et apprehensis per ordinem locutus est.
${}^{7}$~Qui responderunt~: Quare sic loquitur dominus noster, ut servi tui tantum flagitii commiserint~?
${}^{8}$~pecuniam, quam invenimus in summitate saccorum, reportavimus ad te de terra Chanaan~: et quomodo consequens est ut furati simus de domo domini tui aurum vel argentum~?
${}^{9}$~apud quemcumque fuerit inventum servorum tuorum quod qu\ae ris, moriatur, et nos erimus servi domini nostri.
${}^{10}$~Qui dixit eis~: Fiat juxta vestram sententiam~: apud quemcumque fuerit inventum, ipse sit servus meus, vos autem eritis innoxii.
${}^{11}$~Itaque festinato deponentes in terram saccos, aperuerunt singuli.
${}^{12}$~Quos scrutatus, incipiens a majore usque ad minimum, invenit scyphum in sacco Benjamin.
${}^{13}$~At illi, scissis vestibus, oneratisque rursum asinis, reversi sunt in oppidum.
${}^{14}$~Primusque Judas cum fratribus ingressus est ad Joseph (necdum enim de loco abierat) omnesque ante eum pariter in terram corruerunt.
${}^{15}$~Quibus ille ait~: Cur sic agere voluistis~? an ignoratis quod non sit similis mei in augurandi scientia~?
${}^{16}$~Cui Judas~: Quid respondebimus, inquit, domino meo~? vel quid loquemur, aut juste poterimus obtendere~? Deus invenit iniquitatem servorum tuorum~: en omnes servi sumus domini mei, et nos, et apud quem inventus est scyphus.
${}^{17}$~Respondit Joseph~: Absit a me ut sic agam~: qui furatus est scyphum, ipse sit servus meus~: vos autem abite liberi ad patrem vestrum.
${}^{18}$~Accedens autem propius Judas, confidenter ait~: Oro, domine mi, loquatur servus tuus verbum in auribus tuis, et ne irascaris famulo tuo~: tu es enim post Pharaonem
${}^{19}$~dominus meus. Interrogasti prius servos tuos~: Habetis patrem aut fratrem~?
${}^{20}$~et nos respondimus tibi domino meo~: Est nobis pater senex, et puer parvulus, qui in senectute illius natus est~: cujus uterinus frater mortuus est~: et ipsum solum habet mater sua, pater vero tenere diligit eum.
${}^{21}$~Dixistique servis tuis~: Adducite eum ad me, et ponam oculos meos super illum.
${}^{22}$~Suggessimus domino meo~: Non potest puer relinquere patrem suum~: si enim illum dimiserit, morietur.
${}^{23}$~Et dixisti servis tuis~: Nisi venerit frater vester minimus vobiscum, non videbitis amplius faciem meam.
${}^{24}$~Cum ergo ascendissemus ad famulum tuum patrem nostrum, narravimus ei omnia qu\ae\ locutus est dominus meus.
${}^{25}$~Et dixit pater noster~: Revertimini, et emite nobis parum tritici.
${}^{26}$~Cui diximus~: Ire non possumus~: si frater noster minimus descenderit nobiscum, proficiscemur simul~: alioquin illo absente, non audemus videre faciem viri.
${}^{27}$~Ad qu\ae\ ille respondit~: Vos scitis quod duos genuerit mihi uxor mea.
${}^{28}$~Egressus est unus, et dixistis~: Bestia devoravit eum~: et hucusque non comparet.
${}^{29}$~Si tuleritis et istum, et aliquid ei in via contigerit, deducetis canos meos cum mœrore ad inferos.
${}^{30}$~Igitur si intravero ad servum tuum patrem nostrum, et puer defuerit (cum anima illius ex hujus anima pendeat),
${}^{31}$~videritque eum non esse nobiscum, morietur, et deducent famuli tui canos ejus cum dolore ad inferos.
${}^{32}$~Ego proprie servus tuus sim qui in meam hunc recepi fidem, et spopondi dicens~: Nisi reduxero eum, peccati reus ero in patrem meum omni tempore.
${}^{33}$~Manebo itaque servus tuus pro puero in ministerio domini mei, et puer ascendat cum fratribus suis.
${}^{34}$~Non enim possum redire ad patrem meum, absente puero~: ne calamitatis, qu\ae\ oppressura est patrem meum, testis assistam.

\bchapter
\mylettrine{N}on se poterat ultra cohibere Joseph multis coram astantibus~: unde pr\ae cepit ut egrederentur cuncti foras, et nullus interesset alienus agnitioni mutu\ae .
${}^{2}$~Elevavitque vocem cum fletu, quam audierunt \AE gyptii, omnisque domus Pharaonis.
${}^{3}$~Et dixit fratribus suis~: Ego sum Joseph~: adhuc pater meus vivit~? Non poterant respondere fratres nimio terrore perterriti.
${}^{4}$~Ad quos ille clementer~: Accedite, inquit, ad me. Et cum accessissent prope~: Ego sum, ait, Joseph, frater vester, quem vendidistis in \AE gyptum.
${}^{5}$~Nolite pavere, neque vobis durum esse videatur quod vendidistis me in his regionibus~: pro salute enim vestra misit me Deus ante vos in \AE gyptum.
${}^{6}$~Biennium est enim quod cœpit fames esse in terra~: et adhuc quinque anni restant, quibus nec arari poterit, nec meti.
${}^{7}$~Pr\ae misitque me Deus ut reservemini super terram, et escas ad vivendum habere possitis.
${}^{8}$~Non vestro consilio, sed Dei voluntate huc missus sum~: qui fecit me quasi patrem Pharaonis, et dominum univers\ae\ domus ejus, ac principem in omni terra \AE gypti.
${}^{9}$~Festinate, et ascendite ad patrem meum, et dicetis ei~: H\ae c mandat filius tuus Joseph~: Deus fecit me dominum univers\ae\ terr\ae\ \AE gypti~: descende ad me, ne moreris,
${}^{10}$~et habitabis in terra Gessen~: erisque juxta me tu, et filii tui, et filii filiorum tuorum, oves tu\ae , et armenta tua, et universa qu\ae\ possides~:
${}^{11}$~ibique te pascam (adhuc enim quinque anni residui sunt famis) ne et tu pereas, et domus tua, et omnia qu\ae\ possides.
${}^{12}$~En oculi vestri, et oculi fratris mei Benjamin, vident quod os meum loquatur ad vos.
${}^{13}$~Nuntiate patri meo universam gloriam meam, et cuncta qu\ae\ vidistis in \AE gypto~: festinate, et adducite eum ad me.
${}^{14}$~Cumque amplexatus recidisset in collum Benjamin fratris sui, flevit~: illo quoque similiter flente super collum ejus.
${}^{15}$~Osculatusque est Joseph omnes fratres suos, et ploravit super singulos~: post qu\ae\ ausi sunt loqui ad eum.
${}^{16}$~Auditumque est, et celebri sermone vulgatum in aula regis~: Venerunt fratres Joseph~: et gavisus est Pharao, atque omnis familia ejus.
${}^{17}$~Dixitque ad Joseph ut imperaret fratribus suis, dicens~: Onerantes jumenta, ite in terram Chanaan,
${}^{18}$~et tollite inde patrem vestrum et cognationem, et venite ad me~: et ego dabo vobis omnia bona \AE gypti, ut comedatis medullam terr\ae .
${}^{19}$~Pr\ae cipe etiam ut tollant plaustra de terra \AE gypti, ad subvectionem parvulorum suorum ac conjugum~: et dicito~: Tollite patrem vestrum, et properate quantocius venientes.
${}^{20}$~Nec dimittatis quidquam de supellectili vestra~: quia omnes opes \AE gypti vestr\ae\ erunt.


${}^{21}$~Feceruntque filii Isra\"el ut eis mandatum fuerat. Quibus dedit Joseph plaustra, secundum Pharaonis imperium, et cibaria in itinere.
${}^{22}$~Singulis quoque proferri jussit binas stolas~: Benjamin vero dedit trecentos argenteos cum quinque stolis optimis~:
${}^{23}$~tantumdem pecuni\ae\ et vestium mittens patri suo, addens et asinos decem, qui subveherent ex omnibus divitiis \AE gypti, et totidem asinas, triticum in itinere, panesque portantes.
${}^{24}$~Dimisit ergo fratres suos, et proficiscentibus ait~: Ne irascamini in via.
${}^{25}$~Qui ascendentes ex \AE gypto, venerunt in terram Chanaan ad patrem suum Jacob.
${}^{26}$~Et nuntiaverunt ei, dicentes~: Joseph filius tuus vivit~: et ipse dominatur in omni terra \AE gypti. Quo audito Jacob, quasi de gravi somno evigilans, tamen non credebat eis.
${}^{27}$~Illi e contra referebant omnem ordinem rei. Cumque vidisset plaustra et universa qu\ae\ miserat, revixit spiritus ejus,
${}^{28}$~et ait~: Sufficit mihi si adhuc Joseph filius meus vivit~: vadam, et videbo illum antequam moriar.

\bchapter
\mylettrine{P}rofectusque Isra\"el cum omnibus qu\ae\ habebat, venit ad Puteum juramenti~: et mactatis ibi victimis Deo patris sui Isaac,
${}^{2}$~audivit eum per visionem noctis vocantem se, et dicentem sibi~: Jacob, Jacob. Cui respondit~: Ecce adsum.
${}^{3}$~Ait illi Deus~: Ego sum fortissimus Deus patris tui~: noli timere, descende in \AE gyptum, quia in gentem magnam faciam te ibi.
${}^{4}$~Ego descendam tecum illuc, et ego inde adducam te revertentem~: Joseph quoque ponet manus suas super oculos tuos.
${}^{5}$~Surrexit autem Jacob a Puteo juramenti~: tuleruntque eum filii cum parvulis et uxoribus suis in plaustris qu\ae\ miserat Pharao ad portandum senem,
${}^{6}$~et omnia qu\ae\ possederat in terra Chanaan~: venitque in \AE gyptum cum omni semine suo,
${}^{7}$~filii ejus, et nepotes, fili\ae , et cuncta simul progenies.
${}^{8}$~H\ae c sunt autem nomina filiorum Isra\"el, qui ingressi sunt in \AE gyptum, ipse cum liberis suis. Primogenitus Ruben.
${}^{9}$~Filii Ruben~: Henoch et Phallu et Hesron et Charmi.
${}^{10}$~Filii Simeon~: Jamuel et Jamin et Ahod, et Jachin et Sohar, et Saul filius Chanaanitidis.
${}^{11}$~Filii Levi~: Gerson et Caath et Merari.
${}^{12}$~Filii Juda~: Her et Onan et Sela et Phares et Zara~; mortui sunt autem Her et Onan in terra Chanaan. Natique sunt filii Phares~: Hesron et Hamul.
${}^{13}$~Filii Issachar~: Thola et Phua et Job et Semron.
${}^{14}$~Filii Zabulon~: Sared et Elon et Jahelel.
${}^{15}$~Hi filii Li\ae\ quos genuit in Mesopotamia Syri\ae\ cum Dina filia sua~: omnes anim\ae\ filiorum ejus et filiarum, triginta tres.
${}^{16}$~Filii Gad~: Sephion et Haggi et Suni et Esebon et Heri et Arodi et Areli.
${}^{17}$~Filii Aser~: Jamne et Jesua et Jessui et Beria, Sara quoque soror eorum. Filii Beria~: Heber et Melchiel.
${}^{18}$~Hi filii Zelph\ae , quam dedit Laban Li\ae\ fili\ae\ su\ae~: et hos genuit Jacob sedecim animas.
${}^{19}$~Filii Rachel uxoris Jacob~: Joseph et Benjamin.
${}^{20}$~Natique sunt Joseph filii in terra \AE gypti, quos genuit ei Aseneth filia Putiphare sacerdotis Heliopoleos~: Manasses et Ephraim.
${}^{21}$~Filii Benjamin~: Bela et Bechor et Asbel et Gera et Naaman et Echi et Ros et Mophim et Ophim et Ared.
${}^{22}$~Hi filii Rachel quos genuit Jacob~: omnes anim\ae , quatuordecim.
${}^{23}$~Filii Dan~: Husim.
${}^{24}$~Filii Nephthali~: Jasiel et Guni et Jeser et Sallem.
${}^{25}$~Hi filii Bal\ae , quam dedit Laban Racheli fili\ae\ su\ae~: et hos genuit Jacob~: omnes anim\ae , septem.
${}^{26}$~Cunct\ae\ anim\ae , qu\ae\ ingress\ae\ sunt cum Jacob in \AE gyptum, et egress\ae\ sunt de femore illius, absque uxoribus filiorum ejus, sexaginta sex.
${}^{27}$~Filii autem Joseph, qui nati sunt ei in terra \AE gypti, anim\ae\ du\ae . Omnes anim\ae\ domus Jacob, qu\ae\ ingress\ae\ sunt in \AE gyptum, fuere septuaginta.


${}^{28}$~Misit autem Judam ante se ad Joseph, ut nuntiaret ei, et occurreret in Gessen.
${}^{29}$~Quo cum pervenisset, juncto Joseph curro suo, ascendit obviam patri suo ad eumdem locum~: vidensque eum, irruit super collum ejus, et inter amplexus flevit.
${}^{30}$~Dixitque pater ad Joseph~: Jam l\ae tus moriar, quia vidi faciem tuam, et superstitem te relinquo.
${}^{31}$~At ille locutus est ad fratres suos, et ad omnem domum patris sui~: Ascendam, et nuntiabo Pharaoni, dicamque ei~: Fratres mei, et domus patris mei, qui erant in terra Chanaan, venerunt ad me~:
${}^{32}$~et sunt viri pastores ovium, curamque habent alendorum gregum~: pecora sua, et armenta, et omnia qu\ae\ habere potuerunt, adduxerunt secum.
${}^{33}$~Cumque vocaverit vos, et dixerit~: Quod est opus vestrum~?
${}^{34}$~respondebitis~: Viri pastores sumus servi tui, ab infantia nostra usque in pr\ae sens, et nos et patres nostri. H\ae c autem dicetis, ut habitare possitis in terra Gessen~: quia detestantur \AE gyptii omnes pastores ovium.

\bchapter
\mylettrine{I}ngressus ergo Joseph nuntiavit Pharaoni, dicens~: Pater meus et fratres, oves eorum et armenta, et cuncta qu\ae\ possident, venerunt de terra Chanaan~: et ecce consistunt in terra Gessen.
${}^{2}$~Extremos quoque fratrum suorum quinque viros constituit coram rege~:
${}^{3}$~quos ille interrogavit~: Quid habetis operis~? Responderunt~: Pastores ovium sumus servi tui, et nos et patres nostri.
${}^{4}$~Ad peregrinandum in terra tua venimus~: quoniam non est herba gregibus servorum tuorum, ingravescente fame in terra Chanaan~: petimusque ut esse nos jubeas servos tuos in terra Gessen.
${}^{5}$~Dixit itaque rex ad Joseph~: Pater tuus et fratres tui venerunt ad te.
${}^{6}$~Terra \AE gypti in conspectu tuo est~: in optimo loco fac eos habitare, et trade eis terram Gessen. Quod si nosti in eis esse viros industrios, constitue illos magistros pecorum meorum.
${}^{7}$~Post h\ae c introduxit Joseph patrem suum ad regem, et statuit eum coram eo~: qui benedicens illi,
${}^{8}$~et interrogatus ab eo~: Quot sunt dies annorum vit\ae\ tu\ae~?
${}^{9}$~respondit~: Dies peregrinationis me\ae\ centum triginta annorum sunt, parvi et mali, et non pervenerunt usque ad dies patrum meorum quibus peregrinati sunt.
${}^{10}$~Et benedicto rege, egressus est foras.
${}^{11}$~Joseph vero patri et fratribus suis dedit possessionem in \AE gypto in optimo terr\ae\ loco, Ramesses, ut pr\ae ceperat Pharao.
${}^{12}$~Et alebat eos, omnemque domum patris sui, pr\ae bens cibaria singulis.


${}^{13}$~In toto enim orbe panis deerat, et oppresserat fames terram, maxime \AE gypti et Chanaan.
${}^{14}$~E quibus omnem pecuniam congregavit pro venditione frumenti, et intulit eam in \ae rarium regis.
${}^{15}$~Cumque defecisset emptoribus pretium, venit cuncta \AE gyptus ad Joseph, dicens~: Da nobis panes~: quare morimur coram te, deficiente pecunia~?
${}^{16}$~Quibus ille respondit~: Adducite pecora vestra, et dabo vobis pro eis cibos, si pretium non habetis.
${}^{17}$~Qu\ae\ cum adduxissent, dedit eis alimenta pro equis, et ovibus, et bobus, et asinis~: sustentavitque eos illo anno pro commutatione pecorum.
${}^{18}$~Venerunt quoque anno secundo, et dixerunt ei~: Non celabimus dominum nostrum quod deficiente pecunia, pecora simul defecerunt~: nec clam te est, quod absque corporibus et terra nihil habeamus.
${}^{19}$~Cur ergo moriemur te vidente~? et nos et terra nostra tui erimus~: eme nos in servitutem regiam, et pr\ae be semina, ne pereunte cultore redigatur terra in solitudinem.
${}^{20}$~Emit igitur Joseph omnem terram \AE gypti, vendentibus singulis possessiones suas pr\ae\ magnitudine famis. Subjecitque eam Pharaoni,
${}^{21}$~et cunctos populos ejus a novissimis terminis \AE gypti usque ad extremos fines ejus,
${}^{22}$~pr\ae ter terram sacerdotum, qu\ae\ a rege tradita fuerat eis~: quibus et statuta cibaria ex horreis publicis pr\ae bebantur, et idcirco non sunt compulsi vendere possessiones suas.
${}^{23}$~Dixit ergo Joseph ad populos~: En ut cernitis, et vos et terram vestram Pharao possidet~: accipite semina, et serite agros,
${}^{24}$~ut fruges habere possitis. Quintam partem regi dabitis~: quatuor reliquas permitto vobis in sementem, et in cibum familiis et liberis vestris.
${}^{25}$~Qui responderunt~: Salus nostra in manu tua est~: respiciat nos tantum dominus noster, et l\ae ti serviemus regi.
${}^{26}$~Ex eo tempore usque in pr\ae sentem diem, in universa terra \AE gypti regibus quinta pars solvitur, et factum est quasi in legem, absque terra sacerdotali, qu\ae\ libera ab hac conditione fuit.


${}^{27}$~Habitavit ergo Isra\"el in \AE gypto, id est, in terra Gessen, et possedit eam~: auctusque est, et multiplicatus nimis.
${}^{28}$~Et vixit in ea decem et septem annis~: factique sunt omnes dies vit\ae\ illius, centum quadraginta septem annorum.
${}^{29}$~Cumque appropinquare cerneret diem mortis su\ae , vocavit filium suum Joseph, et dixit ad eum~: Si inveni gratiam in conspectu tuo, pone manum tuam sub femore meo~: et facies mihi misericordiam et veritatem, ut non sepelias me in \AE gypto~:
${}^{30}$~sed dormiam cum patribus meis, et auferas me de terra hac, condasque in sepulchro majorum meorum. Cui respondit Joseph~: Ego faciam quod jussisti.
${}^{31}$~Et ille~: Jura ergo, inquit, mihi. Quo jurante, adoravit Isra\"el Deum, conversus ad lectuli caput.

\bchapter
\mylettrine{H}is ita transactis, nuntiatum est Joseph quod \ae grotaret pater suus~: qui, assumptis duobus filiis Manasse et Ephraim, ire perrexit.
${}^{2}$~Dictumque est seni~: Ecce filius tuus Joseph venit ad te. Qui confortatus sedit in lectulo.
${}^{3}$~Et ingresso ad se ait~: Deus omnipotens apparuit mihi in Luza, qu\ae\ est in terra Chanaan~: benedixitque mihi,
${}^{4}$~et ait~: Ego te augebo et multiplicabo, et faciam te in turbas populorum~: daboque tibi terram hanc, et semini tuo post te in possessionem sempiternam.
${}^{5}$~Duo ergo filii tui, qui nati sunt tibi in terra \AE gypti antequam huc venirem ad te, mei erunt~: Ephraim et Manasses, sicut Ruben et Simeon reputabuntur mihi.
${}^{6}$~Reliquos autem quos genueris post eos, tui erunt, et nomine fratrum suorum vocabuntur in possessionibus suis.
${}^{7}$~Mihi enim, quando veniebam de Mesopotamia, mortua est Rachel in terra Chanaan in ipso itinere, eratque vernum tempus~: et ingrediebar Ephratam, et sepelivi eam juxta viam Ephrat\ae , qu\ae\ alio nomine appellatur Bethlehem.
${}^{8}$~Videns autem filios ejus dixit ad eum~: Qui sunt isti~?
${}^{9}$~Respondit~: Filii mei sunt, quos donavit mihi Deus in hoc loco. Adduc, inquit, eos ad me, ut benedicam illis.
${}^{10}$~Oculi enim Isra\"el caligabant pr\ae\ nimia senectute, et clare videre non poterat. Applicitosque ad se, deosculatus et circumplexus eos,
${}^{11}$~dixit ad filium suum~: Non sum fraudatus aspectu tuo~: insuper ostendit mihi Deus semen tuum.
${}^{12}$~Cumque tulisset eos Joseph de gremio patris, adoravit pronus in terram.
${}^{13}$~Et posuit Ephraim ad dexteram suam, id est, ad sinistram Isra\"el~: Manassen vero in sinistra sua, ad dexteram scilicet patris, applicuitque ambos ad eum.
${}^{14}$~Qui extendens manum dexteram, posuit super caput Ephraim minoris fratris~: sinistram autem super caput Manasse qui major natu erat, commutans manus.
${}^{15}$~Benedixitque Jacob filiis Joseph, et ait~: Deus, in cujus conspectu ambulaverunt patres mei Abraham, et Isaac~; Deus qui pascit me ab adolescentia mea usque in pr\ae sentem diem~:
${}^{16}$~angelus, qui eruit me de cunctis malis, benedicat pueris istis~: et invocetur super eos nomen meum, nomina quoque patrum meorum Abraham et Isaac, et crescant in multitudinem super terram.
${}^{17}$~Videns autem Joseph quod posuisset pater suus dexteram manum super caput Ephraim, graviter accepit~: et apprehensam manum patris levare conatus est de capite Ephraim, et transferre super caput Manasse.
${}^{18}$~Dixitque ad patrem~: Non ita convenit, pater~: quia hic est primogenitus, pone dexteram tuam super caput ejus.
${}^{19}$~Qui renuens, ait~: Scio, fili mi, scio~: et iste quidem erit in populos, et multiplicabitur~: sed frater ejus minor, major erit illo~: et semen illius crescet in gentes.
${}^{20}$~Benedixitque eis in tempore illo, dicens~: In te benedicetur Isra\"el, atque dicetur~: Faciat tibi Deus sicut Ephraim, et sicut Manasse. Constituitque Ephraim ante Manassen.
${}^{21}$~Et ait ad Joseph filium suum~: En ego morior, et erit Deus vobiscum, reducetque vos ad terram patrum vestrorum.
${}^{22}$~Do tibi partem unam extra fratres tuos, quam tuli de manu Amorrh\ae i in gladio et arcu meo.

\bchapter
\mylettrine{V}ocavit autem Jacob filios suos, et ait eis~: Congregamini, ut annuntiem qu\ae\ ventura sunt vobis in diebus novissimis.
\begin{flushleft}\begin{verse}\vspace{6pt}${}^{2}$~Congregamini, et audite, filii Jacob,\\ audite Isra\"el patrem vestrum~:\\
${}^{3}$~Ruben, primogenitus meus,\\ tu fortitudo mea, et principium doloris mei~;\\ prior in donis, major in imperio.\\
${}^{4}$~Effusus es sicut aqua, non crescas~: quia ascendisti cubile patris tui,\\ et maculasti stratum ejus.\\
${}^{5}$~Simeon et Levi fratres\\ vasa iniquitatis bellantia.\\
${}^{6}$~In consilium eorum non veniat anima mea,\\ et in cœtu illorum non sit gloria mea~:\\ quia in furore suo occiderunt virum,\\ et in voluntate sua suffoderunt murum.\\
${}^{7}$~Maledictus furor eorum, quia pertinax~:\\ et indignatio eorum, quia dura~:\\ dividam eos in Jacob,\\ et dispergam eos in Isra\"el.\\
${}^{8}$~Juda, te laudabunt fratres tui~:\\ manus tua in cervicibus inimicorum tuorum,\\ adorabunt te filii patris tui.\\
${}^{9}$~Catulus leonis Juda~:\\ ad pr\ae dam, fili mi, ascendisti~:\\ requiescens accubuisti ut leo,\\ et quasi le\ae na~: quis suscitabit eum~?\\
${}^{10}$~Non auferetur sceptrum de Juda,\\ et dux de femore ejus,\\ donec veniat qui mittendus est,\\ et ipse erit expectatio gentium.\\
${}^{11}$~Ligans ad vineam pullum suum,\\ et ad vitem, o fili mi, asinam suam,\\ lavabit in vino stolam suam\\ et in sanguine uv\ae\ pallium suum.\\
${}^{12}$~Pulchriores sunt oculi ejus vino,\\ et dentes ejus lacte candidiores.\\
${}^{13}$~Zabulon in littore maris habitabit,\\ et in statione navium\\ pertingens usque ad Sidonem.\\
${}^{14}$~Issachar asinus fortis\\ accubans inter terminos.\\
${}^{15}$~Vidit requiem, quod esset bona\\ et terram, quod optima~:\\ et supposuit humerum suum ad portandum,\\ factusque est tributis serviens.\\
${}^{16}$~Dan judicabit populum suum\\ sicut et alia tribus in Isra\"el.\\
${}^{17}$~Fiat Dan coluber in via,\\ cerastes in semita,\\ mordens ungulas equi,\\ ut cadat ascensor ejus retro.\\
${}^{18}$~Salutare tuum expectabo, Domine.\\
${}^{19}$~Gad, accinctus pr\ae liabitur ante eum~:\\ et ipse accingetur retrorsum.\\
${}^{20}$~Aser, pinguis panis ejus,\\ et pr\ae bebit delicias regibus.\\
${}^{21}$~Nephthali, cervus emissus,\\ et dans eloquia pulchritudinis.\\
${}^{22}$~Filius accrescens Joseph, filius accrescens et decorus aspectu~:\\ fili\ae\ discurrerunt super murum.\\
${}^{23}$~Sed exasperaverunt eum et jurgati sunt,\\ invideruntque illi habentes jacula.\\
${}^{24}$~Sedit in forti arcus ejus,\\ et dissoluta sunt vincula brachiorum et manuum illius\\ per manus potentis Jacob~:\\ inde pastor egressus est, lapis Isra\"el.\\
${}^{25}$~Deus patris tui erit adjutor tuus,\\ et omnipotens benedicet tibi\\ benedictionibus c\ae li desuper, benedictionibus abyssi jacentis deorsum,\\ benedictionibus uberum et vulv\ae .\\
${}^{26}$~Benedictiones patris tui confortat\ae\ sunt benedictionibus patrum ejus,\\ donec veniret desiderium collium \ae ternorum~:\\ fiant in capite Joseph,\\ et in vertice Nazar\ae i inter fratres suos.\\
${}^{27}$~Benjamin lupus rapax,\\ mane comedat pr\ae dam,\\ et vespere dividet spolia.\end{verse}\end{flushleft}


${}^{28}$~Omnes hi in tribubus Isra\"el duodecim~: h\ae c locutus est eis pater suus, benedixitque singulis benedictionibus propriis.


${}^{29}$~Et pr\ae cepit eis, dicens~: Ego congregor ad populum meum~: sepelite me cum patribus meis in spelunca duplici qu\ae\ est in agro Ephron Heth\ae i,
${}^{30}$~contra Mambre in terra Chanaan, quam emit Abraham cum agro ab Ephron Heth\ae o in possessionem sepulchri.
${}^{31}$~Ibi sepelierunt eum, et Saram uxorem ejus~: ibi sepultus est Isaac cum Rebecca conjuge sua~: ibi et Lia condita jacet.
${}^{32}$~Finitisque mandatis quibus filios instruebat, collegit pedes suos super lectulum, et obiit~: appositusque est ad populum suum.

\bchapter
\mylettrine{Q}uod cernens Joseph, ruit super faciem patris, flens et deosculans eum.
${}^{2}$~Pr\ae cepitque servis suis medicis ut aromatibus condirent patrem.
${}^{3}$~Quibus jussa explentibus, transierunt quadraginta dies~: iste quippe mos erat cadaverum conditorum~: flevitque eum \AE gyptus septuaginta diebus.
${}^{4}$~Et expleto planctus tempore, locutus est Joseph ad familiam Pharaonis~: Si inveni gratiam in conspectu vestro, loquimini in auribus Pharaonis~:
${}^{5}$~eo quod pater meus adjuraverit me dicens~: En morior~: in sepulchro meo, quod fodi mihi in terra Chanaan, sepelies me. Ascendam igitur, et sepeliam patrem meum, ac revertar.
${}^{6}$~Dixitque ei Pharao~: Ascende, et sepeli patrem tuum sicut adjuratus es.
${}^{7}$~Quo ascendente, ierunt cum eo omnes senes domus Pharaonis, cunctique majores natu terr\ae\ \AE gypti~:
${}^{8}$~domus Joseph cum fratribus suis, absque parvulis, et gregibus atque armentis, qu\ae\ dereliquerant in terra Gessen.
${}^{9}$~Habuit quoque in comitatu currus et equites~: et facta est turba non modica.
${}^{10}$~Veneruntque ad Aream Atad, qu\ae\ sita est trans Jordanem~: ubi celebrantes exequias planctu magno atque vehementi impleverunt septem dies.
${}^{11}$~Quod cum vidissent habitatores terr\ae\ Chanaan, dixerunt~: Planctus magnus est iste \AE gyptiis. Et idcirco vocatum est nomen loci illius, Planctus \AE gypti.
${}^{12}$~Fecerunt ergo filii Jacob sicut pr\ae ceperat eis~:
${}^{13}$~et portantes eum in terram Chanaan, sepelierunt eum in spelunca duplici, quam emerat Abraham cum agro in possessionem sepulchri ab Ephron Heth\ae o, contra faciem Mambre.
${}^{14}$~Reversusque est Joseph in \AE gyptum cum fratribus suis, et omni comitatu, sepulto patre.


${}^{15}$~Quo mortuo, timentes fratres ejus, et mutuo colloquentes~: Ne forte memor sit injuri\ae\ quam passus est, et reddat nobis omne malum quod fecimus,
${}^{16}$~mandaverunt ei dicentes~: Pater tuus pr\ae cepit nobis antequam moreretur,
${}^{17}$~ut h\ae c tibi verbis illius diceremus~: Obsecro ut obliviscaris sceleris fratrum tuorum, et peccati atque maliti\ae\ quam exercuerunt in te~: nos quoque oramus ut servis Dei patris tui dimittas iniquitatem hanc. Quibus auditis flevit Joseph.
${}^{18}$~Veneruntque ad eum fratres sui~: et proni adorantes in terram, dixerunt~: Servi tui sumus.
${}^{19}$~Quibus ille respondit~: Nolite timere~: num Dei possumus resistere voluntati~?
${}^{20}$~Vos cogitastis de me malum~: sed Deus vertit illud in bonum, ut exaltaret me, sicut in pr\ae sentiarum cernitis, et salvos faceret multos populos.
${}^{21}$~Nolite timere~: ego pascam vos et parvulos vestros~: consolatusque est eos, et blande ac leniter est locutus.
${}^{22}$~Et habitavit in \AE gypto cum omni domo patris sui~: vixitque centum decem annis. Et vidit Ephraim filios usque ad tertiam generationem. Filii quoque Machir filii Manasse nati sunt in genibus Joseph.
${}^{23}$~Quibus transactis, locutus est fratribus suis~: Post mortem meam Deus visitabit vos, et ascendere vos faciet de terra ista ad terram quam juravit Abraham, Isaac et Jacob.
${}^{24}$~Cumque adjurasset eos atque dixisset~: Deus visitabit vos, asportate ossa mea vobiscum de loco isto~:
${}^{25}$~mortuus est, expletis centum decem vit\ae\ su\ae\ annis. Et conditus aromatibus, repositus est in loculo in \AE gypto.
