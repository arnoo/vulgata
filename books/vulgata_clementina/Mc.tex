\bbook{Evangelium secundum Marcum}
{Marcus}{images/genese_heading}


\bchapter
\mylettrine{I}nitium Evangelii Jesu Christi, Filii Dei.
${}^{2}$~Sicut scriptum est in Isaia propheta~: \begin{flushleft}\begin{verse}Ecce ego mitto angelum meum ante faciem tuam,\\ qui pr\ae parabit viam tuam ante te.\\
${}^{3}$~Vox clamantis in deserto~:\\ Parate viam Domini, rectas facite semitas ejus.\end{verse}\end{flushleft}


${}^{4}$~Fuit Joannes in deserto baptizans, et pr\ae dicans baptismum pœnitenti\ae\ in remissionem peccatorum.
${}^{5}$~Et egrediebatur ad eum omnis Jud\ae \ae\ regio, et Jerosolymit\ae\ universi, et baptizabantur ab illo in Jordanis flumine, confitentes peccata sua.
${}^{6}$~Et erat Joannes vestitus pilis cameli, et zona pellicea circa lumbos ejus, et locustas et mel silvestre edebat.
${}^{7}$~Et pr\ae dicabat dicens~: Venit fortior me post me, cujus non sum dignus procumbens solvere corrigiam calceamentorum ejus.
${}^{8}$~Ego baptizavi vos aqua, ille vero baptizabit vos Spiritu Sancto.


${}^{9}$~Et factum est~: in diebus illis venit Jesus a Nazareth Galil\ae \ae~: et baptizatus est a Joanne in Jordane.
${}^{10}$~Et statim ascendens de aqua, vidit c\ae los apertos, et Spiritum tamquam columbam descendentem, et manentem in ipso.
${}^{11}$~Et vox facta est de c\ae lis~: Tu es Filius meus dilectus, in te complacui.


${}^{12}$~Et statim Spiritus expulit eum in desertum.
${}^{13}$~Et erat in deserto quadraginta diebus, et quadraginta noctibus~: et tentabatur a Satana~: eratque cum bestiis, et angeli ministrabant illi.


${}^{14}$~Postquam autem traditus est Joannes, venit Jesus in Galil\ae am, pr\ae dicans Evangelium regni Dei,
${}^{15}$~et dicens~: Quoniam impletum est tempus, et appropinquavit regnum Dei~: pœnitemini, et credite Evangelio.
${}^{16}$~Et pr\ae teriens secus mare Galil\ae \ae , vidit Simonem, et Andream fratrem ejus, mittentes retia in mare (erant enim piscatores),
${}^{17}$~et dixit eis Jesus~: Venite post me, et faciam vos fieri piscatores hominum.
${}^{18}$~Et protinus relictis retibus, secuti sunt eum.
${}^{19}$~Et progressus inde pusillum, vidit Jacobum Zebed\ae i, et Joannem fratrem ejus, et ipsos componentes retia in navi~:
${}^{20}$~et statim vocavit illos. Et relicto patre suo Zebed\ae o in navi cum mercenariis, secuti sunt eum.


${}^{21}$~Et ingrediuntur Capharnaum~: et statim sabbatis ingressus in synagogam, docebat eos.
${}^{22}$~Et stupebant super doctrina ejus~: erat enim docens eos quasi potestatem habens, et non sicut scrib\ae .
${}^{23}$~Et erat in synagoga eorum homo in spiritu immundo~: et exclamavit,
${}^{24}$~dicens~: Quid nobis et tibi, Jesu Nazarene~? venisti perdere nos~? scio qui sis, Sanctus Dei.
${}^{25}$~Et comminatus est ei Jesus, dicens~: Obmutesce, et exi de homine.
${}^{26}$~Et discerpens eum spiritus immundus, et exclamans voce magna, exiit ab eo.
${}^{27}$~Et mirati sunt omnes, ita ut conquirerent inter se dicentes~: Quidnam est hoc~? qu\ae nam doctrina h\ae c nova~? quia in potestate etiam spiritibus immundis imperat, et obediunt ei.
${}^{28}$~Et processit rumor ejus statim in omnem regionem Galil\ae \ae .


${}^{29}$~Et protinus egredientes de synagoga, venerunt in domum Simonis et Andre\ae , cum Jacobo et Joanne.
${}^{30}$~Decumbebat autem socrus Simonis febricitans~: et statim dicunt ei de illa.
${}^{31}$~Et accedens elevavit eam, apprehensa manu ejus~: et continuo dimisit eam febris, et ministrabat eis.


${}^{32}$~Vespere autem facto cum occidisset sol, afferebant ad eum omnes male habentes, et d\ae monia habentes~:
${}^{33}$~et erat omnis civitas congregata ad januam.
${}^{34}$~Et curavit multos, qui vexabantur variis languoribus, et d\ae monia multa ejiciebat, et non sinebat ea loqui, quoniam sciebant eum.


${}^{35}$~Et diluculo valde surgens, egressus abiit in desertum locum, ibique orabat.
${}^{36}$~Et prosecutus est eum Simon, et qui cum illo erant.
${}^{37}$~Et cum invenissent eum, dixerunt ei~: Quia omnes qu\ae runt te.
${}^{38}$~Et ait illis~: Eamus in proximos vicos, et civitates, ut et ibi pr\ae dicem~: ad hoc enim veni.
${}^{39}$~Et erat pr\ae dicans in synagogis eorum, et in omni Galil\ae a, et d\ae monia ejiciens.


${}^{40}$~Et venit ad eum leprosus deprecans eum~: et genu flexo dixit ei~: Si vis, potes me mundare.
${}^{41}$~Jesus autem misertus ejus, extendit manum suam~: et tangens eum, ait illi~: Volo~: mundare.
${}^{42}$~Et cum dixisset, statim discessit ab eo lepra, et mundatus est.
${}^{43}$~Et comminatus est ei, statimque ejecit illum,
${}^{44}$~et dicit ei~: Vide nemini dixeris~: sed vade, ostende te principi sacerdotum, et offer pro emundatione tua, qu\ae\ pr\ae cepit Moyses in testimonium illis.
${}^{45}$~At ille egressus cœpit pr\ae dicare, et diffamare sermonem, ita ut jam non posset manifeste introire in civitatem, sed foris in desertis locis esset, et conveniebant ad eum undique.

\bchapter
\mylettrine{E}t iterum intravit Capharnaum post dies,
${}^{2}$~et auditum est quod in domo esset, et convenerunt multi, ita ut non caperet neque ad januam, et loquebatur eis verbum.
${}^{3}$~Et venerunt ad eum ferentes paralyticum, qui a quatuor portabatur.
${}^{4}$~Et cum non possent offerre eum illi pr\ae\ turba, nudaverunt tectum ubi erat~: et patefacientes submiserunt grabatum in quo paralyticus jacebat.
${}^{5}$~Cum autem vidisset Jesus fidem illorum, ait paralytico~: Fili, dimittuntur tibi peccata tua.
${}^{6}$~Erant autem illic quidam de scribis sedentes, et cogitantes in cordibus suis~:
${}^{7}$~Quid hic sic loquitur~? blasphemat. Quis potest dimittere peccata, nisi solus Deus~?
${}^{8}$~Quo statim cognito Jesus spiritu suo, quia sic cogitarent intra se, dicit illis~: Quid ista cogitatis in cordibus vestris~?
${}^{9}$~Quid est facilius dicere paralytico~: Dimittuntur tibi peccata~: an dicere~: Surge, tolle grabatum tuum, et ambula~?
${}^{10}$~Ut autem sciatis quia Filius hominis habet potestatem in terra dimittendi peccata (ait paralytico),
${}^{11}$~tibi dico~: Surge, tolle grabatum tuum, et vade in domum tuam.
${}^{12}$~Et statim surrexit ille~: et, sublato grabato, abiit coram omnibus, ita ut mirarentur omnes, et honorificent Deum, dicentes~: Quia numquam sic vidimus.


${}^{13}$~Et egressus est rursus ad mare, omnisque turba veniebat ad eum, et docebat eos.
${}^{14}$~Et cum pr\ae teriret, vidit Levi Alph\ae i sedentem ad telonium, et ait illi~: Sequere me. Et surgens secutus est eum.
${}^{15}$~Et factum est, cum accumberet in domo illius, multi publicani et peccatores simul discumbebant cum Jesu et discipulis ejus~: erant enim multi, qui et sequebantur eum.
${}^{16}$~Et scrib\ae\ et pharis\ae i videntes quia manducaret cum publicanis et peccatoribus, dicebant discipulis ejus~: Quare cum publicanis et peccatoribus manducat et bibit Magister vester~?
${}^{17}$~Hoc audito Jesus ait illis~: Non necesse habent sani medico, sed qui male habent~: non enim veni vocare justos, sed peccatores.
${}^{18}$~Et erant discipuli Joannis et pharis\ae i jejunantes~: et veniunt, et dicunt illi~: Quare discipuli Joannis et pharis\ae orum jejunant, tui autem discipuli non jejunant~?
${}^{19}$~Et ait illis Jesus~: Numquid possunt filii nuptiarum, quamdiu sponsus cum illis est, jejunare~? Quanto tempore habent secum sponsum, non possunt jejunare.
${}^{20}$~Venient autem dies cum auferetur ab eis sponsus~: et tunc jejunabunt in illis diebus.
${}^{21}$~Nemo assumentum panni rudis assuit vestimento veteri~: alioquin aufert supplementum novum a veteri, et major scissura fit.
${}^{22}$~Et nemo mittit vinum novum in utres veteres~: alioquin dirumpet vinum utres, et vinum effundetur, et utres peribunt~: sed vinum novum in utres novos mitti debet.


${}^{23}$~Et factum est iterum cum Dominus sabbatis ambularet per sata, et discipuli ejus cœperunt progredi, et vellere spicas.
${}^{24}$~Pharis\ae i autem dicebant ei~: Ecce, quid faciunt sabbatis quod non licet~?
${}^{25}$~Et ait illis~: Numquam legistis quid fecerit David, quando necessitatem habuit, et esuriit ipse, et qui cum eo erant~?
${}^{26}$~quomodo introivit in domum Dei sub Abiathar principe sacerdotum, et panes propositionis manducavit, quos non licebat manducare, nisi sacerdotibus, et dedit eis qui cum eo erant~?
${}^{27}$~Et dicebat eis~: Sabbatum propter hominem factum est, et non homo propter sabbatum.
${}^{28}$~Itaque Dominus est Filius hominis, etiam sabbati.

\bchapter
\mylettrine{E}t introivit iterum in synagogam~: et erat ibi homo habens manum aridam.
${}^{2}$~Et observabant eum, si sabbatis curaret, ut accusarent illum.
${}^{3}$~Et ait homini habenti manum aridam~: Surge in medium.
${}^{4}$~Et dicit eis~: Licet sabbatis benefacere, an male~? animam salvam facere, an perdere~? At illi tacebant.
${}^{5}$~Et circumspiciens eos cum ira, contristatus super c\ae citate cordis eorum, dicit homini~: Extende manum tuam. Et extendit, et restituta est manus illi.
${}^{6}$~Exeuntes autem pharis\ae i, statim cum Herodianis consilium faciebant adversus eum quomodo eum perderent.


${}^{7}$~Jesus autem cum discipulis suis secessit ad mare~: et multa turba a Galil\ae a et Jud\ae a secuta est eum,
${}^{8}$~et ab Jerosolymis, et ab Idum\ae a, et trans Jordanem~: et qui circa Tyrum et Sidonem multitudo magna, audientes qu\ae\ faciebat, venerunt ad eum.
${}^{9}$~Et dicit discipulis suis ut navicula sibi deserviret propter turbam, ne comprimerent eum~:
${}^{10}$~multos enim sanabat, ita ut irruerent in eum ut illum tangerent, quotquot habebant plagas.
${}^{11}$~Et spiritus immundi, cum illum videbant, procidebant ei~: et clamabant, dicentes~:
${}^{12}$~Tu es Filius Dei. Et vehementer comminabatur eis ne manifestarent illum.


${}^{13}$~Et ascendens in montem vocavit ad se quos voluit ipse~: et venerunt ad eum.
${}^{14}$~Et fecit ut essent duodecim cum illo~: et ut mitteret eos pr\ae dicare.
${}^{15}$~Et dedit illis potestatem curandi infirmitates et ejiciendi d\ae monia.
${}^{16}$~Et imposuit Simoni nomen Petrus~:
${}^{17}$~et Jacobum Zebed\ae i, et Joannem fratrem Jacobi, et imposuit eis nomina Boanerges, quod est, Filii tonitrui~:
${}^{18}$~et Andream, et Philippum, et Bartholom\ae um, et Matth\ae um, et Thomam, et Jacobum Alph\ae i, et Thadd\ae um, et Simonem Canan\ae um,
${}^{19}$~et Judam Iscariotem, qui et tradidit illum.


${}^{20}$~Et veniunt ad domum~: et convenit iterum turba, ita ut non possent neque panem manducare.
${}^{21}$~Et cum audissent sui, exierunt tenere eum~: dicebant enim~: Quoniam in furorem versus est.
${}^{22}$~Et scrib\ae , qui ab Jerosolymis descenderant, dicebant~: Quoniam Beelzebub habet, et quia in principe d\ae moniorum ejicit d\ae monia.


${}^{23}$~Et convocatis eis in parabolis dicebat illis~: Quomodo potest Satanas Satanam ejicere~?
${}^{24}$~Et si regnum in se dividatur, non potest regnum illud stare.
${}^{25}$~Et si domus super semetipsam dispertiatur, non potest domus illa stare.
${}^{26}$~Et si Satanas consurrexerit in semetipsum, dispertitus est, et non poterit stare, sed finem habet.
${}^{27}$~Nemo potest vasa fortis ingressus in domum diripere, nisi prius fortem alliget, et tunc domum ejus diripiet.
${}^{28}$~Amen dico vobis, quoniam omnia dimittentur filiis hominum peccata, et blasphemi\ae\ quibus blasphemaverint~:
${}^{29}$~qui autem blasphemaverit in Spiritum Sanctum, non habebit remissionem in \ae ternum, sed reus erit \ae terni delicti.
${}^{30}$~Quoniam dicebant~: Spiritum immundum habet.


${}^{31}$~Et veniunt mater ejus et fratres~: et foris stantes miserunt ad eum vocantes eum,
${}^{32}$~et sedebat circa eum turba~: et dicunt ei~: Ecce mater tua et fratres tui foris qu\ae runt te.
${}^{33}$~Et respondens eis, ait~: Qu\ae\ est mater mea et fratres mei~?
${}^{34}$~Et circumspiciens eos, qui in circuitu ejus sedebant, ait~: Ecce mater mea et fratres mei.
${}^{35}$~Qui enim fecerit voluntatem Dei, hic frater meus, et soror mea, et mater est.

\bchapter
\mylettrine{E}t iterum cœpit docere ad mare~: et congregata est ad eum turba multa, ita ut navim ascendens sederet in mari, et omnis turba circa mare super terram erat~:
${}^{2}$~et docebat eos in parabolis multa, et dicebat illis in doctrina sua~:
${}^{3}$~Audite~: ecce exiit seminans ad seminandum.
${}^{4}$~Et dum seminat, aliud cecidit circa viam, et venerunt volucres c\ae li, et comederunt illud.
${}^{5}$~Aliud vero cecidit super petrosa, ubi non habuit terram multam~: et statim exortum est, quoniam non habebat altitudinem terr\ae~:
${}^{6}$~et quando exortus est sol, ex\ae stuavit~: et eo quod non habebat radicem, exaruit.
${}^{7}$~Et aliud cecidit in spinas~: et ascenderunt spin\ae , et suffocaverunt illud, et fructum non dedit.
${}^{8}$~Et aliud cecidit in terram bonam~: et dabat fructum ascendentem et crescentem, et afferebat unum triginta, unum sexaginta, et unum centum.
${}^{9}$~Et dicebat~: Qui habet aures audiendi, audiat.
${}^{10}$~Et cum esset singularis, interrogaverunt eum hi qui cum eo erant duodecim, parabolam.
${}^{11}$~Et dicebat eis~: Vobis datum est nosse mysterium regni Dei~: illis autem, qui foris sunt, in parabolis omnia fiunt~:
${}^{12}$~ut videntes videant, et non videant~: et audientes audiant, et non intelligant~: nequando convertantur, et dimittantur eis peccata.


${}^{13}$~Et ait illis~: Nescitis parabolam hanc~? Et quomodo omnes parabolas cognoscetis~?
${}^{14}$~Qui seminat, verbum seminat.
${}^{15}$~Hi autem sunt, qui circa viam, ubi seminatur verbum, et cum audierint, confestim venit Satanas, et aufert verbum, quod seminatum est in cordibus eorum.
${}^{16}$~Et hi sunt similiter, qui super petrosa seminantur~: qui cum audierint verbum, statim cum gaudio accipiunt illud~:
${}^{17}$~et non habent radicem in se, sed temporales sunt~: deinde orta tribulatione et persecutione propter verbum, confestim scandalizantur.
${}^{18}$~Et alii sunt qui in spinas seminantur~: hi sunt qui verbum audiunt,
${}^{19}$~et \ae rumn\ae\ s\ae culi, et deceptio divitiarum, et circa reliqua concupiscenti\ae\ intro\"euntes suffocant verbum, et sine fructu efficitur.
${}^{20}$~Et hi sunt qui super terram bonam seminati sunt, qui audiunt verbum, et suscipiunt, et fructificant, unum triginta, unum sexaginta, et unum centum.


${}^{21}$~Et dicebat illis~: Numquid venit lucerna ut sub modio ponatur, aut sub lecto~? nonne ut super candelabrum ponatur~?
${}^{22}$~Non est enim aliquid absconditum, quod non manifestetur~: nec factum est occultum, sed ut in palam veniat.
${}^{23}$~Si quis habet aures audiendi, audiat.
${}^{24}$~Et dicebat illis~: Videte quid audiatis. In qua mensura mensi fueritis, remetietur vobis, et adjicietur vobis.
${}^{25}$~Qui enim habet, dabitur illi~: et qui non habet, etiam quod habet auferetur ab eo.


${}^{26}$~Et dicebat~: Sic est regnum Dei, quemadmodum si homo jaciat sementem in terram,
${}^{27}$~et dormiat, et exsurgat nocte et die, et semen germinet, et increscat dum nescit ille.
${}^{28}$~Ultro enim terra fructificat, primum herbam, deinde spicam, deinde plenum frumentum in spica.
${}^{29}$~Et cum produxerit fructus, statim mittit falcem, quoniam adest messis.
${}^{30}$~Et dicebat~: Cui assimilabimus regnum Dei~? aut cui parabol\ae\ comparabimus illud~?
${}^{31}$~Sicut granum sinapis, quod cum seminatum fuerit in terra, minus est omnibus seminibus, qu\ae\ sunt in terra~:
${}^{32}$~et cum seminatum fuerit, ascendit, et fit majus omnibus oleribus, et facit ramos magnos, ita ut possint sub umbra ejus aves c\ae li habitare.
${}^{33}$~Et talibus multis parabolis loquebatur eis verbum, prout poterant audire~:
${}^{34}$~sine parabola autem non loquebatur eis~: seorsum autem discipulis suis disserebat omnia.


${}^{35}$~Et ait illis in illa die, cum sero esset factum~: Transeamus contra.
${}^{36}$~Et dimittentes turbam, assumunt eum ita ut erat in navi~: et ali\ae\ naves erant cum illo.
${}^{37}$~Et facta est procella magna venti, et fluctus mittebat in navim, ita ut impleretur navis.
${}^{38}$~Et erat ipse in puppi super cervical dormiens~: et excitant eum, et dicunt illi~: Magister, non ad te pertinet, quia perimus~?
${}^{39}$~Et exsurgens comminatus est vento, et dixit mari~: Tace, obmutesce. Et cessavit ventus~: et facta est tranquillitas magna.
${}^{40}$~Et ait illis~: Quid timidi estis~? necdum habetis fidem~? et timuerunt timore magno, et dicebant ad alterutrum~: Quis, putas, est iste, quia et ventus et mare obediunt ei~?

\bchapter
\mylettrine{E}t venerunt trans fretum maris in regionem Gerasenorum.
${}^{2}$~Et exeunti ei de navi, statim occurrit de monumentis homo in spiritu immundo,
${}^{3}$~qui domicilium habebat in monumentis, et neque catenis jam quisquam poterat eum ligare~:
${}^{4}$~quoniam s\ae pe compedibus et catenis vinctus, dirupisset catenas, et compedes comminuisset, et nemo poterat eum domare~:
${}^{5}$~et semper die ac nocte in monumentis, et in montibus erat, clamans, et concidens se lapidibus.
${}^{6}$~Videns autem Jesum a longe, cucurrit, et adoravit eum~:
${}^{7}$~et clamans voce magna dixit~: Quid mihi et tibi, Jesu Fili Dei altissimi~? adjuro te per Deum, ne me torqueas.
${}^{8}$~Dicebat enim illi~: Exi spiritus immunde ab homine.
${}^{9}$~Et interrogabat eum~: Quod tibi nomen est~? Et dicit ei~: Legio mihi nomen est, quia multi sumus.
${}^{10}$~Et deprecabatur eum multum, ne se expelleret extra regionem.
${}^{11}$~Erat autem ibi circa montem grex porcorum magnus, pascens.
${}^{12}$~Et deprecabantur eum spiritus, dicentes~: Mitte nos in porcos ut in eos intro\"eamus.
${}^{13}$~Et concessit eis statim Jesus. Et exeuntes spiritus immundi introierunt in porcos~: et magno impetu grex pr\ae cipitatus est in mare ad duo millia, et suffocati sunt in mari.
${}^{14}$~Qui autem pascebant eos, fugerunt, et nuntiaverunt in civitatem et in agros. Et egressi sunt videre quid esset factum~:
${}^{15}$~et veniunt ad Jesum~: et vident illum qui a d\ae monio vexabatur, sedentem, vestitum, et san\ae\ mentis, et timuerunt.
${}^{16}$~Et narraverunt illis, qui viderant, qualiter factum esset ei qui d\ae monium habuerat, et de porcis.
${}^{17}$~Et rogare cœperunt eum ut discederet de finibus eorum.
${}^{18}$~Cumque ascenderet navim, cœpit illum deprecari, qui a d\ae monio vexatus fuerat, ut esset cum illo,
${}^{19}$~et non admisit eum, sed ait illi~: Vade in domum tuam ad tuos, et annuntia illis quanta tibi Dominus fecerit, et misertus sit tui.
${}^{20}$~Et abiit, et cœpit pr\ae dicare in Decapoli, quanta sibi fecisset Jesus~: et omnes mirabantur.
${}^{21}$~Et cum transcendisset Jesus in navi rursum trans fretum, convenit turba multa ad eum, et erat circa mare.


${}^{22}$~Et venit quidam de archisynagogis nomine Jairus, et videns eum procidit ad pedes ejus,
${}^{23}$~et deprecabatur eum multum, dicens~: Quoniam filia mea in extremis est, veni, impone manum super eam, ut salva sit, et vivat.
${}^{24}$~Et abiit cum illo, et sequebatur eum turba multa, et comprimebant eum.
${}^{25}$~Et mulier, qu\ae\ erat in profluvio sanguinis annis duodecim,
${}^{26}$~et fuerat multa perpessa a compluribus medicis~: et erogaverat omnia sua, nec quidquam profecerat, sed magis deterius habebat~:
${}^{27}$~cum audisset de Jesu, venit in turba retro, et tetigit vestimentum ejus~:
${}^{28}$~dicebat enim~: Quia si vel vestimentum ejus tetigero, salva ero.
${}^{29}$~Et confestim siccatus est fons sanguinis ejus~: et sensit corpore quia sanata esset a plaga.
${}^{30}$~Et statim Jesus in semetipso cognoscens virtutem qu\ae\ exierat de illo, conversus ad turbam, aiebat~: Quis tetigit vestimenta mea~?
${}^{31}$~Et dicebant ei discipuli sui~: Vides turbam comprimentem te, et dicis~: Quis me tetigit~?
${}^{32}$~Et circumspiciebat videre eam, qu\ae\ hoc fecerat.
${}^{33}$~Mulier vero timens et tremens, sciens quod factum esset in se, venit et procidit ante eum, et dixit ei omnem veritatem.
${}^{34}$~Ille autem dixit ei~: Filia, fides tua te salvam fecit~: vade in pace, et esto sana a plaga tua.
${}^{35}$~Adhuc eo loquente, veniunt ab archisynagogo, dicentes~: Quia filia tua mortua est~: quid ultra vexas magistrum~?
${}^{36}$~Jesus autem audito verbo quod dicebatur, ait archisynagogo~: Noli timere~: tantummodo crede.
${}^{37}$~Et non admisit quemquam se sequi nisi Petrum, et Jacobum, et Joannem fratrem Jacobi.
${}^{38}$~Et veniunt in domum archisynagogi, et videt tumultum, et flentes, et ejulantes multum.
${}^{39}$~Et ingressus, ait illis~: Quid turbamini, et ploratis~? puella non est mortua, sed dormit.
${}^{40}$~Et irridebant eum. Ipse vero ejectis omnibus assumit patrem, et matrem puell\ae , et qui secum erant, et ingreditur ubi puella erat jacens.
${}^{41}$~Et tenens manum puell\ae , ait illi~: Talitha cumi, quod est interpretatum~: Puella (tibi dico), surge.
${}^{42}$~Et confestim surrexit puella, et ambulabat~: erat autem annorum duodecim~: et obstupuerunt stupore magno.
${}^{43}$~Et pr\ae cepit illis vehementer ut nemo id sciret~: et dixit dari illi manducare.

\bchapter
\mylettrine{E}t egressus inde, abiit in patriam suam~: et sequebantur eum discipuli sui~:
${}^{2}$~et facto sabbato cœpit in synagoga docere~: et multi audientes admirabantur in doctrina ejus, dicentes~: Unde huic h\ae c omnia~? et qu\ae\ est sapientia, qu\ae\ data est illi, et virtutes tales, qu\ae\ per manus ejus efficiuntur~?
${}^{3}$~Nonne hic est faber, filius Mari\ae , frater Jacobi, et Joseph, et Jud\ae , et Simonis~? nonne et sorores ejus hic nobiscum sunt~? Et scandalizabantur in illo.
${}^{4}$~Et dicebat illis Jesus~: Quia non est propheta sine honore nisi in patria sua, et in domo sua, et in cognatione sua.
${}^{5}$~Et non poterat ibi virtutem ullam facere, nisi paucos infirmos impositis manibus curavit~:
${}^{6}$~et mirabatur propter incredulitatem eorum, et circuibat castella in circuitu docens.


${}^{7}$~Et vocavit duodecim~: et cœpit eos mittere binos, et dabat illis potestatem spirituum immundorum.
${}^{8}$~Et pr\ae cepit eis ne quid tollerent in via, nisi virgam tantum~: non peram, non panem, neque in zona \ae s,
${}^{9}$~sed calceatos sandaliis, et ne induerentur duabus tunicis.
${}^{10}$~Et dicebat eis~: Quocumque introieritis in domum, illic manete donec exeatis inde~:
${}^{11}$~et quicumque non receperint vos, nec audierint vos, exeuntes inde, excutite pulverem de pedibus vestris in testimonium illis.
${}^{12}$~Et exeuntes pr\ae dicabant ut pœnitentiam agerent~:
${}^{13}$~et d\ae monia multa ejiciebant, et ungebant oleo multos \ae gros, et sanabant.


${}^{14}$~Et audivit rex Herodes (manifestum enim factum est nomen ejus), et dicebat~: Quia Joannes Baptista resurrexit a mortuis~: et propterea virtutes operantur in illo.
${}^{15}$~Alii autem dicebant~: Quia Elias est~; alii vero dicebant~: Quia propheta est, quasi unus ex prophetis.
${}^{16}$~Quo audito Herodes ait~: Quem ego decollavi Joannem, hic a mortuis resurrexit.
${}^{17}$~Ipse enim Herodes misit, ac tenuit Joannem, et vinxit eum in carcere propter Herodiadem uxorem Philippi fratris sui, quia duxerat eam.
${}^{18}$~Dicebat enim Joannes Herodi~: Non licet tibi habere uxorem fratris tui.
${}^{19}$~Herodias autem insidiabatur illi~: et volebat occidere eum, nec poterat.
${}^{20}$~Herodes enim metuebat Joannem, sciens eum virum justum et sanctum~: et custodiebat eum, et audito eo multa faciebat, et libenter eum audiebat.
${}^{21}$~Et cum dies opportunus accidisset, Herodes natalis sui cœnam fecit principibus, et tribunis, et primis Galil\ae \ae~:
${}^{22}$~cumque introisset filia ipsius Herodiadis, et saltasset, et placuisset Herodi, simulque recumbentibus, rex ait puell\ae~: Pete a me quod vis, et dabo tibi~:
${}^{23}$~et juravit illi~: Quia quidquid petieris dabo tibi, licet dimidium regni mei.
${}^{24}$~Qu\ae\ cum exisset, dixit matri su\ae~: Quid petam~? At illa dixit~: Caput Joannis Baptist\ae .
${}^{25}$~Cumque introisset statim cum festinatione ad regem, petivit dicens~: Volo ut protinus des mihi in disco caput Joannis Baptist\ae .
${}^{26}$~Et contristatus est rex~: propter jusjurandum, et propter simul discumbentes, noluit eam contristare~:
${}^{27}$~sed misso spiculatore pr\ae cepit afferri caput ejus in disco. Et decollavit eum in carcere,
${}^{28}$~et attulit caput ejus in disco~: et dedit illud puell\ae , et puella dedit matri su\ae .
${}^{29}$~Quo audito, discipuli ejus venerunt, et tulerunt corpus ejus~: et posuerunt illud in monumento.


${}^{30}$~Et convenientes Apostoli ad Jesum, renuntiaverunt ei omnia qu\ae\ egerant, et docuerant.
${}^{31}$~Et ait illis~: Venite seorsum in desertum locum, et requiescite pusillum. Erant enim qui veniebant et redibant multi~: et nec spatium manducandi habebant.
${}^{32}$~Et ascendentes in navim, abierunt in desertum locum seorsum.
${}^{33}$~Et viderunt eos abeuntes, et cognoverunt multi~: et pedestres de omnibus civitatibus concurrerunt illuc, et pr\ae venerunt eos.
${}^{34}$~Et exiens vidit turbam multam Jesus~: et misertus est super eos, quia erant sicut oves non habentes pastorem, et cœpit docere multa.
${}^{35}$~Et cum jam hora multa fieret, accesserunt discipuli ejus, dicentes~: Desertus est locus hic, et jam hora pr\ae teriit~:
${}^{36}$~dimitte illos, ut euntes in proximas villas et vicos, emant sibi cibos, quos manducent.
${}^{37}$~Et respondens ait illis~: Date illis vos manducare. Et dixerunt ei~: Euntes emamus ducentis denariis panes, et dabimus illis manducare.
${}^{38}$~Et dicit eis~: Quot panes habetis~? ite, et videte. Et cum cognovissent, dicunt~: Quinque, et duos pisces.
${}^{39}$~Et pr\ae cepit illis ut accumbere facerent omnes secundum contubernia super viride fœnum.
${}^{40}$~Et discubuerunt in partes per centenos et quinquagenos.
${}^{41}$~Et acceptis quinque panibus et duobus piscibus, intuens in c\ae lum, benedixit, et fregit panes, et dedit discipulis suis, ut ponerent ante eos~: et duos pisces divisit omnibus.
${}^{42}$~Et manducaverunt omnes, et saturati sunt.
${}^{43}$~Et sustulerunt reliquias, fragmentorum duodecim cophinos plenos, et de piscibus.
${}^{44}$~Erant autem qui manducaverunt quinque millia virorum.


${}^{45}$~Et statim co\"egit discipulos suos ascendere navim, ut pr\ae cederent eum trans fretum ad Bethsaidam, dum ipse dimitteret populum.
${}^{46}$~Et cum dimisisset eos, abiit in montem orare.
${}^{47}$~Et cum sero esset, erat navis in medio mari et ipse solus in terra.
${}^{48}$~Et videns eos laborantes in remigando (erat enim ventus contrarius eis) et circa quartam vigiliam noctis venit ad eos ambulans supra mare~: et volebat pr\ae terire eos.
${}^{49}$~At illi ut viderunt eum ambulantem supra mare, putaverunt phantasma esse, et exclamaverunt.
${}^{50}$~Omnes enim viderunt eum, et conturbati sunt. Et statim locutus est cum eis, et dixit eis~: Confidite, ego sum~: nolite timere.
${}^{51}$~Et ascendit ad illos in navim, et cessavit ventus. Et plus magis intra se stupebant~:
${}^{52}$~non enim intellexerunt de panibus~: erat enim cor eorum obc\ae catum.
${}^{53}$~Et cum transfretassent, venerunt in terram Genesareth, et applicuerunt.
${}^{54}$~Cumque egressi essent de navi, continuo cognoverunt eum~:
${}^{55}$~et percurrentes universam regionem illam, cœperunt in grabatis eos, qui se male habebant, circumferre, ubi audiebant eum esse.
${}^{56}$~Et quocumque introibat, in vicos, vel in villas aut civitates, in plateis ponebant infirmos, et deprecabantur eum, ut vel fimbriam vestimenti ejus tangerent, et quotquot tangebant eum, salvi fiebant.

\bchapter
\mylettrine{E}t conveniunt ad eum pharis\ae i, et quidam de scribis, venientes ab Jerosolymis.
${}^{2}$~Et cum vidissent quosdam ex discipulis ejus communibus manibus, id est non lotis, manducare panes, vituperaverunt.
${}^{3}$~Pharis\ae i enim, et omnes Jud\ae i, nisi crebro laverint manus, non manducant, tenentes traditionem seniorum~:
${}^{4}$~et a foro nisi baptizentur, non comedunt~: et alia multa sunt, qu\ae\ tradita sunt illis servare, baptismata calicum, et urceorum, et \ae ramentorum, et lectorum~:
${}^{5}$~et interrogabant eum pharis\ae i et scrib\ae~: Quare discipuli tui non ambulant juxta traditionem seniorum, sed communibus manibus manducant panem~?
${}^{6}$~At ille respondens, dixit eis~: Bene prophetavit Isaias de vobis hypocritis, sicut scriptum est~: \begin{flushleft}\begin{verse}Populus hic labiis me honorat,\\ cor autem eorum longe est a me~:\\
${}^{7}$~in vanum autem me colunt,\\ docentes doctrinas, et pr\ae cepta hominum.\end{verse}\end{flushleft}


${}^{8}$~Relinquentes enim mandatum Dei, tenetis traditionem hominum, baptismata urceorum et calicum~: et alia similia his facitis multa.
${}^{9}$~Et dicebat illis~: Bene irritum facitis pr\ae ceptum Dei, ut traditionem vestram servetis.
${}^{10}$~Moyses enim dixit~: Honora patrem tuum, et matrem tuam. Et~: Qui maledixerit patri, vel matri, morte moriatur.
${}^{11}$~Vos autem dicitis~: Si dixerit homo patri, aut matri, Corban (quod est donum) quodcumque ex me, tibi profuerit~:
${}^{12}$~et ultra non dimittitis eum quidquam facere patri suo, aut matri,
${}^{13}$~rescindentes verbum Dei per traditionem vestram, quam tradidistis~: et similia hujusmodi multa facitis.


${}^{14}$~Et advocans iterum turbam, dicebat illis~: Audite me omnes, et intelligite.
${}^{15}$~Nihil est extra hominem introiens in eum, quod possit eum coinquinare, sed qu\ae\ de homine procedunt illa sunt qu\ae\ communicant hominem.
${}^{16}$~Si quis habet aures audiendi, audiat.
${}^{17}$~Et cum introisset in domum a turba, interrogabant eum discipuli ejus parabolam.
${}^{18}$~Et ait illis~: Sic et vos imprudentes estis~? Non intelligitis quia omne extrinsecus introiens in hominem, non potest eum communicare~:
${}^{19}$~quia non intrat in cor ejus, sed in ventrum vadit, et in secessum exit, purgans omnes escas~?
${}^{20}$~Dicebat autem, quoniam qu\ae\ de homine exeunt, illa communicant hominem.
${}^{21}$~Ab intus enim de corde hominum mal\ae\ cogitationes procedunt, adulteria, fornicationes, homicidia,
${}^{22}$~furta, avariti\ae , nequiti\ae , dolus, impudiciti\ae , oculus malus, blasphemia, superbia, stultitia.
${}^{23}$~Omnia h\ae c mala ab intus procedunt, et communicant hominem.


${}^{24}$~Et inde surgens abiit in fines Tyri et Sidonis~: et ingressus domum, neminem voluit scire, et non potuit latere.
${}^{25}$~Mulier enim statim ut audivit de eo, cujus filia habebat spiritum immundum, intravit, et procidit ad pedes ejus.
${}^{26}$~Erat enim mulier gentilis, Syrophœnissa genere. Et rogabat eum ut d\ae monium ejiceret de filia ejus.
${}^{27}$~Qui dixit illi~: Sine prius saturari filios~: non est enim bonum sumere panem filiorum, et mittere canibus.
${}^{28}$~At illa respondit, et dixit illi~: Utique Domine, nam et catelli comedunt sub mensa de micis puerorum.
${}^{29}$~Et ait illi~: Propter hunc sermonem vade~: exiit d\ae monium a filia tua.
${}^{30}$~Et cum abiisset domum suam, invenit puellam jacentem supra lectum, et d\ae monium exiisse.


${}^{31}$~Et iterum exiens de finibus Tyri, venit per Sidonem ad mare Galil\ae \ae\ inter medios fines Decapoleos.
${}^{32}$~Et adducunt ei surdum, et mutum, et deprecabantur eum, ut imponat illi manum.
${}^{33}$~Et apprehendens eum de turba seorsum, misit digitos suos in auriculas ejus~: et exspuens, tetigit linguam ejus~:
${}^{34}$~et suspiciens in c\ae lum, ingemuit, et ait illi~: Ephphetha, quod est, Adaperire.
${}^{35}$~Et statim apert\ae\ sunt aures ejus, et solutum est vinculum lingu\ae\ ejus, et loquebatur recte.
${}^{36}$~Et pr\ae cepit illis ne cui dicerent. Quanto autem eis pr\ae cipiebat, tanto magis plus pr\ae dicabant~:
${}^{37}$~et eo amplius admirabantur, dicentes~: Bene omnia fecit~: et surdos fecit audire, et mutos loqui.

\bchapter
\mylettrine{I}n diebus illis iterum cum turba multa esset, nec haberent quod manducarent, convocatis discipulis, ait illis~:
${}^{2}$~Misereor super turbam~: quia ecce jam triduo sustinent me, nec habent quod manducent~:
${}^{3}$~et si dimisero eos jejunos in domum suam, deficient in via~: quidam enim ex eis de longe venerunt.
${}^{4}$~Et responderunt ei discipuli sui~: Unde illos quis poterit saturare panibus in solitudine~?
${}^{5}$~Et interrogavit eos~: Quot panes habetis~? Qui dixerunt~: Septem.
${}^{6}$~Et pr\ae cepit turb\ae\ discumbere super terram. Et accipiens septem panes, gratias agens fregit, et dabat discipulis suis ut apponerent, et apposuerunt turb\ae .
${}^{7}$~Et habebant pisciculos paucos~: et ipsos benedixit, et jussit apponi.
${}^{8}$~Et manducaverunt, et saturati sunt, et sustulerunt quod superaverat de fragmentis, septem sportas.
${}^{9}$~Erant autem qui manducaverunt, quasi quatuor millia~: et dimisit eos.


${}^{10}$~Et statim ascendens navim cum discipulis suis, venit in partes Dalmanutha.
${}^{11}$~Et exierunt pharis\ae i, et cœperunt conquirere cum eo, qu\ae rentes ab illo signum de c\ae lo, tentantes eum.
${}^{12}$~Et ingemiscens spiritu, ait~: Quid generatio ista signum qu\ae rit~? Amen dico vobis, si dabitur generationi isti signum.
${}^{13}$~Et dimittens eos, ascendit iterum navim et abiit trans fretum.


${}^{14}$~Et obliti sunt panes sumere~: et nisi unum panem non habebant secum in navi.
${}^{15}$~Et pr\ae cipiebat eis, dicens~: Videte, et cavete a fermento pharis\ae orum, et fermento Herodis.
${}^{16}$~Et cogitabant ad alterutrum, dicentes~: quia panes non habemus.
${}^{17}$~Quo cognito, ait illis Jesus~: Quid cogitatis, quia panes non habetis~? nondum cognoscetis nec intelligitis~? adhuc c\ae catum habetis cor vestrum~?
${}^{18}$~oculos habentes non videtis~? et aures habentes non auditis~? nec recordamini,
${}^{19}$~quando quinque panes fregi in quinque millia~: quot cophinos fragmentorum plenos sustulistis~? Dicunt ei~: Duodecim.
${}^{20}$~Quando et septem panes in quatuor millia~: quot sportas fragmentorum tulistis~? Et dicunt ei~: Septem.
${}^{21}$~Et dicebat eis~: Quomodo nondum intelligitis~?


${}^{22}$~Et veniunt Bethsaidam, et adducunt ei c\ae cum, et rogabant eum ut illum tangeret.
${}^{23}$~Et apprehensa manu c\ae ci, eduxit eum extra vicum~: et exspuens in oculos ejus impositis manibus suis, interrogavit eum si quid videret.
${}^{24}$~Et aspiciens, ait~: Video homines velut arbores ambulantes.
${}^{25}$~Deinde iterum imposuit manus super oculos ejus~: et cœpit videre~: et restitutus est ita ut clare videret omnia.
${}^{26}$~Et misit illum in domum suam, dicens~: Vade in domum tuam~: et si in vicum introieris, nemini dixeris.


${}^{27}$~Et egressus est Jesus, et discipuli ejus in castella C\ae sare\ae\ Philippi~: et in via interrogabat discipulos suos, dicens eis~: Quem me dicunt esse homines~?
${}^{28}$~Qui responderunt illi, dicentes~: Joannem Baptistam, alii Eliam, alii vero quasi unum de prophetis.
${}^{29}$~Tunc dicit illis~: Vos vero quem me esse dicitis~? Respondens Petrus, ait ei~: Tu es Christus.
${}^{30}$~Et comminatus est eis, ne cui dicerent de illo.


${}^{31}$~Et cœpit docere eos quoniam oportet Filium hominis pati multa, et reprobari a senioribus, et a summis sacerdotibus et scribis, et occidi~: et post tres dies resurgere.
${}^{32}$~Et palam verbum loquebatur. Et apprehendens eum Petrus, cœpit increpare eum.
${}^{33}$~Qui conversus, et videns discipulos suos, comminatus est Petro, dicens~: Vade retro me Satana, quoniam non sapis qu\ae\ Dei sunt, sed qu\ae\ sunt hominum.


${}^{34}$~Et convocata turba cum discipulis suis, dixit eis~: Si quis vult me sequi, deneget semetipsum~: et tollat crucem suam, et sequatur me.
${}^{35}$~Qui enim voluerit animam suam salvam facere, perdet eam~: qui autem perdiderit animam suam propter me, et Evangelium, salvam faciet eam.
${}^{36}$~Quid enim proderit homini, si lucretur mundum totum et detrimentum anim\ae\ su\ae\ faciat~?
${}^{37}$~Aut quid dabit homo commutationis pro anima sua~?
${}^{38}$~Qui enim me confusus fuerit, et verba mea in generatione ista adultera et peccatrice, et Filius hominis confundetur eum, cum venerit in gloria Patris sui cum angelis sanctis.
${}^{39}$~Et dicebat illis~: Amen dico vobis, quia sunt quidam de hic stantibus, qui non gustabunt mortem donec videant regnum Dei veniens in virtute.

\bchapter
\mylettrine{E}t post dies sex assumit Jesus Petrum, et Jacobum, et Joannem, et ducit illos in montem excelsum seorsum solos, et transfiguratus est coram ipsis.
${}^{2}$~Et vestimenta ejus facta sunt splendentia, et candida nimis velut nix, qualia fullo non potest super terram candida facere.
${}^{3}$~Et apparuit illis Elias cum Moyse~: et erant loquentes cum Jesu.
${}^{4}$~Et respondens Petrus, ait Jesu~: Rabbi, bonum est nos hic esse~: et faciamus tria tabernacula, tibi unum, et Moysi unum, et Eli\ae\ unum.
${}^{5}$~Non enim sciebat quid diceret~: erant enim timore exterriti.
${}^{6}$~Et facta est nubes obumbrans eos~: et venit vox de nube, dicens~: Hic est Filius meus carissimus~: audite illum.
${}^{7}$~Et statim circumspicientes, neminem amplius viderunt, nisi Jesum tantum secum.
${}^{8}$~Et descendentibus illis de monte, pr\ae cepit illis ne cuiquam qu\ae\ vidissent, narrarent~: nisi cum Filius hominis a mortuis resurrexerit.
${}^{9}$~Et verbum continuerunt apud se~: conquirentes quid esset, cum a mortuis resurrexerit.
${}^{10}$~Et interrogabant eum, dicentes~: Quid ergo dicunt pharis\ae i et scrib\ae , quia Eliam oportet venire primum~?
${}^{11}$~Qui respondens, ait illis~: Elias cum venerit primo, restituet omnia~: et quomodo scriptum est in Filium hominis, ut multa patiatur et contemnatur.
${}^{12}$~Sed dico vobis quia et Elias venit (et fecerunt illi qu\ae cumque voluerunt) sicut scriptum est de eo.


${}^{13}$~Et veniens ad discipulos suos, vidit turbam magnam circa eos, et scribas conquirentes cum illis.
${}^{14}$~Et confestim omnis populus videns Jesum, stupefactus est, et expaverunt, et accurrentes salutabant eum.
${}^{15}$~Et interrogavit eos~: Quid inter vos conquiritis~?
${}^{16}$~Et respondens unus de turba, dixit~: Magister, attuli filium meum ad te habentem spiritum mutum~:
${}^{17}$~qui ubicumque eum apprehenderit, allidit illum, et spumat, et stridet dentibus, et arescit~: et dixi discipulis tuis ut ejicerent illum, et non potuerunt.
${}^{18}$~Qui respondens eis, dixit~: O generatio incredula, quamdiu apud vos ero~? quamdiu vos patiar~? afferte illum ad me.
${}^{19}$~Et attulerunt eum. Et cum vidisset eum, statim spiritus conturbavit illum~: et elisus in terram, volutabatur spumans.
${}^{20}$~Et interrogavit patrem ejus~: Quantum temporis est ex quo ei hoc accidit~? At ille ait~: Ab infantia~:
${}^{21}$~et frequenter eum in ignem, et in aquas misit ut eum perderet~: sed si quid potes, adjuva nos, misertus nostri.
${}^{22}$~Jesus autem ait illi~: Si potes credere, omnia possibilia sunt credenti.
${}^{23}$~Et continuo exclamans pater pueri, cum lacrimis aiebat~: Credo, Domine~; adjuva incredulitatem meam.
${}^{24}$~Et cum videret Jesus concurrentem turbam, comminatus est spiritui immundo, dicens illi~: Surde et mute spiritus, ego pr\ae cipio tibi, exi ab eo~: et amplius ne intro\"eas in eum.
${}^{25}$~Et exclamans, et multum discerpens eum, exiit ab eo, et factus est sicut mortuus, ita ut multi dicerent~: Quia mortuus est.
${}^{26}$~Jesus autem tenens manum ejus elevavit eum, et surrexit.
${}^{27}$~Et cum introisset in domum, discipuli ejus secreto interrogabant eum~: Quare nos non potuimus ejicere eum~?
${}^{28}$~Et dixit illis~: Hoc genus in nullo potest exire, nisi in oratione et jejunio.
${}^{29}$~Et inde profecti pr\ae tergrediebantur Galil\ae am~: nec volebat quemquam scire.


${}^{30}$~Docebat autem discipulos suos, et dicebat illis~: Quoniam Filius hominis tradetur in manus hominum, et occident eum, et occisus tertia die resurget.
${}^{31}$~At illi ignorabant verbum~: et timebant interrogare eum.


${}^{32}$~Et venerunt Capharnaum. Qui cum domi essent, interrogabat eos~: Quid in via tractabatis~?
${}^{33}$~At illi tacebant~: siquidem in via inter se disputaverunt~: quis eorum major esset.
${}^{34}$~Et residens vocavit duodecim, et ait illis~: Si quis vult primus esse, erit omnium novissimus, et omnium minister.
${}^{35}$~Et accipiens puerum, statuit eum in medio eorum~: quem cum complexus esset, ait illis~:
${}^{36}$~Quisquis unum ex hujusmodi pueris receperit in nomine meo, me recipit~: et quicumque me susceperit, non me suscipit, sed eum qui misit me.


${}^{37}$~Respondit illi Joannes, dicens~: Magister, vidimus quemdam in nomine tuo ejicientem d\ae monia, qui non sequitur nos, et prohibuimus eum.
${}^{38}$~Jesus autem ait~: Nolite prohibere eum~: nemo est enim qui faciat virtutem in nomine meo, et possit cito male loqui de me~:
${}^{39}$~qui enim non est adversum vos, pro vobis est.
${}^{40}$~Quisquis enim potum dederit vobis calicem aqu\ae\ in nomine meo, quia Christi estis~: amen dico vobis, non perdet mercedem suam.


${}^{41}$~Et quisquis scandalizaverit unum ex his pusillis credentibus in me~: bonum est ei magis si circumdaretur mola asinaria collo ejus, et in mare mitteretur.
${}^{42}$~Et si scandalizaverit te manus tua, abscide illam~: bonum est tibi debilem introire in vitam, quam duas manus habentem ire in gehennam, in ignem inextinguibilem,
${}^{43}$~ubi vermis eorum non moritur, et ignis non extinguitur.
${}^{44}$~Et si pes tuus te scandalizat, amputa illum~: bonum est tibi claudum introire in vitam \ae ternam, quam duos pedes habentem mitti in gehennam ignis inextinguibilis,
${}^{45}$~ubi vermis eorum non moritur, et ignis non extinguitur.
${}^{46}$~Quod si oculus tuus scandalizat te, ejice eum~: bonum est tibi luscum introire in regnum Dei, quam duos oculos habentem mitti in gehennam ignis,
${}^{47}$~ubi vermis eorum non moritur, et ignis non extinguitur.
${}^{48}$~Omnis enim igne salietur, et omnis victima sale salietur.
${}^{49}$~Bonum est sal~: quod si sal insulsum fuerit, in quo illud condietis~? Habete in vobis sal, et pacem habete inter vos.

\bchapter
\mylettrine{E}t inde exsurgens venit in fines Jud\ae \ae\ ultra Jordanem~: et conveniunt iterum turb\ae\ ad eum~: et sicut consueverat, iterum docebat illos.
${}^{2}$~Et accedentes pharis\ae i interrogabant eum~: Si licet viro uxorem dimittere~: tentantes eum.
${}^{3}$~At ille respondens, dixit eis~: Quid vobis pr\ae cepit Moyses~?
${}^{4}$~Qui dixerunt~: Moyses permisit libellum repudii scribere, et dimittere.
${}^{5}$~Quibus respondens Jesus, ait~: Ad duritiam cordis vestri scripsit vobis pr\ae ceptum istud~:
${}^{6}$~ab initio autem creatur\ae\ masculum et feminam fecit eos Deus.
${}^{7}$~Propter hoc relinquet homo patrem suum et matrem, et adh\ae rebit ad uxorem suam~:
${}^{8}$~et erunt duo in carne una. Itaque jam non sunt duo, sed una caro.
${}^{9}$~Quod ergo Deus conjunxit, homo non separet.
${}^{10}$~Et in domo iterum discipuli ejus de eodem interrogaverunt eum.
${}^{11}$~Et ait illis~: Quicumque dimiserit uxorem suam, et aliam duxerit, adulterium committit super eam.
${}^{12}$~Et si uxor dimiserit virum suum, et alii nupserit, mœchatur.


${}^{13}$~Et offerebant illi parvulos ut tangeret illos. Discipuli autem comminabantur offerentibus.
${}^{14}$~Quos cum videret Jesus, indigne tulit, et ait illis~: Sinite parvulos venire ad me, et ne prohibueritis eos~: talium enim est regnum Dei.
${}^{15}$~Amen dico vobis~: Quisquis non receperit regnum Dei velut parvulus, non intrabit in illud.
${}^{16}$~Et complexans eos, et imponens manus super illos, benedicebat eos.


${}^{17}$~Et cum egressus esset in viam, procurrens quidam genu flexo ante eum, rogabat eum~: Magister bone, quid faciam ut vitam \ae ternam percipiam~?
${}^{18}$~Jesus autem dixit ei~: Quid me dicis bonum~? nemo bonus, nisi unus Deus.
${}^{19}$~Pr\ae cepta nosti~: ne adulteres, ne occidas, ne fureris, ne falsum testimonium dixeris, ne fraudem feceris, honora patrem tuum et matrem.
${}^{20}$~At ille respondens, ait illi~: Magister, h\ae c omnia observavi a juventute mea.
${}^{21}$~Jesus autem intuitus eum, dilexit eum, et dixit ei~: Unum tibi deest~: vade, qu\ae cumque habes vende, et da pauperibus, et habebis thesaurum in c\ae lo~: et veni, sequere me.
${}^{22}$~Qui contristatus in verbo, abiit mœrens~: erat enim habens multas possessiones.
${}^{23}$~Et circumspiciens Jesus, ait discipulis suis~: Quam difficile qui pecunias habent, in regnum Dei introibunt~!
${}^{24}$~Discipuli autem obstupescebant in verbis ejus. At Jesus rursus respondens ait illis~: Filioli, quam difficile est, confidentes in pecuniis, in regnum Dei introire~!
${}^{25}$~Facilius est camelum per foramen acus transire, quam divitem intrare in regnum Dei.
${}^{26}$~Qui magis admirabantur, dicentes ad semetipsos~: Et quis potest salvus fieri~?
${}^{27}$~Et intuens illos Jesus, ait~: Apud homines impossibile est, sed non apud Deum~: omnia enim possibilia sunt apud Deum.


${}^{28}$~Et cœpit ei Petrus dicere~: Ecce nos dimisimus omnia, et secuti sumus te.
${}^{29}$~Respondens Jesus, ait~: Amen dico vobis~: Nemo est qui reliquerit domum, aut fratres, aut sorores, aut patrem, aut matrem, aut filios, aut agros propter me et propter Evangelium,
${}^{30}$~qui non accipiat centies tantum, nunc in tempore hoc~: domos, et fratres, et sorores, et matres, et filios, et agros, cum persecutionibus, et in s\ae culo futuro vitam \ae ternam.
${}^{31}$~Multi autem erunt primi novissimi, et novissimi primi.


${}^{32}$~Erant autem in via ascendentes Jerosolymam~: et pr\ae cedebat illos Jesus, et stupebant~: et sequentes timebant. Et assumens iterum duodecim, cœpit illis dicere qu\ae\ essent ei eventura.
${}^{33}$~Quia ecce ascendimus Jerosolymam, et Filius hominis tradetur principibus sacerdotum, et scribis, et senioribus, et damnabunt eum morte, et tradent eum gentibus~:
${}^{34}$~et illudent ei, et conspuent eum, et flagellabunt eum, et interficient eum~: et tertia die resurget.


${}^{35}$~Et accedunt ad eum Jacobus et Joannes filii Zebed\ae i, dicentes~: Magister, volumus ut quodcumque petierimus, facias nobis.
${}^{36}$~At ille dixit eis~: Quid vultis ut faciam vobis~?
${}^{37}$~Et dixerunt~: Da nobis ut unus ad dexteram tuam, et alius ad sinistram tuam sedeamus in gloria tua.
${}^{38}$~Jesus autem ait eis~: Nescitis quid petatis~: potestis bibere calicem, quem ego bibo, aut baptismo, quo ego baptizor, baptizari~?
${}^{39}$~At illi dixerunt ei~: Possumus. Jesus autem ait eis~: Calicem quidem, quem ego bibo, bibetis~; et baptismo, quo ego baptizor, baptizabimini~:
${}^{40}$~sedere autem ad dexteram meam, vel ad sinistram, non est meum dare vobis, sed quibus paratum est.
${}^{41}$~Et audientes decem, cœperunt indignari de Jacobo et Joanne.
${}^{42}$~Jesus autem vocans eos, ait illis~: Scitis quia hi, qui videntur principari gentibus, dominantur eis~: et principes eorum potestatem habent ipsorum.
${}^{43}$~Non ita est autem in vobis, sed quicumque voluerit fieri major, erit vester minister~:
${}^{44}$~et quicumque voluerit in vobis primus esse, erit omnium servus.
${}^{45}$~Nam et Filius hominis non venit ut ministraretur ei, sed ut ministraret, et daret animam suam redemptionem pro multis.


${}^{46}$~Et veniunt Jericho~: et proficiscente eo de Jericho, et discipulis ejus, et plurima multitudine, filius Tim\ae i Bartim\ae us c\ae cus, sedebat juxta viam mendicans.
${}^{47}$~Qui cum audisset quia Jesus Nazarenus est, cœpit clamare, et dicere~: Jesu fili David, miserere mei.
${}^{48}$~Et comminabantur ei multi ut taceret. At ille multo magis clamabat~: Fili David, miserere mei.
${}^{49}$~Et stans Jesus pr\ae cepit illum vocari. Et vocant c\ae cum, dicentes ei~: Anim\ae quior esto~: surge, vocat te.
${}^{50}$~Qui projecto vestimento suo exiliens, venit ad eum.
${}^{51}$~Et respondens Jesus dixit illi~: Quid tibi vis faciam~? C\ae cus autem dixit ei~: Rabboni, ut videam.
${}^{52}$~Jesus autem ait illi~: Vade, fides tua te salvum fecit. Et confestim vidit, et sequebatur eum in via.

\bchapter
\mylettrine{E}t cum appropinquarent Jerosolym\ae\ et Bethani\ae\ ad montem Olivarum, mittit duos ex discipulis suis,
${}^{2}$~et ait illis~: Ite in castellum, quod contra vos est, et statim intro\"euntes illuc, invenietis pullum ligatum, super quem nemo adhuc hominum sedit~: solvite illum, et adducite.
${}^{3}$~Et si quis vobis dixerit~: Quid facitis~? dicite, quia Domino necessarius est~: et continuo illum dimittet huc.
${}^{4}$~Et abeuntes invenerunt pullum ligatum ante januam foris in bivio~: et solvunt eum.
${}^{5}$~Et quidam de illic stantibus dicebant illis~: Quid facitis solventes pullum~?
${}^{6}$~Qui dixerunt eis sicut pr\ae ceperat illis Jesus, et dimiserunt eis.
${}^{7}$~Et duxerunt pullum ad Jesum~: et imponunt illi vestimenta sua, et sedit super eum.
${}^{8}$~Multi autem vestimenta sua straverunt in via~: alii autem frondes c\ae debant de arboribus, et sternebant in via.
${}^{9}$~Et qui pr\ae ibant, et qui sequebantur, clamabant, dicentes~: Hosanna~: benedictus qui venit in nomine Domini~:
${}^{10}$~benedictum quod venit regnum patris nostri David~: hosanna in excelsis.
${}^{11}$~Et introivit Jerosolymam in templum~: et circumspectis omnibus, cum jam vespera esset hora, exiit in Bethaniam cum duodecim.


${}^{12}$~Et alia die cum exirent a Bethania, esuriit.
${}^{13}$~Cumque vidisset a longe ficum habentem folia, venit si quid forte inveniret in ea~: et cum venisset ad eam, nihil invenit pr\ae ter folia~: non enim erat tempus ficorum.
${}^{14}$~Et respondens dixit ei~: Jam non amplius in \ae ternum ex te fructum quisquam manducet. Et audiebant discipuli ejus.


${}^{15}$~Et veniunt in Jerosolymam. Et cum introisset in templum, cœpit ejicere vendentes et ementes in templo~: et mensas numulariorum, et cathedras vendentium columbas evertit~:
${}^{16}$~et non sinebat ut quisquam transferret vas per templum~:
${}^{17}$~et docebat, dicens eis~: Nonne scriptum est~: Quia domus mea, domus orationis vocabitur omnibus gentibus~? vos autem fecistis eam speluncam latronum.
${}^{18}$~Quo audito principes sacerdotum et scrib\ae , qu\ae rebant quomodo eum perderent~: timebant enim eum, quoniam universa turba admirabatur super doctrina ejus.
${}^{19}$~Et cum vespera facta esset, egrediebatur de civitate.


${}^{20}$~Et cum mane transirent, viderunt ficum aridam factam a radicibus.
${}^{21}$~Et recordatus Petrus, dixit ei~: Rabbi, ecce ficus, cui maledixisti, aruit.
${}^{22}$~Et respondens Jesus ait illis~: Habete fidem Dei.
${}^{23}$~Amen dico vobis, quia quicumque dixerit huic monti~: Tollere, et mittere in mare, et non h\ae sitaverit in corde suo, sed crediderit, quia quodcumque dixerit fiat, fiet ei.
${}^{24}$~Propterea dico vobis, omnia qu\ae cumque orantes petitis, credite quia accipietis, et evenient vobis.
${}^{25}$~Et cum stabitis ad orandum, dimittite si quid habetis adversus aliquem~: ut et Pater vester, qui in c\ae lis est, dimittat vobis peccata vestra.
${}^{26}$~Quod si vos non dimiseritis~: nec Pater vester, qui in c\ae lis est, dimittet vobis peccata vestra.


${}^{27}$~Et veniunt rursus Jerosolymam. Et cum ambularet in templo, accedunt ad eum summi sacerdotes, et scrib\ae , et seniores~:
${}^{28}$~et dicunt ei~: In qua potestate h\ae c facis~? et quis dedit tibi hanc potestatem ut ista facias~?
${}^{29}$~Jesus autem respondens, ait illis~: Interrogabo vos et ego unum verbum, et respondete mihi~: et dicam vobis in qua potestate h\ae c faciam.
${}^{30}$~Baptismus Joannis, de c\ae lo erat, an ex hominibus~? Respondete mihi.
${}^{31}$~At illi cogitabant secum, dicentes~: Si dixerimus~: De c\ae lo, dicet~: Quare ergo non credidistis ei~?
${}^{32}$~Si dixerimus~: Ex hominibus, timemus populum~: omnes enim habebant Joannem quia vere propheta esset.
${}^{33}$~Et respondentes dicunt Jesu~: Nescimus. Et respondens Jesus ait illis~: Neque ego dico vobis in qua potestate h\ae c faciam.

\bchapter
\mylettrine{E}t cœpit illis in parabolis loqui~: Vineam pastinavit homo, et circumdedit sepem, et fodit lacum, et \ae dificavit turrim, et locavit eam agricolis, et peregre profectus est.
${}^{2}$~Et misit ad agricolas in tempore servum ut ab agricolis acciperet de fructu vine\ae .
${}^{3}$~Qui apprehensum eum ceciderunt, et dimiserunt vacuum.
${}^{4}$~Et iterum misit ad illos alium servum~: et illum in capite vulneraverunt, et contumeliis affecerunt.
${}^{5}$~Et rursum alium misit, et illum occiderunt~: et plures alios~: quosdam c\ae dentes, alios vero occidentes.
${}^{6}$~Adhuc ergo unum habens filium carissimum, et illum misit ad eos novissimum, dicens~: Quia reverebuntur filium meum.
${}^{7}$~Coloni autem dixerunt ad invicem~: Hic est h\ae res~: venite, occidamus eum~: et nostra erit h\ae reditas.
${}^{8}$~Et apprehendentes eum, occiderunt~: et ejecerunt extra vineam.
${}^{9}$~Quid ergo faciet dominus vine\ae~? Veniet, et perdet colonos, et dabit vineam aliis.
${}^{10}$~Nec scripturam hanc legistis~: Lapidem quem reprobaverunt \ae dificantes, hic factus est in caput anguli~:
${}^{11}$~a Domino factum est istud, et est mirabile in oculis nostris~?
${}^{12}$~Et qu\ae rebant eum tenere~: et timuerunt turbam~: cognoverunt enim quoniam ad eos parabolam hanc dixerit. Et relicto eo abierunt.


${}^{13}$~Et mittunt ad eum quosdam ex pharis\ae is, et herodianis, ut eum caperent in verbo.
${}^{14}$~Qui venientes dicunt ei~: Magister, scimus quia verax es, et non curas quemquam~: nec enim vides in faciem hominum, sed in veritate viam Dei doces. Licet dari tributum C\ae sari, an non dabimus~?
${}^{15}$~Qui sciens versutiam illorum, ait illis~: Quid me tentatis~? afferte mihi denarium ut videam.
${}^{16}$~At illi attulerunt ei. Et ait illis~: Cujus est imago h\ae c, et inscriptio~? Dicunt ei~: C\ae saris.
${}^{17}$~Respondens autem Jesus dixit illis~: Reddite igitur qu\ae\ sunt C\ae saris, C\ae sari~: et qu\ae\ sunt Dei, Deo. Et mirabantur super eo.


${}^{18}$~Et venerunt ad eum sadduc\ae i, qui dicunt resurrectionem non esse~: et interrogabant eum, dicentes~:
${}^{19}$~Magister, Moyses nobis scripsit, ut si cujus frater mortuus fuerit, et dimiserit uxorem, et filios non reliquerit, accipiat frater ejus uxorem ipsius, et resuscitet semen fratri suo.
${}^{20}$~Septem ergo fratres erant~: et primus accepit uxorem, et mortuus est non relicto semine.
${}^{21}$~Et secundus accepit eam, et mortuus est~: et nec iste reliquit semen. Et tertius similiter.
${}^{22}$~Et acceperunt eam similiter septem~: et non reliquerunt semen. Novissima omnium defuncta est et mulier.
${}^{23}$~In resurrectione ergo cum resurrexerint, cujus de his erit uxor~? septem enim habuerunt eam uxorem.
${}^{24}$~Et respondens Jesus, ait illis~: Nonne ideo erratis, non scientes Scripturas, neque virtutem Dei~?
${}^{25}$~Cum enim a mortuis resurrexerint, neque nubent, neque nubentur, sed sunt sicut angeli in c\ae lis.
${}^{26}$~De mortuis autem quod resurgant, non legistis in libro Moysi, super rubum, quomodo dixerit illi Deus, inquiens~: Ego sum Deus Abraham, et Deus Isaac, et Deus Jacob~?
${}^{27}$~Non est Deus mortuorum, sed vivorum. Vos ergo multum erratis.


${}^{28}$~Et accessit unus de scribis, qui audierat illos conquirentes, et videns quoniam bene illis responderit, interrogavit eum quod esset primum omnium mandatum.
${}^{29}$~Jesus autem respondit ei~: Quia primum omnium mandatum est~: Audi Isra\"el, Dominus Deus tuus, Deus unus est~:
${}^{30}$~et diliges Dominum Deum tuum ex toto corde tuo, et ex tota anima tua, et ex tota mente tua, et ex tota virtute tua. Hoc est primum mandatum.
${}^{31}$~Secundum autem simile est illi~: Diliges proximum tuum tamquam teipsum. Majus horum aliud mandatum non est.
${}^{32}$~Et ait illi scriba~: Bene, Magister, in veritate dixisti, quia unus est Deus, et non est alius pr\ae ter eum.
${}^{33}$~Et ut diligatur ex toto corde, et ex toto intellectu, et ex tota anima, et ex tota fortitudine, et diligere proximum tamquam seipsum, majus est omnibus holocautomatibus, et sacrificiis.
${}^{34}$~Jesus autem videns quod sapienter respondisset, dixit illi~: Non es longe a regno Dei. Et nemo jam audebat eum interrogare.


${}^{35}$~Et respondens Jesus dicebat, docens in templo~: Quomodo dicunt scrib\ae\ Christum filium esse David~?
${}^{36}$~Ipse enim David dicit in Spiritu Sancto~: Dixit Dominus Domino meo~: Sede a dextris meis, donec ponam inimicos tuos scabellum pedum tuorum.
${}^{37}$~Ipse ergo David dicit eum Dominum, et unde est filius ejus~? Et multa turba eum libenter audivit.
${}^{38}$~Et dicebat eis in doctrina sua~: Cavete a scribis, qui volunt in stolis ambulare, et salutari in foro,
${}^{39}$~et in primis cathedris sedere in synagogis, et primos discubitus in cœnis~:
${}^{40}$~qui devorant domos viduarum sub obtentu prolix\ae\ orationis~: hi accipient prolixius judicium.


${}^{41}$~Et sedens Jesus contra gazophylacium, aspiciebat quomodo turba jactaret \ae s in gazophylacium, et multi divites jactabant multa.
${}^{42}$~Cum venisset autem vidua una pauper, misit duo minuta, quod est quadrans,
${}^{43}$~et convocans discipulos suos, ait illis~: Amen dico vobis, quoniam vidua h\ae c pauper plus omnibus misit, qui miserunt in gazophylacium.
${}^{44}$~Omnes enim ex eo, quod abundabat illis, miserunt~: h\ae c vero de penuria sua omnia qu\ae\ habuit misit totum victum suum.

\bchapter
\mylettrine{E}t cum egrederetur de templo, ait illi unus ex discipulis suis~: Magister, aspice quales lapides, et quales structur\ae .
${}^{2}$~Et respondens Jesus, ait illi~: Vides has omnes magnas \ae dificationes~? Non relinquetur lapis super lapidem, qui non destruatur.


${}^{3}$~Et cum sederet in monte Olivarum contra templum, interrogabant eum separatim Petrus, et Jacobus, et Joannes, et Andreas~:
${}^{4}$~Dic nobis, quando ista fient~? et quod signum erit, quando h\ae c omnia incipient consummari~?
${}^{5}$~Et respondens Jesus cœpit dicere illis~: Videte ne quis vos seducat~:
${}^{6}$~multi enim venient in nomine meo, dicentes quia ego sum~: et multos seducent.
${}^{7}$~Cum audieritis autem bella, et opiniones bellorum, ne timueritis~: oportet enim h\ae c fieri~: sed nondum finis.
${}^{8}$~Exsurget enim gens contra gentem, et regnum super regnum, et erunt terr\ae motus per loca, et fames. Initium dolorum h\ae c.


${}^{9}$~Videte autem vosmetipsos. Tradent enim vos in consiliis, et in synagogis vapulabitis, et ante pr\ae sides et reges stabitis propter me, in testimonium illis.
${}^{10}$~Et in omnes gentes primum oportet pr\ae dicari Evangelium.
${}^{11}$~Et cum duxerint vos tradentes, nolite pr\ae cogitare quid loquamini~: sed quod datum vobis fuerit in illa hora, id loquimini~: non enim vos estis loquentes, sed Spiritus Sanctus.
${}^{12}$~Tradet autem frater fratrem in mortem, et pater filium~: et consurgent filii in parentes, et morte afficient eos.
${}^{13}$~Et eritis odio omnibus propter nomen meum. Qui autem sustinuerit in finem, hic salvus erit.


${}^{14}$~Cum autem videritis abominationem desolationis stantem, ubi non debet, qui legit, intelligat~: tunc qui in Jud\ae a sunt, fugiant in montes~:
${}^{15}$~et qui super tectum, ne descendat in domum, nec intro\"eat ut tollat quid de domo sua~:
${}^{16}$~et qui in agro erit, non revertatur retro tollere vestimentum suum.
${}^{17}$~V\ae\ autem pr\ae gnantibus et nutrientibus in illis diebus.


${}^{18}$~Orate vero ut hieme non fiant.
${}^{19}$~Erunt enim dies illi tribulationes tales quales non fuerunt ab initio creatur\ae , quam condidit Deus usque nunc, neque fient.
${}^{20}$~Et nisi breviasset Dominus dies, non fuisset salva omnis caro~: sed propter electos, quos elegit, breviavit dies.
${}^{21}$~Et tunc si quis vobis dixerit~: Ecce hic est Christus, ecce illic, ne credideritis.
${}^{22}$~Exsurgent enim pseudochristi et pseudoprophet\ae , et dabunt signa et portenta ad seducendos, si fieri potest, etiam electos.
${}^{23}$~Vos ergo videte~: ecce pr\ae dixi vobis omnia.


${}^{24}$~Sed in illis diebus, post tribulationem illam, sol contenebrabitur, et luna non dabit splendorem suum~:
${}^{25}$~et stell\ae\ c\ae li erunt decidentes, et virtutes, qu\ae\ in c\ae lis sunt, movebuntur.
${}^{26}$~Et tunc videbunt Filium hominis venientem in nubibus cum virtute multa et gloria.
${}^{27}$~Et tunc mittet angelos suos, et congregabit electos suos a quatuor ventis, a summo terr\ae\ usque ad summum c\ae li.


${}^{28}$~A ficu autem discite parabolam. Cum jam ramus ejus tener fuerit, et nata fuerint folia, cognoscitis quia in proximo sit \ae stas~:
${}^{29}$~sic et vos cum videritis h\ae c fieri, scitote quod in proximo sit, in ostiis.
${}^{30}$~Amen dico vobis, quoniam non transibit generatio h\ae c, donec omnia ista fiant.
${}^{31}$~C\ae lum et terra transibunt, verba autem mea non transibunt.


${}^{32}$~De die autem illo vel hora nemo scit, neque angeli in c\ae lo, neque Filius, nisi Pater.
${}^{33}$~Videte, vigilate, et orate~: nescitis enim quando tempus sit.
${}^{34}$~Sicut homo qui peregre profectus reliquit domum suam, et dedit servis suis potestatem cujusque operis, et janitori pr\ae cepit ut vigilet,
${}^{35}$~vigilate ergo (nescitis enim quando dominus domus veniat~: sero, an media nocte, an galli cantu, an mane),
${}^{36}$~ne, cum venerit repente, inveniat vos dormientes.
${}^{37}$~Quod autem vobis dico, omnibus dico~: Vigilate.

\bchapter
\mylettrine{E}rat autem Pascha et azyma post biduum~: et qu\ae rebant summi sacerdotes et scrib\ae\ quomodo eum dolo tenerent, et occiderent.
${}^{2}$~Dicebant autem~: Non in die festo, ne forte tumultus fieret in populo.


${}^{3}$~Et cum esset Bethani\ae\ in domo Simonis leprosi, et recumberet, venit mulier habens alabastrum unguenti nardi spicati pretiosi~: et fracto alabastro, effudit super caput ejus.
${}^{4}$~Erant autem quidam indigne ferentes intra semetipsos, et dicentes~: Ut quid perditio ista unguenti facta est~?
${}^{5}$~poterat enim unguentum istud venundari plus quam trecentis denariis, et dari pauperibus. Et fremebant in eam.
${}^{6}$~Jesus autem dixit~: Sinite eam, quid illi molesti estis~? Bonum opus operata est in me~:
${}^{7}$~semper enim pauperes habetis vobiscum~: et cum volueritis, potestis illis benefacere~: me autem non semper habetis.
${}^{8}$~Quod habuit h\ae c, fecit~: pr\ae venit ungere corpus meum in sepulturam.
${}^{9}$~Amen dico vobis~: Ubicumque pr\ae dicatum fuerit Evangelium istud in universo mundo, et quod fecit h\ae c, narrabitur in memoriam ejus.


${}^{10}$~Et Judas Iscariotes, unus de duodecim, abiit ad summos sacerdotes, ut proderet eum illis.
${}^{11}$~Qui audientes gavisi sunt~: et promiserunt ei pecuniam se daturos. Et qu\ae rebat quomodo illum opportune traderet.


${}^{12}$~Et primo die azymorum quando Pascha immolabant, dicunt ei discipuli~: Quo vis eamus, et paremus tibi ut manduces Pascha~?
${}^{13}$~Et mittit duos ex discipulis suis, et dicit eis~: Ite in civitatem, et occurret vobis homo lagenam aqu\ae\ bajulans~: sequimini eum,
${}^{14}$~et quocumque introierit, dicite domino domus, quia magister dicit~: Ubi est refectio mea, ubi Pascha cum discipulis meis manducem~?
${}^{15}$~Et ipse vobis demonstrabit cœnaculum grande, stratum~: et illic parate nobis.
${}^{16}$~Et abierunt discipuli ejus, et venerunt in civitatem~: et invenerunt sicut dixerat illis, et paraverunt Pascha.
${}^{17}$~Vespere autem facto, venit cum duodecim.
${}^{18}$~Et discumbentibus eis, et manducantibus, ait Jesus~: Amen dico vobis, quia unus ex vobis tradet me, qui manducat mecum.
${}^{19}$~At illi cœperunt contristari, et dicere ei singulatim~: Numquid ego~?
${}^{20}$~Qui ait illis~: Unus ex duodecim, qui intingit mecum manum in catino.
${}^{21}$~Et Filius quidem hominis vadit sicut scriptum est de eo~: v\ae\ autem homini illi per quem Filius hominis tradetur~! bonum erat ei, si non esset natus homo ille.


${}^{22}$~Et manducantibus illis, accepit Jesus panem~: et benedicens fregit, et dedit eis, et ait~: Sumite, hoc est corpus meum.
${}^{23}$~Et accepto calice, gratias agens dedit eis~: et biberunt ex illo omnes.
${}^{24}$~Et ait illis~: Hic est sanguis meus novi testamenti, qui pro multis effundetur.
${}^{25}$~Amen dico vobis, quia jam non bibam de hoc genimine vitis usque in diem illum, cum illud bibam novum in regno Dei.


${}^{26}$~Et hymno dicto exierunt in montem Olivarum.
${}^{27}$~Et ait eis Jesus~: Omnes scandalizabimini in me in nocte ista~: quia scriptum est~: Percutiam pastorem, et dispergentur oves.
${}^{28}$~Sed postquam resurrexero, pr\ae cedam vos in Galil\ae am.
${}^{29}$~Petrus autem ait illi~: Et si omnes scandalizati fuerint in te, sed non ego.
${}^{30}$~Et ait illi Jesus~: Amen dico tibi, quia tu hodie in nocte hac, priusquam gallus vocem bis dederit, ter me es negaturus.
${}^{31}$~At ille amplius loquebatur~: Et si oportuerit me simul commori tibi, non te negabo. Similiter autem et omnes dicebant.


${}^{32}$~Et veniunt in pr\ae dium, cui nomen Gethsemani. Et ait discipulis suis~: Sedete hic donec orem.
${}^{33}$~Et assumit Petrum, et Jacobum, et Joannem secum~: et cœpit pavere et t\ae dere.
${}^{34}$~Et ait illis~: Tristis est anima mea usque ad mortem~: sustinete hic, et vigilate.
${}^{35}$~Et cum processisset paululum, procidit super terram, et orabat ut, si fieri posset, transiret ab eo hora.
${}^{36}$~Et dixit~: Abba pater, omnia tibi possibilia sunt~: transfer calicem hunc a me~: sed non quod ego volo, sed quod tu.
${}^{37}$~Et venit, et invenit eos dormientes. Et ait Petro~: Simon, dormis~? non potuisti una hora vigilare~?
${}^{38}$~vigilate et orate, ut non intretis in tentationem. Spiritus quidem promptus est, caro vero infirma.
${}^{39}$~Et iterum abiens oravit, eumdem sermonem dicens.
${}^{40}$~Et reversus, denuo invenit eos dormientes (erant enim oculi eorum gravati), et ignorabant quid responderent ei.
${}^{41}$~Et venit tertio, et ait illis~: Dormite jam, et requiescite. Sufficit~: venit hora~: ecce Filius hominis tradetur in manus peccatorum.
${}^{42}$~Surgite, eamus~: ecce qui me tradet, prope est.


${}^{43}$~Et, adhuc eo loquente, venit Judas Iscariotes unus de duodecim, et cum eo turba multa cum gladiis et lignis, a summis sacerdotibus, et scribis, et senioribus.
${}^{44}$~Dederat autem traditor ejus signum eis, dicens~: Quemcumque osculatus fuero, ipse est, tenete eum, et ducite caute.
${}^{45}$~Et cum venisset, statim accedens ad eum, ait~: Ave Rabbi~: et osculatus est eum.
${}^{46}$~At illi manus injecerunt in eum, et tenuerunt eum.
${}^{47}$~Unus autem quidam de circumstantibus educens gladium, percussit servum summi sacerdotis, et amputavit illi auriculam.
${}^{48}$~Et respondens Jesus, ait illis~: Tamquam ad latronem existis cum gladiis et lignis comprehendere me~?
${}^{49}$~quotidie eram apud vos in templo docens, et non me tenuistis. Sed ut impleantur Scriptur\ae .
${}^{50}$~Tunc discipuli ejus relinquentes eum, omnes fugerunt.
${}^{51}$~Adolescens autem quidam sequebatur eum amictus sindone super nudo~: et tenuerunt eum.
${}^{52}$~At ille rejecta sindone, nudus profugit ab eis.


${}^{53}$~Et adduxerunt Jesum ad summum sacerdotem~: et convenerunt omnes sacerdotes, et scrib\ae , et seniores.
${}^{54}$~Petrus autem a longe secutus est eum usque intro in atrium summi sacerdotis~: et sedebat cum ministris ad ignem, et calefaciebat se.
${}^{55}$~Summi vero sacerdotes et omne concilium qu\ae rebant adversus Jesum testimonium ut eum morti traderent~: nec inveniebant.
${}^{56}$~Multi enim testimonium falsum dicebant adversus eum~: et convenientia testimonia non erant.
${}^{57}$~Et quidam surgentes, falsum testimonium ferebant adversus eum, dicentes~:
${}^{58}$~Quoniam nos audivimus eum dicentem~: Ego dissolvam templum hoc manu factum, et per triduum aliud non manu factum \ae dificabo.
${}^{59}$~Et non erat conveniens testimonium illorum.
${}^{60}$~Et exsurgens summus sacerdos in medium, interrogavit Jesum, dicens~: Non respondes quidquam ad ea qu\ae\ tibi objiciuntur ab his~?
${}^{61}$~Ille autem tacebat, et nihil respondit. Rursum summus sacerdos interrogabat eum, et dixit ei~: Tu es Christus Filius Dei benedicti~?
${}^{62}$~Jesus autem dixit illi~: Ego sum~: et videbitis Filium hominis sedentem a dextris virtutis Dei, et venientem cum nubibus c\ae li.
${}^{63}$~Summus autem sacerdos scindens vestimenta sua, ait~: Quid adhuc desideramus testes~?
${}^{64}$~Audistis blasphemiam~: quid vobis videtur~? Qui omnes condemnaverunt eum esse reum mortis.
${}^{65}$~Et cœperunt quidam conspuere eum, et velare faciem ejus, et colaphis eum c\ae dere, et dicere ei~: Prophetiza~: et ministri alapis eum c\ae debant.


${}^{66}$~Et cum esset Petrus in atrio deorsum, venit una ex ancillis summi sacerdotis~:
${}^{67}$~et cum vidisset Petrum calefacientem se, aspiciens illum, ait~: Et tu cum Jesu Nazareno eras.
${}^{68}$~At ille negavit, dicens~: Neque scio, neque novi quid dicas. Et exiit foras ante atrium, et gallus cantavit.
${}^{69}$~Rursus autem cum vidisset illum ancilla, cœpit dicere circumstantibus~: Quia hic ex illis est.
${}^{70}$~At ille iterum negavit. Et post pusillum rursus qui astabant, dicebant Petro~: Vere ex illis es~: nam et Galil\ae us es.
${}^{71}$~Ille autem cœpit anathematizare et jurare~: Quia nescio hominem istum, quem dicitis.
${}^{72}$~Et statim gallus iterum cantavit. Et recordatus est Petrus verbi quod dixerat ei Jesus~: Priusquam gallus cantet bis, ter me negabis. Et cœpit flere.

\bchapter
\mylettrine{E}t confestim mane consilium facientes summi sacerdotes cum senioribus, et scribis, et universo concilio, vincientes Jesum, duxerunt, et tradiderunt Pilato.
${}^{2}$~Et interrogavit eum Pilatus~: Tu es rex Jud\ae orum~? At ille respondens, ait illi~: Tu dicis.
${}^{3}$~Et accusabant eum summi sacerdotes in multis.
${}^{4}$~Pilatus autem rursum interrogavit eum, dicens~: Non respondes quidquam~? vide in quantis te accusant.
${}^{5}$~Jesus autem amplius nihil respondit, ita ut miraretur Pilatus.
${}^{6}$~Per diem autem festum solebat dimittere illis unum ex vinctis, quemcumque petissent.
${}^{7}$~Erat autem qui dicebatur Barrabas, qui cum seditiosis erat vinctus, qui in seditione fecerat homicidium.
${}^{8}$~Et cum ascendisset turba, cœpit rogare, sicut semper faciebat illis.
${}^{9}$~Pilatus autem respondit eis, et dixit~: Vultis dimittam vobis regem Jud\ae orum~?
${}^{10}$~Sciebat enim quod per invidiam tradidissent eum summi sacerdotes.
${}^{11}$~Pontifices autem concitaverunt turbam, ut magis Barabbam dimitteret eis.
${}^{12}$~Pilatus autem iterum respondens, ait illis~: Quid ergo vultis faciam regi Jud\ae orum~?
${}^{13}$~At illi iterum clamaverunt~: Crucifige eum.
${}^{14}$~Pilatus vero dicebat illis~: Quid enim mali fecit~? At illi magis clamabant~: Crucifige eum.
${}^{15}$~Pilatus autem volens populo satisfacere, dimisit illis Barabbam, et tradidit Jesum flagellis c\ae sum, ut crucifigeretur.


${}^{16}$~Milites autem duxerunt eum in atrium pr\ae torii, et convocant totam cohortem,
${}^{17}$~et induunt eum purpura, et imponunt ei plectentes spineam coronam.
${}^{18}$~Et cœperunt salutare eum~: Ave rex Jud\ae orum.
${}^{19}$~Et percutiebant caput ejus arundine~: et conspuebant eum, et ponentes genua, adorabant eum.
${}^{20}$~Et postquam illuserunt ei, exuerunt illum purpura, et induerunt eum vestimentis suis~: et educunt illum ut crucifigerent eum.


${}^{21}$~Et angariaverunt pr\ae tereuntem quempiam, Simonem Cyren\ae um venientem de villa, patrem Alexandri et Rufi, ut tolleret crucem ejus.
${}^{22}$~Et perducunt illum in Golgotha locum~: quod est interpretatum Calvari\ae\ locus.
${}^{23}$~Et dabant ei bibere myrrhatum vinum~: et non accepit.
${}^{24}$~Et crucifigentes eum, diviserunt vestimenta ejus, mittentes sortem super eis, quis quid tolleret.
${}^{25}$~Erat autem hora tertia~: et crucifixerunt eum.
${}^{26}$~Et erat titulus caus\ae\ ejus inscriptus~: Rex Jud\ae orum.
${}^{27}$~Et cum eo crucifigunt duos latrones~: unum a dextris, et alium a sinistris ejus.
${}^{28}$~Et impleta est Scriptura, qu\ae\ dicit~: Et cum iniquis reputatus est.
${}^{29}$~Et pr\ae tereuntes blasphemabant eum, moventes capita sua, et dicentes~: Vah~! qui destruis templum Dei, et in tribus diebus re\ae dificas,
${}^{30}$~salvum fac temetipsum descendens de cruce.
${}^{31}$~Similiter et summi sacerdotes illudentes, ad alterutrum cum scribis dicebant~: Alios salvos fecit~; seipsum non potest salvum facere.
${}^{32}$~Christus rex Isra\"el descendat nunc de cruce, ut videamus, et credamus. Et qui cum eo crucifixi erant, convitiabantur ei.


${}^{33}$~Et facta hora sexta, tenebr\ae\ fact\ae\ sunt per totam terram usque in horam nonam.
${}^{34}$~Et hora nona exclamavit Jesus voce magna, dicens~: Eloi, eloi, lamma sabacthani~? quod est interpretatum~: Deus meus, Deus meus, ut quid dereliquisti me~?
${}^{35}$~Et quidam de circumstantibus audientes, dicebant~: Ecce Eliam vocat.
${}^{36}$~Currens autem unus, et implens spongiam aceto, circumponensque calamo, potum dabat ei, dicens~: Sinite, videamus si veniat Elias ad deponendum eum.
${}^{37}$~Jesus autem emissa voce magna expiravit.


${}^{38}$~Et velum templi scissum est in duo, a summo usque deorsum.
${}^{39}$~Videns autem centurio, qui ex adverso stabat, quia sic clamans expirasset, ait~: Vere hic homo Filius Dei erat.


${}^{40}$~Erant autem et mulieres de longe aspicientes~: inter quas erat Maria Magdalene, et Maria Jacobi minoris, et Joseph mater, et Salome~:
${}^{41}$~et cum esset in Galil\ae a, sequebantur eum, et ministrabant ei, et ali\ae\ mult\ae , qu\ae\ simul cum eo ascenderant Jerosolymam.
${}^{42}$~Et cum jam sero esset factum (quia erat parasceve, quod est ante sabbatum),
${}^{43}$~venit Joseph ab Arimath\ae a nobilis decurio, qui et ipse erat exspectans regnum Dei, et audacter introivit ad Pilatum, et petiit corpus Jesu.
${}^{44}$~Pilatus autem mirabatur si jam obiisset. Et accersito centurione, interrogavit eum si jam mortuus esset.
${}^{45}$~Et cum cognovisset a centurione, donavit corpus Joseph.
${}^{46}$~Joseph autem mercatus sindonem, et deponens eum involvit sindone, et posuit eum in monumento quod erat excisum de petra, et advolvit lapidem ad ostium monumenti.
${}^{47}$~Maria autem Magdalene et Maria Joseph aspiciebant ubi poneretur.

\bchapter
\mylettrine{E}t cum transisset sabbatum, Maria Magdalene, et Maria Jacobi, et Salome emerunt aromata ut venientes ungerent Jesum.
${}^{2}$~Et valde mane una sabbatorum, veniunt ad monumentum, orto jam sole.
${}^{3}$~Et dicebant ad invicem~: Quis revolvet nobis lapidem ab ostio monumenti~?
${}^{4}$~Et respicientes viderunt revolutum lapidem. Erat quippe magnus valde.
${}^{5}$~Et intro\"euntes in monumentum viderunt juvenem sedentem in dextris, coopertum stola candida, et obstupuerunt.
${}^{6}$~Qui dicit illis~: Nolite expavescere~: Jesum qu\ae ritis Nazarenum, crucifixum~: surrexit, non est hic, ecce locus ubi posuerunt eum.
${}^{7}$~Sed ite, dicite discipulis ejus, et Petro, quia pr\ae cedit vos in Galil\ae am~: ibi eum videbitis, sicut dixit vobis.
${}^{8}$~At ill\ae\ exeuntes, fugerunt de monumento~: invaserat enim eas tremor et pavor~: et nemini quidquam dixerunt~: timebant enim.


${}^{9}$~Surgens autem mane prima sabbati, apparuit primo Mari\ae\ Magdalene, de qua ejecerat septem d\ae monia.
${}^{10}$~Illa vadens nuntiavit his, qui cum eo fuerant, lugentibus et flentibus.
${}^{11}$~Et illi audientes quia viveret, et visus esset ab ea, non crediderunt.


${}^{12}$~Post h\ae c autem duobus ex his ambulantibus ostensus est in alia effigie, euntibus in villam~:
${}^{13}$~et illi euntes nuntiaverunt ceteris~: nec illis crediderunt.
${}^{14}$~Novissime recumbentibus illis undecim apparuit~: et exprobravit incredulitatem eorum et duritiam cordis~: quia iis, qui viderant eum resurrexisse, non crediderunt.
${}^{15}$~Et dixit eis~: Euntes in mundum universum pr\ae dicate Evangelium omni creatur\ae .
${}^{16}$~Qui crediderit, et baptizatus fuerit, salvus erit~: qui vero non crediderit, condemnabitur.
${}^{17}$~Signa autem eos qui crediderint, h\ae c sequentur~: in nomine meo d\ae monia ejicient~: linguis loquentur novis~:
${}^{18}$~serpentes tollent~: et si mortiferum quid biberint, non eis nocebit~: super \ae gros manus imponent, et bene habebunt.


${}^{19}$~Et Dominus quidem Jesus postquam locutus est eis, assumptus est in c\ae lum, et sedet a dextris Dei.
${}^{20}$~Illi autem profecti pr\ae dicaverunt ubique, Domino cooperante, et sermonem confirmante, sequentibus signis.
