\bbook{Evangelium secundum Matthæum}
{Matthæus}{images/genese_heading}


\bchapter
\mylettrine{L}iber generationis Jesu Christi filii David, filii Abraham.
${}^{2}$~Abraham genuit Isaac. Isaac autem genuit Jacob. Jacob autem genuit Judam, et fratres ejus.
${}^{3}$~Judas autem genuit Phares, et Zaram de Thamar. Phares autem genuit Esron. Esron autem genuit Aram.
${}^{4}$~Aram autem genuit Aminadab. Aminadab autem genuit Naasson. Naasson autem genuit Salmon.
${}^{5}$~Salmon autem genuit Booz de Rahab. Booz autem genuit Obed ex Ruth. Obed autem genuit Jesse. Jesse autem genuit David regem.
${}^{6}$~David autem rex genuit Salomonem ex ea qu\ae\ fuit Uri\ae .
${}^{7}$~Salomon autem genuit Roboam. Roboam autem genuit Abiam. Abias autem genuit Asa.
${}^{8}$~Asa autem genuit Josophat. Josophat autem genuit Joram. Joram autem genuit Oziam.
${}^{9}$~Ozias autem genuit Joatham. Joatham autem genuit Achaz. Achaz autem genuit Ezechiam.
${}^{10}$~Ezechias autem genuit Manassen. Manasses autem genuit Amon. Amon autem genuit Josiam.
${}^{11}$~Josias autem genuit Jechoniam, et fratres ejus in transmigratione Babylonis.
${}^{12}$~Et post transmigrationem Babylonis~: Jechonias genuit Salathiel. Salathiel autem genuit Zorobabel.
${}^{13}$~Zorobabel autem genuit Abiud. Abiud autem genuit Eliacim. Eliacim autem genuit Azor.
${}^{14}$~Azor autem genuit Sadoc. Sadoc autem genuit Achim. Achim autem genuit Eliud.
${}^{15}$~Eliud autem genuit Eleazar. Eleazar autem genuit Mathan. Mathan autem genuit Jacob.
${}^{16}$~Jacob autem genuit Joseph virum Mari\ae , de qua natus est Jesus, qui vocatur Christus.
${}^{17}$~Omnes itaque generationes ab Abraham usque ad David, generationes quatuordecim~: et a David usque ad transmigrationem Babylonis, generationes quatuordecim~: et a transmigratione Babylonis usque ad Christum, generationes quatuordecim.


${}^{18}$~Christi autem generatio sic erat~: cum esset desponsata mater ejus Maria Joseph, antequam convenirent inventa est in utero habens de Spiritu Sancto.
${}^{19}$~Joseph autem vir ejus cum esset justus, et nollet eam traducere, voluit occulte dimittere eam.
${}^{20}$~H\ae c autem eo cogitante, ecce angelus Domini apparuit in somnis ei, dicens~: Joseph, fili David, noli timere accipere Mariam conjugem tuam~: quod enim in ea natum est, de Spiritu Sancto est.
${}^{21}$~Pariet autem filium~: et vocabis nomen ejus Jesum~: ipse enim salvum faciet populum suum a peccatis eorum.
${}^{22}$~Hoc autem totum factum est, ut adimpleretur quod dictum est a Domino per prophetam dicentem~:
${}^{23}$~Ecce virgo in utero habebit, et pariet filium~: et vocabunt nomen ejus Emmanuel, quod est interpretatum Nobiscum Deus.
${}^{24}$~Exsurgens autem Joseph a somno, fecit sicut pr\ae cepit ei angelus Domini, et accepit conjugem suam.
${}^{25}$~Et non cognoscebat eam donec peperit filium suum primogenitum~: et vocavit nomen ejus Jesum.

\bchapter
\mylettrine{C}um ergo natus esset Jesus in Bethlehem Juda in diebus Herodis regis, ecce magi ab oriente venerunt Jerosolymam,
${}^{2}$~dicentes~: Ubi est qui natus est rex Jud\ae orum~? vidimus enim stellam ejus in oriente, et venimus adorare eum.
${}^{3}$~Audiens autem Herodes rex, turbatus est, et omnis Jerosolyma cum illo.
${}^{4}$~Et congregans omnes principes sacerdotum, et scribas populi, sciscitabatur ab eis ubi Christus nasceretur.
${}^{5}$~At illi dixerunt~: In Bethlehem Jud\ae~: sic enim scriptum est per prophetam~:
\begin{flushleft}\begin{verse}${}^{6}$~Et tu Bethlehem terra Juda,\\ nequaquam minima es\\ in principibus Juda~:\\ ex te enim exiet dux, qui regat populum meum Isra\"el.\end{verse}\end{flushleft}


${}^{7}$~Tunc Herodes clam vocatis magis diligenter didicit ab eis tempus stell\ae , qu\ae\ apparuit eis~:
${}^{8}$~et mittens illos in Bethlehem, dixit~: Ite, et interrogate diligenter de puero~: et cum inveneritis, renuntiate mihi, ut et ego veniens adorem eum.
${}^{9}$~Qui cum audissent regem, abierunt, et ecce stella, quam viderant in oriente, antecedebat eos, usque dum veniens staret supra, ubi erat puer.
${}^{10}$~Videntes autem stellam gavisi sunt gaudio magno valde.
${}^{11}$~Et intrantes domum, invenerunt puerum cum Maria matre ejus, et procidentes adoraverunt eum~: et apertis thesauris suis obtulerunt ei munera, aurum, thus, et myrrham.
${}^{12}$~Et responso accepto in somnis ne redirent ad Herodem, per aliam viam reversi sunt in regionem suam.


${}^{13}$~Qui cum recessissent, ecce angelus Domini apparuit in somnis Joseph, dicens~: Surge, et accipe puerum, et matrem ejus, et fuge in \AE gyptum, et esto ibi usque dum dicam tibi. Futurum est enim ut Herodes qu\ae rat puerum ad perdendum eum.
${}^{14}$~Qui consurgens accepit puerum et matrem ejus nocte, et secessit in \AE gyptum~:
${}^{15}$~et erat ibi usque ad obitum Herodis~: ut adimpleretur quod dictum est a Domino per prophetam dicentem~: Ex \AE gypto vocavi filium meum.
${}^{16}$~Tunc Herodes videns quoniam illusus esset a magis, iratus est valde, et mittens occidit omnes pueros, qui erant in Bethlehem, et in omnibus finibus ejus, a bimatu et infra secundum tempus, quod exquisierat a magis.
${}^{17}$~Tunc adimpletum est quod dictum est per Jeremiam prophetam dicentem~:
\begin{flushleft}\begin{verse}${}^{18}$~Vox in Rama audita est\\ ploratus, et ululatus multus~:\\ Rachel plorans filios suos,\\ et noluit consolari, quia non sunt.\end{verse}\end{flushleft}


${}^{19}$~Defuncto autem Herode, ecce angelus Domini apparuit in somnis Joseph in \AE gypto,
${}^{20}$~dicens~: Surge, et accipe puerum, et matrem ejus, et vade in terram Isra\"el~: defuncti sunt enim qui qu\ae rebant animam pueri.
${}^{21}$~Qui consurgens, accepit puerum, et matrem ejus, et venit in terram Isra\"el.
${}^{22}$~Audiens autem quod Archelaus regnaret in Jud\ae a pro Herode patre suo, timuit illo ire~: et admonitus in somnis, secessit in partes Galil\ae \ae .
${}^{23}$~Et veniens habitavit in civitate qu\ae\ vocatur Nazareth~: ut adimpleretur quod dictum est per prophetas~: Quoniam Nazar\ae us vocabitur.

\bchapter
\mylettrine{I}n diebus autem illis venit Joannes Baptista pr\ae dicans in deserto Jud\ae \ae ,
${}^{2}$~et dicens~: Pœnitentiam agite~: appropinquavit enim regnum c\ae lorum.
${}^{3}$~Hic est enim, qui dictus est per Isaiam prophetam dicentem~: \begin{flushleft}\begin{verse}Vox clamantis in deserto~:\\ Parate viam Domini~;\\ rectas facite semitas ejus.\end{verse}\end{flushleft}


${}^{4}$~Ipse autem Joannes habebat vestimentum de pilis camelorum, et zonam pelliceam circa lumbos suos~: esca autem ejus erat locust\ae , et mel silvestre.
${}^{5}$~Tunc exibat ad eum Jerosolyma, et omnis Jud\ae a, et omnis regio circa Jordanem~;
${}^{6}$~et baptizabantur ab eo in Jordane, confitentes peccata sua.
${}^{7}$~Videns autem multos pharis\ae orum, et sadduc\ae orum, venientes ad baptismum suum, dixit eis~: Progenies viperarum, quis demonstravit vobis fugere a ventura ira~?
${}^{8}$~Facite ergo fructum dignum pœnitenti\ae .
${}^{9}$~Et ne velitis dicere intra vos~: Patrem habemus Abraham. Dico enim vobis quoniam potens est Deus de lapidibus istis suscitare filios Abrah\ae .
${}^{10}$~Jam enim securis ad radicem arborum posita est. Omnis ergo arbor, qu\ae\ non facit fructum bonum, excidetur, et in ignem mittetur.
${}^{11}$~Ego quidem baptizo vos in aqua in pœnitentiam~: qui autem post me venturus est, fortior me est, cujus non sum dignus calceamenta portare~: ipse vos baptizabit in Spiritu Sancto, et igni.
${}^{12}$~Cujus ventilabrum in manu sua~: et permundabit aream suam~: et congregabit triticum suum in horreum, paleas autem comburet igni inextinguibili.


${}^{13}$~Tunc venit Jesus a Galil\ae a in Jordanem ad Joannem, ut baptizaretur ab eo.
${}^{14}$~Joannes autem prohibebat eum, dicens~: Ego a te debeo baptizari, et tu venis ad me~?
${}^{15}$~Respondens autem Jesus, dixit ei~: Sine modo~: sic enim decet nos implere omnem justitiam. Tunc dimisit eum.
${}^{16}$~Baptizatus autem Jesus, confestim ascendit de aqua, et ecce aperti sunt ei c\ae li~: et vidit Spiritum Dei descendentem sicut columbam, et venientem super se.
${}^{17}$~Et ecce vox de c\ae lis dicens~: Hic est Filius meus dilectus, in quo mihi complacui.

\bchapter
\mylettrine{T}unc Jesus ductus est in desertum a Spiritu, ut tentaretur a diabolo.
${}^{2}$~Et cum jejunasset quadraginta diebus, et quadraginta noctibus, postea esuriit.
${}^{3}$~Et accedens tentator dixit ei~: Si Filius Dei es, dic ut lapides isti panes fiant.
${}^{4}$~Qui respondens dixit~: Scriptum est~: Non in solo pane vivit homo, sed in omni verbo, quod procedit de ore Dei.
${}^{5}$~Tunc assumpsit eum diabolus in sanctam civitatem, et statuit eum super pinnaculum templi,
${}^{6}$~et dixit ei~: Si Filius Dei es, mitte te deorsum. Scriptum est enim~: Quia angelis suis mandavit de te, et in manibus tollent te, ne forte offendas ad lapidem pedem tuum.
${}^{7}$~Ait illi Jesus~: Rursum scriptum est~: Non tentabis Dominum Deum tuum.
${}^{8}$~Iterum assumpsit eum diabolus in montem excelsum valde~: et ostendit ei omnia regna mundi, et gloriam eorum,
${}^{9}$~et dixit ei~: H\ae c omnia tibi dabo, si cadens adoraveris me.
${}^{10}$~Tunc dicit ei Jesus~: Vade Satana~: Scriptum est enim~: Dominum Deum tuum adorabis, et illi soli servies.
${}^{11}$~Tunc reliquit eum diabolus~: et ecce angeli accesserunt, et ministrabant ei.


${}^{12}$~Cum autem audisset Jesus quod Joannes traditus esset, secessit in Galil\ae am~:
${}^{13}$~et, relicta civitate Nazareth, venit, et habitavit in Capharnaum maritima, in finibus Zabulon et Nephthalim~:
${}^{14}$~ut adimpleretur quod dictum est per Isaiam prophetam~:
\begin{flushleft}\begin{verse}${}^{15}$~Terra Zabulon, et terra Nephthalim,\\ via maris trans Jordanem,\\ Galil\ae a gentium~:\\
${}^{16}$~populus, qui sedebat in tenebris,\\ vidit lucem magnam~:\\ et sedentibus in regione umbr\ae\ mortis,\\ lux orta est eis.\end{verse}\end{flushleft}


${}^{17}$~Exinde cœpit Jesus pr\ae dicare, et dicere~: Pœnitentiam agite~: appropinquavit enim regnum c\ae lorum.


${}^{18}$~Ambulans autem Jesus juxta mare Galil\ae \ae , vidit duos fratres, Simonem, qui vocatur Petrus, et Andream fratrem ejus, mittentes rete in mare (erant enim piscatores),
${}^{19}$~et ait illis~: Venite post me, et faciam vos fieri piscatores hominum.
${}^{20}$~At illi continuo relictis retibus secuti sunt eum.
${}^{21}$~Et procedens inde, vidit alios duos fratres, Jacobum Zebed\ae i, et Joannem fratrem ejus, in navi cum Zebed\ae o patre eorum, reficientes retia sua~: et vocavit eos.
${}^{22}$~Illi autem statim relictis retibus et patre, secuti sunt eum.
${}^{23}$~Et circuibat Jesus totam Galil\ae am, docens in synagogis eorum, et pr\ae dicans Evangelium regni~: et sanans omnem languorem, et omnem infirmitatem in populo.
${}^{24}$~Et abiit opinio ejus in totam Syriam, et obtulerunt ei omnes male habentes, variis languoribus, et tormentis comprehensos, et qui d\ae monia habebant, et lunaticos, et paralyticos, et curavit eos~:
${}^{25}$~et secut\ae\ sunt eum turb\ae\ mult\ae\ de Galil\ae a, et Decapoli, et de Jerosolymis, et de Jud\ae a, et de trans Jordanem.

\bchapter
\mylettrine{V}idens autem Jesus turbas, ascendit in montem, et cum sedisset, accesserunt ad eum discipuli ejus,
${}^{2}$~et aperiens os suum docebat eos dicens~:
${}^{3}$~Beati pauperes spiritu~: quoniam ipsorum est regnum c\ae lorum.
${}^{4}$~Beati mites~: quoniam ipsi possidebunt terram.
${}^{5}$~Beati qui lugent~: quoniam ipsi consolabuntur.
${}^{6}$~Beati qui esuriunt et sitiunt justitiam~: quoniam ipsi saturabuntur.
${}^{7}$~Beati misericordes~: quoniam ipsi misericordiam consequentur.
${}^{8}$~Beati mundo corde~: quoniam ipsi Deum videbunt.
${}^{9}$~Beati pacifici~: quoniam filii Dei vocabuntur.
${}^{10}$~Beati qui persecutionem patiuntur propter justitiam~: quoniam ipsorum est regnum c\ae lorum.
${}^{11}$~Beati estis cum maledixerint vobis, et persecuti vos fuerint, et dixerint omne malum adversum vos mentientes, propter me~:
${}^{12}$~gaudete, et exsultate, quoniam merces vestra copiosa est in c\ae lis. Sic enim persecuti sunt prophetas, qui fuerunt ante vos.


${}^{13}$~Vos estis sal terr\ae . Quod si sal evanuerit, in quo salietur~? ad nihilum valet ultra, nisi ut mittatur foras, et conculcetur ab hominibus.
${}^{14}$~Vos estis lux mundi. Non potest civitas abscondi supra montem posita,
${}^{15}$~neque accendunt lucernam, et ponunt eam sub modio, sed super candelabrum, ut luceat omnibus qui in domo sunt.
${}^{16}$~Sic luceat lux vestra coram hominibus~: ut videant opera vestra bona, et glorificent Patrem vestrum, qui in c\ae lis est.


${}^{17}$~Nolite putare quoniam veni solvere legem aut prophetas~: non veni solvere, sed adimplere.
${}^{18}$~Amen quippe dico vobis, donec transeat c\ae lum et terra, jota unum aut unus apex non pr\ae teribit a lege, donec omnia fiant.
${}^{19}$~Qui ergo solverit unum de mandatis istis minimis, et docuerit sic homines, minimus vocabitur in regno c\ae lorum~: qui autem fecerit et docuerit, hic magnus vocabitur in regno c\ae lorum.
${}^{20}$~Dico enim vobis, quia nisi abundaverit justitia vestra plus quam scribarum et pharis\ae orum, non intrabitis in regnum c\ae lorum.


${}^{21}$~Audistis quia dictum est antiquis~: Non occides~: qui autem occiderit, reus erit judicio.
${}^{22}$~Ego autem dico vobis~: quia omnis qui irascitur fratri suo, reus erit judicio. Qui autem dixerit fratri suo, raca~: reus erit concilio. Qui autem dixerit, fatue~: reus erit gehenn\ae\ ignis.
${}^{23}$~Si ergo offers munus tuum ad altare, et ibi recordatus fueris quia frater tuus habet aliquid adversum te~:
${}^{24}$~relinque ibi munus tuum ante altare, et vade prius reconciliari fratri tuo~: et tunc veniens offeres munus tuum.
${}^{25}$~Esto consentiens adversario tuo cito dum es in via cum eo~: ne forte tradat te adversarius judici, et judex tradat te ministro~: et in carcerem mittaris.
${}^{26}$~Amen dico tibi, non exies inde, donec reddas novissimum quadrantem.


${}^{27}$~Audistis quia dictum est antiquis~: Non mœchaberis.
${}^{28}$~Ego autem dico vobis~: quia omnis qui viderit mulierem ad concupiscendum eam, jam mœchatus est eam in corde suo.
${}^{29}$~Quod si oculus tuus dexter scandalizat te, erue eum, et projice abs te~: expedit enim tibi ut pereat unum membrorum tuorum, quam totum corpus tuum mittatur in gehennam.
${}^{30}$~Et si dextra manus tua scandalizat te, abscide eam, et projice abs te~: expedit enim tibi ut pereat unum membrorum tuorum, quam totum corpus tuum eat in gehennam.
${}^{31}$~Dictum est autem~: Quicumque dimiserit uxorem suam, det ei libellum repudii.
${}^{32}$~Ego autem dico vobis~: quia omnis qui dimiserit uxorem suam, excepta fornicationis causa, facit eam mœchari~: et qui dimissam duxerit, adulterat.


${}^{33}$~Iterum audistis quia dictum est antiquis~: Non perjurabis~: reddes autem Domino juramenta tua.
${}^{34}$~Ego autem dico vobis, non jurare omnino, neque per c\ae lum, quia thronus Dei est~:
${}^{35}$~neque per terram, quia scabellum est pedum ejus~: neque per Jerosolymam, quia civitas est magni regis~:
${}^{36}$~neque per caput tuum juraveris, quia non potes unum capillum album facere, aut nigrum.
${}^{37}$~Sit autem sermo vester, est, est~: non, non~: quod autem his abundantius est, a malo est.


${}^{38}$~Audistis quia dictum est~: Oculum pro oculo, et dentem pro dente.
${}^{39}$~Ego autem dico vobis, non resistere malo~: sed si quis te percusserit in dexteram maxillam tuam, pr\ae be illi et alteram~:
${}^{40}$~et ei, qui vult tecum judicio contendere, et tunicam tuam tollere, dimitte ei et pallium~:
${}^{41}$~et quicumque te angariaverit mille passus, vade cum illo et alia duo.
${}^{42}$~Qui petit a te, da ei~: et volenti mutuari a te, ne avertaris.


${}^{43}$~Audistis quia dictum est~: Diliges proximum tuum, et odio habebis inimicum tuum.
${}^{44}$~Ego autem dico vobis~: diligite inimicos vestros, benefacite his qui oderunt vos, et orate pro persequentibus et calumniantibus vos~:
${}^{45}$~ut sitis filii Patris vestri, qui in c\ae lis est~: qui solem suum oriri facit super bonos et malos~: et pluit super justos et injustos.
${}^{46}$~Si enim diligitis eos qui vos diligunt, quam mercedem habebitis~? nonne et publicani hoc faciunt~?
${}^{47}$~Et si salutaveritis fratres vestros tantum, quid amplius facitis~? nonne et ethnici hoc faciunt~?
${}^{48}$~Estote ergo vos perfecti, sicut et Pater vester c\ae lestis perfectus est.

\bchapter
\mylettrine{A}ttendite ne justitiam vestram faciatis coram hominibus, ut videamini ab eis~: alioquin mercedem non habebitis apud Patrem vestrum qui in c\ae lis est.
${}^{2}$~Cum ergo facis eleemosynam, noli tuba canere ante te, sicut hypocrit\ae\ faciunt in synagogis, et in vicis, ut honorificentur ab hominibus. Amen dico vobis, receperunt mercedem suam.
${}^{3}$~Te autem faciente eleemosynam, nesciat sinistra tua quid faciat dextera tua~:
${}^{4}$~ut sit eleemosyna tua in abscondito, et Pater tuus, qui videt in abscondito, reddet tibi.


${}^{5}$~Et cum oratis, non eritis sicut hypocrit\ae\ qui amant in synagogis et in angulis platearum stantes orare, ut videantur ab hominibus~: amen dico vobis, receperunt mercedem suam.
${}^{6}$~Tu autem cum oraveris, intra in cubiculum tuum, et clauso ostio, ora Patrem tuum in abscondito~: et Pater tuus, qui videt in abscondito, reddet tibi.
${}^{7}$~Orantes autem, nolite multum loqui, sicut ethnici, putant enim quod in multiloquio suo exaudiantur.
${}^{8}$~Nolite ergo assimilari eis~: scit enim Pater vester, quid opus sit vobis, antequam petatis eum.
${}^{9}$~Sic ergo vos orabitis~: \begin{flushleft}\begin{verse}Pater noster, qui es in c\ae lis,\\ sanctificetur nomen tuum.\\
${}^{10}$~Adveniat regnum tuum~;\\ fiat voluntas tua, sicut in c\ae lo et in terra.\\
${}^{11}$~Panem nostrum supersubstantialem da nobis hodie,\\
${}^{12}$~et dimitte nobis debita nostra,\\ sicut et nos dimittimus debitoribus nostris.\\
${}^{13}$~Et ne nos inducas in tentationem,\\ sed libera nos a malo. Amen.\end{verse}\end{flushleft}


${}^{14}$~Si enim dimiseritis hominibus peccata eorum~: dimittet et vobis Pater vester c\ae lestis delicta vestra.
${}^{15}$~Si autem non dimiseritis hominibus~: nec Pater vester dimittet vobis peccata vestra.


${}^{16}$~Cum autem jejunatis, nolite fieri sicut hypocrit\ae , tristes. Exterminant enim facies suas, ut appareant hominibus jejunantes. Amen dico vobis, quia receperunt mercedem suam.
${}^{17}$~Tu autem, cum jejunas, unge caput tuum, et faciem tuam lava,
${}^{18}$~ne videaris hominibus jejunans, sed Patri tuo, qui est in abscondito~: et Pater tuus, qui videt in abscondito, reddet tibi.


${}^{19}$~Nolite thesaurizare vobis thesauros in terra~: ubi \ae rugo, et tinea demolitur~: et ubi fures effodiunt, et furantur.
${}^{20}$~Thesaurizate autem vobis thesauros in c\ae lo, ubi neque \ae rugo, neque tinea demolitur, et ubi fures non effodiunt, nec furantur.
${}^{21}$~Ubi enim est thesaurus tuus, ibi est et cor tuum.
${}^{22}$~Lucerna corporis tui est oculus tuus. Si oculus tuus fuerit simplex, totum corpus tuum lucidum erit.
${}^{23}$~Si autem oculus tuus fuerit nequam, totum corpus tuum tenebrosum erit. Si ergo lumen, quod in te est, tenebr\ae\ sunt~: ips\ae\ tenebr\ae\ quant\ae\ erunt~?
${}^{24}$~Nemo potest duobus dominis servire~: aut enim unum odio habebit, et alterum diliget~: aut unum sustinebit, et alterum contemnet. Non potestis Deo servire et mammon\ae .
${}^{25}$~Ideo dico vobis, ne solliciti sitis anim\ae\ vestr\ae\ quid manducetis, neque corpori vestro quid induamini. Nonne anima plus est quam esca, et corpus plus quam vestimentum~?
${}^{26}$~Respicite volatilia c\ae li, quoniam non serunt, neque metunt, neque congregant in horrea~: et Pater vester c\ae lestis pascit illa. Nonne vos magis pluris estis illis~?
${}^{27}$~Quis autem vestrum cogitans potest adjicere ad staturam suam cubitum unum~?
${}^{28}$~Et de vestimento quid solliciti estis~? Considerate lilia agri quomodo crescunt~: non laborant, neque nent.
${}^{29}$~Dico autem vobis, quoniam nec Salomon in omni gloria sua coopertus est sicut unum ex istis.
${}^{30}$~Si autem fœnum agri, quod hodie est, et cras in clibanum mittitur, Deus sic vestit, quanto magis vos modic\ae\ fidei~?
${}^{31}$~Nolite ergo solliciti esse, dicentes~: Quid manducabimus, aut quid bibemus, aut quo operiemur~?
${}^{32}$~h\ae c enim omnia gentes inquirunt. Scit enim Pater vester, quia his omnibus indigetis.
${}^{33}$~Qu\ae rite ergo primum regnum Dei, et justitiam ejus~: et h\ae c omnia adjicientur vobis.
${}^{34}$~Nolite ergo solliciti esse in crastinum. Crastinus enim dies sollicitus erit sibi ipsi~: sufficit diei malitia sua.

\bchapter
\mylettrine{N}olite judicare, ut non judicemini.
${}^{2}$~In quo enim judicio judicaveritis, judicabimini~: et in qua mensura mensi fueritis, remetietur vobis.
${}^{3}$~Quid autem vides festucam in oculo fratris tui, et trabem in oculo tuo non vides~?
${}^{4}$~aut quomodo dicis fratri tuo~: Sine ejiciam festucam de oculo tuo, et ecce trabs est in oculo tuo~?
${}^{5}$~Hypocrita, ejice primum trabem de oculo tuo, et tunc videbis ejicere festucam de oculo fratris tui.
${}^{6}$~Nolite dare sanctum canibus~: neque mittatis margaritas vestras ante porcos, ne forte conculcent eas pedibus suis, et conversi dirumpant vos.


${}^{7}$~Petite, et dabitur vobis~: qu\ae rite, et invenietis~: pulsate, et aperietur vobis.
${}^{8}$~Omnis enim qui petit, accipit~: et qui qu\ae rit, invenit~: et pulsanti aperietur.
${}^{9}$~Aut quis est ex vobis homo, quem si petierit filius suus panem, numquid lapidem porriget ei~?
${}^{10}$~aut si piscem petierit, numquid serpentem porriget ei~?
${}^{11}$~Si ergo vos, cum sitis mali, nostis bona data dare filiis vestris~: quanto magis Pater vester, qui in c\ae lis est, dabit bona petentibus se~?


${}^{12}$~Omnia ergo qu\ae cumque vultis ut faciant vobis homines, et vos facite illis. H\ae c est enim lex, et prophet\ae .
${}^{13}$~Intrate per angustam portam~: quia lata porta, et spatiosa via est, qu\ae\ ducit ad perditionem, et multi sunt qui intrant per eam.
${}^{14}$~Quam angusta porta, et arcta via est, qu\ae\ ducit ad vitam~: et pauci sunt qui inveniunt eam~!


${}^{15}$~Attendite a falsis prophetis, qui veniunt ad vos in vestimentis ovium, intrinsecus autem sunt lupi rapaces~:
${}^{16}$~a fructibus eorum cognoscetis eos. Numquid colligunt de spinis uvas, aut de tribulis ficus~?
${}^{17}$~Sic omnis arbor bona fructus bonos facit~: mala autem arbor malos fructus facit.
${}^{18}$~Non potest arbor bona malos fructus facere~: neque arbor mala bonos fructus facere.
${}^{19}$~Omnis arbor, qu\ae\ non facit fructum bonum, excidetur, et in ignem mittetur.
${}^{20}$~Igitur ex fructibus eorum cognoscetis eos.


${}^{21}$~Non omnis qui dicit mihi, Domine, Domine, intrabit in regnum c\ae lorum~: sed qui facit voluntatem Patris mei, qui in c\ae lis est, ipse intrabit in regnum c\ae lorum.
${}^{22}$~Multi dicent mihi in illa die~: Domine, Domine, nonne in nomine tuo prophetavimus, et in nomine tuo d\ae monia ejecimus, et in nomine tuo virtutes multas fecimus~?
${}^{23}$~Et tunc confitebor illis~: Quia numquam novi vos~: discedite a me, qui operamini iniquitatem.
${}^{24}$~Omnis ergo qui audit verba mea h\ae c, et facit ea, assimilabitur viro sapienti, qui \ae dificavit domum suam supra petram,
${}^{25}$~et descendit pluvia, et venerunt flumina, et flaverunt venti, et irruerunt in domum illam, et non cecidit~: fundata enim erat super petram.
${}^{26}$~Et omnis qui audit verba mea h\ae c, et non facit ea, similis erit viro stulto, qui \ae dificavit domum suam super arenam~:
${}^{27}$~et descendit pluvia, et venerunt flumina, et flaverunt venti, et irruerunt in domum illam, et cecidit, et fuit ruina illius magna.
${}^{28}$~Et factum est~: cum consummasset Jesus verba h\ae c, admirabantur turb\ae\ super doctrina ejus.
${}^{29}$~Erat enim docens eos sicut potestatem habens, et non sicut scrib\ae\ eorum, et pharis\ae i.

\bchapter
\mylettrine{C}um autem descendisset de monte, secut\ae\ sunt eum turb\ae\ mult\ae~:
${}^{2}$~et ecce leprosus veniens, adorabat eum, dicens~: Domine, si vis, potes me mundare.
${}^{3}$~Et extendens Jesus manum, tetigit eum, dicens~: Volo~: mundare. Et confestim mundata est lepra ejus.
${}^{4}$~Et ait illi Jesus~: Vide, nemini dixeris~: sed vade, ostende te sacerdoti, et offer munus, quod pr\ae cepit Moyses, in testimonium illis.


${}^{5}$~Cum autem introisset Capharnaum, accessit ad eum centurio, rogans eum,
${}^{6}$~et dicens~: Domine, puer meus jacet in domo paralyticus, et male torquetur.
${}^{7}$~Et ait illi Jesus~: Ego veniam, et curabo eum.
${}^{8}$~Et respondens centurio, ait~: Domine, non sum dignus ut intres sub tectum meum~: sed tantum dic verbo, et sanabitur puer meus.
${}^{9}$~Nam et ego homo sum sub potestate constitutus, habens sub me milites, et dico huic~: Vade, et vadit~: et alii~: Veni, et venit~: et servo meo~: Fac hoc, et facit.
${}^{10}$~Audiens autem Jesus miratus est, et sequentibus se dixit~: Amen dico vobis, non inveni tantam fidem in Isra\"el.
${}^{11}$~Dico autem vobis, quod multi ab oriente et occidente venient, et recumbent cum Abraham, et Isaac, et Jacob in regno c\ae lorum~:
${}^{12}$~filii autem regni ejicientur in tenebras exteriores~: ibi erit fletus et stridor dentium.
${}^{13}$~Et dixit Jesus centurioni~: Vade, et sicut credidisti, fiat tibi. Et sanatus est puer in illa hora.


${}^{14}$~Et cum venisset Jesus in domum Petri, vidit socrum ejus jacentem, et febricitantem~:
${}^{15}$~et tetigit manum ejus, et dimisit eam febris, et surrexit, et ministrabat eis.


${}^{16}$~Vespere autem facto, obtulerunt ei multos d\ae monia habentes~: et ejiciebat spiritus verbo, et omnes male habentes curavit~:
${}^{17}$~ut adimpleretur quod dictum est per Isaiam prophetam, dicentem~: \begin{flushleft}\begin{verse}Ipse infirmitates nostras accepit~:\\ et \ae grotationes nostras portavit.\end{verse}\end{flushleft}


${}^{18}$~Videns autem Jesus turbas multas circum se, jussit ire trans fretum.


${}^{19}$~Et accedens unus scriba, ait illi~: Magister, sequar te, quocumque ieris.
${}^{20}$~Et dicit ei Jesus~: Vulpes foveas habent, et volucres c\ae li nidos~; Filius autem hominis non habet ubi caput reclinet.
${}^{21}$~Alius autem de discipulis ejus ait illi~: Domine, permitte me primum ire, et sepelire patrem meum.
${}^{22}$~Jesus autem ait illi~: Sequere me, et dimitte mortuos sepelire mortuos suos.


${}^{23}$~Et ascendente eo in naviculam, secuti sunt eum discipuli ejus~:
${}^{24}$~et ecce motus magnus factus est in mari, ita ut navicula operiretur fluctibus~: ipse vero dormiebat.
${}^{25}$~Et accesserunt ad eum discipuli ejus, et suscitaverunt eum, dicentes~: Domine, salva nos~: perimus.
${}^{26}$~Et dicit eis Jesus~: Quid timidi estis, modic\ae\ fidei~? Tunc surgens imperavit ventis, et mari, et facta est tranquillitas magna.
${}^{27}$~Porro homines mirati sunt, dicentes~: Qualis est hic, quia venti et mare obediunt ei~?


${}^{28}$~Et cum venisset trans fretum in regionem Gerasenorum, occurrerunt ei duo habentes d\ae monia, de monumentis exeuntes, s\ae vi nimis, ita ut nemo posset transire per viam illam.
${}^{29}$~Et ecce clamaverunt, dicentes~: Quid nobis et tibi, Jesu fili Dei~? Venisti huc ante tempus torquere nos~?
${}^{30}$~Erat autem non longe ab illis grex multorum porcorum pascens.
${}^{31}$~D\ae mones autem rogabant eum, dicentes~: Si ejicis nos hinc, mitte nos in gregem porcorum.
${}^{32}$~Et ait illis~: Ite. At illi exeuntes abierunt in porcos, et ecce impetu abiit totus grex per pr\ae ceps in mare~: et mortui sunt in aquis.
${}^{33}$~Pastores autem fugerunt~: et venientes in civitatem, nuntiaverunt omnia, et de eis qui d\ae monia habuerant.
${}^{34}$~Et ecce tota civitas exiit obviam Jesu~: et viso eo, rogabant ut transiret a finibus eorum.

\bchapter
\mylettrine{E}t ascendens in naviculam, transfretavit, et venit in civitatem suam.
${}^{2}$~Et ecce offerebant ei paralyticum jacentem in lecto. Et videns Jesus fidem illorum, dixit paralytico~: Confide fili, remittuntur tibi peccata tua.
${}^{3}$~Et ecce quidam de scribis dixerunt intra se~: Hic blasphemat.
${}^{4}$~Et cum vidisset Jesus cogitationes eorum, dixit~: Ut quid cogitatis mala in cordibus vestris~?
${}^{5}$~Quid est facilius dicere~: Dimittuntur tibi peccata tua~: an dicere~: Surge, et ambula~?
${}^{6}$~Ut autem sciatis, quia Filius hominis habet potestatem in terra dimittendi peccata, tunc ait paralytico~: Surge, tolle lectum tuum, et vade in domum tuam.
${}^{7}$~Et surrexit, et abiit in domum suam.
${}^{8}$~Videntes autem turb\ae\ timuerunt, et glorificaverunt Deum, qui dedit potestatem talem hominibus.


${}^{9}$~Et, cum transiret inde Jesus, vidit hominem sedentem in telonio, Matth\ae um nomine. Et ait illi~: Sequere me. Et surgens, secutus est eum.
${}^{10}$~Et factum est, discumbente eo in domo, ecce multi publicani et peccatores venientes, discumbebant cum Jesu, et discipulis ejus.
${}^{11}$~Et videntes pharis\ae i, dicebant discipulis ejus~: Quare cum publicanis et peccatoribus manducat magister vester~?
${}^{12}$~At Jesus audiens, ait~: Non est opus valentibus medicus, sed male habentibus.
${}^{13}$~Euntes autem discite quid est~: Misericordiam volo, et non sacrificium. Non enim veni vocare justos, sed peccatores.
${}^{14}$~Tunc accesserunt ad eum discipuli Joannis, dicentes~: Quare nos, et pharis\ae i, jejunamus frequenter~: discipuli autem tui non jejunant~?
${}^{15}$~Et ait illis Jesus~: Numquid possunt filii sponsi lugere, quamdiu cum illis est sponsus~? Venient autem dies cum auferetur ab eis sponsus~: et tunc jejunabunt.
${}^{16}$~Nemo autem immittit commissuram panni rudis in vestimentum vetus~: tollit enim plenitudinem ejus a vestimento, et pejor scissura fit.
${}^{17}$~Neque mittunt vinum novum in utres veteres~: alioquin rumpuntur utres, et vinum effunditur, et utres pereunt. Sed vinum novum in utres novos mittunt~: et ambo conservantur.


${}^{18}$~H\ae c illo loquente ad eos, ecce princeps unus accessit, et adorabat eum, dicens~: Domine, filia mea modo defuncta est~: sed veni, impone manum tuam super eam, et vivet.
${}^{19}$~Et surgens Jesus, sequebatur eum, et discipuli ejus.
${}^{20}$~Et ecce mulier, qu\ae\ sanguinis fluxum patiebatur duodecim annis, accessit retro, et tetigit fimbriam vestimenti ejus.
${}^{21}$~Dicebat enim intra se~: Si tetigero tantum vestimentum ejus, salva ero.
${}^{22}$~At Jesus conversus, et videns eam, dixit~: Confide, filia, fides tua te salvam fecit. Et salva facta est mulier ex illa hora.


${}^{23}$~Et cum venisset Jesus in domum principis, et vidisset tibicines et turbam tumultuantem, dicebat~:
${}^{24}$~Recedite~: non est enim mortua puella, sed dormit. Et deridebant eum.
${}^{25}$~Et cum ejecta esset turba, intravit~: et tenuit manum ejus, et surrexit puella.
${}^{26}$~Et exiit fama h\ae c in universam terram illam.


${}^{27}$~Et transeunte inde Jesu, secuti sunt eum duo c\ae ci, clamantes, et dicentes~: Miserere nostri, fili David.
${}^{28}$~Cum autem venisset domum, accesserunt ad eum c\ae ci. Et dicit eis Jesus~: Creditis quia hoc possum facere vobis~? Dicunt ei~: Utique, Domine.
${}^{29}$~Tunc tetigit oculos eorum, dicens~: Secundum fidem vestram, fiat vobis.
${}^{30}$~Et aperti sunt oculi eorum~: et comminatus est illis Jesus, dicens~: Videte ne quis sciat.
${}^{31}$~Illi autem exeuntes, diffamaverunt eum in tota terra illa.


${}^{32}$~Egressis autem illis, ecce obtulerunt ei hominem mutum, d\ae monium habentem.
${}^{33}$~Et ejecto d\ae monio, locutus est mutus, et mirat\ae\ sunt turb\ae , dicentes~: Numquam apparuit sic in Isra\"el.
${}^{34}$~Pharis\ae i autem dicebant~: In principe d\ae moniorum ejicit d\ae mones.


${}^{35}$~Et circuibat Jesus omnes civitates, et castella, docens in synagogis eorum, et pr\ae dicans Evangelium regni, et curans omnem languorem, et omnem infirmitatem.
${}^{36}$~Videns autem turbas, misertus est eis~: quia erant vexati, et jacentes sicut oves non habentes pastorem.
${}^{37}$~Tunc dicit discipulis suis~: Messis quidem multa, operarii autem pauci.
${}^{38}$~Rogate ergo Dominum messis, ut mittat operarios in messem suam.

\bchapter
\mylettrine{E}t convocatis duodecim discipulis suis, dedit illis potestatem spirituum immundorum, ut ejicerent eos, et curarent omnem languorem, et omnem infirmitatem.
${}^{2}$~Duodecim autem Apostolorum nomina sunt h\ae c. Primus, Simon, qui dicitur Petrus~: et Andreas frater ejus,
${}^{3}$~Jacobus Zebed\ae i, et Joannes frater ejus, Philippus, et Bartholom\ae us, Thomas, et Matth\ae us publicanus, Jacobus Alph\ae i, et Thadd\ae us,
${}^{4}$~Simon Chanan\ae us, et Judas Iscariotes, qui et tradidit eum.


${}^{5}$~Hos duodecim misit Jesus, pr\ae cipiens eis, dicens~: In viam gentium ne abieritis, et in civitates Samaritanorum ne intraveritis~:
${}^{6}$~sed potius ite ad oves qu\ae\ perierunt domus Isra\"el.
${}^{7}$~Euntes autem pr\ae dicate, dicentes~: Quia appropinquavit regnum c\ae lorum.
${}^{8}$~Infirmos curate, mortuos suscitate, leprosos mundate, d\ae mones ejicite~: gratis accepistis, gratis date.
${}^{9}$~Nolite possidere aurum, neque argentum, neque pecuniam in zonis vestris~:
${}^{10}$~non peram in via, neque duas tunicas, neque calceamenta, neque virgam~: dignus enim est operarius cibo suo.
${}^{11}$~In quamcumque autem civitatem aut castellum intraveritis, interrogate, quis in ea dignus sit~: et ibi manete donec exeatis.
${}^{12}$~Intrantes autem in domum, salutate eam, dicentes~: Pax huic domui.
${}^{13}$~Et siquidem fuerit domus illa digna, veniet pax vestra super eam~: si autem non fuerit digna, pax vestra revertetur ad vos.
${}^{14}$~Et quicumque non receperit vos, neque audierit sermones vestros~: exeuntes foras de domo, vel civitate, excutite pulverem de pedibus vestris.
${}^{15}$~Amen dico vobis~: Tolerabilius erit terr\ae\ Sodomorum et Gomorrh\ae orum in die judicii, quam illi civitati.


${}^{16}$~Ecce ego mitto vos sicut oves in medio luporum. Estote ergo prudentes sicut serpentes, et simplices sicut columb\ae .
${}^{17}$~Cavete autem ab hominibus. Tradent enim vos in conciliis, et in synagogis suis flagellabunt vos~:
${}^{18}$~et ad pr\ae sides, et ad reges ducemini propter me in testimonium illis, et gentibus.
${}^{19}$~Cum autem tradent vos, nolite cogitare quomodo, aut quid loquamini~: dabitur enim vobis in illa hora, quid loquamini~:
${}^{20}$~non enim vos estis qui loquimini, sed Spiritus Patris vestri, qui loquitur in vobis.
${}^{21}$~Tradet autem frater fratrem in mortem, et pater filium~: et insurgent filii in parentes, et morte eos afficient~:
${}^{22}$~et eritis odio omnibus propter nomen meum~: qui autem perseveraverit usque in finem, hic salvus erit.


${}^{23}$~Cum autem persequentur vos in civitate ista, fugite in aliam. Amen dico vobis, non consummabitis civitates Isra\"el, donec veniat Filius hominis.
${}^{24}$~Non est discipulus super magistrum, nec servus super dominum suum~:
${}^{25}$~sufficit discipulo ut sit sicut magister ejus, et servo, sicut dominus ejus. Si patremfamilias Beelzebub vocaverunt, quanto magis domesticos ejus~?
${}^{26}$~Ne ergo timueritis eos. Nihil enim est opertum, quod non revelabitur~: et occultum, quod non scietur.
${}^{27}$~Quod dico vobis in tenebris, dicite in lumine~: et quod in aure auditis, pr\ae dicate super tecta.
${}^{28}$~Et nolite timere eos qui occidunt corpus, animam autem non possunt occidere~: sed potius timete eum, qui potest et animam et corpus perdere in gehennam.
${}^{29}$~Nonne duo passeres asse veneunt~? et unus ex illis non cadet super terram sine Patre vestro.
${}^{30}$~Vestri autem capilli capitis omnes numerati sunt.
${}^{31}$~Nolite ergo timere~: multis passeribus meliores estis vos.
${}^{32}$~Omnis ergo qui confitebitur me coram hominibus, confitebor et ego eum coram Patre meo, qui in c\ae lis est.
${}^{33}$~Qui autem negaverit me coram hominibus, negabo et ego eum coram Patre meo, qui in c\ae lis est.
${}^{34}$~Nolite arbitrari quia pacem venerim mittere in terram~: non veni pacem mittere, sed gladium~:
${}^{35}$~veni enim separare hominem adversus patrem suum, et filiam adversus matrem suam, et nurum adversus socrum suam~:
${}^{36}$~et inimici hominis, domestici ejus.
${}^{37}$~Qui amat patrem aut matrem plus quam me, non est me dignus~: et qui amat filium aut filiam super me, non est me dignus.
${}^{38}$~Et qui non accipit crucem suam, et sequitur me, non est me dignus.
${}^{39}$~Qui invenit animam suam, perdet illam~: et qui perdiderit animam suam propter me, inveniet eam.
${}^{40}$~Qui recipit vos, me recipit~: et qui me recipit, recipit eum qui me misit.
${}^{41}$~Qui recipit prophetam in nomine prophet\ae , mercedem prophet\ae\ accipiet~: et qui recipit justum in nomine justi, mercedem justi accipiet.
${}^{42}$~Et quicumque potum dederit uni ex minimis istis calicem aqu\ae\ frigid\ae\ tantum in nomine discipuli~: amen dico vobis, non perdet mercedem suam.

\bchapter
\mylettrine{E}t factum est, cum consummasset Jesus, pr\ae cipiens duodecim discipulis suis, transiit inde ut doceret, et pr\ae dicaret in civitatibus eorum.


${}^{2}$~Joannes autem cum audisset in vinculis opera Christi, mittens duos de discipulis suis,
${}^{3}$~ait illi~: Tu es, qui venturus es, an alium exspectamus~?
${}^{4}$~Et respondens Jesus ait illis~: Euntes renuntiate Joanni qu\ae\ audistis, et vidistis.
${}^{5}$~C\ae ci vident, claudi ambulant, leprosi mundantur, surdi audiunt, mortui resurgunt, pauperes evangelizantur~:
${}^{6}$~et beatus est, qui non fuerit scandalizatus in me.


${}^{7}$~Illis autem abeuntibus, cœpit Jesus dicere ad turbas de Joanne~: Quid existis in desertum videre~? arundinem vento agitatam~?
${}^{8}$~Sed quid existis videre~? hominem mollibus vestitum~? Ecce qui mollibus vestiuntur, in domibus regum sunt.
${}^{9}$~Sed quid existis videre~? prophetam~? Etiam dico vobis, et plus quam prophetam.
${}^{10}$~Hic est enim de quo scriptum est~: Ecce ego mitto angelum meum ante faciem tuam, qui pr\ae parabit viam tuam ante te.
${}^{11}$~Amen dico vobis, non surrexit inter natos mulierum major Joanne Baptista~: qui autem minor est in regno c\ae lorum, major est illo.
${}^{12}$~A diebus autem Joannis Baptist\ae\ usque nunc, regnum c\ae lorum vim patitur, et violenti rapiunt illud.
${}^{13}$~Omnes enim prophet\ae\ et lex usque ad Joannem prophetaverunt~:
${}^{14}$~et si vultis recipere, ipse est Elias, qui venturus est.
${}^{15}$~Qui habet aures audiendi, audiat.


${}^{16}$~Cui autem similem \ae stimabo generationem istam~? Similis est pueris sedentibus in foro~: qui clamantes co\ae qualibus
${}^{17}$~dicunt~: Cecinimus vobis, et non saltastis~: lamentavimus, et non planxistis.
${}^{18}$~Venit enim Joannes neque manducans, neque bibens, et dicunt~: D\ae monium habet.
${}^{19}$~Venit Filius hominis manducans, et bibens, et dicunt~: Ecce homo vorax, et potator vini, publicanorum et peccatorum amicus. Et justificata est sapientia a filiis suis.


${}^{20}$~Tunc cœpit exprobrare civitatibus, in quibus fact\ae\ sunt plurim\ae\ virtutes ejus, quia non egissent pœnitentiam~:
${}^{21}$~V\ae\ tibi Corozain, v\ae\ tibi Bethsaida~: quia, si in Tyro et Sidone fact\ae\ essent virtutes qu\ae\ fact\ae\ sunt in vobis, olim in cilicio et cinere pœnitentiam egissent.
${}^{22}$~Verumtamen dico vobis~: Tyro et Sidoni remissius erit in die judicii, quam vobis.
${}^{23}$~Et tu Capharnaum, numquid usque in c\ae lum exaltaberis~? usque in infernum descendes, quia si in Sodomis fact\ae\ fuissent virtutes qu\ae\ fact\ae\ sunt in te, forte mansissent usque in hanc diem.
${}^{24}$~Verumtamen dico vobis, quia terr\ae\ Sodomorum remissius erit in die judicii, quam tibi.


${}^{25}$~In illo tempore respondens Jesus dixit~: Confiteor tibi, Pater, Domine c\ae li et terr\ae , quia abscondisti h\ae c a sapientibus, et prudentibus, et revelasti ea parvulis.
${}^{26}$~Ita Pater~: quoniam sic fuit placitum ante te.
${}^{27}$~Omnia mihi tradita sunt a Patre meo. Et nemo novit Filium, nisi Pater~: neque Patrem quis novit, nisi Filius, et cui voluerit Filius revelare.
${}^{28}$~Venite ad me omnes qui laboratis, et onerati estis, et ego reficiam vos.
${}^{29}$~Tollite jugum meum super vos, et discite a me, quia mitis sum, et humilis corde~: et invenietis requiem animabus vestris.
${}^{30}$~Jugum enim meum suave est, et onus meum leve.

\bchapter
\mylettrine{I}n illo tempore abiit Jesus per sata sabbato~: discipuli autem ejus esurientes cœperunt vellere spicas, et manducare.
${}^{2}$~Pharis\ae i autem videntes, dixerunt ei~: Ecce discipuli tui faciunt quod non licet facere sabbatis.
${}^{3}$~At ille dixit eis~: Non legistis quid fecerit David, quando esuriit, et qui cum eo erant~:
${}^{4}$~quomodo intravit in domum Dei, et panes propositionis comedit, quos non licebat ei edere, neque his qui cum eo erant, nisi solis sacerdotibus~?
${}^{5}$~aut non legistis in lege quia sabbatis sacerdotes in templo sabbatum violant, et sine crimine sunt~?
${}^{6}$~Dico autem vobis, quia templo major est hic.
${}^{7}$~Si autem sciretis, quid est~: Misericordiam volo, et non sacrificium~: numquam condemnassetis innocentes~:
${}^{8}$~dominus enim est Filius hominis etiam sabbati.


${}^{9}$~Et cum inde transisset, venit in synagogam eorum.
${}^{10}$~Et ecce homo manum habens aridam, et interrogabant eum, dicentes~: Si licet sabbatis curare~? ut accusarent eum.
${}^{11}$~Ipse autem dixit illis~: Quis erit ex vobis homo, qui habeat ovem unam, et si ceciderit h\ae c sabbatis in foveam, nonne tenebit et levabit eam~?
${}^{12}$~Quanto magis melior est homo ove~? itaque licet sabbatis benefacere.
${}^{13}$~Tunc ait homini~: Extende manum tuam. Et extendit, et restituta est sanitati sicut altera.
${}^{14}$~Exeuntes autem pharis\ae i, consilium faciebant adversus eum, quomodo perderent eum.
${}^{15}$~Jesus autem sciens recessit inde~: et secuti sunt eum multi, et curavit eos omnes~:
${}^{16}$~et pr\ae cepit eis ne manifestum eum facerent.
${}^{17}$~Ut adimpleretur quod dictum est per Isaiam prophetam, dicentem~:
\begin{flushleft}\begin{verse}${}^{18}$~Ecce puer meus, quem elegi, dilectus meus,\\ in quo bene complacuit anim\ae\ me\ae .\\ Ponam spiritum meum super eum,\\ et judicium gentibus nuntiabit.\\
${}^{19}$~Non contendet, neque clamabit,\\ neque audiet aliquis in plateis vocem ejus~:\\
${}^{20}$~arundinem quassatam non confringet,\\ et linum fumigans non extinguet,\\ donec ejiciat ad victoriam judicium~:\\
${}^{21}$~et in nomine ejus gentes sperabunt.\end{verse}\end{flushleft}


${}^{22}$~Tunc oblatus est ei d\ae monium habens, c\ae cus, et mutus, et curavit eum ita ut loqueretur, et videret.
${}^{23}$~Et stupebant omnes turb\ae , et dicebant~: Numquid hic est filius David~?
${}^{24}$~Pharis\ae i autem audientes, dixerunt~: Hic non ejicit d\ae mones nisi in Beelzebub principe d\ae moniorum.
${}^{25}$~Jesus autem sciens cogitationes eorum, dixit eis~: Omne regnum divisum contra se desolabitur~: et omnis civitas vel domus divisa contra se, non stabit.
${}^{26}$~Et si Satanas Satanam ejicit, adversus se divisus est~: quomodo ergo stabit regnum ejus~?
${}^{27}$~Et si ego in Beelzebub ejicio d\ae mones, filii vestri in quo ejiciunt~? ideo ipsi judices vestri erunt.
${}^{28}$~Si autem ego in Spiritu Dei ejicio d\ae mones, igitur pervenit in vos regnum Dei.
${}^{29}$~Aut quomodo potest quisquam intrare in domum fortis, et vasa ejus diripere, nisi prius alligaverit fortem~? et tunc domum illius diripiet.
${}^{30}$~Qui non est mecum, contra me est~; et qui non congregat mecum, spargit.
${}^{31}$~Ideo dico vobis~: Omne peccatum et blasphemia remittetur hominibus, Spiritus autem blasphemia non remittetur.
${}^{32}$~Et quicumque dixerit verbum contra Filium hominis, remittetur ei~: qui autem dixerit contra Spiritum Sanctum, non remittetur ei, neque in hoc s\ae culo, neque in futuro.
${}^{33}$~Aut facite arborem bonam, et fructum ejus bonum~: aut facite arborem malam, et fructum ejus malum~: siquidem ex fructu arbor agnoscitur.
${}^{34}$~Progenies viperarum, quomodo potestis bona loqui, cum sitis mali~? ex abundantia enim cordis os loquitur.
${}^{35}$~Bonus homo de bono thesauro profert bona~: et malus homo de malo thesauro profert mala.
${}^{36}$~Dico autem vobis quoniam omne verbum otiosum, quod locuti fuerint homines, reddent rationem de eo in die judicii.
${}^{37}$~Ex verbis enim tuis justificaberis et ex verbis tuis condemnaberis.


${}^{38}$~Tunc responderunt ei quidam de scribis et pharis\ae is, dicentes~: Magister, volumus a te signum videre.
${}^{39}$~Qui respondens ait illis~: Generatio mala et adultera signum qu\ae rit~: et signum non dabitur ei, nisi signum Jon\ae\ prophet\ae .
${}^{40}$~Sicut enim fuit Jonas in ventre ceti tribus diebus, et tribus noctibus, sic erit Filius hominis in corde terr\ae\ tribus diebus et tribus noctibus.
${}^{41}$~Viri Ninivit\ae\ surgent in judicio cum generatione ista, et condemnabunt eam~: quia pœnitentiam egerunt in pr\ae dicatione Jon\ae , et ecce plus quam Jonas hic.
${}^{42}$~Regina austri surget in judicio cum generatione ista, et condemnabit eam~: quia venit a finibus terr\ae\ audire sapientiam Salomonis, et ecce plus quam Salomon hic.
${}^{43}$~Cum autem immundus spiritus exierit ab homine, ambulat per loca arida, qu\ae rens requiem, et non invenit.
${}^{44}$~Tunc dicit~: Revertar in domum meam, unde exivi. Et veniens invenit eam vacantem, scopis mundatam, et ornatam.
${}^{45}$~Tunc vadit, et assumit septem alios spiritus secum nequiores se, et intrantes habitant ibi~: et fiunt novissima hominis illius pejora prioribus. Sic erit et generationi huic pessim\ae .


${}^{46}$~Adhuc eo loquente ad turbas, ecce mater ejus et fratres stabant foris, qu\ae rentes loqui ei.
${}^{47}$~Dixit autem ei quidam~: Ecce mater tua, et fratres tui foris stant qu\ae rentes te.
${}^{48}$~At ipse respondens dicenti sibi, ait~: Qu\ae\ est mater mea, et qui sunt fratres mei~?
${}^{49}$~Et extendens manum in discipulos suos, dixit~: Ecce mater mea, et fratres mei.
${}^{50}$~Quicumque enim fecerit voluntatem Patris mei, qui in c\ae lis est, ipse meus frater, et soror, et mater est.

\bchapter
\mylettrine{I}n illo die exiens Jesus de domo, sedebat secus mare.
${}^{2}$~Et congregat\ae\ sunt ad eum turb\ae\ mult\ae , ita ut naviculam ascendens sederet~: et omnis turba stabat in littore,
${}^{3}$~et locutus est eis multa in parabolis, dicens~: Ecce exiit qui seminat, seminare.
${}^{4}$~Et dum seminat, qu\ae dam ceciderunt secus viam, et venerunt volucres c\ae li, et comederunt ea.
${}^{5}$~Alia autem ceciderunt in petrosa, ubi non habebant terram multam~: et continuo exorta sunt, quia non habebant altitudinem terr\ae~:
${}^{6}$~sole autem orto \ae stuaverunt~; et quia non habebant radicem, aruerunt.
${}^{7}$~Alia autem ceciderunt in spinas~: et creverunt spin\ae , et suffocaverunt ea.
${}^{8}$~Alia autem ceciderunt in terram bonam~: et dabant fructum, aliud centesimum, aliud sexagesimum, aliud trigesimum.
${}^{9}$~Qui habet aures audiendi, audiat.


${}^{10}$~Et accedentes discipuli dixerunt ei~: Quare in parabolis loqueris eis~?
${}^{11}$~Qui respondens, ait illis~: Quia vobis datum est nosse mysteria regni c\ae lorum~: illis autem non est datum.
${}^{12}$~Qui enim habet, dabitur ei, et abundabit~: qui autem non habet, et quod habet auferetur ab eo.
${}^{13}$~Ideo in parabolis loquor eis~: quia videntes non vident, et audientes non audiunt, neque intelligunt.
${}^{14}$~Et adimpletur in eis prophetia Isai\ae , dicentis~: \begin{flushleft}\begin{verse}Auditu audietis, et non intelligetis~:\\ et videntes videbitis, et non videbitis.\\
${}^{15}$~Incrassatum est enim cor populi hujus,\\ et auribus graviter audierunt,\\ et oculos suos clauserunt~:\\ nequando videant oculis, et auribus audiant,\\ et corde intelligant, et convertantur,\\ et sanem eos.\end{verse}\end{flushleft}


${}^{16}$~Vestri autem beati oculi quia vident, et aures vestr\ae\ quia audiunt.
${}^{17}$~Amen quippe dico vobis, quia multi prophet\ae\ et justi cupierunt videre qu\ae\ videtis, et non viderunt~: et audire qu\ae\ auditis, et non audierunt.


${}^{18}$~Vos ergo audite parabolam seminantis.
${}^{19}$~Omnis qui audit verbum regni, et non intelligit, venit malus, et rapit quod seminatum est in corde ejus~: hic est qui secus viam seminatus est.
${}^{20}$~Qui autem super petrosa seminatus est, hic est qui verbum audit, et continuo cum gaudio accipit illud~:
${}^{21}$~non habet autem in se radicem, sed est temporalis~: facta autem tribulatione et persecutione propter verbum, continuo scandalizatur.
${}^{22}$~Qui autem seminatus est in spinis, hic est qui verbum audit, et sollicitudo s\ae culi istius, et fallacia divitiarum suffocat verbum, et sine fructu efficitur.
${}^{23}$~Qui vero in terram bonam seminatus est, hic est qui audit verbum, et intelligit, et fructum affert, et facit aliud quidem centesimum, aliud autem sexagesimum, aliud vero trigesimum.


${}^{24}$~Aliam parabolam proposuit illis, dicens~: Simile factum est regnum c\ae lorum homini, qui seminavit bonum semen in agro suo~:
${}^{25}$~cum autem dormirent homines, venit inimicus ejus, et superseminavit zizania in medio tritici, et abiit.
${}^{26}$~Cum autem crevisset herba, et fructum fecisset, tunc apparuerunt et zizania.
${}^{27}$~Accedentes autem servi patrisfamilias, dixerunt ei~: Domine, nonne bonum semen seminasti in agro tuo~? unde ergo habet zizania~?
${}^{28}$~Et ait illis~: Inimicus homo hoc fecit. Servi autem dixerunt ei~: Vis, imus, et colligimus ea~?
${}^{29}$~Et ait~: Non~: ne forte colligentes zizania, eradicetis simul cum eis et triticum.
${}^{30}$~Sinite utraque crescere usque ad messem, et in tempore messis dicam messoribus~: Colligite primum zizania, et alligate ea in fasciculos ad comburendum~: triticum autem congregate in horreum meum.


${}^{31}$~Aliam parabolam proposuit eis dicens~: Simile est regnum c\ae lorum grano sinapis, quod accipiens homo seminavit in agro suo~:
${}^{32}$~quod minimum quidem est omnibus seminibus~: cum autem creverit, majus est omnibus oleribus, et fit arbor, ita ut volucres c\ae li veniant, et habitent in ramis ejus.
${}^{33}$~Aliam parabolam locutus est eis~: Simile est regnum c\ae lorum fermento, quod acceptum mulier abscondit in farin\ae\ satis tribus, donec fermentatum est totum.
${}^{34}$~H\ae c omnia locutus est Jesus in parabolis ad turbas~: et sine parabolis non loquebatur eis~:
${}^{35}$~ut impleretur quod dictum erat per prophetam dicentem~: Aperiam in parabolis os meum~; eructabo abscondita a constitutione mundi.


${}^{36}$~Tunc, dimissis turbis, venit in domum~: et accesserunt ad eum discipuli ejus, dicentes~: Edissere nobis parabolam zizaniorum agri.
${}^{37}$~Qui respondens ait illis~: Qui seminat bonum semen, est Filius hominis.
${}^{38}$~Ager autem est mundus. Bonum vero semen, hi sunt filii regni. Zizania autem, filii sunt nequam.
${}^{39}$~Inimicus autem, qui seminavit ea, est diabolus. Messis vero, consummatio s\ae culi est. Messores autem, angeli sunt.
${}^{40}$~Sicut ergo colliguntur zizania, et igni comburuntur~: sic erit in consummatione s\ae culi.
${}^{41}$~Mittet Filius hominis angelos suos, et colligent de regno ejus omnia scandala, et eos qui faciunt iniquitatem~:
${}^{42}$~et mittent eos in caminum ignis. Ibi erit fletus et stridor dentium.
${}^{43}$~Tunc justi fulgebunt sicut sol in regno Patris eorum. Qui habet aures audiendi, audiat.


${}^{44}$~Simile est regnum c\ae lorum thesauro abscondito in agro~: quem qui invenit homo, abscondit, et pr\ae\ gaudio illius vadit, et vendit universa qu\ae\ habet, et emit agrum illum.
${}^{45}$~Iterum simile est regnum c\ae lorum homini negotiatori, qu\ae renti bonas margaritas.
${}^{46}$~Inventa autem una pretiosa margarita, abiit, et vendidit omnia qu\ae\ habuit, et emit eam.


${}^{47}$~Iterum simile est regnum c\ae lorum sagen\ae\ miss\ae\ in mare, et ex omni genere piscium congreganti.
${}^{48}$~Quam, cum impleta esset, educentes, et secus littus sedentes, elegerunt bonos in vasa, malos autem foras miserunt.
${}^{49}$~Sic erit in consummatione s\ae culi~: exibunt angeli, et separabunt malos de medio justorum,
${}^{50}$~et mittent eos in caminum ignis~: ibi erit fletus, et stridor dentium.
${}^{51}$~Intellexistis h\ae c omnia~? Dicunt ei~: Etiam.
${}^{52}$~Ait illis~: Ideo omnis scriba doctus in regno c\ae lorum, similis est homini patrifamilias, qui profert de thesauro suo nova et vetera.
${}^{53}$~Et factum est, cum consummasset Jesus parabolas istas, transiit inde.


${}^{54}$~Et veniens in patriam suam, docebat eos in synagogis eorum, ita ut mirarentur, et dicerent~: Unde huic sapientia h\ae c, et virtutes~?
${}^{55}$~Nonne hic est fabri filius~? nonne mater ejus dicitur Maria, et fratres ejus, Jacobus, et Joseph, et Simon, et Judas~?
${}^{56}$~et sorores ejus, nonne omnes apud nos sunt~? unde ergo huic omnia ista~?
${}^{57}$~Et scandalizabantur in eo. Jesus autem dixit eis~: Non est propheta sine honore, nisi in patria sua, et in domo sua.
${}^{58}$~Et non fecit ibi virtutes multas propter incredulitatem illorum.

\bchapter
\mylettrine{I}n illo tempore audivit Herodes tetrarcha famam Jesu~:
${}^{2}$~et ait pueris suis~: Hic est Joannes Baptista~: ipse surrexit a mortuis, et ideo virtutes operantur in eo.
${}^{3}$~Herodes enim tenuit Joannem, et alligavit eum~: et posuit in carcerem propter Herodiadem uxorem fratris sui.
${}^{4}$~Dicebat enim illi Joannes~: Non licet tibi habere eam.
${}^{5}$~Et volens illum occidere, timuit populum~: quia sicut prophetam eum habebant.
${}^{6}$~Die autem natalis Herodis saltavit filia Herodiadis in medio, et placuit Herodi~:
${}^{7}$~unde cum juramento pollicitus est ei dare quodcumque postulasset ab eo.
${}^{8}$~At illa pr\ae monita a matre sua~: Da mihi, inquit, hic in disco caput Joannis Baptist\ae .
${}^{9}$~Et contristatus est rex~: propter juramentum autem, et eos qui pariter recumbebant, jussit dari.
${}^{10}$~Misitque et decollavit Joannem in carcere.
${}^{11}$~Et allatum est caput ejus in disco, et datum est puell\ae , et attulit matri su\ae .
${}^{12}$~Et accedentes discipuli ejus, tulerunt corpus ejus, et sepelierunt illud~: et venientes nuntiaverunt Jesu.


${}^{13}$~Quod cum audisset Jesus, secessit inde in navicula, in locum desertum seorsum~: et cum audissent turb\ae , secut\ae\ sunt eum pedestres de civitatibus.
${}^{14}$~Et exiens vidit turbam multam, et misertus est eis, et curavit languidos eorum.
${}^{15}$~Vespere autem facto, accesserunt ad eum discipuli ejus, dicentes~: Desertus est locus, et hora jam pr\ae teriit~: dimitte turbas, ut euntes in castella, emant sibi escas.
${}^{16}$~Jesus autem dixit eis~: Non habent necesse ire~: date illis vos manducare.
${}^{17}$~Responderunt ei~: Non habemus hic nisi quinque panes et duos pisces.
${}^{18}$~Qui ait eis~: Afferte mihi illos huc.
${}^{19}$~Et cum jussisset turbam discumbere super fœnum, acceptis quinque panibus et duobus piscibus, aspiciens in c\ae lum benedixit, et fregit, et dedit discipulis panes, discipuli autem turbis.
${}^{20}$~Et manducaverunt omnes, et saturati sunt. Et tulerunt reliquias, duodecim cophinos fragmentorum plenos.
${}^{21}$~Manducantium autem fuit numerus quinque millia virorum, exceptis mulieribus et parvulis.


${}^{22}$~Et statim compulit Jesus discipulos ascendere in naviculam, et pr\ae cedere eum trans fretum, donec dimitteret turbas.
${}^{23}$~Et dimissa turba, ascendit in montem solus orare. Vespere autem facto solus erat ibi~:
${}^{24}$~navicula autem in medio mari jactabatur fluctibus~: erat enim contrarius ventus.
${}^{25}$~Quarta enim vigilia noctis, venit ad eos ambulans super mare.
${}^{26}$~Et videntes eum super mare ambulantem, turbati sunt, dicentes~: Quia phantasma est. Et pr\ae\ timore clamaverunt.
${}^{27}$~Statimque Jesus locutus est eis, dicens~: Habete fiduciam~: ego sum, nolite timere.
${}^{28}$~Respondens autem Petrus, dixit~: Domine, si tu es, jube me ad te venire super aquas.
${}^{29}$~At ipse ait~: Veni. Et descendens Petrus de navicula, ambulabat super aquam ut veniret ad Jesum.
${}^{30}$~Videns vero ventum validum, timuit~: et cum cœpisset mergi, clamavit dicens~: Domine, salvum me fac.
${}^{31}$~Et continuo Jesus extendens manum, apprehendit eum~: et ait illi~: Modic\ae\ fidei, quare dubitasti~?
${}^{32}$~Et cum ascendissent in naviculam, cessavit ventus.
${}^{33}$~Qui autem in navicula erant, venerunt, et adoraverunt eum, dicentes~: Vere Filius Dei es.
${}^{34}$~Et cum transfretassent, venerunt in terram Genesar.
${}^{35}$~Et cum cognovissent eum viri loci illius, miserunt in universam regionem illam, et obtulerunt ei omnes male habentes~:
${}^{36}$~et rogabant eum ut vel fimbriam vestimenti ejus tangerent. Et quicumque tetigerunt, salvi facti sunt.

\bchapter
\mylettrine{T}unc accesserunt ad eum ab Jerosolymis scrib\ae\ et pharis\ae i, dicentes~:
${}^{2}$~Quare discipuli tui transgrediuntur traditionem seniorum~? non enim lavant manus suas cum panem manducant.
${}^{3}$~Ipse autem respondens ait illis~: Quare et vos transgredimini mandatum Dei propter traditionem vestram~? Nam Deus dixit~:
${}^{4}$~Honora patrem, et matrem~: et, Qui maledixerit patri, vel matri, morte moriatur.
${}^{5}$~Vos autem dicitis~: Quicumque dixerit patri, vel matri~: Munus, quodcumque est ex me, tibi proderit~:
${}^{6}$~et non honorificabit patrem suum, aut matrem suam~: et irritum fecistis mandatum Dei propter traditionem vestram.
${}^{7}$~Hypocrit\ae , bene prophetavit de vobis Isaias, dicens~:
\begin{flushleft}\begin{verse}${}^{8}$~Populus hic labiis me honorat~:\\ cor autem eorum longe est a me.\\
${}^{9}$~Sine causa autem colunt me,\\ docentes doctrinas et mandata hominum.\end{verse}\end{flushleft}


${}^{10}$~Et convocatis ad se turbis, dixit eis~: Audite, et intelligite.
${}^{11}$~Non quod intrat in os, coinquinat hominem~: sed quod procedit ex ore, hoc coinquinat hominem.
${}^{12}$~Tunc accedentes discipuli ejus, dixerunt ei~: Scis quia pharis\ae i audito verbo hoc, scandalizati sunt~?
${}^{13}$~At ille respondens ait~: Omnis plantatio, quam non plantavit Pater meus c\ae lestis, eradicabitur.
${}^{14}$~Sinite illos~: c\ae ci sunt, et duces c\ae corum~; c\ae cus autem si c\ae co ducatum pr\ae stet, ambo in foveam cadunt.
${}^{15}$~Respondens autem Petrus dixit ei~: Edissere nobis parabolam istam.
${}^{16}$~At ille dixit~: Adhuc et vos sine intellectu estis~?
${}^{17}$~Non intelligitis quia omne quod in os intrat, in ventrem vadit, et in secessum emittitur~?
${}^{18}$~Qu\ae\ autem procedunt de ore, de corde exeunt, et ea coinquinant hominem~:
${}^{19}$~de corde enim exeunt cogitationes mal\ae , homicidia, adulteria, fornicationes, furta, falsa testimonia, blasphemi\ae~:
${}^{20}$~h\ae c sunt, qu\ae\ coinquinant hominem. Non lotis autem manibus manducare, non coinquinat hominem.


${}^{21}$~Et egressus inde Jesus secessit in partes Tyri et Sidonis.
${}^{22}$~Et ecce mulier chanan\ae a a finibus illis egressa clamavit, dicens ei~: Miserere mei, Domine fili David~: filia mea male a d\ae monio vexatur.
${}^{23}$~Qui non respondit ei verbum. Et accedentes discipuli ejus rogabant eum dicentes~: Dimitte eam~: quia clamat post nos.
${}^{24}$~Ipse autem respondens ait~: Non sum missus nisi ad oves, qu\ae\ perierunt domus Isra\"el.
${}^{25}$~At illa venit, et adoravit eum, dicens~: Domine, adjuva me.
${}^{26}$~Qui respondens ait~: Non est bonum sumere panem filiorum, et mittere canibus.
${}^{27}$~At illa dixit~: Etiam Domine~: nam et catelli edunt de micis qu\ae\ cadunt de mensa dominorum suorum.
${}^{28}$~Tunc respondens Jesus, ait illi~: O mulier, magna est fides tua~: fiat tibi sicut vis. Et sanata est filia ejus ex illa hora.


${}^{29}$~Et cum transisset inde Jesus, venit secus mare Galil\ae \ae~: et ascendens in montem, sedebat ibi.
${}^{30}$~Et accesserunt ad eum turb\ae\ mult\ae , habentes secum mutos, c\ae cos, claudos, debiles, et alios multos~: et projecerunt eos ad pedes ejus, et curavit eos,
${}^{31}$~ita ut turb\ae\ mirarentur, videntes mutos loquentes, claudos ambulantes, c\ae cos videntes~: et magnificabant Deum Isra\"el.
${}^{32}$~Jesus autem, convocatis discipulis suis, dixit~: Misereor turb\ae , quia triduo jam perseverant mecum, et non habent quod manducent~: et dimittere eos jejunos nolo, ne deficiant in via.
${}^{33}$~Et dicunt ei discipuli~: Unde ergo nobis in deserto panes tantos, ut saturemus turbam tantam~?
${}^{34}$~Et ait illis Jesus~: Quot habetis panes~? At illi dixerunt~: Septem, et paucos pisciculos.
${}^{35}$~Et pr\ae cepit turb\ae\ ut discumberent super terram.
${}^{36}$~Et accipiens septem panes, et pisces, et gratias agens, fregit, et dedit discipulis suis, et discipuli dederunt populo.
${}^{37}$~Et comederunt omnes, et saturati sunt. Et quod superfuit de fragmentis, tulerunt septem sportas plenas.
${}^{38}$~Erant autem qui manducaverunt quatuor millia hominum, extra parvulos et mulieres.
${}^{39}$~Et, dimissa turba, ascendit in naviculam~: et venit in fines Magedan.

\bchapter
\mylettrine{E}t accesserunt ad eum pharis\ae i et sadduc\ae i tentantes~: et rogaverunt eum ut signum de c\ae lo ostenderet eis.
${}^{2}$~At ille respondens, ait illis~: Facto vespere dicitis~: Serenum erit, rubicundum est enim c\ae lum.
${}^{3}$~Et mane~: Hodie tempestas, rutilat enim triste c\ae lum.
${}^{4}$~Faciem ergo c\ae li dijudicare nostis~: signa autem temporum non potestis scire~? Generatio mala et adultera signum qu\ae rit~: et signum non dabitur ei, nisi signum Jon\ae\ prophet\ae . Et relictis illis, abiit.


${}^{5}$~Et cum venissent discipuli ejus trans fretum, obliti sunt panes accipere.
${}^{6}$~Qui dixit illis~: Intuemini, et cavete a fermento pharis\ae orum et sadduc\ae orum.
${}^{7}$~At illi cogitabant intra se dicentes~: Quia panes non accepimus.
${}^{8}$~Sciens autem Jesus, dixit~: Quid cogitatis intra vos modic\ae\ fidei, quia panes non habetis~?
${}^{9}$~Nondum intelligitis, neque recordamini quinque panum in quinque millia hominum, et quot cophinos sumpsistis~?
${}^{10}$~neque septem panum in quatuor millia hominum, et quot sportas sumpsistis~?
${}^{11}$~Quare non intelligitis, quia non de pane dixi vobis~: Cavete a fermento pharis\ae orum et sadduc\ae orum~?
${}^{12}$~Tunc intellexerunt quia non dixerit cavendum a fermento panum, sed a doctrina pharis\ae orum et sadduc\ae orum.


${}^{13}$~Venit autem Jesus in partes C\ae sare\ae\ Philippi~: et interrogabat discipulos suos, dicens~: Quem dicunt homines esse Filium hominis~?
${}^{14}$~At illi dixerunt~: Alii Joannem Baptistam, alii autem Eliam, alii vero Jeremiam, aut unum ex prophetis.
${}^{15}$~Dicit illis Jesus~: Vos autem, quem me esse dicitis~?
${}^{16}$~Respondens Simon Petrus dixit~: Tu es Christus, Filius Dei vivi.
${}^{17}$~Respondens autem Jesus, dixit ei~: Beatus es Simon Bar Jona~: quia caro et sanguis non revelavit tibi, sed Pater meus, qui in c\ae lis est.
${}^{18}$~Et ego dico tibi, quia tu es Petrus, et super hanc petram \ae dificabo Ecclesiam meam, et port\ae\ inferi non pr\ae valebunt adversus eam.
${}^{19}$~Et tibi dabo claves regni c\ae lorum. Et quodcumque ligaveris super terram, erit ligatum et in c\ae lis~: et quodcumque solveris super terram, erit solutum et in c\ae lis.
${}^{20}$~Tunc pr\ae cepit discipulis suis ut nemini dicerent quia ipse esset Jesus Christus.


${}^{21}$~Exinde cœpit Jesus ostendere discipulis suis, quia oporteret eum ire Jerosolymam, et multa pati a senioribus, et scribis, et principibus sacerdotum, et occidi, et tertia die resurgere.
${}^{22}$~Et assumens eum Petrus, cœpit increpare illum dicens~: Absit a te, Domine~: non erit tibi hoc.
${}^{23}$~Qui conversus, dixit Petro~: Vade post me Satana, scandalum es mihi~: quia non sapis ea qu\ae\ Dei sunt, sed ea qu\ae\ hominum.
${}^{24}$~Tunc Jesus dixit discipulis suis~: Si quis vult post me venire, abneget semetipsum, et tollat crucem suam, et sequatur me.
${}^{25}$~Qui enim voluerit animam suam salvam facere, perdet eam~: qui autem perdiderit animam suam propter me, inveniet eam.
${}^{26}$~Quid enim prodest homini, si mundum universum lucretur, anim\ae\ vero su\ae\ detrimentum patiatur~? aut quam dabit homo commutationem pro anima sua~?
${}^{27}$~Filius enim hominis venturus est in gloria Patris sui cum angelis suis~: et tunc reddet unicuique secundum opera ejus.
${}^{28}$~Amen dico vobis, sunt quidam de hic stantibus, qui non gustabunt mortem, donec videant Filium hominis venientem in regno suo.

\bchapter
\mylettrine{E}t post dies sex assumit Jesus Petrum, et Jacobum, et Joannem fratrem ejus, et ducit illos in montem excelsum seorsum~:
${}^{2}$~et transfiguratus est ante eos. Et resplenduit facies ejus sicut sol~: vestimenta autem ejus facta sunt alba sicut nix.
${}^{3}$~Et ecce apparuerunt illis Moyses et Elias cum eo loquentes.
${}^{4}$~Respondens autem Petrus, dixit ad Jesum~: Domine, bonum est nos hic esse~: si vis, faciamus tria tabernacula, tibi unum, Moysi unum, et Eli\ae\ unum.
${}^{5}$~Adhuc eo loquente, ecce nubes lucida obumbravit eos. Et ecce vox de nube, dicens~: Hic est Filius meus dilectus, in quo mihi bene complacui~: ipsum audite.
${}^{6}$~Et audientes discipuli ceciderunt in faciem suam, et timuerunt valde.
${}^{7}$~Et accessit Jesus, et tetigit eos~: dixitque eis~: Surgite, et nolite timere.
${}^{8}$~Levantes autem oculos suos, neminem viderunt, nisi solum Jesum.
${}^{9}$~Et descendentibus illis de monte, pr\ae cepit eis Jesus, dicens~: Nemini dixeritis visionem, donec Filius hominis a mortuis resurgat.
${}^{10}$~Et interrogaverunt eum discipuli, dicentes~: Quid ergo scrib\ae\ dicunt, quod Eliam oporteat primum venire~?
${}^{11}$~At ille respondens, ait eis~: Elias quidem venturus est, et restituet omnia.
${}^{12}$~Dico autem vobis, quia Elias jam venit, et non cognoverunt eum, sed fecerunt in eo qu\ae cumque voluerunt. Sic et Filius hominis passurus est ab eis.
${}^{13}$~Tunc intellexerunt discipuli, quia de Joanne Baptista dixisset eis.


${}^{14}$~Et cum venisset ad turbam, accessit ad eum homo genibus provolutus ante eum, dicens~: Domine, miserere filio meo, quia lunaticus est, et male patitur~: nam s\ae pe cadit in ignem, et crebro in aquam.
${}^{15}$~Et obtuli eum discipulis tuis, et non potuerunt curare eum.
${}^{16}$~Respondens autem Jesus, ait~: O generatio incredula, et perversa, quousque ero vobiscum~? usquequo patiar vos~? Afferte huc illum ad me.
${}^{17}$~Et increpavit illum Jesus, et exiit ab eo d\ae monium, et curatus est puer ex illa hora.
${}^{18}$~Tunc accesserunt discipuli ad Jesum secreto, et dixerunt~: Quare nos non potuimus ejicere illum~?
${}^{19}$~Dixit illis Jesus~: Propter incredulitatem vestram. Amen quippe dico vobis, si habueritis fidem sicut granum sinapis, dicetis monti huic~: Transi hinc illuc, et transibit, et nihil impossibile erit vobis.
${}^{20}$~Hoc autem genus non ejicitur nisi per orationem et jejunium.


${}^{21}$~Conversantibus autem eis in Galil\ae a, dixit illis Jesus~: Filius hominis tradendus est in manus hominum~:
${}^{22}$~et occident eum, et tertia die resurget. Et contristati sunt vehementer.


${}^{23}$~Et cum venissent Capharnaum, accesserunt qui didrachma accipiebant ad Petrum, et dixerunt ei~: Magister vester non solvit didrachma~?
${}^{24}$~Ait~: Etiam. Et cum intrasset in domum, pr\ae venit eum Jesus, dicens~: Quid tibi videtur Simon~? reges terr\ae\ a quibus accipiunt tributum vel censum~? a filiis suis, an ab alienis~?
${}^{25}$~Et ille dixit~: Ab alienis. Dixit illi Jesus~: Ergo liberi sunt filii.
${}^{26}$~Ut autem non scandalizemus eos, vade ad mare, et mitte hamum~: et eum piscem, qui primus ascenderit, tolle~: et aperto ore ejus, invenies staterem~: illum sumens, da eis pro me et te.

\bchapter
\mylettrine{I}n illa hora accesserunt discipuli ad Jesum, dicentes~: Quis, putas, major est in regno c\ae lorum~?
${}^{2}$~Et advocans Jesus parvulum, statuit eum in medio eorum,
${}^{3}$~et dixit~: Amen dico vobis, nisi conversi fueritis, et efficiamini sicut parvuli, non intrabitis in regnum c\ae lorum.
${}^{4}$~Quicumque ergo humiliaverit se sicut parvulus iste, hic est major in regno c\ae lorum.
${}^{5}$~Et qui susceperit unum parvulum talem in nomine meo, me suscipit~:
${}^{6}$~qui autem scandalizaverit unum de pusillis istis, qui in me credunt, expedit ei ut suspendatur mola asinaria in collo ejus, et demergatur in profundum maris.
${}^{7}$~V\ae\ mundo a scandalis~! Necesse est enim ut veniant scandala~: verumtamen v\ae\ homini illi, per quem scandalum venit.
${}^{8}$~Si autem manus tua, vel pes tuus scandalizat te, abscide eum, et projice abs te~: bonum tibi est ad vitam ingredi debilem, vel claudum, quam duas manus vel duos pedes habentem mitti in ignem \ae ternum.
${}^{9}$~Et si oculus tuus scandalizat te, erue eum, et projice abs te~: bonum tibi est cum uno oculo in vitam intrare, quam duos oculos habentem mitti in gehennam ignis.
${}^{10}$~Videte ne contemnatis unum ex his pusillis~: dico enim vobis, quia angeli eorum in c\ae lis semper vident faciem Patris mei, qui in c\ae lis est.
${}^{11}$~Venit enim Filius hominis salvare quod perierat.
${}^{12}$~Quid vobis videtur~? si fuerint alicui centum oves, et erravit una ex eis~: nonne relinquit nonaginta novem in montibus, et vadit qu\ae rere eam qu\ae\ erravit~?
${}^{13}$~Et si contigerit ut inveniat eam~: amen dico vobis, quia gaudet super eam magis quam super nonaginta novem, qu\ae\ non erraverunt.
${}^{14}$~Sic non est voluntas ante Patrem vestrum, qui in c\ae lis est, ut pereat unus de pusillis istis.


${}^{15}$~Si autem peccaverit in te frater tuus, vade, et corripe eum inter te, et ipsum solum~: si te audierit, lucratus eris fratrem tuum.
${}^{16}$~Si autem te non audierit, adhibe tecum adhuc unum, vel duos, ut in ore duorum, vel trium testium stet omne verbum.
${}^{17}$~Quod si non audierit eos~: dic ecclesi\ae . Si autem ecclesiam non audierit, sit tibi sicut ethnicus et publicanus.
${}^{18}$~Amen dico vobis, qu\ae cumque alligaveritis super terram, erunt ligata et in c\ae lo~: et qu\ae cumque solveritis super terram, erunt soluta et in c\ae lo.
${}^{19}$~Iterum dico vobis, quia si duo ex vobis consenserint super terram, de omni re quamcumque petierint, fiet illis a Patre meo, qui in c\ae lis est.
${}^{20}$~Ubi enim sunt duo vel tres congregati in nomine meo, ibi sum in medio eorum.


${}^{21}$~Tunc accedens Petrus ad eum, dixit~: Domine, quoties peccabit in me frater meus, et dimittam ei~? usque septies~?
${}^{22}$~Dicit illi Jesus~: Non dico tibi usque septies~: sed usque septuagies septies.
${}^{23}$~Ideo assimilatum est regnum c\ae lorum homini regi, qui voluit rationem ponere cum servis suis.
${}^{24}$~Et cum cœpisset rationem ponere, oblatus est ei unus, qui debebat ei decem millia talenta.
${}^{25}$~Cum autem non haberet unde redderet, jussit eum dominus ejus venundari, et uxorem ejus, et filios, et omnia qu\ae\ habebat, et reddi.
${}^{26}$~Procidens autem servus ille, orabat eum, dicens~: Patientiam habe in me, et omnia reddam tibi.
${}^{27}$~Misertus autem dominus servi illius, dimisit eum, et debitum dimisit ei.
${}^{28}$~Egressus autem servus ille invenit unum de conservis suis, qui debebat ei centum denarios~: et tenens suffocavit eum, dicens~: Redde quod debes.
${}^{29}$~Et procidens conservus ejus, rogabat eum, dicens~: Patientiam habe in me, et omnia reddam tibi.
${}^{30}$~Ille autem noluit~: sed abiit, et misit eum in carcerem donec redderet debitum.
${}^{31}$~Videntes autem conservi ejus qu\ae\ fiebant, contristati sunt valde~: et venerunt, et narraverunt domino suo omnia qu\ae\ facta fuerant.
${}^{32}$~Tunc vocavit illum dominus suus~: et ait illi~: Serve nequam, omne debitum dimisi tibi quoniam rogasti me~:
${}^{33}$~nonne ergo oportuit et te misereri conservi tui, sicut et ego tui misertus sum~?
${}^{34}$~Et iratus dominus ejus tradidit eum tortoribus, quoadusque redderet universum debitum.
${}^{35}$~Sic et Pater meus c\ae lestis faciet vobis, si non remiseritis unusquisque fratri suo de cordibus vestris.

\bchapter
\mylettrine{E}t factum est, cum consummasset Jesus sermones istos, migravit a Galil\ae a, et venit in fines Jud\ae \ae\ trans Jordanem,
${}^{2}$~et secut\ae\ sunt eum turb\ae\ mult\ae , et curavit eos ibi.
${}^{3}$~Et accesserunt ad eum pharis\ae i tentantes eum, et dicentes~: Si licet homini dimittere uxorem suam, quacumque ex causa~?
${}^{4}$~Qui respondens, ait eis~: Non legistis, quia qui fecit hominem ab initio, masculum et feminam fecit eos~? Et dixit~:
${}^{5}$~Propter hoc dimittet homo patrem, et matrem, et adh\ae rebit uxori su\ae , et erunt duo in carne una.
${}^{6}$~Itaque jam non sunt duo, sed una caro. Quod ergo Deus conjunxit, homo non separet.
${}^{7}$~Dicunt illi~: Quid ergo Moyses mandavit dare libellum repudii, et dimittere~?
${}^{8}$~Ait illis~: Quoniam Moyses ad duritiam cordis vestri permisit vobis dimittere uxores vestras~: ab initio autem non fuit sic.
${}^{9}$~Dico autem vobis, quia quicumque dimiserit uxorem suam, nisi ob fornicationem, et aliam duxerit, mœchatur~: et qui dimissam duxerit, mœchatur.
${}^{10}$~Dicunt ei discipuli ejus~: Si ita est causa hominis cum uxore, non expedit nubere.
${}^{11}$~Qui dixit illis~: Non omnes capiunt verbum istud, sed quibus datum est.
${}^{12}$~Sunt enim eunuchi, qui de matris utero sic nati sunt~: et sunt eunuchi, qui facti sunt ab hominibus~: et sunt eunuchi, qui seipsos castraverunt propter regnum c\ae lorum. Qui potest capere capiat.


${}^{13}$~Tunc oblati sunt ei parvuli, ut manus eis imponeret, et oraret. Discipuli autem increpabant eos.
${}^{14}$~Jesus vero ait eis~: Sinite parvulos, et nolite eos prohibere ad me venire~: talium est enim regnum c\ae lorum.
${}^{15}$~Et cum imposuisset eis manus, abiit inde.


${}^{16}$~Et ecce unus accedens, ait illi~: Magister bone, quid boni faciam ut habeam vitam \ae ternam~?
${}^{17}$~Qui dixit ei~: Quid me interrogas de bono~? Unus est bonus, Deus. Si autem vis ad vitam ingredi, serva mandata.
${}^{18}$~Dicit illi~: Qu\ae~? Jesus autem dixit~: Non homicidium facies~; non adulterabis~; non facies furtum~; non falsum testimonium dices~;
${}^{19}$~honora patrem tuum, et matrem tuam, et diliges proximum tuum sicut teipsum.
${}^{20}$~Dicit illi adolescens~: Omnia h\ae c custodivi a juventute mea~: quid adhuc mihi deest~?
${}^{21}$~Ait illi Jesus~: Si vis perfectus esse, vade, vende qu\ae\ habes, et da pauperibus, et habebis thesaurum in c\ae lo~: et veni, sequere me.
${}^{22}$~Cum audisset autem adolescens verbum, abiit tristis~: erat enim habens multas possessiones.
${}^{23}$~Jesus autem dixit discipulis suis~: Amen dico vobis, quia dives difficile intrabit in regnum c\ae lorum.
${}^{24}$~Et iterum dico vobis~: Facilius est camelum per foramen acus transire, quam divitem intrare in regnum c\ae lorum.
${}^{25}$~Auditis autem his, discipuli mirabantur valde, dicentes~: Quis ergo poterit salvus esse~?
${}^{26}$~Aspiciens autem Jesus, dixit illis~: Apud homines hoc impossibile est~: apud Deum autem omnia possibilia sunt.


${}^{27}$~Tunc respondens Petrus, dixit ei~: Ecce nos reliquimus omnia, et secuti sumus te~: quid ergo erit nobis~?
${}^{28}$~Jesus autem dixit illis~: Amen dico vobis, quod vos, qui secuti estis me, in regeneratione cum sederit Filius hominis in sede majestatis su\ae , sedebitis et vos super sedes duodecim, judicantes duodecim tribus Isra\"el.
${}^{29}$~Et omnis qui reliquerit domum, vel fratres, aut sorores, aut patrem, aut matrem, aut uxorem, aut filios, aut agros propter nomen meum, centuplum accipiet, et vitam \ae ternam possidebit.
${}^{30}$~Multi autem erunt primi novissimi, et novissimi primi.

\bchapter
\mylettrine{S}imile est regnum c\ae lorum homini patrifamilias, qui exiit primo mane conducere operarios in vineam suam.
${}^{2}$~Conventione autem facta cum operariis ex denario diurno, misit eos in vineam suam.
${}^{3}$~Et egressus circa horam tertiam, vidit alios stantes in foro otiosos,
${}^{4}$~et dixit illis~: Ite et vos in vineam meam, et quod justum fuerit dabo vobis.
${}^{5}$~Illi autem abierunt. Iterum autem exiit circa sextam et nonam horam~: et fecit similiter.
${}^{6}$~Circa undecimam vero exiit, et invenit alios stantes, et dicit illis~: Quid hic statis tota die otiosi~?
${}^{7}$~Dicunt ei~: Quia nemo nos conduxit. Dicit illis~: Ite et vos in vineam meam.
${}^{8}$~Cum sero autem factum esset, dicit dominus vine\ae\ procuratori suo~: Voca operarios, et redde illis mercedem incipiens a novissimis usque ad primos.
${}^{9}$~Cum venissent ergo qui circa undecimam horam venerant, acceperunt singulos denarios.
${}^{10}$~Venientes autem et primi, arbitrati sunt quod plus essent accepturi~: acceperunt autem et ipsi singulos denarios.
${}^{11}$~Et accipientes murmurabant adversus patremfamilias,
${}^{12}$~dicentes~: Hi novissimi una hora fecerunt, et pares illos nobis fecisti, qui portavimus pondus diei, et \ae stus.
${}^{13}$~At ille respondens uni eorum, dixit~: Amice, non facio tibi injuriam~: nonne ex denario convenisti mecum~?
${}^{14}$~Tolle quod tuum est, et vade~: volo autem et huic novissimo dare sicut et tibi.
${}^{15}$~Aut non licet mihi quod volo, facere~? an oculus tuus nequam est, quia ego bonus sum~?
${}^{16}$~Sic erunt novissimi primi, et primi novissimi. Multi enim sunt vocati, pauci vero electi.


${}^{17}$~Et ascendens Jesus Jerosolymam, assumpsit duodecim discipulos secreto, et ait illis~:
${}^{18}$~Ecce ascendimus Jerosolymam, et Filius hominis tradetur principibus sacerdotum, et scribis, et condemnabunt eum morte,
${}^{19}$~et tradent eum gentibus ad illudendum, et flagellandum, et crucifigendum, et tertia die resurget.


${}^{20}$~Tunc accessit ad eum mater filiorum Zebed\ae i cum filiis suis, adorans et petens aliquid ab eo.
${}^{21}$~Qui dixit ei~: Quid vis~? Ait illi~: Dic ut sedeant hi duo filii mei, unus ad dexteram tuam, et unus ad sinistram in regno tuo.
${}^{22}$~Respondens autem Jesus, dixit~: Nescitis quid petatis. Potestis bibere calicem, quem ego bibiturus sum~? Dicunt ei~: Possumus.
${}^{23}$~Ait illis~: Calicem quidem meum bibetis~: sedere autem ad dexteram meam vel sinistram non est meum dare vobis, sed quibus paratum est a Patre meo.
${}^{24}$~Et audientes decem, indignati sunt de duobus fratribus.
${}^{25}$~Jesus autem vocavit eos ad se, et ait~: Scitis quia principes gentium dominantur eorum~: et qui majores sunt, potestatem exercent in eos.
${}^{26}$~Non ita erit inter vos~: sed quicumque voluerit inter vos major fieri, sit vester minister~:
${}^{27}$~et qui voluerit inter vos primus esse, erit vester servus.
${}^{28}$~Sicut Filius hominis non venit ministrari, sed ministrare, et dare animam suam redemptionem pro multis.


${}^{29}$~Et egredientibus illis ab Jericho, secuta est eum turba multa,
${}^{30}$~et ecce duo c\ae ci sedentes secus viam audierunt quia Jesus transiret~: et clamaverunt, dicentes~: Domine, miserere nostri, fili David.
${}^{31}$~Turba autem increpabat eos ut tacerent. At illi magis clamabant, dicentes~: Domine, miserere nostri, fili David.
${}^{32}$~Et stetit Jesus, et vocavit eos, et ait~: Quid vultis ut faciam vobis~?
${}^{33}$~Dicunt illi~: Domine, ut aperiantur oculi nostri.
${}^{34}$~Misertus autem eorum Jesus, tetigit oculos eorum. Et confestim viderunt, et secuti sunt eum.

\bchapter
\mylettrine{E}t cum appropinquassent Jerosolymis, et venissent Bethphage ad montem Oliveti~: tunc Jesus misit duos discipulos,
${}^{2}$~dicens eis~: Ite in castellum, quod contra vos est, et statim invenietis asinam alligatam, et pullum cum ea~: solvite, et adducite mihi~:
${}^{3}$~et si quis vobis aliquid dixerit, dicite quia Dominus his opus habet~: et confestim dimittet eos.
${}^{4}$~Hoc autem totum factum est, ut adimpleretur quod dictum est per prophetam dicentem~:
\begin{flushleft}\begin{verse}${}^{5}$~Dicite fili\ae\ Sion~:\\ Ecce rex tuus venit tibi\\ mansuetus, sedens super asinam,\\ et pullum filium subjugalis.\end{verse}\end{flushleft}


${}^{6}$~Euntes autem discipuli fecerunt sicut pr\ae cepit illis Jesus.
${}^{7}$~Et adduxerunt asinam, et pullum~: et imposuerunt super eos vestimenta sua, et eum desuper sedere fecerunt.
${}^{8}$~Plurima autem turba straverunt vestimenta sua in via~: alii autem c\ae debant ramos de arboribus, et sternebant in via~:
${}^{9}$~turb\ae\ autem, qu\ae\ pr\ae cedebant, et qu\ae\ sequebantur, clamabant, dicentes~: Hosanna filio David~: benedictus, qui venit in nomine Domini~: hosanna in altissimis.
${}^{10}$~Et cum intrasset Jerosolymam, commota est universa civitas, dicens~: Quis est hic~?
${}^{11}$~Populi autem dicebant~: Hic est Jesus propheta a Nazareth Galil\ae \ae .
${}^{12}$~Et intravit Jesus in templum Dei, et ejiciebat omnes vendentes et ementes in templo, et mensas numulariorum, et cathedras vendentium columbas evertit~:
${}^{13}$~et dicit eis~: Scriptum est~: Domus mea domus orationis vocabitur~: vos autem fecistis illam speluncam latronum.
${}^{14}$~Et accesserunt ad eum c\ae ci, et claudi in templo~: et sanavit eos.
${}^{15}$~Videntes autem principes sacerdotum et scrib\ae\ mirabilia qu\ae\ fecit, et pueros clamantes in templo, et dicentes~: Hosanna filio David~: indignati sunt,
${}^{16}$~et dixerunt ei~: Audis quid isti dicunt~? Jesus autem dixit eis~: Utique. Numquam legistis~: Quia ex ore infantium et lactentium perfecisti laudem~?
${}^{17}$~Et relictis illis, abiit foras extra civitatem in Bethaniam~: ibique mansit.


${}^{18}$~Mane autem revertens in civitatem, esuriit.
${}^{19}$~Et videns fici arborem unam secus viam, venit ad eam~: et nihil invenit in ea nisi folia tantum, et ait illi~: Numquam ex te fructus nascatur in sempiternum. Et arefacta est continuo ficulnea.
${}^{20}$~Et videntes discipuli, mirati sunt, dicentes~: Quomodo continuo aruit~?
${}^{21}$~Respondens autem Jesus, ait eis~: Amen dico vobis, si habueritis fidem, et non h\ae sitaveritis, non solum de ficulnea facietis, sed et si monti huic dixeritis~: Tolle, et jacta te in mare, fiet.
${}^{22}$~Et omnia qu\ae cumque petieritis in oratione credentes, accipietis.


${}^{23}$~Et cum venisset in templum, accesserunt ad eum docentem principes sacerdotum, et seniores populi, dicentes~: In qua potestate h\ae c facis~? et quis tibi dedit hanc potestatem~?
${}^{24}$~Respondens Jesus dixit eis~: Interrogabo vos et ego unum sermonem~: quem si dixeritis mihi, et ego vobis dicam in qua potestate h\ae c facio.
${}^{25}$~Baptismus Joannis unde erat~? e c\ae lo, an ex hominibus~? At illi cogitabant inter se, dicentes~:
${}^{26}$~Si dixerimus, e c\ae lo, dicet nobis~: Quare ergo non credidistis illi~? Si autem dixerimus, ex hominibus, timemus turbam~: omnes enim habebant Joannem sicut prophetam.
${}^{27}$~Et respondentes Jesu, dixerunt~: Nescimus. Ait illis et ipse~: Nec ego dico vobis in qua potestate h\ae c facio.


${}^{28}$~Quid autem vobis videtur~? Homo quidam habebat duos filios, et accedens ad primum, dixit~: Fili, vade hodie, operare in vinea mea.
${}^{29}$~Ille autem respondens, ait~: Nolo. Postea autem, pœnitentia motus, abiit.
${}^{30}$~Accedens autem ad alterum, dixit similiter. At ille respondens, ait~: Eo, domine, et non ivit~:
${}^{31}$~quis ex duobus fecit voluntatem patris~? Dicunt ei~: Primus. Dicit illis Jesus~: Amen dico vobis, quia publicani et meretrices pr\ae cedent vos in regnum Dei.
${}^{32}$~Venit enim ad vos Joannes in via justiti\ae , et non credidistis ei~: publicani autem et meretrices crediderunt ei~: vos autem videntes nec pœnitentiam habuistis postea, ut crederetis ei.


${}^{33}$~Aliam parabolam audite~: Homo erat paterfamilias, qui plantavit vineam, et sepem circumdedit ei, et fodit in ea torcular, et \ae dificavit turrim, et locavit eam agricolis, et peregre profectus est.
${}^{34}$~Cum autem tempus fructuum appropinquasset, misit servos suos ad agricolas, ut acciperent fructus ejus.
${}^{35}$~Et agricol\ae , apprehensis servis ejus, alium ceciderunt, alium occiderunt, alium vero lapidaverunt.
${}^{36}$~Iterum misit alios servos plures prioribus, et fecerunt illis similiter.
${}^{37}$~Novissime autem misit ad eos filium suum, dicens~: Verebuntur filium meum.
${}^{38}$~Agricol\ae\ autem videntes filium dixerunt intra se~: Hic est h\ae res, venite, occidamus eum, et habebimus h\ae reditatem ejus.
${}^{39}$~Et apprehensum eum ejecerunt extra vineam, et occiderunt.
${}^{40}$~Cum ergo venerit dominus vine\ae , quid faciet agricolis illis~?
${}^{41}$~Aiunt illi~: Malos male perdet~: et vineam suam locabit aliis agricolis, qui reddant ei fructum temporibus suis.
${}^{42}$~Dicit illis Jesus~: Numquam legistis in Scripturis~: \begin{flushleft}\begin{verse}Lapidem quem reprobaverunt \ae dificantes,\\ hic factus est in caput anguli~:\\ a Domino factum est istud,\\ et est mirabile in oculis nostris~?\end{verse}\end{flushleft}


${}^{43}$~Ideo dico vobis, quia auferetur a vobis regnum Dei, et dabitur genti facienti fructus ejus.
${}^{44}$~Et qui ceciderit super lapidem istum, confringetur~: super quem vero ceciderit, conteret eum.
${}^{45}$~Et cum audissent principes sacerdotum et pharis\ae i parabolas ejus, cognoverunt quod de ipsis diceret.
${}^{46}$~Et qu\ae rentes eum tenere, timuerunt turbas~: quoniam sicut prophetam eum habebant.

\bchapter
\mylettrine{E}t respondens Jesus, dixit iterum in parabolis eis, dicens~:
${}^{2}$~Simile factum est regnum c\ae lorum homini regi, qui fecit nuptias filio suo.
${}^{3}$~Et misit servos suos vocare invitatos ad nuptias, et nolebant venire.
${}^{4}$~Iterum misit alios servos, dicens~: Dicite invitatis~: Ecce prandium meum paravi, tauri mei et altilia occisa sunt, et omnia parata~: venite ad nuptias.
${}^{5}$~Illi autem neglexerunt~: et abierunt, alius in villam suam, alius vero ad negotiationem suam~:
${}^{6}$~reliqui vero tenuerunt servos ejus, et contumeliis affectos occiderunt.
${}^{7}$~Rex autem cum audisset, iratus est~: et missis exercitibus suis, perdidit homicidas illos, et civitatem illorum succendit.
${}^{8}$~Tunc ait servis suis~: Nupti\ae\ quidem parat\ae\ sunt, sed qui invitati erant, non fuerunt digni~:
${}^{9}$~ite ergo ad exitus viarum, et quoscumque inveneritis, vocate ad nuptias.
${}^{10}$~Et egressi servi ejus in vias, congregaverunt omnes quos invenerunt, malos et bonos~: et implet\ae\ sunt nupti\ae\ discumbentium.
${}^{11}$~Intravit autem rex ut videret discumbentes, et vidit ibi hominem non vestitum veste nuptiali.
${}^{12}$~Et ait illi~: Amice, quomodo huc intrasti non habens vestem nuptialem~? At ille obmutuit.
${}^{13}$~Tunc dicit rex ministris~: Ligatis manibus et pedibus ejus, mittite eum in tenebras exteriores~: ibi erit fletus et stridor dentium.
${}^{14}$~Multi enim sunt vocati, pauci vero electi.


${}^{15}$~Tunc abeuntes pharis\ae i, consilium inierunt ut caperent eum in sermone.
${}^{16}$~Et mittunt ei discipulos suos cum Herodianis, dicentes~: Magister, scimus quia verax es, et viam Dei in veritate doces, et non est tibi cura de aliquo~: non enim respicis personam hominum~:
${}^{17}$~dic ergo nobis quid tibi videtur, licet censum dare C\ae sari, an non~?
${}^{18}$~Cognita autem Jesus nequitia eorum, ait~: Quid me tentatis, hypocrit\ae~?
${}^{19}$~ostendite mihi numisma census. At illi obtulerunt ei denarium.
${}^{20}$~Et ait illis Jesus~: Cujus est imago h\ae c, et superscriptio~?
${}^{21}$~Dicunt ei~: C\ae saris. Tunc ait illis~: Reddite ergo qu\ae\ sunt C\ae saris, C\ae sari~: et qu\ae\ sunt Dei, Deo.
${}^{22}$~Et audientes mirati sunt, et relicto eo abierunt.


${}^{23}$~In illo die accesserunt ad eum sadduc\ae i, qui dicunt non esse resurrectionem~: et interrogaverunt eum,
${}^{24}$~dicentes~: Magister, Moyses dixit~: Si quis mortuus fuerit non habens filium, ut ducat frater ejus uxorem illius, et suscitet semen fratri suo.
${}^{25}$~Erant autem apud nos septem fratres~: et primus, uxore ducta, defunctus est~: et non habens semen, reliquit uxorem suam fratri suo.
${}^{26}$~Similiter secundus, et tertius usque ad septimum.
${}^{27}$~Novissime autem omnium et mulier defuncta est.
${}^{28}$~In resurrectione ergo cujus erit de septem uxor~? omnes enim habuerunt eam.
${}^{29}$~Respondens autem Jesus, ait illis~: Erratis nescientes Scripturas, neque virtutem Dei.
${}^{30}$~In resurrectione enim neque nubent, neque nubentur~: sed erunt sicut angeli Dei in c\ae lo.
${}^{31}$~De resurrectione autem mortuorum non legistis quod dictum est a Deo dicente vobis~:
${}^{32}$~Ego sum Deus Abraham, et Deus Isaac, et Deus Jacob~? Non est Deus mortuorum, sed viventium.
${}^{33}$~Et audientes turb\ae , mirabantur in doctrina ejus.


${}^{34}$~Pharis\ae i autem audientes quod silentium imposuisset sadduc\ae is, convenerunt in unum~:
${}^{35}$~et interrogavit eum unus ex eis legis doctor, tentans eum~:
${}^{36}$~Magister, quod est mandatum magnum in lege~?
${}^{37}$~Ait illi Jesus~: Diliges Dominum Deum tuum ex toto corde tuo, et in tota anima tua, et in tota mente tua.
${}^{38}$~Hoc est maximum, et primum mandatum.
${}^{39}$~Secundum autem simile est huic~: Diliges proximum tuum, sicut teipsum.
${}^{40}$~In his duobus mandatis universa lex pendet, et prophet\ae .


${}^{41}$~Congregatis autem pharis\ae is, interrogavit eos Jesus,
${}^{42}$~dicens~: Quid vobis videtur de Christo~? cujus filius est~? Dicunt ei~: David.
${}^{43}$~Ait illis~: Quomodo ergo David in spiritu vocat eum Dominum, dicens~:
${}^{44}$~Dixit Dominus Domino meo~: Sede a dextris meis, donec ponam inimicos tuos scabellum pedum tuorum~?
${}^{45}$~Si ergo David vocat eum Dominum, quomodo filius ejus est~?
${}^{46}$~Et nemo poterat ei respondere verbum~: neque ausus fuit quisquam ex illa die eum amplius interrogare.

\bchapter
\mylettrine{T}unc Jesus locutus est ad turbas, et ad discipulos suos,
${}^{2}$~dicens~: Super cathedram Moysi sederunt scrib\ae\ et pharis\ae i.
${}^{3}$~Omnia ergo qu\ae cumque dixerint vobis, servate, et facite~: secundum opera vero eorum nolite facere~: dicunt enim, et non faciunt.
${}^{4}$~Alligant enim onera gravia, et importabilia, et imponunt in humeros hominum~: digito autem suo nolunt ea movere.
${}^{5}$~Omnia vero opera sua faciunt ut videantur ab hominibus~: dilatant enim phylacteria sua, et magnificant fimbrias.
${}^{6}$~Amant autem primos recubitus in cœnis, et primas cathedras in synagogis,
${}^{7}$~et salutationes in foro, et vocari ab hominibus Rabbi.
${}^{8}$~Vos autem nolite vocari Rabbi~: unus est enim magister vester, omnes autem vos fratres estis.
${}^{9}$~Et patrem nolite vocare vobis super terram~: unus est enim pater vester qui in c\ae lis est.
${}^{10}$~Nec vocemini magistri~: quia magister vester unus est, Christus.
${}^{11}$~Qui major est vestrum, erit minister vester.
${}^{12}$~Qui autem se exaltaverit, humiliabitur~: et qui se humiliaverit, exaltabitur.


${}^{13}$~V\ae\ autem vobis scrib\ae\ et pharis\ae i hypocrit\ae , quia clauditis regnum c\ae lorum ante homines~! vos enim non intratis, nec intro\"euntes sinitis intrare.
${}^{14}$~V\ae\ vobis scrib\ae\ et pharis\ae i hypocrit\ae , quia comeditis domos viduarum, orationes longas orantes~! propter hoc amplius accipietis judicium.
${}^{15}$~V\ae\ vobis scrib\ae\ et pharis\ae i hypocrit\ae , quia circuitis mare, et aridam, ut faciatis unum proselytum, et cum fuerit factus, facitis eum filium gehenn\ae\ duplo quam vos.
${}^{16}$~V\ae\ vobis duces c\ae ci, qui dicitis~: Quicumque juraverit per templum, nihil est~: qui autem juraverit in auro templi, debet.
${}^{17}$~Stulti et c\ae ci~: quid enim majus est~? aurum, an templum, quod sanctificat aurum~?
${}^{18}$~Et quicumque juraverit in altari, nihil est~: quicumque autem juraverit in dono, quod est super illud, debet.
${}^{19}$~C\ae ci~: quid enim majus est, donum, an altare, quod sanctificat donum~?
${}^{20}$~Qui ergo jurat in altari, jurat in eo, et in omnibus qu\ae\ super illud sunt.
${}^{21}$~Et quicumque juraverit in templo, jurat in illo, et in eo qui habitat in ipso~:
${}^{22}$~et qui jurat in c\ae lo, jurat in throno Dei, et in eo qui sedet super eum.
${}^{23}$~V\ae\ vobis scrib\ae\ et pharis\ae i hypocrit\ae , qui decimatis mentham, et anethum, et cyminum, et reliquistis qu\ae\ graviora sunt legis, judicium, et misericordiam, et fidem~! h\ae c oportuit facere, et illa non omittere.
${}^{24}$~Duces c\ae ci, excolantes culicem, camelum autem glutientes.
${}^{25}$~V\ae\ vobis scrib\ae\ et pharis\ae i hypocrit\ae , quia mundatis quod deforis est calicis et paropsidis~; intus autem pleni estis rapina et immunditia~!
${}^{26}$~Pharis\ae e c\ae ce, munda prius quod intus est calicis, et paropsidis, ut fiat id, quod deforis est, mundum.
${}^{27}$~V\ae\ vobis scrib\ae\ et pharis\ae i hypocrit\ae , quia similes estis sepulchris dealbatis, qu\ae\ a foris parent hominibus speciosa, intus vero pleni sunt ossibus mortuorum, et omni spurcitia~!
${}^{28}$~Sic et vos a foris quidem paretis hominibus justi~: intus autem pleni estis hypocrisi et iniquitate.
${}^{29}$~V\ae\ vobis scrib\ae\ et pharis\ae i hypocrit\ae , qui \ae dificatis sepulchra prophetarum, et ornatis monumenta justorum,
${}^{30}$~et dicitis~: Si fuissemus in diebus patrum nostrorum, non essemus socii eorum in sanguine prophetarum~!
${}^{31}$~itaque testimonio estis vobismetipsis, quia filii estis eorum, qui prophetas occiderunt.
${}^{32}$~Et vos implete mensuram patrum vestrorum.
${}^{33}$~Serpentes, genimina viperarum, quomodo fugietis a judicio gehenn\ae~?


${}^{34}$~Ideo ecce ego mitto ad vos prophetas, et sapientes, et scribas, et ex illis occidetis, et crucifigetis, et ex eis flagellabitis in synagogis vestris, et persequemini de civitate in civitatem~:
${}^{35}$~ut veniat super vos omnis sanguis justus, qui effusus est super terram, a sanguine Abel justi usque ad sanguinem Zachari\ae , filii Barachi\ae , quem occidistis inter templum et altare.
${}^{36}$~Amen dico vobis, venient h\ae c omnia super generationem istam.
${}^{37}$~Jerusalem, Jerusalem, qu\ae\ occidis prophetas, et lapidas eos, qui ad te missi sunt, quoties volui congregare filios tuos, quemadmodum gallina congregat pullos suos sub alas, et noluisti~?
${}^{38}$~Ecce relinquetur vobis domus vestra deserta.
${}^{39}$~Dico enim vobis, non me videbitis amodo, donec dicatis~: Benedictus, qui venit in nomine Domini.

\bchapter
\mylettrine{E}t egressus Jesus de templo, ibat. Et accesserunt discipuli ejus, ut ostenderent ei \ae dificationes templi.
${}^{2}$~Ipse autem respondens dixit illis~: Videtis h\ae c omnia~? amen dico vobis, non relinquetur hic lapis super lapidem, qui non destruatur.


${}^{3}$~Sedente autem eo super montem Oliveti, accesserunt ad eum discipuli secreto, dicentes~: Dic nobis, quando h\ae c erunt~? et quod signum adventus tui, et consummationis s\ae culi~?
${}^{4}$~Et respondens Jesus, dixit eis~: Videte ne quis vos seducat~:
${}^{5}$~multi enim venient in nomine meo, dicentes~: Ego sum Christus~: et multos seducent.
${}^{6}$~Audituri enim estis pr\ae lia, et opiniones pr\ae liorum. Videte ne turbemini~: oportet enim h\ae c fieri, sed nondum est finis~:
${}^{7}$~consurget enim gens in gentem, et regnum in regnum, et erunt pestilenti\ae , et fames, et terr\ae motus per loca~:
${}^{8}$~h\ae c autem omnia initia sunt dolorum.


${}^{9}$~Tunc tradent vos in tribulationem, et occident vos~: et eritis odio omnibus gentibus propter nomen meum.
${}^{10}$~Et tunc scandalizabuntur multi, et invicem tradent, et odio habebunt invicem.
${}^{11}$~Et multi pseudoprophet\ae\ surgent, et seducent multos.
${}^{12}$~Et quoniam abundavit iniquitas, refrigescet caritas multorum~:
${}^{13}$~qui autem perseveraverit usque in finem, hic salvus erit.
${}^{14}$~Et pr\ae dicabitur hoc Evangelium regni in universo orbe, in testimonium omnibus gentibus~: et tunc veniet consummatio.


${}^{15}$~Cum ergo videritis abominationem desolationis, qu\ae\ dicta est a Daniele propheta, stantem in loco sancto, qui legit, intelligat~:
${}^{16}$~tunc qui in Jud\ae a sunt, fugiant ad montes~:
${}^{17}$~et qui in tecto, non descendat tollere aliquid de domo sua~:
${}^{18}$~et qui in agro, non revertatur tollere tunicam suam.
${}^{19}$~V\ae\ autem pr\ae gnantibus et nutrientibus in illis diebus~!


${}^{20}$~Orate autem ut non fiat fuga vestra in hieme, vel sabbato~:
${}^{21}$~erit enim tunc tribulatio magna, qualis non fuit ab initio mundi usque modo, neque fiet.
${}^{22}$~Et nisi breviati fuissent dies illi, non fieret salva omnis caro~: sed propter electos breviabuntur dies illi.
${}^{23}$~Tunc si quis vobis dixerit~: Ecce hic est Christus, aut illic~: nolite credere.
${}^{24}$~Surgent enim pseudochristi, et pseudoprophet\ae~: et dabunt signa magna, et prodigia, ita ut in errorem inducantur (si fieri potest) etiam electi.
${}^{25}$~Ecce pr\ae dixi vobis.
${}^{26}$~Si ergo dixerint vobis~: Ecce in deserto est, nolite exire~; Ecce in penetralibus, nolite credere.
${}^{27}$~Sicut enim fulgur exit ab oriente, et paret usque in occidentem~: ita erit et adventus Filii hominis.
${}^{28}$~Ubicumque fuerit corpus, illic congregabuntur et aquil\ae .


${}^{29}$~Statim autem post tribulationem dierum illorum sol obscurabitur, et luna non dabit lumen suum, et stell\ae\ cadent de c\ae lo, et virtutes c\ae lorum commovebuntur~:
${}^{30}$~et tunc parebit signum Filii hominis in c\ae lo~: et tunc plangent omnes tribus terr\ae~: et videbunt Filium hominis venientem in nubibus c\ae li cum virtute multa et majestate.
${}^{31}$~Et mittet angelos suos cum tuba, et voce magna~: et congregabunt electos ejus a quatuor ventis, a summis c\ae lorum usque ad terminos eorum.


${}^{32}$~Ab arbore autem fici discite parabolam~: cum jam ramus ejus tener fuerit, et folia nata, scitis quia prope est \ae stas~:
${}^{33}$~ita et vos cum videritis h\ae c omnia, scitote quia prope est, in januis.
${}^{34}$~Amen dico vobis, quia non pr\ae teribit generatio h\ae c, donec omnia h\ae c fiant.
${}^{35}$~C\ae lum et terra transibunt, verba autem mea non pr\ae teribunt.


${}^{36}$~De die autem illa et hora nemo scit, neque angeli c\ae lorum, nisi solus Pater.
${}^{37}$~Sicut autem in diebus No\"e, ita erit et adventus Filii hominis~:
${}^{38}$~sicut enim erant in diebus ante diluvium comedentes et bibentes, nubentes et nuptum tradentes, usque ad eum diem, quo intravit No\"e in arcam,
${}^{39}$~et non cognoverunt donec venit diluvium, et tulit omnes~: ita erit et adventus Filii hominis.
${}^{40}$~Tunc duo erunt in agro~: unus assumetur, et unus relinquetur.
${}^{41}$~Du\ae\ molentes in mola~: una assumetur, et una relinquetur.


${}^{42}$~Vigilate ergo, quia nescitis qua hora Dominus vester venturus sit.
${}^{43}$~Illud autem scitote, quoniam si sciret paterfamilias qua hora fur venturus esset, vigilaret utique, et non sineret perfodi domum suam.
${}^{44}$~Ideo et vos estote parati~: quia qua nescitis hora Filius hominis venturus est.
${}^{45}$~Quis, putas, est fidelis servus, et prudens, quem constituit dominus suus super familiam suam ut det illis cibum in tempore~?
${}^{46}$~Beatus ille servus, quem cum venerit dominus ejus, invenerit sic facientem.
${}^{47}$~Amen dico vobis, quoniam super omnia bona sua constituet eum.
${}^{48}$~Si autem dixerit malus servus ille in corde suo~: Moram fecit dominus meus venire~:
${}^{49}$~et cœperit percutere conservos suos, manducet autem et bibat cum ebriosis~:
${}^{50}$~veniet dominus servi illius in die qua non sperat, et hora qua ignorat~:
${}^{51}$~et dividet eum, partemque ejus ponet cum hypocritis~: illic erit fletus et stridor dentium.

\bchapter
\mylettrine{T}unc simile erit regnum c\ae lorum decem virginibus~: qu\ae\ accipientes lampades suas exierunt obviam sponso et spons\ae .
${}^{2}$~Quinque autem ex eis erant fatu\ae , et quinque prudentes~:
${}^{3}$~sed quinque fatu\ae , acceptis lampadibus, non sumpserunt oleum secum~:
${}^{4}$~prudentes vero acceperunt oleum in vasis suis cum lampadibus.
${}^{5}$~Moram autem faciente sponso, dormitaverunt omnes et dormierunt.
${}^{6}$~Media autem nocte clamor factus est~: Ecce sponsus venit, exite obviam ei.
${}^{7}$~Tunc surrexerunt omnes virgines ill\ae , et ornaverunt lampades suas.
${}^{8}$~Fatu\ae\ autem sapientibus dixerunt~: Date nobis de oleo vestro, quia lampades nostr\ae\ extinguuntur.
${}^{9}$~Responderunt prudentes, dicentes~: Ne forte non sufficiat nobis, et vobis, ite potius ad vendentes, et emite vobis.
${}^{10}$~Dum autem irent emere, venit sponsus~: et qu\ae\ parat\ae\ erant, intraverunt cum eo ad nuptias, et clausa est janua.
${}^{11}$~Novissime vero veniunt et reliqu\ae\ virgines, dicentes~: Domine, domine, aperi nobis.
${}^{12}$~At ille respondens, ait~: Amen dico vobis, nescio vos.
${}^{13}$~Vigilate itaque, quia nescitis diem, neque horam.


${}^{14}$~Sicut enim homo peregre proficiscens, vocavit servos suos, et tradidit illis bona sua.
${}^{15}$~Et uni dedit quinque talenta, alii autem duo, alii vero unum, unicuique secundum propriam virtutem~: et profectus est statim.
${}^{16}$~Abiit autem qui quinque talenta acceperat, et operatus est in eis, et lucratus est alia quinque.
${}^{17}$~Similiter et qui duo acceperat, lucratus est alia duo.
${}^{18}$~Qui autem unum acceperat, abiens fodit in terram, et abscondit pecuniam domini sui.
${}^{19}$~Post multum vero temporis venit dominus servorum illorum, et posuit rationem cum eis.
${}^{20}$~Et accedens qui quinque talenta acceperat, obtulit alia quinque talenta, dicens~: Domine, quinque talenta tradidisti mihi, ecce alia quinque superlucratus sum.
${}^{21}$~Ait illi dominus ejus~: Euge serve bone, et fidelis~: quia super pauca fuisti fidelis, super multa te constituam~; intra in gaudium domini tui.
${}^{22}$~Accessit autem et qui duo talenta acceperat, et ait~: Domine, duo talenta tradidisti mihi, ecce alia duo lucratus sum.
${}^{23}$~Ait illi dominus ejus~: Euge serve bone, et fidelis~: quia super pauca fuisti fidelis, super multa te constituam~; intra in gaudium domini tui.
${}^{24}$~Accedens autem et qui unum talentum acceperat, ait~: Domine, scio quia homo durus es~; metis ubi non seminasti, et congregas ubi non sparsisti~:
${}^{25}$~et timens abii, et abscondi talentum tuum in terra~: ecce habes quod tuum est.
${}^{26}$~Respondens autem dominus ejus, dixit ei~: Serve male, et piger, sciebas quia meto ubi non semino, et congrego ubi non sparsi~:
${}^{27}$~oportuit ergo te committere pecuniam meam numulariis, et veniens ego recepissem utique quod meum est cum usura.
${}^{28}$~Tollite itaque ab eo talentum, et date ei qui habet decem talenta~:
${}^{29}$~omni enim habenti dabitur, et abundabit~: ei autem qui non habet, et quod videtur habere, auferetur ab eo.
${}^{30}$~Et inutilem servum ejicite in tenebras exteriores~: illic erit fletus, et stridor dentium.


${}^{31}$~Cum autem venerit Filius hominis in majestate sua, et omnes angeli cum eo, tunc sedebit super sedem majestatis su\ae~:
${}^{32}$~et congregabuntur ante eum omnes gentes, et separabit eos ab invicem, sicut pastor segregat oves ab h\ae dis~:
${}^{33}$~et statuet oves quidem a dextris suis, h\ae dos autem a sinistris.
${}^{34}$~Tunc dicet rex his qui a dextris ejus erunt~: Venite benedicti Patris mei, possidete paratum vobis regnum a constitutione mundi~:
${}^{35}$~esurivi enim, et dedistis mihi manducare~: sitivi, et dedistis mihi bibere~: hospes eram, et collegistis me~:
${}^{36}$~nudus, et cooperuistis me~: infirmus, et visitastis me~: in carcere eram, et venistis ad me.
${}^{37}$~Tunc respondebunt ei justi, dicentes~: Domine, quando te vidimus esurientem, et pavimus te~: sitientem, et dedimus tibi potum~?
${}^{38}$~quando autem te vidimus hospitem, et collegimus te~: aut nudum, et cooperuimus te~?
${}^{39}$~aut quando te vidimus infirmum, aut in carcere, et venimus ad te~?
${}^{40}$~Et respondens rex, dicet illis~: Amen dico vobis, quamdiu fecistis uni ex his fratribus meis minimis, mihi fecistis.
${}^{41}$~Tunc dicet et his qui a sinistris erunt~: Discedite a me maledicti in ignem \ae ternum, qui paratus est diabolo, et angelis ejus~:
${}^{42}$~esurivi enim, et non dedistis mihi manducare~: sitivi, et non dedistis mihi potum~:
${}^{43}$~hospes eram, et non collegistis me~: nudus, et non cooperuistis me~: infirmus, et in carcere, et non visitastis me.
${}^{44}$~Tunc respondebunt ei et ipsi, dicentes~: Domine, quando te vidimus esurientem, aut sitientem, aut hospitem, aut nudum, aut infirmum, aut in carcere, et non ministravimus tibi~?
${}^{45}$~Tunc respondebit illis, dicens~: Amen dico vobis~: Quamdiu non fecistis uni de minoribus his, nec mihi fecistis.
${}^{46}$~Et ibunt hi in supplicium \ae ternum~: justi autem in vitam \ae ternam.

\bchapter
\mylettrine{E}t factum est~: cum consummasset Jesus sermones hos omnes, dixit discipulis suis~:
${}^{2}$~Scitis quia post biduum Pascha fiet, et Filius hominis tradetur ut crucifigatur.
${}^{3}$~Tunc congregati sunt principes sacerdotum, et seniores populi, in atrium principis sacerdotum, qui dicebatur Caiphas~:
${}^{4}$~et consilium fecerunt ut Jesum dolo tenerent, et occiderent.
${}^{5}$~Dicebant autem~: Non in die festo, ne forte tumultus fieret in populo.


${}^{6}$~Cum autem Jesus esset in Bethania in domo Simonis leprosi,
${}^{7}$~accessit ad eum mulier habens alabastrum unguenti pretiosi, et effudit super caput ipsius recumbentis.
${}^{8}$~Videntes autem discipuli, indignati sunt, dicentes~: Ut quid perditio h\ae c~?
${}^{9}$~potuit enim istud venundari multo, et dari pauperibus.
${}^{10}$~Sciens autem Jesus, ait illis~: Quid molesti estis huic mulieri~? opus enim bonum operata est in me.
${}^{11}$~Nam semper pauperes habetis vobiscum~: me autem non semper habetis.
${}^{12}$~Mittens enim h\ae c unguentum hoc in corpus meum, ad sepeliendum me fecit.
${}^{13}$~Amen dico vobis, ubicumque pr\ae dicatum fuerit hoc Evangelium in toto mundo, dicetur et quod h\ae c fecit in memoriam ejus.


${}^{14}$~Tunc abiit unus de duodecim, qui dicebatur Judas Iscariotes, ad principes sacerdotum~:
${}^{15}$~et ait illis~: Quid vultis mihi dare, et ego vobis eum tradam~? At illi constituerunt ei triginta argenteos.
${}^{16}$~Et exinde qu\ae rebat opportunitatem ut eum traderet.


${}^{17}$~Prima autem die azymorum accesserunt discipuli ad Jesum, dicentes~: Ubi vis paremus tibi comedere Pascha~?
${}^{18}$~At Jesus dixit~: Ite in civitatem ad quemdam, et dicite ei~: Magister dicit~: Tempus meum prope est, apud te facio Pascha cum discipulis meis.
${}^{19}$~Et fecerunt discipuli sicut constituit illis Jesus, et paraverunt Pascha.
${}^{20}$~Vespere autem facto, discumbebat cum duodecim discipulis suis.
${}^{21}$~Et edentibus illis, dixit~: Amen dico vobis, quia unus vestrum me traditurus est.
${}^{22}$~Et contristati valde, cœperunt singuli dicere~: Numquid ego sum Domine~?
${}^{23}$~At ipse respondens, ait~: Qui intingit mecum manum in paropside, hic me tradet.
${}^{24}$~Filius quidem hominis vadit, sicut scriptum est de illo~: v\ae\ autem homini illi, per quem Filius hominis tradetur~! bonum erat ei, si natus non fuisset homo ille.
${}^{25}$~Respondens autem Judas, qui tradidit eum, dixit~: Numquid ego sum Rabbi~? Ait illi~: Tu dixisti.


${}^{26}$~Cœnantibus autem eis, accepit Jesus panem, et benedixit, ac fregit, deditque discipulis suis, et ait~: Accipite, et comedite~: hoc est corpus meum.
${}^{27}$~Et accipiens calicem, gratias egit~: et dedit illis, dicens~: Bibite ex hoc omnes.
${}^{28}$~Hic est enim sanguis meus novi testamenti, qui pro multis effundetur in remissionem peccatorum.
${}^{29}$~Dico autem vobis~: non bibam amodo de hoc genimine vitis usque in diem illum, cum illud bibam vobiscum novum in regno Patris mei.


${}^{30}$~Et hymno dicto, exierunt in montem Oliveti.
${}^{31}$~Tunc dicit illis Jesus~: Omnes vos scandalum patiemini in me in ista nocte. Scriptum est enim~: Percutiam pastorem, et dispergentur oves gregis.
${}^{32}$~Postquam autem resurrexero, pr\ae cedam vos in Galil\ae am.
${}^{33}$~Respondens autem Petrus, ait illi~: Et si omnes scandalizati fuerint in te, ego numquam scandalizabor.
${}^{34}$~Ait illi Jesus~: Amen dico tibi, quia in hac nocte, antequam gallus cantet, ter me negabis.
${}^{35}$~Ait illi Petrus~: Etiamsi oportuerit me mori tecum, non te negabo. Similiter et omnes discipuli dixerunt.


${}^{36}$~Tunc venit Jesus cum illis in villam, qu\ae\ dicitur Gethsemani, et dixit discipulis suis~: Sedete hic donec vadam illuc, et orem.
${}^{37}$~Et assumpto Petro, et duobus filiis Zebed\ae i, cœpit contristari et mœstus esse.
${}^{38}$~Tunc ait illis~: Tristis est anima mea usque ad mortem~: sustinete hic, et vigilate mecum.
${}^{39}$~Et progressus pusillum, procidit in faciem suam, orans, et dicens~: Pater mi, si possibile est, transeat a me calix iste~: verumtamen non sicut ego volo, sed sicut tu.
${}^{40}$~Et venit ad discipulos suos, et invenit eos dormientes, et dicit Petro~: Sic non potuistis una hora vigilare mecum~?
${}^{41}$~Vigilate, et orate ut non intretis in tentationem. Spiritus quidem promptus est, caro autem infirma.
${}^{42}$~Iterum secundo abiit, et oravit, dicens~: Pater mi, si non potest hic calix transire nisi bibam illum, fiat voluntas tua.
${}^{43}$~Et venit iterum, et invenit eos dormientes~: erant enim oculi eorum gravati.
${}^{44}$~Et relictis illis, iterum abiit, et oravit tertio, eumdem sermonem dicens.
${}^{45}$~Tunc venit ad discipulos suos, et dicit illis~: Dormite jam, et requiescite~: ecce appropinquavit hora, et Filius hominis tradetur in manus peccatorum.
${}^{46}$~Surgite, eamus~: ecce appropinquavit qui me tradet.


${}^{47}$~Adhuc eo loquente, ecce Judas unus de duodecim venit, et cum eo turba multa cum gladiis et fustibus, missi a principibus sacerdotum, et senioribus populi.
${}^{48}$~Qui autem tradidit eum, dedit illis signum, dicens~: Quemcumque osculatus fuero, ipse est, tenete eum.
${}^{49}$~Et confestim accedens ad Jesum, dixit~: Ave Rabbi. Et osculatus est eum.
${}^{50}$~Dixitque illi Jesus~: Amice, ad quid venisti~? Tunc accesserunt, et manus injecerunt in Jesum, et tenuerunt eum.
${}^{51}$~Et ecce unus ex his qui erant cum Jesu, extendens manum, exemit gladium suum, et percutiens servum principis sacerdotum amputavit auriculam ejus.
${}^{52}$~Tunc ait illi Jesus~: Converte gladium tuum in locum suum~: omnes enim, qui acceperint gladium, gladio peribunt.
${}^{53}$~An putas, quia non possum rogare patrem meum, et exhibebit mihi modo plusquam duodecim legiones angelorum~?
${}^{54}$~Quomodo ergo implebuntur Scriptur\ae , quia sic oportet fieri~?
${}^{55}$~In illa hora dixit Jesus turbis~: Tamquam ad latronem existis cum gladiis et fustibus comprehendere me~: quotidie apud vos sedebam docens in templo, et non me tenuistis.
${}^{56}$~Hoc autem totum factum est, ut adimplerentur Scriptur\ae\ prophetarum. Tunc discipuli omnes, relicto eo, fugerunt.


${}^{57}$~At illi tenentes Jesum, duxerunt ad Caipham principem sacerdotum, ubi scrib\ae\ et seniores convenerant.
${}^{58}$~Petrus autem sequebatur eum a longe, usque in atrium principis sacerdotum. Et ingressus intro, sedebat cum ministris, ut videret finem.
${}^{59}$~Principes autem sacerdotum, et omne concilium, qu\ae rebant falsum testimonium contra Jesum, ut eum morti traderent~:
${}^{60}$~et non invenerunt, cum multi falsi testes accessissent. Novissime autem venerunt duo falsi testes,
${}^{61}$~et dixerunt~: Hic dixit~: Possum destruere templum Dei, et post triduum re\ae dificare illud.
${}^{62}$~Et surgens princeps sacerdotum, ait illi~: Nihil respondes ad ea, qu\ae\ isti adversum te testificantur~?
${}^{63}$~Jesus autem tacebat. Et princeps sacerdotum ait illi~: Adjuro te per Deum vivum, ut dicas nobis si tu es Christus Filius Dei.
${}^{64}$~Dicit illi Jesus~: Tu dixisti. Verumtamen dico vobis, amodo videbitis Filium hominis sedentem a dextris virtutis Dei, et venientem in nubibus c\ae li.
${}^{65}$~Tunc princeps sacerdotum scidit vestimenta sua, dicens~: Blasphemavit~: quid adhuc egemus testibus~? ecce nunc audistis blasphemiam~:
${}^{66}$~quid vobis videtur~? At illi respondentes dixerunt~: Reus est mortis.
${}^{67}$~Tunc exspuerunt in faciem ejus, et colaphis eum ceciderunt, alii autem palmas in faciem ejus dederunt,
${}^{68}$~dicentes~: Prophetiza nobis Christe, quis est qui te percussit~?


${}^{69}$~Petrus vero sedebat foris in atrio~: et accessit ad eum una ancilla, dicens~: Et tu cum Jesu Galil\ae o eras.
${}^{70}$~At ille negavit coram omnibus, dicens~: Nescio quid dicis.
${}^{71}$~Exeunte autem illo januam, vidit eum alia ancilla, et ait his qui erant ibi~: Et hic erat cum Jesu Nazareno.
${}^{72}$~Et iterum negavit cum juramento~: Quia non novi hominem.
${}^{73}$~Et post pusillum accesserunt qui stabant, et dixerunt Petro~: Vere et tu ex illis es~: nam et loquela tua manifestum te facit.
${}^{74}$~Tunc cœpit detestari et jurare quia non novisset hominem. Et continuo gallus cantavit.
${}^{75}$~Et recordatus est Petrus verbi Jesu, quod dixerat~: Priusquam gallus cantet, ter me negabis. Et egressus foras, flevit amare.

\bchapter
\mylettrine{M}ane autem facto, consilium inierunt omnes principes sacerdotum et seniores populi adversus Jesum, ut eum morti traderent.
${}^{2}$~Et vinctum adduxerunt eum, et tradiderunt Pontio Pilato pr\ae sidi.


${}^{3}$~Tunc videns Judas, qui eum tradidit, quod damnatus esset, pœnitentia ductus, retulit triginta argenteos principibus sacerdotum, et senioribus,
${}^{4}$~dicens~: Peccavi, tradens sanguinem justum. At illi dixerunt~: Quid ad nos~? tu videris.
${}^{5}$~Et projectis argenteis in templo, recessit~: et abiens laqueo se suspendit.
${}^{6}$~Principes autem sacerdotum, acceptis argenteis, dixerunt~: Non licet eos mittere in corbonam~: quia pretium sanguinis est.
${}^{7}$~Consilio autem inito, emerunt ex illis agrum figuli, in sepulturam peregrinorum.
${}^{8}$~Propter hoc vocatus est ager ille, Haceldama, hoc est, Ager sanguinis, usque in hodiernum diem.
${}^{9}$~Tunc impletum est quod dictum est per Jeremiam prophetam, dicentem~: Et acceperunt triginta argenteos pretium appretiati, quem appretiaverunt a filiis Isra\"el~:
${}^{10}$~et dederunt eos in agrum figuli, sicut constituit mihi Dominus.


${}^{11}$~Jesus autem stetit ante pr\ae sidem, et interrogavit eum pr\ae ses, dicens~: Tu es rex Jud\ae orum~? Dicit illi Jesus~: Tu dicis.
${}^{12}$~Et cum accusaretur a principibus sacerdotum et senioribus, nihil respondit.
${}^{13}$~Tunc dicit illi Pilatus~: Non audis quanta adversum te dicunt testimonia~?
${}^{14}$~Et non respondit ei ad ullum verbum, ita ut miraretur pr\ae ses vehementer.
${}^{15}$~Per diem autem solemnem consueverat pr\ae ses populo dimittere unum vinctum, quem voluissent~:
${}^{16}$~habebat autem tunc vinctum insignem, qui dicebatur Barabbas.
${}^{17}$~Congregatis ergo illis, dixit Pilatus~: Quem vultis dimittam vobis~: Barabbam, an Jesum, qui dicitur Christus~?
${}^{18}$~Sciebat enim quod per invidiam tradidissent eum.
${}^{19}$~Sedente autem illo pro tribunali, misit ad eum uxor ejus, dicens~: Nihil tibi, et justo illi~: multa enim passa sum hodie per visum propter eum.
${}^{20}$~Principes autem sacerdotum et seniores persuaserunt populis ut peterent Barabbam, Jesum vero perderent.
${}^{21}$~Respondens autem pr\ae ses, ait illis~: Quem vultis vobis de duobus dimitti~? At illi dixerunt~: Barabbam.
${}^{22}$~Dicit illis Pilatus~: Quid igitur faciam de Jesu, qui dicitur Christus~?
${}^{23}$~Dicunt omnes~: Crucifigatur. Ait illis pr\ae ses~: Quid enim mali fecit~? At illi magis clamabant dicentes~: Crucifigatur.
${}^{24}$~Videns autem Pilatus quia nihil proficeret, sed magis tumultus fieret~: accepta aqua, lavit manus coram populo, dicens~: Innocens ego sum a sanguine justi hujus~: vos videritis.
${}^{25}$~Et respondens universus populus, dixit~: Sanguis ejus super nos, et super filios nostros.
${}^{26}$~Tunc dimisit illis Barabbam~: Jesum autem flagellatum tradidit eis ut crucifigeretur.


${}^{27}$~Tunc milites pr\ae sidis suscipientes Jesum in pr\ae torium, congregaverunt ad eum universam cohortem~:
${}^{28}$~et exuentes eum, chlamydem coccineam circumdederunt ei,
${}^{29}$~et plectentes coronam de spinis, posuerunt super caput ejus, et arundinem in dextera ejus. Et genu flexo ante eum, illudebant ei, dicentes~: Ave rex Jud\ae orum.
${}^{30}$~Et exspuentes in eum, acceperunt arundinem, et percutiebant caput ejus.
${}^{31}$~Et postquam illuserunt ei, exuerunt eum chlamyde, et induerunt eum vestimentis ejus, et duxerunt eum ut crucifigerent.


${}^{32}$~Exeuntes autem invenerunt hominem Cyren\ae um, nomine Simonem~: hunc angariaverunt ut tolleret crucem ejus.
${}^{33}$~Et venerunt in locum qui dicitur Golgotha, quod est Calvari\ae\ locus.
${}^{34}$~Et dederunt ei vinum bibere cum felle mistum. Et cum gustasset, noluit bibere.
${}^{35}$~Postquam autem crucifixerunt eum, diviserunt vestimenta ejus, sortem mittentes~: ut impleretur quod dictum est per prophetam dicentem~: Diviserunt sibi vestimenta mea, et super vestem meam miserunt sortem.
${}^{36}$~Et sedentes servabant eum.
${}^{37}$~Et imposuerunt super caput ejus causam ipsius scriptam~: Hic est Jesus rex Jud\ae orum.
${}^{38}$~Tunc crucifixi sunt cum eo duo latrones~: unus a dextris, et unus a sinistris.
${}^{39}$~Pr\ae tereuntes autem blasphemabant eum moventes capita sua,
${}^{40}$~et dicentes~: Vah~! qui destruis templum Dei, et in triduo illud re\ae dificas~: salva temetipsum~: si Filius Dei es, descende de cruce.
${}^{41}$~Similiter et principes sacerdotum illudentes cum scribis et senioribus dicebant~:
${}^{42}$~Alios salvos fecit, seipsum non potest salvum facere~: si rex Isra\"el est, descendat nunc de cruce, et credimus ei~:
${}^{43}$~confidit in Deo~: liberet nunc, si vult eum~: dixit enim~: Quia Filius Dei sum.
${}^{44}$~Idipsum autem et latrones, qui crucifixi erant cum eo, improperabant ei.


${}^{45}$~A sexta autem hora tenebr\ae\ fact\ae\ sunt super universam terram usque ad horam nonam.
${}^{46}$~Et circa horam nonam clamavit Jesus voce magna, dicens~: Eli, Eli, lamma sabacthani~? hoc est~: Deus meus, Deus meus, ut quid dereliquisti me~?
${}^{47}$~Quidam autem illic stantes, et audientes, dicebant~: Eliam vocat iste.
${}^{48}$~Et continuo currens unus ex eis, acceptam spongiam implevit aceto, et imposuit arundini, et dabat ei bibere.
${}^{49}$~Ceteri vero dicebant~: Sine, videamus an veniat Elias liberans eum.
${}^{50}$~Jesus autem iterum clamans voce magna, emisit spiritum.


${}^{51}$~Et ecce velum templi scissum est in duas partes a summo usque deorsum~: et terra mota est, et petr\ae\ sciss\ae\ sunt,
${}^{52}$~et monumenta aperta sunt~: et multa corpora sanctorum, qui dormierant, surrexerunt.
${}^{53}$~Et exeuntes de monumentis post resurrectionem ejus, venerunt in sanctam civitatem, et apparuerunt multis.
${}^{54}$~Centurio autem, et qui cum eo erant, custodientes Jesum, viso terr\ae motu, et his qu\ae\ fiebant, timuerunt valde, dicentes~: Vere Filius Dei erat iste.


${}^{55}$~Erant autem ibi mulieres mult\ae\ a longe, qu\ae\ secut\ae\ erant Jesum a Galil\ae a, ministrantes ei~:
${}^{56}$~inter quas erat Maria Magdalene, et Maria Jacobi, et Joseph mater, et mater filiorum Zebed\ae i.
${}^{57}$~Cum autem sero factum esset, venit quidam homo dives ab Arimath\ae a, nomine Joseph, qui et ipse discipulus erat Jesu~:
${}^{58}$~hic accessit ad Pilatum, et petiit corpus Jesu. Tunc Pilatus jussit reddi corpus.
${}^{59}$~Et accepto corpore, Joseph involvit illud in sindone munda,
${}^{60}$~et posuit illud in monumento suo novo, quod exciderat in petra. Et advolvit saxum magnum ad ostium monumenti, et abiit.
${}^{61}$~Erant autem ibi Maria Magdalene, et altera Maria, sedentes contra sepulchrum.


${}^{62}$~Altera autem die, qu\ae\ est post Parasceven, convenerunt principes sacerdotum et pharis\ae i ad Pilatum,
${}^{63}$~dicentes~: Domine, recordati sumus, quia seductor ille dixit adhuc vivens~: Post tres dies resurgam.
${}^{64}$~Jube ergo custodiri sepulchrum usque in diem tertium~: ne forte veniant discipuli ejus, et furentur eum, et dicant plebi~: Surrexit a mortuis~: et erit novissimus error pejor priore.
${}^{65}$~Ait illis Pilatus~: Habetis custodiam, ite, custodite sicut scitis.
${}^{66}$~Illi autem abeuntes, munierunt sepulchrum, signantes lapidem, cum custodibus.

\bchapter
\mylettrine{V}espere autem sabbati, qu\ae\ lucescit in prima sabbati, venit Maria Magdalene, et altera Maria, videre sepulchrum.
${}^{2}$~Et ecce terr\ae motus factus est magnus. Angelus enim Domini descendit de c\ae lo~: et accedens revolvit lapidem, et sedebat super eum~:
${}^{3}$~erat autem aspectus ejus sicut fulgur~: et vestimentum ejus sicut nix.
${}^{4}$~Pr\ae\ timore autem ejus exterriti sunt custodes, et facti sunt velut mortui.
${}^{5}$~Respondens autem angelus dixit mulieribus~: Nolite timere vos~: scio enim, quod Jesum, qui crucifixus est, qu\ae ritis.
${}^{6}$~Non est hic~: surrexit enim, sicut dixit~: venite, et videte locum ubi positus erat Dominus.
${}^{7}$~Et cito euntes, dicite discipulis ejus quia surrexit~: et ecce pr\ae cedit vos in Galil\ae am~: ibi eum videbitis~: ecce pr\ae dixi vobis.
${}^{8}$~Et exierunt cito de monumento cum timore et gaudio magno, currentes nuntiare discipulis ejus.
${}^{9}$~Et ecce Jesus occurrit illis, dicens~: Avete. Ill\ae\ autem accesserunt, et tenuerunt pedes ejus, et adoraverunt eum.
${}^{10}$~Tunc ait illis Jesus~: Nolite timere~: ite, nuntiate fratribus meis ut eant in Galil\ae am~; ibi me videbunt.


${}^{11}$~Qu\ae\ cum abiissent, ecce quidam de custodibus venerunt in civitatem, et nuntiaverunt principibus sacerdotum omnia qu\ae\ facta fuerant.
${}^{12}$~Et congregati cum senioribus consilio accepto, pecuniam copiosam dederunt militibus,
${}^{13}$~dicentes~: Dicite quia discipuli ejus nocte venerunt, et furati sunt eum, nobis dormientibus.
${}^{14}$~Et si hoc auditum fuerit a pr\ae side, nos suadebimus ei, et securos vos faciemus.
${}^{15}$~At illi, accepta pecunia, fecerunt sicut erant edocti. Et divulgatum est verbum istud apud Jud\ae os, usque in hodiernum diem.
${}^{16}$~Undecim autem discipuli abierunt in Galil\ae am in montem ubi constituerat illis Jesus.


${}^{17}$~Et videntes eum adoraverunt~: quidam autem dubitaverunt.
${}^{18}$~Et accedens Jesus locutus est eis, dicens~: Data est mihi omnis potestas in c\ae lo et in terra~:
${}^{19}$~euntes ergo docete omnes gentes~: baptizantes eos in nomine Patris, et Filii, et Spiritus Sancti~:
${}^{20}$~docentes eos servare omnia qu\ae cumque mandavi vobis~: et ecce ego vobiscum sum omnibus diebus, usque ad consummationem s\ae culi.
