\bbook{Liber Numeri}
{Numeri}{images/genese_heading}
\addcontentsline{toc}{subsection}{Numeri}


\bchapter{1}
\lettrine[lines=10,image=true,loversize=0.05,lraise=-0.03]{L}{}ocutusque est Dominus ad Moysen in deserto Sinai in tabernaculo fœderis, prima die mensis secundi, anno altero egressionis eorum ex \AE gypto, dicens~:
${}^{2}$~Tollite summam univers\ae\ congregationis filiorum Isra\"el per cognationes et domos suas, et nomina singulorum, quidquid sexus est masculini
${}^{3}$~a vigesimo anno et supra, omnium virorum fortium ex Isra\"el, et numerabitis eos per turmas suas, tu et Aaron.
${}^{4}$~Eruntque vobiscum principes tribuum ac domorum in cognationibus suis,
${}^{5}$~quorum ista sunt nomina~: de Ruben, Elisur, filius Sedeur~;
${}^{6}$~de Simeon, Salamiel filius Surisaddai~;
${}^{7}$~de Juda, Nahasson filius Aminadab~;
${}^{8}$~de Issachar, Nathana\"el filius Suar~;
${}^{9}$~de Zabulon, Eliab filius Helon~;
${}^{10}$~filiorum autem Joseph, de Ephraim, Elisama filius Ammiud~; de Manasse, Gamaliel filius Phadassur~;
${}^{11}$~de Benjamin, Abidan filius Gedeonis~;
${}^{12}$~de Dan, Ahiezer filius Ammisaddai~;
${}^{13}$~de Aser, Phegiel filius Ochran~;
${}^{14}$~de Gad, Eliasaph filius Duel~;
${}^{15}$~de Nephthali, Ahira filius Enan.
${}^{16}$~Hi nobilissimi principes multitudinis per tribus et cognationes suas, et capita exercitus Isra\"el,
${}^{17}$~quos tulerunt Moyses et Aaron cum omni vulgi multitudine~:
${}^{18}$~et congregaverunt primo die mensis secundi, recensentes eos per cognationes, et domos, ac familias, et capita, et nomina singulorum a vigesimo anno et supra,
${}^{19}$~sicut pr\ae ceperat Dominus Moysi. Numeratique sunt in deserto Sinai.


${}^{20}$~De Ruben primogenito Isra\"elis per generationes et familias ac domos suas, et nomina capitum singulorum, omne quod sexus est masculini a vigesimo anno et supra, procedentium ad bellum,
${}^{21}$~quadraginta sex millia quingenti.
${}^{22}$~De filiis Simeon per generationes et familias ac domos cognationum suarum recensiti sunt per nomina et capita singulorum, omne quod sexus est masculini a vigesimo anno et supra, procedentium ad bellum,
${}^{23}$~quinquaginta novem millia trecenti.
${}^{24}$~De filiis Gad per generationes et familias ac domos cognationum suarum recensiti sunt per nomina singulorum a viginti annis et supra, omnes qui ad bella procederent,
${}^{25}$~quadraginta quinque millia sexcenti quinquaginta.
${}^{26}$~De filiis Juda per generationes et familias ac domos cognationum suarum, per nomina singulorum a vigesimo anno et supra, omnes qui poterant ad bella procedere,
${}^{27}$~recensiti sunt septuaginta quatuor millia sexcenti.
${}^{28}$~De filiis Issachar, per generationes et familias ac domos cognationum suarum, per nomina singulorum a vigesimo anno et supra, omnes qui ad bella procederent,
${}^{29}$~recensiti sunt quinquaginta quatuor millia quadringenti.
${}^{30}$~De filiis Zabulon per generationes et familias ac domos cognationum suarum recensiti sunt per nomina singulorum a vigesimo anno et supra, omnes qui poterant ad bella procedere,
${}^{31}$~quinquaginta septem millia quadringenti.
${}^{32}$~De filiis Joseph, filiorum Ephraim per generationes et familias ac domos cognationum suarum recensiti sunt per nomina singulorum a vigesimo anno et supra, omnes qui poterant ad bella procedere,
${}^{33}$~quadraginta millia quingenti.
${}^{34}$~Porro filiorum Manasse per generationes et familias ac domos cognationum suarum recensiti sunt per nomina singulorum a viginti annis et supra, omnes qui poterant ad bella procedere,
${}^{35}$~triginta duo millia ducenti.
${}^{36}$~De filiis Benjamin per generationes et familias ac domos cognationum suarum recensiti sunt nominibus singulorum a vigesimo anno et supra, omnes qui poterant ad bella procedere,
${}^{37}$~triginta quinque millia quadringenti.
${}^{38}$~De filiis Dan per generationes et familias ac domos cognationum suarum recensiti sunt nominibus singulorum a vigesimo anno et supra, omnes qui poterant ad bella procedere,
${}^{39}$~sexaginta duo millia septingenti.
${}^{40}$~De filiis Aser per generationes et familias ac domos cognationum suarum recensiti sunt per nomina singulorum a vigesimo anno et supra, omnes qui poterant ad bella procedere,
${}^{41}$~quadraginta millia et mille quingenti.
${}^{42}$~De filiis Nephthali per generationes et familias ac domos cognationum suarum recensiti sunt nominibus singulorum a vigesimo anno et supra, omnes qui poterant ad bella procedere,
${}^{43}$~quinquaginta tria millia quadringenti.


${}^{44}$~Hi sunt, quos numeraverunt Moyses et Aaron, et duodecim principes Isra\"el, singulos per domos cognationum suarum.
${}^{45}$~Fueruntque omnis numerus filiorum Isra\"el per domos et familias suas a vigesimo anno et supra, qui poterant ad bella procedere,
${}^{46}$~sexcenta tria millia virorum quingenti quinquaginta.


${}^{47}$~Levit\ae\ autem in tribu familiarum suarum non sunt numerati cum eis.
${}^{48}$~Locutusque est Dominus ad Moysen, dicens~:
${}^{49}$~Tribum Levi noli numerare, neque pones summam eorum cum filiis Isra\"el~:
${}^{50}$~sed constitue eos super tabernaculum testimonii et cuncta vasa ejus, et quidquid ad c\ae remonias pertinet. Ipsi portabunt tabernaculum et omnia utensilia ejus~: et erunt in ministerio, ac per gyrum tabernaculi metabuntur.
${}^{51}$~Cum proficiscendum fuerit, deponent Levit\ae\ tabernaculum~; cum castrametandum, erigent. Quisquis externorum accesserit, occidetur.
${}^{52}$~Metabuntur autem castra filii Isra\"el unusquisque per turmas et cuneos atque exercitum suum.
${}^{53}$~Porro Levit\ae\ per gyrum tabernaculi figent tentoria, ne fiat indignatio super multitudinem filiorum Isra\"el, et excubabunt in custodiis tabernaculi testimonii.
${}^{54}$~Fecerunt ergo filii Isra\"el juxta omnia qu\ae\ pr\ae ceperat Dominus Moysi.

\bchapter{2}
\lettrine[lines=10,image=true,loversize=0.05,lraise=-0.03]{L}{}ocutusque est Dominus ad Moysen et Aaron, dicens~:
${}^{2}$~Singuli per turmas, signa, atque vexilla, et domos cognationum suarum, castrametabuntur filii Isra\"el, per gyrum tabernaculi fœderis.
${}^{3}$~Ad orientem Judas figet tentoria per turmas exercitus sui~: eritque princeps filiorum ejus Nahasson filius Aminadab.
${}^{4}$~Et omnis de stirpe ejus summa pugnantium, septuaginta quatuor millia sexcenti.
${}^{5}$~Juxta eum castrametati sunt de tribu Issachar, quorum princeps fuit Nathana\"el filius Suar.
${}^{6}$~Et omnis numerus pugnatorum ejus quinquaginta quatuor millia quadringenti.
${}^{7}$~In tribu Zabulon princeps fuit Eliab filius Helon.
${}^{8}$~Omnis de stirpe ejus exercitus pugnatorum, quinquaginta septem millia quadringenti.
${}^{9}$~Universi qui in castris Jud\ae\ enumerati sunt, fuerunt centum octoginta sex millia quadringenti~: et per turmas suas primi egredientur.


${}^{10}$~In castris filiorum Ruben ad meridianam plagam erit princeps Elisur filius Sedeur.
${}^{11}$~Et cunctus exercitus pugnatorum ejus qui numerati sunt, quadraginta sex millia quingenti.
${}^{12}$~Juxta eum castrametati sunt de tribu Simeon~: quorum princeps fuit Salamiel filius Surisaddai.
${}^{13}$~Et cunctus exercitus pugnatorum ejus qui numerati sunt, quinquaginta novem millia trecenti.
${}^{14}$~In tribu Gad princeps fuit Eliasaph filius Duel.
${}^{15}$~Et cunctus exercitus pugnatorum ejus, qui numerati sunt, quadraginta quinque millia sexcenti quinquaginta.
${}^{16}$~Omnes qui recensiti sunt in castris Ruben, centum quinquaginta millia et mille quadringenti quinquaginta per turmas suas~: in secundo loco proficiscentur.
${}^{17}$~Levabitur autem tabernaculum testimonii per officia Levitarum, et turmas eorum~: quomodo erigetur, ita et deponetur. Singuli per loca et ordines suos proficiscentur.


${}^{18}$~Ad occidentalem plagam erunt castra filiorum Ephraim, quorum princeps fuit Elisama filius Ammiud.
${}^{19}$~Cunctus exercitus pugnatorum ejus, qui numerati sunt, quadraginta millia quingenti.
${}^{20}$~Et cum eis tribus filiorum Manasse, quorum princeps fuit Gamaliel filius Phadassur.
${}^{21}$~Cunctusque exercitus pugnatorum ejus, qui numerati sunt, triginta duo millia ducenti.
${}^{22}$~In tribu filiorum Benjamin princeps fuit Abidan filius Gedeonis.
${}^{23}$~Et cunctus exercitus pugnatorum ejus, qui recensiti sunt, triginta quinque millia quadringenti.
${}^{24}$~Omnes qui numerati sunt in castris Ephraim, centum octo millia centum per turmas suas~: tertii proficiscentur.


${}^{25}$~Ad aquilonis partem castrametati sunt filii Dan~: quorum princeps fuit Ahiezer filius Ammisaddai.
${}^{26}$~Cunctus exercitus pugnatorum ejus, qui numerati sunt, sexaginta duo millia septingenti.
${}^{27}$~Juxta eum fixere tentoria de tribu Aser~: quorum princeps fuit Phegiel filius Ochran.
${}^{28}$~Cunctus exercitus pugnatorum ejus, qui numerati sunt, quadraginta millia et mille quingenti.
${}^{29}$~De tribu filiorum Nephthali princeps fuit Ahira filius Enan.
${}^{30}$~Cunctus exercitus pugnatorum ejus, quinquaginta tria millia quadringenti.
${}^{31}$~Omnes qui numerati sunt in castris Dan, fuerunt centum quinquaginta septem millia sexcenti~: et novissimi proficiscentur.


${}^{32}$~Hic numerus filiorum Isra\"el, per domos cognationum suarum et turmas divisi exercitus, sexcenta tria millia quingenti quinquaginta.
${}^{33}$~Levit\ae\ autem non sunt numerati inter filios Isra\"el~: sic enim pr\ae ceperat Dominus Moysi.
${}^{34}$~Feceruntque filii Isra\"el juxta omnia qu\ae\ mandaverat Dominus. Castrametati sunt per turmas suas, et profecti per familias ac domos patrum suorum.

\bchapter{3}
\lettrine[lines=10,image=true,loversize=0.05,lraise=-0.03]{H}{}\ae\ sunt generationes Aaron et Moysi in die qua locutus est Dominus ad Moysen in monte Sinai.
${}^{2}$~Et h\ae c nomina filiorum Aaron~: primogenitus ejus Nadab, deinde Abiu, et Eleazar, et Ithamar.
${}^{3}$~H\ae c nomina filiorum Aaron sacerdotum qui uncti sunt, et quorum replet\ae\ et consecrat\ae\ manus ut sacerdotio fungerentur.
${}^{4}$~Mortui sunt enim Nadab et Abiu cum offerrent ignem alienum in conspectu Domini in deserto Sinai, absque liberis~: functique sunt sacerdotio Eleazar et Ithamar coram Aaron patre suo.
${}^{5}$~Locutusque est Dominus ad Moysen, dicens~:
${}^{6}$~Applica tribum Levi, et fac stare in conspectu Aaron sacerdotis ut ministrent ei, et excubent,
${}^{7}$~et observent quidquid ad cultum pertinet multitudinis coram tabernaculo testimonii,
${}^{8}$~et custodiant vasa tabernaculi, servientes in ministerio ejus.
${}^{9}$~Dabisque dono Levitas
${}^{10}$~Aaron et filiis ejus, quibus traditi sunt a filiis Isra\"el. Aaron autem et filios ejus constitues super cultum sacerdotii. Externus, qui ad ministrandum accesserit, morietur.
${}^{11}$~Locutusque est Dominus ad Moysen, dicens~:
${}^{12}$~Ego tuli Levitas a filiis Isra\"el pro omni primogenito, qui aperit vulvam in filiis Isra\"el, eruntque Levit\ae\ mei.
${}^{13}$~Meum est enim omne primogenitum~: ex quo percussi primogenitos in terra \AE gypti, sanctificavi mihi quidquid primum nascitur in Isra\"el~: ab homine usque ad pecus, mei sunt. Ego Dominus.
${}^{14}$~Locutusque est Dominus ad Moysen in deserto Sinai, dicens~:
${}^{15}$~Numera filios Levi per domos patrum suorum et familias, omnem masculum ab uno mense et supra.
${}^{16}$~Numeravit Moyses, ut pr\ae ceperat Dominus,
${}^{17}$~et inventi sunt filii Levi per nomina sua, Gerson et Caath et Merari.
${}^{18}$~Filii Gerson~: Lebni et Semei.
${}^{19}$~Filii Caath~: Amram et Jesaar, Hebron et Oziel.
${}^{20}$~Filii Merari~: Moholi et Musi.
${}^{21}$~De Gerson fuere famili\ae\ du\ae , Lebnitica, et Semeitica~:
${}^{22}$~quarum numeratus est populus sexus masculini ab uno mense et supra, septem millia quingenti.
${}^{23}$~Hi post tabernaculum metabuntur ad occidentem,
${}^{24}$~sub principe Eliasaph filio La\"el.
${}^{25}$~Et habebunt excubias in tabernaculo fœderis,
${}^{26}$~ipsum tabernaculum et operimentum ejus, tentorium quod trahitur ante fores tecti fœderis, et cortinas atrii~: tentorium quoque quod appenditur in introitu atrii tabernaculi, et quidquid ad ritum altaris pertinet, funes tabernaculi et omnia utensilia ejus.
${}^{27}$~Cognatio Caath habebit populos Amramitas et Jesaaritas et Hebronitas et Ozielitas. H\ae\ sunt famili\ae\ Caathitarum recensit\ae\ per nomina sua.
${}^{28}$~Omnes generis masculini ab uno mense et supra, octo millia sexcenti habebunt excubias sanctuarii,
${}^{29}$~et castrametabuntur ad meridianam plagam.
${}^{30}$~Princepsque eorum erit Elisaphan filius Oziel~:
${}^{31}$~et custodient arcam, mensamque et candelabrum, altaria et vasa sanctuarii, in quibus ministratur, et velum, cunctamque hujuscemodi supellectilem.
${}^{32}$~Princeps autem principum Levitarum Eleazar filius Aaron sacerdotis, erit super excubitores custodi\ae\ sanctuarii.
${}^{33}$~At vero de Merari erunt populi Moholit\ae\ et Musit\ae\ recensiti per nomina sua~:
${}^{34}$~omnes generis masculini ab uno mense et supra, sex millia ducenti.
${}^{35}$~Princeps eorum Suriel filius Abihaiel~: in plaga septentrionali castrametabuntur.
${}^{36}$~Erunt sub custodia eorum tabul\ae\ tabernaculi et vectes, et column\ae\ ac bases earum, et omnia qu\ae\ ad cultum hujuscemodi pertinent~:
${}^{37}$~column\ae que atrii per circuitum cum basibus suis, et paxilli cum funibus.
${}^{38}$~Castrametabuntur ante tabernaculum fœderis, id est, ad orientalem plagam, Moyses et Aaron cum filiis suis, habentes custodiam sanctuarii in medio filiorum Isra\"el. Quisquis alienus accesserit, morietur.
${}^{39}$~Omnes Levit\ae , quos numeraverunt Moyses et Aaron juxta pr\ae ceptum Domini per familias suas in genere masculino a mense uno et supra, fuerunt viginti duo millia.


${}^{40}$~Et ait Dominus ad Moysen~: Numera primogenitos sexus masculini de filiis Isra\"el ab uno mense et supra, et habebis summam eorum.
${}^{41}$~Tollesque Levitas mihi pro omni primogenito filiorum Isra\"el~: ego sum Dominus~: et pecora eorum pro universis primogenitis pecorum filiorum Isra\"el.
${}^{42}$~Recensuit Moyses, sicut pr\ae ceperat Dominus, primogenitos filiorum Isra\"el~:
${}^{43}$~et fuerunt masculi per nomina sua, a mense uno et supra, viginti duo millia ducenti septuaginta tres.
${}^{44}$~Locutusque est Dominus ad Moysen, dicens~:
${}^{45}$~Tolle Levitas pro primogenitis filiorum Isra\"el, et pecora Levitarum pro pecoribus eorum, eruntque Levit\ae\ mei. Ego sum Dominus.
${}^{46}$~In pretio autem ducentorum septuaginta trium, qui excedunt numerum Levitarum de primogenitis filiorum Isra\"el,
${}^{47}$~accipies quinque siclos per singula capita ad mensuram sanctuarii (siclus habet viginti obolos)~:
${}^{48}$~dabisque pecuniam Aaron et filiis ejus pretium eorum qui supra sunt.
${}^{49}$~Tulit igitur Moyses pecuniam eorum, qui fuerant amplius, et quos redemerant a Levitis,
${}^{50}$~pro primogenitis filiorum Isra\"el, mille trecentorum sexaginta quinque siclorum juxta pondus sanctuarii~:
${}^{51}$~et dedit eam Aaron et filiis ejus juxta verbum quod pr\ae ceperat sibi Dominus.

\bchapter{4}
\lettrine[lines=10,image=true,loversize=0.05,lraise=-0.03]{L}{}ocutusque est Dominus ad Moysen et Aaron, dicens~:
${}^{2}$~Tolle summam filiorum Caath de medio Levitarum per domos et familias suas,
${}^{3}$~a trigesimo anno et supra, usque ad quinquagesimum annum, omnium qui ingrediuntur ut stent et ministrent in tabernaculo fœderis.
${}^{4}$~Hic est cultus filiorum Caath~: tabernaculum fœderis, et Sanctum sanctorum
${}^{5}$~ingredientur Aaron et filii ejus, quando movenda sunt castra, et deponent velum quod pendet ante fores, involventque eo arcam testimonii,
${}^{6}$~et operient rursum velamine janthinarum pellium, extendentque desuper pallium totum hyacinthinum, et inducent vectes.
${}^{7}$~Mensam quoque propositionis involvent hyacinthino pallio, et ponent cum ea thuribula et mortariola, cyathos et crateras ad liba fundenda~: panes semper in ea erunt~:
${}^{8}$~extendentque desuper pallium coccineum, quod rursum operient velamento janthinarum pellium, et inducent vectes.
${}^{9}$~Sument et pallium hyacinthinum, quo operient candelabrum cum lucernis et forcipibus suis et emunctoriis et cunctis vasis olei, qu\ae\ ad concinnandas lucernas necessaria sunt~:
${}^{10}$~et super omnia ponent operimentum janthinarum pellium, et inducent vectes.
${}^{11}$~Necnon et altare aureum involvent hyacinthino vestimento, et extendent desuper operimentum janthinarum pellium, inducentque vectes.
${}^{12}$~Omnia vasa, quibus ministratur in sanctuario, involvent hyacinthino pallio, et extendent desuper operimentum janthinarum pellium, inducentque vectes.
${}^{13}$~Sed et altare mundabunt cinere, et involvent illud purpureo vestimento,
${}^{14}$~ponentque cum eo omnia vasa, quibus in ministerio ejus utuntur, id est, ignium receptacula, fuscinulas ac tridentes, uncinos et batilla. Cuncta vasa altaris operient simul velamine janthinarum pellium, et inducent vectes.
${}^{15}$~Cumque involverint Aaron et filii ejus sanctuarium et omnia vasa ejus in commotione castrorum, tunc intrabunt filii Caath ut portent involuta~: et non tangent vasa sanctuarii, ne moriantur. Ista sunt onera filiorum Caath in tabernaculo fœderis~:
${}^{16}$~super quos erit Eleazar filius Aaron sacerdotis, ad cujus curam pertinet oleum ad concinnandas lucernas, et compositionis incensum, et sacrificium, quod semper offertur, et oleum unctionis, et quidquid ad cultum tabernaculi pertinet, omniumque vasorum, qu\ae\ in sanctuario sunt.
${}^{17}$~Locutusque est Dominus ad Moysen et Aaron, dicens~:
${}^{18}$~Nolite perdere populum Caath de medio Levitarum~:
${}^{19}$~sed hoc facite eis, ut vivant, et non moriantur, si tetigerint Sancta sanctorum. Aaron et filii ejus intrabunt, ipsique disponent opera singulorum, et divident quid portare quis debeat.
${}^{20}$~Alii nulla curiositate videant qu\ae\ sunt in sanctuario priusquam involvantur, alioquin morientur.
${}^{21}$~Locutusque est Dominus ad Moysen, dicens~:
${}^{22}$~Tolle summam etiam filiorum Gerson per domos ac familias et cognationes suas,
${}^{23}$~a triginta annis et supra, usque ad annos quinquaginta. Numera omnes qui ingrediuntur et ministrant in tabernaculo fœderis.
${}^{24}$~Hoc est officium famili\ae\ Gersonitarum,
${}^{25}$~ut portent cortinas tabernaculi et tectum fœderis, operimentum aliud, et super omnia velamen janthinum tentoriumque quod pendet in introitu tabernaculi fœderis,
${}^{26}$~cortinas atrii, et velum in introitu quod est ante tabernaculum. Omnia qu\ae\ ad altare pertinent, funiculos, et vasa ministerii,
${}^{27}$~jubente Aaron et filiis ejus, portabunt filii Gerson~: et scient singuli cui debeant oneri mancipari.
${}^{28}$~Hic est cultus famili\ae\ Gersonitarum in tabernaculo fœderis, eruntque sub manu Ithamar filii Aaron sacerdotis.
${}^{29}$~Filios quoque Merari per familias et domos patrum suorum recensebis,
${}^{30}$~a triginta annis et supra, usque ad annos quinquaginta, omnes qui ingrediuntur ad officium ministerii sui et cultum fœderis testimonii.
${}^{31}$~H\ae c sunt onera eorum~: portabunt tabulas tabernaculi et vectes ejus, columnas ac bases earum,
${}^{32}$~columnas quoque atrii per circuitum cum basibus et paxillis et funibus suis. Omnia vasa et supellectilem ad numerum accipient, sicque portabunt.
${}^{33}$~Hoc est officium famili\ae\ Meraritarum et ministerium in tabernaculo fœderis~: eruntque sub manu Ithamar filii Aaron sacerdotis.
${}^{34}$~Recensuerunt igitur Moyses et Aaron et principes synagog\ae\ filios Caath per cognationes et domos patrum suorum,
${}^{35}$~a triginta annis et supra, usque ad annum quinquagesimum, omnes qui ingrediuntur ad ministerium tabernaculi fœderis~:
${}^{36}$~et inventi sunt duo millia septingenti quinquaginta.
${}^{37}$~Hic est numerus populi Caath qui intrant tabernaculum fœderis~: hos numeravit Moyses et Aaron juxta sermonem Domini per manum Moysi.
${}^{38}$~Numerati sunt et filii Gerson per cognationes et domos patrum suorum,
${}^{39}$~a triginta annos et supra, usque ad quinquagesimum annum, omnes qui ingrediuntur ut ministrent in tabernaculo fœderis~:
${}^{40}$~et inventi sunt duo millia sexcenti triginta.
${}^{41}$~Hic est populus Gersonitarum, quos numeraverunt Moyses et Aaron juxta verbum Domini.
${}^{42}$~Numerati sunt et filii Merari per cognationes et domos patrum suorum,
${}^{43}$~a triginta annis et supra, usque ad annum quinquagesimum, omnes qui ingrediuntur ad explendos ritus tabernaculi fœderis~:
${}^{44}$~et inventi sunt tria millia ducenti.
${}^{45}$~Hic est numerus filiorum Merari, quos recensuerunt Moyses et Aaron juxta imperium Domini per manum Moysi.
${}^{46}$~Omnes qui recensiti sunt de Levitis, et quos recenseri fecit ad nomen Moyses et Aaron, et principes Isra\"el per cognationes et domos patrum suorum,
${}^{47}$~a triginta annis et supra, usque ad annum quinquagesimum, ingredientes ad ministerium tabernaculi, et onera portanda,
${}^{48}$~fuerunt simul octo millia quingenti octoginta.
${}^{49}$~Juxta verbum Domini recensuit eos Moyses, unumquemque juxta officium et onera sua, sicut pr\ae ceperat ei Dominus.

\bchapter{5}
\lettrine[lines=10,image=true,loversize=0.05,lraise=-0.03]{L}{}ocutusque est Dominus ad Moysen, dicens~:
${}^{2}$~Pr\ae cipe filiis Isra\"el, ut ejiciant de castris omnem leprosum, et qui semine fluit, pollutusque est super mortuo~:
${}^{3}$~tam masculum quam feminam ejicite de castris, ne contaminent ea cum habitaverint vobiscum.
${}^{4}$~Feceruntque ita filii Isra\"el, et ejecerunt eos extra castra, sicut locutus erat Dominus Moysi.
${}^{5}$~Locutusque est Dominus ad Moysen, dicens~:
${}^{6}$~Loquere ad filios Isra\"el~: Vir, sive mulier, cum fecerint ex omnibus peccatis, qu\ae\ solent hominibus accidere, et per negligentiam transgressi fuerint mandatum Domini, atque deliquerint,
${}^{7}$~confitebuntur peccatum suum, et reddent ipsum caput, quintamque partem desuper, ei in quem peccaverint.
${}^{8}$~Sin autem non fuerit qui recipiat, dabunt Domino, et erit sacerdotis, excepto ariete, qui offertur pro expiatione, ut sit placabilis hostia.
${}^{9}$~Omnes quoque primiti\ae , quas offerunt filii Isra\"el, ad sacerdotem pertinent~:
${}^{10}$~et quidquid in sanctuarium offertur a singulis, et traditur manibus sacerdotis, ipsius erit.


${}^{11}$~Locutusque est Dominus ad Moysen, dicens~:
${}^{12}$~Loquere ad filios Isra\"el, et dices ad eos~: Vir cujus uxor erraverit, maritumque contemnens
${}^{13}$~dormierit cum altero viro, et hoc maritus deprehendere non quiverit, sed latet adulterium, et testibus argui non potest, quia non est inventa in stupro~:
${}^{14}$~si spiritus zelotypi\ae\ concitaverit virum contra uxorem suam, qu\ae\ vel polluta est, vel falsa suspicione appetitur~:
${}^{15}$~adducet eam ad sacerdotem, et offeret oblationem pro illa, decimam partem sati farin\ae\ hordeace\ae~: non fundet super eam oleum, nec imponet thus~: quia sacrificium zelotypi\ae\ est, et oblatio investigans adulterium.
${}^{16}$~Offeret igitur eam sacerdos, et statuet coram Domino,
${}^{17}$~assumetque aquam sanctam in vase fictili, et pauxillum terr\ae\ de pavimento tabernaculi mittet in eam.
${}^{18}$~Cumque steterit mulier in conspectu Domini, discooperiet caput ejus, et ponet super manus illius sacrificium recordationis, et oblationem zelotypi\ae~: ipse autem tenebit aquas amarissimas, in quibus cum execratione maledicta congessit.
${}^{19}$~Adjurabitque eam, et dicet~: Si non dormivit vir alienus tecum, et si non polluta es deserto mariti thoro, non te nocebunt aqu\ae\ ist\ae\ amarissim\ae , in quas maledicta congessi.
${}^{20}$~Sin autem declinasti a viro tuo, atque polluta es, et concubuisti cum altero viro~:
${}^{21}$~his maledictionibus subjacebis~: det te Dominus in maledictionem, exemplumque cunctorum in populo suo~: putrescere faciat femur tuum, et tumens uterus tuus disrumpatur.
${}^{22}$~Ingrediantur aqu\ae\ maledict\ae\ in ventrem tuum, et utero tumescente putrescat femur. Et respondebit mulier~: Amen, amen.
${}^{23}$~Scribetque sacerdos in libello ista maledicta, et delebit ea aquis amarissimis, in quas maledicta congessit,
${}^{24}$~et dabit ei bibere. Quas cum exhauserit,
${}^{25}$~tollet sacerdos de manu ejus sacrificium zelotypi\ae , et elevabit illud coram Domino, imponetque illud super altare, ita dumtaxat ut prius~:
${}^{26}$~pugillum sacrificii tollat de eo, quod offertur, et incendat super altare~: et sic potum det mulieri aquas amarissimas.
${}^{27}$~Quas cum biberit, si polluta est, et contempto viro adulterii rea, pertransibunt eam aqu\ae\ maledictionis, et inflato ventre, computrescet femur~: eritque mulier in maledictionem, et in exemplum omni populo.
${}^{28}$~Quod si polluta non fuerit, erit innoxia, et faciet liberos.
${}^{29}$~Ista est lex zelotypi\ae . Si declinaverit mulier a viro suo, et si polluta fuerit,
${}^{30}$~maritusque zelotypi\ae\ spiritu concitatus adduxerit eam in conspectu Domini, et fecerit ei sacerdos juxta omnia qu\ae\ scripta sunt~:
${}^{31}$~maritus absque culpa erit, et illa recipiet iniquitatem suam.

\bchapter{6}
\lettrine[lines=10,image=true,loversize=0.05,lraise=-0.03]{L}{}ocutusque est Dominus ad Moysen, dicens~:
${}^{2}$~Loquere ad filios Isra\"el, et dices ad eos~: Vir, sive mulier, cum fecerint votum ut sanctificentur, et se voluerint Domino consecrare~:
${}^{3}$~a vino, et omni quod inebriare potest, abstinebunt. Acetum ex vino, et ex qualibet alia potione, et quidquid de uva exprimitur, non bibent~: uvas recentes siccasque non comedent
${}^{4}$~cunctis diebus quibus ex voto Domino consecrantur~: quidquid ex vinea esse potest, ab uva passa usque ad acinum non comedent.
${}^{5}$~Omni tempore separationis su\ae\ novacula non transibit per caput ejus usque ad completum diem, quo Domino consecratur. Sanctus erit, crescente c\ae sarie capitis ejus.
${}^{6}$~Omni tempore consecrationis su\ae , super mortuum non ingredietur,
${}^{7}$~nec super patris quidem et matris et fratris sororisque funere contaminabitur, quia consecratio Dei sui super caput ejus est.
${}^{8}$~Omnibus diebus separationis su\ae\ sanctus erit Domino.
${}^{9}$~Sin autem mortuus fuerit subito quispiam coram eo, polluetur caput consecrationis ejus~: quod radet illico in eadem die purgationis su\ae , et rursum septima.
${}^{10}$~In octava autem die offeret duos turtures, vel duos pullos columb\ae\ sacerdoti in introitu fœderis testimonii.
${}^{11}$~Facietque sacerdos unum pro peccato, et alterum in holocaustum, et deprecabitur pro eo, quia peccavit super mortuo~: sanctificabitque caput ejus in die illo~:
${}^{12}$~et consecrabit Domino dies separationis illius, offerens agnum anniculum pro peccato~: ita tamen ut dies priores irriti fiant, quoniam polluta est sanctificatio ejus.
${}^{13}$~Ista est lex consecrationis. Cum dies, quos ex voto decreverat, complebuntur, adducet eum ad ostium tabernaculi fœderis,
${}^{14}$~et offeret oblationes ejus Domino, agnum anniculum immaculatum in holocaustum, et ovem anniculam immaculatam pro peccato, et arietem immaculatum, hostiam pacificam,
${}^{15}$~canistrum quoque panum azymorum qui conspersi sint oleo, et lagana absque fermento uncta oleo, ac libamina singulorum~:
${}^{16}$~qu\ae\ offeret sacerdos coram Domino, et faciet tam pro peccato, quam in holocaustum.
${}^{17}$~Arietem vero immolabit hostiam pacificam Domino, offerens simul canistrum azymorum, et libamenta qu\ae\ ex more debentur.
${}^{18}$~Tunc radetur nazar\ae us ante ostium tabernaculi fœderis c\ae sarie consecrationis su\ae~: tolletque capillos ejus, et ponet super ignem, qui est suppositus sacrificio pacificorum~:
${}^{19}$~et armum coctum arietis, tortamque absque fermento unam de canistro, et laganum azymum unum, et tradet in manus nazar\ae i, postquam rasum fuerit caput ejus.
${}^{20}$~Susceptaque rursum ab eo, elevabit in conspectu Domini~: et sanctificata sacerdotis erunt, sicut pectusculum, quod separari jussum est, et femur. Post h\ae c, potest bibere nazar\ae us vinum.
${}^{21}$~Ista est lex nazar\ae i, cum voverit oblationem suam Domino tempore consecrationis su\ae , exceptis his, qu\ae\ invenerit manus ejus~: juxta quod mente devoverat, ita faciet ad perfectionem sanctificationis su\ae .


${}^{22}$~Locutusque est Dominus ad Moysen, dicens~:
${}^{23}$~Loquere Aaron et filiis ejus~: Sic benedicetis filiis Isra\"el, et dicetis eis~:
${}^{24}$~Benedicat tibi Dominus, et custodiat te.
${}^{25}$~Ostendat Dominus faciem suam tibi, et misereatur tui.
${}^{26}$~Convertat Dominus vultum suum ad te, et det tibi pacem.
${}^{27}$~Invocabuntque nomen meum super filios Isra\"el, et ego benedicam eis.

\bchapter{7}
\lettrine[lines=10,image=true,loversize=0.05,lraise=-0.03]{F}{}actum est autem in die qua complevit Moyses tabernaculum, et erexit illud, unxitque et sanctificavit cum omnibus vasis suis, altare similiter et omnia vasa ejus~:
${}^{2}$~obtulerunt principes Isra\"el et capita familiarum, qui erant per singulas tribus, pr\ae fectique eorum, qui numerati fuerant,
${}^{3}$~munera coram Domino sex plaustra tecta cum duodecim bobus. Unum plaustrum obtulere duo duces, et unum bovem singuli, obtuleruntque ea in conspectu tabernaculi.
${}^{4}$~Ait autem Dominus ad Moysen~:
${}^{5}$~Suscipe ab eis ut serviant in ministerio tabernaculi, et trades ea Levitis juxta ordinem ministerii sui.
${}^{6}$~Itaque cum suscepisset Moyses plaustra et boves, tradidit eos Levitis.
${}^{7}$~Duo plaustra et quatuor boves dedit filiis Gerson, juxta id quod habebant necessarium.
${}^{8}$~Quatuor alia plaustra et octo boves dedit filiis Merari secundum officia et cultum suum, sub manu Ithamar filii Aaron sacerdotis.
${}^{9}$~Filiis autem Caath non dedit plaustra et boves~: quia in sanctuario serviunt, et onera propriis portant humeris.


${}^{10}$~Igitur obtulerunt duces in dedicationem altaris, die qua unctum est, oblationem suam ante altare.
${}^{11}$~Dixitque Dominus ad Moysen~: Singuli duces per singulos dies offerant munera in dedicationem altaris.
${}^{12}$~Primo die obtulit oblationem suam Nahasson filius Aminadab de tribu Juda~:
${}^{13}$~fueruntque in ea acetabulum argenteum pondo centum triginta siclorum, phiala argentea habens septuaginta siclos, juxta pondus sanctuarii, utrumque plenum simila conspersa oleo in sacrificium~:
${}^{14}$~mortariolum ex decem siclis aureis plenum incenso~:
${}^{15}$~bovem de armento, et arietem, et agnum anniculum in holocaustum~:
${}^{16}$~hircumque pro peccato~:
${}^{17}$~et in sacrificio pacificorum boves duos, arietes quinque, hircos quinque, agnos anniculos quinque. H\ae c est oblatio Nahasson filii Aminadab.
${}^{18}$~Secundo die obtulit Nathana\"el filius Suar, dux de tribu Issachar,
${}^{19}$~acetabulum argenteum appendens centum triginta siclos, phialam argenteam habentem septuaginta siclos, juxta pondus sanctuarii, utrumque plenum simila conspersa oleo in sacrificium~:
${}^{20}$~mortariolum aureum habens decem siclos plenum incenso~:
${}^{21}$~bovem de armento, et arietem, et agnum anniculum in holocaustum~:
${}^{22}$~hircumque pro peccato~:
${}^{23}$~et in sacrificio pacificorum boves duos, arietes quinque, hircos quinque, agnos anniculos quinque. H\ae c fuit oblatio Nathana\"el filii Suar.
${}^{24}$~Tertio die princeps filiorum Zabulon, Eliab filius Helon,
${}^{25}$~obtulit acetabulum argenteum appendens centum triginta siclos, phialam argenteam habentem septuaginta siclos, ad pondus sanctuarii, utrumque plenum simila conspersa oleo in sacrificium~:
${}^{26}$~mortariolum aureum appendens decem siclos, plenum incenso~:
${}^{27}$~bovem de armento, et arietem, et agnum anniculum in holocaustum~:
${}^{28}$~hircumque pro peccato~:
${}^{29}$~et in sacrificio pacificorum boves duos, arietes quinque, hircos quinque, agnos anniculos quinque. H\ae c est oblatio Eliab filii Helon.
${}^{30}$~Die quarto princeps filiorum Ruben, Elisur filius Sedeur,
${}^{31}$~obtulit acetabulum argenteum appendens centum triginta siclos, phialam argenteam habentem septuaginta siclos, ad pondus sanctuarii, utrumque plenum simila conspersa oleo in sacrificum~:
${}^{32}$~mortariolum aureum appendens decem siclos, plenum incenso~:
${}^{33}$~bovem de armento, et arietem, et agnum anniculum in holocaustum~:
${}^{34}$~hircumque pro peccato~:
${}^{35}$~et in hostias pacificorum boves duos, arietes quinque, hircos quinque, agnos anniculos quinque. H\ae c fuit oblatio Elisur filii Sedeur.
${}^{36}$~Die quinto princeps filiorum Simeon, Salamiel filius Surisaddai,
${}^{37}$~obtulit acetabulum argenteum appendens centum triginta siclos, phialam argenteam habentem septuaginta siclos, ad pondus sanctuarii, utrumque plenum simila conspersa oleo in sacrificum~:
${}^{38}$~mortariolum aureum appendens decem siclos, plenum incenso~:
${}^{39}$~bovem de armento, et arietem, et agnum anniculum in holocaustum~:
${}^{40}$~hircumque pro peccato~:
${}^{41}$~et in hostias pacificorum boves duos, arietes quinque, hircos quinque, agnos anniculos quinque. H\ae c fuit oblatio Salamiel filii Surisaddai.
${}^{42}$~Die sexto princeps filiorum Gad, Eliasaph filius Duel,
${}^{43}$~obtulit acetabulum argenteum appendens centum triginta siclos, phialam argenteam habentem septuaginta siclos, ad pondus sanctuarii, utrumque plenum simila conspersa oleo in sacrificum~:
${}^{44}$~mortariolum aureum appendens decem siclos, plenum incenso~:
${}^{45}$~bovem de armento, et arietem, et agnum anniculum in holocaustum~:
${}^{46}$~hircumque pro peccato~:
${}^{47}$~et in hostias pacificorum boves duos, arietes quinque, hircos quinque, agnos anniculos quinque. H\ae c fuit oblatio Eliasaph filii Duel.
${}^{48}$~Die septimo princeps filiorum Ephraim, Elisama filius Ammiud,
${}^{49}$~obtulit acetabulum argenteum appendens centum triginta siclos, phialam argenteam habentem septuaginta siclos, ad pondus sanctuarii, utrumque plenum simila conspersa oleo in sacrificum~:
${}^{50}$~mortariolum aureum appendens decem siclos, plenum incenso~:
${}^{51}$~bovem de armento, et arietem, et agnum anniculum in holocaustum~:
${}^{52}$~hircumque pro peccato~:
${}^{53}$~et in hostias pacificorum boves duos, arietes quinque, hircos quinque, agnos anniculos quinque. H\ae c fuit oblatio Elisama filii Ammiud.
${}^{54}$~Die octavo, princeps filiorum Manasse, Gamaliel filius Phadassur,
${}^{55}$~obtulit acetabulum argenteum appendens centum triginta siclos, phialam argenteam habentem septuaginta siclos, ad pondus sanctuarii, utrumque plenum simila conspersa oleo in sacrificum~:
${}^{56}$~mortariolum aureum appendens decem siclos, plenum incenso~:
${}^{57}$~bovem de armento, et arietem, et agnum anniculum in holocaustum~:
${}^{58}$~hircumque pro peccato~:
${}^{59}$~et in hostias pacificorum boves duos, arietes quinque, hircos quinque, agnos anniculos quinque. H\ae c fuit oblatio Gamaliel filii Phadassur.
${}^{60}$~Die nono princeps filiorum Benjamin, Abidan filius Gedeonis,
${}^{61}$~obtulit acetabulum argenteum appendens centum triginta siclos, phialam argenteam habentem septuaginta siclos, ad pondus sanctuarii, utrumque plenum simila conspersa oleo in sacrificum~:
${}^{62}$~et mortariolum aureum appendens decem siclos, plenum incenso~:
${}^{63}$~bovem de armento, et arietem, et agnum anniculum in holocaustum~:
${}^{64}$~hircumque pro peccato~:
${}^{65}$~et in hostias pacificorum boves duos, arietes quinque, hircos quinque, agnos anniculos quinque. H\ae c fuit oblatio Abidan filii Gedeonis.
${}^{66}$~Die decimo princeps filiorum Dan, Ahiezer filius Ammisaddai,
${}^{67}$~obtulit acetabulum argenteum appendens centum triginta siclos, phialam argenteam habentem septuaginta siclos, ad pondus sanctuarii, utrumque plenum simila conspersa oleo in sacrificum~:
${}^{68}$~mortariolum aureum appendens decem siclos, plenum incenso~:
${}^{69}$~bovem de armento, et arietem, et agnum anniculum in holocaustum~:
${}^{70}$~hircumque pro peccato~:
${}^{71}$~et in hostias pacificorum boves duos, arietes quinque, hircos quinque, agnos anniculos quinque. H\ae c fuit oblatio Ahiezer filii Ammisaddai.
${}^{72}$~Die undecimo princeps filiorum Aser, Phegiel filius Ochran,
${}^{73}$~obtulit acetabulum argenteum appendens centum triginta siclos, phialam argenteam habentem septuaginta siclos, ad pondus sanctuarii, utrumque plenum simila conspersa oleo in sacrificum~:
${}^{74}$~mortariolum aureum appendens decem siclos, plenum incenso~:
${}^{75}$~bovem de armento, et arietem, et agnum anniculum in holocaustum~:
${}^{76}$~hircumque pro peccato~:
${}^{77}$~et in hostias pacificorum boves duos, arietes quinque, hircos quinque, agnos anniculos quinque. H\ae c fuit oblatio Phegiel filii Ochran.
${}^{78}$~Die duodecimo princeps filiorum Nephthali, Ahira filius Enan,
${}^{79}$~obtulit acetabulum argenteum appendens centum triginta siclos, phialam argenteam habentem septuaginta siclos, ad pondus sanctuarii, utrumque plenum simila oleo conspersa in sacrificum~:
${}^{80}$~mortariolum aureum appendens decem siclos, plenum incenso~:
${}^{81}$~bovem de armento, et arietem, et agnum anniculum in holocaustum~:
${}^{82}$~hircumque pro peccato~:
${}^{83}$~et in hostias pacificorum boves duos, arietes quinque, hircos quinque, agnos anniculos quinque. H\ae c fuit oblatio Ahira filii Enan.


${}^{84}$~H\ae c in dedicatione altaris oblata sunt a principibus Isra\"el, in die qua consecratum est~: acetabula argentea duodecim~: phial\ae\ argente\ae\ duodecim~: mortariola aurea duodecim~:
${}^{85}$~ita ut centum triginta siclos argenti haberet unum acetabulum, et septuaginta siclos haberet una phiala~: id est, in commune vasorum omnium ex argento sicli duo millia quadringenti, pondere sanctuarii~:
${}^{86}$~mortariola aurea duodecim plena incenso, denos siclos appendentia pondere sanctuarii~: id est, simul auri sicli centum viginti~:
${}^{87}$~boves de armento in holocaustum duodecim, arietes duodecim, agni anniculi duodecim, et libamenta eorum~: hirci duodecim pro peccato.
${}^{88}$~In hostias pacificorum, boves viginti quatuor, arietes sexaginta, hirci sexaginta, agni anniculi sexaginta. H\ae c oblata sunt in dedicatione altaris, quando unctum est.
${}^{89}$~Cumque ingrederetur Moyses tabernaculum fœderis, ut consuleret oraculum, audiebat vocem loquentis ad se de propitiatorio quod erat super arcam testimonii inter duos cherubim, unde et loquebatur ei.

\bchapter{8}
\lettrine[lines=10,image=true,loversize=0.05,lraise=-0.03]{L}{}ocutusque est Dominus ad Moysen, dicens~:
${}^{2}$~Loquere Aaron, et dices ad eum~: Cum posueris septem lucernas, candelabrum in australi parte erigatur. Hoc igitur pr\ae cipe ut lucern\ae\ contra boream e regione respiciant ad mensam panum propositionis, contra eam partem, quam candelabrum respicit, lucere debebunt.
${}^{3}$~Fecitque Aaron, et imposuit lucernas super candelabrum, ut pr\ae ceperat Dominus Moysi.
${}^{4}$~H\ae c autem erat factura candelabri, ex auro ductili, tam medius stipes, quam cuncta qu\ae\ ex utroque calamorum latere nascebantur~: juxta exemplum quod ostendit Dominus Moysi, ita operatus est candelabrum.


${}^{5}$~Et locutus est Dominus ad Moysen, dicens~:
${}^{6}$~Tolle Levitas de medio filiorum Isra\"el, et purificabis eos
${}^{7}$~juxta hunc ritum~: aspergantur aqua lustrationis, et radant omnes pilos carnis su\ae . Cumque laverint vestimenta sua, et mundati fuerint,
${}^{8}$~tollent bovem de armentis, et libamentum ejus similam oleo conspersam~: bovem autem alterum de armento tu accipies pro peccato~:
${}^{9}$~et applicabis Levitas coram tabernaculo fœderis, convocata omni multitudine filiorum Isra\"el.
${}^{10}$~Cumque Levit\ae\ fuerint coram Domino, ponent filii Isra\"el manus suas super eos.
${}^{11}$~Et offeret Aaron Levitas, munus in conspectu Domini a filiis Isra\"el, ut serviant in ministerio ejus.
${}^{12}$~Levit\ae\ quoque ponent manus suas super capita boum, e quibus unum facies pro peccato, et alterum in holocaustum Domini, ut depreceris pro eis.
${}^{13}$~Statuesque Levitas in conspectu Aaron et filiorum ejus, et consecrabis oblatos Domino,
${}^{14}$~ac separabis de medio filiorum Isra\"el, ut sint mei.
${}^{15}$~Et postea ingredientur tabernaculum fœderis, ut serviant mihi. Sicque purificabis et consecrabis eos in oblationem Domini~: quoniam dono donati sunt mihi a filiis Isra\"el.
${}^{16}$~Pro primogenitis qu\ae\ aperiunt omnem vulvam in Isra\"el, accepi eos.
${}^{17}$~Mea sunt enim omnia primogenita filiorum Isra\"el, tam ex hominibus quam ex jumentis. Ex die quo percussi omne primogenitum in terra \AE gypti, sanctificavi eos mihi~:
${}^{18}$~et tuli Levitas pro cunctis primogenitis filiorum Isra\"el,
${}^{19}$~tradidique eos dono Aaron et filiis ejus de medio populi, ut serviant mihi pro Isra\"el in tabernaculo fœderis, et orent pro eis ne sit in populo plaga, si ausi fuerint accedere ad sanctuarium.
${}^{20}$~Feceruntque Moyses et Aaron et omnis multitudo filiorum Isra\"el super Levitas qu\ae\ pr\ae ceperat Dominus Moysi~:
${}^{21}$~purificatique sunt, et laverunt vestimenta sua. Elevavitque eos Aaron in conspectu Domini, et oravit pro eis,
${}^{22}$~ut purificati ingrederentur ad officia sua in tabernaculum fœderis coram Aaron et filiis ejus. Sicut pr\ae ceperat Dominus Moysi de Levitis, ita factum est.
${}^{23}$~Locutusque est Dominus ad Moysen, dicens~:
${}^{24}$~H\ae c est lex Levitarum~: a viginti quinque annis et supra, ingredientur ut ministrent in tabernaculo fœderis.
${}^{25}$~Cumque quinquagesimum annum \ae tatis impleverint, servire cessabunt,
${}^{26}$~eruntque ministri fratrum suorum in tabernaculo fœderis, ut custodiant qu\ae\ sibi fuerunt commendata~: opera autem ipsa non faciant. Sic dispones Levitis in custodiis suis.

\bchapter{9}
\lettrine[lines=10,image=true,loversize=0.05,lraise=-0.03]{L}{}ocutus est Dominus ad Moysen in deserto Sinai anno secundo, postquam egressi sunt de terra \AE gypti, mense primo, dicens~:
${}^{2}$~Faciant filii Isra\"el Phase in tempore suo,
${}^{3}$~quartadecima die mensis hujus ad vesperam, juxta omnes c\ae remonias et justificationes ejus.
${}^{4}$~Pr\ae cepitque Moyses filiis Isra\"el ut facerent Phase.
${}^{5}$~Qui fecerunt tempore suo, quartadecima die mensis ad vesperam, in monte Sinai. Juxta omnia qu\ae\ mandaverat Dominus Moysi, fecerunt filii Isra\"el.
${}^{6}$~Ecce autem quidam immundi super anima hominis, qui non poterant facere Phase in die illo, accedentes ad Moysen et Aaron,
${}^{7}$~dixerunt eis~: Immundi sumus super anima hominis~: quare fraudamur ut non valeamus oblationem offerre Domino in tempore suo inter filios Isra\"el~?
${}^{8}$~Quibus respondit Moyses~: State ut consulam quid pr\ae cipiat Dominus de vobis.
${}^{9}$~Locutusque est Dominus ad Moysen, dicens~:
${}^{10}$~Loquere filiis Isra\"el~: Homo, qui fuerit immundus super anima, sive in via procul in gente vestra, faciat Phase Domino
${}^{11}$~in mense secundo, quartadecima die mensis ad vesperam. Cum azymis et lactucis agrestibus comedent illud~:
${}^{12}$~non relinquent ex eo quippiam usque mane, et os ejus non confringent~: omnem ritum Phase observabunt.
${}^{13}$~Si quis autem et mundus est, et in itinere non fuit, et tamen non fecit Phase, exterminabitur anima illa de populis suis, quia sacrificium Domino non obtulit tempore suo~: peccatum suum ipse portabit.
${}^{14}$~Peregrinus quoque et advena si fuerint apud vos, facient Phase Domino juxta c\ae remonias et justificationes ejus. Pr\ae ceptum idem erit apud vos tam adven\ae\ quam indigen\ae .


${}^{15}$~Igitur die qua erectum est tabernaculum, operuit illud nubes. A vespere autem super tentorium erat quasi species ignis usque mane.
${}^{16}$~Sic fiebat jugiter~: per diem operiebat illud nubes, et per noctem quasi species ignis.
${}^{17}$~Cumque ablata fuisset nubes, qu\ae\ tabernaculum protegebat, tunc proficiscebantur filii Isra\"el~: et in loco ubi stetisset nubes, ibi castrametabantur.
${}^{18}$~Ad imperium Domini proficiscebantur, et ad imperium illius figebant tabernaculum. Cunctis diebus quibus stabat nubes super tabernaculum, manebant in eodem loco~:
${}^{19}$~et si evenisset ut multo tempore maneret super illud, erant filii Isra\"el in excubiis Domini, et non proficiscebantur
${}^{20}$~quot diebus fuisset nubes super tabernaculum. Ad imperium Domini erigebant tentoria, et ad imperium illius deponebant.
${}^{21}$~Si fuisset nubes a vespere usque mane, et statim diluculo tabernaculum reliquisset, proficiscebantur~: et si post diem et noctem recessisset, dissipabant tentoria.
${}^{22}$~Si vero biduo aut uno mense vel longiori tempore fuisset super tabernaculum, manebant filii Isra\"el in eodem loco, et non proficiscebantur~: statim autem ut recessisset, movebant castra.
${}^{23}$~Per verbum Domini figebant tentoria, et per verbum illius proficiscebantur~: erantque in excubiis Domini juxta imperium ejus per manum Moysi.

\bchapter{10}
\lettrine[lines=10,image=true,loversize=0.05,lraise=-0.03]{L}{}ocutusque est Dominus ad Moysen, dicens~:
${}^{2}$~Fac tibi duas tubas argenteas ductiles, quibus convocare possis multitudinem quando movenda sunt castra.
${}^{3}$~Cumque increpueris tubis, congregabitur ad te omnis turba ad ostium tabernaculi fœderis.
${}^{4}$~Si semel clangueris, venient ad te principes, et capita multitudinis Isra\"el.
${}^{5}$~Si autem prolixior atque concisus clangor increpuerit, movebunt castra primi qui sunt ad orientalem plagam.
${}^{6}$~In secundo autem sonitu et pari ululatu tub\ae , levabunt tentoria qui habitant ad meridiem~; et juxta hunc modum reliqui facient, ululantibus tubis in profectionem.
${}^{7}$~Quando autem congregandus est populus, simplex tubarum clangor erit, et non concise ululabunt.
${}^{8}$~Filii autem Aaron sacerdotes clangent tubis~: eritque hoc legitimum sempiternum in generationibus vestris.
${}^{9}$~Si exieritis ad bellum de terra vestra contra hostes qui dimicant adversum vos, clangetis ululantibus tubis, et erit recordatio vestri coram Domino Deo vestro, ut eruamini de manibus inimicorum vestrorum.
${}^{10}$~Siquando habebitis epulum, et dies festos, et calendas, canetis tubis super holocaustis, et pacificis victimis, ut sint vobis in recordationem Dei vestri. Ego Dominus Deus vester.


${}^{11}$~Anno secundo, mense secundo, vigesima die mensis, elevata est nubes de tabernaculo fœderis~:
${}^{12}$~profectique sunt filii Isra\"el per turmas suas de deserto Sinai, et recubuit nubes in solitudine Pharan.
${}^{13}$~Moveruntque castra primi juxta imperium Domini in manu Moysi.
${}^{14}$~Filii Juda per turmas suas~: quorum princeps erat Nahasson filius Aminadab.
${}^{15}$~In tribu filiorum Issachar fuit princeps Nathana\"el filius Suar.
${}^{16}$~In tribu Zabulon erat princeps Eliab filius Helon.
${}^{17}$~Depositumque est tabernaculum, quod portantes egressi sunt filii Gerson et Merari.
${}^{18}$~Profectique sunt et filii Ruben, per turmas et ordinem suum~: quorum princeps erat Helisur filius Sedeur.
${}^{19}$~In tribu autem filiorum Simeon, princeps fuit Salamiel filius Surisaddai.
${}^{20}$~Porro in tribu Gad erat princeps Eliasaph filius Duel.
${}^{21}$~Profectique sunt et Caathit\ae\ portantes sanctuarium. Tamdiu tabernaculum portabatur, donec venirent ad erectionis locum.
${}^{22}$~Moverunt castra et filii Ephraim per turmas suas, in quorum exercitu princeps erat Elisama filius Ammiud.
${}^{23}$~In tribu autem filiorum Manasse princeps fuit Gamaliel filius Phadassur.
${}^{24}$~Et in tribu Benjamin erat dux Abidan filius Gedeonis.
${}^{25}$~Novissimi castrorum omnium profecti sunt filii Dan per turmas suas, in quorum exercitu princeps fuit Ahiezer filius Ammisaddai.
${}^{26}$~In tribu autem filiorum Aser erat princeps Phegiel filius Ochran.
${}^{27}$~Et in tribu filiorum Nephthali princeps fuit Ahira filius Enan.
${}^{28}$~H\ae c sunt castra, et profectiones filiorum Isra\"el per turmas suas quando egrediebantur.


${}^{29}$~Dixitque Moyses Hobab filio Raguel Madianit\ae , cognato suo~: Proficiscimur ad locum quem Dominus daturus est nobis~: veni nobiscum, ut benefaciamus tibi, quia Dominus bona promisit Isra\"eli.
${}^{30}$~Cui ille respondit~: Non vadam tecum, sed revertar in terram meam, in qua natus sum.
${}^{31}$~Et ille~: Noli, inquit, nos relinquere~: tu enim nosti in quibus locis per desertum castra ponere debeamus, et eris ductor noster.
${}^{32}$~Cumque nobiscum veneris, quidquid optimum fuerit ex opibus, quas nobis traditurus est Dominus, dabimus tibi.
${}^{33}$~Profecti sunt ergo de monte Domini viam trium dierum, arcaque fœderis Domini pr\ae cedebat eos, per dies tres providens castrorum locum.
${}^{34}$~Nubes quoque Domini super eos erat per diem cum incederent.
${}^{35}$~Cumque elevaretur arca, dicebat Moyses~: Surge, Domine, et dissipentur inimici tui, et fugiant qui oderunt te, a facie tua.
${}^{36}$~Cum autem deponeretur, aiebat~: Revertere, Domine, ad multitudinem exercitus Isra\"el.

\bchapter{11}
\lettrine[lines=10,image=true,loversize=0.05,lraise=-0.03]{I}{}nterea ortum est murmur populi, quasi dolentium pro labore, contra Dominum. Quod cum audisset Dominus, iratus est. Et accensus in eos ignis Domini, devoravit extremam castrorum partem.
${}^{2}$~Cumque clamasset populus ad Moysen, oravit Moyses ad Dominum, et absorptus est ignis.
${}^{3}$~Vocavitque nomen loci illius, Incensio~: eo quod incensus fuisset contra eos ignis Domini.
${}^{4}$~Vulgus quippe promiscuum, quod ascenderat cum eis, flagravit desiderio, sedens et flens, junctis sibi pariter filiis Isra\"el, et ait~: Quis dabit nobis ad vescendum carnes~?
${}^{5}$~recordamur piscium quos comedebamus in \AE gypto gratis~: in mentem nobis veniunt cucumeres, et pepones, porrique, et c\ae pe, et allia.
${}^{6}$~Anima nostra arida est~: nihil aliud respiciunt oculi nostri nisi man.
${}^{7}$~Erat autem man quasi semen coriandri, coloris bdellii.
${}^{8}$~Circuibatque populus, et colligens illud, frangebat mola, sive terebat in mortario, coquens in olla, et faciens ex eo tortulas saporis quasi panis oleati.
${}^{9}$~Cumque descenderet nocte super castra ros, descendebat pariter et man.


${}^{10}$~Audivit ergo Moyses flentem populum per familias, singulos per ostia tentorii sui. Iratusque est furor Domini valde~: sed et Moysi intoleranda res visa est,
${}^{11}$~et ait ad Dominum~: Cur afflixisti servum tuum~? quare non invenio gratiam coram te~? et cur imposuisti pondus universi populi hujus super me~?
${}^{12}$~Numquid ego concepi omnem hanc multitudinem, vel genui eam, ut dicas mihi~: Porta eos in sinu tuo sicut portare solet nutrix infantulum, et defer in terram, pro qua jurasti patribus eorum~?
${}^{13}$~Unde mihi carnes ut dem tant\ae\ multitudini~? flent contra me, dicentes~: Da nobis carnes ut comedamus.
${}^{14}$~Non possum solus sustinere omnem hunc populum, quia gravis est mihi.
${}^{15}$~Sin aliter tibi videtur, obsecro ut interficias me, et inveniam gratiam in oculis tuis, ne tantis afficiar malis.


${}^{16}$~Et dixit Dominus ad Moysen~: Congrega mihi septuaginta viros de senibus Isra\"el, quos tu nosti quod senes populi sint ac magistri~: et duces eos ad ostium tabernaculi fœderis, faciesque ibi stare tecum,
${}^{17}$~ut descendam et loquar tibi~: et auferam de spiritu tuo, tradamque eis, ut sustentent tecum onus populi, et non tu solus graveris.
${}^{18}$~Populo quoque dices~: Sanctificamini (cras comedetis carnes~: ego enim audivi vos dicere~: Quis dabit nobis escas carnium~? bene nobis erat in \AE gypto), ut det vobis Dominus carnes, et comedatis~:
${}^{19}$~non uno die, nec duobus, vel quinque aut decem, nec viginti quidem,
${}^{20}$~sed usque ad mensem dierum, donec exeat per nares vestras, et vertatur in nauseam, eo quod repuleritis Dominum, qui in medio vestri est, et fleveritis coram eo, dicentes~: Quare egressi sumus ex \AE gypto~?
${}^{21}$~Et ait Moyses~: Sexcenta millia peditum hujus populi sunt~: et tu dicis~: Dabo eis esum carnium mense integro~?
${}^{22}$~numquid ovium et boum multitudo c\ae detur, ut possit sufficere ad cibum~? vel omnes pisces maris in unum congregabuntur, ut eos satient~?
${}^{23}$~Cui respondit Dominus~: Numquid manus Domini invalida est~? jam nunc videbis utrum meus sermo opere compleatur.
${}^{24}$~Venit igitur Moyses, et narravit populo verba Domini, congregans septuaginta viros de senibus Isra\"el, quos stare fecit circa tabernaculum.
${}^{25}$~Descenditque Dominus per nubem, et locutus est ad eum, auferens de spiritu qui erat in Moyse, et dans septuaginta viris. Cumque requievisset in eis spiritus, prophetaverunt, nec ultra cessaverunt.
${}^{26}$~Remanserat autem in castris duo viri, quorum unus vocabatur Eldad, et alter Medad, super quos requievit spiritus. Nam et ipsi descripti fuerant, et non exierant ad tabernaculum.
${}^{27}$~Cumque prophetarent in castris, cucurrit puer, et nuntiavit Moysi, dicens~: Eldad et Medad prophetant in castris.
${}^{28}$~Statim Josue filius Nun, minister Moysi, et electus e pluribus, ait~: Domine mi Moyses, prohibe eos.
${}^{29}$~At ille~: Quid, inquit \ae mularis pro me~? quis tribuat ut omnis populus prophetet, et det eis Dominus spiritum suum~?
${}^{30}$~Reversusque est Moyses, et majores natu Isra\"el in castra.


${}^{31}$~Ventus autem egrediens a Domino, arreptans trans mare coturnices detulit, et demisit in castra itinere quantum uno die confici potest, ex omni parte castrorum per circuitum, volabantque in a\"ere duobus cubitis altitudine super terram.
${}^{32}$~Surgens ergo populus toto die illo, et nocte, ac die altero, congregavit coturnicum~: qui parum, decem coros~: et siccaverunt eas per gyrum castrorum.
${}^{33}$~Adhuc carnes erant in dentibus eorum, nec defecerat hujuscemodi cibus~: et ecce furor Domini concitatus in populum, percussit eum plaga magna nimis.
${}^{34}$~Vocatusque est ille locus, Sepulchra concupiscenti\ae~: ibi enim sepelierunt populum qui desideraverat. Egressi autem de Sepulchris concupiscenti\ae , venerunt in Haseroth, et manserunt ibi.

\bchapter{12}
\lettrine[lines=10,image=true,loversize=0.05,lraise=-0.03]{L}{}ocutaque est Maria et Aaron contra Moysen propter uxorem ejus \AE thiopissam,
${}^{2}$~et dixerunt~: Num per solum Moysen locutus est Dominus~? nonne et nobis similiter est locutus~? Quod cum audisset Dominus
${}^{3}$~(erat enim Moyses vir mitissimus super omnes homines qui morabantur in terra),
${}^{4}$~statim locutus est ad eum, et ad Aaron et Mariam~: Egredimini vos tantum tres ad tabernaculum fœderis. Cumque fuissent egressi,
${}^{5}$~descendit Dominus in columna nubis, et stetit in introitu tabernaculi, vocans Aaron et Mariam. Qui cum issent,
${}^{6}$~dixit ad eos~: Audite sermones meos~: si quis fuerit inter vos propheta Domini, in visione apparebo ei, vel per somnium loquar ad illum.
${}^{7}$~At non talis servus meus Moyses, qui in omni domo mea fidelissimus est~:
${}^{8}$~ore enim ad os loquor ei~: et palam, et non per \ae nigmata et figuras Dominum videt. Quare ergo non timuistis detrahere servo meo Moysi~?


${}^{9}$~Iratusque contra eos, abiit~:
${}^{10}$~nubes quoque recessit qu\ae\ erat super tabernaculum~: et ecce Maria apparuit candens lepra quasi nix. Cumque respexisset eam Aaron, et vidisset perfusam lepra,
${}^{11}$~ait ad Moysen~: Obsecro, domine mi, ne imponas nobis hoc peccatum quod stulte commisimus,
${}^{12}$~ne fiat h\ae c quasi mortua, et ut abortivum quod projicitur de vulva matris su\ae~: ecce jam medium carnis ejus devoratum est a lepra.
${}^{13}$~Clamavitque Moyses ad Dominum, dicens~: Deus, obsecro, sana eam.
${}^{14}$~Cui respondit Dominus~: Si pater ejus spuisset in faciem illius, nonne debuerat saltem septem diebus rubore suffundi~? separetur septem diebus extra castra, et postea revocabitur.
${}^{15}$~Exclusa est itaque Maria extra castra septem diebus~: et populus non est motus de loco illo, donec revocata est Maria.

\bchapter{13}
\lettrine[lines=10,image=true,loversize=0.05,lraise=-0.03]{P}{}rofectusque est populus de Haseroth, fixis tentoriis in deserto Pharan.
${}^{2}$~Ibique locutus est Dominus ad Moysen, dicens~:
${}^{3}$~Mitte viros, qui considerent terram Chanaan, quam daturus sum filiis Isra\"el, singulos de singulis tribubus, ex principibus.
${}^{4}$~Fecit Moyses quod Dominus imperaverat, de deserto Pharan mittens principes viros, quorum ista sunt nomina.
${}^{5}$~De tribu Ruben, Sammua filium Zechur.
${}^{6}$~De tribu Simeon, Saphat filium Huri.
${}^{7}$~De tribu Juda, Caleb filium Jephone.
${}^{8}$~De tribu Issachar, Igal filium Joseph.
${}^{9}$~De tribu Ephraim, Osee filium Nun.
${}^{10}$~De tribu Benjamin, Phalti filium Raphu.
${}^{11}$~De tribu Zabulon, Geddiel filium Sodi.
${}^{12}$~De tribu Joseph, sceptri Manasse, Gaddi filium Susi.
${}^{13}$~De tribu Dan, Ammiel filium Gemalli.
${}^{14}$~De tribu Aser, Sthur filium Micha\"el.
${}^{15}$~De tribu Nephthali, Nahabi filium Vapsi.
${}^{16}$~De tribu Gad, Guel filium Machi.
${}^{17}$~H\ae c sunt nomina virorum, quos misit Moyses ad considerandam terram~: vocavitque Osee filium Nun, Josue.


${}^{18}$~Misit ergo eos Moyses ad considerandam terram Chanaan, et dixit ad eos~: Ascendite per meridianam plagam. Cumque veneritis ad montes,
${}^{19}$~considerate terram, qualis sit~: et populum qui habitator est ejus, utrum fortis sit an infirmus~: si pauci numero an plures~:
${}^{20}$~ipsa terra, bona an mala~: urbes quales, murat\ae\ an absque muris~:
${}^{21}$~humus, pinguis an sterilis, nemorosa an absque arboribus. Confortamini, et afferte nobis de fructibus terr\ae . Erat autem tempus quando jam pr\ae coqu\ae\ uv\ae\ vesci possunt.
${}^{22}$~Cumque ascendissent, exploraverunt terram a deserto Sin, usque Rohob intrantibus Emath.
${}^{23}$~Ascenderuntque ad meridiem, et venerunt in Hebron, ubi erant Achiman et Sisai et Tholmai filii Enac~: nam Hebron septem annis ante Tanim urbem \AE gypti condita est.
${}^{24}$~Pergentesque usque ad Torrentem botri, absciderunt palmitem cum uva sua, quem portaverunt in vecte duo viri. De malis quoque granatis et de ficis loci illius tulerunt~:
${}^{25}$~qui appellatus est Nehelescol, id est Torrens botri, eo quod botrum portassent inde filii Isra\"el.


${}^{26}$~Reversique exploratores terr\ae\ post quadraginta dies, omni regione circuita,
${}^{27}$~venerunt ad Moysen et Aaron et ad omnem cœtum filiorum Isra\"el in desertum Pharan, quod est in Cades. Locutique eis et omni multitudini ostenderunt fructus terr\ae~:
${}^{28}$~et narraverunt, dicentes~: Venimus in terram, ad quam misisti nos, qu\ae\ revera fluit lacte et melle, ut ex his fructibus cognosci potest~:
${}^{29}$~sed cultores fortissimos habet, et urbes grandes atque muratas. Stirpem Enac vidimus ibi.
${}^{30}$~Amalec habitat in meridie, Heth\ae us et Jebus\ae us et Amorrh\ae us in montanis~: Chanan\ae us vero moratur juxta mare et circa fluenta Jordanis.
${}^{31}$~Inter h\ae c Caleb compescens murmur populi, qui oriebatur contra Moysen, ait~: Ascendamus, et possideamus terram, quoniam poterimus obtinere eam.
${}^{32}$~Alii vero, qui fuerant cum eo, dicebant~: Nequaquam ad hunc populum valemus ascendere, quia fortior nobis est.
${}^{33}$~Detraxeruntque terr\ae , quam inspexerant, apud filios Isra\"el, dicentes~: Terra, quam lustravimus, devorat habitatores suos~: populus, quem aspeximus, procer\ae\ statur\ae\ est.
${}^{34}$~Ibi vidimus monstra qu\ae dam filiorum Enac de genere giganteo~: quibus comparati, quasi locust\ae\ videbamur.

\bchapter{14}
\lettrine[lines=10,image=true,loversize=0.05,lraise=-0.03]{I}{}gitur vociferans omnis turba flevit nocte illa,
${}^{2}$~et murmurati sunt contra Moysen et Aaron cuncti filii Isra\"el, dicentes~:
${}^{3}$~Utinam mortui essemus in \AE gypto~: et in hac vasta solitudine utinam pereamus, et non inducat nos Dominus in terram istam, ne cadamus gladio, et uxores ac liberi nostri ducantur captivi. Nonne melius est reverti in \AE gyptum~?
${}^{4}$~Dixeruntque alter ad alterum~: Constituamus nobis ducem, et revertamur in \AE gyptum.
${}^{5}$~Quo audito, Moyses et Aaron ceciderunt proni in terram coram omni multitudine filiorum Isra\"el.
${}^{6}$~At vero Josue filius Nun et Caleb filius Jephone, qui et ipsi lustraverant terram, sciderunt vestimenta sua,
${}^{7}$~et ad omnem multitudinem filiorum Isra\"el locuti sunt~: Terra, quam circuivimus, valde bona est.
${}^{8}$~Si propitius fuerit Dominus, inducet nos in eam, et tradet humum lacte et melle manantem.
${}^{9}$~Nolite rebelles esse contra Dominum~: neque timeatis populum terr\ae\ hujus, quia sicut panem ita eos possumus devorare. Recessit ab eis omne pr\ae sidium~: Dominus nobiscum est, nolite metuere.
${}^{10}$~Cumque clamaret omnis multitudo, et lapidibus eos vellet opprimere, apparuit gloria Domini super tectum fœderis cunctis filiis Isra\"el.


${}^{11}$~Et dixit Dominus ad Moysen~: Usquequo detrahet mihi populus iste~? quousque non credent mihi, in omnibus signis qu\ae\ feci coram eis~?
${}^{12}$~Feriam igitur eos pestilentia, atque consumam~: te autem faciam principem super gentem magnam, et fortiorem quam h\ae c est.
${}^{13}$~Et ait Moyses ad Dominum~: Ut audiant \AE gyptii, de quorum medio eduxisti populum istum,
${}^{14}$~et habitatores terr\ae\ hujus, qui audierunt quod tu, Domine, in populo isto sis, et facie videaris ad faciem, et nubes tua protegat illos, et in columna nubis pr\ae cedas eos per diem, et in columna ignis per noctem~:
${}^{15}$~quod occideris tantam multitudinem quasi unum hominem, et dicant~:
${}^{16}$~Non poterat introducere populum in terram pro qua juraverat~: idcirco occidit eos in solitudine~?
${}^{17}$~Magnificetur ergo fortitudo Domini sicut jurasti, dicens~:
${}^{18}$~Dominus patiens et mult\ae\ misericordi\ae , auferens iniquitatem et scelera, nullumque innoxium derelinquens, qui visitas peccata patrum in filios in tertiam et quartam generationem.
${}^{19}$~Dimitte, obsecro, peccatum populi hujus secundum magnitudinem misericordi\ae\ tu\ae , sicut propitius fuisti egredientibus de \AE gypto usque ad locum istum.
${}^{20}$~Dixitque Dominus~: Dimisi juxta verbum tuum.
${}^{21}$~Vivo ego~: et implebitur gloria Domini universa terra.
${}^{22}$~Attamen omnes homines qui viderunt majestatem meam, et signa qu\ae\ feci in \AE gypto et in solitudine, et tentaverunt me jam per decem vices, nec obedierunt voci me\ae ,
${}^{23}$~non videbunt terram pro qua juravi patribus eorum, nec quisquam ex illis qui detraxit mihi, intuebitur eam.
${}^{24}$~Servum meum Caleb, qui plenus alio spiritu secutus est me, inducam in terram hanc, quam circuivit~; et semen ejus possidebit eam.
${}^{25}$~Quoniam Amalecites et Chanan\ae us habitant in vallibus. Cras movete castra, et revertimini in solitudinem per viam maris Rubri.
${}^{26}$~Locutusque est Dominus ad Moysen et Aaron, dicens~:
${}^{27}$~Usquequo multitudo h\ae c pessima murmurat contra me~? querelas filiorum Isra\"el audivi.
${}^{28}$~Dic ergo eis~: Vivo ego, ait Dominus~: sicut locuti estis audiente me, sic faciam vobis.
${}^{29}$~In solitudine hac jacebunt cadavera vestra. Omnes qui numerati estis a viginti annis et supra, et murmurastis contra me,
${}^{30}$~non intrabitis terram, super quam levavi manum meam ut habitare vos facerem, pr\ae ter Caleb filium Jephone, et Josue filium Nun.
${}^{31}$~Parvulos autem vestros, de quibus dixistis quod pr\ae d\ae\ hostibus forent, introducam, ut videant terram, qu\ae\ vobis displicuit.
${}^{32}$~Vestra cadavera jacebunt in solitudine.
${}^{33}$~Filii vestri erunt vagi in deserto annis quadraginta, et portabunt fornicationem vestram, donec consumantur cadavera patrum in deserto,
${}^{34}$~juxta numerum quadraginta dierum, quibus considerastis terram~: annus pro die imputabitur. Et quadraginta annis recipietis iniquitates vestras, et scietis ultionem meam~:
${}^{35}$~quoniam sicut locutus sum, ita faciam omni multitudini huic pessim\ae , qu\ae\ consurrexit adversum me~: in solitudine hac deficiet, et morietur.
${}^{36}$~Igitur omnes viri, quos miserat Moyses ad contemplandam terram, et qui reversi murmurare fecerant contra eum omnem multitudinem, detrahentes terr\ae\ quod esset mala,
${}^{37}$~mortui sunt atque percussi in conspectu Domini.
${}^{38}$~Josue autem filius Nun, et Caleb filius Jephone, vixerunt ex omnibus qui perrexerant ad considerandam terram.


${}^{39}$~Locutusque est Moyses universa verba h\ae c ad omnes filios Isra\"el, et luxit populus nimis.
${}^{40}$~Et ecce mane primo surgentes ascenderunt verticem montis, atque dixerunt~: Parati sumus ascendere ad locum, de quo Dominus locutus est~: quia peccavimus.
${}^{41}$~Quibus Moyses~: Cur, inquit, transgredimini verbum Domini, quod vobis non cedet in prosperum~?
${}^{42}$~nolite ascendere~: non enim est Dominus vobiscum~: ne corruatis coram inimicis vestris.
${}^{43}$~Amalecites et Chanan\ae us ante vos sunt, quorum gladio corruetis, eo quod nolueritis acquiescere Domino~: nec erit Dominus vobiscum.
${}^{44}$~At illi contenebrati ascenderunt in verticem montis. Arca autem testamenti Domini et Moyses non recesserunt de castris.
${}^{45}$~Descenditque Amalecites et Chanan\ae us, qui habitabat in monte~: et percutiens eos atque concidens, persecutus est eos usque Horma.

\bchapter{15}
\lettrine[lines=10,image=true,loversize=0.05,lraise=-0.03]{L}{}ocutus est Dominus ad Moysen, dicens~:
${}^{2}$~Loquere ad filios Isra\"el, et dices ad eos~: Cum ingressi fueritis terram habitationis vestr\ae , quam ego dabo vobis,
${}^{3}$~et feceritis oblationem Domino in holocaustum, aut victimam, vota solventes, vel sponte offerentes munera, aut in solemnitatibus vestris adolentes odorem suavitatis Domino, de bobus sive de ovibus~:
${}^{4}$~offeret quicumque immolaverit victimam, sacrificium simil\ae , decimam partem ephi, conspers\ae\ oleo, quod mensuram habebit quartam partem hin~:
${}^{5}$~et vinum ad liba fundenda ejusdem mensur\ae\ dabit in holocaustum sive in victimam. Per agnos singulos
${}^{6}$~et arietes erit sacrificium simil\ae\ duarum decimarum, qu\ae\ conspersa sit oleo terti\ae\ partis hin~:
${}^{7}$~et vinum ad libamentum terti\ae\ partis ejusdem mensur\ae\ offeret in odorem suavitatis Domino.
${}^{8}$~Quando vero de bobus feceris holocaustum aut hostiam, ut impleas votum vel pacificas victimas,
${}^{9}$~dabis per singulos boves simil\ae\ tres decimas conspers\ae\ oleo, quod habeat medium mensur\ae\ hin~:
${}^{10}$~et vinum ad liba fundenda ejusdem mensur\ae\ in oblationem suavissimi odoris Domino.
${}^{11}$~Sic facies
${}^{12}$~per singulos boves et arietis et agnos et h\ae dos.
${}^{13}$~Tam indigen\ae\ quam peregrini
${}^{14}$~eodem ritu offerent sacrificia.
${}^{15}$~Unum pr\ae ceptum erit atque judicium tam vobis quam advenis terr\ae .


${}^{16}$~Locutus est Dominus ad Moysen, dicens~:
${}^{17}$~Loquere filiis Isra\"el, et dices ad eos~:
${}^{18}$~Cum veneritis in terram, quam dabo vobis,
${}^{19}$~et comederitis de panibus regionis illius, separabitis primitias Domino
${}^{20}$~de cibis vestris. Sicut de areis primitias separatis,
${}^{21}$~ita et de pulmentis dabitis primitiva Domino.


${}^{22}$~Quod si per ignorantiam pr\ae terieritis quidquam horum, qu\ae\ locutus est Dominus ad Moysen,
${}^{23}$~et mandavit per eum ad vos, a die qua cœpit jubere et ultra,
${}^{24}$~oblitaque fuerit facere multitudo~: offeret vitulum de armento, holocaustum in odorem suavissimum Domino, et sacrificum ejus ac liba, ut c\ae remoni\ae\ postulant, hircumque pro peccato~:
${}^{25}$~et rogabit sacerdos pro omni multitudine filiorum Isra\"el, et dimittetur eis, quoniam non sponte peccaverunt, nihilominus offerentes incensum Domino pro se et pro peccato atque errore suo~:
${}^{26}$~et dimittetur univers\ae\ plebi filiorum Isra\"el, et advenis qui peregrinantur inter eos~: quoniam culpa est omnis populi per ignorantiam.
${}^{27}$~Quod si anima una nesciens peccaverit, offeret capram anniculam pro peccato suo~:
${}^{28}$~et deprecabitur pro ea sacerdos, quod inscia peccaverit coram Domino~: impetrabitque ei veniam, et dimittetur illi.
${}^{29}$~Tam indigenis quam advenis una lex erit omnium, qui peccaverint ignorantes.
${}^{30}$~Anima vero, qu\ae\ per superbiam aliquid commiserit, sive civis sit ille, sive peregrinus (quoniam adversus Dominum rebellis fuit), peribit de populo suo~:
${}^{31}$~verbum enim Domini contempsit, et pr\ae ceptum illius fecit irritum~: idcirco delebitur, et portabit iniquitatem suam.


${}^{32}$~Factum est autem, cum essent filii Isra\"el in solitudine, et invenissent hominem colligentem ligna in die sabbati,
${}^{33}$~obtulerunt eum Moysi et Aaron et univers\ae\ multitudini.
${}^{34}$~Qui recluserunt eum in carcerem, nescientes quid super eo facere deberent.
${}^{35}$~Dixitque Dominus ad Moysen~: Morte moriatur homo iste~: obruat eum lapidibus omnis turba extra castra.
${}^{36}$~Cumque eduxissent eum foras, obruerunt lapidibus, et mortuus est, sicut pr\ae ceperat Dominus.


${}^{37}$~Dixit quoque Dominus ad Moysen~:
${}^{38}$~Loquere filiis Isra\"el, et dices ad eos ut faciant sibi fimbrias per angulos palliorum, ponentes in eis vittas hyacinthinas~:
${}^{39}$~quas cum viderint, recordentur omnium mandatorum Domini, nec sequantur cogitationes suas et oculos per res varias fornicantes,
${}^{40}$~sed magis memores pr\ae ceptorum Domini faciant ea, sintque sancti Deo suo.
${}^{41}$~Ego Dominus Deus vester, qui eduxi vos de terra \AE gypti, ut essem Deus vester.

\bchapter{16}
\lettrine[lines=10,image=true,loversize=0.05,lraise=-0.03]{E}{}cce autem Core filius Isaar, filii Caath, filii Levi, et Dathan atque Abiron filii Eliab, Hon quoque filius Pheleth de filiis Ruben,
${}^{2}$~surrexerunt contra Moysen, aliique filiorum Isra\"el ducenti quinquaginta viri proceres synagog\ae , et qui tempore concilii per nomina vocabantur.
${}^{3}$~Cumque stetissent adversum Moysen et Aaron, dixerunt~: Sufficiat vobis, quia omnis multitudo sanctorum est, et in ipsis est Dominus~: cur elevamini super populum Domini~?
${}^{4}$~Quod cum audisset Moyses, cecidit pronus in faciem~:
${}^{5}$~locutusque ad Core et ad omnem multitudinem~: Mane, inquit, notum faciet Dominus qui ad se pertineant, et sanctos applicabit sibi~: et quos elegerit, appropinquabunt ei.
${}^{6}$~Hoc igitur facite~: tollat unusquisque thuribula sua, tu Core, et omne concilium tuum~:
${}^{7}$~et hausto cras igne, ponite desuper thymiama coram Domino~: et quemcumque elegerit, ipse erit sanctus~: multum erigimini filii Levi.
${}^{8}$~Dixitque rursum ad Core~: Audite, filii Levi~:
${}^{9}$~num parum vobis est quod separavit vos Deus Isra\"el ab omni populo, et junxit sibi, ut serviretis ei in cultu tabernaculi, et staretis coram frequentia populi, et ministraretis ei~?
${}^{10}$~idcirco ad se fecit accedere te et omnes fratres tuos filios Levi, ut vobis etiam sacerdotium vindicetis,
${}^{11}$~et omnis globus tuus stet contra Dominum~? quid est enim Aaron ut murmuretis contra eum~?


${}^{12}$~Misit ergo Moyses ut vocaret Dathan et Abiron filios Eliab. Qui responderunt~: Non venimus.
${}^{13}$~Numquid parum est tibi quod eduxisti nos de terra, qu\ae\ lacte et melle manabat, ut occideres in deserto, nisi et dominatus fueris nostri~?
${}^{14}$~Revera induxisti nos in terram, qu\ae\ fluit rivis lactis et mellis, et dedisti nobis possessiones agrorum et vinearum~: an et oculos nostros vis eruere~? non venimus.
${}^{15}$~Iratusque Moyses valde, ait ad Dominum~: Ne respicias sacrificia eorum~: tu scis quod ne asellum quidem umquam acceperim ab eis, nec afflixerim quempiam eorum.


${}^{16}$~Dixitque ad Core~: Tu, et omnis congregatio tua, state seorsum coram Domino, et Aaron die crastino separatim.
${}^{17}$~Tollite singuli thuribula vestra, et ponite super ea incensum, offerentes Domino ducenta quinquaginta thuribula~: Aaron quoque teneat thuribulum suum.
${}^{18}$~Quod cum fecissent, stantibus Moyses et Aaron,
${}^{19}$~et coacervassent adversum eos omnem multitudinem ad ostium tabernaculi, apparuit cunctis gloria Domini.
${}^{20}$~Locutusque Dominus ad Moysen et Aaron, ait~:
${}^{21}$~Separamini de medio congregationis hujus, ut eos repente disperdam.
${}^{22}$~Qui ceciderunt proni in faciem, atque dixerunt~: Fortissime Deus spirituum univers\ae\ carnis, num uno peccante, contra omnes ira tua des\ae viet~?
${}^{23}$~Et ait Dominus ad Moysen~:
${}^{24}$~Pr\ae cipe universo populo ut separetur a tabernaculis Core et Dathan et Abiron.


${}^{25}$~Surrexitque Moyses, et abiit ad Dathan et Abiron~: et sequentibus eum senioribus Isra\"el,
${}^{26}$~dixit ad turbam~: Recedite a tabernaculis hominum impiorum, et nolite tangere qu\ae\ ad eos pertinent, ne involvamini in peccatis eorum.
${}^{27}$~Cumque recessissent a tentoriis eorum per circuitum, Dathan et Abiron egressi stabant in introitu papilionum suorum cum uxoribus et liberis, omnique frequentia.
${}^{28}$~Et ait Moyses~: In hoc scietis quod Dominus miserit me ut facerem universa qu\ae\ cernitis, et non ex proprio ea corde protulerim~:
${}^{29}$~si consueta hominum morte interierint, et visitaverit eos plaga, qua et ceteri visitari solent, non misit me Dominus~:
${}^{30}$~sin autem novam rem fecerit Dominus, ut aperiens terra os suum deglutiat eos et omnia qu\ae\ ad illos pertinent, descenderintque viventes in infernum, scietis quod blasphemaverint Dominum.


${}^{31}$~Confestim igitur ut cessavit loqui, dirupta est terra sub pedibus eorum~:
${}^{32}$~et aperiens os suum, devoravit illos cum tabernaculis suis et universa substantia eorum,
${}^{33}$~descenderuntque vivi in infernum operti humo, et perierunt de medio multitudinis.
${}^{34}$~At vero omnis Isra\"el, qui stabat per gyrum, fugit ad clamorem pereuntium, dicens~: Ne forte et nos terra deglutiat.
${}^{35}$~Sed et ignis egressus a Domino interfecit ducentos quinquaginta viros, qui offerebant incensum.
${}^{36}$~Locutusque est Dominus ad Moysen, dicens~:
${}^{37}$~Pr\ae cipe Eleazaro filio Aaron sacerdoti ut tollat thuribula qu\ae\ jacent in incendio, et ignem huc illucque dispergat~: quoniam sanctificata sunt
${}^{38}$~in mortibus peccatorum~: producatque ea in laminas, et affigat altari, eo quod oblatum sit in eis incensum Domino, et sanctificata sint, ut cernant ea pro signo et monimento filii Isra\"el.
${}^{39}$~Tulit ergo Eleazar sacerdos thuribula \ae nea, in quibus obtulerant hi quos incendium devoravit, et produxit ea in laminas, affigens altari~:
${}^{40}$~ut haberent postea filii Isra\"el, quibus commonerentur ne quis accedat alienigena, et qui non est de semine Aaron ad offerendum incensum Domino, ne patiatur sicut passus est Core, et omnis congregatio ejus, loquente Domino ad Moysen.


${}^{41}$~Murmuravit autem omnis multitudo filiorum Isra\"el sequenti die contra Moysen et Aaron, dicens~: Vos interfecistis populum Domini.
${}^{42}$~Cumque oriretur seditio, et tumultus incresceret,
${}^{43}$~Moyses et Aaron fugerunt ad tabernaculum fœderis. Quod, postquam ingressi sunt, operuit nubes, et apparuit gloria Domini.
${}^{44}$~Dixitque Dominus ad Moysen~:
${}^{45}$~Recedite de medio hujus multitudinis, etiam nunc delebo eos. Cumque jacerent in terra,
${}^{46}$~dixit Moyses ad Aaron~: Tolle thuribulum, et hausto igne de altari, mitte incensum desuper, pergens cito ad populum, ut roges pro eis~: jam enim egressa est ira a Domino, et plaga des\ae vit.
${}^{47}$~Quod cum fecisset Aaron, et cucurrisset ad mediam multitudinem, quam jam vastabat incendium, obtulit thymiama~:
${}^{48}$~et stans inter mortuos ac viventes, pro populo deprecatus est, et plaga cessavit.
${}^{49}$~Fuerunt autem qui percussi sunt, quatuordecim millia hominum, et septingenti, absque his qui perierant in seditione Core.
${}^{50}$~Reversusque est Aaron ad Moysen ad ostium tabernaculi fœderis postquam quievit interitus.

\bchapter{17}
\lettrine[lines=10,image=true,loversize=0.05,lraise=-0.03]{E}{}t locutus est Dominus ad Moysen, dicens~:
${}^{2}$~Loquere ad filios Isra\"el, et accipe ab eis virgas singulas per cognationes suas, a cunctis principibus tribuum, virgas duodecim, et uniuscujusque nomen superscribes virg\ae\ su\ae .
${}^{3}$~Nomen autem Aaron erit in tribu Levi, et una virga cunctas seorsum familias continebit~:
${}^{4}$~ponesque eas in tabernaculo fœderis coram testimonio, ubi loquar ad te.
${}^{5}$~Quem ex his elegero, germinabit virga ejus~: et cohibebo a me querimonias filiorum Isra\"el, quibus contra vos murmurant.
${}^{6}$~Locutusque est Moyses ad filios Isra\"el~: et dederunt ei omnes principes virgas per singulas tribus~: fueruntque virg\ae\ duodecim absque virga Aaron.
${}^{7}$~Quas cum posuisset Moyses coram Domino in tabernaculo testimonii,
${}^{8}$~sequenti die regressus invenit germinasse virgam Aaron in domo Levi~: et turgentibus gemmis eruperant flores, qui, foliis dilatatis, in amygdalas deformati sunt.
${}^{9}$~Protulit ergo Moyses omnes virgas de conspectu Domini ad cunctos filios Isra\"el~: videruntque, et receperunt singuli virgas suas.
${}^{10}$~Dixitque Dominus ad Moysen~: Refer virgam Aaron in tabernaculum testimonii, ut servetur ibi in signum rebellium filiorum Isra\"el, et quiescant querel\ae\ eorum a me, ne moriantur.
${}^{11}$~Fecitque Moyses sicut pr\ae ceperat Dominus.
${}^{12}$~Dixerunt autem filii Isra\"el ad Moysen~: Ecce consumpti sumus, omnes perivimus.
${}^{13}$~Quicumque accedit ad tabernaculum Domini, moritur. Num usque ad internecionem cuncti delendi sumus~?

\bchapter{18}
\lettrine[lines=10,image=true,loversize=0.05,lraise=-0.03]{D}{}ixitque Dominus ad Aaron~: Tu, et filii tui, et domus patris tui tecum, portabitis iniquitatem sanctuarii~: et tu et filii tui simul sustinebitis peccata sacerdotii vestri.
${}^{2}$~Sed et fratres tuos de tribu Levi, et sceptrum patris tui sume tecum, pr\ae stoque sint, et ministrent tibi~: tu autem et filii tui ministrabitis in tabernaculo testimonii.
${}^{3}$~Excubabuntque Levit\ae\ ad pr\ae cepta tua, et ad cuncta opera tabernaculi~: ita dumtaxat ut ad vasa sanctuarii et ad altare non accedant, ne et illi moriantur, et vos pereatis simul.
${}^{4}$~Sint autem tecum, et excubent in custodiis tabernaculi, et in omnibus c\ae remoniis ejus. Alienigena non miscebitur vobis.
${}^{5}$~Excubate in custodia sanctuarii, et in ministerio altaris~: ne oriatur indignatio super filios Isra\"el.
${}^{6}$~Ego dedi vobis fratres vestros Levitas de medio filiorum Isra\"el, et tradidi donum Domino, ut serviant in ministeriis tabernaculi ejus.
${}^{7}$~Tu autem et filii tui custodite sacerdotium vestrum~: et omnia qu\ae\ ad cultum altaris pertinent, et intra velum sunt, per sacerdotes administrabuntur~: si quis externus accesserit, occidetur.


${}^{8}$~Locutusque est Dominus ad Aaron~: Ecce dedi tibi custodiam primitiarum mearum. Omnia qu\ae\ sanctificantur a filiis Isra\"el, tradidi tibi et filiis tuis pro officio sacerdotali legitima sempiterna.
${}^{9}$~H\ae c ergo accipies de his, qu\ae\ sanctificantur et oblata sunt Domino. Omnis oblatio, et sacrificium, et quidquid pro peccato atque delicto redditur mihi, et cedit in Sancta sanctorum, tuum erit, et filiorum tuorum.
${}^{10}$~In sanctuario comedes illud~: mares tantum edent ex eo, quia consecratum est tibi.
${}^{11}$~Primitias autem, quas voverint et obtulerint filii Isra\"el, tibi dedi, et filiis tuis, ac filiabus tuis, jure perpetuo~: qui mundus est in domo tua, vescetur eis.
${}^{12}$~Omnem medullam olei, et vini, ac frumenti, quidquid offerunt primitiarum Domino, tibi dedi.
${}^{13}$~Universa frugum initia, quas gignit humus, et Domino deportantur, cedent in usus tuos~: qui mundus est in domo tua, vescetur eis.
${}^{14}$~Omne quod ex voto reddiderint filii Isra\"el, tuum erit.
${}^{15}$~Quidquid primum erumpit e vulva cunct\ae\ carnis, quam offerunt Domino, sive ex hominibus, sive de pecoribus fuerit, tui juris erit~: ita dumtaxat ut pro hominis primogenito pretium accipias, et omne animal quod immundum est redimi facias,
${}^{16}$~cujus redemptio erit post unum mensem, siclis argenti quinque, pondere sanctuarii. Siclus viginti obolos habet.
${}^{17}$~Primogenitum autem bovis, et ovis, et capr\ae , non facies redimi, quia sanctificata sunt Domino. Sanguinem tantum eorum fundes super altare, et adipes adolebis in suavissimum odorem Domino.
${}^{18}$~Carnes vero in usum tuum cedent, sicut pectusculum consecratum, et armus dexter~: tua erunt.
${}^{19}$~Omnes primitias sanctuarii, quas offerunt filii Isra\"el Domino, tibi dedi, et filiis, ac filiabus tuis, jure perpetuo. Pactum salis est sempiternum coram Domino, tibi ac filiis tuis.


${}^{20}$~Dixitque Dominus ad Aaron~: In terra eorum nihil possidebitis, nec habebitis partem inter eos~: ego pars et h\ae reditas tua in medio filiorum Isra\"el.
${}^{21}$~Filiis autem Levi dedi omnes decimas Isra\"elis in possessionem, pro ministerio, quo serviunt mihi in tabernaculo fœderis~:
${}^{22}$~ut non accedant ultra filii Isra\"el ad tabernaculum, nec committant peccatum mortiferum,
${}^{23}$~solis filiis Levi mihi in tabernaculo servientibus, et portantibus peccata populi. Legitimum sempiternum erit in generationibus vestris. Nihil aliud possidebunt,
${}^{24}$~decimarum oblatione contenti, quas in usus eorum et necessaria separavi.


${}^{25}$~Locutusque est Dominus ad Moysen, dicens~:
${}^{26}$~Pr\ae cipe Levitis, atque denuntia~: Cum acceperitis a filiis Isra\"el decimas, quas dedi vobis, primitias earum offerte Domino, id est, decimam partem decim\ae ,
${}^{27}$~ut reputetur vobis in oblationem primitivorum, tam de areis quam de torcularibus~:
${}^{28}$~et universis quorum accipitis primitias, offerte Domino, et date Aaron sacerdoti.
${}^{29}$~Omnia qu\ae\ offeretis ex decimis, et in donaria Domini separabitis, optima et electa erunt.
${}^{30}$~Dicesque ad eos~: Si pr\ae clara et meliora qu\ae que obtuleritis ex decimis, reputabitur vobis quasi de area, et torculari dederitis primitias~:
${}^{31}$~et comedetis eas in omnibus locis vestris, tam vos quam famili\ae\ vestr\ae~: quia pretium est pro ministerio, quo servitis in tabernaculo testimonii.
${}^{32}$~Et non peccabitis super hoc, egregia vobis et pinguia reservantes, ne polluatis oblationes filiorum Isra\"el, et moriamini.

\bchapter{19}
\lettrine[lines=10,image=true,loversize=0.05,lraise=-0.03]{L}{}ocutusque est Dominus ad Moysen et Aaron, dicens~:
${}^{2}$~Ista est religio victim\ae , quam constituit Dominus. Pr\ae cipe filiis Isra\"el, ut adducant ad te vaccam rufam \ae tatis integr\ae , in qua nulla sit macula, nec portaverit jugum~:
${}^{3}$~tradetisque eam Eleazaro sacerdoti, qui eductam extra castra, immolabit in conspectu omnium~:
${}^{4}$~et tingens digitum in sanguine ejus, asperget contra fores tabernaculi septem vicibus,
${}^{5}$~comburetque eam cunctis videntibus, tam pelle et carnibus ejus quam sanguine et fimo flamm\ae\ traditis.
${}^{6}$~Lignum quoque cedrinum, et hyssopum, coccumque bis tinctum sacerdos mittet in flammam, qu\ae\ vaccam vorat.
${}^{7}$~Et tunc demum, lotis vestibus et corpore suo, ingredietur in castra, commaculatusque erit usque ad vesperum.
${}^{8}$~Sed et ille qui combusserit eam, lavabit vestimenta sua et corpus, et immundus erit usque ad vesperum.
${}^{9}$~Colliget autem vir mundus cineres vacc\ae , et effundet eos extra castra in loco purissimo, ut sint multitudini filiorum Isra\"el in custodiam, et in aquam aspersionis~: quia pro peccato vacca combusta est.
${}^{10}$~Cumque laverit qui vacc\ae\ portaverat cineres vestimenta sua, immundus erit usque ad vesperum. Habebunt hoc filii Isra\"el, et adven\ae\ qui habitant inter eos, sanctum jure perpetuo.


${}^{11}$~Qui tetigerit cadaver hominis, et propter hoc septem diebus fuerit immundus,
${}^{12}$~aspergetur ex hac aqua die tertio et septimo, et sic mundabitur. Si die tertio aspersus non fuerit, septimo non poterit emundari.
${}^{13}$~Omnis qui tetigerit human\ae\ anim\ae\ morticinum, et aspersus hac commistione non fuerit, polluet tabernaculum Domini et peribit ex Isra\"el~: quia aqua expiationis non est aspersus, immundus erit, et manebit spurcitia ejus super eum.
${}^{14}$~Ista est lex hominis qui moritur in tabernaculo~: omnes qui ingrediuntur tentorium illius, et universa vasa qu\ae\ ibi sunt, polluta erunt septem diebus.
${}^{15}$~Vas, quod non habuerit operculum nec ligaturam desuper, immundum erit.
${}^{16}$~Si quis in agro tetigerit cadaver occisi hominis, aut per se mortui, sive os illius, vel sepulchrum, immundus erit septem diebus.
${}^{17}$~Tollentque de cineribus combustionis atque peccati, et mittent aquas vivas super eos in vas~:
${}^{18}$~in quibus cum homo mundus tinxerit hyssopum, asperget ex eo omne tentorium, et cunctam supellectilem, et homines hujuscemodi contagione pollutos~:
${}^{19}$~atque hoc modo mundus lustrabit immundum tertio et septimo die~: expiatusque die septimo, lavabit et se et vestimenta sua, et immundus erit usque ad vesperum.
${}^{20}$~Si quis hoc ritu non fuerit expiatus, peribit anima illius de medio ecclesi\ae~: quia sanctuarium Domini polluit, et non est aqua lustrationis aspersus.
${}^{21}$~Erit hoc pr\ae ceptum legitimum sempiternum. Ipse quoque qui aspergit aquas, lavabit vestimenta sua. Omnis qui tetigerit aquas expiationis, immundus erit usque ad vesperum.
${}^{22}$~Quidquid tetigerit immundus, immundum faciet~: et anima, qu\ae\ horum quippiam tetigerit, immunda erit usque ad vesperum.

\bchapter{20}
\lettrine[lines=10,image=true,loversize=0.05,lraise=-0.03]{V}{}eneruntque filii Isra\"el et omnis multitudo in desertum Sin, mense primo, et mansit populus in Cades. Mortuaque est ibi Maria, et sepulta in eodem loco.
${}^{2}$~Cumque indigeret aqua populus, convenerunt adversum Moysen et Aaron~:
${}^{3}$~et versi in seditionem, dixerunt~: Utinam periissemus inter fratres nostros coram Domino.
${}^{4}$~Cur eduxistis ecclesiam Domini in solitudinem, ut et nos et nostra jumenta moriamur~?
${}^{5}$~quare nos fecistis ascendere de \AE gypto, et adduxistis in locum istum pessimum, qui seri non potest, qui nec ficum gignit, nec vineas, nec malogranata, insuper et aquam non habet ad bibendum~?
${}^{6}$~Ingressusque Moyses et Aaron, dimissa multitudine, tabernaculum fœderis, corruerunt proni in terram, clamaveruntque ad Dominum, atque dixerunt~: Domine Deus, audi clamorem hujus populi, et aperi eis thesaurum tuum fontem aqu\ae\ viv\ae , ut satiati, cesset murmuratio eorum. Et apparuit gloria Domini super eos.


${}^{7}$~Locutusque est Dominus ad Moysen, dicens~:
${}^{8}$~Tolle virgam, et congrega populum, tu et Aaron frater tuus, et loquimini ad petram coram eis, et illa dabit aquas. Cumque eduxeris aquam de petra, bibet omnis multitudo et jumenta ejus.
${}^{9}$~Tulit igitur Moyses virgam, qu\ae\ erat in conspectu Domini, sicut pr\ae ceperat ei,
${}^{10}$~congregata multitudine ante petram~: dixitque eis~: Audite, rebelles et increduli~: num de petra hac vobis aquam poterimus ejicere~?
${}^{11}$~Cumque elevasset Moyses manum, percutiens virga bis silicem, egress\ae\ sunt aqu\ae\ largissim\ae , ita ut populus biberet et jumenta.
${}^{12}$~Dixitque Dominus ad Moysen et Aaron~: Quia non credidistis mihi, ut sanctificaretis me coram filiis Isra\"el, non introducetis hos populos in terram, quam dabo eis.
${}^{13}$~H\ae c est aqua contradictionis, ubi jurgati sunt filii Isra\"el contra Dominum, et sanctificatus est in eis.


${}^{14}$~Misit interea nuntios Moyses de Cades ad regem Edom, qui dicerent~: H\ae c mandat frater tuus Isra\"el~: Nosti omnem laborem qui apprehendit nos,
${}^{15}$~quomodo descenderint patres nostri in \AE gyptum, et habitaverimus ibi multo tempore, afflixerintque nos \AE gyptii, et patres nostros~:
${}^{16}$~et quomodo clamaverimus ad Dominum, et exaudierit nos, miseritque angelum, qui eduxerit nos de \AE gypto. Ecce in urbe Cades, qu\ae\ est in extremis finibus tuis, positi,
${}^{17}$~obsecramus ut nobis transire liceat per terram tuam. Non ibimus per agros, nec per vineas~; non bibemus aquas de puteis tuis~: sed gradiemur via publica, nec ad dexteram nec ad sinistram declinantes, donec transeamus terminos tuos.
${}^{18}$~Cui respondit Edom~: Non transibis per me, alioquin armatus occurram tibi.
${}^{19}$~Dixeruntque filii Isra\"el~: Per tritam gradiemur viam~: et si biberimus aquas tuas, nos et pecora nostra, dabimus quod justum est~: nulla erit in pretio difficultas, tantum velociter transeamus.
${}^{20}$~At ille respondit~: Non transibis. Statimque egressus est obvius, cum infinita multitudo, et manu forti,
${}^{21}$~nec voluit acquiescere deprecanti, ut concederet transitum per fines suos. Quam ob rem divertit ab eo Isra\"el.


${}^{22}$~Cumque castra movissent de Cades, venerunt in montem Hor, qui est in finibus terr\ae\ Edom~:
${}^{23}$~ubi locutus est Dominus ad Moysen~:
${}^{24}$~Pergat, inquit, Aaron ad populos suos~: non enim intrabit terram, quam dedi filiis Isra\"el, eo quod incredulus fuerit ori meo, ad aquas contradictionis.
${}^{25}$~Tolle Aaron et filium ejus cum eo, et duces eos in montem Hor.
${}^{26}$~Cumque nudaveris patrem veste sua, indues ea Eleazarum filium ejus~: Aaron colligetur, et morietur ibi.
${}^{27}$~Fecit Moyses ut pr\ae ceperat Dominus~: et ascenderunt in montem Hor coram omni multitudine.
${}^{28}$~Cumque Aaron spoliasset vestibus suis, induit eis Eleazarum filium ejus.
${}^{29}$~Illo mortuo in montis supercilio, descendit cum Eleazaro.
${}^{30}$~Omnis autem multitudo videns occubuisse Aaron, flevit super eo triginta diebus per cunctas familias suas.

\bchapter{21}
\lettrine[lines=10,image=true,loversize=0.05,lraise=-0.03]{Q}{}uod cum audisset Chanan\ae us rex Arad, qui habitabat ad meridiem, venisse scilicet Isra\"el per exploratorum viam, pugnavit contra illum, et victor existens, duxit ex eo pr\ae dam.
${}^{2}$~At Isra\"el voto se Domino obligans, ait~: Si tradideris populum istum in manu mea, delebo urbes ejus.
${}^{3}$~Exaudivitque Dominus preces Isra\"el, et tradidit Chanan\ae um, quem ille interfecit subversis urbibus ejus~: et vocavit nomen loci illius Horma, id est, anathema.


${}^{4}$~Profecti sunt autem et de monte Hor, per viam qu\ae\ ducit ad mare Rubrum, ut circumirent terram Edom. Et t\ae dere cœpit populum itineris ac laboris~:
${}^{5}$~locutusque contra Deum et Moysen, ait~: Cur eduxisti nos de \AE gypto, ut moreremur in solitudine~? deest panis, non sunt aqu\ae~: anima nostra jam nauseat super cibo isto levissimo.
${}^{6}$~Quam ob rem misit Dominus in populum ignitos serpentes, ad quorum plagas et mortes plurimorum,
${}^{7}$~venerunt ad Moysen, atque dixerunt~: Peccavimus, quia locuti sumus contra Dominum et te~: ora ut tollat a nobis serpentes. Oravitque Moyses pro populo,
${}^{8}$~et locutus est Dominus ad eum~: Fac serpentem \ae neum, et pone eum pro signo~: qui percussus aspexerit eum, vivet.
${}^{9}$~Fecit ergo Moyses serpentem \ae neum, et posuit eum pro signo~: quem cum percussi aspicerent, sanabantur.


${}^{10}$~Profectique filii Isra\"el castrametati sunt in Oboth.
${}^{11}$~Unde egressi fixere tentoria in Jeabarim, in solitudine qu\ae\ respicit Moab contra orientalem plagam.
${}^{12}$~Et inde moventes, venerunt ad torrentem Zared.
${}^{13}$~Quem relinquentes castrametati sunt contra Arnon, qu\ae\ est in deserto, et prominet in finibus Amorrh\ae i. Siquidem Arnon terminus est Moab, dividens Moabitas et Amorrh\ae os.
${}^{14}$~Unde dicitur in libro bellorum Domini~: \begin{flushleft}\begin{verse}Sicut fecit in mari Rubro,\\ sic faciet in torrentibus Arnon.\\
${}^{15}$~Scopuli torrentium inclinati sunt, ut requiescerent in Ar,\\ et recumberent in finibus Moabitarum.\end{verse}\end{flushleft}


${}^{16}$~Ex eo loco apparuit puteus, super quo locutus est Dominus ad Moysen~: Congrega populum, et dabo ei aquam.
${}^{17}$~Tunc cecinit Isra\"el carmen istud~: \begin{flushleft}\begin{verse}Ascendat puteus.\end{verse}\end{flushleft}

 Concinebant~:
\begin{flushleft}\begin{verse}${}^{18}$~Puteus, quem foderunt principes\\ et paraverunt duces multitudinis\\ in datore legis, et in baculis suis.\end{verse}\end{flushleft}

 De solitudine, Matthana.
${}^{19}$~De Matthana in Nahaliel~: de Nahaliel in Bamoth.
${}^{20}$~De Bamoth, vallis est in regione Moab, in vertice Phasga, quod respicit contra desertum.


${}^{21}$~Misit autem Isra\"el nuntios ad Sehon regem Amorrh\ae orum, dicens~:
${}^{22}$~Obsecro ut transire mihi liceat per terram tuam~: non declinabimus in agros et vineas~; non bibemus aquas ex puteis~: via regia gradiemur, donec transeamus terminos tuos.
${}^{23}$~Qui concedere noluit ut transiret Isra\"el per fines suos~: quin potius exercitu congregato, egressus est obviam in desertum, et venit in Jasa, pugnavitque contra eum.
${}^{24}$~A quo percussus est in ore gladii, et possessa est terra ejus ab Arnon usque Jeboc, et filios Ammon~: quia forti pr\ae sidio tenebantur termini Ammonitarum.
${}^{25}$~Tulit ergo Isra\"el omnes civitates ejus, et habitavit in urbibus Amorrh\ae i, in Hesebon scilicet, et viculis ejus.
${}^{26}$~Urbs Hesebon fuit Sehon regis Amorrh\ae i, qui pugnavit contra regem Moab~: et tulit omnem terram, qu\ae\ ditionis illius fuerat usque Arnon.
${}^{27}$~Idcirco dicitur in proverbio~: \begin{flushleft}\begin{verse}Venite in Hesebon~: \ae dificetur, et construatur civitas Sehon~:\\
${}^{28}$~ignis egressus est de Hesebon, flamma de oppido Sehon,\\ et devoravit Ar Moabitarum, et habitatores excelsorum Arnon.\\
${}^{29}$~V\ae\ tibi Moab~; peristi popule Chamos.\\ Dedit filios ejus in fugam, et filias in captivitatem regi Amorrh\ae orum Sehon.\\
${}^{30}$~Jugum ipsorum disperiit ad Hesebon usque Dibon~:\\ lassi pervenerunt in Nophe, et usque Medaba.\end{verse}\end{flushleft}


${}^{31}$~Habitavit itaque Isra\"el in terra Amorrh\ae i.


${}^{32}$~Misitque Moyses qui explorarent Jazer~: cujus ceperunt viculos, et possederunt habitatores.
${}^{33}$~Verteruntque se, et ascenderunt per viam Basan, et occurrit eis Og, rex Basan, cum omni populo suo, pugnaturus in Edrai.
${}^{34}$~Dixitque Dominus ad Moysen~: Ne timeas eum, quia in manu tua tradidi illum, et omnem populum ac terram ejus~: faciesque illi sicut fecisti Sehon, regi Amorrh\ae orum habitatori Hesebon.
${}^{35}$~Percusserunt igitur et hunc cum filiis suis, universumque populum ejus usque ad internecionem, et possederunt terram illius.

\bchapter{22}
\lettrine[lines=10,image=true,loversize=0.05,lraise=-0.03]{P}{}rofectique castrametati sunt in campestribus Moab, ubi trans Jordanem Jericho sita est.
${}^{2}$~Videns autem Balac filius Sephor omnia qu\ae\ fecerat Isra\"el Amorrh\ae o,
${}^{3}$~et quod pertimuissent eum Moabit\ae , et impetum ejus ferre non possent,
${}^{4}$~dixit ad majores natu Madian~: Ita delebit hic populus omnes, qui in nostris finibus commorantur, quomodo solet bos herbas usque ad radices carpere. Ipse erat eo tempore rex in Moab.
${}^{5}$~Misit ergo nuntios ad Balaam filium Beor ariolum, qui habitabat super flumen terr\ae\ filiorum Ammon, ut vocarent eum, et dicerent~: Ecce egressus est populus ex \AE gypto, qui operuit superficiem terr\ae , sedens contra me.
${}^{6}$~Veni igitur, et maledic populo huic, quia fortior me est~: si quomodo possim percutere et ejicere eum de terra mea. Novi enim quod benedictus sit cui benedixeris, et maledictus in quem maledicta congesseris.
${}^{7}$~Perrexeruntque seniores Moab, et majores natu Madian, habentes divinationis pretium in manibus. Cumque venissent ad Balaam, et narrassent ei omnia verba Balac,
${}^{8}$~ille respondit~: Manete hic nocte, et respondebo quidquid mihi dixerit Dominus. Manentibus illis apud Balaam, venit Deus, et ait ad eum~:
${}^{9}$~Quid sibi volunt homines isti apud te~?
${}^{10}$~Respondit~: Balac filius Sephor rex Moabitarum misit ad me,
${}^{11}$~dicens~: Ecce populus qui egressus est de \AE gypto, operuit superficiem terr\ae~: veni, et maledic ei, si quomodo possim pugnans abigere eum.
${}^{12}$~Dixitque Deus ad Balaam~: Noli ire cum eis, neque maledicas populo~: quia benedictus est.
${}^{13}$~Qui mane consurgens dixit ad principes~: Ite in terram vestram, quia prohibuit me Dominus venire vobiscum.
${}^{14}$~Reversi principes dixerunt ad Balac~: Noluit Balaam venire nobiscum.


${}^{15}$~Rursum ille multo plures et nobiliores quam ante miserat, misit.
${}^{16}$~Qui cum venissent ad Balaam, dixerunt~: Sic dicit Balac filius Sephor~: Ne cuncteris venire ad me~:
${}^{17}$~paratus sum honorare te, et quidquid volueris, dabo tibi~: veni, et maledic populo isti.
${}^{18}$~Respondit Balaam~: Si dederit mihi Balac plenam domum suam argenti et auri, non potero immutare verbum Domini Dei mei, ut vel plus, vel minus loquar.
${}^{19}$~Obsecro ut hic maneatis etiam hac nocte, et scire queam quid mihi rursum respondeat Dominus.
${}^{20}$~Venit ergo Deus ad Balaam nocte, et ait ei~: Si vocare te venerunt homines isti, surge, et vade cum eis~: ita dumtaxat, ut quod tibi pr\ae cepero, facias.
${}^{21}$~Surrexit Balaam mane, et strata asina sua profectus est cum eis.


${}^{22}$~Et iratus est Deus. Stetitque angelus Domini in via contra Balaam, qui insidebat asin\ae , et duos pueros habebat secum.
${}^{23}$~Cernens asina angelum stantem in via, evaginato gladio, avertit se de itinere, et ibat per agrum. Quam cum verberaret Balaam, et vellet ad semitam reducere,
${}^{24}$~stetit angelus in angustiis duarum maceriarum, quibus vine\ae\ cingebantur.
${}^{25}$~Quem videns asina, junxit se parieti, et attrivit sedentis pedem. At ille iterum verberabat eam~:
${}^{26}$~et nihilominus angelus ad locum angustum transiens, ubi nec ad dexteram, nec ad sinistram poterat deviare, obvius stetit.
${}^{27}$~Cumque vidisset asina stantem angelum, concidit sub pedibus sedentis~: qui iratus, vehementius c\ae debat fuste latera ejus.
${}^{28}$~Aperuitque Dominus os asin\ae , et locuta est~: Quid feci tibi~? cur percutis me ecce jam tertio~?
${}^{29}$~Respondit Balaam~: Quia commeruisti, et illusisti mihi~: utinam haberem gladium, ut te percuterem~!
${}^{30}$~Dixit asina~: Nonne animal tuum sum, cui semper sedere consuevisti usque in pr\ae sentem diem~? dic quid simile umquam fecerim tibi. At ille ait~: Numquam.
${}^{31}$~Protinus aperuit Dominus oculos Balaam, et vidit angelum stantem in via, evaginato gladio, adoravitque eum pronus in terram.
${}^{32}$~Cui angelus~: Cur, inquit, tertio verberas asinam tuam~? ego veni ut adversarer tibi, quia perversa est via tua, mihique contraria~:
${}^{33}$~et nisi asina declinasset de via, dans locum resistenti, te occidissem, et illa viveret.


${}^{34}$~Dixit Balaam~: Peccavi, nesciens quod tu stares contra me~: et nunc si displicet tibi ut vadam, revertar.
${}^{35}$~Ait angelus~: Vade cum istis, et cave ne aliud quam pr\ae cepero tibi loquaris. Ivit igitur cum principibus.
${}^{36}$~Quod cum audisset Balac, egressus est in occursum ejus in oppido Moabitarum, quod situm est in extremis finibus Arnon.
${}^{37}$~Dixitque ad Balaam~: Misi nuntios ut vocarent te~: cur non statim venisti ad me~? an quia mercedem adventui tuo reddere nequeo~?
${}^{38}$~Cui ille respondit~: Ecce adsum~: numquid loqui potero aliud, nisi quod Deus posuerit in ore meo~?
${}^{39}$~Perrexerunt ergo simul, et venerunt in urbem, qu\ae\ in extremis regni ejus finibus erat.
${}^{40}$~Cumque occidisset Balac boves et oves, misit ad Balaam, et principes qui cum eo erant, munera.
${}^{41}$~Mane autem facto, duxit eum ad excelsa Baal, et intuitus est extremam partem populi.

\bchapter{23}
\lettrine[lines=10,image=true,loversize=0.05,lraise=-0.03]{D}{}ixitque Balaam ad Balac~: \AE difica mihi hic septem aras, et para totidem vitulos, ejusdemque numeri arietes.
${}^{2}$~Cumque fecisset juxta sermonem Balaam, imposuerunt simul vitulum et arietem super aram.
${}^{3}$~Dixitque Balaam ad Balac~: Sta paulisper juxta holocaustum tuum, donec vadam, si forte occurrat mihi Dominus, et quodcumque imperaverit, loquar tibi.
${}^{4}$~Cumque abiisset velociter, occurrit illi Deus. Locutusque ad eum Balaam~: Septem, inquit, aras erexi, et imposui vitulum et arietem desuper.
${}^{5}$~Dominus autem posuit verbum in ore ejus, et ait~: Revertere ad Balac, et h\ae c loqueris.
${}^{6}$~Reversus invenit stantem Balac juxta holocaustum suum, et omnes principes Moabitarum~:
${}^{7}$~assumptaque parabola sua, dixit~: \begin{flushleft}\begin{verse}De Aram adduxit me Balac\\ rex Moabitarum, de montibus orientis~:\\ Veni, inquit, et maledic Jacob~;\\ propera, et detestare Isra\"el.\\
${}^{8}$~Quomodo maledicam, cui non maledixit Deus~?\\ qua ratione detester, quem Dominus non detestatur~?\\
${}^{9}$~De summis silicibus videbo eum,\\ et de collibus considerabo illum.\\ Populus solus habitabit,\\ et inter gentes non reputabitur.\\
${}^{10}$~Quis dinumerare possit pulverem Jacob,\\ et nosse numerum stirpis Isra\"el~?\\ Moriatur anima mea morte justorum,\\ et fiant novissima mea horum similia.\end{verse}\end{flushleft}


${}^{11}$~Dixitque Balac ad Balaam~: Quid est hoc quod agis~? ut malediceres inimicis meis vocavi te, et tu e contrario benedicis eis.
${}^{12}$~Cui ille respondit~: Num aliud possum loqui, nisi quod jusserit Dominus~?
${}^{13}$~Dixit ergo Balac~: Veni mecum in alterum locum unde partem Isra\"el videas, et totum videre non possis~: inde maledicito ei.
${}^{14}$~Cumque duxisset eum in locum sublimem, super verticem montis Phasga, \ae dificavit Balaam septem aras, et impositis supra vitulo atque ariete,
${}^{15}$~dixit ad Balac~: Sta hic juxta holocaustum tuum, donec ego obvius pergam.
${}^{16}$~Cui cum Dominus occurrisset, posuissetque verbum in ore ejus, ait~: Revertere ad Balac, et h\ae c loqueris ei.
${}^{17}$~Reversus invenit eum stantem juxta holocaustum suum, et principes Moabitarum cum eo. Ad quem Balac~: Quid, inquit, locutus est Dominus~?
${}^{18}$~At ille, assumpta parabola sua, ait~: \begin{flushleft}\begin{verse}Sta, Balac, et ausculta~; audi, fili Sephor~:\\
${}^{19}$~non est Deus quasi homo, ut mentiatur,\\ nec ut filius hominis, ut mutetur.\\ Dixit ergo, et non faciet~?\\ locutus est, et non implebit~?\\
${}^{20}$~Ad benedicendum adductus sum~:\\ benedictionem prohibere non valeo.\\
${}^{21}$~Non est idolum in Jacob,\\ nec videtur simulacrum in Isra\"el.\\ Dominus Deus ejus cum eo est,\\ et clangor victori\ae\ regis in illo.\\
${}^{22}$~Deus eduxit illum de \AE gypto,\\ cujus fortitudo similis est rhinocerotis.\\
${}^{23}$~Non est augurium in Jacob,\\ nec divinatio in Isra\"el~:\\ temporibus suis dicetur Jacob et Isra\"eli quid operatus sit Deus.\\
${}^{24}$~Ecce populus ut le\ae na consurget,\\ et quasi leo erigetur~:\\ non accubabit donec devoret pr\ae dam,\\ et occisorum sanguinem bibat.\end{verse}\end{flushleft}


${}^{25}$~Dixitque Balac ad Balaam~: Nec maledicas ei, nec benedicas.
${}^{26}$~Et ille ait~: Nonne dixi tibi quod quidquid mihi Deus imperaret, hoc facerem~?
${}^{27}$~Et ait Balac ad eum~: Veni, et ducam te ad alium locum~: si forte placeat Deo ut inde maledicas eis.
${}^{28}$~Cumque duxisset eum super verticem montis Phogor, qui respicit solitudinem,
${}^{29}$~dixit ei Balaam~: \AE difica mihi hic septem aras, et para totidem vitulos, ejusdemque numeri arietes.
${}^{30}$~Fecit Balac ut Balaam dixerat~: imposuitque vitulos et arietes per singulas aras.

\bchapter{24}
\lettrine[lines=10,image=true,loversize=0.05,lraise=-0.03]{C}{}umque vidisset Balaam quod placeret Domino ut benediceret Isra\"eli, nequaquam abiit ut ante perrexerat, ut augurium qu\ae reret~: sed dirigens contra desertum vultum suum,
${}^{2}$~et elevans oculos, vidit Isra\"el in tentoriis commorantem per tribus suas~: et irruente in se spiritu Dei,
${}^{3}$~assumpta parabola, ait~: \begin{flushleft}\begin{verse}Dixit Balaam filius Beor~:\\ dixit homo, cujus obturatus est oculus~:\\
${}^{4}$~dixit auditor sermonum Dei,\\ qui visionem Omnipotentis intuitus est,\\ qui cadit, et sic aperiuntur oculi ejus~:\\
${}^{5}$~Quam pulchra tabernacula tua, Jacob,\\ et tentoria tua, Isra\"el~!\\
${}^{6}$~ut valles nemoros\ae ,\\ ut horti juxta fluvios irrigui,\\ ut tabernacula qu\ae\ fixit Dominus,\\ quasi cedri prope aquas.\\
${}^{7}$~Fluet aqua de situla ejus,\\ et semen illius erit in aquas multas.\\ Tolletur propter Agag, rex ejus,\\ et auferetur regnum illius.\\
${}^{8}$~Deus eduxit illum de \AE gypto,\\ cujus fortitudo similis est rhinocerotis.\\ Devorabunt gentes hostes illius,\\ ossaque eorum confringent, et perforabunt sagittis.\\
${}^{9}$~Accubans dormivit ut leo,\\ et quasi le\ae na, quam suscitare nullus audebit.\\ Qui benedixerit tibi, erit et ipse benedictus~:\\ qui maledixerit, in maledictione reputabitur.\end{verse}\end{flushleft}


${}^{10}$~Iratusque Balac contra Balaam, complosis manibus ait~: Ad maledicendum inimicis meis vocavi te, quibus e contrario tertio benedixisti~:
${}^{11}$~revertere ad locum tuum. Decreveram quidem magnifice honorare te, sed Dominus privavit te honore disposito.
${}^{12}$~Respondit Balaam ad Balac~: Nonne nuntiis tuis, quos misisti ad me, dixi~:
${}^{13}$~Si dederit mihi Balac plenam domum suam argenti et auri, non potero pr\ae terire sermonem Domini Dei mei, ut vel boni quid vel mali proferam ex corde meo~: sed quidquid Dominus dixerit, hoc loquar~?
${}^{14}$~verumtamen pergens ad populum meum, dabo consilium, quid populus tuus populo huic faciat extremo tempore.
${}^{15}$~Sumpta igitur parabola, rursum ait~: \begin{flushleft}\begin{verse}Dixit Balaam filius Beor~:\\ dixit homo, cujus obturatus est oculus~:\\
${}^{16}$~dixit auditor sermonum Dei,\\ qui novit doctrinam Altissimi,\\ et visiones Omnipotentis videt,\\ qui cadens apertos habet oculos~:\\
${}^{17}$~Videbo eum, sed non modo~:\\ intuebor illum, sed non prope.\\ Orietur stella ex Jacob,\\ et consurget virga de Isra\"el~:\\ et percutiet duces Moab,\\ vastabitque omnes filios Seth.\\
${}^{18}$~Et erit Idum\ae a possessio ejus~:\\ h\ae reditas Seir cedet inimicis suis~:\\ Isra\"el vero fortiter aget.\\
${}^{19}$~De Jacob erit qui dominetur,\\ et perdat reliquias civitatis.\\\end{verse}\end{flushleft}


${}^{20}$~Cumque vidisset Amalec, assumens parabolam, ait~: \begin{flushleft}\begin{verse}Principium gentium Amalec,\\ cujus extrema perdentur.\end{verse}\end{flushleft}


${}^{21}$~Vidit quoque Cin\ae um~: et assumpta parabola, ait~: \begin{flushleft}\begin{verse}Robustum quidem est habitaculum tuum~:\\ sed si in petra posueris nidum tuum,\\
${}^{22}$~et fueris electus de stirpe Cin,\\ quamdiu poteris permanere~? Assur enim capiet te.\end{verse}\end{flushleft}


${}^{23}$~Assumptaque parabola iterum locutus est~: \begin{flushleft}\begin{verse}Heu~! quis victurus est,\\ quando ista faciet Deus~?\\
${}^{24}$~Venient in trieribus de Italia~:\\ superabunt Assyrios, vastabuntque Hebr\ae os, et ad extremum etiam ipsi peribunt.\end{verse}\end{flushleft}


${}^{25}$~Surrexitque Balaam, et reversus est in locum suum~: Balac quoque via, qua venerat, rediit.

\bchapter{25}
\lettrine[lines=10,image=true,loversize=0.05,lraise=-0.03]{M}{}orabatur autem eo tempore Isra\"el in Settim, et fornicatus est populus cum filiabus Moab,
${}^{2}$~qu\ae\ vocaverunt eos ad sacrificia sua. At ille comederunt et adoraverunt deos earum.
${}^{3}$~Initiatusque est Isra\"el Beelphegor~: et iratus Dominus,
${}^{4}$~ait ad Moysen~: Tolle cunctos principes populi, et suspende eos contra solem in patibulis, ut avertatur furor meus ab Isra\"el.
${}^{5}$~Dixitque Moyses ad judices Isra\"el~: Occidat unusquisque proximos suos, qui initiati sunt Beelphegor.
${}^{6}$~Et ecce unus de filiis Isra\"el intravit coram fratribus suis ad scortum Madianitidem, vidente Moyse, et omni turba filiorum Isra\"el, qui flebant ante fores tabernaculi.


${}^{7}$~Quod cum vidisset Phinees filius Eleazari filii Aaron sacerdotis, surrexit de medio multitudinis, et arrepto pugione,
${}^{8}$~ingressus est post virum Isra\"elitem in lupanar, et perfodit ambos simul, virum scilicet et mulierem in locis genitalibus. Cessavitque plaga a filiis Isra\"el~:
${}^{9}$~et occisi sunt viginti quatuor millia hominum.
${}^{10}$~Dixitque Dominus ad Moysen~:
${}^{11}$~Phinees filius Eleazari filii Aaron sacerdotis avertit iram meam a filiis Isra\"el~: quia zelo meo commotus est contra eos, ut non ipse delerem filios Isra\"el in zelo meo.
${}^{12}$~Idcirco loquere ad eum~: Ecce do ei pacem fœderis mei,
${}^{13}$~et erit tam ipsi quam semini ejus pactum sacerdotii sempiternum~: quia zelatus est pro Deo suo, et expiavit scelus filiorum Isra\"el.
${}^{14}$~Erat autem nomen viri Isra\"elit\ae , qui occisus est cum Madianitide, Zambri filius Salu, dux de cognatione et tribu Simeonis.
${}^{15}$~Porro mulier Madianitis, qu\ae\ pariter interfecta est, vocabatur Cozbi filia Sur principis nobilissimi Madianitarum.
${}^{16}$~Locutusque est Dominus ad Moysen, dicens~:
${}^{17}$~Hostes vos sentiant Madianit\ae , et percutite eos~:
${}^{18}$~quia et ipsi hostiliter egerunt contra vos, et decepere insidiis per idolum Phogor, et Cozbi filiam ducis Madian sororem suam, qu\ae\ percussa est in die plag\ae\ pro sacrilegio Phogor.

\bchapter{26}
\lettrine[lines=10,image=true,loversize=0.05,lraise=-0.03]{P}{}ostquam noxiorum sanguis effusus est, dixit Dominus ad Moysen et Eleazarum filium Aaron sacerdotem~:
${}^{2}$~Numerate omnem summam filiorum Isra\"el a viginti annis et supra, per domos et cognationes suas, cunctos qui possunt ad bella procedere.
${}^{3}$~Locuti sunt itaque Moyses et Eleazar sacerdos, in campestribus Moab super Jordanem contra Jericho, ad eos qui erant
${}^{4}$~a viginti annis et supra, sicut Dominus imperaverat, quorum iste est numerus.
${}^{5}$~Ruben primogenitus Isra\"el~: hujus filius, Henoch, a quo familia Henochitarum~: et Phallu, a quo familia Phalluitarum~:
${}^{6}$~et Hesron, a quo familia Hesronitarum~: et Charmi, a quo familia Charmitarum.
${}^{7}$~H\ae\ sunt famili\ae\ de stirpe Ruben~: quarum numerus inventus est quadraginta tria millia, et septingenti triginta.
${}^{8}$~Filius Phallu, Eliab~:
${}^{9}$~hujus filii, Namuel et Dathan et Abiron~: isti sunt Dathan et Abiron principes populi, qui surrexerunt contra Moysen et Aaron in seditione Core, quando adversus Dominum rebellaverunt~:
${}^{10}$~et aperiens terra os suum devoravit Core, morientibus plurimis, quando combussit ignis ducentos quinquaginta viros. Et factum est grande miraculum,
${}^{11}$~ut, Core pereunte, filii illius non perirent.
${}^{12}$~Filii Simeon per cognationes suas~: Namuel, ab hoc familia Namuelitarum~: Jamin, ab hoc familia Jaminitarum~: Jachin, ab hoc familia Jachinitarum~:
${}^{13}$~Zare, ab hoc familia Zareitarum~: Saul, ab hoc familia Saulitarum.
${}^{14}$~H\ae\ sunt famili\ae\ de stirpe Simeon, quarum omnis numerus fuit viginti duo millia ducenti.
${}^{15}$~Filii Gad per cognationes suas~: Sephon, ab hoc familia Sephonitarum~: Aggi, ab hoc familia Aggitarum~: Suni, ab hoc familia Sunitarum~:
${}^{16}$~Ozni, ab hoc familia Oznitarum~: Her, ab hoc familia Heritarum~:
${}^{17}$~Arod, ab hoc familia Aroditarum~: Ariel, ab hoc familia Arielitarum.
${}^{18}$~Ist\ae\ sunt famili\ae\ Gad, quarum omnis numerus fuit quadraginta millia quingenti.
${}^{19}$~Filii Juda, Her et Onan, qui ambo mortui sunt in terra Chanaan.
${}^{20}$~Fueruntque filii Juda per cognationes suas~: Sela, a quo familia Selaitarum~: Phares, a quo familia Pharesitarum~: Zare, a quo familia Zareitarum.
${}^{21}$~Porro filii Phares~: Hesron, a quo familia Hesronitarum~: et Hamul, a quo familia Hamulitarum.
${}^{22}$~Ist\ae\ sunt famili\ae\ Juda, quarum omnis numerus fuit septuaginta sex millia quingenti.
${}^{23}$~Filii Issachar per cognationes suas~: Thola, a quo familia Tholaitarum~: Phua, a quo familia Phuaitarum~:
${}^{24}$~Jasub, a quo familia Jasubitarum~: Semran, a quo familia Semranitarum.
${}^{25}$~H\ae\ sunt cognationes Issachar, quarum numerus fuit sexaginta quatuor millia trecenti.
${}^{26}$~Filii Zabulon per cognationes suas~: Sared, a quo familia Sareditarum~: Elon, a quo familia Elonitarum~: Jalel, a quo familia Jalelitarum.
${}^{27}$~H\ae\ sunt cognationes Zabulon, quarum numerus fuit sexaginta millia quingenti.
${}^{28}$~Filii Joseph per cognationes suas, Manasse et Ephraim.
${}^{29}$~De Manasse ortus est Machir, a quo familia Machiritarum. Machir genuit Galaad, a quo familia Galaaditarum.
${}^{30}$~Galaad habuit filios~: Jezer, a quo familia Jezeritarum~: et Helec, a quo familia Helecitarum~:
${}^{31}$~et Asriel, a quo familia Asrielitarum~: et Sechem, a quo familia Sechemitarum~:
${}^{32}$~et Semida, a quo familia Semidaitarum~: et Hepher, a quo familia Hepheritarum.
${}^{33}$~Fuit autem Hepher pater Salphaad, qui filios non habebat, sed tantum filias~: quarum ista sunt nomina~: Maala, et Noa, et Hegla, et Melcha, et Thersa.
${}^{34}$~H\ae\ sunt famili\ae\ Manasse, et numerus earum quinquaginta duo millia septingenti.
${}^{35}$~Filii autem Ephraim per cognationes suas fuerunt hi~: Suthala, a quo familia Suthalaitarum~: Becher, a quo familia Becheritarum~: Thehen, a quo familia Thehenitarum.
${}^{36}$~Porro filius Suthala fuit Heran, a quo familia Heranitarum.
${}^{37}$~H\ae\ sunt cognationes filiorum Ephraim~: quarum numerus fuit triginta duo millia quingenti.
${}^{38}$~Isti sunt filii Joseph per familias suas. Filii Benjamin in cognationibus suis~: Bela, a quo familia Belaitarum~: Asbel, a quo familia Asbelitarum~: Ahiram, a quo familia Ahiramitarum~:
${}^{39}$~Supham, a quo familia Suphamitarum~: Hupham, a quo familia Huphamitarum.
${}^{40}$~Filii Bela~: Hered, et No\"eman. De Hered, familia Hereditarum~: de No\"eman, familia No\"emanitarum.
${}^{41}$~Hi sunt filii Benjamin per cognationes suas~: quorum numerus fuit quadraginta quinque millia sexcenti.
${}^{42}$~Filii Dan per cognationes suas~: Suham, a quo familia Suhamitarum. H\ae\ sunt cognationes Dan per familias suas.
${}^{43}$~Omnes fuere Suhamit\ae , quorum numerus erat sexaginta quatuor millia quadringenti.
${}^{44}$~Filii Aser per cognationes suas~: Jemna, a quo familia Jemnaitarum~: Jessui, a quo familia Jessuitarum~: Brie, a quo familia Brieitarum.
${}^{45}$~Filii Brie~: Heber, a quo familia Heberitarum~: et Melchiel, a quo familia Melchielitarum.
${}^{46}$~Nomen autem fili\ae\ Aser fuit Sara.
${}^{47}$~H\ae\ cognationes filiorum Aser, et numerus eorum quinquaginta tria millia quadringenti.
${}^{48}$~Filii Nephthali per cognationes suas~: Jesiel, a quo familia Jesielitarum~: Guni, a quo familia Gunitarum~:
${}^{49}$~Jeser, a quo familia Jeseritarum~: Sellem, a quo familia Sellemitarum.
${}^{50}$~H\ae\ sunt cognationes filiorum Nephthali per familias suas~: quorum numerus quadraginta quinque millia quadringenti.
${}^{51}$~Ista est summa filiorum Isra\"el, qui recensiti sunt, sexcenta millia, et mille septingenti triginta.
${}^{52}$~Locutusque est Dominus ad Moysen, dicens~:
${}^{53}$~Istis dividetur terra juxta numerum vocabulorum in possessiones suas.
${}^{54}$~Pluribus majorem partem dabis, et paucioribus minorem~: singulis, sicut nunc recensiti sunt, tradetur possessio~:
${}^{55}$~ita dumtaxat ut sors terram tribubus dividat et familiis.
${}^{56}$~Quidquid sorte contigerit, hoc vel plures accipiant, vel pauciores.


${}^{57}$~Hic quoque est numerus filiorum Levi per familias suas~: Gerson, a quo familia Gersonitarum~: Caath, a quo familia Caathitarum~: Merari, a quo familia Meraritarum.
${}^{58}$~H\ae\ sunt famili\ae\ Levi~: familia Lobni, familia Hebroni, familia Moholi, familia Musi, familia Core. At vero Caath genuit Amram~:
${}^{59}$~qui habuit uxorem Jochabed filiam Levi, qu\ae\ nata est ei in \AE gypto. H\ae c genuit Amram viro suo filios, Aaron, et Moysen, et Mariam sororem eorum.
${}^{60}$~De Aaron orti sunt Nadab et Abiu, et Eleazar et Ithamar~:
${}^{61}$~quorum Nadab et Abiu mortui sunt, cum obtulissent ignem alienum coram Domino.
${}^{62}$~Fueruntque omnes qui numerati sunt, viginti tria millia generis masculini ab uno mense et supra~: quia non sunt recensiti inter filios Isra\"el, nec eis cum ceteris data possessio est.
${}^{63}$~Hic est numerus filiorum Isra\"el, qui descripti sunt a Moyse et Eleazaro sacerdote, in campestribus Moab supra Jordanem contra Jericho~:
${}^{64}$~inter quos, nullus fuit eorum qui ante numerati sunt a Moyse et Aaron in deserto Sinai~:
${}^{65}$~pr\ae dixerat enim Dominus quod omnes morerentur in solitudine. Nullusque remansit ex eis, nisi Caleb filius Jephone, et Josue filius Nun.

\bchapter{27}
\lettrine[lines=10,image=true,loversize=0.05,lraise=-0.03]{A}{}ccesserunt autem fili\ae\ Salphaad, filii Hepher, filii Galaad, filii Machir, filii Manasse, qui fuit filius Joseph~: quarum sunt nomina, Maala, et Noa, et Hegla, et Melcha, et Thersa.
${}^{2}$~Steteruntque coram Moyse et Eleazaro sacerdote et cunctis principibus populi ad ostium tabernaculi fœderis, atque dixerunt~:
${}^{3}$~Pater noster mortuus est in deserto, nec fuit in seditione, qu\ae\ concitata est contra Dominum sub Core, sed in peccato suo mortuus est~: hic non habuit mares filios. Cur tollitur nomen illius de familia sua, quia non habuit filium~? date nobis possessionem inter cognatos patris nostri.
${}^{4}$~Retulitque Moyses causam earum ad judicium Domini.
${}^{5}$~Qui dixit ad eum~:
${}^{6}$~Justam rem postulant fili\ae\ Salphaad~: da eis possessionem inter cognatos patris sui, et ei in h\ae reditatem succedant.
${}^{7}$~Ad filios autem Isra\"el loqueris h\ae c~:
${}^{8}$~Homo cum mortuus fuerit absque filio, ad filiam ejus transibit h\ae reditas.
${}^{9}$~Si filiam non habuerit, habebit successores fratres suos.
${}^{10}$~Quod si et fratres non fuerint, dabitis h\ae reditatem fratribus patris ejus.
${}^{11}$~Sin autem nec patruos habuerit, dabitur h\ae reditas his qui ei proximi sunt. Eritque hoc filiis Isra\"el sanctum lege perpetua, sicut pr\ae cepit Dominus Moysi.


${}^{12}$~Dixit quoque Dominus ad Moysen~: Ascende in montem istum Abarim, et contemplare inde terram, quam daturus sum filiis Isra\"el.
${}^{13}$~Cumque videris eam, ibis et tu ad populum tuum, sicut ivit frater tuus Aaron~:
${}^{14}$~quia offendistis me in deserto Sin in contradictione multitudinis, nec sanctificare me voluistis coram ea super aquas. H\ae\ sunt aqu\ae\ contradictionis in Cades deserti Sin.
${}^{15}$~Cui respondit Moyses~:
${}^{16}$~Provideat Dominus Deus spirituum omnis carnis hominem, qui sit super multitudinem hanc~:
${}^{17}$~et possit exire et intrare ante eos, et educere eos vel introducere~: ne sit populus Domini sicut oves absque pastore.
${}^{18}$~Dixitque Dominus ad eum~: Tolle Josue filium Nun, virum in quo est spiritus, et pone manum tuam super eum.
${}^{19}$~Qui stabit coram Eleazaro sacerdote et omni multitudine~:
${}^{20}$~et dabis ei pr\ae cepta cunctis videntibus, et partem glori\ae\ tu\ae , ut audiat eum omnis synagoga filiorum Isra\"el.
${}^{21}$~Pro hoc, si quid agendum erit, Eleazar sacerdos consulet Dominum. Ad verbum ejus egredietur et ingredietur ipse, et omnes filii Isra\"el cum eo, et cetera multitudo.
${}^{22}$~Fecit Moyses ut pr\ae ceperat Dominus~: cumque tulisset Josue, statuit eum coram Eleazaro sacerdote et omni frequentia populi.
${}^{23}$~Et impositis capiti ejus manibus, cuncta replicavit qu\ae\ mandaverat Dominus.

\bchapter{28}
\lettrine[lines=10,image=true,loversize=0.05,lraise=-0.03]{D}{}ixit quoque Dominus ad Moysen~:
${}^{2}$~Pr\ae cipe filiis Isra\"el, et dices ad eos~: Oblationem meam et panes, et incensum odoris suavissimi offerte per tempora sua.
${}^{3}$~H\ae c sunt sacrificia qu\ae\ offerre debetis~: agnos anniculos immaculatos duos quotidie in holocaustum sempiternum~:
${}^{4}$~unum offeretis mane, et alterum ad vesperum~:
${}^{5}$~decimam partem ephi simil\ae , qu\ae\ conspersa sit oleo purissimo, et habeat quartam partem hin.
${}^{6}$~Holocaustum juge est quod obtulistis in monte Sinai in odorem suavissimum incensi Domini.
${}^{7}$~Et libabitis vini quartam partem hin per agnos singulos in sanctuario Domini.
${}^{8}$~Alterumque agnum similiter offeretis ad vesperam juxta omnem ritum sacrificii matutini, et libamentorum ejus, oblationem suavissimi odoris Domino.
${}^{9}$~Die autem sabbati offeretis duos agnos anniculos immaculatos, et duas decimas simil\ae\ oleo conspers\ae\ in sacrificio, et liba~:
${}^{10}$~qu\ae\ rite funduntur per singula sabbata in holocaustum sempiternum.
${}^{11}$~In calendis autem offeretis holocaustum Domino, vitulos de armento duos, arietem unum, agnos anniculos septem immaculatos,
${}^{12}$~et tres decimas simil\ae\ oleo conspers\ae\ in sacrificio per singulos vitulos~: et duas decimas simil\ae\ oleo conspers\ae\ per singulos arietes~:
${}^{13}$~et decimam decim\ae\ simil\ae\ ex oleo in sacrificio per agnos singulos~: holocaustum suavissimi odoris atque incensi est Domino.
${}^{14}$~Libamenta autem vini, qu\ae\ per singulas fundenda sunt victimas, ista erunt~: media pars hin per singulos vitulos, tertia per arietem, quarta per agnum. Hoc erit holocaustum per omnes menses, qui sibi anno vertente succedunt.
${}^{15}$~Hircus quoque offeretur Domino pro peccatis in holocaustum sempiternum cum libamentis suis.
${}^{16}$~Mense autem primo, quartadecima die mensis, Phase Domini erit,
${}^{17}$~et quintadecima die solemnitas~: septem diebus vescentur azymis.
${}^{18}$~Quarum dies prima venerabilis et sancta erit~: omne opus servile non facietis in ea.
${}^{19}$~Offeretisque incensum holocaustum Domino, vitulos de armento duos, arietem unum, agnos anniculos immaculatos septem~:
${}^{20}$~et sacrificia singulorum ex simila qu\ae\ conspersa sit oleo, tres decimas per singulos vitulos, et duas decimas per arietem,
${}^{21}$~et decimam decim\ae\ per agnos singulos, id est, per septem agnos.
${}^{22}$~Et hircum pro peccato unum, ut expietur pro vobis,
${}^{23}$~pr\ae ter holocaustum matutinum, quod semper offeretis.
${}^{24}$~Ita facietis per singulos dies septem dierum in fomitem ignis, et in odorem suavissimum Domino, qui surget de holocausto, et de libationibus singulorum.
${}^{25}$~Dies quoque septimus celeberrimus et sanctus erit vobis~: omne opus servile non facietis in eo.
${}^{26}$~Dies etiam primitivorum, quando offeretis novas fruges Domino, expletis hebdomadibus, venerabilis et sancta erit~: omne opus servile non facietis in ea.
${}^{27}$~Offeretisque holocaustum in odorem suavissimum Domino, vitulos de armento duos, arietem unum, et agnos anniculos immaculatos septem~:
${}^{28}$~atque in sacrificiis eorum, simil\ae\ oleo conspers\ae\ tres decimas per singulos vitulos, per arietes duas,
${}^{29}$~per agnos decimam decim\ae , qui simul sunt agni septem. Hircum quoque,
${}^{30}$~qui mactatur pro expiatione~: pr\ae ter holocaustum sempiternum et liba ejus.
${}^{31}$~Immaculata offeretis omnia cum libationibus suis.

\bchapter{29}
\lettrine[lines=10,image=true,loversize=0.05,lraise=-0.03]{M}{}ensis etiam septimi prima dies venerabilis et sancta erit vobis. Omne opus servile non facietis in ea, quia dies clangoris est et tubarum.
${}^{2}$~Offeretisque holocaustum in odorem suavissimum Domino, vitulum de armento unum, arietem unum, et agnos anniculos immaculatos septem~:
${}^{3}$~et in sacrificiis eorum, simil\ae\ oleo conspers\ae\ tres decimas per singulos vitulos, duas decimas per arietem,
${}^{4}$~unam decimam per agnum, qui simul sunt agni septem~:
${}^{5}$~et hircum pro peccato, qui offertur in expiationem populi,
${}^{6}$~pr\ae ter holocaustum calendarum cum sacrificiis suis, et holocaustum sempiternum cum libationibus solitis~: eisdem c\ae remoniis offeretis in odorem suavissimum incensum Domino.
${}^{7}$~Decima quoque dies mensis hujus septimi erit vobis sancta atque venerabilis, et affligetis animas vestras~: omne opus servile non facietis in ea.
${}^{8}$~Offeretisque holocaustum Domino in odorem suavissimum, vitulum de armento unum, arietem unum, agnos anniculos immaculatos septem~:
${}^{9}$~et in sacrificiis eorum simil\ae\ oleo conspers\ae\ tres decimas per singulos vitulos, duas decimas per arietem,
${}^{10}$~decimam decim\ae\ per agnos singulos, qui sunt simul agni septem~:
${}^{11}$~et hircum pro peccato, absque his qu\ae\ offerri pro delicto solent in expiationem, et holocaustum sempiternum, cum sacrificio et libaminibus eorum.
${}^{12}$~Quintadecima vero die mensis septimi, qu\ae\ vobis sancta erit atque venerabilis, omne opus servile non facietis in ea, sed celebrabitis solemnitatem Domino septem diebus.
${}^{13}$~Offeretisque holocaustum in odorem suavissimum Domino, vitulos de armento tredecim, arietes duos, agnos anniculos immaculatos quatuordecim~:
${}^{14}$~et in libamentis eorum, simil\ae\ oleo conspers\ae\ tres decimas per vitulos singulos, qui sunt simul vituli tredecim, et duas decimas arieti uno, id est, simul arietibus duobus,
${}^{15}$~et decimam decim\ae\ agnis singulis, qui sunt simul agni quatuordecim~:
${}^{16}$~et hircum pro peccato, absque holocausto sempiterno, et sacrificio, et libamine ejus.
${}^{17}$~In die altero offeretis vitulos de armento duodecim, arietes duos, agnos anniculos immaculatos quatuordecim~:
${}^{18}$~sacrificiaque et libamina singulorum, per vitulos, et arietes, et agnos rite celebrabitis~:
${}^{19}$~et hircum pro peccato, absque holocausto sempiterno, sacrificioque et libamine ejus.
${}^{20}$~Die tertio offeretis vitulos undecim, arietes duos, agnos anniculos immaculatos quatuordecim~:
${}^{21}$~sacrificiaque et libamina singulorum, per vitulos, et arietes, et agnos rite celebrabitis~:
${}^{22}$~et hircum pro peccato, absque holocausto sempiterno, sacrificioque et libamine ejus.
${}^{23}$~Die quarto offeretis vitulos decem, arietes duos, agnos anniculos immaculatos quatuordecim~:
${}^{24}$~sacrificiaque et libamina singulorum, per vitulos, et arietes, et agnos rite celebrabitis~:
${}^{25}$~et hircum pro peccato, absque holocausto sempiterno, sacrificioque ejus et libamine.
${}^{26}$~Die quinto offeretis vitulos novem, arietes duos, agnos anniculos immaculatos quatuordecim~:
${}^{27}$~sacrificiaque et libamina singulorum, per vitulos, et arietes, et agnos rite celebrabitis~:
${}^{28}$~et hircum pro peccato, absque holocausto sempiterno, sacrificioque ejus et libamine.
${}^{29}$~Die sexto offeretis vitulos octo, arietes duos, agnos anniculos immaculatos quatuordecim~:
${}^{30}$~sacrificiaque et libamina singulorum, per vitulos, et arietes, et agnos rite celebrabitis~:
${}^{31}$~et hircum pro peccato, absque holocausto sempiterno, sacrificioque ejus et libamine.
${}^{32}$~Die septimo offeretis vitulos septem, et arietes duos, agnos anniculos immaculatos quatuordecim~:
${}^{33}$~sacrificiaque et libamina singulorum, per vitulos, et arietes, et agnos rite celebrabitis~:
${}^{34}$~et hircum pro peccato, absque holocausto sempiterno, sacrificioque ejus et libamine.
${}^{35}$~Die octavo, qui est celeberrimus, omne opus servile non facietis,
${}^{36}$~offerentes holocaustum in odorem suavissimum Domino, vitulum unum, arietem unum, agnos anniculos immaculatos septem~:
${}^{37}$~sacrificiaque et libamina singulorum, per vitulos, et arietes, et agnos rite celebrabitis~:
${}^{38}$~et hircum pro peccato, absque holocausto sempiterno, sacrificioque ejus et libamine.
${}^{39}$~H\ae c offeretis Domino in solemnitatibus vestris~: pr\ae ter vota et oblationes spontaneas in holocausto, in sacrificio, in libamine, et in hostiis pacificis.

\bchapter{30}
\lettrine[lines=10,image=true,loversize=0.05,lraise=-0.03]{N}{}arravitque Moyses filiis Isra\"el omnia qu\ae\ ei Dominus imperarat.
${}^{2}$~Et locutus est ad principes tribuum filiorum Isra\"el~: Iste est sermo quem pr\ae cepit Dominus~:
${}^{3}$~Si quis virorum votum Domino voverit, aut se constrinxerit juramento~: non faciet irritum verbum suum, sed omne quod promisit, implebit.
${}^{4}$~Mulier si quippiam voverit, et se constrinxerit juramento, qu\ae\ est in domo patris sui, et in \ae tate adhuc puellari~: si cognoverit pater votum quod pollicita est, et juramentum quo obligavit animam suam, et tacuerit, voti rea erit~:
${}^{5}$~quidquid pollicita est, et juravit, opere complebit.
${}^{6}$~Sin autem statim ut audierit, contradixerit pater~: et vota et juramenta ejus irrita erunt, nec obnoxia tenebitur sponsioni, eo quod contradixerit pater.
${}^{7}$~Si maritum habuerit, et voverit aliquid, et semel de ore ejus verbum egrediens animam ejus obligaverit juramento~:
${}^{8}$~quo die audierit vir, et non contradixerit, voti rea erit, reddetque quodcumque promiserat.
${}^{9}$~Sin autem audiens statim contradixerit, et irritas fecerit pollicitationes ejus, verbaque quibus obstrinxerat animam suam, propitius erit ei Dominus.
${}^{10}$~Vidua et repudiata quidquid voverint, reddent.
${}^{11}$~Uxor in domo viri cum se voto constrinxerit et juramento,
${}^{12}$~si audierit vir, et tacuerit, nec contradixerit sponsioni, reddet quodcumque promiserat.
${}^{13}$~Sin autem extemplo contradixerit, non tenebitur promissionis rea~: quia maritus contradixit, et Dominus ei propitius erit.
${}^{14}$~Si voverit, et juramento se constrinxerit, ut per jejunium, vel ceterarum rerum abstinentiam affligat animam suam, in arbitrio viri erit ut faciat, sive non faciat.
${}^{15}$~Quod si audiens vir tacuerit, et in alteram diem distulerit sententiam, quidquid voverat atque promiserat, reddet~: quia statim ut audivit, tacuit.
${}^{16}$~Sin autem contradixerit postquam rescivit, portabit ipse iniquitatem ejus.
${}^{17}$~Ist\ae\ sunt leges, quas constituit Dominus Moysi inter virum et uxorem, inter patrem et filiam, qu\ae\ in puellari adhuc \ae tate est, vel qu\ae\ manet in parentis domo.

\bchapter{31}
\lettrine[lines=10,image=true,loversize=0.05,lraise=-0.03]{L}{}ocutusque est Dominus ad Moysen, dicens~:
${}^{2}$~Ulciscere prius filios Isra\"el de Madianitis, et sic colligeris ad populum tuum.
${}^{3}$~Statimque Moyses~: Armate, inquit, ex vobis viros ad pugnam, qui possint ultionem Domini expetere de Madianitis.
${}^{4}$~Mille viri de singulis tribubus eligantur ex Isra\"el qui mittantur ad bellum.
${}^{5}$~Dederuntque millenos de singulis tribubus, id est, duodecim millia expeditorum ad pugnam~:
${}^{6}$~quos misit Moyses cum Phinees filio Eleazari sacerdotis, vasa quoque sancta, et tubas ad clangendum tradidit ei.
${}^{7}$~Cumque pugnassent contra Madianitas atque vicissent, omnes mares occiderunt,
${}^{8}$~et reges eorum, Evi, et Recem, et Sur, et Hur, et Rebe, quinque principes gentis~: Balaam quoque filium Beor interfecerunt gladio.
${}^{9}$~Ceperuntque mulieres eorum, et parvulos, omniaque pecora, et cunctam supellectilem~: quidquid habere potuerant depopulati sunt~:
${}^{10}$~tam urbes quam viculos et castella flamma consumpsit.
${}^{11}$~Et tulerunt pr\ae dam, et universa qu\ae\ ceperant tam ex hominibus quam ex jumentis,
${}^{12}$~et adduxerunt ad Moysen, et Eleazarum sacerdotem, et ad omnem multitudinem filiorum Isra\"el~: reliqua autem utensilia portaverunt ad castra in campestribus Moab juxta Jordanem contra Jericho.


${}^{13}$~Egressi sunt autem Moyses et Eleazar sacerdos, et omnes principes synagog\ae , in occursum eorum extra castra.
${}^{14}$~Iratusque Moyses principibus exercitus, tribunis, et centurionibus qui venerant de bello,
${}^{15}$~ait~: Cur feminas reservastis~?
${}^{16}$~nonne ist\ae\ sunt, qu\ae\ deceperunt filios Isra\"el ad suggestionem Balaam, et pr\ae varicari vos fecerunt in Domino super peccato Phogor, unde et percussus est populus~?
${}^{17}$~ergo cunctos interficite quidquid est generis masculini, etiam in parvulis~: et mulieres, qu\ae\ noverunt viros in coitu, jugulate~:
${}^{18}$~puellas autem et omnes feminas virgines reservate vobis~:
${}^{19}$~et manete extra castra septem diebus. Qui occiderit hominem, vel occisum tetigerit, lustrabitur die tertio et septimo.
${}^{20}$~Et de omni pr\ae da, sive vestimentum fuerit, sive vas, et aliquid in utensilia pr\ae paratum, de caprarum pellibus, et pilis, et ligno, expiabitur.


${}^{21}$~Eleazar quoque sacerdos ad viros exercitus, qui pugnaverunt, sic locutus est~: Hoc est pr\ae ceptum legis, quod mandavit Dominus Moysi~:
${}^{22}$~aurum, et argentum, et \ae s, et ferrum, et plumbum, et stannum,
${}^{23}$~et omne, quod potest transire per flammas, igne purgabitur~: quidquid autem ignem non potest sustinere, aqua expiationis sanctificabitur~:
${}^{24}$~et lavabitis vestimenta vestra die septimo, et purificati postea castra intrabitis.
${}^{25}$~Dixit quoque Dominus ad Moysen~:
${}^{26}$~Tollite summam eorum qu\ae\ capta sunt, ab homine usque ad pecus, tu et Eleazar sacerdos et principes vulgi~:
${}^{27}$~dividesque ex \ae quo pr\ae dam inter eos qui pugnaverunt egressique sunt ad bellum, et inter omnem reliquam multitudinem.
${}^{28}$~Et separabis partem Domino ab his qui pugnaverunt et fuerunt in bello, unam animam de quingentis, tam ex hominibus quam ex bobus et asinis et ovibus,
${}^{29}$~et dabis eam Eleazaro sacerdoti, quia primiti\ae\ Domini sunt.
${}^{30}$~Ex media quoque parte filiorum Isra\"el accipies quinquagesimum caput hominum, et boum, et asinorum, et ovium, cunctorum animantium, et dabis ea Levitis, qui excubant in custodiis tabernaculi Domini.


${}^{31}$~Feceruntque Moyses et Eleazar sicut pr\ae ceperat Dominus.
${}^{32}$~Fuit autem pr\ae da, quam exercitus ceperat, ovium sexcenta septuaginta quinque millia,
${}^{33}$~boum septuaginta duo millia,
${}^{34}$~asinorum sexaginta millia et mille~:
${}^{35}$~anim\ae\ hominum sexus feminei, qu\ae\ non cognoverant viros, triginta duo millia.
${}^{36}$~Dataque est media pars his qui in pr\ae lio fuerant, ovium trecenta triginta septem millia quingent\ae~:
${}^{37}$~e quibus in partem Domini supputat\ae\ sunt oves sexcent\ae\ septuaginta quinque~:
${}^{38}$~et de bobus triginta sex millibus, boves septuaginta et duo~:
${}^{39}$~de asinis triginta millibus quingentis, asini sexaginta unus~:
${}^{40}$~de animabus hominum sedecim millibus, cesserunt in partem Domini triginta du\ae\ anim\ae .
${}^{41}$~Tradiditque Moyses numerum primitiarum Domini Eleazaro sacerdoti, sicut fuerat ei imperatum,
${}^{42}$~ex media parte filiorum Isra\"el, quam separaverat his qui in pr\ae lio fuerant.
${}^{43}$~De media vero parte, qu\ae\ contigerat reliqu\ae\ multitudini, id est, de ovibus trecentis triginta septem millibus quingentis,
${}^{44}$~et de bobus triginta sex millibus,
${}^{45}$~et de asinis triginta millibus quingentis,
${}^{46}$~et de hominibus sedecim millibus,
${}^{47}$~tulit Moyses quinquagesimum caput, et dedit Levitis, qui excubabant in tabernaculo Domini, sicut pr\ae ceperat Dominus.


${}^{48}$~Cumque accessissent principes exercitus ad Moysen, et tribuni, centurionesque, dixerunt~:
${}^{49}$~Nos servi tui recensuimus numerum pugnatorum, quos habuimus sub manu nostra~: et ne unus quidem defuit.
${}^{50}$~Ob hanc causam offerimus in donariis Domini singuli quod in pr\ae da auri potuimus invenire, periscelides et armillas, annulos et dextralia, ac mur\ae nulas, ut depreceris pro nobis Dominum.
${}^{51}$~Susceperuntque Moyses et Eleazar sacerdos omne aurum in diversis speciebus,
${}^{52}$~pondo sedecim millia septingentos quinquaginta siclos, a tribunis et centurionibus.
${}^{53}$~Unusquisque enim quod in pr\ae da rapuerat, suum erat.
${}^{54}$~Et susceptum intulerunt in tabernaculum testimonii, in monimentum filiorum Isra\"el coram Domino.

\bchapter{32}
\lettrine[lines=10,image=true,loversize=0.05,lraise=-0.03]{F}{}ilii autem Ruben et Gad habebant pecora multa, et erat illis in jumentis infinita substantia. Cumque vidissent Jazer et Galaad aptas animalibus alendis terras,
${}^{2}$~venerunt ad Moysen, et ad Elezarum sacerdotem, et principes multitudinis, atque dixerunt~:
${}^{3}$~Ataroth, et Dibon, et Jazer, et Nemra, Hesebon, et Eleale, et Saban, et Nebo, et Beon,
${}^{4}$~terra, quam percussit Dominus in conspectu filiorum Isra\"el, regio uberrima est ad pastum animalium~: et nos servi tui habemus jumenta plurima,
${}^{5}$~precamurque si invenimus gratiam coram te, ut des nobis famulis tuis eam in possessionem, nec facias nos transire Jordanem.
${}^{6}$~Quibus respondit Moyses~: Numquid fratres vestri ibunt ad pugnam, et vos hic sedebitis~?
${}^{7}$~cur subvertitis mentes filiorum Isra\"el, ne transire audeant in locum, quem eis daturus est Dominus~?
${}^{8}$~Nonne ita egerunt patres vestri, quando misi de Cadesbarne ad explorandam terram~?
${}^{9}$~cumque venissent usque ad Vallem botri, lustrata omni regione, subverterunt cor filiorum Isra\"el, ut non intrarent fines, quos eis Dominus dedit.
${}^{10}$~Qui iratus juravit, dicens~:
${}^{11}$~Si videbunt homines isti, qui ascenderunt ex \AE gypto a viginti annis et supra, terram, quam sub juramento pollicitus sum Abraham, Isaac, et Jacob~: et noluerunt sequi me,
${}^{12}$~pr\ae ter Caleb filium Jephone Cenez\ae um, et Josue filium Nun~: isti impleverunt voluntatem meam.
${}^{13}$~Iratusque Dominus adversum Isra\"el, circumduxit eum per desertum quadraginta annis, donec consumeretur universa generatio, qu\ae\ fecerat malum in conspectu ejus.
${}^{14}$~Et ecce, inquit, vos surrexistis pro patribus vestris, incrementa et alumni hominum peccatorum, ut augeretis furorem Domini contra Isra\"el.
${}^{15}$~Quod si nolueritis sequi eum, in solitudine populum derelinquet, et vos causa eritis necis omnium.
${}^{16}$~At illi prope accedentes, dixerunt~: Caulas ovium fabricabimus, et stabula jumentorum, parvulis quoque nostris urbes munitas~:
${}^{17}$~nos autem ipsi armati et accincti pergemus ad pr\ae lium ante filios Isra\"el, donec introducamus eos ad loca sua. Parvuli nostri, et quidquid habere possumus, erunt in urbibus muratis, propter habitatorum insidias.
${}^{18}$~Non revertemur in domos nostras, usque dum possideant filii Isra\"el h\ae reditatem suam~:
${}^{19}$~nec quidquam qu\ae remus trans Jordanem, quia jam habemus nostram possessionem in orientali ejus plaga.


${}^{20}$~Quibus Moyses ait~: Si facitis quod promittitis, expediti pergite coram Domino ad pugnam~:
${}^{21}$~et omnis vir bellator armatus Jordanem transeat, donec subvertat Dominus inimicos suos,
${}^{22}$~et subjiciatur ei omnis terra~: tunc eritis inculpabiles apud Dominum et apud Isra\"el, et obtinebitis regiones, quas vultis, coram Domino.
${}^{23}$~Sin autem quod dicitis, non feceritis, nulli dubium est quin peccetis in Deum~: et scitote quoniam peccatum vestrum apprehendet vos.
${}^{24}$~\AE dificate ergo urbes parvulis vestris, et caulas, et stabula ovibus ac jumentis~: et quod polliciti estis, implete.
${}^{25}$~Dixeruntque filii Gad et Ruben ad Moysen~: Servi tui sumus~: faciemus quod jubet dominus noster.
${}^{26}$~Parvulos nostros, et mulieres, et pecora, ac jumenta relinquemus in urbibus Galaad~:
${}^{27}$~nos autem famuli tui omnes expediti pergemus ad bellum, sicut tu, domine, loqueris.
${}^{28}$~Pr\ae cepit ergo Moyses Eleazaro sacerdoti, et Josue filio Nun, et principibus familiarum per tribus Isra\"el, et dixit ad eos~:
${}^{29}$~Si transierint filii Gad et filii Ruben vobiscum Jordanem omnes armati ad bellum coram Domino, et vobis fuerit terra subjecta, date eis Galaad in possessionem.
${}^{30}$~Sin autem noluerint transire armati vobiscum in terram Chanaan, inter vos habitandi accipiant loca.
${}^{31}$~Responderuntque filii Gad et filii Ruben~: Sicut locutus est Dominus servis suis, ita faciemus~:
${}^{32}$~ipsi armati pergemus coram Domino in terram Chanaan, et possessionem jam suscepisse nos confitemur trans Jordanem.


${}^{33}$~Dedit itaque Moyses filiis Gad et Ruben, et dimidi\ae\ tribui Manasse filii Joseph, regnum Sehon regis Amorrh\ae i, et regnum Og regis Basan, et terram eorum cum urbibus suis per circuitum.
${}^{34}$~Igitur exstruxerunt filii Gad, Dibon, et Ataroth, et Aro\"er,
${}^{35}$~et Etroth, et Sophan, et Jazer, et Jegbaa,
${}^{36}$~et Bethnemra, et Betharan, urbes munitas, et caulas pecoribus suis.
${}^{37}$~Filii vero Ruben \ae dificaverunt Hesebon, et Eleale, et Cariathaim,
${}^{38}$~et Nabo, et Baalmeon versis nominibus, Sabama quoque~: imponentes vocabula urbibus, quas exstruxerunt.
${}^{39}$~Porro filii Machir filii Manasse, perrexerunt in Galaad, et vastaverunt eam interfecto Amorrh\ae o habitatore ejus.
${}^{40}$~Dedit ergo Moyses terram Galaad Machir filio Manasse, qui habitavit in ea.
${}^{41}$~Jair autem filius Manasse abiit, et occupavit vicos ejus, quos appellavit Havoth Jair, id est, Villas Jair.
${}^{42}$~Nobe quoque perrexit, et apprehendit Chanath cum viculis suis~: vocavitque eam ex nomine suo Nobe.

\bchapter{33}
\lettrine[lines=10,image=true,loversize=0.05,lraise=-0.03]{H}{}\ae\ sunt mansiones filiorum Isra\"el, qui egressi sunt de \AE gypto per turmas suas in manu Moysi et Aaron,
${}^{2}$~quas descripsit Moyses juxta castrorum loca, qu\ae\ Domini jussione mutabant.
${}^{3}$~Profecti igitur de Ramesse mense primo, quintadecima die mensis primi, altera die Phase, filii Isra\"el in manu excelsa, videntibus cunctis \AE gyptiis,
${}^{4}$~et sepelientibus primogenitos, quos percusserat Dominus (nam et in diis eorum exercuerat ultionem),
${}^{5}$~castrametati sunt in Soccoth.
${}^{6}$~Et de Soccoth venerunt in Etham, qu\ae\ est in extremis finibus solitudinis.
${}^{7}$~Inde egressi venerunt contra Phihahiroth, qu\ae\ respicit Beelsephon, et castrametati sunt ante Magdalum.
${}^{8}$~Profectique de Phihahiroth, transierunt per medium mare in solitudinem~: et ambulantes tribus diebus per desertum Etham, castrametati sunt in Mara.
${}^{9}$~Profectique de Mara, venerunt in Elim, ubi erant duodecim fontes aquarum, et palm\ae\ septuaginta~: ibique castrametati sunt.
${}^{10}$~Sed et inde egressi, fixerunt tentoria super mare Rubrum. Profectique de mari Rubro,
${}^{11}$~castrametati sunt in deserto Sin.
${}^{12}$~Unde egressi, venerunt in Daphca.
${}^{13}$~Profectique de Daphca, castrametati sunt in Alus.
${}^{14}$~Egressique de Alus, in Raphidim fixere tentoria, ubi populo defuit aqua ad bibendum.
${}^{15}$~Profectique de Raphidim, castrametati sunt in deserto Sinai.
${}^{16}$~Sed et de solitudine Sinai egressi, venerunt ad sepulchra concupiscenti\ae .
${}^{17}$~Profectique de sepulchris concupiscenti\ae , castrametati sunt in Haseroth.
${}^{18}$~Et de Haseroth venerunt in Rethma.
${}^{19}$~Profectique de Rethma, castrametati sunt in Remmomphares.
${}^{20}$~Unde egressi venerunt in Lebna.
${}^{21}$~De Lebna castrametati sunt in Ressa.
${}^{22}$~Egressique de Ressa, venerunt in Ceelatha.
${}^{23}$~Unde profecti, castrametati sunt in monte Sepher.
${}^{24}$~Egressi de monte Sepher, venerunt in Arada.
${}^{25}$~Inde proficiscentes, castrametati sunt in Maceloth.
${}^{26}$~Profectique de Maceloth, venerunt in Thahath.
${}^{27}$~De Thahath, castrametati sunt in Thare.
${}^{28}$~Unde egressi, fixere tentoria in Methca.
${}^{29}$~Et de Methca, castrametati sunt in Hesmona.
${}^{30}$~Profectique de Hesmona, venerunt in Moseroth.
${}^{31}$~Et de Moseroth, castrametati sunt in Benejaacan.
${}^{32}$~Profectique de Benejaacan, venerunt in montem Gadgad.
${}^{33}$~Unde profecti, castrametati sunt in Jetebatha.
${}^{34}$~Et de Jetebatha venerunt in Hebrona.
${}^{35}$~Egressique de Hebrona, castrametati sunt in Asiongaber.
${}^{36}$~Inde profecti, venerunt in desertum Sin, h\ae c est Cades.
${}^{37}$~Egressique de Cades, castrametati sunt in monte Hor, in extremis finibus terr\ae\ Edom.
${}^{38}$~Ascenditque Aaron sacerdos in montem Hor jubente Domino~: et ibi mortuus est anno quadragesimo egressionis filiorum Isra\"el ex \AE gypto, mense quinto, prima die mensis,
${}^{39}$~cum esset annorum centum viginti trium.
${}^{40}$~Audivitque Chanan\ae us rex Arad, qui habitabat ad meridiem, in terram Chanaan venisse filios Isra\"el.
${}^{41}$~Et profecti de monte Hor, castrametati sunt in Salmona.
${}^{42}$~Unde egressi, venerunt in Phunon.
${}^{43}$~Profectique de Phunon, castrametati sunt in Oboth.
${}^{44}$~Et de Oboth venerunt in Ijeabarim, qu\ae\ est in finibus Moabitarum.
${}^{45}$~Profectique de Ijeabarim, fixere tentoria in Dibongad.
${}^{46}$~Unde egressi, castrametati sunt in Helmondeblathaim.
${}^{47}$~Egressique de Helmondeblathaim, venerunt ad montes Abarim contra Nabo.
${}^{48}$~Profectique de montibus Abarim, transierunt ad campestria Moab, supra Jordanem, contra Jericho.
${}^{49}$~Ibique castrametati sunt de Bethsimoth usque ad Abelsatim in planioribus locis Moabitarum.


${}^{50}$~Ubi locutus est Dominus ad Moysen~:
${}^{51}$~Pr\ae cipe filiis Isra\"el, et dic ad eos~: Quando transieritis Jordanem, intrantes terram Chanaan,
${}^{52}$~disperdite cunctos habitatores terr\ae\ illius~: confringite titulos, et statuas comminuite, atque omnia excelsa vastate,
${}^{53}$~mundantes terram, et habitantes in ea. Ego enim dedi vobis illam in possessionem,
${}^{54}$~quam dividetis vobis sorte. Pluribus dabitis latiorem, et paucis angustiorem. Singulis ut sors ceciderit, ita tribuetur h\ae reditas. Per tribus et familias possessio dividetur.
${}^{55}$~Sin autem nolueritis interficere habitatores terr\ae~: qui remanserint, erunt vobis quasi clavi in oculis, et lance\ae\ in lateribus, et adversabuntur vobis in terra habitationis vestr\ae~:
${}^{56}$~et quidquid illis cogitaveram facere, vobis faciam.

\bchapter{34}
\lettrine[lines=10,image=true,loversize=0.05,lraise=-0.03]{L}{}ocutusque est Dominus ad Moysen, dicens~:
${}^{2}$~Pr\ae cipe filiis Isra\"el, et dices ad eos~: Cum ingressi fueritis terram Chanaan, et in possessionem vobis sorte ceciderit, his finibus terminabitur.
${}^{3}$~Pars meridiana incipiet a solitudine Sin, qu\ae\ est juxta Edom~: et habebit terminos contra orientem mare salsissimum.
${}^{4}$~Qui circuibunt australem plagam per ascensum Scorpionis, ita ut transeant in Senna, et perveniant a meridie usque ad Cadesbarne, unde egredientur confinia ad villam nomine Adar, et tendent usque ad Asemona.
${}^{5}$~Ibitque per gyrum terminus ab Asemona usque ad torrentem \AE gypti, et maris magni littore finietur.
${}^{6}$~Plaga autem occidentalis a mari magno incipiet, et ipso fine claudetur.
${}^{7}$~Porro ad septentrionalem plagam a mari magno termini incipient, pervenientes usque ad montem altissimum,
${}^{8}$~a quo venient in Emath usque ad terminos Sedada~:
${}^{9}$~ibuntque confinia usque ad Zephrona, et villam Enan. Hi erunt termini in parte aquilonis.
${}^{10}$~Inde metabuntur fines contra orientalem plagam de villa Enan usque Sephama,
${}^{11}$~et de Sephama descendent termini in Rebla contra fontem Daphnim~: inde pervenient contra orientem ad mare Cenereth,
${}^{12}$~et tendent usque ad Jordanem, et ad ultimum salsissimo claudentur mari. Hanc habebitis terram per fines suos in circuitu.


${}^{13}$~Pr\ae cepitque Moyses filiis Isra\"el, dicens~: H\ae c erit terra, quam possidebitis sorte, et quam jussit Dominus dari novem tribubus, et dimidi\ae\ tribui.
${}^{14}$~Tribus enim filiorum Ruben per familias suas, et tribus filiorum Gad juxta cognationum numerum, media quoque tribus Manasse,
${}^{15}$~id est, du\ae\ semis tribus, acceperunt partem suam trans Jordanem contra Jericho ad orientalem plagam.
${}^{16}$~Et ait Dominus ad Moysen~:
${}^{17}$~H\ae c sunt nomina virorum qui terram vobis divident, Eleazar sacerdos, et Josue filius Nun,
${}^{18}$~et singuli principes de tribubus singulis,
${}^{19}$~quorum ista sunt vocabula. De tribu Juda, Caleb filius Jephone.
${}^{20}$~De tribu Simeon, Samuel filius Ammiud.
${}^{21}$~De tribu Benjamin, Elidad filius Chaselon.
${}^{22}$~De tribu filiorum Dan, Bocci filius Jogli.
${}^{23}$~Filiorum Joseph de tribu Manasse, Hanniel filius Ephod.
${}^{24}$~De tribu Ephraim, Camuel filius Sephthan.
${}^{25}$~De tribu Zabulon, Elisaphan filius Pharnach.
${}^{26}$~De tribu Issachar, dux Phaltiel filius Ozan.
${}^{27}$~De tribu Aser, Ahiud filius Salomi.
${}^{28}$~De tribu Nephthali, Pheda\"el filius Ammiud.
${}^{29}$~Hi sunt, quibus pr\ae cepit Dominus ut dividerent filiis Isra\"el terram Chanaan.

\bchapter{35}
\lettrine[lines=10,image=true,loversize=0.05,lraise=-0.03]{H}{}\ae c quoque locutus est Dominus ad Moysen in campestribus Moab supra Jordanem, contra Jericho~:
${}^{2}$~Pr\ae cipe filiis Isra\"el ut dent Levitis de possessionibus suis
${}^{3}$~urbes ad habitandum, et suburbana earum per circuitum~: ut ipsi in oppidis maneant, et suburbana sint pecoribus ac jumentis~:
${}^{4}$~qu\ae\ a muris civitatum forinsecus, per circuitum, mille passuum spatio tendentur.
${}^{5}$~Contra orientem duo millia erunt cubiti, et contra meridiem similiter erunt duo millia~: ad mare quoque, quod respicit ad occidentem, eadem mensura erit, et septentrionalis plaga \ae quali termino finietur, eruntque urbes in medio, et foris suburbana.
${}^{6}$~De ipsis autem oppidis, qu\ae\ Levitis dabitis, sex erunt in fugitivorum auxilia separata, ut fugiat ad ea qui fuderit sanguinem~: et exceptis his, alia quadraginta duo oppida,
${}^{7}$~id est, simul quadraginta octo cum suburbanis suis.
${}^{8}$~Ips\ae que urbes, qu\ae\ dabuntur de possessionibus filiorum Isra\"el, ab his qui plus habent, plures auferentur~: et qui minus, pauciores~: singuli juxta mensuram h\ae reditatis su\ae\ dabunt oppida Levitis.


${}^{9}$~Ait Dominus ad Moysen~:
${}^{10}$~Loquere filiis Isra\"el, et dices ad eos~: Quando transgressi fueritis Jordanem in terram Chanaan,
${}^{11}$~decernite qu\ae\ urbes esse debeant in pr\ae sidia fugitivorum, qui nolentes sanguinem fuderint~:
${}^{12}$~in quibus cum fuerit profugus, cognatus occisi non poterit eum occidere, donec stet in conspectu multitudinis, et causa illius judicetur.
${}^{13}$~De ipsis autem urbibus, qu\ae\ ad fugitivorum subsidia separantur,
${}^{14}$~tres erunt trans Jordanem, et tres in terra Chanaan,
${}^{15}$~tam filiis Isra\"el quam advenis atque peregrinis, ut confugiat ad eas qui nolens sanguinem fuderit.
${}^{16}$~Si quis ferro percusserit, et mortuus fuerit qui percussus est, reus erit homicidii, et ipse morietur.
${}^{17}$~Si lapidem jecerit, et ictus occubuerit, similiter punietur.
${}^{18}$~Si ligno percussus interierit, percussoris sanguine vindicabitur.
${}^{19}$~Propinquus occisi, homicidam interficiet~: statim ut apprehenderit eum, interficiet.
${}^{20}$~Si per odium quis hominem impulerit, vel jecerit quippiam in eum per insidias~:
${}^{21}$~aut cum esset inimicus, manu percusserit, et ille mortuus fuerit~: percussor homicidii reus erit~: cognatus occisi statim ut invenerit eum, jugulabit.
${}^{22}$~Quod si fortuitu, et absque odio
${}^{23}$~et inimicitiis quidquam horum fecerit,
${}^{24}$~et hoc audiente populo fuerit comprobatum, atque inter percussorem et propinquum sanguinis qu\ae stio ventilata~:
${}^{25}$~liberabitur innocens de ultoris manu, et reducetur per sententiam in urbem, ad quam confugerat, manebitque ibi, donec sacerdos magnus, qui oleo sancto unctus est, moriatur.
${}^{26}$~Si interfector extra fines urbium, qu\ae\ exulibus deputat\ae\ sunt,
${}^{27}$~fuerit inventus, et percussus ab eo qui ultor est sanguinis~: absque noxa erit qui eum occiderit.
${}^{28}$~Debuerat enim profugus usque ad mortem pontificis in urbe residere. Postquam autem ille obierit, homicida revertetur in terram suam.
${}^{29}$~H\ae c sempiterna erunt, et legitima in cunctis habitationibus vestris.


${}^{30}$~Homicida sub testibus punietur~: ad unius testimonium nullus condemnabitur.
${}^{31}$~Non accipietis pretium ab eo qui reus est sanguinis, statim et ipse morietur.
${}^{32}$~Exules et profugi ante mortem pontificis nullo modo in urbes suas reverti poterunt,
${}^{33}$~ne polluatis terram habitationis vestr\ae , qu\ae\ insontium cruore maculatur~: nec aliter expiari potest, nisi per ejus sanguinem, qui alterius sanguinem fuderit.
${}^{34}$~Atque ita emundabitur vestra possessio me commorante vobiscum. Ego enim sum Dominus qui habito inter filios Isra\"el.

\bchapter{36}
\lettrine[lines=10,image=true,loversize=0.05,lraise=-0.03]{A}{}ccesserunt autem et principes familiarum Galaad filii Machir filii Manasse, de stirpe filiorum Joseph~: locutique sunt Moysi coram principibus Isra\"el, atque dixerunt~:
${}^{2}$~Tibi domino nostro pr\ae cepit Dominus ut terram sorte divideres filiis Isra\"el, et ut filiabus Salphaad fratris nostri dares possessionem debitam patri~:
${}^{3}$~quas si alterius tribus homines uxores acceperint, sequetur possessio sua, et translata ad aliam tribum, de nostra h\ae reditate minuetur.
${}^{4}$~Atque ita fiet, ut cum jubil\ae us, id est, quinquagesimus annus remissionis advenerit, confundatur sortium distributio, et aliorum possessio ad alios transeat.
${}^{5}$~Respondit Moyses filiis Isra\"el, et Domino pr\ae cipiente ait~: Recte tribus filiorum Joseph locuta est.
${}^{6}$~Et h\ae c lex super filiabus Salphaad a Domino promulgata est~: nubant quibus volunt, tantum ut su\ae\ tribus hominibus~:
${}^{7}$~ne commisceatur possessio filiorum Isra\"el de tribu in tribum. Omnes enim viri ducent uxores de tribu et cognatione sua~:
${}^{8}$~et cunct\ae\ femin\ae\ de eadem tribu maritos accipient~: ut h\ae reditas permaneat in familiis,
${}^{9}$~nec sibi misceantur tribus, sed ita maneant
${}^{10}$~ut a Domino separat\ae\ sunt. Feceruntque fili\ae\ Salphaad ut fuerat imperatum~:
${}^{11}$~et nupserunt Maala, et Thersa, et Hegla, et Melcha, et Noa, filiis patrui sui
${}^{12}$~de familia Manasse, qui fuit filius Joseph~: et possessio, qu\ae\ illis fuerat attributa, mansit in tribu et familia patris earum.
${}^{13}$~H\ae c sunt mandata atque judicia, qu\ae\ mandavit Dominus per manum Moysi ad filios Isra\"el, in campestribus Moab supra Jordanem contra Jericho.
