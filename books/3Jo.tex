\newpage
\addcontentsline{toc}{subsection}{Joannis III}
{Joannis III}{images/genese_heading}

\noindent Senior Gajo carissimo, quem ego diligo in veritate.
${}^{2}$~Carissime, de omnibus orationem facio prospere te ingredi, et valere sicut prospere agit anima tua.


${}^{3}$~Gavisus sum valde venientibus fratribus, et testimonium perhibentibus veritati tu\ae , sicut tu in veritate ambulas.
${}^{4}$~Majorem horum non habeo gratiam, quam ut audiam filios meos in veritate ambulare.
${}^{5}$~Carissime, fideliter facis quidquid operaris in fratres, et hoc in peregrinos,
${}^{6}$~qui testimonium reddiderunt caritati tu\ae\ in conspectu ecclesi\ae~: quos, benefaciens, deduces digne Deo.
${}^{7}$~Pro nomine enim ejus profecti sunt, nihil accipientes a gentibus.
${}^{8}$~Nos ergo debemus suscipere hujusmodi, ut cooperatores simus veritatis.


${}^{9}$~Scripsissem forsitan ecclesi\ae~: sed is qui amat primatum gerere in eis, Diotrephes, non recipit nos~:
${}^{10}$~propter hoc si venero, commonebo ejus opera, qu\ae\ facit, verbis malignis garriens in nos~: et quasi non ei ista sufficiant, neque ipse suscipit fratres~: et eos qui suscipiunt, prohibet, et de ecclesia ejicit.
${}^{11}$~Carissime, noli imitari malum, sed quod bonum est. Qui benefacit, ex Deo est~: qui malefacit, non vidit Deum.


${}^{12}$~Demetrio testimonium redditur ab omnibus, et ab ipsa veritate, sed et nos testimonium perhibemus~: et nosti quoniam testimonium nostrum verum est.


${}^{13}$~Multa habui tibi scribere~: sed nolui per atramentum et calamum scribere tibi.
${}^{14}$~Spero autem protinus te videre, et os ad os loquemur. Pax tibi. Salutant te amici. Saluta amicos nominatim.
