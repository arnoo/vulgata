{\centering \section*{Evangelium secundum Lucam}}\thispagestyle{empty}
\addcontentsline{toc}{subsection}{Lucas}
\fancyhead[C]{\textsc{Lucas}}

\Needspace{2.5\baselineskip}\versal{1}~\lettrine[lines=10,image=true,loversize=0.05,lraise=-0.03]{Q}{}uoniam quidem multi conati sunt ordinare narrationem, qu\ae\ in nobis complet\ae\ sunt, rerum~:
${}^{2}$~sicut tradiderunt nobis, qui ab initio ipsi viderunt, et ministri fuerunt sermonis~:
${}^{3}$~visum est et mihi, assecuto omnia a principio diligenter, ex ordine tibi scribere, optime Theophile,
${}^{4}$~ut cognoscas eorum verborum, de quibus eruditus es, veritatem.


${}^{5}$~Fuit in diebus Herodis, regis Jud\ae \ae , sacerdos quidam nomine Zacharias de vice Abia, et uxor illius de filiabus Aaron, et nomen ejus Elisabeth.
${}^{6}$~Erant autem justi ambo ante Deum, incedentes in omnibus mandatis et justificationibus Domini sine querela.
${}^{7}$~Et non erat illis filius, eo quod esset Elisabeth sterilis, et ambo processissent in diebus suis.
${}^{8}$~Factum est autem, cum sacerdotio fungeretur in ordine vicis su\ae\ ante Deum,
${}^{9}$~secundum consuetudinem sacerdotii, sorte exiit ut incensum poneret, ingressus in templum Domini~:
${}^{10}$~et omnis multitudo populi erat orans foris hora incensi.
${}^{11}$~Apparuit autem illi angelus Domini, stans a dextris altaris incensi.
${}^{12}$~Et Zacharias turbatus est videns, et timor irruit super eum.
${}^{13}$~Ait autem ad illum angelus~: Ne timeas, Zacharia, quoniam exaudita est deprecatio tua~: et uxor tua Elisabeth pariet tibi filium, et vocabis nomen ejus Joannem~:
${}^{14}$~et erit gaudium tibi, et exsultatio, et multi in nativitate ejus gaudebunt~:
${}^{15}$~erit enim magnus coram Domino~: et vinum et siceram non bibet, et Spiritu Sancto replebitur adhuc ex utero matris su\ae~:
${}^{16}$~et multos filiorum Isra\"el convertet ad Dominum Deum ipsorum~:
${}^{17}$~et ipse pr\ae cedet ante illum in spiritu et virtute Eli\ae~: ut convertat corda patrum in filios, et incredulos ad prudentiam justorum, parare Domino plebem perfectam.
${}^{18}$~Et dixit Zacharias ad angelum~: Unde hoc sciam~? ego enim sum senex, et uxor mea processit in diebus suis.
${}^{19}$~Et respondens angelus dixit ei~: Ego sum Gabriel, qui asto ante Deum~: et missus sum loqui ad te, et h\ae c tibi evangelizare.
${}^{20}$~Et ecce eris tacens, et non poteris loqui usque in diem quo h\ae c fiant, pro eo quod non credidisti verbis meis, qu\ae\ implebuntur in tempore suo.
${}^{21}$~Et erat plebs exspectans Zachariam~: et mirabantur quod tardaret ipse in templo.
${}^{22}$~Egressus autem non poterat loqui ad illos, et cognoverunt quod visionem vidisset in templo. Et ipse erat innuens illis, et permansit mutus.
${}^{23}$~Et factum est, ut impleti sunt dies officii ejus, abiit in domum suam~:
${}^{24}$~post hos autem dies concepit Elisabeth uxor ejus, et occultabat se mensibus quinque, dicens~:
${}^{25}$~Quia sic fecit mihi Dominus in diebus, quibus respexit auferre opprobrium meum inter homines.


${}^{26}$~In mense autem sexto, missus est angelus Gabriel a Deo in civitatem Galil\ae \ae , cui nomen Nazareth,
${}^{27}$~ad virginem desponsatam viro, cui nomen erat Joseph, de domo David~: et nomen virginis Maria.
${}^{28}$~Et ingressus angelus ad eam dixit~: Ave gratia plena~: Dominus tecum~: benedicta tu in mulieribus.
${}^{29}$~Qu\ae\ cum audisset, turbata est in sermone ejus, et cogitabat qualis esset ista salutatio.
${}^{30}$~Et ait angelus ei~: Ne timeas, Maria~: invenisti enim gratiam apud Deum.
${}^{31}$~Ecce concipies in utero, et paries filium, et vocabis nomen ejus Jesum~:
${}^{32}$~hic erit magnus, et Filius Altissimi vocabitur, et dabit illi Dominus Deus sedem David patris ejus~: et regnabit in domo Jacob in \ae ternum,
${}^{33}$~et regni ejus non erit finis.
${}^{34}$~Dixit autem Maria ad angelum~: Quomodo fiet istud, quoniam virum non cognosco~?
${}^{35}$~Et respondens angelus dixit ei~: Spiritus Sanctus superveniet in te, et virtus Altissimi obumbrabit tibi. Ideoque et quod nascetur ex te sanctum, vocabitur Filius Dei.
${}^{36}$~Et ecce Elisabeth cognata tua, et ipsa concepit filium in senectute sua~: et hic mensis sextus est illi, qu\ae\ vocatur sterilis~:
${}^{37}$~quia non erit impossibile apud Deum omne verbum.
${}^{38}$~Dixit autem Maria~: Ecce ancilla Domini~: fiat mihi secundum verbum tuum. Et discessit ab illa angelus.


${}^{39}$~Exsurgens autem Maria in diebus illis, abiit in montana cum festinatione, in civitatem Juda~:
${}^{40}$~et intravit in domum Zachari\ae , et salutavit Elisabeth.
${}^{41}$~Et factum est, ut audivit salutationem Mari\ae\ Elisabeth, exsultavit infans in utero ejus~: et repleta est Spiritu Sancto Elisabeth~:
${}^{42}$~et exclamavit voce magna, et dixit~: Benedicta tu inter mulieres, et benedictus fructus ventris tui.
${}^{43}$~Et unde hoc mihi, ut veniat mater Domini mei ad me~?
${}^{44}$~Ecce enim ut facta est vox salutationis tu\ae\ in auribus meis, exsultavit in gaudio infans in utero meo.
${}^{45}$~Et beata, qu\ae\ credidisti, quoniam perficientur ea, qu\ae\ dicta sunt tibi a Domino.
${}^{46}$~Et ait Maria~: \begin{flushleft}\begin{verse}Magnificat anima mea Dominum~:\\
${}^{47}$~et exsultavit spiritus meus in Deo salutari meo.\\
${}^{48}$~Quia respexit humilitatem ancill\ae\ su\ae~:\\ ecce enim ex hoc beatam me dicent omnes generationes,\\
${}^{49}$~quia fecit mihi magna qui potens est~:\\ et sanctum nomen ejus,\\
${}^{50}$~et misericordia ejus a progenie in progenies\\ timentibus eum.\\
${}^{51}$~Fecit potentiam in brachio suo~:\\ dispersit superbos mente cordis sui.\\
${}^{52}$~Deposuit potentes de sede,\\ et exaltavit humiles.\\
${}^{53}$~Esurientes implevit bonis~:\\ et divites dimisit inanes.\\
${}^{54}$~Suscepit Isra\"el puerum suum,\\ recordatus misericordi\ae\ su\ae~:\\
${}^{55}$~sicut locutus est ad patres nostros,\\ Abraham et semini ejus in s\ae cula.\end{verse}\end{flushleft}


${}^{56}$~Mansit autem Maria cum illa quasi mensibus tribus~: et reversa est in domum suam.


${}^{57}$~Elisabeth autem impletum est tempus pariendi, et peperit filium.
${}^{58}$~Et audierunt vicini et cognati ejus quia magnificavit Dominus misericordiam suam cum illa, et congratulabantur ei.
${}^{59}$~Et factum est in die octavo, venerunt circumcidere puerum, et vocabant eum nomine patris sui Zachariam.
${}^{60}$~Et respondens mater ejus, dixit~: Nequaquam, sed vocabitur Joannes.
${}^{61}$~Et dixerunt ad illam~: Quia nemo est in cognatione tua, qui vocetur hoc nomine.
${}^{62}$~Innuebant autem patri ejus, quem vellet vocari eum.
${}^{63}$~Et postulans pugillarem scripsit, dicens~: Joannes est nomen ejus. Et mirati sunt universi.
${}^{64}$~Apertum est autem illico os ejus, et lingua ejus, et loquebatur benedicens Deum.
${}^{65}$~Et factus est timor super omnes vicinos eorum~: et super omnia montana Jud\ae \ae\ divulgabantur omnia verba h\ae c~:
${}^{66}$~et posuerunt omnes qui audierant in corde suo, dicentes~: Quis, putas, puer iste erit~? etenim manus Domini erat cum illo.
${}^{67}$~Et Zacharias pater ejus repletus est Spiritu Sancto~: et prophetavit, dicens~:
\begin{flushleft}\begin{verse}${}^{68}$~Benedictus Dominus Deus Isra\"el,\\ quia visitavit, et fecit redemptionem plebis su\ae~:\\
${}^{69}$~et erexit cornu salutis nobis\\ in domo David pueri sui,\\
${}^{70}$~sicut locutus est per os sanctorum,\\ qui a s\ae culo sunt, prophetarum ejus~:\\
${}^{71}$~salutem ex inimicis nostris,\\ et de manu omnium qui oderunt nos~:\\
${}^{72}$~ad faciendam misericordiam cum patribus nostris~:\\ et memorari testamenti sui sancti~:\\
${}^{73}$~jusjurandum, quod juravit ad Abraham patrem nostrum,\\ daturum se nobis\\
${}^{74}$~ut sine timore, de manu inimicorum nostrorum liberati,\\ serviamus illi\\
${}^{75}$~in sanctitate et justitia coram ipso,\\ omnibus diebus nostris.\\
${}^{76}$~Et tu puer, propheta Altissimi vocaberis~:\\ pr\ae ibis enim ante faciem Domini parare vias ejus,\\
${}^{77}$~ad dandam scientiam salutis plebi ejus\\ in remissionem peccatorum eorum\\
${}^{78}$~per viscera misericordi\ae\ Dei nostri,\\ in quibus visitavit nos, oriens ex alto~:\\
${}^{79}$~illuminare his qui in tenebris et in umbra mortis sedent~:\\ ad dirigendos pedes nostros in viam pacis.\end{verse}\end{flushleft}


${}^{80}$~Puer autem crescebat, et confortabatur spiritu~: et erat in desertis usque in diem ostensionis su\ae\ ad Isra\"el.
\Needspace{2.5\baselineskip}\versal{2}~\lettrine[lines=10,image=true,loversize=0.05,lraise=-0.03]{F}{}actum est autem in diebus illis, exiit edictum a C\ae sare Augusto ut describeretur universus orbis.
${}^{2}$~H\ae c descriptio prima facta est a pr\ae side Syri\ae\ Cyrino~:
${}^{3}$~et ibant omnes ut profiterentur singuli in suam civitatem.
${}^{4}$~Ascendit autem et Joseph a Galil\ae a de civitate Nazareth in Jud\ae am, in civitatem David, qu\ae\ vocatur Bethlehem~: eo quod esset de domo et familia David,
${}^{5}$~ut profiteretur cum Maria desponsata sibi uxore pr\ae gnante.
${}^{6}$~Factum est autem, cum essent ibi, impleti sunt dies ut pareret.
${}^{7}$~Et peperit filium suum primogenitum, et pannis eum involvit, et reclinavit eum in pr\ae sepio~: quia non erat eis locus in diversorio.
${}^{8}$~Et pastores erant in regione eadem vigilantes, et custodientes vigilias noctis super gregem suum.
${}^{9}$~Et ecce angelus Domini stetit juxta illos, et claritas Dei circumfulsit illos, et timuerunt timore magno.
${}^{10}$~Et dixit illis angelus~: Nolite timere~: ecce enim evangelizo vobis gaudium magnum, quod erit omni populo~:
${}^{11}$~quia natus est vobis hodie Salvator, qui est Christus Dominus, in civitate David.
${}^{12}$~Et hoc vobis signum~: invenietis infantem pannis involutum, et positum in pr\ae sepio.
${}^{13}$~Et subito facta est cum angelo multitudo militi\ae\ c\ae lestis laudantium Deum, et dicentium~:
\begin{flushleft}\begin{verse}${}^{14}$~Gloria in altissimis Deo,\\ et in terra pax hominibus bon\ae\ voluntatis.\end{verse}\end{flushleft}


${}^{15}$~Et factum est, ut discesserunt ab eis angeli in c\ae lum~: pastores loquebantur ad invicem~: Transeamus usque Bethlehem, et videamus hoc verbum, quod factum est, quod Dominus ostendit nobis.
${}^{16}$~Et venerunt festinantes~: et invenerunt Mariam, et Joseph, et infantem positum in pr\ae sepio.
${}^{17}$~Videntes autem cognoverunt de verbo, quod dictum erat illis de puero hoc.
${}^{18}$~Et omnes qui audierunt, mirati sunt~: et de his qu\ae\ dicta erant a pastoribus ad ipsos.
${}^{19}$~Maria autem conservabat omnia verba h\ae c, conferens in corde suo.
${}^{20}$~Et reversi sunt pastores glorificantes et laudantes Deum in omnibus qu\ae\ audierant et viderant, sicut dictum est ad illos.


${}^{21}$~Et postquam consummati sunt dies octo, ut circumcideretur puer, vocatum est nomen ejus Jesus, quod vocatum est ab angelo priusquam in utero conciperetur.
${}^{22}$~Et postquam impleti sunt dies purgationis ejus secundum legem Moysi, tulerunt illum in Jerusalem, ut sisterent eum Domino,
${}^{23}$~sicut scriptum est in lege Domini~: Quia omne masculinum adaperiens vulvam, sanctum Domino vocabitur~:
${}^{24}$~et ut darent hostiam secundum quod dictum est in lege Domini, par turturum, aut duos pullos columbarum.
${}^{25}$~Et ecce homo erat in Jerusalem, cui nomen Simeon, et homo iste justus, et timoratus, exspectans consolationem Isra\"el~: et Spiritus Sanctus erat in eo.
${}^{26}$~Et responsum acceperat a Spiritu Sancto, non visurum se mortem, nisi prius videret Christum Domini.
${}^{27}$~Et venit in spiritu in templum. Et cum inducerent puerum Jesum parentes ejus, ut facerent secundum consuetudinem legis pro eo,
${}^{28}$~et ipse accepit eum in ulnas suas~: et benedixit Deum, et dixit~:
\begin{flushleft}\begin{verse}${}^{29}$~Nunc dimittis servum tuum Domine,\\ secundum verbum tuum in pace~:\\
${}^{30}$~quia viderunt oculi mei salutare tuum,\\
${}^{31}$~quod parasti ante faciem omnium populorum~:\\
${}^{32}$~lumen ad revelationem gentium,\\ et gloriam plebis tu\ae\ Isra\"el.\end{verse}\end{flushleft}


${}^{33}$~Et erat pater ejus et mater mirantes super his qu\ae\ dicebantur de illo.
${}^{34}$~Et benedixit illis Simeon, et dixit ad Mariam matrem ejus~: Ecce positus est hic in ruinam et in resurrectionem multorum in Isra\"el, et in signum cui contradicetur~:
${}^{35}$~et tuam ipsius animam pertransibit gladius ut revelentur ex multis cordibus cogitationes.
${}^{36}$~Et erat Anna prophetissa, filia Phanuel, de tribu Aser~: h\ae c processerat in diebus multis, et vixerat cum viro suo annis septem a virginitate sua.
${}^{37}$~Et h\ae c vidua usque ad annos octoginta quatuor~: qu\ae\ non discedebat de templo, jejuniis et obsecrationibus serviens nocte ac die.
${}^{38}$~Et h\ae c, ipsa hora superveniens, confitebatur Domino~: et loquebatur de illo omnibus, qui exspectabant redemptionem Isra\"el.
${}^{39}$~Et ut perfecerunt omnia secundum legem Domini, reversi sunt in Galil\ae am in civitatem suam Nazareth.


${}^{40}$~Puer autem crescebat, et confortabatur plenus sapientia~: et gratia Dei erat in illo.
${}^{41}$~Et ibant parentes ejus per omnes annos in Jerusalem, in die solemni Pasch\ae .
${}^{42}$~Et cum factus esset annorum duodecim, ascendentibus illis Jerosolymam secundum consuetudinem diei festi,
${}^{43}$~consummatisque diebus, cum redirent, remansit puer Jesus in Jerusalem, et non cognoverunt parentes ejus.
${}^{44}$~Existimantes autem illum esse in comitatu, venerunt iter diei, et requirebant eum inter cognatos et notos.
${}^{45}$~Et non invenientes, regressi sunt in Jerusalem, requirentes eum.
${}^{46}$~Et factum est, post triduum invenerunt illum in templo sedentem in medio doctorum, audientem illos, et interrogantem eos.
${}^{47}$~Stupebant autem omnes qui eum audiebant, super prudentia et responsis ejus.
${}^{48}$~Et videntes admirati sunt. Et dixit mater ejus ad illum~: Fili, quid fecisti nobis sic~? ecce pater tuus et ego dolentes qu\ae rebamus te.
${}^{49}$~Et ait ad illos~: Quid est quod me qu\ae rebatis~? nesciebatis quia in his qu\ae\ Patris mei sunt, oportet me esse~?
${}^{50}$~Et ipsi non intellexerunt verbum quod locutus est ad eos.
${}^{51}$~Et descendit cum eis, et venit Nazareth~: et erat subditus illis. Et mater ejus conservabat omnia verba h\ae c in corde suo.
${}^{52}$~Et Jesus proficiebat sapientia, et \ae tate, et gratia apud Deum et homines.
\Needspace{2.5\baselineskip}\versal{3}~\lettrine[lines=10,image=true,loversize=0.05,lraise=-0.03]{A}{}nno autem quintodecimo imperii Tiberii C\ae saris, procurante Pontio Pilato Jud\ae am, tetrarcha autem Galil\ae \ae\ Herode, Philippo autem fratre ejus tetrarcha Itur\ae \ae , et Trachonitidis regionis, et Lysania Abilin\ae\ tetrarcha,
${}^{2}$~sub principibus sacerdotum Anna et Caipha~: factum est verbum Domini super Joannem, Zachari\ae\ filium, in deserto.
${}^{3}$~Et venit in omnem regionem Jordanis, pr\ae dicans baptismum pœnitenti\ae\ in remissionem peccatorum,
${}^{4}$~sicut scriptum est in libro sermonum Isai\ae\ prophet\ae~: \begin{flushleft}\begin{verse}Vox clamantis in deserto~:\\ Parate viam Domini~; rectas facite semitas ejus~:\\
${}^{5}$~omnis vallis implebitur,\\ et omnis mons, et collis humiliabitur~:\\ et erunt prava in directa, et aspera in vias planas~:\\
${}^{6}$~et videbit omnis caro salutare Dei.\end{verse}\end{flushleft}


${}^{7}$~Dicebat ergo ad turbas qu\ae\ exibant ut baptizarentur ab ipso~: Genimina viperarum, quis ostendit vobis fugere a ventura ira~?
${}^{8}$~Facite ergo fructus dignos pœnitenti\ae , et ne cœperitis dicere~: Patrem habemus Abraham. Dico enim vobis quia potens est Deus de lapidibus istis suscitare filios Abrah\ae .
${}^{9}$~Jam enim securis ad radicem arborum posita est. Omnis ergo arbor non faciens fructum bonum, excidetur, et in ignem mittetur.
${}^{10}$~Et interrogabant eum turb\ae , dicentes~: Quid ergo faciemus~?
${}^{11}$~Respondens autem dicebat illis~: Qui habet duas tunicas, det non habenti~: et qui habet escas, similiter faciat.
${}^{12}$~Venerunt autem et publicani ut baptizarentur, et dixerunt ad illum~: Magister, quid faciemus~?
${}^{13}$~At ille dixit ad eos~: Nihil amplius, quam quod constitutum est vobis, faciatis.
${}^{14}$~Interrogabant autem eum et milites, dicentes~: Quid faciemus et nos~? Et ait illis~: Neminem concutiatis, neque calumniam faciatis~: et contenti estote stipendiis vestris.
${}^{15}$~Existimante autem populo, et cogitantibus omnibus in cordibus suis de Joanne, ne forte ipse esset Christus,
${}^{16}$~respondit Joannes, dicens omnibus~: Ego quidem aqua baptizo vos~: veniet autem fortior me, cujus non sum dignus solvere corrigiam calceamentorum ejus~: ipse vos baptizabit in Spiritu Sancto et igni~:
${}^{17}$~cujus ventilabrum in manu ejus, et purgabit aream suam, et congregabit triticum in horreum suum, paleas autem comburet igni inextinguibili.
${}^{18}$~Multa quidem et alia exhortans evangelizabat populo.


${}^{19}$~Herodes autem tetrarcha cum corriperetur ab illo de Herodiade uxore fratris sui, et de omnibus malis qu\ae\ fecit Herodes,
${}^{20}$~adjecit et hoc super omnia, et inclusit Joannem in carcere.


${}^{21}$~Factum est autem cum baptizaretur omnis populus, et Jesu baptizato, et orante, apertum est c\ae lum~:
${}^{22}$~et descendit Spiritus Sanctus corporali specie sicut columba in ipsum~: et vox de c\ae lo facta est~: Tu es filius meus dilectus, in te complacui mihi.
${}^{23}$~Et ipse Jesus erat incipiens quasi annorum triginta, ut putabatur, filius Joseph, qui fuit Heli, qui fuit Mathat,
${}^{24}$~qui fuit Levi, qui fuit Melchi, qui fuit Janne, qui fuit Joseph,
${}^{25}$~qui fuit Mathathi\ae , qui fuit Amos, qui fuit Nahum, qui fuit Hesli, qui fuit Nagge,
${}^{26}$~qui fuit Mahath, qui fuit Mathathi\ae , qui fuit Semei, qui fuit Joseph, qui fuit Juda,
${}^{27}$~qui fuit Joanna, qui fuit Resa, qui fuit Zorobabel, qui fuit Salathiel, qui fuit Neri,
${}^{28}$~qui fuit Melchi, qui fuit Addi, qui fuit Cosan, qui fuit Elmadan, qui fuit Her,
${}^{29}$~qui fuit Jesu, qui fuit Eliezer, qui fuit Jorim, qui fuit Mathat, qui fuit Levi,
${}^{30}$~qui fuit Simeon, qui fuit Juda, qui fuit Joseph, qui fuit Jona, qui fuit Eliakim,
${}^{31}$~qui fuit Melea, qui fuit Menna, qui fuit Mathatha, qui fuit Natham, qui fuit David,
${}^{32}$~qui fuit Jesse, qui fuit Obed, qui fuit Booz, qui fuit Salmon, qui fuit Naasson,
${}^{33}$~qui fuit Aminadab, qui fuit Aram, qui fuit Esron, qui fuit Phares, qui fuit Jud\ae ,
${}^{34}$~qui fuit Jacob, qui fuit Isaac, qui fuit Abrah\ae , qui fuit Thare, qui fuit Nachor,
${}^{35}$~qui fuit Sarug, qui fuit Ragau, qui fuit Phaleg, qui fuit Heber, qui fuit Sale,
${}^{36}$~qui fuit Cainan, qui fuit Arphaxad, qui fuit Sem, qui fuit No\"e, qui fuit Lamech,
${}^{37}$~qui fuit Methusale, qui fuit Henoch, qui fuit Jared, qui fuit Malaleel, qui fuit Cainan,
${}^{38}$~qui fuit Henos, qui fuit Seth, qui fuit Adam, qui fuit Dei.
\Needspace{2.5\baselineskip}\versal{4}~\lettrine[lines=10,image=true,loversize=0.05,lraise=-0.03]{J}{}esus autem plenus Spiritu Sancto regressus est a Jordane~: et agebatur a Spiritu in desertum
${}^{2}$~diebus quadraginta, et tentabatur a diabolo. Et nihil manducavit in diebus illis~: et consummatis illis esuriit.
${}^{3}$~Dixit autem illi diabolus~: Si Filius Dei es, dic lapidi huic ut panis fiat.
${}^{4}$~Et respondit ad illum Jesus~: Scriptum est~: Quia non in solo pane vivit homo, sed in omni verbo Dei.
${}^{5}$~Et duxit illum diabolus in montem excelsum, et ostendit illi omnia regna orbis terr\ae\ in momento temporis,
${}^{6}$~et ait illi~: Tibi dabo potestatem hanc universam, et gloriam illorum~: quia mihi tradita sunt, et cui volo do illa.
${}^{7}$~Tu ergo si adoraveris coram me, erunt tua omnia.
${}^{8}$~Et respondens Jesus, dixit illi~: Scriptum est~: Dominum Deum tuum adorabis, et illi soli servies.
${}^{9}$~Et duxit illum in Jerusalem, et statuit eum super pinnam templi, et dixit illi~: Si Filius Dei es, mitte te hinc deorsum.
${}^{10}$~Scriptum est enim quod angelis suis mandavit de te, ut conservent te~:
${}^{11}$~et quia in manibus tollent te, ne forte offendas ad lapidem pedem tuum.
${}^{12}$~Et respondens Jesus, ait illi~: Dictum est~: Non tentabis Dominum Deum tuum.
${}^{13}$~Et consummata omni tentatione, diabolus recessit ab illo, usque ad tempus.


${}^{14}$~Et regressus est Jesus in virtute Spiritus in Galil\ae am, et fama exiit per universam regionem de illo.
${}^{15}$~Et ipse docebat in synagogis eorum, et magnificabatur ab omnibus.
${}^{16}$~Et venit Nazareth, ubi erat nutritus, et intravit secundum consuetudinem suam die sabbati in synagogam, et surrexit legere.
${}^{17}$~Et traditus est illi liber Isai\ae\ prophet\ae . Et ut revolvit librum, invenit locum ubi scriptum erat~:
${}^{18}$~Spiritus Domini super me~: propter quod unxit me, evangelizare pauperibus misit me, sanare contritos corde,
${}^{19}$~pr\ae dicare captivis remissionem, et c\ae cis visum, dimittere confractos in remissionem, pr\ae dicare annum Domini acceptum et diem retributionis.
${}^{20}$~Et cum plicuisset librum, reddit ministro, et sedit. Et omnium in synagoga oculi erant intendentes in eum.
${}^{21}$~Cœpit autem dicere ad illos~: Quia hodie impleta est h\ae c scriptura in auribus vestris.
${}^{22}$~Et omnes testimonium illi dabant~: et mirabantur in verbis grati\ae , qu\ae\ procedebant de ore ipsius, et dicebant~: Nonne hic est filius Joseph~?
${}^{23}$~Et ait illis~: Utique dicetis mihi hanc similitudinem~: Medice cura teipsum~: quanta audivimus facta in Capharnaum, fac et hic in patria tua.
${}^{24}$~Ait autem~: Amen dico vobis, quia nemo propheta acceptus est in patria sua.
${}^{25}$~In veritate dico vobis, mult\ae\ vidu\ae\ erant in diebus Eli\ae\ in Isra\"el, quando clausum est c\ae lum annis tribus et mensibus sex, cum facta esset fames magna in omni terra~:
${}^{26}$~et ad nullam illarum missus est Elias, nisi in Sarepta Sidoni\ae , ad mulierem viduam.
${}^{27}$~Et multi leprosi erant in Isra\"el sub Eliseo propheta~: et nemo eorum mundatus est nisi Naaman Syrus.
${}^{28}$~Et repleti sunt omnes in synagoga ira, h\ae c audientes.
${}^{29}$~Et surrexerunt, et ejecerunt illum extra civitatem~: et duxerunt illum usque ad supercilium montis, super quem civitas illorum erat \ae dificata, ut pr\ae cipitarent eum.
${}^{30}$~Ipse autem transiens per medium illorum, ibat.


${}^{31}$~Et descendit in Capharnaum civitatem Galil\ae \ae , ibique docebat illos sabbatis.
${}^{32}$~Et stupebant in doctrina ejus, quia in potestate erat sermo ipsius.
${}^{33}$~Et in synagoga erat homo habens d\ae monium immundum, et exclamavit voce magna,
${}^{34}$~dicens~: Sine, quid nobis et tibi, Jesu Nazarene~? venisti perdere nos~? scio te quis sis, Sanctus Dei.
${}^{35}$~Et increpavit illum Jesus, dicens~: Obmutesce, et exi ab eo. Et cum projecisset illum d\ae monium in medium, exiit ab illo, nihilque illum nocuit.
${}^{36}$~Et factus est pavor in omnibus, et colloquebantur ad invicem, dicentes~: Quod est hoc verbum, quia in potestate et virtute imperat immundis spiritibus, et exeunt~?
${}^{37}$~Et divulgabatur fama de illo in omnem locum regionis.


${}^{38}$~Surgens autem Jesus de synagoga, introivit in domum Simonis. Socrus autem Simonis tenebatur magnis febribus~: et rogaverunt illum pro ea.
${}^{39}$~Et stans super illam imperavit febri~: et dimisit illam. Et continuo surgens, ministrabat illis.
${}^{40}$~Cum autem sol occidisset, omnes qui habebant infirmos variis languoribus, ducebant illos ad eum. At ille singulis manus imponens, curabat eos.
${}^{41}$~Exibant autem d\ae monia a multis clamantia, et dicentia~: Quia tu es Filius Dei~: et increpans non sinebat ea loqui~: quia sciebant ipsum esse Christum.


${}^{42}$~Facta autem die egressus ibat in desertum locum, et turb\ae\ requirebant eum, et venerunt usque ad ipsum~: et detinebant illum ne discederet ab eis.
${}^{43}$~Quibus ille ait~: Quia et aliis civitatibus oportet me evangelizare regnum Dei~: quia ideo missus sum.
${}^{44}$~Et erat pr\ae dicans in synagogis Galil\ae \ae .
\Needspace{2.5\baselineskip}\versal{5}~\lettrine[lines=10,image=true,loversize=0.05,lraise=-0.03]{F}{}actum est autem, cum turb\ae\ irruerunt in eum ut audirent verbum Dei, et ipse stabat secus stagnum Genesareth.
${}^{2}$~Et vidit duas naves stantes secus stagnum~: piscatores autem descenderant, et lavabant retia.
${}^{3}$~Ascendens autem in unam navim, qu\ae\ erat Simonis, rogavit eum a terra reducere pusillum. Et sedens docebat de navicula turbas.


${}^{4}$~Ut cessavit autem loqui, dixit ad Simonem~: Duc in altum, et laxate retia vestra in capturam.
${}^{5}$~Et respondens Simon, dixit illi~: Pr\ae ceptor, per totam noctem laborantes nihil cepimus~: in verbo autem tuo laxabo rete.
${}^{6}$~Et cum hoc fecissent, concluserunt piscium multitudinem copiosam~: rumpebatur autem rete eorum.
${}^{7}$~Et annuerunt sociis, qui erant in alia navi, ut venirent, et adjuvarent eos. Et venerunt, et impleverunt ambas naviculas, ita ut pene mergerentur.
${}^{8}$~Quod cum videret Simon Petrus, procidit ad genua Jesu, dicens~: Exi a me, quia homo peccator sum, Domine.
${}^{9}$~Stupor enim circumdederat eum, et omnes qui cum illo erant, in captura piscium, quam ceperant~:
${}^{10}$~similiter autem Jacobum et Joannem, filios Zebed\ae i, qui erant socii Simonis. Et ait ad Simonem Jesus~: Noli timere~: ex hoc jam homines eris capiens.
${}^{11}$~Et subductis ad terram navibus, relictis omnibus, secuti sunt eum.


${}^{12}$~Et factum est, cum esset in una civitatum, et ecce vir plenus lepra, et videns Jesum, et procidens in faciem, rogavit eum, dicens~: Domine, si vis, potes me mundare.
${}^{13}$~Et extendens manum, tetigit eum dicens~: Volo~: mundare. Et confestim lepra discessit ab illo.
${}^{14}$~Et ipse pr\ae cepit illi ut nemini diceret~: sed, Vade, ostende te sacerdoti, et offer pro emundatione tua, sicut pr\ae cepit Moyses, in testimonium illis.
${}^{15}$~Perambulabat autem magis sermo de illo~: et conveniebant turb\ae\ mult\ae\ ut audirent, et curarentur ab infirmitatibus suis.
${}^{16}$~Ipse autem secedebat in desertum, et orabat.


${}^{17}$~Et factum est in una dierum, et ipse sedebat docens. Et erant pharis\ae i sedentes, et legis doctores, qui venerant ex omni castello Galil\ae \ae , et Jud\ae \ae , et Jerusalem~: et virtus Domini erat ad sanandum eos.
${}^{18}$~Et ecce viri portantes in lecto hominem, qui erat paralyticus~: et qu\ae rebant eum inferre, et ponere ante eum.
${}^{19}$~Et non invenientes qua parte illum inferrent pr\ae\ turba, ascenderunt supra tectum, et per tegulas summiserunt eum cum lecto in medium ante Jesum.
${}^{20}$~Quorum fidem ut vidit, dixit~: Homo, remittuntur tibi peccata tua.
${}^{21}$~Et cœperunt cogitare scrib\ae\ et pharis\ae i, dicentes~: Quis est hic, qui loquitur blasphemias~? quis potest dimittere peccata, nisi solus Deus~?
${}^{22}$~Ut cognovit autem Jesus cogitationes eorum, respondens, dixit ad illos~: Quid cogitatis in cordibus vestris~?
${}^{23}$~Quid est facilius dicere~: Dimittuntur tibi peccata~: an dicere~: Surge, et ambula~?
${}^{24}$~Ut autem sciatis quia Filius hominis habet potestatem in terra dimittendi peccata, (ait paralytico) tibi dico, surge, tolle lectum tuum, et vade in domum tuam.
${}^{25}$~Et confestim consurgens coram illis, tulit lectum in quo jacebat~: et abiit in domum suam, magnificans Deum.
${}^{26}$~Et stupor apprehendit omnes, et magnificabant Deum. Et repleti sunt timore, dicentes~: Quia vidimus mirabilia hodie.


${}^{27}$~Et post h\ae c exiit, et vidit publicanum nomine Levi, sedentem ad telonium, et ait illi~: Sequere me.
${}^{28}$~Et relictis omnibus, surgens secutus est eum.
${}^{29}$~Et fecit ei convivium magnum Levi in domo sua~: et erat turba multa publicanorum, et aliorum qui cum illis erant discumbentes.
${}^{30}$~Et murmurabant pharis\ae i et scrib\ae\ eorum, dicentes ad discipulos ejus~: Quare cum publicanis et peccatoribus manducatis et bibitis~?
${}^{31}$~Et respondens Jesus, dixit ad illos~: Non egent qui sani sunt medico, sed qui male habent.
${}^{32}$~Non veni vocare justos, sed peccatores ad pœnitentiam.


${}^{33}$~At illi dixerunt ad eum~: Quare discipuli Joannis jejunant frequenter, et obsecrationes faciunt, similiter et pharis\ae orum~: tui autem edunt et bibunt~?
${}^{34}$~Quibus ipse ait~: Numquid potestis filios sponsi, dum cum illis est sponsus, facere jejunare~?
${}^{35}$~Venient autem dies, cum ablatus fuerit ab illis sponsus~: tunc jejunabunt in illis diebus.
${}^{36}$~Dicebat autem et similitudinem ad illos~: Quia nemo commissuram a novo vestimento immittit in vestimentum vetus~: alioquin et novum rumpit, et veteri non convenit commissura a novo.
${}^{37}$~Et nemo mittit vinum novum in utres veteres~: alioquin rumpet vinum novum utres, et ipsum effundetur, et utres peribunt~:
${}^{38}$~sed vinum novum in utres novos mittendum est, et utraque conservantur.
${}^{39}$~Et nemo bibens vetus, statim vult novum~: dicit enim~: Vetus melius est.
\Needspace{2.5\baselineskip}\versal{6}~\lettrine[lines=10,image=true,loversize=0.05,lraise=-0.03]{F}{}actum est autem in sabbato secundo, primo, cum transiret per sata, vellebant discipuli ejus spicas, et manducabant confricantes manibus.
${}^{2}$~Quidam autem pharis\ae orum, dicebant illis~: Quid facitis quod non licet in sabbatis~?
${}^{3}$~Et respondens Jesus ad eos, dixit~: Nec hoc legistis quod fecit David, cum esurisset ipse, et qui cum illo erant~?
${}^{4}$~quomodo intravit in domum Dei, et panes propositionis sumpsit, et manducavit, et dedit his qui cum ipso erant~: quos non licet manducare nisi tantum sacerdotibus~?
${}^{5}$~Et dicebat illis~: Quia dominus est Filius hominis etiam sabbati.


${}^{6}$~Factum est autem in alio sabbato, ut intraret in synagogam, et doceret. Et erat ibi homo, et manus ejus dextra erat arida.
${}^{7}$~Observabant autem scrib\ae\ et pharis\ae i si in sabbato curaret, ut invenirent unde accusarent eum.
${}^{8}$~Ipse vero sciebat cogitationes eorum~: et ait homini qui habebat manum aridam~: Surge, et sta in medium. Et surgens stetit.
${}^{9}$~Ait autem ad illos Jesus~: Interrogo vos si licet sabbatis benefacere, an male~: animam salvam facere, an perdere~?
${}^{10}$~Et circumspectis omnibus dixit homini~: Extende manum tuam. Et extendit~: et restituta est manus ejus.
${}^{11}$~Ipsi autem repleti sunt insipientia, et colloquebantur ad invicem, quidnam facerent Jesu.


${}^{12}$~Factum est autem in illis diebus, exiit in montem orare, et erat pernoctans in oratione Dei.
${}^{13}$~Et cum dies factus esset, vocavit discipulos suos~: et elegit duodecim ex ipsis (quos et apostolos nominavit)~:
${}^{14}$~Simonem, quem cognominavit Petrum, et Andream fratrem ejus, Jacobum, et Joannem, Philippum, et Bartholom\ae um,
${}^{15}$~Matth\ae um, et Thomam, Jacobum Alph\ae i, et Simonem, qui vocatur Zelotes,
${}^{16}$~et Judam Jacobi, et Judam Iscariotem, qui fuit proditor.
${}^{17}$~Et descendens cum illis, stetit in loco campestri, et turba discipulorum ejus, et multitudo copiosa plebis ab omni Jud\ae a, et Jerusalem, et maritima, et Tyri, et Sidonis,
${}^{18}$~qui venerant ut audirent eum, et sanarentur a languoribus suis. Et qui vexabantur a spiritibus immundis, curabantur.
${}^{19}$~Et omnis turba qu\ae rebat eum tangere~: quia virtus de illo exibat, et sanabat omnes.


${}^{20}$~Et ipse elevatis oculis in discipulis suis, dicebat~: Beati pauperes, quia vestrum est regnum Dei.
${}^{21}$~Beati qui nunc esuritis, quia saturabimini. Beati qui nunc fletis, quia ridebitis.
${}^{22}$~Beati eritis cum vos oderint homines, et cum separaverint vos, et exprobraverint, et ejicerint nomen vestrum tamquam malum propter Filium hominis.
${}^{23}$~Gaudete in illa die, et exsultate~: ecce enim merces vestra multa est in c\ae lo~: secundum h\ae c enim faciebant prophetis patres eorum.
${}^{24}$~Verumtamen v\ae\ vobis divitibus, quia habetis consolationem vestram.
${}^{25}$~V\ae\ vobis, qui saturati estis~: quia esurietis. V\ae\ vobis, qui ridetis nunc~: quia lugebitis et flebitis.
${}^{26}$~V\ae\ cum benedixerint vobis homines~: secundum h\ae c enim faciebant pseudoprophetis patres eorum.


${}^{27}$~Sed vobis dico, qui auditis~: diligite inimicos vestros, benefacite his qui oderunt vos.
${}^{28}$~Benedicite maledicentibus vobis, et orate pro calumniantibus vos.
${}^{29}$~Et qui te percutit in maxillam, pr\ae be et alteram. Et ab eo qui aufert tibi vestimentum, etiam tunicam noli prohibere.
${}^{30}$~Omni autem petenti te, tribue~: et qui aufert qu\ae\ tua sunt, ne repetas.
${}^{31}$~Et prout vultis ut faciant vobis homines, et vos facite illis similiter.
${}^{32}$~Et si diligitis eos qui vos diligunt, qu\ae\ vobis est gratia~? nam et peccatores diligentes se diligunt.
${}^{33}$~Et si benefeceritis his qui vobis benefaciunt, qu\ae\ vobis est gratia~? siquidem et peccatores hoc faciunt.
${}^{34}$~Et si mutuum dederitis his a quibus speratis recipere, qu\ae\ gratia est vobis~? nam et peccatores peccatoribus fœnerantur, ut recipiant \ae qualia.
${}^{35}$~Verumtamen diligite inimicos vestros~: benefacite, et mutuum date, nihil inde sperantes~: et erit merces vestra multa, et eritis filii Altissimi, quia ipse benignus est super ingratos et malos.
${}^{36}$~Estote ergo misericordes sicut et Pater vester misericors est.


${}^{37}$~Nolite judicare, et non judicabimini~: nolite condemnare, et non condemnabimini. Dimittite, et dimittemini.
${}^{38}$~Date, et dabitur vobis~: mensuram bonam, et confertam, et coagitatam, et supereffluentem dabunt in sinum vestrum. Eadem quippe mensura, qua mensi fueritis, remetietur vobis.
${}^{39}$~Dicebat autem illis et similitudinem~: Numquid potest c\ae cus c\ae cum ducere~? nonne ambo in foveam cadunt~?
${}^{40}$~Non est discipulus super magistrum~: perfectus autem omnis erit, si sit sicut magister ejus.
${}^{41}$~Quid autem vides festucam in oculo fratris tui, trabem autem, qu\ae\ in oculo tuo est, non consideras~?
${}^{42}$~aut quomodo potes dicere fratri tuo~: Frater, sine ejiciam festucam de oculo tuo~: ipse in oculo tuo trabem non videns~? Hypocrita, ejice primum trabem de oculo tuo~: et tunc perspicies ut educas festucam de oculo fratris tui.
${}^{43}$~Non est enim arbor bona, qu\ae\ facit fructus malos~: neque arbor mala, faciens fructum bonum.
${}^{44}$~Unaqu\ae que enim arbor de fructu suo cognoscitur. Neque enim de spinis colligunt ficus~: neque de rubo vindemiant uvam.
${}^{45}$~Bonus homo de bono thesauro cordis sui profert bonum~: et malus homo de malo thesauro profert malum. Ex abundantia enim cordis os loquitur.
${}^{46}$~Quid autem vocatis me Domine, Domine~: et non facitis qu\ae\ dico~?


${}^{47}$~Omnis qui venit ad me, et audit sermones meos, et facit eos, ostendam vobis cui similis sit~:
${}^{48}$~similis est homini \ae dificanti domum, qui fodit in altum, et posuit fundamentum super petram~: inundatione autem facta, illisum est flumen domui illi, et non potuit eam movere~: fundata enim erat super petram.
${}^{49}$~Qui autem audit, et non facit, similis est homini \ae dificanti domum suam super terram sine fundamento~: in quam illisus est fluvius, et continuo cecidit~: et facta est ruina domus illius magna.
\Needspace{2.5\baselineskip}\versal{7}~\lettrine[lines=10,image=true,loversize=0.05,lraise=-0.03]{C}{}um autem implesset omnia verba sua in aures plebis, intravit Capharnaum.
${}^{2}$~Centurionis autem cujusdam servus male habens, erat moriturus~: qui illi erat pretiosus.
${}^{3}$~Et cum audisset de Jesu, misit ad eum seniores Jud\ae orum, rogans eum ut veniret et salvaret servum ejus.
${}^{4}$~At illi cum venissent ad Jesum, rogabant eum sollicite, dicentes ei~: Quia dignus est ut hoc illi pr\ae stes~:
${}^{5}$~diligit enim gentem nostram, et synagogam ipse \ae dificavit nobis.
${}^{6}$~Jesus autem ibat cum illis. Et cum jam non longe esset a domo, misit ad eum centurio amicos, dicens~: Domine, noli vexari~: non enim sum dignus ut sub tectum meum intres~:
${}^{7}$~propter quod et meipsum non sum dignum arbitratus ut venirem ad te~: sed dic verbo, et sanabitur puer meus.
${}^{8}$~Nam et ego homo sum sub potestate constitutus, habens sub me milites~: et dico huic, Vade, et vadit~: et alii, Veni, et venit~: et servo meo, Fac hoc, et facit.
${}^{9}$~Quo audito Jesus miratus est~: et conversus sequentibus se turbis, dixit~: Amen dico vobis, nec in Isra\"el tantam fidem inveni.
${}^{10}$~Et reversi, qui missi fuerant, domum, invenerunt servum, qui languerat, sanum.


${}^{11}$~Et factum est~: deinceps ibat in civitatem qu\ae\ vocatur Naim~: et ibant cum eo discipuli ejus et turba copiosa.
${}^{12}$~Cum autem appropinquaret port\ae\ civitatis, ecce defunctus efferebatur filius unicus matris su\ae~: et h\ae c vidua erat~: et turba civitatis multa cum illa.
${}^{13}$~Quam cum vidisset Dominus, misericordia motus super eam, dixit illi~: Noli flere.
${}^{14}$~Et accessit, et tetigit loculum. (Hi autem qui portabant, steterunt.) Et ait~: Adolescens, tibi dico, surge.
${}^{15}$~Et resedit qui erat mortuus, et cœpit loqui. Et dedit illum matri su\ae .
${}^{16}$~Accepit autem omnes timor~: et magnificabant Deum, dicentes~: Quia propheta magnus surrexit in nobis~: et quia Deus visitavit plebem suam.
${}^{17}$~Et exiit hic sermo in universam Jud\ae am de eo, et in omnem circa regionem.


${}^{18}$~Et nuntiaverunt Joanni discipuli ejus de omnibus his.
${}^{19}$~Et convocavit duos de discipulis suis Joannes, et misit ad Jesum, dicens~: Tu es qui venturus es, an alium exspectamus~?
${}^{20}$~Cum autem venissent ad eum viri, dixerunt~: Joannes Baptista misit nos ad te dicens~: Tu es qui venturus es, an alium exspectamus~?
${}^{21}$~(In ipsa autem hora multos curavit a languoribus, et plagis, et spiritibus malis, et c\ae cis multis donavit visum.)
${}^{22}$~Et respondens, dixit illis~: Euntes renuntiate Joanni qu\ae\ audistis et vidistis~: quia c\ae ci vident, claudi ambulant, leprosi mundantur, surdi audiunt, mortui resurgunt, pauperes evangelizantur~:
${}^{23}$~et beatus est quicumque non fuerit scandalizatus in me.


${}^{24}$~Et cum discessissent nuntii Joannis, cœpit de Joanne dicere ad turbas~: Quid existis in desertum videre~? arundinem vento agitatam~?
${}^{25}$~Sed quid existis videre~? hominem mollibus vestibus indutum~? Ecce qui in veste pretiosa sunt et deliciis, in domibus regum sunt.
${}^{26}$~Sed quid existis videre~? prophetam~? Utique dico vobis, et plus quam prophetam~:
${}^{27}$~hic est, de quo scriptum est~: Ecce mitto angelum meum ante faciem tuam, qui pr\ae parabit viam tuam ante te.
${}^{28}$~Dico enim vobis~: major inter natos mulierum propheta Joanne Baptista nemo est~: qui autem minor est in regno Dei, major est illo.


${}^{29}$~Et omnis populus audiens et publicani, justificaverunt Deum, baptizati baptismo Joannis.
${}^{30}$~Pharis\ae i autem et legisperiti consilium Dei spreverunt in semetipsos, non baptizati ab eo.
${}^{31}$~Ait autem Dominus~: Cui ergo similes dicam homines generationis hujus~? et cui similes sunt~?
${}^{32}$~Similes sunt pueris sedentibus in foro, et loquentibus ad invicem, et dicentibus~: Cantavimus vobis tibiis, et non saltastis~: lamentavimus, et non plorastis.
${}^{33}$~Venit enim Joannes Baptista, neque manducans panem, neque bibens vinum, et dicitis~: D\ae monium habet.
${}^{34}$~Venit Filius hominis manducans, et bibens, et dicitis~: Ecce homo devorator, et bibens vinum, amicus publicanorum et peccatorum.
${}^{35}$~Et justificata est sapientia ab omnibus filiis suis.


${}^{36}$~Rogabat autem illum quidam de pharis\ae is ut manducaret cum illo. Et ingressus domum pharis\ae i discubuit.
${}^{37}$~Et ecce mulier, qu\ae\ erat in civitate peccatrix, ut cognovit quod accubuisset in domo pharis\ae i, attulit alabastrum unguenti~:
${}^{38}$~et stans retro secus pedes ejus, lacrimis cœpit rigare pedes ejus, et capillis capitis sui tergebat, et osculabatur pedes ejus, et unguento ungebat.
${}^{39}$~Videns autem pharis\ae us, qui vocaverat eum, ait intra se dicens~: Hic si esset propheta, sciret utique qu\ae\ et qualis est mulier, qu\ae\ tangit eum~: quia peccatrix est.
${}^{40}$~Et respondens Jesus, dixit ad illum~: Simon, habeo tibi aliquid dicere. At ille ait~: Magister, dic.
${}^{41}$~Duo debitores erant cuidam fœneratori~: unus debebat denarios quingentos, et alius quinquaginta.
${}^{42}$~Non habentibus illis unde redderent, donavit utrisque. Quis ergo eum plus diligit~?
${}^{43}$~Respondens Simon dixit~: \AE stimo quia is cui plus donavit. At ille dixit ei~: Recte judicasti.
${}^{44}$~Et conversus ad mulierem, dixit Simoni~: Vides hanc mulierem~? Intravi in domum tuam, aquam pedibus meis non dedisti~: h\ae c autem lacrimis rigavit pedes meos, et capillis suis tersit.
${}^{45}$~Osculum mihi non dedisti~: h\ae c autem ex quo intravit, non cessavit osculari pedes meos.
${}^{46}$~Oleo caput meum non unxisti~: h\ae c autem unguento unxit pedes meos.
${}^{47}$~Propter quod dico tibi~: remittuntur ei peccata multa, quoniam dilexit multum. Cui autem minus dimittitur, minus diligit.
${}^{48}$~Dixit autem ad illam~: Remittuntur tibi peccata.
${}^{49}$~Et cœperunt qui simul accumbebant, dicere intra se~: Quis est hic qui etiam peccata dimittit~?
${}^{50}$~Dixit autem ad mulierem~: Fides tua te salvam fecit~: vade in pace.
\Needspace{2.5\baselineskip}\versal{8}~\lettrine[lines=10,image=true,loversize=0.05,lraise=-0.03]{E}{}t factum est deinceps, et ipse iter faciebat per civitates, et castella pr\ae dicans, et evangelizans regnum Dei~: et duodecim cum illo,
${}^{2}$~et mulieres aliqu\ae , qu\ae\ erant curat\ae\ a spiritibus malignis et infirmantibus~: Maria, qu\ae\ vocatur Magdalene, de qua septem d\ae monia exierant,
${}^{3}$~et Joanna uxor Chus\ae\ procuratoris Herodis, et Susanna, et ali\ae\ mult\ae , qu\ae\ ministrabant ei de facultatibus suis.


${}^{4}$~Cum autem turba plurima convenirent, et de civitatibus properarent ad eum, dixit per similitudinem~:
${}^{5}$~Exiit qui seminat, seminare semen suum. Et dum seminat, aliud cecidit secus viam, et conculcatum est, et volucres c\ae li comederunt illud.
${}^{6}$~Et aliud cecidit supra petram~: et natum aruit, quia non habebat humorem.
${}^{7}$~Et aliud cecidit inter spinas, et simul exort\ae\ spin\ae\ suffocaverunt illud.
${}^{8}$~Et aliud cecidit in terram bonam~: et ortum fecit fructum centuplum. H\ae c dicens clamabat~: Qui habet aures audiendi, audiat.


${}^{9}$~Interrogabant autem eum discipuli ejus, qu\ae\ esset h\ae c parabola.
${}^{10}$~Quibus ipse dixit~: Vobis datum est nosse mysterium regni Dei, ceteris autem in parabolis~: ut videntes non videant, et audientes non intelligant.


${}^{11}$~Est autem h\ae c parabola~: Semen est verbum Dei.
${}^{12}$~Qui autem secus viam, hi sunt qui audiunt~: deinde venit diabolus, et tollit verbum de corde eorum, ne credentes salvi fiant.
${}^{13}$~Nam qui supra petram, qui cum audierint, cum gaudio suscipiunt verbum~: et hi radices non habent~: qui ad tempus credunt, et in tempore tentationis recedunt.
${}^{14}$~Quod autem in spinas cecidit~: hi sunt qui audierunt, et a sollicitudinibus, et divitiis, et voluptatibus vit\ae\ euntes, suffocantur, et non referunt fructum.
${}^{15}$~Quod autem in bonam terram~: hi sunt qui in corde bono et optimo audientes verbum retinent, et fructum afferunt in patientia.


${}^{16}$~Nemo autem lucernam accendens, operit eam vase, aut subtus lectum ponit~: sed supra candelabrum ponit, ut intrantes videant lumen.
${}^{17}$~Non est enim occultum, quod non manifestetur~: nec absconditum, quod non cognoscatur, et in palam veniat.
${}^{18}$~Videte ergo quomodo audiatis~? Qui enim habet, dabitur illi~: et quicumque non habet, etiam quod putat se habere, auferetur ab illo.


${}^{19}$~Venerunt autem ad illum mater et fratres ejus, et non poterant adire eum pr\ae\ turba.
${}^{20}$~Et nuntiatum est illi~: Mater tua et fratres tui stant foris, volentes te videre.
${}^{21}$~Qui respondens, dixit ad eos~: Mater mea et fratres mei hi sunt, qui verbum Dei audiunt et faciunt.


${}^{22}$~Factum est autem in una dierum~: et ipse ascendit in naviculam, et discipuli ejus, et ait ad illos~: Transfretemus trans stagnum. Et ascenderunt.
${}^{23}$~Et navigantibus illis, obdormivit, et descendit procella venti in stagnum, et complebantur, et periclitabantur.
${}^{24}$~Accedentes autem suscitaverunt eum, dicentes~: Pr\ae ceptor, perimus. At ille surgens, increpavit ventum, et tempestatem aqu\ae , et cessavit~: et facta est tranquillitas.
${}^{25}$~Dixit autem illis~: Ubi est fides vestra~? Qui timentes, mirati sunt ad invicem, dicentes~: Quis putas hic est, quia et ventis, et mari imperat, et obediunt ei~?


${}^{26}$~Et navigaverunt ad regionem Gerasenorum, qu\ae\ est contra Galil\ae am.
${}^{27}$~Et cum egressus esset ad terram, occurrit illi vir quidam, qui habebat d\ae monium jam temporibus multis, et vestimento non induebatur, neque in domo manebat, sed in monumentis.
${}^{28}$~Is, ut vidit Jesum, procidit ante illum~: et exclamans voce magna, dixit~: Quid mihi et tibi est, Jesu Fili Dei Altissimi~? obsecro te, ne me torqueas.
${}^{29}$~Pr\ae cipiebat enim spiritui immundo ut exiret ab homine. Multis enim temporibus arripiebat illum, et vinciebatur catenis, et compedibus custoditus. Et ruptis vinculis agebatur a d\ae monio in deserta.
${}^{30}$~Interrogavit autem illum Jesus, dicens~: Quod tibi nomen est~? At ille dixit~: Legio~: quia intraverant d\ae monia multa in eum.
${}^{31}$~Et rogabant illum ne imperaret illis ut in abyssum irent.
${}^{32}$~Erat autem ibi grex porcorum multorum pascentium in monte~: et rogabant eum, ut permitteret eis in illos ingredi. Et permisit illis.
${}^{33}$~Exierunt ergo d\ae monia ab homine, et intraverunt in porcos~: et impetu abiit grex per pr\ae ceps in stagnum, et suffocatus est.
${}^{34}$~Quod ut viderunt factum qui pascebant, fugerunt, et nuntiaverunt in civitatem et in villas.
${}^{35}$~Exierunt autem videre quod factum est, et venerunt ad Jesum, et invenerunt hominem sedentem, a quo d\ae monia exierant, vestitum ac sana mente, ad pedes ejus, et timuerunt.
${}^{36}$~Nuntiaverunt autem illis et qui viderant, quomodo sanus factus esset a legione~:
${}^{37}$~et rogaverunt illum omnis multitudo regionis Gerasenorum ut discederet ab ipsis~: quia magno timore tenebantur. Ipse autem ascendens navim, reversus est.
${}^{38}$~Et rogabat illum vir, a quo d\ae monia exierant, ut cum eo esset. Dimisit autem eum Jesus, dicens~:
${}^{39}$~Redi in domum tuam, et narra quanta tibi fecit Deus. Et abiit per universam civitatem, pr\ae dicans quanta illi fecisset Jesus.


${}^{40}$~Factum est autem cum rediisset Jesus, excepit illum turba~: erunt enim omnes exspectantes eum.
${}^{41}$~Et ecce venit vir, cui nomen Jairus, et ipse princeps synagog\ae\ erat~: et cecidit ad pedes Jesu, rogans eum ut intraret in domum ejus,
${}^{42}$~quia unica filia erat ei fere annorum duodecim, et h\ae c moriebatur. Et contigit, dum iret, a turba comprimebatur.
${}^{43}$~Et mulier qu\ae dam erat in fluxu sanguinis ab annis duodecim, qu\ae\ in medicos erogaverat omnem substantiam suam, nec ab ullo potuit curari~:
${}^{44}$~accessit retro, et tetigit fimbriam vestimenti ejus~: et confestim stetit fluxus sanguinis ejus.
${}^{45}$~Et ait Jesus~: Quis est, qui me tetigit~? Negantibus autem omnibus, dixit Petrus, et qui cum illo erant~: Pr\ae ceptor, turb\ae\ te comprimunt, et affligunt, et dicis~: Quis me tetigit~?
${}^{46}$~Et dicit Jesus~: Tetigit me aliquis~: nam ego novi virtutem de me exiisse.
${}^{47}$~Videns autem mulier, quia non latuit, tremens venit, et procidit ante pedes ejus~: et ob quam causam tetigerit eum, indicavit coram omni populo~: et quemadmodum confestim sanata sit.
${}^{48}$~At ipse dixit ei~: Filia, fides tua salvam te fecit~: vade in pace.
${}^{49}$~Adhuc illo loquente, venit quidam ad principem synagog\ae , dicens ei~: Quia mortua est filia tua, noli vexare illum.
${}^{50}$~Jesus autem, audito hoc verbo, respondit patri puell\ae~: Noli timere, crede tantum, et salva erit.
${}^{51}$~Et cum venisset domum, non permisit intrare secum quemquam, nisi Petrum, et Jacobum, et Joannem, et patrem, et matrem puell\ae .
${}^{52}$~Flebant autem omnes, et plangebant illam. At ille dixit~: Nolite flere~: non est mortua puella, sed dormit.
${}^{53}$~Et deridebant eum, scientes quod mortua esset.
${}^{54}$~Ipse autem tenens manum ejus clamavit, dicens~: Puella, surge.
${}^{55}$~Et reversus est spiritus ejus, et surrexit continuo. Et jussit illi dari manducare.
${}^{56}$~Et stupuerunt parentes ejus, quibus pr\ae cepit ne alicui dicerent quod factum erat.
\Needspace{2.5\baselineskip}\versal{9}~\lettrine[lines=10,image=true,loversize=0.05,lraise=-0.03]{C}{}onvocatis autem duodecim Apostolis, dedit illis virtutem et potestatem super omnia d\ae monia, et ut languores curarent.
${}^{2}$~Et misit illos pr\ae dicare regnum Dei, et sanare infirmos.
${}^{3}$~Et ait ad illos~: Nihil tuleritis in via, neque virgam, neque peram, neque panem, neque pecuniam, neque duas tunicas habeatis.
${}^{4}$~Et in quamcumque domum intraveritis, ibi manete, et inde ne exeatis.
${}^{5}$~Et quicumque non receperint vos~: exeuntes de civitate illa, etiam pulverem pedum vestrorum excutite in testimonium supra illos.
${}^{6}$~Egressi autem circuibant per castella evangelizantes, et curantes ubique.


${}^{7}$~Audivit autem Herodes tetrarcha omnia qu\ae\ fiebant ab eo, et h\ae sitabat eo quod diceretur
${}^{8}$~a quibusdam~: Quia Joannes surrexit a mortuis~: a quibusdam vero~: Quia Elias apparuit~: ab aliis autem~: Quia propheta unus de antiquis surrexit.
${}^{9}$~Et ait Herodes~: Joannem ego decollavi~: quis est autem iste, de quo ego talia audio~? Et qu\ae rebat videre eum.


${}^{10}$~Et reversi Apostoli, narraverunt illi qu\ae cumque fecerunt~: et assumptis illis secessit seorsum in locum desertum, qui est Bethsaid\ae .
${}^{11}$~Quod cum cognovissent turb\ae , secut\ae\ sunt illum~: et excepit eos, et loquebatur illis de regno Dei, et eos, qui cura indigebant, sanabat.
${}^{12}$~Dies autem cœperat declinare, et accedentes duodecim dixerunt illi~: Dimitte turbas, ut euntes in castella villasque qu\ae\ circa sunt, divertant, et inveniant escas~: quia hic in loco deserto sumus.
${}^{13}$~Ait autem ad illos~: Vos date illis manducare. At illi dixerunt~: Non sunt nobis plus quam quinque panes et duo pisces~: nisi forte nos eamus, et emamus in omnem hanc turbam escas.
${}^{14}$~Erant autem fere viri quinque millia. Ait autem ad discipulos suos~: Facite illos discumbere per convivia quinquagenos.
${}^{15}$~Et ita fecerunt~: et discumbere fecerunt omnes.
${}^{16}$~Acceptis autem quinque panibus et duobus piscibus, respexit in c\ae lum, et benedixit illis~: et fregit, et distribuit discipulis suis, ut ponerent ante turbas.
${}^{17}$~Et manducaverunt omnes, et saturati sunt. Et sublatum est quod superfuit illis, fragmentorum cophini duodecim.


${}^{18}$~Et factum est cum solus esset orans, erant cum illo et discipuli~: et interrogavit illos, dicens~: Quem me dicunt esse turb\ae~?
${}^{19}$~At illi responderunt, et dixerunt~: Joannem Baptistam, alii autem Eliam, alii vero quia unus propheta de prioribus surrexit.
${}^{20}$~Dixit autem illis~: Vos autem quem me esse dicitis~? Respondens Simon Petrus, dixit~: Christum Dei.
${}^{21}$~At ille increpans illos, pr\ae cepit ne cui dicerent hoc,
${}^{22}$~dicens~: Quia oportet Filium hominis multa pati, et reprobari a senioribus, et principibus sacerdotum, et scribis, et occidi, et tertia die resurgere.


${}^{23}$~Dicebat autem ad omnes~: Si quis vult post me venire, abneget semetipsum, et tollat crucem suam quotidie, et sequatur me.
${}^{24}$~Qui enim voluerit animam suam salvam facere, perdet illam~: nam qui perdiderit animam suam propter me, salvam faciet illam.
${}^{25}$~Quid enim proficit homo, si lucretur universum mundum, se autem ipsum perdat, et detrimentum sui faciat~?
${}^{26}$~Nam qui me erubuerit, et meos sermones~: hunc Filius hominis erubescet cum venerit in majestate sua, et Patris, et sanctorum angelorum.
${}^{27}$~Dico autem vobis vere~: sunt aliqui hic stantes, qui non gustabunt mortem donec videant regnum Dei.


${}^{28}$~Factum est autem post h\ae c verba fere dies octo, et assumpsit Petrum, et Jacobum, et Joannem, et ascendit in montem ut oraret.
${}^{29}$~Et facta est, dum oraret, species vultus ejus altera~: et vestitus ejus albus et refulgens.
${}^{30}$~Et ecce duo viri loquebantur cum illo. Erant autem Moyses et Elias,
${}^{31}$~visi in majestate~: et dicebant excessum ejus, quem completurus erat in Jerusalem.
${}^{32}$~Petrus vero, et qui cum illo erant, gravati erant somno. Et evigilantes viderunt majestatem ejus, et duos viros qui stabant cum illo.
${}^{33}$~Et factum est cum discederent ab illo, ait Petrus ad Jesum~: Pr\ae ceptor, bonum est nos hic esse~: et faciamus tria tabernacula, unum tibi, et unum Moysi, et unum Eli\ae~: nesciens quid diceret.
${}^{34}$~H\ae c autem illo loquente, facta est nubes, et obumbravit eos~: et timuerunt, intrantibus illis in nubem.
${}^{35}$~Et vox facta est de nube, dicens~: Hic est Filius meus dilectus, ipsum audite.
${}^{36}$~Et dum fieret vox, inventus est Jesus solus. Et ipsi tacuerunt, et nemini dixerunt in illis diebus quidquam ex his qu\ae\ viderant.


${}^{37}$~Factum est autem in sequenti die, descendentibus illis de monte, occurrit illis turba multa.
${}^{38}$~Et ecce vir de turba exclamavit, dicens~: Magister, obsecro te, respice in filium meum quia unicus est mihi~:
${}^{39}$~et ecce spiritus apprehendit eum, et subito clamat, et elidit, et dissipat eum cum spuma, et vix discedit dilanians eum~:
${}^{40}$~et rogavi discipulos tuos ut ejicerent illum, et non potuerunt.
${}^{41}$~Respondens autem Jesus, dixit~: O generatio infidelis, et perversa, usquequo ero apud vos, et patiar vos~? adduc huc filium tuum.
${}^{42}$~Et cum accederet, elisit illum d\ae monium, et dissipavit.
${}^{43}$~Et increpavit Jesus spiritum immundum, et sanavit puerum, et reddidit illum patri ejus.


${}^{44}$~Stupebant autem omnes in magnitudine Dei~: omnibusque mirantibus in omnibus qu\ae\ faciebat, dixit ad discipulos suos~: Ponite vos in cordibus vestris sermones istos~: Filius enim hominis futurum est ut tradatur in manus hominum.
${}^{45}$~At illi ignorabant verbum istud, et erat velatum ante eos ut non sentirent illud~: et timebant eum interrogare de hoc verbo.


${}^{46}$~Intravit autem cogitatio in eos quis eorum major esset.
${}^{47}$~At Jesus videns cogitationes cordis illorum, apprehendit puerum, et statuit illum secus se,
${}^{48}$~et ait illis~: Quicumque susceperit puerum istum in nomine meo, me recipit~: et quicumque me receperit, recipit eum qui me misit. Nam qui minor est inter vos omnes, hic major est.


${}^{49}$~Respondens autem Joannes dixit~: Pr\ae ceptor, vidimus quemdam in nomine tuo ejicientem d\ae monia, et prohibuimus eum~: quia non sequitur nobiscum.
${}^{50}$~Et ait ad illum Jesus~: Nolite prohibere~: qui enim non est adversum vos, pro vobis est.


${}^{51}$~Factum est autem dum complerentur dies assumptionis ejus, et ipse faciem suam firmavit ut iret in Jerusalem.
${}^{52}$~Et misit nuntios ante conspectum suum~: et euntes intraverunt in civitatem Samaritanorum ut parerent illi.
${}^{53}$~Et non receperunt eum, quia facies ejus erat euntis in Jerusalem.
${}^{54}$~Cum vidissent autem discipuli ejus Jacobus et Joannes, dixerunt~: Domine, vis dicimus ut ignis descendat de c\ae lo, et consumat illos~?
${}^{55}$~Et conversus increpavit illos, dicens~: Nescitis cujus spiritus estis.
${}^{56}$~Filius hominis non venit animas perdere, sed salvare. Et abierunt in aliud castellum.


${}^{57}$~Factum est autem~: ambulantibus illis in via, dixit quidam ad illum~: Sequar te quocumque ieris.
${}^{58}$~Dixit illi Jesus~: Vulpes foveas habent, et volucres c\ae li nidos~: Filius autem hominis non habet ubi caput reclinet.
${}^{59}$~Ait autem ad alterum~: Sequere me~: ille autem dixit~: Domine, permitte mihi primum ire, et sepelire patrem meum.
${}^{60}$~Dixitque ei Jesus~: Sine ut mortui sepeliant mortuos suos~: tu autem vade, et annuntia regnum Dei.
${}^{61}$~Et ait alter~: Sequar te Domine, sed permitte mihi primum renuntiare his qu\ae\ domi sunt.
${}^{62}$~Ait ad illum Jesus~: Nemo mittens manum suam ad aratrum, et respiciens retro, aptus est regno Dei.
\Needspace{2.5\baselineskip}\versal{10}~\lettrine[lines=10,image=true,loversize=0.05,lraise=-0.03]{P}{}ost h\ae c autem designavit Dominus et alios septuaginta duos~: et misit illos binos ante faciem suam in omnem civitatem et locum, quo erat ipse venturus.
${}^{2}$~Et dicebat illis~: Messis quidem multa, operarii autem pauci. Rogate ergo dominum messis ut mittat operarios in messem suam.
${}^{3}$~Ite~: ecce ego mitto vos sicut agnos inter lupos.
${}^{4}$~Nolite portare sacculum, neque peram, neque calceamenta, et neminem per viam salutaveritis.
${}^{5}$~In quamcumque domum intraveritis, primum dicite~: Pax huic domui~:
${}^{6}$~et si ibi fuerit filius pacis, requiescet super illum pax vestra~: sin autem, ad vos revertetur.
${}^{7}$~In eadem autem domo manete, edentes et bibentes qu\ae\ apud illos sunt~: dignus est enim operarius mercede sua. Nolite transire de domo in domum.
${}^{8}$~Et in quamcumque civitatem intraveritis, et susceperint vos, manducate qu\ae\ apponuntur vobis~:
${}^{9}$~et curate infirmos, qui in illa sunt, et dicite illis~: Appropinquavit in vos regnum Dei.
${}^{10}$~In quamcumque autem civitatem intraveritis, et non susceperint vos, exeuntes in plateas ejus, dicite~:
${}^{11}$~Etiam pulverem, qui adh\ae sit nobis de civitate vestra, extergimus in vos~: tamen hoc scitote, quia appropinquavit regnum Dei.
${}^{12}$~Dico vobis, quia Sodomis in die illa remissius erit, quam illi civitati.


${}^{13}$~V\ae\ tibi Corozain~! v\ae\ tibi Bethsaida~! quia si in Tyro et Sidone fact\ae\ fuissent virtutes qu\ae\ fact\ae\ sunt in vobis, olim in cilicio et cinere sedentes pœniterent.
${}^{14}$~Verumtamen Tyro et Sidoni remissius erit in judicio, quam vobis.
${}^{15}$~Et tu Capharnaum, usque ad c\ae lum exaltata, usque ad infernum demergeris.
${}^{16}$~Qui vos audit, me audit~: et qui vos spernit, me spernit. Qui autem me spernit, spernit eum qui misit me.


${}^{17}$~Reversi sunt autem septuaginta duo cum gaudio, dicentes~: Domine, etiam d\ae monia subjiciuntur nobis in nomine tuo.
${}^{18}$~Et ait illis~: Videbam Satanam sicut fulgor de c\ae lo cadentem.
${}^{19}$~Ecce dedi vobis potestatem calcandi supra serpentes, et scorpiones, et super omnem virtutem inimici~: et nihil vobis nocebit.
${}^{20}$~Verumtamen in hoc nolite gaudere quia spiritus vobis subjiciuntur~: gaudete autem, quod nomina vestra scripta sunt in c\ae lis.


${}^{21}$~In ipsa hora exsultavit Spiritu Sancto, et dixit~: Confiteor tibi Pater, Domine c\ae li et terr\ae , quod abscondisti h\ae c a sapientibus et prudentibus, et revelasti ea parvulis. Etiam Pater~: quoniam sic placuit ante te.
${}^{22}$~Omnia mihi tradita sunt a Patre meo. Et nemo scit quis sit Filius, nisi Pater~: et quis sit Pater, nisi Filius, et cui voluerit Filius revelare.
${}^{23}$~Et conversus ad discipulos suos, dixit~: Beati oculi qui vident qu\ae\ vos videtis.
${}^{24}$~Dico enim vobis quod multi prophet\ae\ et reges voluerunt videre qu\ae\ vos videtis, et non viderunt~: et audire qu\ae\ auditis, et non audierunt.


${}^{25}$~Et ecce quidam legisperitus surrexit tentans illum, et dicens~: Magister, quid faciendo vitam \ae ternam possidebo~?
${}^{26}$~At ille dixit ad eum~: In lege quid scriptum est~? quomodo legis~?
${}^{27}$~Ille respondens dixit~: Diliges Dominum Deum tuum ex toto corde tuo, et ex tota anima tua, et ex omnibus virtutibus tuis, et ex omni mente tua~: et proximum tuum sicut teipsum.
${}^{28}$~Dixitque illi~: Recte respondisti~: hoc fac, et vives.


${}^{29}$~Ille autem volens justificare seipsum, dixit ad Jesum~: Et quis est meus proximus~?
${}^{30}$~Suscipiens autem Jesus, dixit~: Homo quidam descendebat ab Jerusalem in Jericho, et incidit in latrones, qui etiam despoliaverunt eum~: et plagis impositis abierunt semivivo relicto.
${}^{31}$~Accidit autem ut sacerdos quidam descenderet eadem via~: et viso illo pr\ae terivit.
${}^{32}$~Similiter et Levita, cum esset secus locum, et videret eum, pertransiit.
${}^{33}$~Samaritanus autem quidam iter faciens, venit secus eum~: et videns eum, misericordia motus est.
${}^{34}$~Et appropians alligavit vulnera ejus, infundens oleum et vinum~: et imponens illum in jumentum suum, duxit in stabulum, et curam ejus egit.
${}^{35}$~Et altera die protulit duos denarios, et dedit stabulario, et ait~: Curam illius habe~: et quodcumque supererogaveris, ego cum rediero reddam tibi.
${}^{36}$~Quis horum trium videtur tibi proximus fuisse illi, qui incidit in latrones~?
${}^{37}$~At ille dixit~: Qui fecit misericordiam in illum. Et ait illi Jesus~: Vade, et tu fac similiter.


${}^{38}$~Factum est autem, dum irent, et ipse intravit in quoddam castellum~: et mulier qu\ae dam, Martha nomine, excepit illum in domum suam,
${}^{39}$~et huic erat soror nomine Maria, qu\ae\ etiam sedens secus pedes Domini, audiebat verbum illius.
${}^{40}$~Martha autem satagebat circa frequens ministerium~: qu\ae\ stetit, et ait~: Domine, non est tibi cur\ae\ quod soror mea reliquit me solam ministrare~? dic ergo illi ut me adjuvet.
${}^{41}$~Et respondens dixit illi Dominus~: Martha, Martha, sollicita es, et turbaris erga plurima,
${}^{42}$~porro unum est necessarium. Maria optimam partem elegit, qu\ae\ non auferetur ab ea.
\Needspace{2.5\baselineskip}\versal{11}~\lettrine[lines=10,image=true,loversize=0.05,lraise=-0.03]{E}{}t factum est~: cum esset in quodam loco orans, ut cessavit, dixit unus ex discipulis ejus ad eum~: Domine, doce nos orare, sicut docuit et Joannes discipulos suos.
${}^{2}$~Et ait illis~: Cum oratis, dicite~: Pater, sanctificetur nomen tuum. Adveniat regnum tuum.
${}^{3}$~Panem nostrum quotidianum da nobis hodie.
${}^{4}$~Et dimitte nobis peccata nostra, siquidem et ipsi dimittimus omni debenti nobis. Et ne nos inducas in tentationem.


${}^{5}$~Et ait ad illos~: Quis vestrum habebit amicum, et ibit ad illum media nocte, et dicet illi~: Amice, commoda mihi tres panes,
${}^{6}$~quoniam amicus meus venit de via ad me, et non habeo quod ponam ante illum,
${}^{7}$~et ille de intus respondens dicat~: Noli mihi molestus esse, jam ostium clausum est, et pueri mei mecum sunt in cubili~: non possum surgere, et dare tibi.
${}^{8}$~Et si ille perseveraverit pulsans~: dico vobis, etsi non dabit illi surgens eo quod amicus ejus sit, propter improbitatem tamen ejus surget, et dabit illi quotquot habet necessarios.
${}^{9}$~Et ego dico vobis~: Petite, et dabitur vobis~; qu\ae rite, et invenietis~; pulsate, et aperietur vobis.
${}^{10}$~Omnis enim qui petit, accipit~: et qui qu\ae rit, invenit~: et pulsanti aperietur.
${}^{11}$~Quis autem ex vobis patrem petit panem, numquid lapidem dabit illi~? aut piscem, numquid pro pisce serpentem dabit illi~?
${}^{12}$~aut si petierit ovum, numquid porriget illi scorpionem~?
${}^{13}$~Si ergo vos, cum sitis mali, nostis bona data dare filiis vestris~: quanto magis Pater vester de c\ae lo dabit spiritum bonum petentibus se~?


${}^{14}$~Et erat ejiciens d\ae monium, et illud erat mutum. Et cum ejecisset d\ae monium, locutus est mutus, et admirat\ae\ sunt turb\ae .
${}^{15}$~Quidam autem ex eis dixerunt~: In Beelzebub principe d\ae moniorum ejicit d\ae monia.
${}^{16}$~Et alii tentantes, signum de c\ae lo qu\ae rebant ab eo.
${}^{17}$~Ipse autem ut vidit cogitationes eorum, dixit eis~: Omne regnum in seipsum divisum desolabitur, et domus supra domum cadet.
${}^{18}$~Si autem et Satanas in seipsum divisus est, quomodo stabit regnum ejus~? quia dicitis in Beelzebub me ejicere d\ae monia.
${}^{19}$~Si autem ego in Beelzebub ejicio d\ae monia~: filii vestri in quo ejiciunt~? ideo ipsi judices vestri erunt.
${}^{20}$~Porro si in digito Dei ejicio d\ae monia~: profecto pervenit in vos regnum Dei.
${}^{21}$~Cum fortis armatus custodit atrium suum, in pace sunt ea qu\ae\ possidet.
${}^{22}$~Si autem fortior eo superveniens vicerit eum, universa arma ejus auferet, in quibus confidebat, et spolia ejus distribuet.
${}^{23}$~Qui non est mecum, contra me est~: et qui non colligit mecum, dispergit.
${}^{24}$~Cum immundus spiritus exierit de homine, ambulat per loca inaquosa, qu\ae rens requiem~: et non inveniens dicit~: Revertar in domum meam unde exivi.
${}^{25}$~Et cum venerit, invenit eam scopis mundatam, et ornatam.
${}^{26}$~Tunc vadit, et assumit septem alios spiritus secum, nequiores se, et ingressi habitant ibi. Et fiunt novissima hominis illius pejora prioribus.


${}^{27}$~Factum est autem, cum h\ae c diceret~: extollens vocem qu\ae dam mulier de turba dixit illi~: Beatus venter qui te portavit, et ubera qu\ae\ suxisti.
${}^{28}$~At ille dixit~: Quinimmo beati, qui audiunt verbum Dei et custodiunt illud.


${}^{29}$~Turbis autem concurrentibus cœpit dicere~: Generatio h\ae c, generatio nequam est~: signum qu\ae rit, et signum non dabitur ei, nisi signum Jon\ae\ prophet\ae .
${}^{30}$~Nam sicut fuit Jonas signum Ninivitis, ita erit et Filius hominis generationi isti.
${}^{31}$~Regina austri surget in judicio cum viris generationis hujus, et condemnabit illos~: quia venit a finibus terr\ae\ audire sapientiam Salomonis~: et ecce plus quam Salomon hic.
${}^{32}$~Viri Ninivit\ae\ surgent in judicio cum generatione hac, et condemnabunt illam~: quia pœnitentiam egerunt ad pr\ae dicationem Jon\ae , et ecce plus quam Jonas hic.


${}^{33}$~Nemo lucernam accendit, et in abscondito ponit, neque sub modio~: sed supra candelabrum, ut qui ingrediuntur, lumen videant.
${}^{34}$~Lucerna corporis tui est oculus tuus. Si oculus tuus fuerit simplex, totum corpus tuum lucidum erit~: si autem nequam fuerit, etiam corpus tuum tenebrosum erit.
${}^{35}$~Vide ergo ne lumen quod in te est, tenebr\ae\ sint.
${}^{36}$~Si ergo corpus tuum totum lucidum fuerit, non habens aliquam partem tenebrarum, erit lucidum totum, et sicut lucerna fulgoris illuminabit te.
${}^{37}$~Et cum loqueretur, rogavit illum quidam pharis\ae us ut pranderet apud se. Et ingressus recubuit.


${}^{38}$~Pharis\ae us autem cœpit intra se reputans dicere, quare non baptizatus esset ante prandium.
${}^{39}$~Et ait Dominus ad illum~: Nunc vos pharis\ae i, quod deforis est calicis et catini, mundatis~: quod autem intus est vestrum, plenum est rapina et iniquitate.
${}^{40}$~Stulti~! nonne qui fecit quod deforis est, etiam id quod deintus est fecit~?
${}^{41}$~Verumtamen quod superest, date eleemosynam~: et ecce omnia munda sunt vobis.
${}^{42}$~Sed v\ae\ vobis, pharis\ae is, quia decimatis mentham, et rutam, et omne olus, et pr\ae teritis judicium et caritatem Dei~: h\ae c autem oportuit facere, et illa non omittere.
${}^{43}$~V\ae\ vobis, pharis\ae is, quia diligitis primas cathedras in synagogis, et salutationes in foro.
${}^{44}$~V\ae\ vobis, quia estis ut monumenta, qu\ae\ non apparent, et homines ambulantes supra, nesciunt.
${}^{45}$~Respondens autem quidam ex legisperitis, ait illi~: Magister, h\ae c dicens etiam contumeliam nobis facis.
${}^{46}$~At ille ait~: Et vobis legisperitis v\ae~: quia oneratis homines oneribus, qu\ae\ portare non possunt, et ipsi uno digito vestro non tangitis sarcinas.
${}^{47}$~V\ae\ vobis, qui \ae dificatis monumenta prophetarum~: patres autem vestri occiderunt illos.
${}^{48}$~Profecto testificamini quod consentitis operibus patrum vestrorum~: quoniam ipsi quidem eos occiderunt, vos autem \ae dificatis eorum sepulchra.
${}^{49}$~Propterea et sapientia Dei dixit~: Mittam ad illos prophetas, et apostolos, et ex illis occident, et persequentur~:
${}^{50}$~ut inquiratur sanguis omnium prophetarum, qui effusus est a constitutione mundi a generatione ista,
${}^{51}$~a sanguine Abel, usque ad sanguinem Zachari\ae , qui periit inter altare et \ae dem. Ita dico vobis, requiretur ab hac generatione.
${}^{52}$~V\ae\ vobis, legisperitis, quia tulistis clavem scienti\ae~: ipsi non introistis, et eos qui introibant, prohibuistis.
${}^{53}$~Cum autem h\ae c ad illos diceret, cœperunt pharis\ae i et legisperiti graviter insistere, et os ejus opprimere de multis,
${}^{54}$~insidiantes ei, et qu\ae rentes aliquid capere de ore ejus, ut accusarent eum.
\Needspace{2.5\baselineskip}\versal{12}~\lettrine[lines=10,image=true,loversize=0.05,lraise=-0.03]{M}{}ultis autem turbis circumstantibus, ita ut se invicem conculcarent, cœpit dicere ad discipulos suos~: Attendite a fermento pharis\ae orum, quod est hypocrisis.
${}^{2}$~Nihil autem opertum est, quod non reveletur~: neque absconditum, quod non sciatur.
${}^{3}$~Quoniam qu\ae\ in tenebris dixistis, in lumine dicentur~: et quod in aurem locuti estis in cubiculis, pr\ae dicabitur in tectis.


${}^{4}$~Dico autem vobis amicis meis~: Ne terreamini ab his qui occidunt corpus, et post h\ae c non habent amplius quid faciant.
${}^{5}$~Ostendam autem vobis quem timeatis~: timete eum qui, postquam occiderit, habet potestatem mittere in gehennam~: ita dico vobis, hunc timete.
${}^{6}$~Nonne quinque passeres veneunt dipondio, et unus ex illis non est in oblivione coram Deo~?
${}^{7}$~sed et capilli capitis vestri omnes numerati sunt. Nolite ergo timere~: multis passeribus pluris estis vos.


${}^{8}$~Dico autem vobis~: Omnis quicumque confessus fuerit me coram hominibus, et Filius hominis confitebitur illum coram angelis Dei~:
${}^{9}$~qui autem negaverit me coram hominibus, negabitur coram angelis Dei.
${}^{10}$~Et omnis qui dicit verbum in Filium hominis, remittetur illi~: ei autem qui in Spiritum Sanctum blasphemaverit, non remittetur.
${}^{11}$~Cum autem inducent vos in synagogas, et ad magistratus, et potestates, nolite solliciti esse qualiter, aut quid respondeatis, aut quid dicatis.
${}^{12}$~Spiritus enim Sanctus docebit vos in ipsa hora quid oporteat vos dicere.
${}^{13}$~Ait autem ei quidam de turba~: Magister, dic fratri meo ut dividat mecum h\ae reditatem.
${}^{14}$~At ille dixit illi~: Homo, quis me constituit judicem, aut divisorem super vos~?


${}^{15}$~Dixitque ad illos~: Videte, et cavete ab omni avaritia~: quia non in abundantia cujusquam vita ejus est ex his qu\ae\ possidet.
${}^{16}$~Dixit autem similitudinem ad illos, dicens~: Hominis cujusdam divitis uberes fructus ager attulit~:
${}^{17}$~et cogitabat intra se dicens~: Quid faciam, quia non habeo quo congregem fructus meos~?
${}^{18}$~Et dixit~: Hoc faciam~: destruam horrea mea, et majora faciam~: et illuc congregabo omnia qu\ae\ nata sunt mihi, et bona mea,
${}^{19}$~et dicam anim\ae\ me\ae~: Anima, habes multa bona posita in annos plurimos~: requiesce, comede, bibe, epulare.
${}^{20}$~Dixit autem illi Deus~: Stulte, hac nocte animam tuam repetunt a te~: qu\ae\ autem parasti, cujus erunt~?
${}^{21}$~Sic est qui sibi thesaurizat, et non est in Deum dives.


${}^{22}$~Dixitque ad discipulos suos~: Ideo dico vobis, nolite solliciti esse anim\ae\ vestr\ae\ quid manducetis, neque corpori quid induamini.
${}^{23}$~Anima plus est quam esca, et corpus plus quam vestimentum.
${}^{24}$~Considerate corvos, quia non seminant, neque metunt, quibus non est cellarium, neque horreum, et Deus pascit illos. Quanto magis vos pluris estis illis~?
${}^{25}$~Quis autem vestrum cogitando potest adjicere ad staturam suam cubitum unum~?
${}^{26}$~Si ergo neque quod minimum est potestis, quid de ceteris solliciti estis~?
${}^{27}$~Considerate lilia quomodo crescunt~: non laborant, neque nent~: dico autem vobis, nec Salomon in omni gloria sua vestiebatur sicut unum ex istis.
${}^{28}$~Si autem fœnum, quod hodie est in agro, et cras in clibanum mittitur, Deus sic vestit~: quanto magis vos pusill\ae\ fidei~?
${}^{29}$~Et vos nolite qu\ae rere quid manducetis, aut quid bibatis~: et nolite in sublime tolli~:
${}^{30}$~h\ae c enim omnia gentes mundi qu\ae runt. Pater autem vester scit quoniam his indigetis.
${}^{31}$~Verumtamen qu\ae rite primum regnum Dei, et justitiam ejus~: et h\ae c omnia adjicientur vobis.
${}^{32}$~Nolite timere pusillus grex, quia complacuit Patri vestro dare vobis regnum.
${}^{33}$~Vendite qu\ae\ possidetis, et date eleemosynam. Facite vobis sacculos, qui non veterascunt, thesaurum non deficientem in c\ae lis~: quo fur non appropriat, neque tinea corrumpit.
${}^{34}$~Ubi enim thesaurus vester est, ibi et cor vestrum erit.


${}^{35}$~Sint lumbi vestri pr\ae cincti, et lucern\ae\ ardentes in manibus vestris,
${}^{36}$~et vos similes hominibus exspectantibus dominum suum quando revertatur a nuptiis~: ut, cum venerit et pulsaverit, confestim aperiant ei.
${}^{37}$~Beati servi illi quos, cum venerit dominus, invenerit vigilantes~: amen dico vobis, quod pr\ae cinget se, et faciet illos discumbere, et transiens ministrabit illis.
${}^{38}$~Et si venerit in secunda vigilia, et si in tertia vigilia venerit, et ita invenerit, beati sunt servi illi.
${}^{39}$~Hoc autem scitote, quoniam si sciret paterfamilias, qua hora fur veniret, vigilaret utique, et non sineret perfodi domum suam.
${}^{40}$~Et vos estote parati~: quia qua hora non putatis, Filius hominis veniet.
${}^{41}$~Ait autem ei Petrus~: Domine, ad nos dicis hanc parabolam, an et ad omnes~?
${}^{42}$~Dixit autem Dominus~: Quis, putas, est fidelis dispensator, et prudens, quem constituit dominus supra familiam suam, ut det illis in tempore tritici mensuram~?
${}^{43}$~Beatus ille servus quem, cum venerit dominus, invenerit ita facientem.
${}^{44}$~Vere dico vobis, quoniam supra omnia qu\ae\ possidet, constituet illum.
${}^{45}$~Quod si dixerit servus ille in corde suo~: Moram facit dominus meus venire~: et cœperit percutere servos, et ancillas, et edere, et bibere, et inebriari~:
${}^{46}$~veniet dominus servi illius in die qua non sperat, et hora qua nescit, et dividet eum, partemque ejus cum infidelibus ponet.
${}^{47}$~Ille autem servus qui cognovit voluntatem domini sui, et non pr\ae paravit, et non facit secundum voluntatem ejus, vapulabit multis~:
${}^{48}$~qui autem non cognovit, et fecit digna plagis, vapulabit paucis. Omni autem cui multum datum est, multum qu\ae retur ab eo~: et cui commendaverunt multum, plus petent ab eo.


${}^{49}$~Ignem veni mittere in terram, et quid volo nisi ut accendatur~?
${}^{50}$~Baptismo autem habeo baptizari~: et quomodo coarctor usque dum perficiatur~?
${}^{51}$~Putatis quia pacem veni dare in terram~? non, dico vobis, sed separationem~:
${}^{52}$~erunt enim ex hoc quinque in domo una divisi, tres in duos, et duo in tres
${}^{53}$~dividentur~: pater in filium, et filius in patrem suum, mater in filiam, et filia in matrem, socrus in nurum suam, et nurus in socrum suam.


${}^{54}$~Dicebat autem et ad turbas~: Cum videritis nubem orientem ab occasu, statim dicitis~: Nimbus venit~: et ita fit.
${}^{55}$~Et cum austrum flantem, dicitis~: Quia \ae stus erit~: et fit.
${}^{56}$~Hypocrit\ae~! faciem c\ae li et terr\ae\ nostis probare~: hoc autem tempus quomodo non probatis~?
${}^{57}$~quid autem et a vobis ipsis non judicatis quod justum est~?


${}^{58}$~Cum autem vadis cum adversario tuo ad principem, in via da operam liberari ab illo, ne forte trahat te ad judicem, et judex tradat te exactori, et exactor mittat te in carcerem.
${}^{59}$~Dico tibi, non exies inde, donec etiam novissimum minutum reddas.
\Needspace{2.5\baselineskip}\versal{13}~\lettrine[lines=10,image=true,loversize=0.05,lraise=-0.03]{A}{}derant autem quidam ipso in tempore, nuntiantes illi de Galil\ae is, quorum sanguinem Pilatus miscuit cum sacrificiis eorum.
${}^{2}$~Et respondens dixit illis~: Putatis quod hi Galil\ae i pr\ae\ omnibus Galil\ae is peccatores fuerint, quia talia passi sunt~?
${}^{3}$~Non, dico vobis~: sed nisi pœnitentiam habueritis, omnes similiter peribitis.
${}^{4}$~Sicut illi decem et octo, supra quos cecidit turris in Silo\"e, et occidit eos~: putatis quia et ipsi debitores fuerint pr\ae ter omnes homines habitantes in Jerusalem~?
${}^{5}$~Non, dico vobis~: sed si pœnitentiam non egeritis, omnes similiter peribitis.


${}^{6}$~Dicebat autem et hanc similitudinem~: Arborem fici habebat quidam plantatam in vinea sua, et venit qu\ae rens fructum in illa, et non invenit.
${}^{7}$~Dixit autem ad cultorem vine\ae~: Ecce anni tres sunt ex quo venio qu\ae rens fructum in ficulnea hac, et non invenio~: succide ergo illam~: ut quid etiam terram occupat~?
${}^{8}$~At ille respondens, dicit illi~: Domine dimitte illam et hoc anno, usque dum fodiam circa illam, et mittam stercora,
${}^{9}$~et siquidem fecerit fructum~: sin autem, in futurum succides eam.


${}^{10}$~Erat autem docens in synagoga eorum sabbatis.
${}^{11}$~Et ecce mulier, qu\ae\ habebat spiritum infirmitatis annis decem et octo~: et erat inclinata, nec omnino poterat sursum respicere.
${}^{12}$~Quam cum videret Jesus, vocavit eam ad se, et ait illi~: Mulier, dimissa es ab infirmitate tua.
${}^{13}$~Et imposuit illi manus, et confestim erecta est, et glorificabat Deum.
${}^{14}$~Respondens autem archisynagogus, indignans quia sabbato curasset Jesus, dicebat turb\ae~: Sex dies sunt in quibus oportet operari~: in his ergo venite, et curamini, et non in die sabbati.
${}^{15}$~Respondens autem ad illum Dominus, dixit~: Hypocrit\ae , unusquisque vestrum sabbato non solvit bovem suum, aut asinum a pr\ae sepio, et ducit adaquare~?
${}^{16}$~Hanc autem filiam Abrah\ae , quam alligavit Satanas, ecce decem et octo annis, non oportuit solvi a vinculo isto die sabbati~?
${}^{17}$~Et cum h\ae c diceret, erubescebant omnes adversarii ejus~: et omnis populus gaudebat in universis, qu\ae\ gloriose fiebant ab eo.


${}^{18}$~Dicebat ergo~: Cui simile est regnum Dei, et cui simile \ae stimabo illud~?
${}^{19}$~Simile est grano sinapis, quod acceptum homo misit in hortum suum, et crevit, et factum est in arborem magnam~: et volucres c\ae li requieverunt in ramis ejus.
${}^{20}$~Et iterum dixit~: Cui simile \ae stimabo regnum Dei~?
${}^{21}$~Simile est fermento, quod acceptum mulier abscondit in farin\ae\ sata tria, donec fermentaretur totum.


${}^{22}$~Et ibat per civitates et castella, docens, et iter faciens in Jerusalem.
${}^{23}$~Ait autem illi quidam~: Domine, si pauci sunt, qui salvantur~? Ipse autem dixit ad illos~:
${}^{24}$~Contendite intrare per angustam portam~: quia multi, dico vobis, qu\ae rent intrare, et non poterunt.
${}^{25}$~Cum autem intraverit paterfamilias, et clauserit ostium, incipietis foris stare, et pulsare ostium, dicentes~: Domine, aperi nobis~: et respondens dicet vobis~: Nescio vos unde sitis~:
${}^{26}$~tunc incipietis dicere~: Manducavimus coram te, et bibimus, et in plateis nostris docuisti.
${}^{27}$~Et dicet vobis~: Nescio vos unde sitis~: discedite a me omnes operarii iniquitatis.
${}^{28}$~Ibi erit fletus et stridor dentium~: cum videritis Abraham, et Isaac, et Jacob, et omnes prophetas in regno Dei, vos autem expelli foras.
${}^{29}$~Et venient ab oriente, et occidente, et aquilone, et austro, et accumbent in regno Dei.
${}^{30}$~Et ecce sunt novissimi qui erunt primi, et sunt primi qui erunt novissimi.


${}^{31}$~In ipsa die accesserunt quidam pharis\ae orum, dicentes illi~: Exi, et vade hinc~: quia Herodes vult te occidere.
${}^{32}$~Et ait illis~: Ite, et dicite vulpi illi~: Ecce ejicio d\ae monia, et sanitates perficio hodie, et cras, et tertia die consummor.
${}^{33}$~Verumtamen oportet me hodie et cras et sequenti die ambulare~: quia non capit prophetam perire extra Jerusalem.
${}^{34}$~Jerusalem, Jerusalem, qu\ae\ occidis prophetas, et lapidas eos qui mittuntur ad te, quoties volui congregare filios tuos quemadmodum avis nidum suum sub pennis, et noluisti~?
${}^{35}$~Ecce relinquetur vobis domus vestra deserta. Dico autem vobis, quia non videbitis me donec veniat cum dicetis~: Benedictus qui venit in nomine Domini.
\Needspace{2.5\baselineskip}\versal{14}~\lettrine[lines=10,image=true,loversize=0.05,lraise=-0.03]{E}{}t factum est cum intraret Jesus in domum cujusdam principis pharis\ae orum sabbato manducare panem, et ipsi observabant eum.
${}^{2}$~Et ecce homo quidam hydropicus erat ante illum.
${}^{3}$~Et respondens Jesus dixit ad legisperitos et pharis\ae os, dicens~: Si licet sabbato curare~?
${}^{4}$~At illi tacuerunt. Ipse vero apprehensum sanavit eum, ac dimisit.
${}^{5}$~Et respondens ad illos dixit~: Cujus vestrum asinus, aut bos in puteum cadet, et non continuo extrahet illum die sabbati~?
${}^{6}$~Et non poterant ad h\ae c respondere illi.


${}^{7}$~Dicebat autem et ad invitatos parabolam, intendens quomodo primos accubitus eligerent, dicens ad illos~:
${}^{8}$~Cum invitatus fueris ad nuptias, non discumbas in primo loco, ne forte honoratior te sit invitatus ab illo.
${}^{9}$~Et veniens is, qui te et illum vocavit, dicat tibi~: Da huic locum~: et tunc incipias cum rubore novissimum locum tenere.
${}^{10}$~Sed cum vocatus fueris, vade, recumbe in novissimo loco~: ut, cum venerit qui te invitavit, dicat tibi~: Amice, ascende superius. Tunc erit tibi gloria coram simul discumbentibus~:
${}^{11}$~quia omnis, qui se exaltat, humiliabitur~: et qui se humiliat, exaltabitur.


${}^{12}$~Dicebat autem et ei, qui invitaverat~: Cum facis prandium, aut cœnam, noli vocare amicos tuos, neque fratres tuos, neque cognatos, neque vicinos divites~: ne forte te et ipsi reinvitent, et fiat tibi retributio~;
${}^{13}$~sed cum facis convivium, voca pauperes, debiles, claudos, et c\ae cos~:
${}^{14}$~et beatus eris, quia non habent retribuere tibi~: retribuetur enim tibi in resurrectione justorum.


${}^{15}$~H\ae c cum audisset quidam de simul discumbentibus, dixit illi~: Beatus qui manducabit panem in regno Dei.
${}^{16}$~At ipse dixit ei~: Homo quidam fecit cœnam magnam, et vocavit multos.
${}^{17}$~Et misit servum suum hora cœn\ae\ dicere invitatis ut venirent, quia jam parata sunt omnia.
${}^{18}$~Et cœperunt simul omnes excusare. Primus dixit ei~: Villam emi, et necesse habeo exire, et videre illam~: rogo te, habe me excusatum.
${}^{19}$~Et alter dixit~: Juga boum emi quinque, et eo probare illa~: rogo te, habe me excusatum.
${}^{20}$~Et alius dixit~: Uxorem duxi, et ideo non possum venire.
${}^{21}$~Et reversus servus nuntiavit h\ae c domino suo. Tunc iratus paterfamilias, dixit servo suo~: Exi cito in plateas et vicos civitatis~: et pauperes, ac debiles, et c\ae cos, et claudos introduc huc.
${}^{22}$~Et ait servus~: Domine, factum est ut imperasti, et adhuc locus est.
${}^{23}$~Et ait dominus servo~: Exi in vias, et s\ae pes~: et compelle intrare, ut impleatur domus mea.
${}^{24}$~Dico autem vobis quod nemo virorum illorum qui vocati sunt, gustabit cœnam meam.


${}^{25}$~Ibant autem turb\ae\ mult\ae\ cum eo~: et conversus dixit ad illos~:
${}^{26}$~Si quis venit ad me, et non odit patrem suum, et matrem, et uxorem, et filios, et fratres, et sorores, adhuc autem et animam suam, non potest meus esse discipulus.
${}^{27}$~Et qui non bajulat crucem suam, et venit post me, non potest meus esse discipulus.
${}^{28}$~Quis enim ex vobis volens turrim \ae dificare, non prius sedens computat sumptus, qui necessarii sunt, si habeat ad perficiendum,
${}^{29}$~ne, posteaquam posuerit fundamentum, et non potuerit perficere, omnes qui vident, incipiant illudere ei,
${}^{30}$~dicentes~: Quia hic homo cœpit \ae dificare, et non potuit consummare~?
${}^{31}$~Aut quis rex iturus committere bellum adversus alium regem, non sedens prius cogitat, si possit cum decem millibus occurrere ei, qui cum viginti millibus venit ad se~?
${}^{32}$~Alioquin adhuc illo longe agente, legationem mittens rogat ea qu\ae\ pacis sunt.
${}^{33}$~Sic ergo omnis ex vobis, qui non renuntiat omnibus qu\ae\ possidet, non potest meus esse discipulus.
${}^{34}$~Bonum est sal~: si autem sal evanuerit, in quo condietur~?
${}^{35}$~Neque in terram, neque in sterquilinium utile est, sed foras mittetur. Qui habet aures audiendi, audiat.
\Needspace{2.5\baselineskip}\versal{15}~\lettrine[lines=10,image=true,loversize=0.05,lraise=-0.03]{E}{}rant autem appropinquantes ei publicani, et peccatores ut audirent illum.
${}^{2}$~Et murmurabant pharis\ae i, et scrib\ae , dicentes~: Quia hic peccatores recipit, et manducat cum illis.
${}^{3}$~Et ait ad illos parabolam istam dicens~:
${}^{4}$~Quis ex vobis homo, qui habet centum oves, et si perdiderit unam ex illis, nonne dimittit nonaginta novem in deserto, et vadit ad illam qu\ae\ perierat, donec inveniat eam~?
${}^{5}$~Et cum invenerit eam, imponit in humeros suos gaudens~:
${}^{6}$~et veniens domum convocat amicos et vicinos, dicens illis~: Congratulamini mihi, quia inveni ovem meam, qu\ae\ perierat.
${}^{7}$~Dico vobis quod ita gaudium erit in c\ae lo super uno peccatore pœnitentiam agente, quam super nonaginta novem justis, qui non indigent pœnitentia.


${}^{8}$~Aut qu\ae\ mulier habens drachmas decem, si perdiderit drachmam unam, nonne accendit lucernam, et everrit domum, et qu\ae rit diligenter, donec inveniat~?
${}^{9}$~Et cum invenerit convocat amicas et vicinas, dicens~: Congratulamini mihi, quia inveni drachmam quam perdideram.
${}^{10}$~Ita, dico vobis, gaudium erit coram angelis Dei super uno peccatore pœnitentiam agente.


${}^{11}$~Ait autem~: Homo quidam habuit duos filios~:
${}^{12}$~et dixit adolescentior ex illis patri~: Pater, da mihi portionem substanti\ae , qu\ae\ me contingit. Et divisit illis substantiam.
${}^{13}$~Et non post multos dies, congregatis omnibus, adolescentior filius peregre profectus est in regionem longinquam, et ibi dissipavit substantiam suam vivendo luxuriose.
${}^{14}$~Et postquam omnia consummasset, facta est fames valida in regione illa, et ipse cœpit egere.
${}^{15}$~Et abiit, et adh\ae sit uni civium regionis illius~: et misit illum in villam suam ut pasceret porcos.
${}^{16}$~Et cupiebat implere ventrem suum de siliquis, quas porci manducabant~: et nemo illi dabat.
${}^{17}$~In se autem reversus, dixit~: Quanti mercenarii in domo patris mei abundant panibus, ego autem hic fame pereo~!
${}^{18}$~surgam, et ibo ad patrem meum, et dicam ei~: Pater, peccavi in c\ae lum, et coram te~:
${}^{19}$~jam non sum dignus vocari filius tuus~: fac me sicut unum de mercenariis tuis.
${}^{20}$~Et surgens venit ad patrem suum. Cum autem adhuc longe esset, vidit illum pater ipsius, et misericordia motus est, et accurrens cecidit super collum ejus, et osculatus est eum.
${}^{21}$~Dixitque ei filius~: Pater, peccavi in c\ae lum, et coram te~: jam non sum dignus vocari filius tuus.
${}^{22}$~Dixit autem pater ad servos suos~: Cito proferte stolam primam, et induite illum, et date annulum in manum ejus, et calceamenta in pedes ejus~:
${}^{23}$~et adducite vitulum saginatum, et occidite, et manducemus, et epulemur~:
${}^{24}$~quia hic filius meus mortuus erat, et revixit~: perierat, et inventus est. Et cœperunt epulari.
${}^{25}$~Erat autem filius ejus senior in agro~: et cum veniret, et appropinquaret domui, audivit symphoniam et chorum~:
${}^{26}$~et vocavit unum de servis, et interrogavit quid h\ae c essent.
${}^{27}$~Isque dixit illi~: Frater tuus venit, et occidit pater tuus vitulum saginatum, quia salvum illum recepit.
${}^{28}$~Indignatus est autem, et nolebat introire. Pater ergo illius egressus, cœpit rogare illum.
${}^{29}$~At ille respondens, dixit patri suo~: Ecce tot annis servio tibi, et numquam mandatum tuum pr\ae terivi~: et numquam dedisti mihi h\ae dum ut cum amicis meis epularer.
${}^{30}$~Sed postquam filius tuus hic, qui devoravit substantiam suam cum meretricibus, venit, occidisti illi vitulum saginatum.
${}^{31}$~At ipse dixit illi~: Fili, tu semper mecum es, et omnia mea tua sunt~:
${}^{32}$~epulari autem, et gaudere oportebat, quia frater tuus hic mortuus erat, et revixit~; perierat, et inventus est.
\Needspace{2.5\baselineskip}\versal{16}~\lettrine[lines=10,image=true,loversize=0.05,lraise=-0.03]{D}{}icebat autem et ad discipulos suos~: Homo quidam erat dives, qui habebat villicum~: et hic diffamatus est apud illum quasi dissipasset bona ipsius.
${}^{2}$~Et vocavit illum, et ait illi~: Quid hoc audio de te~? redde rationem villicationis tu\ae~: jam enim non poteris villicare.
${}^{3}$~Ait autem villicus intra se~: Quid faciam, quia dominus meus aufert a me villicationem~? Fodere non valeo, mendicare erubesco.
${}^{4}$~Scio quid faciam, ut, cum amotus fuero a villicatione, recipiant me in domos suas.
${}^{5}$~Convocatis itaque singulis debitoribus domini sui, dicebat primo~: Quantum debes domino meo~?
${}^{6}$~At ille dixit~: Centum cados olei. Dixitque illi~: Accipe cautionem tuam~: et sede cito, scribe quinquaginta.
${}^{7}$~Deinde alii dixit~: Tu vero quantum debes~? Qui ait~: Centum coros tritici. Ait illi~: Accipe litteras tuas, et scribe octoginta.
${}^{8}$~Et laudavit dominus villicum iniquitatis, quia prudenter fecisset~: quia filii hujus s\ae culi prudentiores filiis lucis in generatione sua sunt.
${}^{9}$~Et ego vobis dico~: facite vobis amicos de mammona iniquitatis~: ut, cum defeceritis, recipiant vos in \ae terna tabernacula.
${}^{10}$~Qui fidelis est in minimo, et in majori fidelis est~: et qui in modico iniquus est, et in majori iniquus est.
${}^{11}$~Si ergo in iniquo mammona fideles non fuistis quod verum est, quis credet vobis~?
${}^{12}$~Et si in alieno fideles non fuistis, quod vestrum est, quis dabit vobis~?
${}^{13}$~Nemo servus potest duobus dominis servire~: aut enim unum odiet, et alterum diliget~: aut uni adh\ae rebit, et alterum contemnet. Non potestis Deo servire et mammon\ae .


${}^{14}$~Audiebant autem omnia h\ae c pharis\ae i, qui erant avari~: et deridebant illum.
${}^{15}$~Et ait illis~: Vos estis qui justificatis vos coram hominibus~: Deus autem novit corda vestra~: quia quod hominibus altum est, abominatio est ante Deum.
${}^{16}$~Lex et prophet\ae\ usque ad Joannem~: ex eo regnum Dei evangelizatur, et omnis in illud vim facit.
${}^{17}$~Facilius est autem c\ae lum et terram pr\ae terire, quam de lege unum apicem cadere.
${}^{18}$~Omnis qui dimittit uxorem suam et alteram ducit, mœchatur~: et qui dimissam a viro ducit, mœchatur.


${}^{19}$~Homo quidam erat dives, qui induebatur purpura et bysso, et epulabatur quotidie splendide.
${}^{20}$~Et erat quidam mendicus, nomine Lazarus, qui jacebat ad januam ejus, ulceribus plenus,
${}^{21}$~cupiens saturari de micis qu\ae\ cadebant de mensa divitis, et nemo illi dabat~: sed et canes veniebant, et lingebant ulcera ejus.
${}^{22}$~Factum est autem ut moreretur mendicus, et portaretur ab angelis in sinum Abrah\ae . Mortuus est autem et dives, et sepultus est in inferno.
${}^{23}$~Elevans autem oculos suos, cum esset in tormentis, vidit Abraham a longe, et Lazarum in sinu ejus~:
${}^{24}$~et ipse clamans dixit~: Pater Abraham, miserere mei, et mitte Lazarum ut intingat extremum digiti sui in aquam, ut refrigeret linguam meam, quia crucior in hac flamma.
${}^{25}$~Et dixit illi Abraham~: Fili, recordare quia recepisti bona in vita tua, et Lazarus similiter mala~: nunc autem hic consolatur, tu vero cruciaris~:
${}^{26}$~et in his omnibus inter nos et vos chaos magnum firmatum est~: ut hi qui volunt hinc transire ad vos, non possint, neque inde huc transmeare.
${}^{27}$~Et ait~: Rogo ergo te, pater, ut mittas eum in domum patris mei~:
${}^{28}$~habeo enim quinque fratres~: ut testetur illis, ne et ipsi veniant in hunc locum tormentorum.
${}^{29}$~Et ait illi Abraham~: Habent Moysen et prophetas~: audiant illos.
${}^{30}$~At ille dixit~: Non, pater Abraham~: sed si quis ex mortuis ierit ad eos, pœnitentiam agent.
${}^{31}$~Ait autem illi~: Si Moysen et prophetas non audiunt, neque si quis ex mortuis resurrexerit, credent.
\Needspace{2.5\baselineskip}\versal{17}~\lettrine[lines=10,image=true,loversize=0.05,lraise=-0.03]{E}{}t ait ad discipulos suos~: Impossibile est ut non veniant scandala~: v\ae\ autem illi per quem veniunt.
${}^{2}$~Utilius est illi si lapis molaris imponatur circa collum ejus, et projiciatur in mare quam ut scandalizet unum de pusillis istis.
${}^{3}$~Attendite vobis~: Si peccaverit in te frater tuus, increpa illum~: et si pœnitentiam egerit, dimitte illi.
${}^{4}$~Et si septies in die peccaverit in te, et septies in die conversus fuerit ad te, dicens~: Pœnitet me, dimitte illi.
${}^{5}$~Et dixerunt apostoli Domino~: Adauge nobis fidem.
${}^{6}$~Dixit autem Dominus~: Si habueritis fidem sicut granum sinapis, dicetis huic arbori moro~: Eradicare, et transplantare in mare, et obediet vobis.
${}^{7}$~Quis autem vestrum habens servum arantem aut pascentem, qui regresso de agro dicat illi~: Statim transi, recumbe~:
${}^{8}$~et non dicat ei~: Para quod cœnem, et pr\ae cinge te, et ministra mihi donec manducem, et bibam, et post h\ae c tu manducabis, et bibes~?
${}^{9}$~Numquid gratiam habet servo illi, quia fecit qu\ae\ ei imperaverat~?
${}^{10}$~non puto. Sic et vos cum feceritis omnia qu\ae\ pr\ae cepta sunt vobis, dicite~: Servi inutiles sumus~: quod debuimus facere, fecimus.


${}^{11}$~Et factum est, dum iret in Jerusalem, transibat per mediam Samariam et Galil\ae am.
${}^{12}$~Et cum ingrederetur quoddam castellum, occurrerunt ei decem viri leprosi, qui steterunt a longe~:
${}^{13}$~et levaverunt vocem, dicentes~: Jesu pr\ae ceptor, miserere nostri.
${}^{14}$~Quos ut vidit, dixit~: Ite, ostendite vos sacerdotibus. Et factum est, dum irent, mundati sunt.
${}^{15}$~Unus autem ex illis, ut vidit quia mundatus est, regressus est, cum magna voce magnificans Deum,
${}^{16}$~et cecidit in faciem ante pedes ejus, gratias agens~: et hic erat Samaritanus.
${}^{17}$~Respondens autem Jesus, dixit~: Nonne decem mundati sunt~? et novem ubi sunt~?
${}^{18}$~Non est inventus qui rediret, et daret gloriam Deo, nisi hic alienigena.
${}^{19}$~Et ait illi~: Surge, vade~: quia fides tua te salvum fecit.


${}^{20}$~Interrogatus autem a pharis\ae is~: Quando venit regnum Dei~? respondens eis, dixit~: Non venit regnum Dei cum observatione~:
${}^{21}$~neque dicent~: Ecce hic, aut ecce illic. Ecce enim regnum Dei intra vos est.
${}^{22}$~Et ait ad discipulos suos~: Venient dies quando desideretis videre unum diem Filii hominis, et non videbitis.
${}^{23}$~Et dicent vobis~: Ecce hic, et ecce illic. Nolite ire, neque sectemini~:
${}^{24}$~nam, sicut fulgur coruscans de sub c\ae lo in ea qu\ae\ sub c\ae lo sunt, fulget~: ita erit Filius hominis in die sua.
${}^{25}$~Primum autem oportet illum multa pati, et reprobari a generatione hac.
${}^{26}$~Et sicut factum est in diebus No\"e, ita erit et in diebus Filii hominis~:
${}^{27}$~edebant et bibebant~: uxores ducebant et dabantur ad nuptias, usque in diem, qua intravit No\"e in arcam~: et venit diluvium, et perdidit omnes.
${}^{28}$~Similiter sicut factum est in diebus Lot~: edebant et bibebant, emebant et vendebant, plantabant et \ae dificabant~:
${}^{29}$~qua die autem exiit Lot a Sodomis, pluit ignem et sulphur de c\ae lo, et omnes perdidit~:
${}^{30}$~secundum h\ae c erit qua die Filius hominis revelabitur.
${}^{31}$~In illa hora, qui fuerit in tecto, et vasa ejus in domo, ne descendat tollere illa~: et qui in agro, similiter non redeat retro.
${}^{32}$~Memores estote uxoris Lot.
${}^{33}$~Quicumque qu\ae sierit animam suam salvam facere, perdet illam~: et quicumque perdiderit illam, vivificabit eam.
${}^{34}$~Dico vobis~: In illa nocte erunt duo in lecto uno~: unus assumetur, et alter relinquetur~:
${}^{35}$~du\ae\ erunt molentes in unum~: una assumetur, et altera relinquetur~: duo in agro~: unus assumetur, et alter relinquetur.
${}^{36}$~Respondentes dicunt illi~: Ubi Domine~?
${}^{37}$~Qui dixit illis~: Ubicumque fuerit corpus, illuc congregabuntur et aquil\ae .
\Needspace{2.5\baselineskip}\versal{18}~\lettrine[lines=10,image=true,loversize=0.05,lraise=-0.03]{D}{}icebat autem et parabolam ad illos, quoniam oportet semper orare et non deficere,
${}^{2}$~dicens~: Judex quidam erat in quadam civitate, qui Deum non timebat, et hominem non reverebatur.
${}^{3}$~Vidua autem qu\ae dam erat in civitate illa, et veniebat ad eum, dicens~: Vindica me de adversario meo.
${}^{4}$~Et nolebat per multum tempus. Post h\ae c autem dixit intra se~: Etsi Deum non timeo, nec hominem revereor~:
${}^{5}$~tamen quia molesta est mihi h\ae c vidua, vindicabo illam, ne in novissimo veniens sugillet me.
${}^{6}$~Ait autem Dominus~: Audite quid judex iniquitatis dicit~:
${}^{7}$~Deus autem non faciet vindictam electorum suorum clamantium ad se die ac nocte, et patientiam habebit in illis~?
${}^{8}$~Dico vobis quia cito faciet vindictam illorum. Verumtamen Filius hominis veniens, putas, inveniet fidem in terra~?


${}^{9}$~Dixit autem et ad quosdam qui in se confidebant tamquam justi, et aspernabantur ceteros, parabolam istam~:
${}^{10}$~Duo homines ascenderunt in templum ut orarent~: unus pharis\ae us et alter publicanus.
${}^{11}$~Pharis\ae us stans, h\ae c apud se orabat~: Deus, gratias ago tibi, quia non sum sicut ceteri hominum~: raptores, injusti, adulteri, velut etiam hic publicanus~:
${}^{12}$~jejuno bis in sabbato, decimas do omnium qu\ae\ possideo.
${}^{13}$~Et publicanus a longe stans, nolebat nec oculos ad c\ae lum levare~: sed percutiebat pectus suum, dicens~: Deus propitius esto mihi peccatori.
${}^{14}$~Dico vobis, descendit hic justificatus in domum suam ab illo~: quia omnis qui se exaltat, humiliabitur, et qui se humiliat, exaltabitur.


${}^{15}$~Afferebant autem ad illum et infantes, ut eos tangeret. Quod cum viderent discipuli, increpabant illos.
${}^{16}$~Jesus autem convocans illos, dixit~: Sinite pueros venire ad me, et nolite vetare eos~: talium est enim regnum Dei.
${}^{17}$~Amen dico vobis, quicumque non acceperit regnum Dei sicut puer, non intrabit in illud.


${}^{18}$~Et interrogavit eum quidam princeps, dicens~: Magister bone, quid faciens vitam \ae ternam possidebo~?
${}^{19}$~Dixit autem ei Jesus~: Quid me dicis bonum~? nemo bonus nisi solus Deus.
${}^{20}$~Mandata nosti~: non occides~; non mœchaberis~; non furtum facies~; non falsum testimonium dices~; honora patrem tuum et matrem.
${}^{21}$~Qui ait~: H\ae c omnia custodivi a juventute mea.
${}^{22}$~Quo audito, Jesus ait ei~: Adhuc unum tibi deest~: omnia qu\ae cumque habes vende, et da pauperibus, et habebis thesaurum in c\ae lo~: et veni, sequere me.
${}^{23}$~His ille auditis, contristatus est~: quia dives erat valde.
${}^{24}$~Videns autem Jesus illum tristem factum, dixit~: Quam difficile, qui pecunias habent, in regnum Dei intrabunt~!
${}^{25}$~facilius est enim camelum per foramen acus transire quam divitem intrare in regnum Dei.
${}^{26}$~Et dixerunt qui audiebant~: Et quis potest salvus fieri~?
${}^{27}$~Ait illis~: Qu\ae\ impossibilia sunt apud homines, possibilia sunt apud Deum.
${}^{28}$~Ait autem Petrus~: Ecce nos dimisimus omnia et secuti sumus te.
${}^{29}$~Qui dixit eis~: Amen dico vobis, nemo est qui reliquit domum, aut parentes, aut fratres, aut uxorem, aut filios propter regnum Dei,
${}^{30}$~et non recipiat multo plura in hoc tempore, et in s\ae culo venturo vitam \ae ternam.


${}^{31}$~Assumpsit autem Jesus duodecim, et ait illis~: Ecce ascendimus Jerosolymam, et consummabuntur omnia qu\ae\ scripta sunt per prophetas de Filio hominis~:
${}^{32}$~tradetur enim gentibus, et illudetur, et flagellabitur, et conspuetur~:
${}^{33}$~et postquam flagellaverint, occident eum, et tertia die resurget.
${}^{34}$~Et ipsi nihil horum intellexerunt, et erat verbum istud absconditum ab eis, et non intelligebant qu\ae\ dicebantur.


${}^{35}$~Factum est autem, cum appropinquaret Jericho, c\ae cus quidam sedebat secus viam, mendicans.
${}^{36}$~Et cum audiret turbam pr\ae tereuntem, interrogabat quid hoc esset.
${}^{37}$~Dixerunt autem ei quod Jesus Nazarenus transiret.
${}^{38}$~Et clamavit, dicens~: Jesu, fili David, miserere mei.
${}^{39}$~Et qui pr\ae ibant, increpabant eum ut taceret. Ipse vero multo magis clamabat~: Fili David, miserere mei.
${}^{40}$~Stans autem Jesus jussit illum adduci ad se. Et cum appropinquasset, interrogavit illum,
${}^{41}$~dicens~: Quid tibi vis faciam~? At ille dixit~: Domine, ut videam.
${}^{42}$~Et Jesus dixit illi~: Respice, fides tua te salvum fecit.
${}^{43}$~Et confestim vidit, et sequebatur illum magnificans Deum. Et omnis plebs ut vidit, dedit laudem Deo.
\Needspace{2.5\baselineskip}\versal{19}~\lettrine[lines=10,image=true,loversize=0.05,lraise=-0.03]{E}{}t ingressus perambulabat Jericho.
${}^{2}$~Et ecce vir nomine Zach\ae us~: et hic princeps erat publicanorum, et ipse dives~:
${}^{3}$~et qu\ae rebat videre Jesum, quis esset~: et non poterat pr\ae\ turba, quia statura pusillus erat.
${}^{4}$~Et pr\ae currens ascendit in arborem sycomorum ut videret eum~: quia inde erat transiturus.
${}^{5}$~Et cum venisset ad locum, suspiciens Jesus vidit illum, et dixit ad eum~: Zach\ae e, festinans descende~: quia hodie in domo tua oportet me manere.
${}^{6}$~Et festinans descendit, et excepit illum gaudens.
${}^{7}$~Et cum viderent omnes, murmurabant, dicentes quod ad hominem peccatorem divertisset.
${}^{8}$~Stans autem Zach\ae us, dixit ad Dominum~: Ecce dimidium bonorum meorum, Domine, do pauperibus~: et si quid aliquem defraudavi, reddo quadruplum.
${}^{9}$~Ait Jesus ad eum~: Quia hodie salus domui huic facta est~: eo quod et ipse filius sit Abrah\ae .
${}^{10}$~Venit enim Filius hominis qu\ae rere, et salvum facere quod perierat.


${}^{11}$~H\ae c illis audientibus adjiciens, dixit parabolam, eo quod esset prope Jerusalem~: et quia existimarent quod confestim regnum Dei manifestaretur.
${}^{12}$~Dixit ergo~: Homo quidam nobilis abiit in regionem longinquam accipere sibi regnum, et reverti.
${}^{13}$~Vocatis autem decem servis suis, dedit eis decem mnas, et ait ad illos~: Negotiamini dum venio.
${}^{14}$~Cives autem ejus oderant eum~: et miserunt legationem post illum, dicentes~: Nolumus hunc regnare super nos.
${}^{15}$~Et factum est ut rediret accepto regno~: et jussit vocari servos, quibus dedit pecuniam, ut sciret quantum quisque negotiatus esset.
${}^{16}$~Venit autem primus dicens~: Domine, mna tua decem mnas acquisivit.
${}^{17}$~Et ait illi~: Euge bone serve, quia in modico fuisti fidelis, eris potestatem habens super decem civitates.
${}^{18}$~Et alter venit, dicens~: Domine, mna tua fecit quinque mnas.
${}^{19}$~Et huic ait~: Et tu esto super quinque civitates.
${}^{20}$~Et alter venit, dicens~: Domine, ecce mna tua, quam habui repositam in sudario~:
${}^{21}$~timui enim te, quia homo austerus es~: tollis quod non posuisti, et metis quod non seminasti.
${}^{22}$~Dicit ei~: De ore tuo te judico, serve nequam. Sciebas quod ego homo austerus sum, tollens quod non posui, et metens quod non seminavi~:
${}^{23}$~et quare non dedisti pecuniam meam ad mensam, ut ego veniens cum usuris utique exegissem illam~?
${}^{24}$~Et astantibus dixit~: Auferte ab illo mnam, et date illi qui decem mnas habet.
${}^{25}$~Et dixerunt ei~: Domine, habet decem mnas.
${}^{26}$~Dico autem vobis, quia omni habenti dabitur, et abundabit~: ab eo autem qui non habet, et quod habet auferetur ab eo.
${}^{27}$~Verumtamen inimicos meos illos, qui noluerunt me regnare super se, adducite huc~: et interficite ante me.
${}^{28}$~Et his dictis, pr\ae cedebat ascendens Jerosolymam.


${}^{29}$~Et factum est, cum appropinquasset ad Bethphage et Bethaniam, ad montem qui vocatur Oliveti, misit duos discipulos suos,
${}^{30}$~dicens~: Ite in castellum quod contra est~: in quod intro\"euntes, invenietis pullum asin\ae\ alligatum, cui nemo umquam hominum sedit~: solvite illum, et adducite.
${}^{31}$~Et si quis vos interrogaverit~: Quare solvitis~? sic dicetis ei~: Quia Dominus operam ejus desiderat.
${}^{32}$~Abierunt autem qui missi erant~: et invenerunt, sicut dixit illis, stantem pullum.
${}^{33}$~Solventibus autem illis pullum, dixerunt domini ejus ad illos~: Quid solvitis pullum~?
${}^{34}$~At illi dixerunt~: Quia Dominus eum necessarium habet.
${}^{35}$~Et duxerunt illum ad Jesum. Et jactantes vestimenta sua supra pullum, imposuerunt Jesum.
${}^{36}$~Eunte autem illo, substernebant vestimenta sua in via~:
${}^{37}$~et cum appropinquaret jam ad descensum montis Oliveti, cœperunt omnes turb\ae\ discipulorum gaudentes laudare Deum voce magna super omnibus, quas viderant, virtutibus,
${}^{38}$~dicentes~: Benedictus, qui venit rex in nomine Domini~: pax in c\ae lo, et gloria in excelsis.
${}^{39}$~Et quidam pharis\ae orum de turbis dixerunt ad illum~: Magister, increpa discipulos tuos.
${}^{40}$~Quibus ipse ait~: Dico vobis, quia si hi tacuerint, lapides clamabunt.


${}^{41}$~Et ut appropinquavit, videns civitatem flevit super illam, dicens~:
${}^{42}$~Quia si cognovisses et tu, et quidem in hac die tua, qu\ae\ ad pacem tibi~: nunc autem abscondita sunt ab oculis tuis.
${}^{43}$~Quia venient dies in te~: et circumdabunt te inimici tui vallo, et circumdabunt te~: et coangustabunt te undique~:
${}^{44}$~et ad terram prosternent te, et filios tuos, qui in te sunt, et non relinquent in te lapidem super lapidem~: eo quod non cognoveris tempus visitationis tu\ae .
${}^{45}$~Et ingressus in templum, cœpit ejicere vendentes in illo, et ementes,
${}^{46}$~dicens illis~: Scriptum est~: Quia domus mea domus orationis est~: vos autem fecistis illam speluncam latronum.
${}^{47}$~Et erat docens quotidie in templo. Principes autem sacerdotum, et scrib\ae , et princeps plebis qu\ae rebant illum perdere~:
${}^{48}$~et non inveniebant quid facerent illi. Omnis enim populus suspensus erat, audiens illum.
\Needspace{2.5\baselineskip}\versal{20}~\lettrine[lines=10,image=true,loversize=0.05,lraise=-0.03]{E}{}t factum est in una dierum, docente illo populum in templo, et evangelizante, convenerunt principes sacerdotum, et scrib\ae\ cum senioribus,
${}^{2}$~et aiunt dicentes ad illum~: Dic nobis in qua potestate h\ae c facis~? aut quis est qui dedit tibi hanc potestatem~?
${}^{3}$~Respondens autem Jesus, dixit ad illos~: Interrogabo vos et ego unum verbum. Respondete mihi~:
${}^{4}$~baptismus Joannis de c\ae lo erat, an ex hominibus~?
${}^{5}$~At illi cogitabant intra se, dicentes~: Quia si dixerimus~: De c\ae lo, dicet~: Quare ergo non credidistis illi~?
${}^{6}$~Si autem dixerimus~: Ex hominibus, plebs universa lapidabit nos~: certi sunt enim Joannem prophetam esse.
${}^{7}$~Et responderunt se nescire unde esset.
${}^{8}$~Et Jesus ait illis~: Neque ego dico vobis in qua potestate h\ae c facio.


${}^{9}$~Cœpit autem dicere ad plebem parabolam hanc~: Homo plantavit vineam, et locavit eam colonis~: et ipse peregre fuit multis temporibus.
${}^{10}$~Et in tempore misit ad cultores servum, ut de fructu vine\ae\ darent illi. Qui c\ae sum dimiserunt eum inanem.
${}^{11}$~Et addidit alterum servum mittere. Illi autem hunc quoque c\ae dentes, et afficientes contumelia, dimiserunt inanem.
${}^{12}$~Et addidit tertium mittere~: qui et illum vulnerantes ejecerunt.
${}^{13}$~Dixit autem dominus vine\ae~: Quid faciam~? Mittam filium meum dilectum~: forsitan, cum hunc viderint, verebuntur.
${}^{14}$~Quem cum vidissent coloni, cogitaverunt intra se, dicentes~: Hic est h\ae res, occidamus illum, ut nostra fiat h\ae reditas.
${}^{15}$~Et ejectum illum extra vineam, occiderunt. Quid ergo faciet illis dominus vine\ae~?
${}^{16}$~veniet, et perdet colonos istos, et dabit vineam aliis. Quo audito, dixerunt illi~: Absit.
${}^{17}$~Ille autem aspiciens eos, ait~: Quid est ergo hoc quod scriptum est~: Lapidem quem reprobaverunt \ae dificantes, hic factus est in caput anguli~?
${}^{18}$~Omnis qui ceciderit super illum lapidem, conquassabitur~: super quem autem ceciderit, comminuet illum.


${}^{19}$~Et qu\ae rebant principes sacerdotum et scrib\ae\ mittere in illum manus illa hora, et timuerunt populum~: cognoverunt enim quod ad ipsos dixerit similitudinem hanc.
${}^{20}$~Et observantes miserunt insidiatores, qui se justos simularent, ut caperent eum in sermone, ut traderent illum principatui, et potestati pr\ae sidis.
${}^{21}$~Et interrogaverunt eum, dicentes~: Magister, scimus quia recte dicis et doces~: et non accipis personam, sed viam Dei in veritate doces.
${}^{22}$~Licet nobis tributum dare C\ae sari, an non~?
${}^{23}$~Considerans autem dolum illorum, dixit ad eos~: Quid me tentatis~?
${}^{24}$~ostendite mihi denarium. Cujus habet imaginem et inscriptionem~? Respondentes dixerunt ei~: C\ae saris.
${}^{25}$~Et ait illis~: Reddite ergo qu\ae\ sunt C\ae saris, C\ae sari~: et qu\ae\ sunt Dei, Deo.
${}^{26}$~Et non potuerunt verbum ejus reprehendere coram plebe~: et mirati in responso ejus, tacuerunt.


${}^{27}$~Accesserunt autem quidam sadduc\ae orum, qui negant esse resurrectionem, et interrogaverunt eum,
${}^{28}$~dicentes~: Magister, Moyses scripsit nobis~: Si frater alicujus mortuus fuerit habens uxorem, et hic sine liberis fuerit, ut accipiat eam frater ejus uxorem, et suscitet semen fratri suo.
${}^{29}$~Septem ergo fratres erant~: et primus accepit uxorem, et mortuus est sine filiis.
${}^{30}$~Et sequens accepit illam, et ipse mortuus est sine filio.
${}^{31}$~Et tertius accepit illam. Similiter et omnes septem, et non reliquerunt semen, et mortui sunt.
${}^{32}$~Novissime omnium mortua est et mulier.
${}^{33}$~In resurrectione ergo, cujus eorum erit uxor~? siquidem septem habuerunt eam uxorem.
${}^{34}$~Et ait illis Jesus~: Filii hujus s\ae culi nubunt, et traduntur ad nuptias~:
${}^{35}$~illi vero qui digni habebuntur s\ae culo illo, et resurrectione ex mortuis, neque nubent, neque ducent uxores~:
${}^{36}$~neque enim ultra mori potuerunt~: \ae quales enim angelis sunt, et filii sunt Dei, cum sint filii resurrectionis.
${}^{37}$~Quia vero resurgant mortui, et Moyses ostendit secus rubum, sicut dicit Dominum, Deum Abraham, et Deum Isaac, et Deum Jacob.
${}^{38}$~Deus autem non est mortuorum, sed vivorum~: omnes enim vivunt ei.
${}^{39}$~Respondentes autem quidam scribarum, dixerunt ei~: Magister, bene dixisti.
${}^{40}$~Et amplius non audebant eum quidquam interrogare.


${}^{41}$~Dixit autem ad illos~: Quomodo dicunt Christum filium esse David~?
${}^{42}$~et ipse David dicit in libro Psalmorum~: Dixit Dominus Domino meo~: sede a dextris meis,
${}^{43}$~donec ponam inimicos tuos scabellum pedum tuorum.
${}^{44}$~David ergo Dominum illum vocat~: et quomodo filius ejus est~?
${}^{45}$~Audiente autem omni populo, dixit discipulis suis~:
${}^{46}$~Attendite a scribis, qui volunt ambulare in stolis, et amant salutationes in foro, et primas cathedras in synagogis, et primos discubitus in conviviis,
${}^{47}$~qui devorant domos viduarum, simulantes longam orationem~: hi accipient damnationem majorem.
\Needspace{2.5\baselineskip}\versal{21}~\lettrine[lines=10,image=true,loversize=0.05,lraise=-0.03]{R}{}espiciens autem, vidit eos qui mittebant munera sua in gazophylacium, divites.
${}^{2}$~Vidit autem et quamdam viduam pauperculam mittentem \ae ra minuta duo.
${}^{3}$~Et dixit~: Vere dico vobis, quia vidua h\ae c pauper plus quam omnes misit.
${}^{4}$~Nam omnes hi ex abundanti sibi miserunt in munera Dei~: h\ae c autem ex eo quod deest illi, omnem victum suum quem habuit, misit.


${}^{5}$~Et quibusdam dicentibus de templo quod bonis lapidibus et donis ornatum esset, dixit~:
${}^{6}$~H\ae c qu\ae\ videtis, venient dies in quibus non relinquetur lapis super lapidem, qui non destruatur.
${}^{7}$~Interrogaverunt autem illum, dicentes~: Pr\ae ceptor, quando h\ae c erunt, et quod signum cum fieri incipient~?
${}^{8}$~Qui dixit~: Videte ne seducamini~: multi enim venient in nomine meo, dicentes quia ego sum~: et tempus appropinquavit~: nolite ergo ire post eos.
${}^{9}$~Cum autem audieritis pr\ae lia et seditiones, nolite terreri~: oportet primum h\ae c fieri, sed nondum statim finis.
${}^{10}$~Tunc dicebat illis~: Surget gens contra gentem, et regnum adversus regnum.
${}^{11}$~Et terr\ae motus magni erunt per loca, et pestilenti\ae , et fames, terroresque de c\ae lo, et signa magna erunt.


${}^{12}$~Sed ante h\ae c omnia injicient vobis manus suas, et persequentur tradentes in synagogas et custodias, trahentes ad reges et pr\ae sides propter nomen meum~:
${}^{13}$~continget autem vobis in testimonium.
${}^{14}$~Ponite ergo in cordibus vestris non pr\ae meditari quemadmodum respondeatis~:
${}^{15}$~ego enim dabo vobis os et sapientiam, cui non poterunt resistere et contradicere omnes adversarii vestri.
${}^{16}$~Trademini autem a parentibus, et fratribus, et cognatis, et amicis, et morte afficient ex vobis~:
${}^{17}$~et eritis odio omnibus propter nomen meum~:
${}^{18}$~et capillus de capite vestro non peribit.
${}^{19}$~In patientia vestra possidebitis animas vestras.


${}^{20}$~Cum autem videritis circumdari ab exercitu Jerusalem, tunc scitote quia appropinquavit desolatio ejus~:
${}^{21}$~tunc qui in Jud\ae a sunt, fugiant ad montes, et qui in medio ejus, discedant~: et qui in regionibus, non intrent in eam,
${}^{22}$~quia dies ultionis hi sunt, ut impleantur omnia qu\ae\ scripta sunt.
${}^{23}$~V\ae\ autem pr\ae gnantibus et nutrientibus in illis diebus~! erit enim pressura magna super terram, et ira populo huic.
${}^{24}$~Et cadent in ore gladii, et captivi ducentur in omnes gentes, et Jerusalem calcabitur a gentibus, donec impleantur tempora nationum.


${}^{25}$~Et erunt signa in sole, et luna, et stellis, et in terris pressura gentium pr\ae\ confusione sonitus maris, et fluctuum~:
${}^{26}$~arescentibus hominibus pr\ae\ timore, et exspectatione, qu\ae\ supervenient universo orbi~: nam virtutes c\ae lorum movebuntur~:
${}^{27}$~et tunc videbunt Filium hominis venientem in nube cum potestate magna et majestate.


${}^{28}$~His autem fieri incipientibus, respicite, et levate capita vestra~: quoniam appropinquat redemptio vestra.
${}^{29}$~Et dixit illis similitudinem~: Videte ficulneam, et omnes arbores~:
${}^{30}$~cum producunt jam ex se fructum, scitis quoniam prope est \ae stas.
${}^{31}$~Ita et vos cum videritis h\ae c fieri, scitote quoniam prope est regnum Dei.
${}^{32}$~Amen dico vobis, quia non pr\ae teribit generatio h\ae c, donec omnia fiant.
${}^{33}$~C\ae lum et terra transibunt~: verba autem mea non transibunt.


${}^{34}$~Attendite autem vobis, ne forte graventur corda vestra in crapula, et ebrietate, et curis hujus vit\ae , et superveniat in vos repentina dies illa~:
${}^{35}$~tamquam laqueus enim superveniet in omnes qui sedent super faciem omnis terr\ae .
${}^{36}$~Vigilate itaque, omni tempore orantes, ut digni habeamini fugere ista omnia qu\ae\ futura sunt, et stare ante Filium hominis.


${}^{37}$~Erat autem diebus docens in templo~: noctibus vero exiens, morabatur in monte qui vocatur Oliveti.
${}^{38}$~Et omnis populus manicabat ad eum in templo audire eum.
\Needspace{2.5\baselineskip}\versal{22}~\lettrine[lines=10,image=true,loversize=0.05,lraise=-0.03]{A}{}ppropinquabat autem dies festus azymorum, qui dicitur Pascha~:
${}^{2}$~et qu\ae rebant principes sacerdotum, et scrib\ae , quomodo Jesum interficerent~: timebant vero plebem.
${}^{3}$~Intravit autem Satanas in Judam, qui cognominabatur Iscariotes, unum de duodecim~:
${}^{4}$~et abiit, et locutus est cum principibus sacerdotum, et magistratibus, quemadmodum illum traderet eis.
${}^{5}$~Et gavisi sunt, et pacti sunt pecuniam illi dare.
${}^{6}$~Et spopondit, et qu\ae rebat opportunitatem ut traderet illum sine turbis.


${}^{7}$~Venit autem dies azymorum, in qua necesse erat occidi pascha.
${}^{8}$~Et misit Petrum et Joannem, dicens~: Euntes parate nobis pascha, ut manducemus.
${}^{9}$~At illi dixerunt~: Ubi vis paremus~?
${}^{10}$~Et dixit ad eos~: Ecce intro\"euntibus vobis in civitatem occurret vobis homo quidam amphoram aqu\ae\ portans~: sequimini eum in domum, in quam intrat,
${}^{11}$~et dicetis patrifamilias domus~: Dicit tibi Magister~: Ubi est diversorium, ubi pascha cum discipulis meis manducem~?
${}^{12}$~Et ipse ostendet vobis cœnaculum magnum stratum, et ibi parate.
${}^{13}$~Euntes autem invenerunt sicut dixit illis, et paraverunt pascha.
${}^{14}$~Et cum facta esset hora, discubuit, et duodecim apostoli cum eo.
${}^{15}$~Et ait illis~: Desiderio desideravi hoc pascha manducare vobiscum, antequam patiar.
${}^{16}$~Dico enim vobis, quia ex hoc non manducabo illud, donec impleatur in regno Dei.
${}^{17}$~Et accepto calice gratias egit, et dixit~: Accipite, et dividite inter vos.
${}^{18}$~Dico enim vobis quod non bibam de generatione vitis donec regnum Dei veniat.


${}^{19}$~Et accepto pane gratias egit, et fregit, et dedit eis, dicens~: Hoc est corpus meum, quod pro vobis datur~: hoc facite in meam commemorationem.
${}^{20}$~Similiter et calicem, postquam cœnavit, dicens~: Hic est calix novum testamentum in sanguine meo, qui pro vobis fundetur.
${}^{21}$~Verumtamen ecce manus tradentis me, mecum est in mensa.
${}^{22}$~Et quidem Filius hominis, secundum quod definitum est, vadit~: verumtamen v\ae\ homini illi per quem tradetur.
${}^{23}$~Et ipsi cœperunt qu\ae rere inter se quis esset ex eis qui hoc facturus esset.


${}^{24}$~Facta est autem et contentio inter eos, quis eorum videretur esse major.
${}^{25}$~Dixit autem eis~: Reges gentium dominantur eorum~: et qui potestatem habent super eos, benefici vocantur.
${}^{26}$~Vos autem non sic~: sed qui major est in vobis, fiat sicut minor~: et qui pr\ae cessor est, sicut ministrator.
${}^{27}$~Nam quis major est, qui recumbit, an qui ministrat~? nonne qui recumbit~? Ego autem in medio vestrum sum, sicut qui ministrat~:
${}^{28}$~vos autem estis, qui permansistis mecum in tentationibus meis.
${}^{29}$~Et ego dispono vobis sicut disposuit mihi Pater meus regnum,
${}^{30}$~ut edatis et bibatis super mensam meam in regno meo, et sedeatis super thronos judicantes duodecim tribus Isra\"el.


${}^{31}$~Ait autem Dominus~: Simon, Simon, ecce Satanas expetivit vos ut cribraret sicut triticum~:
${}^{32}$~ego autem rogavi pro te ut non deficiat fides tua~: et tu aliquando conversus, confirma fratres tuos.
${}^{33}$~Qui dixit ei~: Domine, tecum paratus sum et in carcerem et in mortem ire.
${}^{34}$~At ille dixit~: Dico tibi, Petre, non cantabit hodie gallus, donec ter abneges nosse me. Et dixit eis~:
${}^{35}$~Quando misi vos sine sacculo, et pera, et calceamentis, numquid aliquid defuit vobis~?
${}^{36}$~At illi dixerunt~: Nihil. Dixit ergo eis~: Sed nunc qui habet sacculum, tollat~; similiter et peram~: et qui non habet, vendat tunicam suam et emat gladium.
${}^{37}$~Dico enim vobis, quoniam adhuc hoc quod scriptum est, oportet impleri in me~: Et cum iniquis deputatus est. Etenim ea qu\ae\ sunt de me finem habent.
${}^{38}$~At illi dixerunt~: Domine, ecce duo gladii hic. At ille dixit eis~: Satis est.


${}^{39}$~Et egressus ibat secundum consuetudinem in monte Olivarum. Secuti sunt autem illum et discipuli.
${}^{40}$~Et cum pervenisset ad locum, dixit illis~: Orate ne intretis in tentationem.
${}^{41}$~Et ipse avulsus est ab eis quantum jactus est lapidis~: et positis genibus orabat,
${}^{42}$~dicens~: Pater, si vis, transfer calicem istum a me~: verumtamen non mea voluntas, sed tua fiat.
${}^{43}$~Apparuit autem illi angelus de c\ae lo, confortans eum. Et factus in agonia, prolixius orabat.
${}^{44}$~Et factus est sudor ejus sicut gutt\ae\ sanguinis decurrentis in terram.
${}^{45}$~Et cum surrexisset ab oratione et venisset ad discipulos suos, invenit eos dormientes pr\ae\ tristitia.
${}^{46}$~Et ait illis~: Quid dormitis~? surgite, orate, ne intretis in tentationem.


${}^{47}$~Adhuc eo loquente, ecce turba~: et qui vocabatur Judas, unus de duodecim, antecedebat eos, et appropinquavit Jesu ut oscularetur eum.
${}^{48}$~Jesus autem dixit illi~: Juda, osculo Filium hominis tradis~?
${}^{49}$~Videntes autem hi qui circa ipsum erant, quod futurum erat, dixerunt ei~: Domine, si percutimus in gladio~?
${}^{50}$~Et percussit unus ex illis servum principis sacerdotum, et amputavit auriculam ejus dexteram.
${}^{51}$~Respondens autem Jesus, ait~: Sinite usque huc. Et cum tetigisset auriculam ejus, sanavit eum.
${}^{52}$~Dixit autem Jesus ad eos qui venerant ad se principes sacerdotum, et magistratus templi, et seniores~: Quasi ad latronem existis cum gladiis et fustibus~?
${}^{53}$~Cum quotidie vobiscum fuerim in templo, non extendistis manus in me~: sed h\ae c est hora vestra, et potestas tenebrarum.
${}^{54}$~Comprehendentes autem eum, duxerunt ad domum principis sacerdotum~: Petrus vero sequebatur a longe.


${}^{55}$~Accenso autem igne in medio atrii et circumsedentibus illis, erat Petrus in medio eorum.
${}^{56}$~Quem cum vidisset ancilla qu\ae dam sedentem ad lumen, et eum fuisset intuita, dixit~: Et hic cum illo erat.
${}^{57}$~At ille negavit eum, dicens~: Mulier, non novi illum.
${}^{58}$~Et post pusillum alius videns eum, dixit~: Et tu de illis es. Petrus vero ait~: O homo, non sum.
${}^{59}$~Et intervallo facto quasi hor\ae\ unius, alius quidam affirmabat, dicens~: Vere et hic cum illo erat~: nam et Galil\ae us est.
${}^{60}$~Et ait Petrus~: Homo, nescio quid dicis. Et continuo, adhuc illo loquente, cantavit gallus.
${}^{61}$~Et conversus Dominus respexit Petrum, et recordatus est Petrus verbi Domini, sicut dixerat~: Quia priusquam gallus cantet, ter me negabis.
${}^{62}$~Et egressus foras Petrus flevit amare.


${}^{63}$~Et viri qui tenebant illum, illudebant ei, c\ae dentes.
${}^{64}$~Et velaverunt eum, et percutiebant faciem ejus~: et interrogabant eum, dicentes~: Prophetiza, quis est, qui te percussit~?
${}^{65}$~Et alia multa blasphemantes dicebant in eum.


${}^{66}$~Et ut factus est dies, convenerunt seniores plebis, et principes sacerdotum, et scrib\ae , et duxerunt illum in concilium suum, dicentes~: Si tu es Christus, dic nobis.
${}^{67}$~Et ait illis~: Si vobis dixero, non credetis mihi~:
${}^{68}$~si autem et interrogavero, non respondebitis mihi, neque dimittetis.
${}^{69}$~Ex hoc autem erit Filius hominis sedens a dextris virtutis Dei.
${}^{70}$~Dixerunt autem omnes~: Tu ergo es Filius Dei~? Qui ait~: Vos dicitis, quia ego sum.
${}^{71}$~At illi dixerunt~: Quid adhuc desideramus testimonium~? ipsi enim audivimus de ore ejus.
\Needspace{2.5\baselineskip}\versal{23}~\lettrine[lines=10,image=true,loversize=0.05,lraise=-0.03]{E}{}t surgens omnis multitudo eorum, duxerunt illum ad Pilatum.
${}^{2}$~Cœperunt autem illum accusare, dicentes~: Hunc invenimus subvertentem gentem nostram, et prohibentem tributa dare C\ae sari, et dicentem se Christum regem esse.
${}^{3}$~Pilatus autem interrogavit eum, dicens~: Tu es rex Jud\ae orum~? At ille respondens ait~: Tu dicis.
${}^{4}$~Ait autem Pilatus ad principes sacerdotum et turbas~: Nihil invenio caus\ae\ in hoc homine.
${}^{5}$~At illi invalescebant, dicentes~: Commovet populum docens per universam Jud\ae am, incipiens a Galil\ae a usque huc.
${}^{6}$~Pilatus autem audiens Galil\ae am, interrogavit si homo Galil\ae us esset.
${}^{7}$~Et ut cognovit quod de Herodis potestate esset, remisit eum ad Herodem, qui et ipse Jerosolymis erat illis diebus.


${}^{8}$~Herodes autem viso Jesu, gavisus est valde. Erat enim cupiens ex multo tempore videre eum, eo quod audierat multa de eo, et sperabat signum aliquod videre ab eo fieri.
${}^{9}$~Interrogabat autem eum multis sermonibus. At ipse nihil illi respondebat.
${}^{10}$~Stabant autem principes sacerdotum et scrib\ae\ constanter accusantes eum.
${}^{11}$~Sprevit autem illum Herodes cum exercitu suo~: et illusit indutum veste alba, et remisit ad Pilatum.
${}^{12}$~Et facti sunt amici Herodes et Pilatus in ipsa die~: nam antea inimici erant ad invicem.


${}^{13}$~Pilatus autem, convocatis principibus sacerdotum, et magistratibus, et plebe,
${}^{14}$~dixit ad illos~: Obtulistis mihi hunc hominem, quasi avertentem populum, et ecce ego coram vobis interrogans, nullam causam inveni in homine isto ex his in quibus eum accusatis.
${}^{15}$~Sed neque Herodes~: nam remisi vos ad illum, et ecce nihil dignum morte actum est ei.
${}^{16}$~Emendatum ergo illum dimittam.
${}^{17}$~Necesse autem habebat dimittere eis per diem festum unum.
${}^{18}$~Exclamavit autem simul universa turba, dicens~: Tolle hunc, et dimitte nobis Barabbam~:
${}^{19}$~qui erat propter seditionem quamdam factam in civitate et homicidium missus in carcerem.
${}^{20}$~Iterum autem Pilatus locutus est ad eos, volens dimittere Jesum.
${}^{21}$~At illi succlamabant, dicentes~: Crucifige, crucifige eum.
${}^{22}$~Ille autem tertio dixit ad illos~: Quid enim mali fecit iste~? nullam causam mortis invenio in eo~: corripiam ergo illum et dimittam.
${}^{23}$~At illi instabant vocibus magnis postulantes ut crucifigeretur~: et invalescebant voces eorum.


${}^{24}$~Et Pilatus adjudicavit fieri petitionem eorum.
${}^{25}$~Dimisit autem illis eum qui propter homicidium et seditionem missus fuerat in carcerem, quem petebant~: Jesum vero tradidit voluntati eorum.
${}^{26}$~Et cum ducerent eum, apprehenderunt Simonem quemdam Cyrenensem venientem de villa~: et imposuerunt illi crucem portare post Jesum.
${}^{27}$~Sequebatur autem illum multa turba populi et mulierum, qu\ae\ plangebant et lamentabantur eum.
${}^{28}$~Conversus autem ad illas Jesus, dixit~: Fili\ae\ Jerusalem, nolite flere super me, sed super vos ipsas flete et super filios vestros.
${}^{29}$~Quoniam ecce venient dies in quibus dicent~: Beat\ae\ steriles, et ventres qui non genuerunt, et ubera qu\ae\ non lactaverunt.
${}^{30}$~Tunc incipient dicere montibus~: Cadite super nos~; et collibus~: Operite nos.
${}^{31}$~Quia si in viridi ligno h\ae c faciunt, in arido quid fiet~?
${}^{32}$~Ducebantur autem et alii duo nequam cum eo, ut interficerentur.


${}^{33}$~Et postquam venerunt in locum qui vocatur Calvari\ae , ibi crucifixerunt eum~: et latrones, unum a dextris, et alterum a sinistris.
${}^{34}$~Jesus autem dicebat~: Pater, dimitte illis~: non enim sciunt quid faciunt. Dividentes vero vestimenta ejus, miserunt sortes.
${}^{35}$~Et stabat populus spectans, et deridebant eum principes cum eis, dicentes~: Alios salvos fecit, se salvum faciat, si hic est Christus Dei electus.
${}^{36}$~Illudebant autem ei et milites accedentes, et acetum offerentes ei,
${}^{37}$~et dicentes~: Si tu es rex Jud\ae orum, salvum te fac.
${}^{38}$~Erat autem et superscriptio scripta super eum litteris gr\ae cis, et latinis, et hebraicis~: Hic est rex Jud\ae orum.
${}^{39}$~Unus autem de his, qui pendebant, latronibus, blasphemabat eum, dicens~: Si tu es Christus, salvum fac temetipsum et nos.
${}^{40}$~Respondens autem alter increpabat eum, dicens~: Neque tu times Deum, quod in eadem damnatione es.
${}^{41}$~Et nos quidem juste, nam digna factis recipimus~: hic vero nihil mali gessit.
${}^{42}$~Et dicebat ad Jesum~: Domine, memento mei cum veneris in regnum tuum.
${}^{43}$~Et dixit illi Jesus~: Amen dico tibi~: hodie mecum eris in paradiso.


${}^{44}$~Erat autem fere hora sexta, et tenebr\ae\ fact\ae\ sunt in universam terram usque ad horam nonam.
${}^{45}$~Et obscuratus est sol, et velum templi scissum est medium.
${}^{46}$~Et clamans voce magna Jesus ait~: Pater, in manus tuas commendo spiritum meum. Et h\ae c dicens, expiravit.
${}^{47}$~Videns autem centurio quod factum fuerat, glorificavit Deum, dicens~: Vere hic homo justus erat.
${}^{48}$~Et omnis turba eorum, qui simul aderant ad spectaculum istud, et videbant qu\ae\ fiebant, percutientes pectora sua revertebantur.
${}^{49}$~Stabant autem omnes noti ejus a longe, et mulieres, qu\ae\ secut\ae\ eum erant a Galil\ae a, h\ae c videntes.


${}^{50}$~Et ecce vir nomine Joseph, qui erat decurio, vir bonus et justus~:
${}^{51}$~hic non consenserat consilio, et actibus eorum~: ab Arimath\ae a civitate Jud\ae \ae , qui exspectabat et ipse regnum Dei~:
${}^{52}$~hic accessit ad Pilatum et petiit corpus Jesu~:
${}^{53}$~et depositum involvit sindone, et posuit eum in monumento exciso, in quo nondum quisquam positus fuerat.
${}^{54}$~Et dies erat parasceves, et sabbatum illucescebat.
${}^{55}$~Subsecut\ae\ autem mulieres, qu\ae\ cum eo venerant de Galil\ae a, viderunt monumentum, et quemadmodum positum erat corpus ejus.
${}^{56}$~Et revertentes paraverunt aromata, et unguenta~: et sabbato quidem siluerunt secundum mandatum.
\Needspace{2.5\baselineskip}\versal{24}~\lettrine[lines=10,image=true,loversize=0.05,lraise=-0.03]{U}{}na autem sabbati valde diluculo venerunt ad monumentum, portantes qu\ae\ paraverant aromata~:
${}^{2}$~et invenerunt lapidem revolutum a monumento.
${}^{3}$~Et ingress\ae\ non invenerunt corpus Domini Jesu.
${}^{4}$~Et factum est, dum mente consternat\ae\ essent de isto, ecce duo viri steterunt secus illas in veste fulgenti.
${}^{5}$~Cum timerent autem, et declinarent vultum in terram, dixerunt ad illas~: Quid qu\ae ritis viventem cum mortuis~?
${}^{6}$~non est hic, sed surrexit~: recordamini qualiter locutus est vobis, cum adhuc in Galil\ae a esset,
${}^{7}$~dicens~: Quia oportet Filium hominis tradi in manus hominum peccatorum, et crucifigi, et die tertia resurgere.
${}^{8}$~Et recordat\ae\ sunt verborum ejus.
${}^{9}$~Et regress\ae\ a monumento nuntiaverunt h\ae c omnia illis undecim, et ceteris omnibus.
${}^{10}$~Erat autem Maria Magdalene, et Joanna, et Maria Jacobi, et ceter\ae\ qu\ae\ cum eis erant, qu\ae\ dicebant ad apostolos h\ae c.
${}^{11}$~Et visa sunt ante illos sicut deliramentum verba ista, et non crediderunt illis.
${}^{12}$~Petrus autem surgens cucurrit ad monumentum~: et procumbens vidit linteamina sola posita, et abiit secum mirans quod factum fuerat.


${}^{13}$~Et ecce duo ex illis ibant ipsa die in castellum, quod erat in spatio stadiorum sexaginta ab Jerusalem, nomine Emmaus.
${}^{14}$~Et ipsi loquebantur ad invicem de his omnibus qu\ae\ acciderant.
${}^{15}$~Et factum est, dum fabularentur, et secum qu\ae rerent~: et ipse Jesus appropinquans ibat cum illis~:
${}^{16}$~oculi autem illorum tenebantur ne eum agnoscerent.
${}^{17}$~Et ait ad illos~: Qui sunt hi sermones, quos confertis ad invicem ambulantes, et estis tristes~?
${}^{18}$~Et respondens unus, cui nomen Cleophas, dixit ei~: Tu solus peregrinus es in Jerusalem, et non cognovisti qu\ae\ facta sunt in illa his diebus~?
${}^{19}$~Quibus ille dixit~: Qu\ae~? Et dixerunt~: De Jesu Nazareno, qui fuit vir propheta, potens in opere et sermone coram Deo et omni populo~:
${}^{20}$~et quomodo eum tradiderunt summi sacerdotes et principes nostri in damnationem mortis, et crucifixerunt eum~:
${}^{21}$~nos autem sperabamus quia ipse esset redempturus Isra\"el~: et nunc super h\ae c omnia, tertia dies est hodie quod h\ae c facta sunt.
${}^{22}$~Sed et mulieres qu\ae dam ex nostris terruerunt nos, qu\ae\ ante lucem fuerunt ad monumentum,
${}^{23}$~et non invento corpore ejus, venerunt, dicentes se etiam visionem angelorum vidisse, qui dicunt eum vivere.
${}^{24}$~Et abierunt quidam ex nostris ad monumentum~: et ita invenerunt sicut mulieres dixerunt, ipsum vero non invenerunt.
${}^{25}$~Et ipse dixit ad eos~: O stulti, et tardi corde ad credendum in omnibus qu\ae\ locuti sunt prophet\ae~!
${}^{26}$~Nonne h\ae c oportuit pati Christum, et ita intrare in gloriam suam~?
${}^{27}$~Et incipiens a Moyse, et omnibus prophetis, interpretabatur illis in omnibus scripturis qu\ae\ de ipso erant.
${}^{28}$~Et appropinquaverunt castello quo ibant~: et ipse se finxit longius ire.
${}^{29}$~Et co\"egerunt illum, dicentes~: Mane nobiscum, quoniam advesperascit, et inclinata est jam dies. Et intravit cum illis.
${}^{30}$~Et factum est, dum recumberet cum eis, accepit panem, et benedixit, ac fregit, et porrigebat illis.
${}^{31}$~Et aperti sunt oculi eorum, et cognoverunt eum~: et ipse evanuit ex oculis eorum.
${}^{32}$~Et dixerunt ad invicem~: Nonne cor nostrum ardens erat in nobis dum loqueretur in via, et aperiret nobis Scripturas~?
${}^{33}$~Et surgentes eadem hora regressi sunt in Jerusalem~: et invenerunt congregatos undecim, et eos qui cum illis erant,
${}^{34}$~dicentes~: Quod surrexit Dominus vere, et apparuit Simoni.
${}^{35}$~Et ipsi narrabant qu\ae\ gesta erant in via, et quomodo cognoverunt eum in fractione panis.


${}^{36}$~Dum autem h\ae c loquuntur, stetit Jesus in medio eorum, et dicit eis~: Pax vobis~: ego sum, nolite timere.
${}^{37}$~Conturbati vero et conterriti, existimabant se spiritum videre.
${}^{38}$~Et dixit eis~: Quid turbati estis, et cogitationes ascendunt in corda vestra~?
${}^{39}$~videte manus meas, et pedes, quia ego ipse sum~; palpate et videte, quia spiritus carnem et ossa non habet, sicut me videtis habere.
${}^{40}$~Et cum hoc dixisset, ostendit eis manus et pedes.
${}^{41}$~Adhuc autem illis non credentibus, et mirantibus pr\ae\ gaudio, dixit~: Habetis hic aliquid quod manducetur~?
${}^{42}$~At illi obtulerunt ei partem piscis assi et favum mellis.
${}^{43}$~Et cum manducasset coram eis, sumens reliquias dedit eis.


${}^{44}$~Et dixit ad eos~: H\ae c sunt verba qu\ae\ locutus sum ad vos cum adhuc essem vobiscum, quoniam necesse est impleri omnia qu\ae\ scripta sunt in lege Moysi, et prophetis, et Psalmis de me.
${}^{45}$~Tunc aperuit illis sensum ut intelligerent Scripturas,
${}^{46}$~et dixit eis~: Quoniam sic scriptum est, et sic oportebat Christum pati, et resurgere a mortuis tertia die~:
${}^{47}$~et pr\ae dicari in nomine ejus pœnitentiam, et remissionem peccatorum in omnes gentes, incipientibus ab Jerosolyma.
${}^{48}$~Vos autem testes estis horum.
${}^{49}$~Et ego mitto promissum Patris mei in vos~; vos autem sedete in civitate, quoadusque induamini virtute ex alto.


${}^{50}$~Eduxit autem eos foras in Bethaniam, et elevatis manibus suis benedixit eis.
${}^{51}$~Et factum est, dum benediceret illis, recessit ab eis, et ferebatur in c\ae lum.
${}^{52}$~Et ipsi adorantes regressi sunt in Jerusalem cum gaudio magno~:
${}^{53}$~et erant semper in templo, laudantes et benedicentes Deum. Amen.
